\newcommand{\titulus}{\nomenFesti{In Dedicatione Ecclesiæ.}
\celebratio{In Tempore Paschalis.}}
\newcommand{\temporepaschalis}{T.P.}
\newcommand{\tedeumromanum}{Romanum.}
\newcommand{\invitatorium}{\pars{Invitatorium.}

\vspace{-2mm}

\antiphona{VI*}{temporalia/inv-christumdominumquidilexit-tp.gtex}}
\newcommand{\matutinum}{\pars{Psalmus 1.} \scriptura{Ps. 23, 7; \textbf{H327}}

\vspace{-4mm}

\antiphona{III a}{temporalia/ant-tolliteportas-tp.gtex}

\scriptura{Psalmus 23.}

\initiumpsalmi{temporalia/ps23-initium-iii-a-auto.gtex}

\input{temporalia/ps23-iii-a.tex} \Abardot{}

\vfill
\pagebreak

\pars{Psalmus 2.} \scriptura{Gn. 28, 12; \textbf{H328}}

\vspace{-4mm}

\antiphona{VII c}{temporalia/ant-nonesthic-tp.gtex}

\scriptura{Psalmus 83.}

\initiumpsalmi{temporalia/ps83-initium-vii-c-auto.gtex}

\input{temporalia/ps83-vii-c.tex} \Abardot{}

\vfill
\pagebreak

\pars{Psalmus 3.} \scriptura{Gn. 28, 12; \textbf{H328}}

\vspace{-4mm}

\antiphona{VII a}{temporalia/ant-viditjacobscalam-tp.gtex}

\scriptura{Psalmus 86.}

\initiumpsalmi{temporalia/ps86-initium-vii-a-auto.gtex}

\input{temporalia/ps86-vii-a.tex} \Abardot{}

\vfill
\pagebreak}
\newcommand{\matversus}{\pars{Versus.}

\noindent \Vbardot{} Adorábo ad templum sanctum tuum.

\noindent \Rbardot{} Et confitébor nómini tuo, Dómine.}
\newcommand{\lectioi}{\pars{Lectio I.} \scriptura{Petr. 2, 1-9}

\noindent De Epístola prima beáti Petri apóstoli.

\noindent Caríssimi: Deponéntes omnem malítiam et omnem dolum et simulatiónes et invídias et omnes detractiónes, sicut modo géniti infántes, rationále sine dolo lac concupíscite, ut in eo crescátis in salútem, si gustástis quóniam dulcis Dóminus. Ad quem accedéntes, lápidem vivum, ab homínibus quidem reprobátum, coram Deo autem eléctum, pretiósum, et ipsi tamquam lápides vivi ædificámini domus spiritális in sacerdótium sanctum offérre spiritáles hóstias acceptábiles Deo per Iesum Christum. Propter quod cóntinet Scriptúra: «Ecce pono in Sion lápidem angulárem, eléctum, pretiósum; et qui credit in eo, non confundétur».

\noindent Vobis ígitur honor credéntibus; non credéntibus autem «Lapis, quem reprobavérunt ædificántes, hic factus est in caput ánguli» et «lapis offensiónis et petra scándali»; qui offéndunt verbo non credéntes, in quod et pósiti sunt.

\noindent Vos autem genus eléctum, regále sacerdótium, gens sancta, pópulus in acquisitiónem, ut virtútes annuntiétis eius, qui de ténebris vos vocávit in admirábile lumen suum; qui aliquándo non pópulus, nunc autem pópulus Dei; qui non consecúti misericórdiam nunc autem misericórdiam consecúti.

\noindent Caríssimi, óbsecro tamquam ádvenas et peregrínos abstinére vos a carnálibus desidériis, quæ mílitant advérsus ánimam; conversatiónem vestram inter gentes habéntes bonam, ut in eo, quod detréctant de vobis tamquam de malefactóribus, ex bonis opéribus considerántes gloríficent Deum in die visitatiónis.

\noindent Subiécti estóte omni humánæ creatúræ propter Dóminum: sive regi quasi præcellénti sive dúcibus tamquam ab eo missis ad vindíctam malefactórum, laudem vero bonórum; quia sic est volúntas Dei, ut benefaciéntes obmutéscere faciátis imprudéntium hóminum ignorántiam, quasi líberi, et non quasi velámen habéntes malítiæ libertátem, sed sicut servi Dei.

\noindent Omnes honoráte, fraternitátem dilígite, Deum timéte, regem honorificáte.}
\newcommand{\responsoriumi}{\pars{Responsorium 1.} \scriptura{\Rbardot{} Gn. 28, 17 \Vbardot{} ibid., 16; \textbf{H150}}

\vspace{-5mm}

\responsorium{I}{temporalia/resp-dumexiretiacob-CROCHU-tp.gtex}{}}
\newcommand{\lectioii}{\pars{Lectio II.} \scriptura{Sermo 336, 1. 6: PL 38 [edit. 1861], 1471-1472. 1475}

\noindent Ex Sermónibus sancti Augustíni epíscopi.

\noindent Celébritas huius congregatiónis, dedicátio est domus oratiónis. Domus ergo nostrárum oratiónum ista est, domus Dei nos ipsi. Si domus Dei nos ipsi, nos in hoc sǽculo ædificámur, ut in fine sǽculi dedicémur. Ædifícium, immo ædificátio, habet labórem, dedicátio exsultatiónem.

\noindent Quod hic fiébat, quando ista surgébant, hoc fit modo cum congregántur credéntes in Christum. Credéndo enim quasi de silvis et móntibus ligna et lápides præcidúntur; cum vero catechizántur, baptizántur, formántur, tamquam inter manus fabrórum et opíficum dolántur, collineántur, complanántur.

\noindent Verúmtamen domum Dómini non fáciunt, nisi quando caritáte compaginántur. Ligna ista et lápides si non sibi certo órdine cohærérent, si non se pacífice innécterent, si non se ínvicem cohæréndo sibi quodam modo amárent; nemo huc intráret. Dénique, quando vides in áliqua fábrica lápides et ligna bene sibi cohærére, secúrus intras, ruínam non times.}
\newcommand{\responsoriumii}{\pars{Responsorium 2.} \scriptura{\Rbardot{} Gn. 28, 20-22 \Vbardot{} ibid., 16; \textbf{H150}}

\vspace{-5mm}

\responsorium{I}{temporalia/resp-sidominusdeusmeus-CROCHU-tp.gtex}{}}
\newcommand{\lectioiii}{\pars{Lectio III.}

\noindent Volens ergo Dóminus Christus intráre, et in nobis habitáre, tamquam ædificándo dicébat: Mandátum novum do vobis, ut vos ínvicem diligátis. Mandátum, inquit, do vobis. Véteres enim erátis, domum mihi nondum faciebátis, in vestra ruína iacebátis. Ergo, ut eruámini de vestræ ruínæ vetustáte, vos ínvicem amáte.

\noindent Consíderet ergo cáritas vestra, ædificári adhuc istam domum toto, sicut prædíctum est et promíssum, orbe terrárum. Cum enim ædificarétur domus post captivitátem, sicut habet álius psalmus, dicebátur: Cantáte Dómino cánticum novum; cantáte Dómino, omnis terra. Quod ibi dixit: cánticum novum; hoc Dóminus dixit: mandátum novum. Quid enim habet cánticum novum, nisi amórem novum? Cantáre amántis est. Vox huius cantóris fervor est sancti amóris.

\noindent Quod ígitur hic factum corporáliter vidémus in pariétibus, spiritáliter fiat in méntibus; et quod hic perféctum cérnimus in lapídibus et lignis, hoc, ædificánte grátia Dei, perficiátur in corpóribus vestris.

\noindent Principáliter ergo grátias agámus Dómino Deo nostro, a quo est omne datum óptimum, et omne donum perféctum; et eius bonitátem tota cordis alacritáte laudémus, quóniam ad construéndam istam domum oratiónis fidélium suórum visitávit ánimum, excitávit afféctum, surrogávit auxílium; inspirávit necdum voléntibus ut vellent, adiúvit bonæ voluntátis conátus ut fácerent; ac per hoc Deus, qui operátur in suis et velle et perfícere pro bona voluntáte, hæc ómnia ipse cœpit, ipse perfécit.}
\newcommand{\responsoriumiii}{\pars{Responsorium 3.} \scriptura{\Rbardot{} Cf. 1 Reg. 8, 28 \Vbardot{} 2 Paralip. 6, 14; \textbf{H396}}

\vspace{-5mm}

\responsorium{IV}{temporalia/resp-exaudisti-CROCHU-cumdox-tp.gtex}{}}
\newcommand{\lectiobrevis}{\pars{Lectio brevis.} \scriptura{Is. 56, 7}

\noindent Addúcam eos in montem sanctum meum et lætificábo eos in domo oratiónis meæ: holocáusta eórum et víctimæ eórum placébunt mihi super altári meo, quia domus mea domus oratiónis vocábitur cunctis pópulis.}
\newcommand{\preces}{\noindent Nos, lápides vivi, superædificáti lápidi elécto Christo,~\gredagger{} Patrem omnipoténtem pro Ecclésia deprecémur eius dilécta,~\grestar{} et fidem in eam profiteámur, dicéntes:

\Rbardot{} Hic domus Dei est et porta cæli.

\noindent Pater agrícola,~\gredagger{} víneam tuam emúnda, custódi, multíplica,~\grestar{} ut in conspéctu tuo omnem terram ímpleat.

\Rbardot{} Hic domus Dei est et porta cæli.

\noindent Pastor ætérne,~\gredagger{} ovíle tuum prótege ac proténde,~\grestar{} ut sub único pastóre, Fílio tuo, oves congregéntur in unum.

\Rbardot{} Hic domus Dei est et porta cæli.

\noindent Sator omnípotens,~\gredagger{} sémina verbum in agro tuo,~\gredagger{} ut céntuplum fructum áfferat in messem ætérnam.

\Rbardot{} Hic domus Dei est et porta cæli.

\noindent Sápiens ædificátor,~\gredagger{} tuam domum familiámque sanctífica,~\grestar{} ut cívitas de cælo, Ierúsalem nova, sponsa appáreat ómnibus gloriósa.

\Rbardot{} Hic domus Dei est et porta cæli.}
\newcommand{\oratioLaudes}{\noindent Deus, qui de vivis et eléctis lapídibus ætérnum habitáculum tuæ prǽparas maiestáti, multíplica in Ecclésia tua spíritum grátiæ, quem dedísti, ut fidélis tibi pópulus in cæléstis ædificatiónem Ierúsalem semper accréscat.

\noindent Per Dóminum nostrum Iesum Christum, Fílium tuum, qui tecum vivit et regnat in unitáte Spíritus Sancti, Deus, per ómnia sǽcula sæculórum.

\noindent \Rbardot{} Amen.}
\newcommand{\oratio}{\pars{Oratio.}

\oratioLaudes}
\newcommand{\benedictus}{\pars{Canticum Zachariæ.} \scriptura{Lc. 19, 5.6.9; \textbf{H330}}

\vspace{-6mm}

\antiphona{VIII G\textsuperscript{2}}{temporalia/ant-zachaeefestinans.gtex}

%\vspace{-2mm}

\scriptura{Lc. 1, 68-79}

%\vspace{-2mm}

\cantusSineNeumas
\initiumpsalmi{temporalia/benedictus-initium-viiisoll-G5-auto.gtex}

%\vspace{-1.5mm}

\input{temporalia/benedictus-viiisoll-G5.tex}

\vfill

\antiphona{}{temporalia/ant-zachaeefestinans.gtex}}
\newcommand{\laudes}{
\vfill
\pagebreak

\pars{Hymnus.}

{
\grechangedim{interwordspacetext}{0.20 cm plus 0.15 cm minus 0.05 cm}{scalable}%
\cuminitiali{IV}{temporalia/hym-AngularisFundamentum.gtex}
\grechangedim{interwordspacetext}{0.22 cm plus 0.15 cm minus 0.05 cm}{scalable}%
}
\vfill
\pagebreak

\pars{Psalmus 1.} \scriptura{Is. 56, 7; Mt. 21, 13; Mc. 11, 17; \textbf{H329}}

\antiphona{I d}{temporalia/ant-domusmea-tp.gtex}

\vspace{-4mm}

\scriptura{Psalmus 62.}

\initiumpsalmi{temporalia/ps62-initium-i-d-auto.gtex}

\input{temporalia/ps62-i-d.tex} \Abardot{}

\vfill
\pagebreak

\pars{Psalmus 2.} \scriptura{Dn. 3, 53}

\vspace{-4mm}

\antiphona{VIII G}{temporalia/ant-benedictusesintemplo-tp.gtex}

\scriptura{Canticum trium puerorum, Dan. 3, 57-88 et 56}

\initiumpsalmi{temporalia/dan3-initium-viii-G-auto.gtex}

\input{temporalia/dan3-viii-G-sinedox.tex}

\rubrica{Hic non dicitur Gloria Patri, neque Amen.}

\vfill

\vspace{-6mm}

\antiphona{}{temporalia/ant-benedictusesintemplo-tp.gtex} % repeat the antiphon - new page

\vfill
\pagebreak

\pars{Psalmus 3.}

\vspace{-4mm}

\antiphona{VIII G}{temporalia/ant-lausdeiinecclesia.gtex}

%\vspace{-2mm}

\scriptura{Psalmus 149.}

%\vspace{-2mm}

\initiumpsalmi{temporalia/ps149-initium-viii-G-auto.gtex}

\input{temporalia/ps149-viii-G.tex} \Abardot{}

\vfill
\pagebreak

\lectiobrevis

\vfill

\pars{Responsorium breve.} \scriptura{Ps. 47, 2}

\cuminitiali{VI}{temporalia/resp-magnusdominus-tp.gtex}

\vfill
\pagebreak

\benedictus

\vspace{-1cm}

\vfill
\pagebreak

\pars{Preces.}

\sineinitiali{}{temporalia/tonusprecum.gtex}

\preces

\vfill

\pars{Oratio Dominica.}

\cuminitiali{}{temporalia/oratiodominicaalt.gtex}

\vfill
\pagebreak

\rubrica{vel:}

\pars{Supplicatio Litaniæ.}

\cuminitiali{}{temporalia/supplicatiolitaniae.gtex}

\vfill

\pars{Oratio Dominica.}

\cuminitiali{}{temporalia/oratiodominica.gtex}

\vfill
\pagebreak

\oratio

\vspace{-1mm}

\vfill

\rubrica{Hebdomadarius dicit Dominus vobiscum, vel, absente sacerdote vel diacono, sic concluditur:}

\vspace{2mm}

\antiphona{C}{temporalia/dominusnosbenedicat.gtex}

\rubrica{Postea cantatur a cantore:}

\vspace{2mm}

\cuminitiali{VII}{temporalia/benedicamus-tempore-paschali.gtex}

\vspace{1mm}

\vfill
\pagebreak}
\newcommand{\magnificat}{\pars{Canticum B. Mariæ V.} \scriptura{Cf. Ps. 45, 5; \textbf{H330}}

\vspace{-4mm}

\antiphona{I g}{temporalia/ant-sanctificavitdominus.gtex}

%\vspace{-3mm}

\scriptura{Lc. 1, 46-55}

%\vspace{-2mm}

\initiumpsalmi{temporalia/magnificat-initium-isoll-g.gtex}

%\vspace{-1.5mm}

\input{temporalia/magnificat-isoll-g.tex}

\antiphona{}{temporalia/ant-sanctificavitdominus.gtex}}
% LuaLaTeX

\documentclass[a4paper, twoside, 12pt]{article}
\usepackage[latin]{babel}
%\usepackage[landscape, left=3cm, right=1.5cm, top=2cm, bottom=1cm]{geometry} % okraje stranky
%\usepackage[landscape, a4paper, mag=1166, truedimen, left=2cm, right=1.5cm, top=1.6cm, bottom=0.95cm]{geometry} % okraje stranky
\usepackage[landscape, a4paper, mag=1400, truedimen, left=0.5cm, right=0.5cm, top=0.5cm, bottom=0.5cm]{geometry} % okraje stranky

\usepackage{fontspec}
\setmainfont[FeatureFile={junicode.fea}, Ligatures={Common, TeX}, RawFeature=+fixi]{Junicode}
%\setmainfont{Junicode}

% shortcut for Junicode without ligatures (for the Czech texts)
\newfontfamily\nlfont[FeatureFile={junicode.fea}, Ligatures={Common, TeX}, RawFeature=+fixi]{Junicode}

\usepackage{multicol}
\usepackage{color}
\usepackage{lettrine}
\usepackage{fancyhdr}

% usual packages loading:
\usepackage{luatextra}
\usepackage{graphicx} % support the \includegraphics command and options
\usepackage{gregoriotex} % for gregorio score inclusion
\usepackage{gregoriosyms}
\usepackage{wrapfig} % figures wrapped by the text
\usepackage{parcolumns}
\usepackage[contents={},opacity=1,scale=1,color=black]{background}
\usepackage{tikzpagenodes}
\usepackage{calc}
\usepackage{longtable}
\usetikzlibrary{calc}

\setlength{\headheight}{14.5pt}

% Commands used to produce a typical "Conventus" booklet

\newenvironment{titulusOfficii}{\begin{center}}{\end{center}}
\newcommand{\dies}[1]{#1

}
\newcommand{\nomenFesti}[1]{\textbf{\Large #1}

}
\newcommand{\celebratio}[1]{#1

}

\newcommand{\hora}[1]{%
\vspace{0.5cm}{\large \textbf{#1}}

\fancyhead[LE]{\thepage\ / #1}
\fancyhead[RO]{#1 / \thepage}
\addcontentsline{toc}{subsection}{#1}
}

% larger unit than a hora
\newcommand{\divisio}[1]{%
\begin{center}
{\Large \textsc{#1}}
\end{center}
\fancyhead[CO,CE]{#1}
\addcontentsline{toc}{section}{#1}
}

% a part of a hora, larger than pars
\newcommand{\subhora}[1]{
\begin{center}
{\large \textit{#1}}
\end{center}
%\fancyhead[CO,CE]{#1}
\addcontentsline{toc}{subsubsection}{#1}
}

% rubricated inline text
\newcommand{\rubricatum}[1]{\textit{#1}}

% standalone rubric
\newcommand{\rubrica}[1]{\vspace{3mm}\rubricatum{#1}}

\newcommand{\notitia}[1]{\textcolor{red}{#1}}

\newcommand{\scriptura}[1]{\hfill \small\textit{#1}}

\newcommand{\translatioCantus}[1]{\vspace{1mm}%
{\noindent\footnotesize \nlfont{#1}}}

% pruznejsi varianta nasledujiciho - umoznuje nastavit sirku sloupce
% s prekladem
\newcommand{\psalmusEtTranslatioB}[3]{
  \vspace{0.5cm}
  \begin{parcolumns}[colwidths={2=#3}, nofirstindent=true]{2}
    \colchunk{
      \input{#1}
    }

    \colchunk{
      \vspace{-0.5cm}
      {\footnotesize \nlfont
        \input{#2}
      }
    }
  \end{parcolumns}
}

\newcommand{\psalmusEtTranslatio}[2]{
  \psalmusEtTranslatioB{#1}{#2}{8.5cm}
}


\newcommand{\canticumMagnificatEtTranslatio}[1]{
  \psalmusEtTranslatioB{#1}{temporalia/extra-adventum-vespers/magnificat-boh.tex}{12cm}
}
\newcommand{\canticumBenedictusEtTranslatio}[1]{
  \psalmusEtTranslatioB{#1}{temporalia/extra-adventum-laudes/benedictus-boh.tex}{10.5cm}
}

% volne misto nad antifonami, kam si zpevaci dokresli neumy
\newcommand{\hicSuntNeumae}{\vspace{0.5cm}}

% prepinani mista mezi notovymi osnovami: pro neumovane a neneumovane zpevy
\newcommand{\cantusCumNeumis}{
  \setgrefactor{17}
  \global\advance\grespaceabovelines by 5mm%
}
\newcommand{\cantusSineNeumas}{
  \setgrefactor{17}
  \global\advance\grespaceabovelines by -5mm%
}

% znaky k umisteni nad inicialu zpevu
\newcommand{\superInitialam}[1]{\gresetfirstlineaboveinitial{\small {\textbf{#1}}}{\small {\textbf{#1}}}}

% pars officii, i.e. "oratio", ...
\newcommand{\pars}[1]{\textbf{#1}}

\newenvironment{psalmus}{
  \setlength{\parindent}{0pt}
  \setlength{\parskip}{5pt}
}{
  \setlength{\parindent}{10pt}
  \setlength{\parskip}{10pt}
}

%%%% Prejmenovat na latinske:
\newcommand{\nadpisZalmu}[1]{
  \hspace{2cm}\textbf{#1}\vspace{2mm}%
  \nopagebreak%

}

% mode, score, translation
\newcommand{\antiphona}[3]{%
\hicSuntNeumae
\superInitialam{#1}
\includescore{#2}

#3
}
 % Often used macros
%%%% Preklady jednotlivych zpevu (nektere se opakuji, a je dobre mit je
% vsechny na jedne hromade)

\newcommand{\trOratioAnteOfficium}{\translatioCantus{Otevři, Pane, má ústa, abych chválil tvé svaté jméno.
Očisti mé srdce od všech marnivých, zvrácených a~jiných myšlenek, osvěť rozum, rozněť cit,
abych mohl důstojně, soustředěně a~zbožně recitovat a~vysloužil si být
vyslyšen před tváří tvé velebnosti. Skrze Krista…}}

\newcommand{\trOratioPostOfficium}{\translatioCantus{\textit{Následující modlitbu
opatřil pro ty, kdo ji zbožně vyřknou po hodinkách, zesnulý papež Lev X.
odpustky za hříchy vzniklé při konání hodinek z~lidské křehkosti. Říká se
vkleče.}
Svatosvaté a~nerozdílné Trojici, ukřižovanému lidství našeho Pána Ježíše
Krista, přeblažené a~přeslavné plodné neporušenosti vždy Panny Marie
i~souhrnu všech svatých buď ode všeho stvoření věčná chvála, čest a~sláva, nám
pak buď dáno odpuštění všech hříchů, po nekonečné věky věků. Amen.}}

% HOURS ---

\newcommand{\trAntI}{\translatioCantus{Jasné narození slavné Panny Marie,
z pokolení (dosl. ze semene) Abrahámova, vzešlé z kmene Judova, z rodu Davidova.}}
\newcommand{\trAntII}{\translatioCantus{Dnes je Narození svaté Panny 
Marie, jejíž předrahý život osvěcuje všechny církve.}}

\newcommand{\trAntIII}{\translatioCantus{Maria, jež vzešla 
z královského rodu, září; myslí i duchem ji zbožně prosíme, aby 
nám pomáhala svými přímluvami.}}

\newcommand{\trAntIV}{\translatioCantus{Srdcem i duchem pějme Kristu 
k slávě o této svaté slavnosti vznešené Rodičky Boží Marie.}}

\newcommand{\trAntV}{\translatioCantus{Příjemně \notitia{?} 
oslavujme Narození blahoslavené Marie,
aby se ona za nás přimlouvala u Pána Ježíše Krista.}}

\newcommand{\trCapituli}{\translatioCantus{Před věky, na počátku mě stvořil, potrvám věčně. Ve svatém Stanu jsem před ním konala službu.}}

\newcommand{\trRespVesp}{\translatioCantus{Buď zdráva, Maria,
plná milosti: \grestar{} Pán s tebou. \Vbardot{} Požehnaná jsi mezi ženami,
a požehnaný plod života (ve smyslu lůna, břicha) tvého.}}

\newcommand{\trVersus}{\translatioCantus{\Vbardot{} Dnes je Narození svaté Panny Marie. \Rbardot{} Jejíž předrahý život osvěcuje všechny církve.}}

\newcommand{\trAntMagnificatI}{\translatioCantus{Konejme památku
veledůstojného narození slavné Panny Marie,
jíž se dostalo mateřské důstojnosti bez ztráty panenské cudnosti.}}

% Tento preklad je vice nez nejisty a ani alternativy, ktere jsem
% videl, me nepresvedcily...
\newcommand{\trAntBenedictus}{\translatioCantus{Slavnostně slavme 
dnešní narození Marie, vždy Panny a Rodičky Boží: v něm se objevuje
vysokost trůnu (totiž Marie, trůnu Božího Syna), aleluja.}}

\newcommand{\trAntMagnificatII}{\translatioCantus{Tvé narození,
Bohorodičko Panno, vyhlásilo radost celému světu:
z tebe totiž vzešlo Slunce spravedlnosti, Kristus, náš Bůh:
jenž zrušil kletbu a dal nám požehnání: přemohl smrt a dal nám život věčný.}}

\newcommand{\trOrationis}{\translatioCantus{Prosíme tě, Bože, 
uděl svým služebníkům dar nebeské milosti,
aby těm, jimž slehnutím blahoslavené Panny vyvstal počátek spásy, 
slavnost k poctě jejího narození přinesla
rozhojnění pokoje.
Skrze tvého Syna, našeho Pána Ježíše Krista, který s tebou žije a kraluje,
Bůh, v jednotě Ducha svatého po všechny věky věků.}}

\newcommand{\trFideliumAnimae}{\translatioCantus{\Vbardot{} Duše věrných ať pro
milosrdenství Boží odpočívají v~pokoji. \Rbardot{} Amen.}}

% Completorium

\newcommand{\trJubeDomne}{\translatioCantus{Rač, pane, požehnat.}}

\newcommand{\trComplBenedictio}{\translatioCantus{Pokojnou noc a~svatou smrt
nechť nám dopřeje všemohoucí Pán. \Rbardot{} Amen.}}

\newcommand{\trComplLectioBr}{\translatioCantus{Buďte střízliví, bděte.
Váš protivník Ďábel obchází jako lev řvoucí a~hledá, koho by pohltil.
Postavte se proti němu pevní ve víře.  Ale ty, Pane, smiluj se nad námi.
\Rbardot{} Bohu díky.}}

\newcommand{\trComplAntI}{\translatioCantus{Rač se smilovati nade mnou,
Hospodine, a vyslyš mou modlitbu.}}

\newcommand{\trComplCapituli}{\translatioCantus{Jsi přece, Hospodine,
uprostřed nás a~jmenujeme se po tobě.  Neopouštěj nás, Pane, náš Bože.}}

\newcommand{\trRespCompl}{\translatioCantus{Do tvých rukou, Pane, \grestar{}
poroučím svého ducha. \Vbardot{} Ty mne zachráníš, Pane, Bože věrný.}}

\newcommand{\trComplVersus}{\translatioCantus{\Vbardot{} Střez mne jako zřítelnici oka,
aleluja. \Rbardot{} Ve stínu svých křídel uschovej mne, aleluja.}}

\newcommand{\trAntSalvaNos}{\translatioCantus{Ochraňuj nás, Pane, když
bdíme, a~buď s~námi, když spíme, ať bdíme s~Kristem a~odpočíváme v~pokoji.}}

\newcommand{\trComplOrationis}{\translatioCantus{Zavítej, prosíme, Pane, sem
do našeho příbytku a~daleko od něj zažeň všechny úklady nepřítele. Ať tu
bydlí tví svatí andělé a~tvoje požehnání buď nad ním stále. Skrze…}}

\newcommand{\trSalveRegina}{\translatioCantus{Zdrávas Královno, matko
milosrdenství, živote, sladkosti a naděje naše, buď zdráva!
K tobě voláme, vyhnaní synové Evy,
k tobě vzdycháme, lkajíce a plačíce
v tomto slzavém údolí.
A proto, orodovnice naše,
obrať k nám své milosrdné oči
a Ježíše, požehnaný plod života svého,
nám po tomto putování ukaž,
ó milostivá, ó přívětivá,
ó přesladká, Panno Maria!}}

\newcommand{\trOraProNobis}{\translatioCantus{\Vbardot{} 
Oroduj za nás, svatá Boží Rodičko,
\Rbardot{} aby nám Kristus dal účast na svých zaslíbeních.}}

% Matutinum

\newcommand{\trMatInvitatorium}{\translatioCantus{}}

\newcommand{\trMatVeniteA}{\translatioCantus{Pojďte, chvalme s~radostí Pána,
s~jásotem slavme Boha, svou spásu; předstupme před tvář jeho s~díky, písně plesu pějme jemu.}}

\newcommand{\trMatVeniteB}{\translatioCantus{Neboť Bůh veliký jest Hospodin, a~král nade všecky bohy.
Jsouť v~jeho ruce všecky hlubiny země, temena hor jsou majetek jeho.}}

\newcommand{\trMatVeniteC}{\translatioCantus{Jehoť jest moře, neb on je učinil; i~souš
je dílo jeho rukou. Pojďme, klanějme se, padněme, klekněme před Pánem, svým
tvůrcem. Jeť on Pán, náš Bůh, a~my jsme lid, jejž on vodí a~ovce, jež pase.}}

\newcommand{\trMatVeniteD}{\translatioCantus{Kéž byste poslechli dnes hlasu jeho:
,,Nezatvrzujte svých srdcí jak v~Hádce, jak v~Pokušení na poušti, kde vaši otcové pokoušeli mne,
zkoušeli mne, ač vídali skutky mé.``}}

\newcommand{\trMatVeniteE}{\translatioCantus{Čtyřicet roků mrzel jsem se na to pokolení
a~řekl jsem: ,,Lid je to myslí stále bloudící``! Oni však nechtěli znáti mé cesty, takže jsem
přisáhl ve svém hněvu: ,,Nedojdou odpočinku mého!\mbox{}``}}

\newcommand{\trMatAntI}{\translatioCantus{}}

\newcommand{\trMatAntII}{\translatioCantus{}}

\newcommand{\trMatAntIII}{\translatioCantus{}}

\newcommand{\trMatVersusI}{\translatioCantus{}}

\newcommand{\trMatAbsolutioI}{\translatioCantus{Vyslyš Pane Ježíši Kriste
prosby svých služebníků \gredagger{} a~smiluj se nad námi, \grestar{} jenž
s~Otcem a~Duchem…}}

\newcommand{\trMatBenedictioI}{\translatioCantus{Rač, pane, požehnat.
Věčný Otec nám stále žehnej. \Rbardot{} Amen.}}

\newcommand{\trMatLecI}{\translatioCantus{Kéž by mě zulíbal polibky svých úst. 
Tvé milování je nad víno lahodnější;
vybraně voní tvé voňavky;
rozlévající se olej je tvé jméno,
proto tě dívky milují.
Strhni mě za sebou, poběžme!
Král mě uvedl do svých komnat;
budeš nám radostí a jásotem.
Víc než víno oslavíme tvé milování;
věru po právu jsi milován!
Snědá jsem, a přece krásná, jeruzalémské dcery,
jako stany kedarské,
jako šalmské závěsy.
}}

\newcommand{\trMatRespI}{\translatioCantus{}}

\newcommand{\trMatBenedictioII}{\translatioCantus{Rač, pane, požehnat.
Jednorozený Boží Syn nám žehnej \grestar{} a nám pomáhej. \Rbardot{} Amen.}}

\newcommand{\trMatLecII}{\translatioCantus{Nehleďte na mou osmahlou pleť:
to mě slunce ožehlo.
Synové mé matky se na mne rozzlobili,
poslali mě hlídat vinice.
A svou vinici, tu jsem nehlídala!
Pověz mi tedy, ty, jehož miluje mé srdce:
kam zavedeš své stádo pást,
kde ho necháš za poledne odpočívat?
Abych už nebloudila jako tulačka
poblíž stád druhů tvých.
Nevíš-li to, nejrásnější z žen,
jdi po stopách stáda
a kůzlata svá zaveď, ať se pasou
poblíž obydlí pastýřů.
Ke své klisně zapřažené do vozu faraonova
tebe, mé milá, přirovnávám.
Stále krásné jsou tvé líce s náušnicemi
i tvé hrdlo v náhrdelnících.}}

\newcommand{\trMatRespII}{\translatioCantus{}}

\newcommand{\trMatBenedictioIII}{\translatioCantus{Rač, pane, požehnat.
Milost Ducha Svatého ať osvítí nám smysly \grestar{} i srdce. \Rbardot{} Amen.}}

\newcommand{\trMatLecIII}{\translatioCantus{Zhotovíme ti zlaté náušnice
a kuličky ze stříbra.
Když král stoluje,
vydechuje můj nard svou vůni.
Můj milý je polštářek s myrhou,
jenž mi odpočívá mezi ňadry.
Můj milý je hrozen šáchoru
ve vinicích v Engadi.
Jak jsi krásná, milá moje,
jak jsi krásná!
Tvé oči jsou holubice.
Jak jsi krásný, můj milý,
jak líbezný!
Naše lože je samá zeleň.
Trámoví našeho domu je z cedru,
naše ostění z cypřiše.}}

\newcommand{\trMatRespIII}{\translatioCantus{}}

\newcommand{\trMatAntIV}{\translatioCantus{}}

\newcommand{\trMatAntV}{\translatioCantus{}}

\newcommand{\trMatAntVI}{\translatioCantus{}}

\newcommand{\trMatVersusII}{\translatioCantus{}}

\newcommand{\trMatAbsolutioII}{\translatioCantus{
Tvá milost a laskavost nechť nám pomáhá, jenž žiješ a vládneš s Otcem a Svatým Duchem na věky věků.}}

\newcommand{\trMatBenedictioIV}{\translatioCantus{Rač, pane, požehnat.
Bůh Otec všemohoucí, \grestar{} buď k nám milostivý a odpouštějící. \Rbardot{} Amen.}}

\newcommand{\trMatLecIV}{\translatioCantus{}}

\newcommand{\trMatRespIV}{\translatioCantus{}}

\newcommand{\trMatBenedictioV}{\translatioCantus{}}

\newcommand{\trMatLecV}{\translatioCantus{}}

\newcommand{\trMatRespV}{\translatioCantus{}}

\newcommand{\trMatBenedictioVI}{\translatioCantus{Rač, pane, požehnat.
Bůh rozněť v nás oheň své lásky. \Rbardot{} Amen.}}

\newcommand{\trMatLecVI}{\translatioCantus{}}

\newcommand{\trMatRespVI}{\translatioCantus{}}

\newcommand{\trMatAntVII}{\translatioCantus{}}

\newcommand{\trMatAntVIII}{\translatioCantus{}}

\newcommand{\trMatAntIX}{\translatioCantus{}}

\newcommand{\trMatVersusIII}{\translatioCantus{}}

\newcommand{\trMatAbsolutioIII}{\translatioCantus{Z okovů našich hříchů,
\grestar{} vysvoboď nás všemohoucí a milosrdný Pán. \Rbardot{} Amen.}}

\newcommand{\trMatBenedictioVII}{\translatioCantus{Rač, pane, požehnat.
Čtení evangelia nechť je nám \grestar{} spásou a ochranou. \Rbardot{} Amen.}}

\newcommand{\trMatLecVIIa}{\translatioCantus{
  Rodokmen Ježíše Krista, syna Davidova, syna Abrahámova:
  Abrahám zplodil Izáka,
  Izák zplodil Jakuba.}}

\newcommand{\trMatLecVIIb}{\translatioCantus{}}

\newcommand{\trMatRespVII}{\translatioCantus{}}

\newcommand{\trMatBenedictioVIII}{\translatioCantus{Rač, pane, požehnat.
\Rbardot{} Amen.}}

\newcommand{\trMatLecVIII}{\translatioCantus{}}

\newcommand{\trMatRespVIII}{\translatioCantus{}}

\newcommand{\trMatBenedictioIX}{\translatioCantus{Rač, pane, požehnat.
Do společnosti občanů nebes \grestar{} ať nás dovede král andělů.
\Rbardot{} Amen.}}

\newcommand{\trMatLecIX}{\translatioCantus{}}

% from the Czech Liturgia horarum
\newcommand{\trTeDeum}{\begin{translatioMulticol}{3}

Bože, tebe chválíme, 
tebe, Pane, velebíme.

Tebe, věčný Otče, 
oslavuje celá země.

Všichni andělé, 
cherubové i~serafové,

všechny mocné nebeské zástupy 
bez ustání volají:

Svatý, Svatý, Svatý, 
Pán, Bůh zástupů.

Plná jsou nebesa i~země 
tvé vznešené slávy.

Oslavuje tě 
sbor tvých apoštolů,

chválí tě 
velký počet proroků,

vydává o~tobě svědectví 
zástup mučedníků;

a~po celém světě 
vyznává tě tvá církev:

neskonale velebný, 
všemohoucí Otče,

úctyhodný Synu Boží, 
pravý a~jediný,

božský Utěšiteli, 
Duchu svatý.

Kriste, Králi slávy, 
tys od věků Syn Boha Otce;

abys člověka vykoupil, 
stal ses člověkem a~narodil ses z~Panny;

zlomil jsi osten smrti 
a~otevřel věřícím nebe;

sedíš po Otcově pravici 
a~máš účast na jeho slávě.

Věříme, že přijdeš soudit, 

a~proto tě prosíme:
přispěj na pomoc svým služebníkům, 
vždyť jsi je vykoupil svou předrahou krví;

dej, ať se radují s~tvými svatými 
ve věčné slávě.

Zachraň, Pane, svůj lid, žehnej svému dědictví, 
veď ho a~stále pozvedej.

Každý den tě budeme velebit 
a~chválit tvé jméno po všechny věky.

Pomáhej nám i~dnes, 
ať se nedostaneme do područí hříchu.

Smiluj se nad námi, Pane, 
smiluj se nad námi.

Ať spočine na nás tvé milosrdenství, 
jak doufáme v~tebe.

Pane, k~tobě se utíkáme, 
ať nejsme zahanbeni na věky. 
\end{translatioMulticol}}

\newcommand{\trMatEvangelium}{\translatioCantus{
  Rodokmen Ježíše Krista, syna Davidova, syna Abrahámova:
  Abrahám zplodil Izáka,
  Izák zplodil Jakuba,
  Jakub zplodil Judu a jeho bratry,
  Juda zplodil Farese a Zaru z Tamary,
  Fares zplodil Esroma,
  Esrom zplodil Arama,
  Aram zplodil Aminadaba,
  Aminadab zplodil Naasona,
  Naason zplodil Salmona,
  Salmon zplodil Boaze z Rahaby,
  Boaz zplodil Jobeda z Rut,
  Jobed zplodil Jessea,
  Jesse zplodil krále Davida.
  David zplodil Šalomouna z Uriášovy ženy,
  Šalomoun zplodil Roboama,
  Roboam zplodil Abiu,
  Abia zplodil Asu,
  Asa zplodil Josafata,
  Josafat zplodil Jorama,
  Joram zplodil Oziáše,
  Oziáš zplodil Joatama,
  Joatam zplodil Achaze,
  Achaz zplodil Ezechiáše,
  Ezechiáš zplodil Manasesa,
  Manases zplodil Amona,
  Amon zplodil Josiáše,
  Josiáš zplodil Jechoniáše a jeho bratry;
  tehdy došlo k odvlečení do Babylonu.
  Po odvlečení do Babylonu:
  Jechoniáš zplodil Salatiela,
  Salatiel zplodil Zorobabela,
  Zorobabel zplodil Abiuda,
  Abiud zplodil Eljakima,
  Eljakim zplodil Azora,
  Ator zplodil Sadoka,
  Sadok zplodil Achima,
  Achim zplodil Eliuda,
  Eliud zplodil Eleazara,
  Eleatar zplodil Matana,
  Matan zplodil Jakuba,
  Jakub zplodil Josefa, manžela Marie,
  z níž se narodil Ježíš, který se nazývá Kristus.}}

\newcommand{\trTeDecetLaus}{\translatioCantus{Tobě chvála, Tobě zpěvy, Tobě
sláva, Bohu Otci i~Synu i~Svatému Duchu, na věky věků. \Rbardot{} Amen.}}

% MASS ---

\newcommand{\trIntroitus}{\translatioCantus{Radujme se všichni
v Pánu, slavíce svátek ke cti Panny Marie: z něj se radují andělé
a spoluchválí Božího Syna. \textit{\color{red}Žl.} Má ústa vydala dobré slovo,
přednáším svá díla králi.}}

\newcommand{\trGraduale}{\translatioCantus{Požehnaná a ctihodná jsi,
Panno Maria: nedotčená (co do panenství) jsi byla shledána matkou
Spasitele. \Vbardot{} Panno Boží Rodičko, ten, jehož nepojme ani celý svět,
se uzavřel do tvých útrob, když se stal člověkem.}}

\newcommand{\trAlleluia}{\translatioCantus{Aleluja. \Vbardot{} Skvělá slavnost
slavné Panny Marie, z pokolení (dosl. ze semene) Abrahámova, vzešlé z kmene 
Judova, z rodu Davidova.}}

\newcommand{\trOffertorium}{\translatioCantus{Blažená jsi, Panno Maria,
tys nosila Stvořitele všeho; porodila jsi toho, který tě utvořil,
a na věky zůstáváš Pannou.}}

\newcommand{\trCommunio}{\translatioCantus{Budou mě blahoslavit
všechna pokolení, protože mi učinil veliké věci ten, který je mocný.}}

% LITTLE HOURS ---

\newcommand{\trVersusTertia}{\translatioCantus{\Vbardot{} \Rbardot{}}}

\newcommand{\trCapituliEtSic}{\translatioCantus{
Tak jsem se usadila na Sionu a v milovaném městě jsem nalezla odpočinek,
v Jeruzalémě vykonávám svou moc.
Zakořenila jsem u lidu plného slývy, na panství Páně, v jeho dědictví.}}

\newcommand{\trVersusSexta}{\translatioCantus{\Vbardot{} \Rbardot{}}}

\newcommand{\trCapituliInPlateis}{\translatioCantus{
Na planině jako skořicovník a akant jsem vydávala vůni, jako vybraná myrha
jsem voněla.}}

\newcommand{\trVersusNona}{\translatioCantus{\Vbardot{} \Rbardot{}}}
 % Czech translations of the proper texts

\newcommand{\annusEditionis}{2020}

%%%% Vicekrat opakovane kousky

\newcommand{\anteOrationem}{
  \rubrica{Ante Orationem, cantatur a Superiore:}

  \pars{Supplicatio Litaniæ.}

  \cuminitiali{}{temporalia/supplicatiolitaniae.gtex}

  \pars{Oratio Dominica.}

  \cuminitiali{}{temporalia/oratiodominica.gtex}

  \rubrica{Deinde dicitur ab Hebdomadario:}

  \cuminitiali{}{temporalia/dominusvobiscum-solemnis.gtex}

  \rubrica{In choro monialium loco Dominus vobiscum dicitur:}

  \sineinitiali{temporalia/domineexaudi.gtex}
}

\setlength{\columnsep}{30pt} % prostor mezi sloupci

%%%%%%%%%%%%%%%%%%%%%%%%%%%%%%%%%%%%%%%%%%%%%%%%%%%%%%%%%%%%%%%%%%%%%%%%%%%%%%%%%%%%%%%%%%%%%%%%%%%%%%%%%%%%%
\begin{document}

% Here we set the space around the initial.
% Please report to http://home.gna.org/gregorio/gregoriotex/details for more details and options
\grechangedim{afterinitialshift}{2.2mm}{scalable}
\grechangedim{beforeinitialshift}{2.2mm}{scalable}
\grechangedim{interwordspacetext}{0.22 cm plus 0.15 cm minus 0.05 cm}{scalable}%
\grechangedim{annotationraise}{-0.2cm}{scalable}

% Here we set the initial font. Change 38 if you want a bigger initial.
% Emit the initials in red.
\grechangestyle{initial}{\color{red}\fontsize{38}{38}\selectfont}

\pagestyle{empty}

%%%% Titulni stranka
\begin{titulusOfficii}
\titulus
\end{titulusOfficii}

\vfill

\begin{center}
%Ad usum et secundum consuetudines chori \guillemotright{}Conventus Choralis\guillemotleft.

%Editio Sancti Wolfgangi \annusEditionis
\end{center}

\pagebreak

\renewcommand{\headrulewidth}{0pt} % no horiz. rule at the header
\fancyhf{}
\pagestyle{fancy}

\cantusSineNeumas

\pars{Oratio ante divinum Officium.}

\lettrine{{\color{red}A}}{peri,} Dómine, os meum ad benedicéndum nomen sanctum tuum:
munda quoque cor meum ab ómnibus vanis, pervérsis, et aliénis
cogitatiónibus:
intelléctum illúmina, afféctum inflámma,
ut digne, atténte ac devóte hoc Offícium recitáre váleam,
et exaudíri mérear ante conspéctum Divínæ Maiestátis tuæ.
Per Christum, Dóminum nostrum.
\Rbardot{} Amen.

Dómine, in unióne illíus divínæ intentiónis,
qua ipse in terris laudes Deo persolvísti,
has tibi Horas \rubricatum{(vel \textnormal{hanc tibi Horam})} persólvo.

%\trOratioAnteOfficium

\vfill

\pars{Oratio post divinum Officium.}

\rubrica{
  Orationem sequentem devote post Officium recitantibus
  Leo Papa X. defectus, et culpas in eo persolvendo ex humana
  fragilitate contractas, indulsit, et dicitur flexis genibus.
}

\lettrine{{\color{red}S}}{acrosánctæ} et indivíduæ Trinitáti,
crucifíxi Dómini nostri Iesu Christi humanitáti,
beatíssimæ et gloriosíssimæ sempérque Vírginis Maríæ
fecúndæ integritáti, 
et ómnium Sanctórum universitáti
sit sempitérna laus, honor, virtus et glória
ab omni creatúra,
nobísque remíssio ómnium peccatórum,
per infiníta sǽcula sæculórum.
\Rbardot{} Amen.

\noindent \Vbardot{} Beáta víscera Maríæ Virginis, quæ portavérunt
ætérni Patris Fílium.\\
\Rbardot{} Et beáta úbera, quæ lactavérunt Christum Dominum.

\rubrica{Et dicitur secreto \textnormal{Pater noster.} et \textnormal{Ave María.}}

%\trOratioPostOfficium

\vfill

\pars{} \scriptura{}

\hora{Ad Matutinum.} %%%%%%%%%%%%%%%%%%%%%%%%%%%%%%%%%%%%%%%%%%%%%%%%%%%%%%%%%%
%\sideThumbs{Matutinum}

\vspace{2mm}

\cantusSineNeumas

\cuminitiali{}{temporalia/dominelabiamea.gtex}

%\vspace{5mm}

\vfill
\pagebreak

\pars{Invitatorium.} \scriptura{\textbf{Aug. LX 182r}}

\vspace{-2mm}

\antiphona{II}{temporalia/inv-exsultemusdomino.gtex}

\vfill
\pagebreak

\pars{Hymnus.}

\vspace{-5mm}

\antiphona{IV}{temporalia/hym-UrbsJerusalem.gtex}

\vfill
\pagebreak

\pars{Psalmus 1.} \scriptura{Ps. 23, 7; \textbf{H327}}

\vspace{-4mm}

\antiphona{III a}{temporalia/ant-tolliteportas.gtex}

%\trMatAntI

\scriptura{Psalmus 23.}

\initiumpsalmi{temporalia/ps23-initium-iii-a-auto.gtex}

%\psalmusEtTranslatioT{temporalia/ps23-comb.tex}{10cm}
\input{temporalia/ps23.tex} \Abardot{}

\vfill
\pagebreak

\pars{Psalmus 2.} \scriptura{Gn. 28, 21-22; \textbf{H327}}

\vspace{-4mm}

\antiphona{II* b}{temporalia/ant-eritmihi.gtex}

%\trMatAntII

\scriptura{Psalmus 45.}

\initiumpsalmi{temporalia/ps45-initium-ii_-B-auto.gtex}

%\psalmusEtTranslatioT{temporalia/ps45-comb.tex}{10cm}
\input{temporalia/ps45.tex} \Abardot{}

\vfill
\pagebreak

\pars{Psalmus 3.} \scriptura{Ex. 24, 4; \textbf{H327}}

\vspace{-4mm}

\antiphona{VI F}{temporalia/ant-aedificavitmoyses.gtex}

%\trMatAntIII

\scriptura{Psalmus 47.}

\initiumpsalmi{temporalia/ps47-initium-vi-F-auto.gtex}

%\psalmusEtTranslatioT{temporalia/ps47-comb.tex}{10cm}
\input{temporalia/ps47.tex} \Abardot{}

\vfill
\pagebreak

\pars{Psalmus 4.} \scriptura{Gn. 28, 12; \textbf{H328}}

\vspace{-4mm}

\antiphona{VII c}{temporalia/ant-nonesthic.gtex}

%\trMatAntIV

\scriptura{Psalmus 83.}

\initiumpsalmi{temporalia/ps83-initium-vii-c-auto.gtex}

%\psalmusEtTranslatioT{temporalia/ps83-comb.tex}{10cm}
\input{temporalia/ps83.tex} \Abardot{}

\vfill
\pagebreak

\pars{Psalmus 5.} \scriptura{Gn. 28, 12; \textbf{H328}}

\vspace{-4mm}

\antiphona{VII a}{temporalia/ant-viditjacobscalam.gtex}

%\trMatAntV

\scriptura{Psalmus 86.}

\initiumpsalmi{temporalia/ps86-initium-vii-a-auto.gtex}

%\psalmusEtTranslatioT{temporalia/ps86-comb.tex}{10cm}
\input{temporalia/ps86.tex} \Abardot{}

\vfill
\pagebreak

\pars{Psalmus 6.} \scriptura{Bar. 2, 16}

\vspace{-4mm}

\antiphona{VI F}{temporalia/ant-respicedomine.gtex}

%\trMatAntVI

%\vspace{-2mm}

\scriptura{Psalmus 87.}

%\vspace{-2mm}

\initiumpsalmi{temporalia/ps87-initium-vi-F-auto.gtex}

%\vspace{-1mm}

%\psalmusEtTranslatioT{temporalia/ps87-comb.tex}{10cm}
\input{temporalia/ps87.tex}

\vfill

\antiphona{}{temporalia/ant-respicedomine.gtex}

\vfill
\pagebreak

\pars{Versus.} \scriptura{Ps. 92, 5}

\sineinitiali{temporalia/versus-domumtuam-communis.gtex}

\vspace{5mm}

\sineinitiali{temporalia/oratiodominica-mat.gtex}

\vspace{5mm}

\pars{Absolutio.}

\cuminitiali{}{temporalia/absolutio-exaudi.gtex}

%\trMatAbsolutioI

\vfill
\pagebreak

\cuminitiali{}{temporalia/benedictio-solemn-benedictione.gtex}

%\trMatBenedictioI

\vspace{7mm}

\lectioi

\noindent \Vbardot{} Tu autem, Dómine, miserére nobis.
\noindent \Rbardot{} Deo grátias.

\vfill
\pagebreak

\pars{Responsorium 1.} \scriptura{\Rbardot{} Cf. 2 Chr. 7, 6 \Vbardot{} Is. 2, 2; \textbf{H327}}

\vspace{-5mm}

\responsorium{I}{temporalia/resp-indedicationetempli-CROCHU.gtex}{}

\vfill
\pagebreak

\cuminitiali{}{temporalia/benedictio-solemn-unigenitus.gtex}

%\trMatBenedictioII

\vspace{7mm}

\lectioii

\noindent \Vbardot{} Tu autem, Dómine, miserére nobis.
\noindent \Rbardot{} Deo grátias.

\vfill
\pagebreak

\pars{Responsorium 2.} \scriptura{\Rbardot{} Is. 2, 2 \Vbardot{} Ps. 125, 6; \textbf{H328}}

\vspace{-5mm}

\responsorium{II}{temporalia/resp-fundataest-CROCHU.gtex}{}

\vfill
\pagebreak

\cuminitiali{}{temporalia/benedictio-solemn-spiritus.gtex}

%\trMatBenedictioIII

\vspace{7mm}

\lectioiii

\noindent \Vbardot{} Tu autem, Dómine, miserére nobis.
\noindent \Rbardot{} Deo grátias.

\vfill
\pagebreak

\pars{Responsorium 3.} \scriptura{\Rbardot{} 1 Sam. 7, 29 \Vbardot{} Ps. 92, 5; \textbf{H328}}

\vspace{-5mm}

\responsorium{VIII}{temporalia/resp-benedicdominedomum-CROCHU.gtex}{}

\vfill
\pagebreak

\ifx\duplexmaius\undefined
% Te Deum

\pars{Hymnus Ambrosianus} \scriptura{Tonus Solemnis}

\vspace{-2mm}

\grechangedim{interwordspacetext}{0.26 cm plus 0.15 cm minus 0.05 cm}{scalable}%
\cuminitiali{III}{temporalia/tedeum-solemnis-gn.gtex}
\grechangedim{interwordspacetext}{0.22 cm plus 0.15 cm minus 0.05 cm}{scalable}%

%\trTeDeum

\vfill
\pagebreak
\fi

\sineinitiali{temporalia/domineexaudi.gtex}

\vfill

\pars{Oratio.}

\ifx\duplexmaius\undefined
\cuminitiali{}{temporalia/oratio.gtex}
\else
\cuminitiali{}{temporalia/oratio2.gtex}
\fi
%\trOrationis

\vfill

\noindent \Vbardot{} Dómine, exáudi oratiónem meam.
\Rbardot{} Et clamor meus ad te véniat.

\vfill

% Nocturnale Romanum 2002, p. LXXVI Benedicamus Domino seems to match
% the one from Solemn Laudes.
\cuminitiali{V}{temporalia/benedicamus-solemnis-laud.gtex}

\vfill

\noindent \Vbardot{} Fidélium ánimæ per misericórdiam Dei requiéscant in pace.
\Rbardot{} Amen.

%\trFideliumAnimae

\vfill
\pagebreak

\hora{Ad Laudes.} %%%%%%%%%%%%%%%%%%%%%%%%%%%%%%%%%%%%%%%%%%%%%%%%%%%%%%%%%%
%\sideThumbs{Laudes}

% Psalmi festivi (AM33, pg. 721):
% 66 // 92, 99, 62, Dan3, 148+149+150

\vspace{4mm}

\cuminitiali{}{temporalia/deusinadiutorium-alter.gtex}
%\vspace{1cm}

\cantusSineNeumas

\vspace{4mm}

\pars{Psalmus 1.} \scriptura{Ps. 92, 5; \textbf{H329}}

\vspace{-4mm}

\antiphona{VII a}{temporalia/ant-domumtuam.gtex}

%\trAntI

%\vspace{-2mm}

\scriptura{Ps. 92}

%\vspace{-2mm}

\initiumpsalmi{temporalia/ps92-initium-vii-a-auto.gtex}

%\vspace{-1.5mm}

%\psalmusEtTranslatioT{temporalia/ps92-comb.tex}{10cm}
\input{temporalia/ps92.tex} \Abardot{}

\vfill
\pagebreak

\pars{Psalmus 2.} \scriptura{Mt. 7, 24.25; \textbf{H329}}

\antiphona{I f}{temporalia/ant-haecestdomus.gtex}

%\trAntII

\scriptura{Ps. 99}

\initiumpsalmi{temporalia/ps99-initium-i-f-auto.gtex}

%\psalmusEtTranslatioT{temporalia/ps99-comb.tex}{10cm}
\input{temporalia/ps99.tex} \Abardot{}

\vfill
\pagebreak

\pars{Psalmus 3.} \scriptura{Is. 56, 8; Mt. 21, 13; Mc. 11, 17; \textbf{H329}}

\vspace{-4mm}

\antiphona{I D}{temporalia/ant-domusmea.gtex}

%\vspace{-2mm}

%\trAntIII

\scriptura{Ps. 62.}

\initiumpsalmi{temporalia/ps62-initium-i-D-auto.gtex}

%\vspace{-6mm}

%\psalmusEtTranslatioT{temporalia/ps62-comb.tex}{10cm}
\input{temporalia/ps62.tex} \Abardot{}

\vfill
\pagebreak

\pars{Psalmus 4.} \scriptura{Mt. 7, 24.25; \textbf{H329}}

\vspace{-4mm}

\antiphona{VIII c}{temporalia/ant-benefundataest.gtex}

\vspace{-2mm}

%\trAntIV

\scriptura{Canticum trium puerorum, Dan. 3, 57-88 et 56}

\vspace{-2mm}

\initiumpsalmi{temporalia/dan3-initium-viii-c-auto.gtex}

%\psalmusEtTranslatioT{temporalia/dan3-comb.tex}{10cm}
\input{temporalia/dan3.tex}

\rubrica{Hic non dicitur Gloria Patri, neque Amen.}
\vspace{1cm}

\antiphona{}{temporalia/ant-benefundataest.gtex} % repeat the antiphon - new page

\vfill
\pagebreak

\pars{Psalmus 5.} \scriptura{Ap. 21, 18.19; \textbf{H329}}

\vspace{-4mm}

\antiphona{I g}{temporalia/ant-lapidespretiosi.gtex}

\vspace{-2mm}

%\trAntV

\scriptura{Ps. 148}

\vspace{-2mm}

\initiumpsalmi{temporalia/ps148-initium-i-g-auto.gtex}

%\vspace{-1.5mm}

%\psalmusEtTranslatioT{temporalia/ps148-comb.tex}{10cm}
\input{temporalia/ps148.tex}

\rubrica{Hic non dicitur Gloria Patri.}

\vspace{-5mm}

\vfill
\pagebreak

%
\scriptura{Ps. 149}

\initiumpsalmi{temporalia/ps149-initium-i-g-auto.gtex}

%\psalmusEtTranslatioT{temporalia/ps149-comb.tex}{10cm}
\input{temporalia/ps149.tex}

\rubrica{Hic non dicitur Gloria Patri.}

\vfill
\pagebreak

%
\scriptura{Ps. 150}

\initiumpsalmi{temporalia/ps150-initium-i-g-auto.gtex}

%\psalmusEtTranslatioT{temporalia/ps150-comb.tex}{10cm}
\input{temporalia/ps150.tex}

\antiphona{}{temporalia/ant-lapidespretiosi.gtex} % repeat the antiphon - new page

\vfill
\pagebreak

\cantusSineNeumas

\pars{Capitulum.} \scriptura{Ap. 21, 2}

\cuminitiali{}{temporalia/capitulum-VidiCivitatem.gtex}

% preklad Jeruz. bible
%\trCapituli

\vfill
\pars{Responsorium breve.} \scriptura{Ps. 92, 5}

\antiphona{VI}{temporalia/resp-domumtuam.gtex}

%\trRespVesp

\vfill
\pagebreak

% Hymnus. %%%
\pars{Hymnus.}

{
\grechangedim{interwordspacetext}{0.20 cm plus 0.15 cm minus 0.05 cm}{scalable}%
\ifx\duplexmaius\undefined
\cuminitiali{IV}{temporalia/hym-AngularisFundamentum.gtex}
\else
\cuminitiali{IV}{temporalia/hym-IamBone.gtex}
\fi
\grechangedim{interwordspacetext}{0.22 cm plus 0.15 cm minus 0.05 cm}{scalable}%
}
%\input{cantus/amon33/hym-TibiChriste-bohtext.tex}

\vfill

\pars{Versus.} \scriptura{Mt. 7, 24.25}

% Versus. %%%
\sineinitiali{temporalia/versus-haecest.gtex}
    
\noindent %\trVersus

\vfill
\pagebreak

\pars{Canticum Zachariæ.} \scriptura{Gn. 28, 18.17; \textbf{H329}}

\vspace{-4mm}

\antiphona{IV E}{temporalia/ant-manesurgensjacob.gtex}

%\trAntBenedictus

%\vspace{-3mm}

\scriptura{Lc. 1, 68-79}

%\vspace{-2mm}

\initiumpsalmi{temporalia/benedictus-initium-ivsoll-E.gtex}

%\vspace{-1.5mm}

%\psalmusEtTranslatioT{temporalia/benedictus-comb.tex}{10cm}
\input{temporalia/benedictus.tex}

\vfill

\antiphona{}{temporalia/ant-manesurgensjacob.gtex}

\vfill
\pagebreak

\cantusSineNeumas

\anteOrationem

\pagebreak

% Oratio. %%%
\pars{Oratio.}

\ifx\duplexmaius\undefined
\cuminitiali{}{temporalia/oratio.gtex}
\else
\cuminitiali{}{temporalia/oratio2.gtex}
\fi
%\trOrationis

\vfill

\rubrica{Hebdomadarius dicit iterum Dominus vobiscum, vel cantor dicit:}

\vspace{2mm}

\sineinitiali{temporalia/domineexaudi.gtex}

\rubrica{Postea cantatur a cantore:}

\vspace{2mm}

\ifx\duplexmaius\undefined
\cuminitiali{II}{temporalia/benedicamus-solemnism-laud.gtex}
\else
\cuminitiali{VIII}{temporalia/benedicamus-duplexmajus-laudes.gtex}
\fi

\vspace{1mm}

\vfill
\pagebreak

\hora{Ad Vesperas.} %%%%%%%%%%%%%%%%%%%%%%%%%%%%%%%%%%%%%%%%%%%%%%%%%%%%%
%\sideThumbs{II. Vesperæ}

%\vspace{5mm}
\grechangedim{interwordspacetext}{0.18 cm plus 0.15 cm minus 0.05 cm}{scalable}%
\cuminitiali{}{temporalia/deusinadiutorium-solemnis.gtex}
\grechangedim{interwordspacetext}{0.22 cm plus 0.15 cm minus 0.05 cm}{scalable}%

\vspace{3mm}

%\vfill
%\pagebreak

\pars{Psalmus 1.} \scriptura{Ps. 92, 5; \textbf{H329}}

\vspace{-5mm}

\antiphona{VII a}{temporalia/ant-domumtuam.gtex}

\vspace{-2mm}

%\trAntI

\scriptura{Ps. 109}

\vspace{-2mm}

\initiumpsalmi{temporalia/ps109-initium-vii-a-auto.gtex}

\vspace{-1mm}

%\psalmusEtTranslatioT{temporalia/ps109-comb.tex}{10cm}
\input{temporalia/ps109.tex} \Abardot{}

\vfill
\pagebreak

\pars{Psalmus 2.} \scriptura{Mt. 7, 24.25; \textbf{H329}}

\antiphona{I f}{temporalia/ant-haecestdomus.gtex}

%\trAntII

\scriptura{Ps. 121}

\initiumpsalmi{temporalia/ps121-initium-i-f-auto.gtex}
%\psalmusEtTranslatioT{temporalia/ps121-comb.tex}{10cm}
\input{temporalia/ps121.tex} \Abardot{}

\vfill
\pagebreak

\pars{Psalmus 3.} \scriptura{Is. 56, 7; Mt. 21, 13; Mc. 11, 17; \textbf{H329}}

\antiphona{I D}{temporalia/ant-domusmea.gtex}

%\trAntIII

\scriptura{Ps. 126}

\initiumpsalmi{temporalia/ps126-initium-i-D-auto.gtex}
%\psalmusEtTranslatioT{temporalia/ps126-comb.tex}{10cm}
\input{temporalia/ps126.tex} \Abardot{}

\vfill
\pagebreak

\pars{Psalmus 4.} \scriptura{Ap. 21, 18.19; \textbf{H329}}

\antiphona{I g}{temporalia/ant-lapidespretiosi.gtex}

%\trAntIV

\scriptura{Ps. 147}

\initiumpsalmi{temporalia/ps147-initium-i-g-auto.gtex}
%\psalmusEtTranslatioT{temporalia/ps147-comb.tex}{10cm}
\input{temporalia/ps147.tex} \Abardot{}

\vfill
\pagebreak

% Capitulum. %%%
\pars{Capitulum.} \scriptura{Ap. 21, 2}

\cuminitiali{}{temporalia/capitulum-VidiCivitatem.gtex}

% preklad Jeruz. bible
%\trCapituli

\vfill
\pars{Responsorium breve.} \scriptura{Ps. 92, 5}

\vspace{-5mm}

\responsorium{VI}{temporalia/resp-domumtuam.gtex}{}

%\trRespVesp

\vfill
\pagebreak

% Hymnus. %%%
\pars{Hymnus.}

{
\grechangedim{interwordspacetext}{0.20 cm plus 0.15 cm minus 0.05 cm}{scalable}%
\cuminitiali{IV}{temporalia/hym-UrbsBeata.gtex}
\grechangedim{interwordspacetext}{0.22 cm plus 0.15 cm minus 0.05 cm}{scalable}%
}
%\input{cantus/amon33/hym-UrbsBeata-bohtext.tex}

\vfill
%\pagebreak

\pars{Versus.} \scriptura{Ps. 92, 5}

% Versus. %%%
\sineinitiali{temporalia/versus-domumtuam.gtex}
    
\noindent %\trVersus

\vfill
\pagebreak

\pars{Canticum B. Mariæ V.} \scriptura{Lc. 19, 5.6.9; \textbf{H330}}

\vspace{-6mm}

\antiphona{VIII G\textsuperscript{2}}{temporalia/ant-zachaeefestinans.gtex}

%\trAntMagnificatII

\vspace{-3mm}

\scriptura{Lc. 1, 46-55}

\vspace{-2mm}

\initiumpsalmi{temporalia/magnificat-initium-viiisoll-G2.gtex}

\vspace{-1.5mm}

%\psalmusEtTranslatioT{temporalia/magnificat-comb.tex}{10.3cm}
\input{temporalia/magnificat.tex} \Abardot{}

\vfill
\pagebreak

\anteOrationem

\pagebreak

%% Oratio. %%%
\pars{Oratio.}

\ifx\duplexmaius\undefined
\cuminitiali{}{temporalia/oratio.gtex}
\else
\cuminitiali{}{temporalia/oratio2.gtex}
\fi
%\trOrationis

\vfill

\rubrica{Hebdomadarius dicit iterum Dominus vobiscum, vel cantor dicit:}

\vspace{2mm}

\sineinitiali{temporalia/domineexaudi.gtex}

\rubrica{Postea cantatur a cantore:}

\vspace{2mm}

\ifx\duplexmaius\undefined
\cuminitiali{II}{temporalia/benedicamus-solemnism-2vesp.gtex}
\else
\cuminitiali{II}{temporalia/benedicamus-duplexmajus-vesperae.gtex}
\fi

\vspace{1mm}

\vfill
\pagebreak

\end{document}

