\newcommand{\feria}{Sabbato in Albis.}
\newcommand{\versusMatutinum}{\noindent \Vbardot{} Gavísi sunt discípuli, allelúia.
\noindent \Rbardot{} Viso Dómino, allelúia.}
\newcommand{\absolutio}{\cuminitiali{}{temporalia/absolutio-avinculis.gtex}}
\newcommand{\lectioi}{\pars{Lectio I.} \scriptura{Io. 20, 1-9}

\noindent Léctio sancti Evangélii secúndum Ioánnem.

\noindent In illo témpore: Una sabbáti María Magdaléne venit mane, cum adhuc ténebræ essent, ad monuméntum. Et réliqua.

\vspace{7mm}

\noindent Homilía sancti Gregórii Papæ.

\scriptura{Homilia 22 in Evangelia}

\noindent Léctio sancti Evangélii, quam modo, fratres, audístis, valde in superfície histórica est apérta: sed eius nobis sunt mystéria sub brevitáte requirénda. María Magdaléne, cum adhuc ténebræ essent, venit ad monuméntum. Iuxta históriam notátur hora: iuxta intelléctum vero mýsticum, requiréntis signátur intellegéntia. María étenim auctórem ómnium, quem in carne víderat mórtuum, quærébat in monuménto; et quia hunc mínime invénit, furátum crédidit. Adhuc ergo erant ténebræ, cum venit ad monuméntum. Cucúrrit cítius, discípulis nuntiávit: sed illi præ céteris cucurrérunt, qui præ céteris amavérunt, vidélicet Petrus et Ioánnes.}
\newcommand{\responsoriumi}{\pars{Responsorium 1.} \scriptura{\Rbardot{} Aug. Serm. 362; \Vbardot{} ibid. Enarr. in Psalmos: Ps: 102, sermo 2; \textbf{H203}}

\vspace{-5mm}

\responsorium{II}{temporalia/resp-christusresurgens.gtex}{}}
\newcommand{\lectioii}{\pars{Lectio II.}

\noindent Currébant autem duo simul: sed Ioánnes præcucúrrit cítius Petro. Venit prior ad monuméntum, et íngredi non præsúmpsit. Venit ergo postérior Petrus, et intrávit. Quid, fratres, quid cursus signíficat? Numquid hæc tam subtílis Evangelístæ descríptio a mystériis vacáre credénda est? Mínime. Neque enim se Ioánnes et præísse, et non intrásse díceret, si in ipsa sui trepidatióne mystérium defuísse credidísset. Quid ergo per Ioánnem, nisi synagóga: quid per Petrum, nisi Ecclésia designátur?}
\newcommand{\responsoriumii}{\pars{Responsorium 2.} \scriptura{\Rbardot{} Apoc. 7, 17; \Vbardot{} ibid. 7, 9; \textbf{H236}}

\vspace{-5mm}

\responsorium{VII}{temporalia/resp-istisuntagni.gtex}{}}
\newcommand{\lectioiii}{\pars{Lectio III.}

\noindent Nec mirum esse videátur, quod per iuniórem synagóga, per seniórem vero Ecclésia signári perhibétur: quia etsi ad Dei cultum prior est synagóga, quam Ecclésia géntium, ad usum tamen sǽculi prior est multitúdo géntium, quam synagóga, Paulo attestánte, qui ait: Quia non prius quod spiritále est, sed quod animále. Per seniórem ergo Petrum significátur Ecclésia géntium: per iuniórem vero Ioánnem synagóga Iudæórum. Currunt ambo simul: quia ab ortus sui témpore usque ad occásum, pari et commúni via, etsi non pari et commúni sensu, gentílitas cum synagóga cucúrrit. Venit synagóga prior ad monuméntum, sed mínime intrávit: quia legis quidem mandáta percépit, prophetías de incarnatióne ac passióne Domínica audívit, sed crédere in mórtuum nóluit.}
\newcommand{\oratioMatutinum}{\noindent Concéde, quǽsumus, omnípotens Deus:~\gredagger{} ut, qui festa paschália venerándo égimus,~\grestar{} per hæc contíngere ad gáudia ætérna mereámur. Per Dóminum.}
\newcommand{\oratioLaudes}{\cuminitiali{}{temporalia/oratio7.gtex}}
\newcommand{\benedictus}{\pars{Canticum Zachariæ.} \scriptura{\Abardot{} Io. 20, 4; \textbf{H238}}

\vspace{-7mm}

\antiphona{I f}{temporalia/ant-currebant.gtex}

\vspace{-3mm}

\scriptura{Lc. 1, 68-79}

\vspace{-2.5mm}

\initiumpsalmi{temporalia/benedictus-initium-i-f-auto.gtex}

\vspace{-1.5mm}

%\psalmusEtTranslatioT{temporalia/benedictus3-comb.tex}{10cm}
\input{temporalia/benedictus3.tex} \Abardot{}}
% LuaLaTeX

\documentclass[a4paper, twoside, 12pt]{article}
\usepackage[latin]{babel} 
%\usepackage[landscape, left=3cm, right=1.5cm, top=2cm, bottom=1cm]{geometry} % okraje stranky
%\usepackage[landscape, a4paper, mag=1166, truedimen, left=2cm, right=1.5cm, top=1.6cm, bottom=0.95cm]{geometry} % okraje stranky
\usepackage[landscape, a4paper, mag=1400, truedimen, left=0.5cm, right=0.5cm, top=0.5cm, bottom=0.5cm]{geometry} % okraje stranky

\usepackage{fontspec}
\setmainfont[FeatureFile={junicode.fea}, Ligatures={Common, TeX}, RawFeature=+fixi]{Junicode}
%\setmainfont{Junicode}

% shortcut for Junicode without ligatures (for the Czech texts)
\newfontfamily\nlfont[FeatureFile={junicode.fea}, Ligatures={Common, TeX}, RawFeature=+fixi]{Junicode}

% Hebrew font:
% http://scripts.sil.org/cms/scripts/page.php?site_id=nrsi&id=SILHebrUnic2
\newfontfamily\hebfont[Scale=1]{Ezra SIL}

\usepackage{multicol}
\usepackage{color}
\usepackage{lettrine}
\usepackage{fancyhdr}

% usual packages loading:
\usepackage{luatextra}
\usepackage{graphicx} % support the \includegraphics command and options
\usepackage{gregoriotex} % for gregorio score inclusion
\usepackage{gregoriosyms}
\usepackage{wrapfig} % figures wrapped by the text
\usepackage{parcolumns}
\usepackage[contents={},opacity=1,scale=1,color=black]{background}
\usepackage{tikzpagenodes}
\usepackage{calc}
\usepackage{longtable}
\usetikzlibrary{calc}

\setlength{\headheight}{14.5pt}

% Commands used to produce a typical "Conventus" booklet

\newenvironment{titulusOfficii}{\begin{center}}{\end{center}}
\newcommand{\dies}[1]{#1

}
\newcommand{\nomenFesti}[1]{\textbf{\Large #1}

}
\newcommand{\celebratio}[1]{#1

}

\newcommand{\hora}[1]{%
\vspace{0.5cm}{\large \textbf{#1}}

\fancyhead[LE]{\thepage\ / #1}
\fancyhead[RO]{#1 / \thepage}
\addcontentsline{toc}{subsection}{#1}
}

% larger unit than a hora
\newcommand{\divisio}[1]{%
\begin{center}
{\Large \textsc{#1}}
\end{center}
\fancyhead[CO,CE]{#1}
\addcontentsline{toc}{section}{#1}
}

% a part of a hora, larger than pars
\newcommand{\subhora}[1]{
\begin{center}
{\large \textit{#1}}
\end{center}
%\fancyhead[CO,CE]{#1}
\addcontentsline{toc}{subsubsection}{#1}
}

% rubricated inline text
\newcommand{\rubricatum}[1]{\textit{#1}}

% standalone rubric
\newcommand{\rubrica}[1]{\vspace{3mm}\rubricatum{#1}}

\newcommand{\notitia}[1]{\textcolor{red}{#1}}

\newcommand{\scriptura}[1]{\hfill \small\textit{#1}}

\newcommand{\translatioCantus}[1]{\vspace{1mm}%
{\noindent\footnotesize \nlfont{#1}}}

% pruznejsi varianta nasledujiciho - umoznuje nastavit sirku sloupce
% s prekladem
\newcommand{\psalmusEtTranslatioB}[3]{
  \vspace{0.5cm}
  \begin{parcolumns}[colwidths={2=#3}, nofirstindent=true]{2}
    \colchunk{
      \input{#1}
    }

    \colchunk{
      \vspace{-0.5cm}
      {\footnotesize \nlfont
        \input{#2}
      }
    }
  \end{parcolumns}
}

\newcommand{\psalmusEtTranslatio}[2]{
  \psalmusEtTranslatioB{#1}{#2}{8.5cm}
}


\newcommand{\canticumMagnificatEtTranslatio}[1]{
  \psalmusEtTranslatioB{#1}{temporalia/extra-adventum-vespers/magnificat-boh.tex}{12cm}
}
\newcommand{\canticumBenedictusEtTranslatio}[1]{
  \psalmusEtTranslatioB{#1}{temporalia/extra-adventum-laudes/benedictus-boh.tex}{10.5cm}
}

% volne misto nad antifonami, kam si zpevaci dokresli neumy
\newcommand{\hicSuntNeumae}{\vspace{0.5cm}}

% prepinani mista mezi notovymi osnovami: pro neumovane a neneumovane zpevy
\newcommand{\cantusCumNeumis}{
  \setgrefactor{17}
  \global\advance\grespaceabovelines by 5mm%
}
\newcommand{\cantusSineNeumas}{
  \setgrefactor{17}
  \global\advance\grespaceabovelines by -5mm%
}

% znaky k umisteni nad inicialu zpevu
\newcommand{\superInitialam}[1]{\gresetfirstlineaboveinitial{\small {\textbf{#1}}}{\small {\textbf{#1}}}}

% pars officii, i.e. "oratio", ...
\newcommand{\pars}[1]{\textbf{#1}}

\newenvironment{psalmus}{
  \setlength{\parindent}{0pt}
  \setlength{\parskip}{5pt}
}{
  \setlength{\parindent}{10pt}
  \setlength{\parskip}{10pt}
}

%%%% Prejmenovat na latinske:
\newcommand{\nadpisZalmu}[1]{
  \hspace{2cm}\textbf{#1}\vspace{2mm}%
  \nopagebreak%

}

% mode, score, translation
\newcommand{\antiphona}[3]{%
\hicSuntNeumae
\superInitialam{#1}
\includescore{#2}

#3
}
 % Often used macros
%%%% Preklady jednotlivych zpevu (nektere se opakuji, a je dobre mit je
% vsechny na jedne hromade)

\newcommand{\trOratioAnteOfficium}{\translatioCantus{Otevři, Pane, má ústa, abych chválil tvé svaté jméno.
Očisti mé srdce od všech marnivých, zvrácených a~jiných myšlenek, osvěť rozum, rozněť cit,
abych mohl důstojně, soustředěně a~zbožně recitovat a~vysloužil si být
vyslyšen před tváří tvé velebnosti. Skrze Krista…}}

\newcommand{\trOratioPostOfficium}{\translatioCantus{\textit{Následující modlitbu
opatřil pro ty, kdo ji zbožně vyřknou po hodinkách, zesnulý papež Lev X.
odpustky za hříchy vzniklé při konání hodinek z~lidské křehkosti. Říká se
vkleče.}
Svatosvaté a~nerozdílné Trojici, ukřižovanému lidství našeho Pána Ježíše
Krista, přeblažené a~přeslavné plodné neporušenosti vždy Panny Marie
i~souhrnu všech svatých buď ode všeho stvoření věčná chvála, čest a~sláva, nám
pak buď dáno odpuštění všech hříchů, po nekonečné věky věků. Amen.}}

% HOURS ---

\newcommand{\trAntI}{\translatioCantus{Jasné narození slavné Panny Marie,
z pokolení (dosl. ze semene) Abrahámova, vzešlé z kmene Judova, z rodu Davidova.}}
\newcommand{\trAntII}{\translatioCantus{Dnes je Narození svaté Panny 
Marie, jejíž předrahý život osvěcuje všechny církve.}}

\newcommand{\trAntIII}{\translatioCantus{Maria, jež vzešla 
z královského rodu, září; myslí i duchem ji zbožně prosíme, aby 
nám pomáhala svými přímluvami.}}

\newcommand{\trAntIV}{\translatioCantus{Srdcem i duchem pějme Kristu 
k slávě o této svaté slavnosti vznešené Rodičky Boží Marie.}}

\newcommand{\trAntV}{\translatioCantus{Příjemně \notitia{?} 
oslavujme Narození blahoslavené Marie,
aby se ona za nás přimlouvala u Pána Ježíše Krista.}}

\newcommand{\trCapituli}{\translatioCantus{Před věky, na počátku mě stvořil, potrvám věčně. Ve svatém Stanu jsem před ním konala službu.}}

\newcommand{\trRespVesp}{\translatioCantus{Buď zdráva, Maria,
plná milosti: \grestar{} Pán s tebou. \Vbardot{} Požehnaná jsi mezi ženami,
a požehnaný plod života (ve smyslu lůna, břicha) tvého.}}

\newcommand{\trVersus}{\translatioCantus{\Vbardot{} Dnes je Narození svaté Panny Marie. \Rbardot{} Jejíž předrahý život osvěcuje všechny církve.}}

\newcommand{\trAntMagnificatI}{\translatioCantus{Konejme památku
veledůstojného narození slavné Panny Marie,
jíž se dostalo mateřské důstojnosti bez ztráty panenské cudnosti.}}

% Tento preklad je vice nez nejisty a ani alternativy, ktere jsem
% videl, me nepresvedcily...
\newcommand{\trAntBenedictus}{\translatioCantus{Slavnostně slavme 
dnešní narození Marie, vždy Panny a Rodičky Boží: v něm se objevuje
vysokost trůnu (totiž Marie, trůnu Božího Syna), aleluja.}}

\newcommand{\trAntMagnificatII}{\translatioCantus{Tvé narození,
Bohorodičko Panno, vyhlásilo radost celému světu:
z tebe totiž vzešlo Slunce spravedlnosti, Kristus, náš Bůh:
jenž zrušil kletbu a dal nám požehnání: přemohl smrt a dal nám život věčný.}}

\newcommand{\trOrationis}{\translatioCantus{Prosíme tě, Bože, 
uděl svým služebníkům dar nebeské milosti,
aby těm, jimž slehnutím blahoslavené Panny vyvstal počátek spásy, 
slavnost k poctě jejího narození přinesla
rozhojnění pokoje.
Skrze tvého Syna, našeho Pána Ježíše Krista, který s tebou žije a kraluje,
Bůh, v jednotě Ducha svatého po všechny věky věků.}}

\newcommand{\trFideliumAnimae}{\translatioCantus{\Vbardot{} Duše věrných ať pro
milosrdenství Boží odpočívají v~pokoji. \Rbardot{} Amen.}}

% Completorium

\newcommand{\trJubeDomne}{\translatioCantus{Rač, pane, požehnat.}}

\newcommand{\trComplBenedictio}{\translatioCantus{Pokojnou noc a~svatou smrt
nechť nám dopřeje všemohoucí Pán. \Rbardot{} Amen.}}

\newcommand{\trComplLectioBr}{\translatioCantus{Buďte střízliví, bděte.
Váš protivník Ďábel obchází jako lev řvoucí a~hledá, koho by pohltil.
Postavte se proti němu pevní ve víře.  Ale ty, Pane, smiluj se nad námi.
\Rbardot{} Bohu díky.}}

\newcommand{\trComplAntI}{\translatioCantus{Rač se smilovati nade mnou,
Hospodine, a vyslyš mou modlitbu.}}

\newcommand{\trComplCapituli}{\translatioCantus{Jsi přece, Hospodine,
uprostřed nás a~jmenujeme se po tobě.  Neopouštěj nás, Pane, náš Bože.}}

\newcommand{\trRespCompl}{\translatioCantus{Do tvých rukou, Pane, \grestar{}
poroučím svého ducha. \Vbardot{} Ty mne zachráníš, Pane, Bože věrný.}}

\newcommand{\trComplVersus}{\translatioCantus{\Vbardot{} Střez mne jako zřítelnici oka,
aleluja. \Rbardot{} Ve stínu svých křídel uschovej mne, aleluja.}}

\newcommand{\trAntSalvaNos}{\translatioCantus{Ochraňuj nás, Pane, když
bdíme, a~buď s~námi, když spíme, ať bdíme s~Kristem a~odpočíváme v~pokoji.}}

\newcommand{\trComplOrationis}{\translatioCantus{Zavítej, prosíme, Pane, sem
do našeho příbytku a~daleko od něj zažeň všechny úklady nepřítele. Ať tu
bydlí tví svatí andělé a~tvoje požehnání buď nad ním stále. Skrze…}}

\newcommand{\trSalveRegina}{\translatioCantus{Zdrávas Královno, matko
milosrdenství, živote, sladkosti a naděje naše, buď zdráva!
K tobě voláme, vyhnaní synové Evy,
k tobě vzdycháme, lkajíce a plačíce
v tomto slzavém údolí.
A proto, orodovnice naše,
obrať k nám své milosrdné oči
a Ježíše, požehnaný plod života svého,
nám po tomto putování ukaž,
ó milostivá, ó přívětivá,
ó přesladká, Panno Maria!}}

\newcommand{\trOraProNobis}{\translatioCantus{\Vbardot{} 
Oroduj za nás, svatá Boží Rodičko,
\Rbardot{} aby nám Kristus dal účast na svých zaslíbeních.}}

% Matutinum

\newcommand{\trMatInvitatorium}{\translatioCantus{}}

\newcommand{\trMatVeniteA}{\translatioCantus{Pojďte, chvalme s~radostí Pána,
s~jásotem slavme Boha, svou spásu; předstupme před tvář jeho s~díky, písně plesu pějme jemu.}}

\newcommand{\trMatVeniteB}{\translatioCantus{Neboť Bůh veliký jest Hospodin, a~král nade všecky bohy.
Jsouť v~jeho ruce všecky hlubiny země, temena hor jsou majetek jeho.}}

\newcommand{\trMatVeniteC}{\translatioCantus{Jehoť jest moře, neb on je učinil; i~souš
je dílo jeho rukou. Pojďme, klanějme se, padněme, klekněme před Pánem, svým
tvůrcem. Jeť on Pán, náš Bůh, a~my jsme lid, jejž on vodí a~ovce, jež pase.}}

\newcommand{\trMatVeniteD}{\translatioCantus{Kéž byste poslechli dnes hlasu jeho:
,,Nezatvrzujte svých srdcí jak v~Hádce, jak v~Pokušení na poušti, kde vaši otcové pokoušeli mne,
zkoušeli mne, ač vídali skutky mé.``}}

\newcommand{\trMatVeniteE}{\translatioCantus{Čtyřicet roků mrzel jsem se na to pokolení
a~řekl jsem: ,,Lid je to myslí stále bloudící``! Oni však nechtěli znáti mé cesty, takže jsem
přisáhl ve svém hněvu: ,,Nedojdou odpočinku mého!\mbox{}``}}

\newcommand{\trMatAntI}{\translatioCantus{}}

\newcommand{\trMatAntII}{\translatioCantus{}}

\newcommand{\trMatAntIII}{\translatioCantus{}}

\newcommand{\trMatVersusI}{\translatioCantus{}}

\newcommand{\trMatAbsolutioI}{\translatioCantus{Vyslyš Pane Ježíši Kriste
prosby svých služebníků \gredagger{} a~smiluj se nad námi, \grestar{} jenž
s~Otcem a~Duchem…}}

\newcommand{\trMatBenedictioI}{\translatioCantus{Rač, pane, požehnat.
Věčný Otec nám stále žehnej. \Rbardot{} Amen.}}

\newcommand{\trMatLecI}{\translatioCantus{Kéž by mě zulíbal polibky svých úst. 
Tvé milování je nad víno lahodnější;
vybraně voní tvé voňavky;
rozlévající se olej je tvé jméno,
proto tě dívky milují.
Strhni mě za sebou, poběžme!
Král mě uvedl do svých komnat;
budeš nám radostí a jásotem.
Víc než víno oslavíme tvé milování;
věru po právu jsi milován!
Snědá jsem, a přece krásná, jeruzalémské dcery,
jako stany kedarské,
jako šalmské závěsy.
}}

\newcommand{\trMatRespI}{\translatioCantus{}}

\newcommand{\trMatBenedictioII}{\translatioCantus{Rač, pane, požehnat.
Jednorozený Boží Syn nám žehnej \grestar{} a nám pomáhej. \Rbardot{} Amen.}}

\newcommand{\trMatLecII}{\translatioCantus{Nehleďte na mou osmahlou pleť:
to mě slunce ožehlo.
Synové mé matky se na mne rozzlobili,
poslali mě hlídat vinice.
A svou vinici, tu jsem nehlídala!
Pověz mi tedy, ty, jehož miluje mé srdce:
kam zavedeš své stádo pást,
kde ho necháš za poledne odpočívat?
Abych už nebloudila jako tulačka
poblíž stád druhů tvých.
Nevíš-li to, nejrásnější z žen,
jdi po stopách stáda
a kůzlata svá zaveď, ať se pasou
poblíž obydlí pastýřů.
Ke své klisně zapřažené do vozu faraonova
tebe, mé milá, přirovnávám.
Stále krásné jsou tvé líce s náušnicemi
i tvé hrdlo v náhrdelnících.}}

\newcommand{\trMatRespII}{\translatioCantus{}}

\newcommand{\trMatBenedictioIII}{\translatioCantus{Rač, pane, požehnat.
Milost Ducha Svatého ať osvítí nám smysly \grestar{} i srdce. \Rbardot{} Amen.}}

\newcommand{\trMatLecIII}{\translatioCantus{Zhotovíme ti zlaté náušnice
a kuličky ze stříbra.
Když král stoluje,
vydechuje můj nard svou vůni.
Můj milý je polštářek s myrhou,
jenž mi odpočívá mezi ňadry.
Můj milý je hrozen šáchoru
ve vinicích v Engadi.
Jak jsi krásná, milá moje,
jak jsi krásná!
Tvé oči jsou holubice.
Jak jsi krásný, můj milý,
jak líbezný!
Naše lože je samá zeleň.
Trámoví našeho domu je z cedru,
naše ostění z cypřiše.}}

\newcommand{\trMatRespIII}{\translatioCantus{}}

\newcommand{\trMatAntIV}{\translatioCantus{}}

\newcommand{\trMatAntV}{\translatioCantus{}}

\newcommand{\trMatAntVI}{\translatioCantus{}}

\newcommand{\trMatVersusII}{\translatioCantus{}}

\newcommand{\trMatAbsolutioII}{\translatioCantus{
Tvá milost a laskavost nechť nám pomáhá, jenž žiješ a vládneš s Otcem a Svatým Duchem na věky věků.}}

\newcommand{\trMatBenedictioIV}{\translatioCantus{Rač, pane, požehnat.
Bůh Otec všemohoucí, \grestar{} buď k nám milostivý a odpouštějící. \Rbardot{} Amen.}}

\newcommand{\trMatLecIV}{\translatioCantus{}}

\newcommand{\trMatRespIV}{\translatioCantus{}}

\newcommand{\trMatBenedictioV}{\translatioCantus{}}

\newcommand{\trMatLecV}{\translatioCantus{}}

\newcommand{\trMatRespV}{\translatioCantus{}}

\newcommand{\trMatBenedictioVI}{\translatioCantus{Rač, pane, požehnat.
Bůh rozněť v nás oheň své lásky. \Rbardot{} Amen.}}

\newcommand{\trMatLecVI}{\translatioCantus{}}

\newcommand{\trMatRespVI}{\translatioCantus{}}

\newcommand{\trMatAntVII}{\translatioCantus{}}

\newcommand{\trMatAntVIII}{\translatioCantus{}}

\newcommand{\trMatAntIX}{\translatioCantus{}}

\newcommand{\trMatVersusIII}{\translatioCantus{}}

\newcommand{\trMatAbsolutioIII}{\translatioCantus{Z okovů našich hříchů,
\grestar{} vysvoboď nás všemohoucí a milosrdný Pán. \Rbardot{} Amen.}}

\newcommand{\trMatBenedictioVII}{\translatioCantus{Rač, pane, požehnat.
Čtení evangelia nechť je nám \grestar{} spásou a ochranou. \Rbardot{} Amen.}}

\newcommand{\trMatLecVIIa}{\translatioCantus{
  Rodokmen Ježíše Krista, syna Davidova, syna Abrahámova:
  Abrahám zplodil Izáka,
  Izák zplodil Jakuba.}}

\newcommand{\trMatLecVIIb}{\translatioCantus{}}

\newcommand{\trMatRespVII}{\translatioCantus{}}

\newcommand{\trMatBenedictioVIII}{\translatioCantus{Rač, pane, požehnat.
\Rbardot{} Amen.}}

\newcommand{\trMatLecVIII}{\translatioCantus{}}

\newcommand{\trMatRespVIII}{\translatioCantus{}}

\newcommand{\trMatBenedictioIX}{\translatioCantus{Rač, pane, požehnat.
Do společnosti občanů nebes \grestar{} ať nás dovede král andělů.
\Rbardot{} Amen.}}

\newcommand{\trMatLecIX}{\translatioCantus{}}

% from the Czech Liturgia horarum
\newcommand{\trTeDeum}{\begin{translatioMulticol}{3}

Bože, tebe chválíme, 
tebe, Pane, velebíme.

Tebe, věčný Otče, 
oslavuje celá země.

Všichni andělé, 
cherubové i~serafové,

všechny mocné nebeské zástupy 
bez ustání volají:

Svatý, Svatý, Svatý, 
Pán, Bůh zástupů.

Plná jsou nebesa i~země 
tvé vznešené slávy.

Oslavuje tě 
sbor tvých apoštolů,

chválí tě 
velký počet proroků,

vydává o~tobě svědectví 
zástup mučedníků;

a~po celém světě 
vyznává tě tvá církev:

neskonale velebný, 
všemohoucí Otče,

úctyhodný Synu Boží, 
pravý a~jediný,

božský Utěšiteli, 
Duchu svatý.

Kriste, Králi slávy, 
tys od věků Syn Boha Otce;

abys člověka vykoupil, 
stal ses člověkem a~narodil ses z~Panny;

zlomil jsi osten smrti 
a~otevřel věřícím nebe;

sedíš po Otcově pravici 
a~máš účast na jeho slávě.

Věříme, že přijdeš soudit, 

a~proto tě prosíme:
přispěj na pomoc svým služebníkům, 
vždyť jsi je vykoupil svou předrahou krví;

dej, ať se radují s~tvými svatými 
ve věčné slávě.

Zachraň, Pane, svůj lid, žehnej svému dědictví, 
veď ho a~stále pozvedej.

Každý den tě budeme velebit 
a~chválit tvé jméno po všechny věky.

Pomáhej nám i~dnes, 
ať se nedostaneme do područí hříchu.

Smiluj se nad námi, Pane, 
smiluj se nad námi.

Ať spočine na nás tvé milosrdenství, 
jak doufáme v~tebe.

Pane, k~tobě se utíkáme, 
ať nejsme zahanbeni na věky. 
\end{translatioMulticol}}

\newcommand{\trMatEvangelium}{\translatioCantus{
  Rodokmen Ježíše Krista, syna Davidova, syna Abrahámova:
  Abrahám zplodil Izáka,
  Izák zplodil Jakuba,
  Jakub zplodil Judu a jeho bratry,
  Juda zplodil Farese a Zaru z Tamary,
  Fares zplodil Esroma,
  Esrom zplodil Arama,
  Aram zplodil Aminadaba,
  Aminadab zplodil Naasona,
  Naason zplodil Salmona,
  Salmon zplodil Boaze z Rahaby,
  Boaz zplodil Jobeda z Rut,
  Jobed zplodil Jessea,
  Jesse zplodil krále Davida.
  David zplodil Šalomouna z Uriášovy ženy,
  Šalomoun zplodil Roboama,
  Roboam zplodil Abiu,
  Abia zplodil Asu,
  Asa zplodil Josafata,
  Josafat zplodil Jorama,
  Joram zplodil Oziáše,
  Oziáš zplodil Joatama,
  Joatam zplodil Achaze,
  Achaz zplodil Ezechiáše,
  Ezechiáš zplodil Manasesa,
  Manases zplodil Amona,
  Amon zplodil Josiáše,
  Josiáš zplodil Jechoniáše a jeho bratry;
  tehdy došlo k odvlečení do Babylonu.
  Po odvlečení do Babylonu:
  Jechoniáš zplodil Salatiela,
  Salatiel zplodil Zorobabela,
  Zorobabel zplodil Abiuda,
  Abiud zplodil Eljakima,
  Eljakim zplodil Azora,
  Ator zplodil Sadoka,
  Sadok zplodil Achima,
  Achim zplodil Eliuda,
  Eliud zplodil Eleazara,
  Eleatar zplodil Matana,
  Matan zplodil Jakuba,
  Jakub zplodil Josefa, manžela Marie,
  z níž se narodil Ježíš, který se nazývá Kristus.}}

\newcommand{\trTeDecetLaus}{\translatioCantus{Tobě chvála, Tobě zpěvy, Tobě
sláva, Bohu Otci i~Synu i~Svatému Duchu, na věky věků. \Rbardot{} Amen.}}

% MASS ---

\newcommand{\trIntroitus}{\translatioCantus{Radujme se všichni
v Pánu, slavíce svátek ke cti Panny Marie: z něj se radují andělé
a spoluchválí Božího Syna. \textit{\color{red}Žl.} Má ústa vydala dobré slovo,
přednáším svá díla králi.}}

\newcommand{\trGraduale}{\translatioCantus{Požehnaná a ctihodná jsi,
Panno Maria: nedotčená (co do panenství) jsi byla shledána matkou
Spasitele. \Vbardot{} Panno Boží Rodičko, ten, jehož nepojme ani celý svět,
se uzavřel do tvých útrob, když se stal člověkem.}}

\newcommand{\trAlleluia}{\translatioCantus{Aleluja. \Vbardot{} Skvělá slavnost
slavné Panny Marie, z pokolení (dosl. ze semene) Abrahámova, vzešlé z kmene 
Judova, z rodu Davidova.}}

\newcommand{\trOffertorium}{\translatioCantus{Blažená jsi, Panno Maria,
tys nosila Stvořitele všeho; porodila jsi toho, který tě utvořil,
a na věky zůstáváš Pannou.}}

\newcommand{\trCommunio}{\translatioCantus{Budou mě blahoslavit
všechna pokolení, protože mi učinil veliké věci ten, který je mocný.}}

% LITTLE HOURS ---

\newcommand{\trVersusTertia}{\translatioCantus{\Vbardot{} \Rbardot{}}}

\newcommand{\trCapituliEtSic}{\translatioCantus{
Tak jsem se usadila na Sionu a v milovaném městě jsem nalezla odpočinek,
v Jeruzalémě vykonávám svou moc.
Zakořenila jsem u lidu plného slývy, na panství Páně, v jeho dědictví.}}

\newcommand{\trVersusSexta}{\translatioCantus{\Vbardot{} \Rbardot{}}}

\newcommand{\trCapituliInPlateis}{\translatioCantus{
Na planině jako skořicovník a akant jsem vydávala vůni, jako vybraná myrha
jsem voněla.}}

\newcommand{\trVersusNona}{\translatioCantus{\Vbardot{} \Rbardot{}}}
 % Czech translations of the proper texts

\newcommand{\annusEditionis}{2020}

\def\hebinitial#1{%
\leavevmode{\newbox\hebbox\setbox\hebbox\hbox{\hebfont{#1}\hskip 1mm}\kern -\wd\hebbox\hbox{\hebfont{#1}\hskip 1mm}}%
}

%%%% Vicekrat opakovane kousky

\newcommand{\anteOrationem}{
  \rubrica{Ante Orationem, cantatur a Superiore:}

  \pars{Supplicatio Litaniæ.}

  \cuminitiali{}{temporalia/supplicatiolitaniae.gtex}

  \pars{Oratio Dominica.}

  \cuminitiali{}{temporalia/oratiodominica.gtex}

  \rubrica{Deinde dicitur ab Hebdomadario:}

  \cuminitiali{}{temporalia/dominusvobiscum-solemnis.gtex}

  \rubrica{In choro monialium loco Dominus vobiscum dicitur:}

  \sineinitiali{temporalia/domineexaudi.gtex}
}

\setlength{\columnsep}{30pt} % prostor mezi sloupci

%%%%%%%%%%%%%%%%%%%%%%%%%%%%%%%%%%%%%%%%%%%%%%%%%%%%%%%%%%%%%%%%%%%%%%%%%%%%%%%%%%%%%%%%%%%%%%%%%%%%%%%%%%%%%
\begin{document}

% Here we set the space around the initial.
% Please report to http://home.gna.org/gregorio/gregoriotex/details for more details and options
\grechangedim{afterinitialshift}{2.2mm}{scalable}
\grechangedim{beforeinitialshift}{2.2mm}{scalable}

\grechangedim{interwordspacetext}{0.32 cm plus 0.15 cm minus 0.05 cm}{scalable}%
\grechangedim{annotationraise}{-0.2cm}{scalable}

% Here we set the initial font. Change 38 if you want a bigger initial.
% Emit the initials in red.
\grechangestyle{initial}{\color{red}\fontsize{38}{38}\selectfont}

\pagestyle{empty}

%%%% Titulni stranka
\begin{titulusOfficii}
\nomenFesti{\feria{}}
\celebratio{Duplex 1. classis.}
\end{titulusOfficii}

\pagebreak

% graphic
\renewcommand{\headrulewidth}{0pt} % no horiz. rule at the header
\fancyhf{}
\pagestyle{fancy}

\cantusSineNeumas

\hora{Ad Matutinum.}

\vspace{2mm}

\cuminitiali{}{temporalia/dominelabiamea.gtex}

\vspace{2mm}

\pars{Invitatorium.} \scriptura{Lc. 24, 34; Psalmus 94; \textbf{H232}}

\vspace{-6mm}

\antiphona{VI}{temporalia/inv-surrexitdominusvere.gtex}

\rubrica{Hymnus non dicatur.}

\vfill
\pagebreak

\pars{Psalmus 1.} \scriptura{\Abardot{} Ex. 3, 14; \textbf{H226}}

\vspace{-5mm}

\antiphona{I f}{temporalia/an-egosumquisum.gtex}

%\vspace{-5mm}

\scriptura{Ps. 1}

%\vspace{-2mm}

\initiumpsalmi{temporalia/ps1-initium-i-F-auto.gtex}

%\psalmusEtTranslatioT{temporalia/ps1-comb.tex}{10cm}
\input{temporalia/ps1.tex} \Abardot{}

\vfill
\pagebreak

\pars{Psalmus 2.} \scriptura{\Abardot{} Ps. 2, 8; \textbf{H226}}

\vspace{-2mm}

\antiphona{I f}{temporalia/an-postulavi.gtex}

\vspace{-2mm}

\scriptura{Ps. 2}

\initiumpsalmi{temporalia/ps2-initium-i-f-auto.gtex}

%\psalmusEtTranslatioT{temporalia/ps2-comb.tex}{10cm}
\input{temporalia/ps2.tex} \Abardot{}

\vfill
\pagebreak

\pars{Psalmus 3.} \scriptura{\Abardot{} Ps. 3, 6; \textbf{H226}}

%\vspace{-4mm}

\antiphona{VIII C}{temporalia/an-egodormivi.gtex}

%\vspace{-0.3cm}

\scriptura{Ps. 3}

\initiumpsalmi{temporalia/ps3-initium-viii-c-auto.gtex}

%\psalmusEtTranslatioT{temporalia/ps3-comb.tex}{10cm}
\input{temporalia/ps3.tex} \Abardot{}

\vfill
\pagebreak

\versusMatutinum

\vspace{5mm}

\sineinitiali{temporalia/oratiodominica-mat.gtex}

\vspace{5mm}

\pars{Absolutio.}

\absolutio

\vfill
\pagebreak

\cuminitiali{}{temporalia/benedictio-solemn-evangelica.gtex}

\vspace{7mm}

\lectioi

\noindent \Vbardot{} Tu autem, Dómine, miserére nobis.
\noindent \Rbardot{} Deo grátias.

\vfill
\pagebreak

\responsoriumi

\vfill
\pagebreak

\cuminitiali{}{temporalia/benedictio-solemn-divinum.gtex}

\vspace{7mm}

\lectioii

\noindent \Vbardot{} Tu autem, Dómine, miserére nobis.
\noindent \Rbardot{} Deo grátias.

\vfill
\pagebreak

\responsoriumii

\vfill
\pagebreak

\cuminitiali{}{temporalia/benedictio-solemn-adsocietatem.gtex}

\vspace{7mm}

\pars{Lectio III.}

\noindent Et quóniam sermo huc noster evásit, considerémus qua grátia secúndum Ioánnem credíderint Apóstoli, qui gavísi sunt; secúndum Lucam quasi incréduli redarguántur: ibi Spíritum Sanctum accéperint, hic sedére in civitáte iubeántur, quoadúsque induántur virtúte ex alto. Et vidétur mihi ille quasi Apóstolus maióra et altióra tetigísse, hic sequéntia et humánis próxima: hic histórico usus circúitu, ille compéndio: quia et de illo dubitári non potest, qui testimónium pérhibet de iis, quibus ipse intérfuit, et verum est testimónium eius: et ab hoc quoque, qui Evangelísta esse méruit, vel negligéntiæ, vel mendácii suspiciónem æquum est propulsári. Et ídeo verum putámus utrúmque, non sententiárum varietáte, nec personárum diversitáte distínctum. Nam etsi primo Lucas eos non credidísse dicat, póstea tamen credidísse demónstrat: et si prima considerémus, contrária sunt: si sequéntia, certum est conveníre.

\noindent \Vbardot{} Tu autem, Dómine, miserére nobis.
\noindent \Rbardot{} Deo grátias.

\vfill
\pagebreak

% Te Deum

%\pars{Hymnus Ambrosianus}

\vspace{-5mm}

{
\grechangedim{interwordspacetext}{0.22 cm plus 0.15 cm minus 0.05 cm}{scalable}%
\cuminitiali{III}{temporalia/tedeum-solemnis.gtex}
\grechangedim{interwordspacetext}{0.32 cm plus 0.15 cm minus 0.05 cm}{scalable}%
}

\vfill
\pagebreak

\rubrica{Reliqua omittuntur, nisi Laudes separandæ sint.}

\pars{Oratio}

\noindent \Vbardot{} Dómine, exáudi oratiónem meam.

\noindent \Rbardot{} Et clamor meus ad te véniat.

Orémus:

\oratioMatutinum

\noindent \Rbardot{} Amen.

\vspace{7mm}

\pars{Conclusio}

\noindent \Vbardot{} Dómine, exáudi oratiónem meam.

\noindent \Rbardot{} Et clamor meus ad te véniat.

\noindent \Vbardot{} Benedicámus Dómino, allelúia, allelúia.

\noindent \Rbardot{} Deo grátias, allelúia, allelúia.

\noindent \Vbardot{} Fidélium ánimæ per misericórdiam Dei requiéscant in pace.

\noindent \Rbardot{} Amen.

\vfill
\pagebreak

\hora{Ad Laudes.} %%%%%%%%%%%%%%%%%%%%%%%%%%%%%%%%%%%%%%%%%%%%%%%%%%%%%
%\sideThumbs{Laudes}

\cantusSineNeumas

\vspace{0.5cm}
\grechangedim{interwordspacetext}{0.18 cm plus 0.15 cm minus 0.05 cm}{scalable}%
\cuminitiali{}{temporalia/deusinadiutorium-alter.gtex}
\grechangedim{interwordspacetext}{0.22 cm plus 0.15 cm minus 0.05 cm}{scalable}%

\vfill
%\pagebreak

\pars{Psalmus 1.} \scriptura{Mt. 28, 2; \textbf{H230}}

\vspace{-0.4cm}

\antiphona{VIII G}{temporalia/ant-angelusautemdomini.gtex}

\scriptura{Psalmus 92.}

\initiumpsalmi{temporalia/ps92-initium-viii-G-auto.gtex}

%\psalmusEtTranslatioT{temporalia/ps92-comb.tex}{10cm}
\input{temporalia/ps92.tex}

\vfill

\antiphona{}{temporalia/ant-angelusautemdomini.gtex}

\vspace{-1cm}

\vfill
\pagebreak

\pars{Psalmus 2.} \scriptura{Mt. 28, 2; \textbf{H226}}

\vspace{-0.4cm}

\antiphona{VII c}{temporalia/ant-etecceterraemotus.gtex}

\scriptura{Psalmus 99.}

\initiumpsalmi{temporalia/ps99-initium-vii-c-auto.gtex}

%\psalmusEtTranslatioT{temporalia/ps99-comb.tex}{10cm}
\input{temporalia/ps99.tex} \Abardot{}

\vfill
\pagebreak

\pars{Psalmus 3.} \scriptura{Mt. 28, 3; \textbf{H226}}

\vspace{-0.4cm}

\antiphona{VIII c}{temporalia/ant-eratautem.gtex}

\scriptura{Psalmus 62.}

\initiumpsalmi{temporalia/ps62-initium-viii-C-auto.gtex}

%\psalmusEtTranslatioT{temporalia/ps62-comb.tex}{10cm}
\input{temporalia/ps62.tex} \Abardot{}

\vfill
\pagebreak

\pars{Psalmus 4.} \scriptura{Mt. 28, 4; \textbf{H226}}

\vspace{-0.4cm}

\antiphona{VII a}{temporalia/ant-praetimoreautem.gtex}

\scriptura{Canticum trium puerorum, Dan. 3, 57-88 et 56}

\initiumpsalmi{temporalia/dan3-initium-vii-a-auto.gtex}

%\psalmusEtTranslatioT{temporalia/dan3-comb.tex}{10cm}
\input{temporalia/dan3.tex}

\rubrica{Hic non dicitur Gloria Patri, neque Amen.}

\vfill

\vspace{-6mm}

\antiphona{}{temporalia/ant-praetimoreautem.gtex} % repeat the antiphon - new page

\vfill
\pagebreak

\pars{Psalmus 5.} \scriptura{Mt. 28, 5; \textbf{H230}}

\vspace{-7mm}

\antiphona{VIII G}{temporalia/ant-respondensautemangelus.gtex}

\scriptura{Psalmus 148.}

%\vspace{-4mm}

\initiumpsalmi{temporalia/ps148-initium-viii-g-auto.gtex}

%\psalmusEtTranslatioT{temporalia/ps148-comb.tex}{10cm}
\input{temporalia/ps148.tex}

\rubrica{Hic non dicitur Gloria Patri.}

\vfill
\pagebreak

%
\scriptura{Psalmus 149.}

\initiumpsalmi{temporalia/ps149-initium-viii-g-auto.gtex}

%\psalmusEtTranslatioT{temporalia/ps149-comb.tex}{10cm}
\input{temporalia/ps149.tex}

\rubrica{Hic non dicitur Gloria Patri.}

\vfill
\pagebreak

%
\scriptura{Psalmus 150.}

\initiumpsalmi{temporalia/ps150-initium-viii-g-auto.gtex}

%\psalmusEtTranslatioT{temporalia/ps150-comb.tex}{10cm}
\input{temporalia/ps150.tex}

\vfill

\vspace{-6mm}

\antiphona{}{temporalia/ant-respondensautemangelus.gtex} % repeat the antiphon - new page

\vfill
\pagebreak

\newcommand{\capitulumLaudes}{\pars{Capitulum.} \scriptura{1 Cor. 5, 7}

\grechangedim{interwordspacetext}{0.12 cm plus 0.15 cm minus 0.05 cm}{scalable}%
\cuminitiali{}{temporalia/capitulum-FratresExpurgate.gtex}
\grechangedim{interwordspacetext}{0.22 cm plus 0.15 cm minus 0.05 cm}{scalable}}
\capitulumLaudes

% preklad Jeruz. bible
%\trCapituliI

\vfill

\pars{Responsorium breve.} \scriptura{Cf. Mt. 28, 6; Cf. Gal. 3, 13}

\cuminitiali{VI}{temporalia/respbr-laud.gtex}

%\trResp

\vfill
\pagebreak

\pars{Hymnus}

\cuminitiali{VIII}{temporalia/hym-AuroraLucis.gtex}
\vspace{-3mm}
%\input{hym-AuroraLucis-bohtext.tex}

\vfill
%\pagebreak

\pars{Versus.} \scriptura{Ps. 117, 24}

% Versus. %%%
\sineinitiali{temporalia/versus-haecdies.gtex}

%\noindent \trVersus

\vfill
\pagebreak

\benedictus

\vspace{-1cm}

\vfill
\pagebreak

%\sideThumbs{{\scriptsize{}Fine horarum}}

\anteOrationem

\pagebreak

% Oratio. %%%
\oratioLaudes

\vspace{-1mm}
%\trOrationisI

\vfill

\rubrica{Hebdomadarius dicit iterum Dominus vobiscum, vel cantor dicit:}

\vspace{2mm}

\sineinitiali{temporalia/domineexaudi.gtex}

\rubrica{Postea cantatur a cantore:}

\vspace{2mm}

\cuminitiali{}{temporalia/benedicamus-octava-paschae.gtex}

\vspace{1mm}

\ifx\magnificat\undefined
\else
\vfill
\pagebreak

\hora{Ad Vesperas.} %%%%%%%%%%%%%%%%%%%%%%%%%%%%%%%%%%%%%%%%%%%%%%%%%%%%%
%\sideThumbs{Vesperæ}

\cantusSineNeumas

\vspace{0.5cm}
\grechangedim{interwordspacetext}{0.18 cm plus 0.15 cm minus 0.05 cm}{scalable}%
\cuminitiali{}{temporalia/deusinadiutorium-solemnis.gtex}
\grechangedim{interwordspacetext}{0.22 cm plus 0.15 cm minus 0.05 cm}{scalable}%

\vfill
%\pagebreak

\vspace{-2mm}

\pars{Psalmus 1.} \scriptura{Mt. 28, 2; \textbf{H230}}

\vspace{-0.4cm}

\antiphona{VIII G}{temporalia/ant-angelusautemdomini.gtex}

\scriptura{Psalmus 109.}

\initiumpsalmi{temporalia/ps109-initium-viii-G-auto.gtex}

%\psalmusEtTranslatioT{temporalia/ps109-comb.tex}{10cm}
\input{temporalia/ps109.tex}

\antiphona{}{temporalia/ant-angelusautemdomini.gtex}

\vspace{-1cm}

\vfill
\pagebreak

\pars{Psalmus 2.} \scriptura{Mt. 28, 2; \textbf{H226}}

\vspace{-0.4cm}

\antiphona{VII c}{temporalia/ant-etecceterraemotus.gtex}

\scriptura{Psalmus 110.}

\initiumpsalmi{temporalia/ps110-initium-vii-c-auto.gtex}

%\psalmusEtTranslatioT{temporalia/ps110-comb.tex}{10cm}
\input{temporalia/ps110.tex} \Abardot{}

\vfill
\pagebreak

\pars{Psalmus 3.} \scriptura{Mt. 28, 3; \textbf{H226}}

\vspace{-0.4cm}

\antiphona{VIII c}{temporalia/ant-eratautem.gtex}

\scriptura{Psalmus 111.}

\initiumpsalmi{temporalia/ps111-initium-viii-C-auto.gtex}

%\psalmusEtTranslatioT{temporalia/ps111-comb.tex}{10cm}
\input{temporalia/ps111.tex} \Abardot{}

\vfill
\pagebreak

\pars{Psalmus 4.} \scriptura{Mt. 28, 5; \textbf{H230}}

\vspace{-0.4cm}

\antiphona{VIII G}{temporalia/ant-respondensautemangelus.gtex}

\scriptura{Psalmus 112.}

\initiumpsalmi{temporalia/ps112-initium-viii-g-auto.gtex}

%\psalmusEtTranslatioT{temporalia/ps112-comb.tex}{10cm}
\input{temporalia/ps112.tex} \Abardot{}

\vfill
\pagebreak

\capitulumLaudes

% preklad Jeruz. bible
%\trCapituliI

\vfill

\pars{Responsorium breve.} \scriptura{Lc. 24, 34}

\cuminitiali{VI}{temporalia/respbr-vesp.gtex}

%\trResp

\vfill
\pagebreak

\pars{Hymnus}

\cuminitiali{VIII}{temporalia/hym-AdCoenam.gtex}
\vspace{-3mm}
%\begin{translatioMulticol}{4}
U~Beránkovy hostiny\\
oděni rouchy bílými,\\
když Rudým mořem prošli jsme,\\
Vladaři Kristu zpívejme.\\
\\
Když jeho tělem posvátným,\\
na kříži obětovaným,\\
se sytíme a~pijeme\\
jeho krev, v~Bohu žijeme.\columnbreak

Chráněni tímto pokrmem\\
před smrtonosným andělem,\\
svrhli jsme z~beder kruté jho\\
tyrana bezohledného.\\
\\
Kristus je naší paschou teď,\\
on sám se vydal za oběť\\
a~místo přesnic našim rtům\\
své tělo dává za pokrm.\columnbreak

Tys, nejčistější Oběti,\\
zlomila vládu podsvětí.\\
Z~otroctví lid je vykoupen,\\
odměna žití kyne všem.\\
\\
Hle, Kristus, když vstal ze hrobu,\\
jde z~pekel v~slavném průvodu\\
a~brány nebes otevřev,\\
vládce tmy vleče v~okovech.\columnbreak

Buď věčně, Kriste, věrným svým\\
plesáním velikonočním.\\
Nás, milostí tvou vzkříšené,\\
vem k~oslavě své vítězné. \\
\\
Sláva tobě, Pane,\\
jenž jsi vstal z~mrtvých,\\
s~Otcem i~Svatým Duchem\\
na věčné věky.\\
Amen.
\end{translatioMulticol}


\vfill
\pagebreak

\pars{Versus.} \scriptura{Ps. 117, 24}

% Versus. %%%
\sineinitiali{temporalia/versus-haecdies.gtex}

%\noindent \trVersus

\vfill
\pagebreak

\magnificat

\vspace{-1cm}

\vfill
\pagebreak

%\sideThumbs{{\scriptsize{}Fine horarum}}

\anteOrationem

\pagebreak

% Oratio. %%%
\oratioLaudes

\vspace{-1mm}
%\trOrationisI

\vfill

\rubrica{Hebdomadarius dicit iterum Dominus vobiscum, vel cantor dicit:}

\vspace{2mm}

\sineinitiali{temporalia/domineexaudi.gtex}

\rubrica{Postea cantatur a cantore:}

\vspace{2mm}

\cuminitiali{}{temporalia/benedicamus-octava-paschae.gtex}

\vspace{1mm}
\fi

\end{document}

