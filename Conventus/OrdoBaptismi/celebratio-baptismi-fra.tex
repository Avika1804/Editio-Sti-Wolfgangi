\rubricatum{Lorsqu'on est en place près des fonts baptismaux,
le prêtre dit par exemple~:}

Mes frères, priez avec moi le Seigneur tout-puissant~:
qu'il donne en abondance à cet enfant cette vie nouvelle qui
vient de l'eau et de l'Esprit.

\rubricatum{Ensuite le prêtre se tourne vers l'eau et dit~:}

Dieu, dont la puissance invisible accomplit des merveilles par
les sacrements, tu as voulu, au cours des temps, que l'eau, ta
créature, révèle ce que serait la grâce du baptême. Que cette
eau reçoive de l'Esprit Saint la grâce de ton Fils unique, afin
que l'homme, créé à ta ressemblance puisse naître de l'eau et de
l'Esprit pour la vie d'enfant de Dieu. (…) Nous t'en prions,
Seigneur notre Dieu : Par la grâce de ton Fils, que vienne sur
cette eau la puissance de l'Esprit Saint, afin que tout homme,
toute femme, qui sera baptisé, ressuscite avec le Christ pour
la vie, car il est vivant pour les siècles des siècles.

\rubricatum{Célébrant touche l’eau avec main droit et continue~:}

Daigne maintenant bénir \grecross{} cette eau où il va naître de
l'Esprit Saint pour vivre de la vie éternelle. Par Jésus,
le Christ, notre Seigneur.

\rubricatum{Tous~:} Amen.

\pars{}

\rubricatum{Le prêtre s'adresse aux parents et aux parrains en ces termes~:}

L'enfant que vous présentez, parents, parrain et marraine, va recevoir
le sacrement du baptême~: dans son amour, Dieu lui donnera une vie nouvelle.
Il (elle) va renaître de l'eau et de l'Esprit Saint. Ayez le souci de le
(la) faire grandir dans la foi pour que cette vie divine ne soit pas
affaiblie par l'indifférence et le péché, mais se développe en lui (elle)
de jour en jour.

Et donc, si vous êtes disposés à prendre cette responsabilité, et si vous
êtes conduits par la foi, en vous rappelant votre baptême, rejetez tout
attachement au péché et proclamez la foi en Jésus Christ, proclamez la
foi de l'Église dans laquelle tout enfant est baptisé.

\rubricatum{Après quoi le prêtre interroge les parents et les parrains~:} Pour vivre dans
la liberté des enfants de Dieu, rejetez-vous le péché~?

\rubricatum{Parents et parrains~:} Oui, je le rejette.

\rubricatum{Célébrant~:} Pour échapper au pouvoir du péché, rejetez-vous ce qui
conduit au mal~?

\rubricatum{Parents et parrains~:} Oui, je le rejette.

\rubricatum{Célébrant~:} Pour suivre Jésus Christ, rejetez-vous Satan qui est
l'auteur du péché~?

\rubricatum{Parents et parrains~:} Oui, je le rejette.

\rubricatum{Le prêtre demande aux parents et aux parrains une triple
profession de foi~:}

Croyez-vous en Dieu le Père tout-puissant, créateur du ciel et de la terre~?

\rubricatum{Parents et parrains~:} Je crois.

\rubricatum{Célébrant~:} Croyez-vous en Jésus Christ, son Fils unique, notre
Seigneur, qui est né de la Vierge Marie, a souffert la passion, a été
enseveli, est ressuscité d'entre les morts, et qui est assis à la droite
du Père~?

\rubricatum{Parents et parrains~:} Je crois.

\rubricatum{Célébrant~:} Croyez-vous en l'Esprit Saint, à la Sainte Église
catholique, à la communion des saints, au pardon des péchés, à la
résurrection de la chair, et à la vie éternelle~?

\rubricatum{Parents et parrains~:} Je crois.

\rubricatum{Célébrant~:} Telle est notre foi. Telle est la foi de l'Église
que nous sommes fiers de proclamer dans le Christ Jésus notre Seigneur.

\rubricatum{Tous~:} Amen.

\pars{}

\rubricatum{Le prêtre invite les parents et parrains à s'approcher avec
l'enfant de l'eau baptismale. II leur pose la question suivante~:}

Voulez-vous que {\color{red}N.} soit baptisé(e) dans cette foi de
l'Église que tous ensemble nous venons d'exprimer avec vous~?

\rubricatum{Parents et parrains~:} Oui, nous le voulons.

\rubricatum{Et aussitôt le prêtre baptise l'enfant en disant de façon
à ce que tous puissent l'entendre~:}

{\color{red}N.}, je te baptise, au nom du Père, 

\rubricatum{il plonge l'enfant dans l'eau ou verse l'eau une première foi}

et du Fils

\rubricatum{il plonge l'enfant dans l'eau ou verse l'eau une deuxième fois}

et du Saint-Esprit.

\rubricatum{il plonge l'enfant dans l'eau ou verse l'eau une troisième fois.}
