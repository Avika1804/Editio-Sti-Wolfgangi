\rubricatum{\Rbardot{}} Sláva tobě, Pane.

Když nadešel den jejich očišťování podle Mojžíšova Zákona, přinesli Ježíše
do Jeruzaléma, aby ho představili Pánu, jak je psáno v Zákoně Páně: 'Všechno
prvorozené mužského rodu ať je zasvěceno Pánu!' Přitom chtěli také podat
oběť, jak je to nařízeno v Zákoně Páně: pár hrdliček nebo dvě holoubata. 
Tehdy žil v Jeruzalémě jeden člověk, jmenoval se Simeon: byl to člověk
spravedlivý a bohabojný, očekával potěšení Izraele a byl v něm Duch svatý. 
Od Ducha svatého mu bylo zjeveno, že neuzří smrt, dokud neuvidí Pánova
Mesiáše.  Z vnuknutí Ducha přišel do chrámu, právě když rodiče přinesli dítě
Ježíše, aby s ním vykonali, co bylo obvyklé podle Zákona.  Vzal si ho do
náručí a takto velebil Boha: »Nyní můžeš, Pane, propustit svého služebníka
podle svého slova v pokoji, neboť moje oči uviděly tvou spásu, kterou jsi
připravil pro všechny národy: světlo k osvícení pohanům a k slávě tvého
izraelského lidu.« Jeho otec i matka byli plni údivu nad slovy, která o něm
slyšeli.  Simeon jim požehnal a jeho matce Marii prohlásil: »On je ustanoven
k pádu a k povstání mnohých v Izraeli a jako znamení, kterému se bude
odporovat - i tvou vlastní duší pronikne meč - aby vyšlo najevo smýšlení
mnoha srdcí.« Také tam byla prorokyně Anna, dcera Fanuelova z Aserova kmene. 
Byla značně pokročilého věku: mladá se vdala a sedm roků žila v manželství,
potom sama jako vdova - bylo jí už čtyřiaosmdesát let.  Nevycházela z chrámu
a sloužila Bohu posty a modlitbami ve dne v noci.  Přišla tam právě v tu
chvíli, velebila Boha a mluvila o tom dítěti všem, kdo očekávali vykoupení
Jeruzaléma.  Když vykonali všechno podle Zákona Páně, vrátili se do Galileje
do svého města Nazareta.  Dítě rostlo a sílilo, bylo plné moudrosti a milost
Boží byla s ním.

Slyšeli jsme slovo Boží.

\rubricatum{\Rbardot{}} Chvála tobě, Kriste.
