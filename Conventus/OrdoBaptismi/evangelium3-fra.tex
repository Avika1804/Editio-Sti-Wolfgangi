\rubricatum{\Rbardot{}} Gloire à toi, Seigneur.

Jésus est présenté au temple et offert à Dieu comme premier-né; un
vieillard, homme juste inspiré par l'Esprit, va à sa rencontre — c'est une
sorte de synthèse et d'image de toute l'espérance messianique d'Israël.
L'attente de Syméon est terminée, maintenant il peut mourir. En lui et dans
son attente de Rédemption, c'est tout l'Ancien Testament, l'ancienne loi,
qui trouve sa réalisation en même temps que le salut s'ouvre et la lumière
s'allume pour tous les peuples.

Cependant, le jugement et la crise ne
manquent pas d'arriver; l'enfant sera la référence discriminante, le point
de comparaison: un signe de contradiction. Il sera accueilli ou rejeté.
L'épreuve se reflétera aussi sur Marie. Dans la présentation au temple on
peut percevoir déjà la croix, le Crucifié et la Vierge des douleurs. La
prophétesse Anne aussi perçoit la rédemption dans cet enfant, en rend grâce
à Dieu et l'annonce.

Acclamons la Parole de Dieu.

\rubricatum{\Rbardot{}} Louange à toi, Seigneur Jésus.
