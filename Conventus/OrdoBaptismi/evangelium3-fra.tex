\rubricatum{\Rbardot{}} Gloire à toi, Seigneur.

Quand fut arrivé le huitième jour, celui de la circoncision, l'enfant reçut
le nom de Jésus, le nom que l'ange lui avait donné avant sa conception.
Quand arriva le jour fixé par la loi de Moïse pour la purification, les
parents de Jésus le portèrent à Jérusalem pour le présenter au Seigneur,
selon ce qui est écrit dans la Loi : Tout premier-né de sexe masculin sera
consacré au Seigneur. Ils venaient aussi présenter en offrande le sacrifice
prescrit par la loi du Seigneur : un couple de tourterelles ou deux petites
colombes. Or, il y avait à Jérusalem un homme appelé Syméon. C'était un
homme juste et religieux, qui attendait la consolation d'Israël, et
l'Esprit Saint était sur lui. L'Esprit lui avait révélé qu'il ne verrait
pas la mort avant d'avoir vu le Messie du Seigneur. Poussé par l'Esprit,
Syméon vint au Temple. Les parents y entraient avec l'enfant Jésus pour
accomplir les rites de la Loi qui le concernaient. Syméon prit l'enfant
dans ses bras, et il bénit Dieu en disant : « Maintenant, ô Maître, tu peux
laisser ton serviteur s'en aller dans la paix, selon ta parole. Car mes
yeux ont vu ton salut, que tu as préparé à la face de tous les peuples :
lumière pour éclairer les nations païennes, et gloire d'Israël ton peuple.
» Le père et la mère de l'enfant s'étonnaient de ce qu'on disait de lui.
Syméon les bénit, puis il dit à Marie sa mère : « Vois, ton fils qui est là
provoquera la chute et le relèvement de beaucoup en Israël. Il sera un
signe de division. — Et toi-même, ton cœur sera transpercé par une épée. —
Ainsi seront dévoilées les pensées secrètes d'un grand nombre. » Il y avait
là une femme qui était prophète, Anne, fille de Phanuel, de la tribu
d'Aser. Demeurée veuve après sept ans de mariage, elle avait atteint l'âge
de quatre-vingt-quatre ans. Elle ne s'éloignait pas du Temple, servant Dieu
jour et nuit dans le jeûne et la prière. S'approchant d'eux à ce moment,
elle proclamait les louanges de Dieu et parlait de l'enfant à tous ceux qui
attendaient la délivrance de Jérusalem. Lorsqu'ils eurent accompli tout ce
que prescrivait la loi du Seigneur, ils retournèrent en Galilée, dans leur
ville de Nazareth. L'enfant grandissait et se fortifiait, tout rempli de
sagesse, et la grâce de Dieu était sur lui.

Acclamons la Parole de Dieu.

\rubricatum{\Rbardot{}} Louange à toi, Seigneur Jésus.
