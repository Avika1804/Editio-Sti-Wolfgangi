%%%% Preklady jednotlivych zpevu (nektere se opakuji, a je dobre mit je
% vsechny na jedne hromade)

\newcommand{\trMatLecI}{\translatioCantus{
\hebinitial{ח} Hospodin se rozhodl zničit hradby sijónské dcery. Natáhl měřicí šňůru, už neodvrátí svou ruku od ničení. Tak zarmoutil val i hradbu, rázem je po nich veta.\\
\hebinitial{ט} Do země se ponořily její brány, její závory zničil a roztříštil, její král a velmožové jsou mezi pronárody a zákona není; ani její proroci nemívají vidění od Hospodina.\\
\hebinitial{י} Usedli na zem a ztichli starcové sijónské dcery, házeli si na hlavu prach, oděli se do žíněných suknic. Jeruzalémské panny svěsily hlavu k zemi.\\
\hebinitial{כ} Zrak mi pro slzy slábne, v mém nitru to bouří, játra mi vyhřezla na zem pro těžkou ránu dcery mého lidu, neboť skomírá pachole a kojenec na prostranstvích města.
Jeruzaléme, Jeruzaléme, obrať se k Pánu, Bohu svému.}}

\newcommand{\trMatLecII}{\translatioCantus{
\hebinitial{ל} Svých matek se ptají: ,,Kde je obilí a víno?\mbox{}``, když jako smrtelně raněný skomírají na prostranstvích města a na klíně svých matek vypouštějí duši.\\
\hebinitial{מ} Jaké svědectví o tobě vydám, čemu tě připodobním, jeruzalémská dcero? K čemu tě přirovnám, čím tě potěším, panno, dcero sijónská? Tvá těžká rána je veliká jak moře, kdo tě uzdraví?\\
\hebinitial{נ} Tvoji proroci ti zvěstovali šalebná a bláhová vidění, neodhalovali tvoje nepravosti, aby změnili tvůj úděl. Vidění, která ti zvěstovali, byla šalba a svody.\\
\hebinitial{ס} Tleskají nad tebou rukama všichni, kdo jdou kolem, syknou a potřesou hlavou nad jeruzalémskou dcerou. ,,Toto je město, o němž se říkalo: Místo dokonalé krásy k potěše celé země?\mbox{}``
Jeruzaléme, Jeruzaléme, obrať se k Pánu, Bohu svému.}}

\newcommand{\trMatLecIII}{\translatioCantus{
\hebinitial{א} Já jsem muž, jenž zakusil ponížení pod holí jeho prchlivosti. Hnal mě a odvedl do temnoty beze světla. Ano, obrací znovu a znovu svou ruku proti mně každého dne.\\
\hebinitial{ב} Vetchým učinil mé tělo i kůži, roztříštil mé kosti. Obstavěl a obklíčil mě jedem a útrapami. Do temnot mě vsadil jako od věků mrtvé.\\
\hebinitial{ג} Postavil kolem mě zeď, že nemohu vyjít, obtížil mě bronzovým řetězem. Jakkoli úpím a o pomoc volám, umlčuje mou modlitbu. Zazdil mé cesty kvádry, mé stezky rozvrátil.
Jeruzaléme, Jeruzaléme, obrať se k Pánu, Bohu svému.}}

\newcommand{\trMatLecIV}{\translatioCantus{
,,Dej mi úkryt před ničemnou smečkou, dorážením těch, kdo pášou křivdy``.
Pohleďme na Krista, naši hlavu.
Mnozí mučedníci vytrpěli velké soužení, nikdo však nevyzařuje tak, jako hlava mučedníků.
Na něm lépe vidíme, co mučedníci zakusili.
Byl ochráněn před ničemnou smečkou, jsa chráněn Bohem.
Jeho tělo bylo chráněno samotným Synem a také člověkem, jímž se oděl, neboť je synem člověka i Synem Božím.
Syn Boží kvůli přirozenosti Boha, syn člověka kvůli přirozenosti služebníka, měl v moci dát svůj život a přijmout jej.
Co mu mohli nepřátelé učinit?
Zabili tělo, duši nezabili.
Nestačilo by tedy, aby Pán povzbuzoval mučedníky slovem, kdyby je neupevnil také příkladem.}}

\newcommand{\trMatLecV}{\translatioCantus{
Víte, jaká to byla ničemná smečka Židů, jak doráželi ti, kdo pášou křivdy. 
O jakou křivdu se jedná?
Chtěli zabít Pána Ježíše Krista.
,,Ukázal jsme vám mnoho dobrých skutků,`` praví, ,,Pro který z nich mne chcete zabít?\mbox{}``
Snesl všechny jejich nemocné, uzdravil všechny jejich choré, kázal jim nebeské království,
nemlčel o jejich neřestech, aby se jim znechutily spíše ony, a ne lékař, jímž byli uzdravováni.
Nevděční vůči všemu tomuto jeho léčení, jako pomatení velkou horečkou,
běsnící proti lékaři, který je přišel léčit, vymysleli plán jeho záhuby.
Jako by chtěli vyzkoušet, zda je skutečně člověk, který může zemřít,
anebo zda je nad lidmi a nedovolí svoji smrt.
Jejich slovo rozpoznáváme v moudrosti Šalomounově:
,,Odsuďme ho k potupné smrti, vždyť se mu, jak říká, dostane navštívení.
Je-li to skutečně Syn Boží, ať jej Bůh osvobodí.``}}

\newcommand{\trMatLecVI}{\translatioCantus{
,,Jazyky si brousí jako meče.``
Ať Židé neříkají: nezabili jsme Krista.
Proto jej totiž dali soudit Pilátovi, aby se zdáli být zproštěni jeho smrti.
Neboť když jim Pilát řekl: ,,Zabijte ho vy sami``, odpověděli:
,,My nemáme právo nikoho popravit.``
Nespravedlnost svého zločinu chtěli přenést na soudce -- člověka,
ale cožpak oklamali soudce -- Boha?
To, co udělal Pilát, bylo významnou účastí na zločinu.
Ale ve srovnání s nimi má mnohem menší vinu.
Snažil se totiž, jak mohl, aby jej z jejich rukou vysvobodil, neboť proto jej před ně předvedl zbičovaného.
Nezbičoval Pána z nepřátelství, nýbrž aby uspokojil jejich zuřivost,
aby se obměkčili a přestali toužit po jeho smrti, když by jej viděli zbičovaného.
Ale když neustali, víte, že si umyl ruce a řekl, že nemá vinu na jeho smrti.
Udělal to však.
Je-li však vinen, neboť to udělal, byť proti své vůli, jsou snad nevinní ti, kdo jej nutili, aby to udělal?
V žádném případě.
Pilát nad Ježíšem pronesl rozsudek a odsoudil jej k ukřižování, a jakoby on sám jej zabil.
I vy, ó Židé, jste jej zabili.
Jak?
Mečem jazyka.
Nabrousili jste si totiž vaše jazyky.
A kdy jste udeřili, ne-li tehdy, když jste zvolali: ,,Ukřižuj ho! Ukřižuj!\mbox{}``?}}

\newcommand{\trMatLecVII}{\translatioCantus{
A tak usilujme, abychom vešli do toho odpočinutí a nikdo pro neposlušnost nepadl jako ti na poušti.
Slovo Boží je živé, mocné a ostřejší než jakýkoli dvousečný meč; proniká až na rozhraní duše a ducha, kostí a morku, a rozsuzuje touhy i myšlenky srdce.
Není tvora, který by se před ním mohl skrýt. Nahé a odhalené je všechno před očima toho, jemuž se budeme ze všeho odpovídat.
Protože máme mocného velekněze, který vstoupil až před Boží tvář, Ježíše, Syna Božího, držme se toho, co vyznáváme.
Nemáme přece velekněze, který není schopen mít soucit s našimi slabostmi; vždyť na sobě zakusil všechna pokušení jako my, ale nedopustil se hříchu.}}

\newcommand{\trMatLecVIII}{\translatioCantus{
Přistupme tedy směle k trůnu milosti, abychom došli milosrdenství a nalezli milost a pomoc v pravý čas.
Každý velekněz, vybraný z lidí, bývá ustanoven jako zástupce lidí před Bohem, aby přinášel dary i oběti za hříchy.
Má mít soucit s těmi, kdo chybují a bloudí, protože sám také podléhá slabosti.
A proto je povinen přinášet oběti za hřích nejenom za lid, ale i sám za sebe.}}

\newcommand{\trMatLecIX}{\translatioCantus{
Hodnost velekněze si nikdo nemůže přisvojit sám, nýbrž povolává ho Bůh jako kdysi Árona.
Tak ani Kristus si nepřisvojil slávu velekněze sám, ale dal mu ji ten, který řekl: ‚,Ty jsi můj Syn, já jsem tě dnes zplodil.``
A na jiném místě říká: ,,Ty jsi kněz navěky podle řádu Melchisedechova.``
Ježíš za svého pozemského života přinesl s bolestným voláním a slzami oběť modliteb a úpěnlivých proseb Bohu, který ho mohl zachránit před smrtí; a Bůh ho pro jeho pokoru slyšel.
Ačkoli to byl Boží Syn, naučil se poslušnosti z utrpení, jímž prošel,
tak dosáhl dokonalosti a všem, kteří ho poslouchají, stal se původcem věčné spásy,
když ho Bůh prohlásil veleknězem podle řádu Melchisedechova.}}
