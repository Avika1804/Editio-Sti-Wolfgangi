%%%% Preklady jednotlivych zpevu (nektere se opakuji, a je dobre mit je
% vsechny na jedne hromade)

\newcommand{\trOratioAnteOfficium}{\translatioCantus{Otevři, Pane, má ústa, abych chválil tvé svaté jméno.
Očisti mé srdce od všech marnivých, zvrácených a \ji{}ných myšlenek, osvěť rozum, rozněť cit,
abych mohl důstojně, soustředěně a zbožně recitovat a vysloužil si být
vyslyšen před tváří tvé velebnosti. Skrze Krista…}}

\newcommand{\trOratioPostOfficium}{\translatioCantus{\textit{Následující modlitbu
opatřil pro ty, kdo \ji{} zbožně vyřknou po hodinkách, zesnulý papež Lev X.
odpustky za hříchy vzniklé při konání hodinek z lidské křehkosti. Říká se
vkleče.}
Svatosvaté a nerozdílné Tro\ji{}ci, ukřižovanému lidství našeho Pána Ježíše
Krista, přeblažené a přeslavné plodné neporušenosti vždy Panny Marie i
souhrnu všech svatých buď ode všeho stvoření věčná chvála, čest a sláva, nám
pak buď dáno odpuštění všech hříchů, po nekonečné věky věků. Amen.}}

% HOURS ---

\newcommand{\trAntI}{\translatioCantus{Náš Pán a Spasitel, před
\ji{}třenkou a před věky zplozený, se dnes zjevil světu.}}

\newcommand{\trAntII}{\translatioCantus{Vzešlo tvé světlo, \grestar{}
Jeruzaléme, a Hospodinova velebnost září nad tebou a národy budou kráčet v
tvém světle. Aleluja.}}

\newcommand{\trAntIII}{\translatioCantus{Mudrci otevřeli své
pokladnice a obětovali Pánu zlato, kadidlo a myrhu. Aleluja.}}

\newcommand{\trAntIV}{\translatioCantus{Moře a řeky, \grestar{} velebte Pána,
velebte ho, všechny prameny. Aleluja.}}

\newcommand{\trAntV}{\translatioCantus{Tato hvězda \grestar{} se jako plamen třepotá a
zjevuje Boha, krále králů. Mágové \ji{} viděli a velkému králi dary odváděli.}}

\newcommand{\trCapituli}{\translatioCantus{Vzhůru! Zaskvěj se, Jeruzaléme! Neboť zde je tvoje světlo, a nad tebou
vzchází Hospodinova sláva.}}

\newcommand{\trCapituliLeva}{\translatioCantus{Pozdvihni oči a rozhlédni se: všichni se shromáždili, přicházejí k tobě.
Tví synové přicházejí z daleka a tvé dcery jsou neseny na zádech.}}

\newcommand{\trCapituliOmnes}{\translatioCantus{Př\ij{}dou všichni ze Sáby, přinesou zlato a kadidlo a budou hlásat Hospodinovu
chválu.}}

\newcommand{\trRespVesp}{\translatioCantus{Př\ij{}dou všichni ze Sáby, \grestar{} aleluja, aleluja. \Vbardot{} Přinesou zlato a kadidlo.}}

\newcommand{\trRespLaudes}{\translatioCantus{Králové Tarsu a z ostrova mu odvedou tribut. \grestar{} aleluja, aleluja.
\Vbardot{} Králové Arabští a Sebejští mu přinesou dárky.}}

\newcommand{\trVersus}{\translatioCantus{\Vbardot{} Králové Tarsu a z ostrova mu odvedou tribut.
\Rbardot{} Králové Arabští a Sebejští mu přinesou dárky.}}

\newcommand{\trVersusLaudes}{\translatioCantus{\Vbardot{} Klanějte se Bohu, aleluja. \Rbardot{} Všichni jeho andělé, aleluja.}}

\newcommand{\trVersusSexta}{\translatioCantus{\Vbardot{} Př\ij{}dou všichni ze Sáby, aleluja. \Rbardot{} Přinesou zlato a kadidlo, aleluja.}}

\newcommand{\trAntMagnificatI}{\translatioCantus{Když mágové \grestar{} spatřili
hvězdu, řekli si: To je znamení velkého krále; pojďme a vyptejme se a
přinesme mu dary — zlato, kadidlo a myrhu.}}

\newcommand{\trAntBenedictus}{\translatioCantus{Dnes \grestar{} se církev zasnoubila
nebeskému Ženichovi, neboť v Jordánu smyl Kristus její hřích. S dary mágové
spěchají ke svatbě královské, z vody víno je najednou a hosté se u stolu radují.
Aleluja.}}

\newcommand{\trAntMagnificatII}{\translatioCantus{Třemi zázraky \grestar{} ozdobený
slavíme svatý den: dnes hvězda dovedla mágy k jesličkám; dnes se stalo
na svatbě z vody víno; dnes chtěl být Kristus pokřtěn od Jana v~Jordáně, aby
nás vykoupil, aleluja.}}

\newcommand{\trOrationis}{\translatioCantus{Bože, jenžs dnes pohanům vedeným
hvězdou zjevil svého Jednorozeného, uděl ve své milosti, aby kdo tě \ji{}ž
vírou znají, došli též zření tvé tváře. Skrze…}}

\newcommand{\trFideliumAnimae}{\translatioCantus{\Vbardot{} Duše věrných ať pro
milosrdenství Boží odpočívají v poko\ji{}. \Rbardot{} Amen.}}

% Completorium

\newcommand{\trJubeDomne}{\translatioCantus{Rač, pane, požehnat.}}

\newcommand{\trComplBenedictio}{\translatioCantus{Pokojnou noc a svatou smrt
nechť nám dopřeje všemohoucí Pán. \Rbardot{} Amen.}}

\newcommand{\trComplLectioBr}{\translatioCantus{Buďte střízliví, bděte.
Váš protivník Ďábel obchází jako lev řvoucí a hledá, koho by pohltil.
Postavte se proti němu pevní ve víře.  Ale ty, Pane, smiluj se nad námi.
\Rbardot{} Bohu díky.}}

\newcommand{\trComplAntI}{\translatioCantus{Rač se smilovati nade mnou,
Hospodine, a vyslyš mou modlitbu.}}

\newcommand{\trComplCapituli}{\translatioCantus{Jsi přece, Hospodine,
uprostřed nás a jmenujeme se po tobě.  Neopouštěj nás, Pane, náš Bože.}}

\newcommand{\trRespCompl}{\translatioCantus{Do tvých rukou, Pane, \grestar{}
poroučím svého ducha. \Vbardot{} Ty mne zachráníš, Pane, Bože věrný.}}

\newcommand{\trComplVersus}{\translatioCantus{\Vbardot{} Střez mne jako zřítelnici oka,
aleluja. \Rbardot{} Ve stínu svých křídel uschovej mne, aleluja.}}

\newcommand{\trAntSalvaNos}{\translatioCantus{Ochraňuj nás, Pane, když
bdíme, a buď s námi, když spíme, ať bdíme s Kristem a odpočíváme v poko\ji{}.}}

\newcommand{\trComplOrationis}{\translatioCantus{Zavítej, prosíme, Pane, sem
do našeho příbytku a daleko od něj zažeň všechny úklady nepřítele. Ať tu
bydlí tví svatí andělé a tvoje požehnání buď nad ním stále. Skrze…}}

\newcommand{\trAlmaRedemptoris}{\begin{translatioMulticol}{2}
Blažená \grestar{} Matko Vykupitele,\\
jež jsi přístupnou bránou do nebes\\
a hvězdou moří; pomoz povstat tomu,\\
kdo klesá, ochránkyně lidu;\columnbreak

jež jsi zrodila k údivu přírody,\\
svého svatého zploditele;\\
Panno dříve a Panno potom.\\
Kdyžs vyslechla ono Ave z úst Gabriele,\\
hříšníkům nakloň srdce své.\end{translatioMulticol}}

% Matutinum

\newcommand{\trMatAntI}{\translatioCantus{Přineste Hospodinu, \grestar{}
synové Boží, padejte před Pánem ve svaté síni jeho.}}

\newcommand{\trMatAntII}{\translatioCantus{Hojnost vod \grestar{}
obveseluje, aleluja, město Boží, aleluja.}}

\newcommand{\trMatAntIII}{\translatioCantus{Chvalte nám Boha, \grestar{} chvalte;
chvalte nám krále, chvalte!}}

\newcommand{\trMatVersusI}{\translatioCantus{\Vbardot{} Všecky země klanějí
se tobě, zpívají tobě. \Rbardot{} Zpívají tvému jménu.}}

\newcommand{\trMatAbsolutioI}{\translatioCantus{Vyslyš Pane Ježíši Kriste
prosby svých služebníků \gredagger{} a smiluj se nad námi, \grestar{} jenž s Otcem a Duchem…}}

\newcommand{\trMatBenedictioI}{\translatioCantus{Rač, pane, požehnat.
Věčný Otec nám stále žehnej. \Rbardot{} Amen.}}

\newcommand{\trMatLecI}{\translatioCantus{
    Ach! Vy všichni, kdo žízníte, př\ij{}ďte k vodě, i když nemáte peníze,
    př\ij{}ďte, kupujte a jezte; př\ij{}ďte, kupujte bez peněz, bez placení
    víno a mléko.  Proč utrácet peníze za něco \ji{}ného než za chléb a to,
    co jste vydělali, za to, co nesytí?  Slyšte, slyšte mě a jezte, co je
    dobré; budete mít potěšení ze šťavnatých jídel.  Nakloňte ucho a př\ij{}ďte
    ke mně, slyšte a budete žít.  Uzavřu s~vámi smlouvu věčnou, naplňuje
    dobrodiní přislíbená Davidovi.  Hle, udělal jsem z něho svědka pro
    národy, vůdce a zákonodárce národů.}}

\newcommand{\trMatRespI}{\translatioCantus{Když byl dnes Pán pokřtěn v~Jordáně,
otevřela se nebesa, jako holubice spočinul na něm Duch a zazněl
Otcův hlas: To je můj milý Syn, v něm jsem si zalíbil.
\Vbardot{}~Duch Svatý v tělesné podobě jako holubice sestoupil na něj a zazněl hlas
z nebe.}}

\newcommand{\trMatBenedictioII}{\translatioCantus{Rač, pane, požehnat.
Jednorozený Boží Syn nám žehnej \grestar{} a nám pomáhej. \Rbardot{} Amen.}}

\newcommand{\trMatLecII}{\translatioCantus{
    Vzhůru! Zaskvěj se, Jeruzaléme! Neboť zde je tvoje světlo, a nad tebou vzchází Hospodinova
    sláva.  Zatímco nad zemí se rozprostírají temnoty a~nad národy tma,
    nad tebou vzchází Hospodin a jeho sláva se zjevuje nad tebou.  Ve tvém
    světle budou kráčet národy a králové ve tvé vycházející záři.  Pozdvihni
    oči a rozhlédni se: všichni se shromáždili, přicházejí k~tobě.  Tví
    synové přicházejí z daleka a tvé dcery jsou neseny na zádech.  A tak
    uvidíš a rozzáříš se, tvé srdce se zachvěje a povznese, neboť k tobě
    bude proudit bohatství moře a př\ij{}dou k tobě poklady národů.  Pokryje tě
    množství velbloudů, mladá zvířata z~Madianu a~z~Efy, př\ij{}dou všichni ze
    Sáby, přinesou zlato a kadidlo a budou hlásat Hospodinovu chválu.}}

\newcommand{\trMatRespII}{\translatioCantus{Ducha Svatého v podobě holubice
uzřeli a Otcův hlas slyšeli: \gredagger{} To je můj milý Syn, v tom jsem si dobře
zalíbil, poslouchejte ho. \Vbardot{} Rozevřelo se nad ním nebe a~zazněl Otcův hlas.
\gredagger{} To…}}

\newcommand{\trMatBenedictioIII}{\translatioCantus{Rač, pane, požehnat.
Milost Ducha Svatého ať osvítí nám smysly \grestar{} i srdce. \Rbardot{} Amen.}}

\newcommand{\trMatLecIII}{\translatioCantus{
    Překypu\ji{} jásotem v Hospodinovi, má duše plesá v mém Bohu, neboť mě oděl
    šatem spásy, zahalil mě do pláště spravedlnosti jako ženicha, jenž si
    nasazuje čelenku, jež se zdobí svými šperky.  Vždyť jako země dává
    vzejít svým výhonkům a zahrada dává vzejít svým semenům, tak Pán Hospodin
    dává vzejít spravedlnosti a chvále přede všemi národy.  Kvůli Sionu
    nebudu mlčet, kvůli Jeruzalému si nedopře\ji{} odpočinku, dokud jeho
    spravedlnost nevyšlehne jako jasné světlo a jeho spása jako zapálená
    pochodeň.}}

\newcommand{\trMatRespIII}{\translatioCantus{
Králové Taršíše \grestar{} a ostrovů odvedou daně. \gredagger{}
Králové ze Sáby a ze Seby podají Pánu Bohu obětní dar. \Vbardot{}
Př\ij{}dou všichni ze Sáby, přinesou zlato a kadidlo.}}

\newcommand{\trMatAntIV}{\translatioCantus{Všecky země klanějí se tobě, \grestar{}
zpívají tobě, zpívají tvému jménu, Hospodine.}}

\newcommand{\trMatAntV}{\translatioCantus{Králové Tarsisu \grestar{}
a přímoří, budou podávat dary králi Hospodinu.}}

\newcommand{\trMatAntVI}{\translatioCantus{
Všichni národové, které jsi učinil, př\ij{}dou a sklánět se budou před tebou, Pane.}}

\newcommand{\trMatAbsolutioII}{\translatioCantus{
Tvá milost a laskavost nechť nám pomáhá, jenž ž\ij{}eš a vládneš s Otcem a Svatým Duchem na věky věků.}}

\newcommand{\trMatBenedictioIV}{\translatioCantus{Rač, pane, požehnat.
Bůh Otec všemohoucí, \grestar{} buď k nám milostivý a odpouštějící. \Rbardot{} Amen.}}

% Nea Marie Kuchařová, http://www.stjoseph.cz/slavime-zjeveni-pane/
\newcommand{\trMatLecIV}{\translatioCantus{
    Milovaní bratři, radujte se v Pánu; znovu říkám, radujte se, neboť jen
    krátký čas uběhl od slavnosti Kristova Narození a už zazářil svátek Jeho
    Zjevení. Tehdy Ho přivedla Panna na svět, dnes Ho svět poznal. Slovo učiněné
    Tělem tak uspořádalo počátek našeho spasení, že narozený Ježíš se současně
    ukázal věřícím a zůstal skryt pronásledovatelům. Věru \ji{}ž nebesa hlásala
    slávu Boží, a je\ji{}ch hlas se rozšířil do všech zemí, když se pastýřům
    zjevilo andělské vojsko a zvěstovalo Spasitele a hvězda vedla mágy, aby se
    mu mohli poklonit; od východu slunce do západu zazářilo zrození Krále, když
    se zvěst dověděla království východu od mágů a když nezůstala skryta Římské
    říši.}}

\newcommand{\trMatRespIV}{\translatioCantus{
Zaskvěj se! \grestar{} Zaskvěj se! Zde je tvoje světlo, \grestar{}
a nad tebou vzchází Hospodinova sláva. \Vbardot{} Ve tvém světle budou
kráčet národy a králové ve tvé vycházející záři.}}

\newcommand{\trMatBenedictioV}{\translatioCantus{Rač, pane, požehnat.
Nechť nám Kristus dá radost věčného života. \Rbardot{} Amen.}}

% Nea Marie Kuchařová, http://www.stjoseph.cz/slavime-zjeveni-pane/
\newcommand{\trMatLecV}{\translatioCantus{
    Velká Herodova krutost, se kterou se snažil zničit tohoto obávaného Krále
    \ji{}ž brzy po Jeho narození, nevědomky posloužila Milosti tak, že zatímco se
    tento tyran pokoušel zabít neznámé malé dítě vražděním náhodných neviňátek,
    jeho neslavný skutek posloužil k rozšíření zprávy o příchodu nebeského
    panovníka; rychlejšímu rozšíření zprávy napomohla jak její novost velikého
    významu, tak krutost pronásledovatele. Poté byl Spa\-si\-tel odnesen do Egypta,
    národa dlouhodobě utvrzeného v modlářství, který mohl díky tajemné síle
    vycházející ze Spasitele, i když Jeho přítomnost byla neznáma, př\ij{}mout
    světlo spásy velmi brzo, a př\ij{}mout pravdu ještě dříve, než odvrhli
    pověru.}}

\newcommand{\trMatRespV}{\translatioCantus{Všichni \grestar{} ze Sáby
př\ij{}dou, přinesou zlato a kadidlo a budou hlásat Hospodinovu chválu. \gredagger{}
Aleluja, aleluja, aleluja. \Vbardot{}
Králové Taršíše a ostrovů odvedou daně,
králové ze Sáby a ze Seby podají obětní dar.}}

\newcommand{\trMatBenedictioVI}{\translatioCantus{Rač, pane, požehnat.
Bůh rozněť v nás oheň své lásky. \Rbardot{} Amen.}}

% Nea Marie Kuchařová, http://www.stjoseph.cz/slavime-zjeveni-pane/
\newcommand{\trMatLecVI}{\translatioCantus{
    Milovaní bratři, poznejme v mudrcích, kteří se přišli poklonit Kristu, první
    ovoce naší víry a našeho povolání; a v jásotu duše slavme počátek blažené
    naděje. Od této chvíle jsme totiž začali přicházet k věčnému dědictví; od
    této chvíle se nám otevřela tajemství Písma, která hovoří o Kristu; a
    pravda, kterou židovská zaslepenost nepř\ij{}ala, předala své světlo všem
    národům. Oslavujme tedy tento přesvatý den, kdy se zjevil původce naší
    spásy: a tomu, kterého mágové uctívali v jeslích, se klanějme jako
    všemohoucímu v nebesích.  A jako mu oni ze svých pokladů přinesli mystické
    dary, tak i my přinesme ze svých srdcí, co je Boha hodno.}}

\newcommand{\trMatRespVI}{\translatioCantus{Mudrci přišli od Východu do
Jeruzaléma a řekli: ,,Kde je ten, který se právě narodil, jehož hvězdu jsme
viděli? \gredagger{} A přišli jsme se mu poklonit.`` \Vbardot{}
Viděli jsme jeho hvězdu na Východě.}}

\newcommand{\trMatAntVII}{\translatioCantus{Pojďme, klaňme se mu: jeť on Pán, náš Bůh.}}

\newcommand{\trMatAntVIII}{\translatioCantus{Klaňte se Pánu \grestar{} ve svatém
jeho dvoře, aleluja.}}

\newcommand{\trMatAntIX}{\translatioCantus{Klanějte se Bohu, \grestar{}
aleluja, všichni jeho andělé, aleluja.}}

\newcommand{\trMatAbsolutioIII}{\translatioCantus{Z okovů našich hříchů,
\grestar{} vysvoboď nás všemohoucí a milosrdný Pán. \Rbardot{} Amen.}}

\newcommand{\trMatVersusIII}{\translatioCantus{\Vbardot{} Klaňte se Pánu, aleluja.
\Rbardot{} Ve svatém jeho dvoře, aleluja.}}

\newcommand{\trMatBenedictioVII}{\translatioCantus{Rač, pane, požehnat.
Čtení evangelia nechť je nám \grestar{} spásou a ochranou. \Rbardot{} Amen.}}

\newcommand{\trMatLecVIIa}{\translatioCantus{
    Když se Ježíš narodil za dnů krále Heroda v judském Betlémě, hle, do
    Jeruzaléma přišli mudrcové od Východu a řekli: ,,Kde je ten židovský
    král, který se právě narodil?\mbox{}``}}

\newcommand{\trMatLecVIIb}{\translatioCantus{
Jak jste slyšeli, nejdražší bratři, v evangeliu, když se narodil nebeský
král, znepoko\ji{}l se král pozemský. Pozemská vznešenost je totiž velmi
zahanbena, když se otevírá nebeská sláva. Musíme ale vyzkoumat, co znamená,
že při narození Páně se zjevil pastýřům v Jude\ji{} anděl, avšak mágy
přicházející se klanět od východu nevedl anděl, leč hvězda. Židům totiž,
jakožto těm, kteří užívají svého rozumu, to musel sdělit rozumný živý tvor,
to je anděl. Ale pohané, kteří neznají rozumu, jsou vedeni k poznání Pána ne
hlasem, leč znameními. Proto také Pavel říká: Proroctví je pro věřící, ne
pro nevěřící; znamení pro nevěřící, ne pro věřící. Ale i oněm je dáno
proroctví jako věřícím a těmto znamení jako nevěřícím.}}

\newcommand{\trMatRespVII}{\translatioCantus{
Hvězda, \grestar{} kterou mudrci viděli na Východě, šla před nimi, až přišli
na místo, kde bylo dítě. \gredagger{}
Když \ji{} uviděli, velice se zaradovali. \Vbardot{}
Vešli tedy do příbytku, uviděli dítě s jeho matkou Marií, padli na zem a
klaněli se mu.}}

\newcommand{\trMatBenedictioVIII}{\translatioCantus{Rač, pane, požehnat.
Boží pomoc \grestar{} buď vždy s námi. \Rbardot{} Amen.}}

\newcommand{\trMatLecVIII}{\translatioCantus{
Také si povšimněme, že našeho Spasitele, když byl \ji{}ž plného věku, právě
těmto pohanům zvěstovali apoštolové, ale když byl malý a ještě pro povinnost
k lidskému tělu nemluvil, zvěstovala \ji{}m ho hvězda. Rozumný řád totiž
vyžadoval, aby Pána hovořícího nám zvěstovali hovořící kazatelé, kdežto ještě
nemluvícího němé prvky. Ale při všech těchto znameních, jež se ukázala ať
\ji{}ž při narození Páně, nebo při jeho smrti, si všimněme, jaká byla
zatvrzelost srdce u Židů, kteří jej nepoznali ani skrze dar proroctví ani
skrze zázraky.}}

\newcommand{\trMatRespVIII}{\translatioCantus{
Když mudrci uviděli hvězdu, velice se zaradovali. \gredagger{}
Vešli tedy do příbytku, uviděli dítě s jeho matkou Marií, padli na zem a
klaněli se mu; \ddag{} potom otevřeli své pokladnice a podali mu dary:
zlato, kadidlo a myrhu. \Vbardot{} Hvězda, kterou mudrci viděli na Východě,
šla před nimi, až se zastavila nad místem, kde bylo dítě.}}

\newcommand{\trMatBenedictioIX}{\translatioCantus{Rač, pane, požehnat.
Slova evangelia \grestar{} nechť smyjí naše provinění. \Rbardot{} Amen.}}

\newcommand{\trMatLecIX}{\translatioCantus{
Všechny prvky svědčily o příchodu svého původce. Abych po lidsku o tom něco
pověděl: Nebesa poznala, že jde o Boha, neboť hned vyslala hvězdu. Poznalo
ho i moře, neboť pod jeho chodidly se ukázalo schůdné. Poznala ho země,
neboť se otřásla, když umíral. Poznalo jej slunce, neboť skrylo své paprsky.
Skály a zdi ho poznaly, neboť v čas jeho smrti pukaly. A podsvětí jej
poznalo, neboť vrátilo mrtvé, jež drželo. A přesto jej, jehož všechno
nerozumné stvoření vycítilo coby Pána, nepoznala srdce nevěřících Židů, nad
kámen tvrdší, jež se nedala zkrušit k pokání.}}

% from the Czech Liturgia horarum
\newcommand{\trTeDeum}{\begin{translatioMulticol}{3}

Bože, tebe chválíme, 
tebe, Pane, velebíme.

Tebe, věčný Otče, 
oslavuje celá země.

Všichni andělé, 
cherubové i serafové,

všechny mocné nebeské zástupy 
bez ustání volají:

Svatý, Svatý, Svatý, 
Pán, Bůh zástupů.

Plná jsou nebesa i země 
tvé vznešené slávy.

Oslavuje tě 
sbor tvých apoštolů,

chválí tě 
velký počet proroků,

vydává o tobě svědectví 
zástup mučedníků;

a po celém světě 
vyznává tě tvá církev:

neskonale velebný, 
všemohoucí Otče,

úctyhodný Synu Boží, 
pravý a jediný,

božský Utěšiteli, 
Duchu svatý.

Kriste, Králi slávy, 
tys od věků Syn Boha Otce;

abys člověka vykoupil, 
stal ses člověkem a narodil ses z Panny;

zlomil jsi osten smrti 
a otevřel věřícím nebe;

sedíš po Otcově pravici 
a máš účast na jeho slávě.

Věříme, že př\ij{}deš soudit, 

a proto tě prosíme:
přispěj na pomoc svým služebníkům, 
vždyť jsi je vykoupil svou předrahou krví;

dej, ať se radují s tvými svatými 
ve věčné slávě.

Zachraň, Pane, svůj lid, žehnej svému dědictví, 
veď ho a stále pozvedej.

Každý den tě budeme velebit 
a chválit tvé jméno po všechny věky.

Pomáhej nám i dnes, 
ať se nedostaneme do područí hříchu.

Smiluj se nad námi, Pane, 
smiluj se nad námi.

Ať spočine na nás tvé milosrdenství, 
jak doufáme v tebe.

Pane, k tobě se utíkáme, 
ať nejsme zahanbeni na věky. 
\end{translatioMulticol}}

\newcommand{\trMatEvangelium}{\translatioCantus{
    Když se Ježíš narodil za dnů krále Heroda v judském Betlémě, hle, do
    Jeruzaléma přišli mudrcové od Východu a řekli: ,,Kde je ten židovský
    král, který se právě narodil? Viděli jsme totiž, jak vyšla hvězda, a
    přišli jsme se mu poklonit.`` Když se to dozvěděl král Herodes,
    znepoko\ji{}l se a s ním celý Jeruzalém.  Shromáždil všechny velekněze
    spolu se zákoníky lidu a vyptával se \ji{}ch na místo, kde se má Kristus
    narodit. ,,V judském Betlémě,`` řekli mu, ,,takto je totiž napsáno u
    proroka: \textit{A ty Betléme, zemi judská, nejsi nikterak nejmenší
    z předních Judových měst, neboť z tebe vzejde vévoda, jenž bude pást
    můj izraelský lid.}``
    Tu Herodes potají předvolal mudrce, vyptal se \ji{}ch na přesnou dobu, kdy
    se ta hvězda ukázala, poslal je do Betléma a~řekl: ,,Jděte a pečlivě po
    tom dítěti pátrejte, a až je naleznete, oznamte mi to, abych se mu i já
    přišel poklonit.`` Poté, co krále vyslechli, vydali se na cestu; a hle,
    hvězda, kterou předtím viděli vycházet, šla před nimi, až se zastavila
    nad místem, kde bylo to dítě. Když tu hvězdu uviděli, velice se
    zaradovali. Vešli tedy do příbytku, uviděli dítě s jeho matkou Marií,
    padli na zem a klaněli se mu; potom otevřeli své pokladnice a podali mu
    dary: zlato, kadidlo a myrhu. Když pak ve snu dostali pokyn, aby se k
    Herodovi nevraceli, vydali se \ji{}nou cestou a navrátili se do své země.}}

\newcommand{\trTeDecetLaus}{\translatioCantus{Tobě chvála, Tobě zpěvy, Tobě
sláva, Bohu Otci i Synu i Svatému Duchu, na věky věků. \Rbardot{} Amen.}}

% MASS ---

\newcommand{\trIntroitus}{\translatioCantus{
Ejhle přišel vládce a Pán, v jeho ruce je královská moc a vláda.
{\color{red}\textit{Žl.}} Bože, předej svůj soud králi \grestar{} a právo královskému synu. \Abardot{}
Králové Tarsu a ostrovů poskytnou tribut, \grestar{} Arabští a Sebejští králové přinesou dary. \Abardot{}
Vzývat ho budou všichni králové země, \grestar{} všechny národy mu budou sloužit. \Abardot{}}}

\newcommand{\trGraduale}{\translatioCantus{Všichni ze Sáby př\ij{}dou se
zlatem a kadidlem a rozhlašovat budou chválu Páně. \Vbardot{} Povstaň do světla
Jeruzaléme, neboť nad tebou vzešla sláva Páně.}}

\newcommand{\trAlleluia}{\translatioCantus{Aleluja. \Vbardot{} Viděli jsme jeho hvězdu na
východě a s dary jsme přišli poklonit se Pánu.}}

\newcommand{\trAnnuntiatio}{\translatioCantus{Víte, bratři nejmilejší, že z
přízně Božího milosrdenství jsme se mohli radovat z Narození našeho Pána
Ježíše Krista. A tak vám také oznamujeme o Vzkříšení Spasitele:
Dne {\color{red}…} bude popelec a začátek postu svaté církve a
{\color{red}…} budete oslavovat svatou Paschu Pána našeho Ježíše Krista.
Dne {\color{red}…} bude Nanebevstoupení našeho Pána Ježíše Krista, dne
{\color{red}…} Svátek Letnic. Dne {\color{red}…} Svátek přesvatého Těla Páně, dne
{\color{red}…} první neděle adventu přicházejícího Pána našeho Ježíše Krista, jemuž
čest a~sláva na věky věků.}}

\newcommand{\trOffertorium}{\translatioCantus{Králové Tarsu \grestar{} a ostrovů
poskytnou tribut, Arabští a Sebejští králové přinesou dary a vzývat ho budou
všichni králové země. \grestar{} Všechny národy mu budou sloužit.\\
\Vbardot{} {\color{red}\textit{1.}} Dej, Bože, králi svou spravedlnost,
a své právo královu synu, aby soudil tvůj lid dle práva, tvé utiskované spravedlivě!
Všechny národy mu budou sloužit.\\
\Vbardot{} {\color{red}\textit{2.}} Kéž nesou hory pokoj lidu, pahorky pak spravedlnost!\\
\Vbardot{} {\color{red}\textit{3.}} Kvésti bude v jeho dnech právo a hojnost
blaha, až pomine měsíc. Panovat bude od moře k moři. \grestar{}
Všechny národy mu budou sloužit.}}

\newcommand{\trCommunio}{\translatioCantus{Na východě jsme viděli jeho
hvězdu a s dary jsme přišli klanět se Pánu.}}
