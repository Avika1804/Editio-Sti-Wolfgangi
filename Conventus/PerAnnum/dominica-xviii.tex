\newcommand{\titulus}{\nomenFesti{Transfiguratio D.N.I.C.}
\dies{Die 6. Augusti.}}
\newcommand{\oratio}{\pars{Oratio.}

\noindent Deus, qui fídei sacraménta in Unigéniti tui gloriósa Transfiguratióne patrum testimónio roborásti et adoptiónem filiórum perféctam mirabíliter præsignásti, concéde nobis fámulis tuis, ut, ipsíus dilécti Fílii tui vocem audiéntes, eiúsdem coherédes éffici mereámur.

\pars{Pro pace in Ucraina.} \scriptura{Sir. 50, 25; 2 Esdr. 4, 20; \textbf{H416}}

\vspace{-4mm}

\antiphona{II D}{temporalia/ant-dapacemdomine.gtex}

\vfill

\noindent Deus, a quo sancta desidéria, recta consília et iusta sunt ópera: da servis tuis illam, quam mundus dare non potest, pacem; ut et corda nostra mandátis tuis dédita, et hóstium subláta formídine, témpora sint tua protectióne tranquílla.

\noindent Per Dóminum nostrum Iesum Christum, Fílium tuum, qui tecum vivit et regnat in unitáte Spíritus Sancti, Deus, per ómnia sǽcula sæculórum.

\noindent \Rbardot{} Amen.}
\newcommand{\invitatorium}{\pars{Invitatorium.} \scriptura{Cantor; Psalmus 94; \textbf{H447}}

\vspace{-6mm}

\antiphona{V}{temporalia/inv-summumregem.gtex}}
\newcommand{\hymnusmatutinum}{\pars{Hymnus.}

\cuminitiali{IV}{temporalia/hym-CaelestisFormam.gtex}}
\newcommand{\matutinum}{\subhora{In I. Nocturno}

\pars{Psalmus 1.} \scriptura{Mt. 17, 4; \textbf{H149}}

\vspace{-4mm}

\antiphona{I g\textsuperscript{3}}{temporalia/ant-dominebonumest.gtex}

%\vspace{-2mm}

\scriptura{Ps. 83}

%\vspace{-2mm}

\initiumpsalmi{temporalia/ps83-initium-i-g3-auto.gtex}

%\vspace{-1.5mm}

\input{temporalia/ps83-i-g3.tex}

\vfill

\antiphona{}{temporalia/ant-dominebonumest.gtex}

\vfill
\pagebreak

\pars{Psalmus 2.} \scriptura{Mt. 17, 5}

\vspace{-4mm}

\antiphona{VI F}{temporalia/ant-adhuceoloquente.gtex}

%\vspace{-2mm}

\scriptura{Ps. 96}

%\vspace{-2mm}

\initiumpsalmi{temporalia/ps96-initium-vi-F-auto.gtex}

%\vspace{-1.5mm}

\input{temporalia/ps96-vi-F.tex} \Abardot{}

\vfill
\pagebreak

\pars{Psalmus 3.} \scriptura{Cf. Mt. 17, 5; \textbf{H78}}

\vspace{-4mm}

\antiphona{IV e}{temporalia/ant-voxdecaelissonuit.gtex}

%\vspace{-2mm}

\scriptura{Ps. 98}

%\vspace{-2mm}

\initiumpsalmi{temporalia/ps98-initium-iv-e2-auto.gtex}

\input{temporalia/ps98-iv-e2.tex} \Abardot{}

\vfill
\pagebreak}
\newcommand{\matversus}{\noindent \Vbardot{} In colúmna nubis loquebátur ad eos.

\noindent \Rbardot{} Custodiébant testimónia eius.}
\newcommand{\lectioi}{\pars{Lectio I.} \scriptura{2 Cor. 3, 7-18; 4, 1-6}

\noindent De Epístola secúnda beáti Pauli apóstoli ad Corínthios.

\noindent Fratres: Si ministrátio mortis lítteris deformáta in lapídibus fuit in glória, ita ut non possent inténdere fílii Israel in fáciem Móysis propter glóriam vultus eius, quæ evacuátur, quómodo non magis ministrátio Spíritus erit in glória? Nam si ministérium damnatiónis glória est, multo magis abúndat ministérium iustítiæ in glória. Nam nec glorificátum est, quod cláruit in hac parte, propter excelléntem glóriam; si enim, quod evacuátur, per glóriam est, multo magis, quod manet, in glória est.

\noindent Habéntes ígitur talem spem multa fidúcia útimur, et non sicut Móyses: ponébat velámen super fáciem suam, ut non inténderent fílii Israel in finem illíus, quod evacuátur. Sed obtúsi sunt sensus eórum. Usque in hodiérnum enim diem idípsum velámen in lectióne Véteris Testaménti manet non revelátum, quóniam in Christo evacuátur; sed usque in hodiérnum diem, cum légitur Móyses, velámen est pósitum super cor eórum. Quando autem convérsus fúerit ad Dóminum, aufértur velámen. Dóminus autem Spíritus est; ubi autem Spíritus Dómini, ibi libértas. Nos vero omnes reveláta fácie glóriam Dómini speculántes, in eándem imáginem transformámur a claritáte in claritátem tamquam a Dómini Spíritu.

\noindent Ideo habéntes hanc ministratiónem, iuxta quod misericórdiam consecúti sumus, non defícimus, sed abdicávimus occúlta dedécoris non ambulántes in astútia neque adulterántes verbum Dei, sed in manifestatióne veritátis commendántes nosmetípsos ad omnem consciéntiam hóminum coram Deo.

\noindent Quod si étiam velátum est evangélium nostrum, in his, qui péreunt, est velátum, in quibus deus huius sǽculi excæcávit mentes infidélium, ut non fúlgeat illuminátio evangélii glóriæ Christi, qui est imágo Dei. Non enim nosmetípsos prædicámus sed Iesum Christum Dóminum; nos autem servos vestros per Iesum. Quóniam Deus qui dixit: «De ténebris lux splendéscat», ipse illúxit in córdibus nostris ad illuminatiónem sciéntiæ claritátis Dei in fácie Iesu Christi.}
\newcommand{\responsoriumi}{\pars{Responsorium 1.} \scriptura{\Rbardot{} Cf. Mt. 17, 2.5.6 \Vbardot{} ibid. 17, 3; \textbf{H160}}

\vspace{-5mm}

\responsorium{VIII}{temporalia/resp-splendidafactaest.gtex}{}}
\newcommand{\lectioii}{\pars{Lectio II.} \scriptura{Nn. 6-10: Mélanges d’archéologie et d’histoire 67 [1955], 241-244}

\noindent Ex Sermóne Anastásii Sinaítæ epíscopi in die Transfiguratiónis Dómini.

\noindent Mystérium hoc manifestávit Iesus discípulis suis in monte Thabor. Postquam enim inter eos ámbulans sermónes de regno deque suo áltero in glória advéntu díxerat, ut, qui fortásse non satis certi erant de iis quæ circa regnum nuntiáverat, firmíssime tandem in cordis íntimis convinceréntur, utque étiam ex præséntibus futúra créderent, divínam manifestatiónem in monte Thabor mirabíliter exhíbuit tamquam præfiguratívam imáginem regni cælórum. Ita prorsus ac si díceret: «Ne témporis intervállum incredulitátem in vobis gignat, statim, in præsénti, amen dico vobis, sunt quidam de hic stántibus, qui non gustábunt mortem, donec vídeant Fílium hóminis veniéntem in glória Patris sui».

\noindent Osténdens autem Evangelísta poténtiam Christi cum ipsíus voluntáte conveníre, addit: Et post dies sex assúmit Iesus Petrum et Iacóbum et Ioánnem, et ducit illos in montem excélsum seórsum. Et transfigurátus est ante eos, et resplénduit fácies eius sicut sol, vestiménta autem eius facta sunt sicut nix. Et ecce apparuérunt Móyses et Elías cum eo loquéntes.

\noindent Hæc sunt præséntis sollemnitátis mirácula, hoc est nobis salutáre mystérium quod in monte nunc est adimplétum; simul enim nos modo cóngregat et mors et festívitas Christi. Ut ígitur íntima ineffabílium horum sacrorúmque mysteriórum una cum eléctis inter discípulos a Deo inspirátos penetrémus, vocem divínam sacrámque audiámus, quæ ex alto, e vértice montis nos instánter cónvocat.}
\newcommand{\responsoriumii}{\pars{Responsorium 2.} \scriptura{\Rbardot{} Cf. Mt. 17, 1.2}

\vspace{-5mm}

\responsorium{III}{temporalia/resp-assumensiesus-sinedox.gtex}{}}
\newcommand{\lectioiii}{\pars{Lectio III.}

\noindent Illuc nos opórtet festináre —audénter dico— sicut Iesus, qui hic in cælis dux noster est ac præcúrsor, quocum fulgébimus óculis spiritálibus, lineaméntis quodam modo ánimæ nostræ renováti, ad eius conformáti imáginem, ac sicut ipse sine intermissióne transfiguráti divinǽque natúræ facti consórtes et ad superióra paráti.

\noindent Illuc currámus, animósi ac lætántes, et intrémus in íntimam nubem, facti tamquam Móyses et Elías, vel Iacóbus et Ioánnes. Esto sicut Petrus, in divínam visiónem et apparitiónem raptus, pulchra hac Transfiguratióne transfigurátus, elátus e mundo, abstráctus a terra; relínque carnem, désere creatiónem et convértere ad Creatórem, cui Petrus a se abréptus: Dómine, inquit, bonum est nos hic esse.

\noindent Equidem, Petre, vere bonum est nos hic esse cum Iesu atque hic in sǽcula manére. Quid felícius, quid áltius, quid est præstántius quam esse cum Deo, ipsi conformári, in luce inveníri? Certe unusquísque nostrum, cum Deum in se hábeat et sit in divínam eius imáginem transfigurátus, cum lætítia exclámet: Bonum est nos hic esse, ubi ómnia sunt lúcida, ubi gáudium est et beatitúdo et iucúnditas, ubi ómnia in corde tranquílla sunt et seréna et dúlcia, ubi (Christus) Deus conspícitur; ubi mansiónem ipse cum Patre facit et advéniens ait: Hódie salus dómui huic facta est; ubi cum Christo thesáuri exstant et cumulántur bonórum æternórum; ubi primítiæ et imágines futurórum sæculórum velut in spéculo describúntur.}
\newcommand{\responsoriumiii}{\pars{Responsorium 3.} \scriptura{\Rbardot{} Cf. Mt. 17, 2.3.4}

\vspace{-5mm}

\responsorium{VI}{temporalia/resp-videnspetrusmoysen-cumdox.gtex}{}

\vfill
\pagebreak

\subhora{In II. Nocturno}

\pars{Cantica.}

\scriptura{Cf. Mt. 17, 7}

\vspace{-2mm}

\antiphona{VIII G}{temporalia/ant-accessitiesus.gtex}

\vspace{3mm}

\scriptura{Canticum ex Liber Chronicorum, 1 Chr. 29, 10-13}

%\vspace{-2mm}

\initiumpsalmi{temporalia/chronicorum-initium-viii-G-auto.gtex}

\input{temporalia/chronicorum-viii-G.tex}

\vfill
\pagebreak

\scriptura{Canticum Isaiæ Prophetæ, Is. 12, 1-7}

%\vspace{-2mm}

\initiumpsalmi{temporalia/isaiae-initium-viii-G-auto.gtex}

\vfill

\input{temporalia/isaiae-viii-G.tex}

\vfill
\pagebreak

\scriptura{Canticum Isaiaæ, Is. 61, 10-11; 62, 1-7}

\vspace{-3mm}

\initiumpsalmi{temporalia/isaiae4-initium-viii-G-auto.gtex}

\vspace{-1.5mm}

\input{temporalia/isaiae4-viii-G.tex}

\antiphona{}{temporalia/ant-accessitiesus.gtex}

\vfill
\pagebreak

\pars{Versus.}

\noindent \Vbardot{} Magna est glória eius in salutári tuo.

\noindent \Rbardot{} Glóriam et magnum decórem impónes super eum.

\vspace{5mm}

\sineinitiali{temporalia/oratiodominica-mat.gtex}

\vspace{5mm}

\pars{Absolutio.}

\cuminitiali{}{temporalia/absolutio-avinculis.gtex}

\vfill
\pagebreak

\cuminitiali{}{temporalia/benedictio-solemn-evangelica.gtex}

\vspace{7mm}

\pars{Evangelium} \scriptura{Mt. 17, 1-9}

\noindent Léctio sancti Evangélii secúndum Matthǽum.

\noindent In illo témpore: Assúmit Iesus Petrum et Iacóbum et Ioánnem fratrem eius et ducit illos in montem excélsum seórsum. Et transfigurátus est ante eos; et resplénduit fácies eius sicut sol, vestiménta autem eius facta sunt alba sicut lux. Et ecce appáruit illis Móyses et Elías cum eo loquéntes.

\noindent Respóndens autem Petrus dixit ad Iesum: «Dómine, bonum est nos hic esse. Si vis, fáciam hic tria tabernácula: tibi unum et Móysi unum et Elíæ unum».

\noindent Adhuc eo loquénte, ecce nubes lúcida obumbrávit eos; et ecce vox de nube dicens: «Hic est Fílius meus diléctus, in quo mihi bene complácui; ipsum audíte». Et audiéntes discípuli cecidérunt in fáciem suam et timuérunt valde.

\noindent Et accéssit Iesus et tétigit eos dixítque eis: «Súrgite et nolíte timére». Levántes autem óculos suos, néminem vidérunt nisi solum Iesum.

\noindent Et descendéntibus illis de monte, præcépit eis Iesus dicens: «Némini dixéritis visiónem, donec Fílius hóminis a mórtuis resúrgat».

\vspace{5mm}

\scriptura{Sermo 51, 3-4. 8: PL 54, 310.311. 313}

\noindent Ex Sermónibus sancti Leónis Magni papæ.

\noindent Aperit Dóminus coram eléctis téstibus glóriam suam, et commúnem illam cum céteris córporis formam tanto splendóre claríficat, ut et fácies eius solis fulgóri símilis, et vestítus candóri nívium esset æquális. In qua transfiguratióne illud quidem principáliter agebátur, ut de córdibus discipulórum crucis scándalum tollerétur; nec conturbáret eórum fidem voluntáriæ humílitas passiónis, quibus reveláta esset abscónditæ excelléntia dignitátis.

\noindent Sed non minóre providéntia spes sanctæ Ecclésiæ fundabátur, ut totum corpus Christi agnósceret quali esset commutatióne donándum, et eius sibi honóris consórtium membra promítterent, qui in cápite præfulsísset.

\noindent De quo idem Dóminus díxerat, cum de advéntus sui maiestáte loquerétur: \emph{Tunc iusti fulgébunt sicut sol in regno Patris sui;} protestánte hoc ipsum beáto Paulo apóstolo et dicénte: \emph{Exístimo enim quod non sunt condígnæ passiónes huius témporis ad futúram glóriam, quæ revelábitur in nobis;} et íterum: \emph{Mórtui enim estis, et vita vestra abscóndita est cum Christo in Deo. Cum enim Christus apparúerit vita vestra, tunc et vos apparébitis cum ipso in glória.}

\noindent {\color{gray} Confirmándis vero apóstolis et ad omnem sciéntiam provehéndis, ália quoque in illo miráculo accéssit instrúctio. Móyses enim et Elías, lex scílicet et prophétæ, apparuérunt cum Dómino loquéntes, ut veríssime in illa quinque virórum præséntia complerétur quod dictum est: \emph{In duóbus vel tribus téstibus stat omne verbum.}

\noindent Quid hoc stabílius, quid fírmius verbo, in cuius prædicatióne véteris et novi testaménti cóncinit tuba et cum evangélica doctrína antiquárum protestatiónum instruménta concúrrunt?

\noindent Astipulántur enim sibi ínvicem utriúsque fœ́deris páginæ; et, quem sub velámine mysteriórum præcedéntia signa promíserant, maniféstum atque perspícuum præséntis glóriæ splendor osténdit; quia, sicut ait beátus Ioánnes, \emph{lex per Móysen data est, grátia autem et véritas per Iesum Christum facta est}; in quo et propheticárum promíssio impléta est figurárum et legálium rátio præceptórum, dum et veram docet prophetíam per sui præséntiam, et possibília facit mandáta per grátiam.}

\noindent Confirmétur ergo secúndum prædicatiónem sacratíssimi Evangélii ómnium fides, et nemo de Christi cruce, per quam mundus redémptus est, erubéscat.

\noindent Nec ídeo quisquam aut pati pro iustítia tímeat, aut de promissórum retributióne diffídat, quia per labórem ad réquiem, et per mortem transítur ad vitam; cum omnem humilitátis nostræ infirmitátem ille suscéperit, in quo, si in confessióne et in dilectióne ipsíus permaneámus, et quod vicit víncimus, et quod promísit accípimus.

\noindent Quia sive ad faciénda mandáta, sive ad toleránda advérsa, præmíssa Patris vox debet semper áuribus nostris insonáre, dicéntis: \emph{Hic est Fílius meus diléctus, in quo mihi bene complácui: ipsum audíte.}

\vfill
\pagebreak

\cuminitiali{I}{temporalia/tedecetlaus.gtex}

\vfill

\rubrica{vel ad libitum:}

\vspace{3mm}

\cuminitiali{II}{temporalia/tedecetlausii.gtex}

\vfill
\pagebreak

\pars{Hymnus Ambrosianus} \scriptura{Alio modo, iuxta morem Romanum}

\vspace{-2mm}

{
\grechangedim{interwordspacetext}{0.26 cm plus 0.15 cm minus 0.05 cm}{scalable}%
\cuminitiali{III}{temporalia/tedeum-romanum-gn.gtex}
\grechangedim{interwordspacetext}{0.22 cm plus 0.15 cm minus 0.05 cm}{scalable}%
}}
\newcommand{\deusinadiutorium}{\grechangedim{interwordspacetext}{0.18 cm plus 0.15 cm minus 0.05 cm}{scalable}%
\cuminitiali{}{temporalia/deusinadiutorium-alter.gtex}
\grechangedim{interwordspacetext}{0.22 cm plus 0.15 cm minus 0.05 cm}{scalable}}
\newcommand{\hymnuslaudes}{\pars{Hymnus}

\cuminitiali{D}{temporalia/hym-DulcisIesuMemoria.gtex}}
\newcommand{\laudes}{\pars{Psalmus 1.} \scriptura{Cf. Mt. 17, 2}

\vspace{-4mm}

\antiphona{VIII G\textsuperscript{5}}{temporalia/ant-hodiedominusiesus.gtex}

%\vspace{-2mm}

\scriptura{Psalmus 62}

%\vspace{-2mm}

\initiumpsalmi{temporalia/ps62-initium-viii-G6-auto.gtex}

%\vspace{-1.5mm}

\input{temporalia/ps62-viii-G6.tex} \Abardot{}

\vfill
\pagebreak

\pars{Psalmus 2.}

\vspace{-4mm}

\antiphona{I d}{temporalia/ant-hodietransfigurato.gtex}

%\vspace{-2mm}

\scriptura{Canticum trium puerorum, Dan. 3, 57-88 et 56}

\initiumpsalmi{temporalia/dan3-initium-i-d-auto.gtex}

\input{temporalia/dan3-i-d-sinedox.tex}

\rubrica{Hic non dicitur Gloria Patri, neque Amen.}

\vfill

\antiphona{}{temporalia/ant-hodietransfigurato.gtex}

\vfill
\pagebreak

\pars{Psalmus 3.}

\vspace{-4mm}

\antiphona{I g}{temporalia/ant-lexpermoysen.gtex}

%\vspace{-2mm}

\scriptura{Psalmus 149}

%\vspace{-2mm}

\initiumpsalmi{temporalia/ps149-initium-i-g-auto.gtex}

\input{temporalia/ps149-i-g.tex} \Abardot{}

\vfill
\pagebreak}
\newcommand{\lectiobrevis}{\pars{Lectio Brevis.} \scriptura{Ap. 21, 10.23}

\noindent Sústulit me ángelus in spíritu super montem magnum et altum et osténdit mihi civitátem sanctam Ierúsalem descendéntem de cælo a Deo. Et cívitas non eget sole neque luna, ut lúceant ei, nam cláritas Dei illuminávit eam, et lucérna eius est Agnus.}
\newcommand{\responsoriumbreve}{\pars{Responsorium breve.} \scriptura{Ps. 8, 6-7}

\cuminitiali{VI}{temporalia/resp-gloriaethonore.gtex}}
\newcommand{\preces}{\noindent Deum Patrem Dómini et Salvatóris nostri Iesu Christi, qui in monte ante discípulos mirabíliter transfigurátus est, \gredagger{} fidénter deprecémur:

\Rbardot{} In lúmine tuo, Dómine, lumen videámus.

\noindent Pater clementíssime, qui Fílium tuum diléctum transfigurásti et in nube lúcida teípsum manifestásti, \gredagger{} fac ut verbum Christi fidéliter audiámus.

\Rbardot{} In lúmine tuo, Dómine, lumen videámus.

\noindent Deus, qui eléctos inebriásti ab ubertáte domus tuæ et torrénte voluptátis tuæ illos potásti, \gredagger{} concéde, ut in córpore Christi fontem vitæ nostræ inveniámus.

\Rbardot{} In lúmine tuo, Dómine, lumen videámus.

\noindent Deus, qui fecísti de ténebris lumen splendéscere et illuxísti in córdibus nostris ad contemplándam claritátem tuam in fácie Christi Iesu, \gredagger{} fove in nobis spíritum contemplatiónis Fílii tui dilécti.

\Rbardot{} In lúmine tuo, Dómine, lumen videámus.

\noindent Deus, qui nos vocásti vocatióne tua sancta, secúndum grátiam tuam nunc manifestátam per illuminatiónem salvatóris nostri Iesu Christi, \gredagger{} illústra per Evangélium inter hómines vitam incorruptíbilem.

\Rbardot{} In lúmine tuo, Dómine, lumen videámus.

\noindent Pater amantíssime, qui talem caritátem dedísti nobis ut fílii Dei nominémur et simus, \gredagger{} præsta, ut, cum apparúerit Christus, símiles ei fiámus.

\Rbardot{} In lúmine tuo, Dómine, lumen videámus.}
\newcommand{\benedictus}{\pars{Canticum Zachariæ.} \scriptura{Mt. 17, 5}

\vspace{-4mm}

\antiphona{VII a}{temporalia/ant-eteccevoxdenube.gtex}


\vspace{-2mm}

\scriptura{Lc. 1, 68-79}

\vspace{-2mm}

\cantusSineNeumas
\initiumpsalmi{temporalia/benedictus-initium-viisoll-a.gtex}

%\vspace{-1.5mm}

\input{temporalia/benedictus-viisoll-a.tex} \Abardot{}}
\newcommand{\dominusnosbenedicat}{\antiphona{D}{temporalia/dominusnosbenedicat-d.gtex}}
\newcommand{\benedicamuslaudes}{\cuminitiali{II}{temporalia/benedicamus-solemnism-laud.gtex}}
\newcommand{\hebdomada}{infra Hebdom. XVIII post Pentecosten.}
\newcommand{\oratioLaudes}{\cuminitiali{}{temporalia/oratio18.gtex}}

% LuaLaTeX

\documentclass[a4paper, twoside, 12pt]{article}
\usepackage[latin]{babel}
%\usepackage[landscape, left=3cm, right=1.5cm, top=2cm, bottom=1cm]{geometry} % okraje stranky
%\usepackage[landscape, a4paper, mag=1166, truedimen, left=2cm, right=1.5cm, top=1.6cm, bottom=0.95cm]{geometry} % okraje stranky
\usepackage[landscape, a4paper, mag=1400, truedimen, left=0.5cm, right=0.5cm, top=0.5cm, bottom=0.5cm]{geometry} % okraje stranky

\usepackage{fontspec}
\setmainfont[FeatureFile={junicode.fea}, Ligatures={Common, TeX}, RawFeature=+fixi]{Junicode}
%\setmainfont{Junicode}

% shortcut for Junicode without ligatures (for the Czech texts)
\newfontfamily\nlfont[FeatureFile={junicode.fea}, Ligatures={Common, TeX}, RawFeature=+fixi]{Junicode}

\usepackage{multicol}
\usepackage{color}
\usepackage{lettrine}
\usepackage{fancyhdr}

% usual packages loading:
\usepackage{luatextra}
\usepackage{graphicx} % support the \includegraphics command and options
\usepackage{gregoriotex} % for gregorio score inclusion
\usepackage{gregoriosyms}
\usepackage{wrapfig} % figures wrapped by the text
\usepackage{parcolumns}
\usepackage[contents={},opacity=1,scale=1,color=black]{background}
\usepackage{tikzpagenodes}
\usepackage{calc}
\usepackage{longtable}
\usetikzlibrary{calc}

\setlength{\headheight}{14.5pt}

% Commands used to produce a typical "Conventus" booklet

\newenvironment{titulusOfficii}{\begin{center}}{\end{center}}
\newcommand{\dies}[1]{#1

}
\newcommand{\nomenFesti}[1]{\textbf{\Large #1}

}
\newcommand{\celebratio}[1]{#1

}

\newcommand{\hora}[1]{%
\vspace{0.5cm}{\large \textbf{#1}}

\fancyhead[LE]{\thepage\ / #1}
\fancyhead[RO]{#1 / \thepage}
\addcontentsline{toc}{subsection}{#1}
}

% larger unit than a hora
\newcommand{\divisio}[1]{%
\begin{center}
{\Large \textsc{#1}}
\end{center}
\fancyhead[CO,CE]{#1}
\addcontentsline{toc}{section}{#1}
}

% a part of a hora, larger than pars
\newcommand{\subhora}[1]{
\begin{center}
{\large \textit{#1}}
\end{center}
%\fancyhead[CO,CE]{#1}
\addcontentsline{toc}{subsubsection}{#1}
}

% rubricated inline text
\newcommand{\rubricatum}[1]{\textit{#1}}

% standalone rubric
\newcommand{\rubrica}[1]{\vspace{3mm}\rubricatum{#1}}

\newcommand{\notitia}[1]{\textcolor{red}{#1}}

\newcommand{\scriptura}[1]{\hfill \small\textit{#1}}

\newcommand{\translatioCantus}[1]{\vspace{1mm}%
{\noindent\footnotesize \nlfont{#1}}}

% pruznejsi varianta nasledujiciho - umoznuje nastavit sirku sloupce
% s prekladem
\newcommand{\psalmusEtTranslatioB}[3]{
  \vspace{0.5cm}
  \begin{parcolumns}[colwidths={2=#3}, nofirstindent=true]{2}
    \colchunk{
      \input{#1}
    }

    \colchunk{
      \vspace{-0.5cm}
      {\footnotesize \nlfont
        \input{#2}
      }
    }
  \end{parcolumns}
}

\newcommand{\psalmusEtTranslatio}[2]{
  \psalmusEtTranslatioB{#1}{#2}{8.5cm}
}


\newcommand{\canticumMagnificatEtTranslatio}[1]{
  \psalmusEtTranslatioB{#1}{temporalia/extra-adventum-vespers/magnificat-boh.tex}{12cm}
}
\newcommand{\canticumBenedictusEtTranslatio}[1]{
  \psalmusEtTranslatioB{#1}{temporalia/extra-adventum-laudes/benedictus-boh.tex}{10.5cm}
}

% volne misto nad antifonami, kam si zpevaci dokresli neumy
\newcommand{\hicSuntNeumae}{\vspace{0.5cm}}

% prepinani mista mezi notovymi osnovami: pro neumovane a neneumovane zpevy
\newcommand{\cantusCumNeumis}{
  \setgrefactor{17}
  \global\advance\grespaceabovelines by 5mm%
}
\newcommand{\cantusSineNeumas}{
  \setgrefactor{17}
  \global\advance\grespaceabovelines by -5mm%
}

% znaky k umisteni nad inicialu zpevu
\newcommand{\superInitialam}[1]{\gresetfirstlineaboveinitial{\small {\textbf{#1}}}{\small {\textbf{#1}}}}

% pars officii, i.e. "oratio", ...
\newcommand{\pars}[1]{\textbf{#1}}

\newenvironment{psalmus}{
  \setlength{\parindent}{0pt}
  \setlength{\parskip}{5pt}
}{
  \setlength{\parindent}{10pt}
  \setlength{\parskip}{10pt}
}

%%%% Prejmenovat na latinske:
\newcommand{\nadpisZalmu}[1]{
  \hspace{2cm}\textbf{#1}\vspace{2mm}%
  \nopagebreak%

}

% mode, score, translation
\newcommand{\antiphona}[3]{%
\hicSuntNeumae
\superInitialam{#1}
\includescore{#2}

#3
}
 % Often used macros

\newcommand{\annusEditionis}{2021}

%%%% Vicekrat opakovane kousky

\newcommand{\anteOrationem}{
  \rubrica{Ante Orationem, cantatur a Superiore:}

  \pars{Supplicatio Litaniæ.}

  \cuminitiali{}{temporalia/supplicatiolitaniae.gtex}

  \pars{Oratio Dominica.}

  \cuminitiali{}{temporalia/oratiodominica.gtex}

  \rubrica{Deinde dicitur ab Hebdomadario:}

  \cuminitiali{}{temporalia/dominusvobiscum-solemnis.gtex}

  \rubrica{In choro monialium loco Dominus vobiscum dicitur:}

  \sineinitiali{temporalia/domineexaudi.gtex}
}

\setlength{\columnsep}{30pt} % prostor mezi sloupci

%%%%%%%%%%%%%%%%%%%%%%%%%%%%%%%%%%%%%%%%%%%%%%%%%%%%%%%%%%%%%%%%%%%%%%%%%%%%%%%%%%%%%%%%%%%%%%%%%%%%%%%%%%%%%
\begin{document}

% Here we set the space around the initial.
% Please report to http://home.gna.org/gregorio/gregoriotex/details for more details and options
\grechangedim{afterinitialshift}{2.2mm}{scalable}
\grechangedim{beforeinitialshift}{2.2mm}{scalable}
\grechangedim{interwordspacetext}{0.22 cm plus 0.15 cm minus 0.05 cm}{scalable}%
\grechangedim{annotationraise}{-0.2cm}{scalable}

% Here we set the initial font. Change 38 if you want a bigger initial.
% Emit the initials in red.
\grechangestyle{initial}{\color{red}\fontsize{38}{38}\selectfont}

\pagestyle{empty}

%%%% Titulni stranka
\begin{titulusOfficii}
\ifx\titulus\undefined
\nomenFesti{Feria II \hebdomada{}}
\else
\titulus
\fi
\end{titulusOfficii}

\vfill

\begin{center}
%Ad usum et secundum consuetudines chori \guillemotright{}Conventus Choralis\guillemotleft.

%Editio Sancti Wolfgangi \annusEditionis
\end{center}

\scriptura{}

\pars{}

\pagebreak

\renewcommand{\headrulewidth}{0pt} % no horiz. rule at the header
\fancyhf{}
\pagestyle{fancy}

\cantusSineNeumas

\hora{Ad Matutinum.} %%%%%%%%%%%%%%%%%%%%%%%%%%%%%%%%%%%%%%%%%%%%%%%%%%%%%

\vspace{2mm}

\cuminitiali{}{temporalia/dominelabiamea.gtex}

\vfill
%\pagebreak

\vspace{2mm}

\ifx\invitatorium\undefined
\pars{Invitatorium.} \scriptura{Lc. 24, 34; Psalmus 94; \textbf{H232}}

\vspace{-6mm}

\antiphona{VI}{temporalia/inv-surrexitdominusvere.gtex}
\else
\invitatorium
\fi

\vfill
\pagebreak

\ifx\hymnusmatutinum\undefined
\pars{Hymnus}

\cuminitiali{VIII}{temporalia/hym-LaetareCaelum.gtex}
\else
\hymnusmatutinum
\fi

\vspace{-3mm}

\vfill
\pagebreak

\ifx\matua\undefined
\else
% MAT A
\pars{Psalmus 1.} \scriptura{Ps. 6, 3}

\vspace{-4mm}

\antiphona{IV E}{temporalia/ant-misereremihi.gtex}

%\vspace{-2mm}

\scriptura{Ps. 6}

%\vspace{-2mm}

\initiumpsalmi{temporalia/ps6-initium-iv-E-auto.gtex}

\input{temporalia/ps6-iv-E.tex} \Abardot{}

\vfill
\pagebreak

\pars{Psalmus 2.} \scriptura{Ps. 110, 1; \textbf{H91}}

\vspace{-4mm}

\antiphona{VIII G}{temporalia/ant-confitebortibi.gtex}

%\vspace{-2mm}

\scriptura{Ps. 9, 2-11}

%\vspace{-2mm}

\initiumpsalmi{temporalia/ps9ii_xi-initium-viii-G-auto.gtex}

\input{temporalia/ps9ii_xi-viii-G.tex} \Abardot{}

\vfill
\pagebreak

\pars{Psalmus 3.} \scriptura{Ps. 9, 20}

\vspace{-4mm}

\antiphona{I g\textsuperscript{3}}{temporalia/ant-exsurgedominenon.gtex}

%\vspace{-2mm}

\scriptura{Ps. 9, 12-21}

%\vspace{-2mm}

\initiumpsalmi{temporalia/ps9xii_xxi-initium-i-g3-auto.gtex}

\input{temporalia/ps9xii_xxi-i-g3.tex} \Abardot{}

\vfill
\pagebreak
\fi
\ifx\matub\undefined
\else
% MAT B
\pars{Psalmus 1.} \scriptura{Ps. 30, 2; \textbf{H90}}

\vspace{-4mm}

\antiphona{VIII G}{temporalia/ant-intuaiustitia.gtex}

%\vspace{-2mm}

\scriptura{Ps. 30, 2-9}

%\vspace{-2mm}

\initiumpsalmi{temporalia/ps30i-initium-viii-G-auto.gtex}

\vspace{-1.5mm}

\input{temporalia/ps30i-viii-G.tex} \Abardot{}

\vfill
\pagebreak

\pars{Psalmus 2.} \scriptura{Ps. 66, 2}

\vspace{-4mm}

\antiphona{E}{temporalia/ant-illuminadomine.gtex}

%\vspace{-2mm}

\scriptura{Ps. 30, 10-17}

%\vspace{-2mm}

\initiumpsalmi{temporalia/ps30ii-initium-e-a-auto.gtex}

\input{temporalia/ps30ii-e-a.tex} \Abardot{}

\vfill
\pagebreak

\pars{Psalmus 3.} \scriptura{Ps. 30, 24}

\vspace{-4mm}

\antiphona{II D}{temporalia/ant-diligitedominum.gtex}

%\vspace{-5mm}

\scriptura{Ps. 30, 20-25}

%\vspace{-2mm}

\initiumpsalmi{temporalia/ps30iii-initium-ii-D-auto.gtex}

\input{temporalia/ps30iii-ii-D.tex} \Abardot{}

\vfill
\pagebreak
\fi
\ifx\matuc\undefined
\else
% MAT C
\pars{Psalmus 1.}

\vspace{-4mm}

\antiphona{VIII G\textsuperscript{3}}{temporalia/ant-alleluia-bv21-n4.gtex}

%\vspace{-2mm}

\scriptura{Ps. 49, 1-6}

%\vspace{-2mm}

\initiumpsalmi{temporalia/ps49i_vi-initium-viii-G3.gtex}

\input{temporalia/ps49i_vi-viii-G2.tex}

\vfill
\pagebreak

\pars{Psalmus 2.}

\scriptura{Ps. 49, 7-15}

%\vspace{-2mm}

\initiumpsalmi{temporalia/ps49vii_xv-initium-viii-G3.gtex}

\input{temporalia/ps49vii_xv-viii-G2.tex}

\vfill
\pagebreak

\pars{Psalmus 3.}

\scriptura{Ps. 49, 16-23}

%\vspace{-2mm}

\initiumpsalmi{temporalia/ps49xvi_xxiii-initium-viii-G3.gtex}

\input{temporalia/ps49xvi_xxiii-viii-G2.tex} \Abardot{}

\vfill
\pagebreak
\fi
\ifx\matud\undefined
\else
% MAT D
\pars{Psalmus 1.} \scriptura{Ps. 72, 1}

\vspace{-4mm}

\antiphona{VIII c}{temporalia/ant-quambonusdeus.gtex}

%\vspace{-2mm}

\scriptura{Ps. 72, 1-12}

%\vspace{-2mm}

\initiumpsalmi{temporalia/ps72i-initium-viii-c-auto.gtex}

%\vspace{-1.5mm}

\input{temporalia/ps72i-viii-c.tex} \Abardot{}

\vfill
\pagebreak

\pars{Psalmus 2.} \scriptura{Ps. 15, 7; \textbf{H99}}

\vspace{-4mm}

\antiphona{II D}{temporalia/ant-benedicamdomino.gtex}

%\vspace{-2mm}

\scriptura{Ps. 72, 13-20}

%\vspace{-2mm}

\initiumpsalmi{temporalia/ps72ii-initium-ii-D-auto.gtex}

\input{temporalia/ps72ii-ii-D.tex} \Abardot{}

\vfill
\pagebreak

\pars{Psalmus 3.} \scriptura{Ps. 72, 28}

\vspace{-4mm}

\antiphona{III b}{temporalia/ant-adhaereredeobonumest.gtex}

%\vspace{-2mm}

\scriptura{Ps. 72, 21-28}

%\vspace{-2mm}

\initiumpsalmi{temporalia/ps72iii-initium-iii-b.gtex}

\input{temporalia/ps72iii-iii-b.tex} \Abardot{}

\vfill
\pagebreak
\fi

\pars{Versus.}

\ifx\matversus\undefined
\ifx\matua\undefined
\else
\noindent \Vbardot{} Da mihi intelléctum et servábo legem tuam. 

\noindent \Rbardot{} Et custódiam illam in toto corde meo.
\fi
\ifx\matub\undefined
\else
\noindent \Vbardot{} Dírige me, Dómine, in veritáte tua, et doce me.

\noindent \Rbardot{} Quia tu es Deus salútis meæ.
\fi
\ifx\matuc\undefined
\else
\noindent \Vbardot{} Cor meum et caro mea, allelúia.

\noindent \Rbardot{} Exsultavérunt in Deum vivum, allelúia.
\fi
\ifx\matud\undefined
\else
\noindent \Vbardot{} Quam dúlcia fáucibus meis elóquia tua, Dómine.

\noindent \Rbardot{} Super mel ori meo.
\fi
\else
\matversus
\fi

\vspace{5mm}

\sineinitiali{temporalia/oratiodominica-mat.gtex}

\vspace{5mm}

\pars{Absolutio.}

\cuminitiali{}{temporalia/absolutio-exaudi.gtex}

\vfill
\pagebreak

\cuminitiali{}{temporalia/benedictio-solemn-benedictione.gtex}

\vspace{7mm}

\lectioi

\noindent \Vbardot{} Tu autem, Dómine, miserére nobis.
\noindent \Rbardot{} Deo grátias.

\vfill
\pagebreak

\responsoriumi

\vfill
\pagebreak

\cuminitiali{}{temporalia/benedictio-solemn-unigenitus.gtex}

\vspace{7mm}

\lectioii

\noindent \Vbardot{} Tu autem, Dómine, miserére nobis.
\noindent \Rbardot{} Deo grátias.

\vfill
\pagebreak

\responsoriumii

\vfill
\pagebreak

\cuminitiali{}{temporalia/benedictio-solemn-spiritus.gtex}

\vspace{7mm}

\lectioiii

\noindent \Vbardot{} Tu autem, Dómine, miserére nobis.
\noindent \Rbardot{} Deo grátias.

\vfill
\pagebreak

\responsoriumiii

\vfill
\pagebreak

\rubrica{Reliqua omittuntur, nisi Laudes separandæ sint.}

\sineinitiali{temporalia/domineexaudi.gtex}

\vfill

\oratio

\vfill

\noindent \Vbardot{} Dómine, exáudi oratiónem meam.
\Rbardot{} Et clamor meus ad te véniat.

\vfill

\noindent \Vbardot{} Benedicámus Dómino.
\noindent \Rbardot{} Deo grátias.

\vfill

\noindent \Vbardot{} Fidélium ánimæ per misericórdiam Dei requiéscant in pace.
\Rbardot{} Amen.

\vfill
\pagebreak

\hora{Ad Laudes.} %%%%%%%%%%%%%%%%%%%%%%%%%%%%%%%%%%%%%%%%%%%%%%%%%%%%%

\cantusSineNeumas

\vspace{0.5cm}
\grechangedim{interwordspacetext}{0.18 cm plus 0.15 cm minus 0.05 cm}{scalable}%
\cuminitiali{}{temporalia/deusinadiutorium-communis.gtex}
\grechangedim{interwordspacetext}{0.22 cm plus 0.15 cm minus 0.05 cm}{scalable}%

\vfill
\pagebreak

\ifx\hymnuslaudes\undefined
\ifx\laudac\undefined
\else
\pars{Hymnus}

\cuminitiali{I}{temporalia/hym-ChorusNovae-praglia.gtex}
\fi
\ifx\laudbd\undefined
\else
\pars{Hymnus}

\cuminitiali{I}{temporalia/hym-ChorusNovae.gtex}
\vspace{-3mm}
\fi
\else
\hymnuslaudes
\fi

\vfill
\pagebreak

\ifx\lauda\undefined
\else
\pars{Psalmus 1.} \scriptura{Ps. 5, 2; \textbf{H93}}

\vspace{-6mm}

\antiphona{VIII a}{temporalia/ant-intellegeclamorem.gtex}

\vspace{-4mm}

\scriptura{Psalmus 5.}

\vspace{-2mm}

\initiumpsalmi{temporalia/ps5-initium-viii-A-auto.gtex}

\vspace{-1.5mm}

\input{temporalia/ps5-viii-A.tex} \Abardot{}

\vfill
\pagebreak

\pars{Psalmus 2.} \scriptura{1 Par. 29, 13}

\vspace{-4mm}

\antiphona{I f}{temporalia/ant-laudamusnomentuum.gtex}

%\vspace{-2mm}

\scriptura{Canticum David, 1 Par. 29, 10-13}

%\vspace{-2mm}

\initiumpsalmi{temporalia/david-initium-i-f-auto.gtex}

\input{temporalia/david-i-f.tex} \Abardot{}

\vfill
\pagebreak

\pars{Psalmus 3.} \scriptura{Ps. 28, 1.2; \textbf{H72}}

\vspace{-4mm}

\antiphona{VII a}{temporalia/ant-affertedomino.gtex}

\scriptura{Psalmus. 28}

\initiumpsalmi{temporalia/ps28-initium-vii-a-auto.gtex}

\input{temporalia/ps28-vii-a.tex} \Abardot{}

\vfill
\pagebreak
\fi
\ifx\laudb\undefined
\else
\pars{Psalmus 1.} \scriptura{Ps. 41, 3; \textbf{H391}}

\vspace{-4mm}

\antiphona{II D}{temporalia/ant-sitivitanima.gtex}

%\vspace{-2mm}

\scriptura{Psalmus 41}

%\vspace{-2mm}

\initiumpsalmi{temporalia/ps41-initium-ii-D-auto.gtex}

%\vspace{-1.5mm}

\input{temporalia/ps41-ii-D.tex}

\vfill

\antiphona{}{temporalia/ant-sitivitanima.gtex}

\vfill
\pagebreak

\pars{Psalmus 2.}

\vspace{-4mm}

\antiphona{III a}{temporalia/ant-ostendenobisdomine.gtex}

%\vspace{-2mm}

\scriptura{Canticum Ecclesiastici, Sir. 36, 1-7.13-16}

%\vspace{-3mm}

\initiumpsalmi{temporalia/ecclesiastici-initium-iii-a-auto.gtex}

\input{temporalia/ecclesiastici-iii-a.tex} \Abardot{}

\vfill
\pagebreak

\pars{Psalmus 3.}

\vspace{-4mm}

\antiphona{II D}{temporalia/ant-operamanuumeius.gtex}

\scriptura{Psalmus 18, 1-7}

\initiumpsalmi{temporalia/ps18i-initium-ii-D-auto.gtex}

\input{temporalia/ps18i-ii-D.tex} \Abardot{}

\vfill
\pagebreak
\fi
\ifx\laudc\undefined
\else
\pars{Psalmus 1.}

\vspace{-4mm}

\antiphona{VIII G}{temporalia/ant-alleluia-turco12.gtex}

%\vspace{-2mm}

\scriptura{Psalmus 83}

%\vspace{-2mm}

\initiumpsalmi{temporalia/ps83-initium-viii-G-auto.gtex}

%\vspace{-1.5mm}

\input{temporalia/ps83-viii-G.tex} \Abardot{}

\vfill
\pagebreak

\pars{Psalmus 2.} \scriptura{Mi. 4, 2}

\vspace{-4mm}

\antiphona{VIII G\textsuperscript{2}}{temporalia/ant-veniteascendamus-tp.gtex}

%\vspace{-2mm}

\scriptura{Canticum Isaiæ, Is. 2, 2-5}

%\vspace{-2mm}

\initiumpsalmi{temporalia/isaiae11-initium-viii-g5-auto.gtex}

\input{temporalia/isaiae11-viii-g5.tex} \Abardot{}

\vfill
\pagebreak

\pars{Psalmus 3.}

\vspace{-4mm}

\antiphona{II D}{temporalia/ant-alleluia-turco8.gtex}

\scriptura{Psalmus 95}

\initiumpsalmi{temporalia/ps95-initium-ii-D-auto.gtex}

\input{temporalia/ps95-ii-D.tex} \Abardot{}

\vfill
\pagebreak
\fi
\ifx\laudd\undefined
\else
\pars{Psalmus 1.} \scriptura{Ps. 89, 1; \textbf{H98}}

\vspace{-4mm}

\antiphona{VI F}{temporalia/ant-dominerefugium.gtex}

%\vspace{-2mm}

\scriptura{Psalmus 89}

%\vspace{-2mm}

\initiumpsalmi{temporalia/ps89-initium-vi-F-auto.gtex}

%\vspace{-1.5mm}

\input{temporalia/ps89-vi-F.tex}

\vfill

\antiphona{}{temporalia/ant-dominerefugium.gtex}

\vfill
\pagebreak

\pars{Psalmus 2.} \scriptura{Is. 42, 10; \textbf{H98}}

\vspace{-4mm}

\antiphona{VI F}{temporalia/ant-cantatedominocanticum.gtex}

%\vspace{-2mm}

\scriptura{Canticum Isaiæ, Is. 42, 10-16}

%\vspace{-3mm}

\initiumpsalmi{temporalia/isaiae10-initium-vi-F-auto.gtex}

\input{temporalia/isaiae10-vi-F.tex} \Abardot{}

\vfill
\pagebreak

\pars{Psalmus 3.} \scriptura{Ps. 134, 1-2}

\vspace{-4mm}

\antiphona{I a}{temporalia/ant-laudatenomendomini.gtex}

\scriptura{Psalmus 134, 1-12}

\initiumpsalmi{temporalia/ps134i-initium-i-a-auto.gtex}

\input{temporalia/ps134i-i-a.tex} \Abardot{}

\vfill
\pagebreak
\fi

\ifx\lectiobrevis\undefined
\ifx\lauda\undefined
\else
\pars{Lectio Brevis.} \scriptura{2 Th. 3, 10-13}

\noindent Si quis non vult operári, nec mandúcet. Audímus enim inter vos quosdam ambuláre inordináte, nihil operántes sed curióse agéntes; his autem, qui eiúsmodi sunt, præcípimus et obsecrámus in Dómino Iesu Christo, ut cum quiéte operántes suum panem mandúcent. Vos autem, fratres, nolíte defícere benefaciéntes.
\fi
\ifx\laudb\undefined
\else
\pars{Lectio Brevis.} \scriptura{Ier. 15, 16}

\noindent Invénti sunt sermónes tui, et comédi eos, et factum est mihi verbum tuum in gáudium et in lætítiam cordis mei, quóniam invocátum est nomen tuum super me, Dómine Deus exercítuum.
\fi
\ifx\laudc\undefined
\else
\pars{Lectio Brevis.} \scriptura{Rom. 10, 8-10}

\noindent Prope te est verbum, in ore tuo et in corde tuo; hoc est verbum fídei, quod prædicámus. Quia si confiteáris in ore tuo: «Dóminum Iesum!», et in corde tuo credíderis quod Deus illum excitávit ex mórtuis, salvus eris. Corde enim créditur ad iustítiam, ore autem conféssio fit in salútem.
\fi
\ifx\laudd\undefined
\else
\pars{Lectio Brevis.} \scriptura{Idt. 8, 25-27}

\noindent Grátias agámus Dómino Deo nostro, qui temptat nos sicut et patres nostros. Mémores estóte quanta fécerit cum Abraham et Isaac, et quanta facta sint Iacob. Quia non sicut illos combússit in inquisitiónem cordis illórum et in nos non ultus est, sed in monitiónem flagéllat Dóminus appropinquántes sibi.
\fi
\else
\lectiobrevis
\fi

\vfill

\ifx\responsoriumbreve\undefined
\pars{Responsorium breve.} \scriptura{Cf. Mt. 28, 6; Cf. Gal. 3, 13}

\cuminitiali{VI}{temporalia/resp-surrexitdominusdesepulcro.gtex}
\else
\responsoriumbreve
\fi

\vfill
\pagebreak

\benedictus

\vspace{-1cm}

\vfill
\pagebreak

\pars{Preces.}

\sineinitiali{}{temporalia/tonusprecum.gtex}

\ifx\preces\undefined
\ifx\lauda\undefined
\else
\noindent Christum magnificémus, plenum grátia et Spíritu Sancto, \gredagger{} et fidénter eum implorémus:

\Rbardot{} Spíritum tuum da nobis, Dómine.

\noindent Concéde nobis diem istum iucúndum, pacíficum et sine mácula, \gredagger{} ut, ad vésperam perdúcti, cum gáudio et mundo corde te collaudáre valeámus.

\Rbardot{} Spíritum tuum da nobis, Dómine.

\noindent Sit hódie splendor tuus super nos, \gredagger{} et opus mánuum nostrárum dírige.

\Rbardot{} Spíritum tuum da nobis, Dómine.

\noindent Osténde fáciem tuam super nos ad bonum in pace, \gredagger{} ut hódie manu tua válida contegámur.

\Rbardot{} Spíritum tuum da nobis, Dómine.

\noindent Réspice propítius omnes, qui oratiónibus nostris confídunt, \gredagger{} eos adímple bonis ánimæ et córporis univérsis.

\Rbardot{} Spíritum tuum da nobis, Dómine.
\fi
\ifx\laudb\undefined
\else
\noindent Salvátor noster fecit nos regnum et sacerdótium, ut hóstias Deo acceptábiles offerámus. \gredagger{} Grati ígitur eum invocémus:

\Rbardot{} Serva nos in tuo ministério, Dómine.

\noindent Christe, sacérdos ætérne, qui sanctum pópulo tuo sacerdótium concessísti, \gredagger{} concéde, ut spiritáles hóstias Deo acceptábiles iúgiter offerámus.

\Rbardot{} Serva nos in tuo ministério, Dómine.

\noindent Spíritus tui fructus nobis largíre propítius, \gredagger{} patiéntiam, benignitátem et mansuetúdinem.

\Rbardot{} Serva nos in tuo ministério, Dómine.

\noindent Da nobis te amáre, ut te, qui es cáritas, possideámus, \gredagger{} et bene ágere, ut per vitam étiam nostram te laudémus.

\Rbardot{} Serva nos in tuo ministério, Dómine.

\noindent Quæ frátribus nostris sunt utília, nos quǽrere concéde, \gredagger{} ut salútem facílius consequántur.

\Rbardot{} Serva nos in tuo ministério, Dómine.
\fi
\ifx\laudc\undefined
\else
\noindent Iesum, quem Pater glorificávit et herédem ómnium géntium constítuit, \gredagger{} exaltémus, orántes:

\Rbardot{} Per victóriam tuam salva nos, Dómine.

\noindent Christe, qui victória tua portas contrivísti infernáles, peccátum delens et mortem, \gredagger{} fac nos hódie peccáti victóres.

\Rbardot{} Per victóriam tuam salva nos, Dómine.

\noindent Tu, qui mortem evacuásti, vitam nobis impértiens novam, \gredagger{} da ut hódie in hac vitæ novitáte ambulémus.

\Rbardot{} Per victóriam tuam salva nos, Dómine.

\noindent Qui vitam mórtuis tribuísti, totum genus humánum de morte ad vitam redúcens, \gredagger{} ómnibus, qui nobis occúrrent, ætérnam vitam concéde.

\Rbardot{} Per victóriam tuam salva nos, Dómine.

\noindent Qui, sepúlcri tui custódes confúndens, discípulos tuos lætificásti, \gredagger{} plenam tibi serviéntibus largíre lætítiam.

\Rbardot{} Per victóriam tuam salva nos, Dómine.
\fi
\ifx\laudd\undefined
\else
\noindent Christum, qui exáudit et salvos facit sperántes in se, \gredagger{} precémur acclamántes:

\Rbardot{} Te laudámus, in te sperámus, Dómine.

\noindent Grátias ágimus tibi, qui dives es in misericórdia, \gredagger{} propter nímiam caritátem, qua dilexísti nos.

\Rbardot{} Te laudámus, in te sperámus, Dómine.

\noindent Qui omni témpore in mundo cum Patre operáris, \gredagger{} nova fac ómnia per Spíritus Sancti virtútem.

\Rbardot{} Te laudámus, in te sperámus, Dómine.

\noindent Aperi óculos nostros et fratrum nostrórum, \gredagger{} ut videámus hódie mirabília tua.

\Rbardot{} Te laudámus, in te sperámus, Dómine.

\noindent Qui nos hódie ad tuum servítium vocas, \gredagger{} nos erga fratres multifórmis grátiæ tuæ fac minístros.

\Rbardot{} Te laudámus, in te sperámus, Dómine.
\fi
\else
\preces
\fi

\vfill

\pars{Oratio Dominica.}

\cuminitiali{}{temporalia/oratiodominicaalt.gtex}

\vfill
\pagebreak

\rubrica{vel:}

\pars{Supplicatio Litaniæ.}

\cuminitiali{}{temporalia/supplicatiolitaniae.gtex}

\vfill

\pars{Oratio Dominica.}

\cuminitiali{}{temporalia/oratiodominica.gtex}

\vfill
\pagebreak

% Oratio. %%%
\oratio

\vspace{-1mm}

\vfill

\rubrica{Hebdomadarius dicit Dominus vobiscum, vel, absente sacerdote vel diacono, sic concluditur:}

\vspace{2mm}

\antiphona{C}{temporalia/dominusnosbenedicat.gtex}

\rubrica{Postea cantatur a cantore:}

\vspace{2mm}

\cuminitiali{VII}{temporalia/benedicamus-tempore-paschali.gtex}

\vspace{1mm}

\vfill
\pagebreak

\end{document}

