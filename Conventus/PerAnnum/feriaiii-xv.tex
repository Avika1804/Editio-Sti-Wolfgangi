\newcommand{\lectioi}{\pars{Lectio I.} \scriptura{1 Reg. 19, 1-8}

\noindent De libro primo Regum.

\noindent In diébus illis: Nuntiávit Achab Iézabel ómnia, quæ fécerat Elías et quómodo occidísset univérsos prophétas gládio. 

\noindent Misítque Iézabel núntium ad Elíam dicens: «Hæc mihi fáciant dii et hæc addant, nisi hac hora cras posúero ánimam tuam sicut ánimam uníus ex illis». 

\noindent Tímuit ergo Elías et surgens ábiit, ut ánimam suam salváret, venítque in Bersabée Iudæ et dimísit ibi púerum suum. 

\noindent Et perréxit in desértum via uníus diéi; cumque venísset et sedéret subter unam iuníperum, petívit ánimæ suæ, ut morerétur, et ait: «Súfficit mihi, Dómine! Tolle ánimam meam; neque enim mélior sum quam patres mei». 

\noindent Proiecítque se et obdormívit in umbra iuníperi; et ecce ángelus tétigit eum et dixit illi: «Surge, cómede!». 

\noindent Respéxit, et ecce ad caput suum subcinerícius panis et vas aquæ; comédit ergo et bibit et rursum obdormívit. 

\noindent Reversúsque est ángelus Dómini secúndo et tétigit eum dixítque illi: «Surge, cómede! Grandis enim tibi restat via». 

\noindent Qui, cum surrexísset, comédit et bibit et ambulávit in fortitúdine cibi illíus quadragínta diébus et quadragínta nóctibus usque ad montem Dei Horeb.}
\newcommand{\responsoriumi}{\pars{Responsorium 1.} \scriptura{\Rbardot{} 3 Reg. 19, 6.8 \Vbardot{} Io. 6, 52; \textbf{LU\textsubscript{927}}}

\vspace{-2mm}

\responsorium{III}{temporalia/resp-respexitelias-sinedox.gtex}{}}
\newcommand{\lectioii}{\pars{Lectio II.} \scriptura{1 Reg. 19, 9a-21}

\noindent Cumque venísset illuc, mansit in spelúnca. 

\noindent Et ecce sermo Dómini ad eum. Et ait ei: «Egrédere et sta in monte coram Dómino». 

\noindent Et ecce Dóminus transit, et ventus grandis et fortis subvértens montes et cónterens petras ante Dóminum: non in vento Dóminus. 

\noindent Et post ventum, commótio: non in commotióne Dóminus. 

\noindent Et post commotiónem, ignis: non in igne Dóminus. 

\noindent Et post ignem, síbilus auræ ténuis. 

\noindent Quod cum audísset Elías, opéruit vultum suum pállio et egréssus stetit in óstio spelúncæ; et ecce vox ad eum dicens: «Quid agis hic, Elía?». 

\noindent Et ille respóndit: «Zelo zelátus sum pro Dómino Deo exercítuum, quia dereliquérunt pactum tuum fílii Israel, altária tua destruxérunt et prophétas tuos occidérunt gládio: et derelíctus sum ego solus, et quærunt ánimam meam, ut áuferant eam».

\noindent Et ait Dóminus ad eum: «Vade et revértere in viam tuam per desértum in Damáscum; cumque pervéneris, unges Házael regem super Sýriam, et Iehu fílium Namsi unges regem super Israel, Eliséum autem fílium Saphat, qui est de Abelméhula, unges prophétam pro te. 

\noindent Et erit: quicúmque fúgerit gládium Házael occídet eum Iehu, et qui fúgerit gládium Iehu interfíciet eum Eliséus. 

\noindent Et relínquam mihi in Israel septem mília: universórum génua, quæ non sunt incurváta ante Baal, et omne os, quod non osculátum est eum».

\noindent Proféctus ergo inde répperit Eliséum fílium Saphat arántem duódecim iugis boum; et ipse cum duodécimo erat. 

\noindent Cumque venísset Elías ad eum, misit pállium suum super illum, qui statim, relíctis bobus, cucúrrit post Elíam et ait: «Osculer, oro, patrem meum et matrem meam, et sic sequar te». 

\noindent Dixítque ei: «Vade et revértere; quid enim feci tibi?».

\noindent Revérsus autem ab eo tulit par boum et mactávit illud et in iugo boum coxit carnes et dedit pópulo, et comedérunt. 

\noindent Consurgénsque ábiit et secútus est Elíam et ministrábat ei.}
\newcommand{\responsoriumii}{\pars{Responsorium 2.} \scriptura{\Rbardot{} Ios. 1, 5-6; \textbf{H161}}

\vspace{-5mm}

\responsorium{II}{temporalia/resp-sicutfuicummoyse-CROCHU.gtex}{}}
\newcommand{\lectioiii}{\pars{Lectio III.} \scriptura{Nn. 12-16. 19: SCh 25 bis, 162-164}

\noindent Ex Tractátu sancti Ambrósii epíscopi De mystériis.

\noindent Te Apóstolus docet \emph{quóniam patres nostri omnes sub nube fuérunt et omnes mare transiérunt et omnes in Móysen baptizáti sunt in nube et in mari.} Dénique et ipse Móyses dicit in cántico: \emph{Misísti spíritum tuum et coopéruit eos mare.} 

\noindent Advértis quod in illo Hebræórum tránsitu iam tunc sacri baptísmatis figúra præcésserit, in quo Ægýptius intériit et Hebrǽus evásit. 

\noindent Quid enim áliud in hoc cotídie sacraménto docémur, nisi quia culpa demérgitur et error abolétur, píetas autem et innocéntia tota pertránsit?

\noindent Audis quia sub nube fuérunt patres nostri, et bona nube quæ carnálium refrigerávit incéndia passiónum, bona nube: obúmbrat quos revísit Spíritus Sanctus.

\noindent Dénique supervénit in Maríam Vírginem, et virtus Altíssimi obumbrávit ei, quando redemptiónem genti generávit humánæ. 

\noindent Et illud miráculum per Móysen in figúra factum est. Si ergo in figúra fuit Spíritus, non adest in veritáte cum Scriptúra tibi dicat \emph{quia lex per Móysen data est, grátia autem et véritas per Iesum Christum facta est?}

\noindent Mara fons amárus erat, misit in eum Móyses lignum et dulcis est factus. 

\noindent Aqua enim sine prædicatióne domínicæ crucis ad nullos usus futúræ salútis est; cum vero salutáris fúerit crucis mystério consecráta, tunc ad usum spiritális lavácri et salutáris póculi temperátur. 

\noindent Sicut ergo in illum fontem Móyses lignum misit, hoc est prophéta, et in hunc fontem sacérdos prædicatiónem domínicæ crucis mittit et aqua fit ad grátiam dulcis.

\noindent Non ergo solis córporis tui credas óculis: magis vidétur quod non vidétur, quia istud temporále illud ætérnum. 

\noindent Magis aspícitur quod óculis non comprehénditur, ánimo autem ac mente cérnitur.

\noindent Dénique dóceat te decúrsa Regnórum léctio. Náaman Syrus erat et lepram habébat nec ab ullo mundári póterat. 

\noindent Tunc ait puélla ex captívis quóniam esset prophéta in Israel qui posset eum a lepræ contagióne mundáre. Sumpto, inquit, auro et argénto perréxit ad regem Israel. 

\noindent Qui, cógnita advéntus eius causa, scidit vestiménta sua, dicens quod tentarétur magis cum de se ea quæ non essent potestátis régiæ posceréntur. 

\noindent Eliséus autem regi intimávit ut ad se dirígeret Syrum, quo cognósceret quod esset Deus in Israel. 

\noindent Et cum venísset, mandávit ei ut sépties mérgeret in Iordáne flúvio. 

\noindent Tunc ille secum tractáre cœpit quod melióres aquas flúmina habérent pátriæ suæ in quibus sæpe mersísset et numquam a lepra esset ablútus, eóque revocátus non obœdiébat mandátis prophétæ. 

\noindent Sed admónitu et persuasiónibus servulórum acquiévit et mersit, mundatúsque íllico intelléxit non aquárum esse quod unusquísque mundátur, sed grátiæ.

\noindent Et ille dubitávit ántequam sanarétur; tu iam sanátus es et ídeo dubitáre non debes.}
\newcommand{\responsoriumiii}{\pars{Responsorium 3.} \scriptura{\Rbardot{} Ps. 76, 20 \Vbardot{} ibid., 19; \textbf{H159}}

\vspace{-5mm}

\responsorium{II}{temporalia/resp-inmariviatua-CROCHU-cumdox.gtex}{}}
\newcommand{\hebdomada}{infra Hebdom. XV post Pentecosten.}
\newcommand{\oratioLaudes}{\cuminitiali{}{temporalia/oratio15.gtex}}

% LuaLaTeX

\documentclass[a4paper, twoside, 12pt]{article}
\usepackage[latin]{babel}
%\usepackage[landscape, left=3cm, right=1.5cm, top=2cm, bottom=1cm]{geometry} % okraje stranky
%\usepackage[landscape, a4paper, mag=1166, truedimen, left=2cm, right=1.5cm, top=1.6cm, bottom=0.95cm]{geometry} % okraje stranky
\usepackage[landscape, a4paper, mag=1400, truedimen, left=0.5cm, right=0.5cm, top=0.5cm, bottom=0.5cm]{geometry} % okraje stranky

\usepackage{fontspec}
\setmainfont[FeatureFile={junicode.fea}, Ligatures={Common, TeX}, RawFeature=+fixi]{Junicode}
%\setmainfont{Junicode}

% shortcut for Junicode without ligatures (for the Czech texts)
\newfontfamily\nlfont[FeatureFile={junicode.fea}, Ligatures={Common, TeX}, RawFeature=+fixi]{Junicode}

\usepackage{multicol}
\usepackage{color}
\usepackage{lettrine}
\usepackage{fancyhdr}

% usual packages loading:
\usepackage{luatextra}
\usepackage{graphicx} % support the \includegraphics command and options
\usepackage{gregoriotex} % for gregorio score inclusion
\usepackage{gregoriosyms}
\usepackage{wrapfig} % figures wrapped by the text
\usepackage{parcolumns}
\usepackage[contents={},opacity=1,scale=1,color=black]{background}
\usepackage{tikzpagenodes}
\usepackage{calc}
\usepackage{longtable}
\usetikzlibrary{calc}

\setlength{\headheight}{14.5pt}

% Commands used to produce a typical "Conventus" booklet

\newenvironment{titulusOfficii}{\begin{center}}{\end{center}}
\newcommand{\dies}[1]{#1

}
\newcommand{\nomenFesti}[1]{\textbf{\Large #1}

}
\newcommand{\celebratio}[1]{#1

}

\newcommand{\hora}[1]{%
\vspace{0.5cm}{\large \textbf{#1}}

\fancyhead[LE]{\thepage\ / #1}
\fancyhead[RO]{#1 / \thepage}
\addcontentsline{toc}{subsection}{#1}
}

% larger unit than a hora
\newcommand{\divisio}[1]{%
\begin{center}
{\Large \textsc{#1}}
\end{center}
\fancyhead[CO,CE]{#1}
\addcontentsline{toc}{section}{#1}
}

% a part of a hora, larger than pars
\newcommand{\subhora}[1]{
\begin{center}
{\large \textit{#1}}
\end{center}
%\fancyhead[CO,CE]{#1}
\addcontentsline{toc}{subsubsection}{#1}
}

% rubricated inline text
\newcommand{\rubricatum}[1]{\textit{#1}}

% standalone rubric
\newcommand{\rubrica}[1]{\vspace{3mm}\rubricatum{#1}}

\newcommand{\notitia}[1]{\textcolor{red}{#1}}

\newcommand{\scriptura}[1]{\hfill \small\textit{#1}}

\newcommand{\translatioCantus}[1]{\vspace{1mm}%
{\noindent\footnotesize \nlfont{#1}}}

% pruznejsi varianta nasledujiciho - umoznuje nastavit sirku sloupce
% s prekladem
\newcommand{\psalmusEtTranslatioB}[3]{
  \vspace{0.5cm}
  \begin{parcolumns}[colwidths={2=#3}, nofirstindent=true]{2}
    \colchunk{
      \input{#1}
    }

    \colchunk{
      \vspace{-0.5cm}
      {\footnotesize \nlfont
        \input{#2}
      }
    }
  \end{parcolumns}
}

\newcommand{\psalmusEtTranslatio}[2]{
  \psalmusEtTranslatioB{#1}{#2}{8.5cm}
}


\newcommand{\canticumMagnificatEtTranslatio}[1]{
  \psalmusEtTranslatioB{#1}{temporalia/extra-adventum-vespers/magnificat-boh.tex}{12cm}
}
\newcommand{\canticumBenedictusEtTranslatio}[1]{
  \psalmusEtTranslatioB{#1}{temporalia/extra-adventum-laudes/benedictus-boh.tex}{10.5cm}
}

% volne misto nad antifonami, kam si zpevaci dokresli neumy
\newcommand{\hicSuntNeumae}{\vspace{0.5cm}}

% prepinani mista mezi notovymi osnovami: pro neumovane a neneumovane zpevy
\newcommand{\cantusCumNeumis}{
  \setgrefactor{17}
  \global\advance\grespaceabovelines by 5mm%
}
\newcommand{\cantusSineNeumas}{
  \setgrefactor{17}
  \global\advance\grespaceabovelines by -5mm%
}

% znaky k umisteni nad inicialu zpevu
\newcommand{\superInitialam}[1]{\gresetfirstlineaboveinitial{\small {\textbf{#1}}}{\small {\textbf{#1}}}}

% pars officii, i.e. "oratio", ...
\newcommand{\pars}[1]{\textbf{#1}}

\newenvironment{psalmus}{
  \setlength{\parindent}{0pt}
  \setlength{\parskip}{5pt}
}{
  \setlength{\parindent}{10pt}
  \setlength{\parskip}{10pt}
}

%%%% Prejmenovat na latinske:
\newcommand{\nadpisZalmu}[1]{
  \hspace{2cm}\textbf{#1}\vspace{2mm}%
  \nopagebreak%

}

% mode, score, translation
\newcommand{\antiphona}[3]{%
\hicSuntNeumae
\superInitialam{#1}
\includescore{#2}

#3
}
 % Often used macros

\newcommand{\annusEditionis}{2021}

%%%% Vicekrat opakovane kousky

\newcommand{\anteOrationem}{
  \rubrica{Ante Orationem, cantatur a Superiore:}

  \pars{Supplicatio Litaniæ.}

  \cuminitiali{}{temporalia/supplicatiolitaniae.gtex}

  \pars{Oratio Dominica.}

  \cuminitiali{}{temporalia/oratiodominica.gtex}

  \rubrica{Deinde dicitur ab Hebdomadario:}

  \cuminitiali{}{temporalia/dominusvobiscum-solemnis.gtex}

  \rubrica{In choro monialium loco Dominus vobiscum dicitur:}

  \sineinitiali{temporalia/domineexaudi.gtex}
}

\setlength{\columnsep}{30pt} % prostor mezi sloupci

%%%%%%%%%%%%%%%%%%%%%%%%%%%%%%%%%%%%%%%%%%%%%%%%%%%%%%%%%%%%%%%%%%%%%%%%%%%%%%%%%%%%%%%%%%%%%%%%%%%%%%%%%%%%%
\begin{document}

% Here we set the space around the initial.
% Please report to http://home.gna.org/gregorio/gregoriotex/details for more details and options
\grechangedim{afterinitialshift}{2.2mm}{scalable}
\grechangedim{beforeinitialshift}{2.2mm}{scalable}
\grechangedim{interwordspacetext}{0.22 cm plus 0.15 cm minus 0.05 cm}{scalable}%
\grechangedim{annotationraise}{-0.2cm}{scalable}

% Here we set the initial font. Change 38 if you want a bigger initial.
% Emit the initials in red.
\grechangestyle{initial}{\color{red}\fontsize{38}{38}\selectfont}

\pagestyle{empty}

%%%% Titulni stranka
\begin{titulusOfficii}
\ifx\titulus\undefined
\nomenFesti{Feria III \hebdomada{}}
\else
\titulus
\fi
\end{titulusOfficii}

\vfill

\begin{center}
%Ad usum et secundum consuetudines chori \guillemotright{}Conventus Choralis\guillemotleft.

%Editio Sancti Wolfgangi \annusEditionis
\end{center}

\scriptura{}

\pars{}

\pagebreak

\renewcommand{\headrulewidth}{0pt} % no horiz. rule at the header
\fancyhf{}
\pagestyle{fancy}

\cantusSineNeumas

\hora{Ad Matutinum.} %%%%%%%%%%%%%%%%%%%%%%%%%%%%%%%%%%%%%%%%%%%%%%%%%%%%%

\vspace{2mm}

\cuminitiali{}{temporalia/dominelabiamea.gtex}

\vfill
%\pagebreak

\vspace{2mm}

\ifx\invitatorium\undefined
\pars{Invitatorium.} \scriptura{Lc. 24, 34; Psalmus 94; \textbf{H232}}

\vspace{-6mm}

\antiphona{VI}{temporalia/inv-surrexitdominusvere.gtex}
\else
\invitatorium
\fi

\vfill
\pagebreak

\ifx\hymnusmatutinum\undefined
\pars{Hymnus}

\cuminitiali{VIII}{temporalia/hym-LaetareCaelum.gtex}
\else
\hymnusmatutinum
\fi

\vspace{-3mm}

\vfill
\pagebreak

\ifx\matutinum\undefined
\ifx\matua\undefined
\else
% MAT A
\pars{Psalmus 1.}

\vspace{-4mm}

\antiphona{II D}{temporalia/ant-alleluia-turco7.gtex}

%\vspace{-2mm}

\scriptura{Ps. 9, 22-32}

%\vspace{-2mm}

\initiumpsalmi{temporalia/ps9xxii_xxxii-initium-ii-D-auto.gtex}

\input{temporalia/ps9xxii_xxxii-ii-D.tex}

\vfill
\pagebreak

\pars{Psalmus 2.} \scriptura{Ps. 9, 33-39}

%\vspace{-2mm}

\initiumpsalmi{temporalia/ps9xxxiii_xxxix-initium-ii-D-auto.gtex}

\input{temporalia/ps9xxxiii_xxxix-ii-D.tex}

\vfill
\pagebreak

\pars{Psalmus 3.} \scriptura{Ps. 11}

%\vspace{-2mm}

\initiumpsalmi{temporalia/ps11-initium-ii-D-auto.gtex}

\input{temporalia/ps11-ii-D.tex}

\vfill

\antiphona{}{temporalia/ant-alleluia-turco7.gtex}

\vfill
\pagebreak
\fi
\ifx\matub\undefined
\else
% MAT B
\pars{Psalmus 1.}

\vspace{-4mm}

\antiphona{VI F}{temporalia/ant-alleluia-turco6.gtex}

%\vspace{-2mm}

\scriptura{Ps. 36, 1-11}

%\vspace{-2mm}

\initiumpsalmi{temporalia/ps36i_xi-initium-vi-F-auto.gtex}

\input{temporalia/ps36i_xi-vi-F.tex}

\vfill
\pagebreak

\pars{Psalmus 2.}

\scriptura{Ps. 36, 12-29}

\vspace{-2mm}

\initiumpsalmi{temporalia/ps36xii_xxix-initium-vi-F-auto.gtex}

\input{temporalia/ps36xii_xxix-vi-F.tex}

\vfill
\pagebreak

\pars{Psalmus 3.}

\scriptura{Ps. 36, 30-40}

%\vspace{-2mm}

\initiumpsalmi{temporalia/ps36iii-initium-vi-F-auto.gtex}

\input{temporalia/ps36iii-vi-F.tex}

\antiphona{}{temporalia/ant-alleluia-turco6.gtex}

\vfill
\pagebreak
\fi
\ifx\matuc\undefined
\else
% MAT C
\pars{Psalmus 1.}

\vspace{-4mm}

\antiphona{I g\textsuperscript{5}}{temporalia/ant-alleluia-auglx2.gtex}

%\vspace{-2mm}

\scriptura{Ps. 67, 2-11}

\initiumpsalmi{temporalia/ps67i-initium-i-g5.gtex}

\input{temporalia/ps67i-i-g.tex}

\vfill
\pagebreak

\pars{Psalmus 2.}

\scriptura{Ps. 67, 12-24}

%\vspace{-2mm}

\initiumpsalmi{temporalia/ps67ii-initium-i-g5.gtex}

\input{temporalia/ps67ii-i-g.tex}

\vfill
\pagebreak

\pars{Psalmus 3.}

\scriptura{Ps. 67, 25-36}

\initiumpsalmi{temporalia/ps67iii-initium-i-g5.gtex}

\input{temporalia/ps67iii-i-g.tex}

\vfill

\antiphona{}{temporalia/ant-alleluia-auglx2.gtex}

\vfill
\pagebreak
\fi
\ifx\matud\undefined
\else
% MAT D
\pars{Psalmus 1.}

\vspace{-4mm}

\antiphona{I d\textsuperscript{3}}{temporalia/ant-alleluia-auglx6.gtex}

%\vspace{-2mm}

\scriptura{Ps. 101, 2-12}

%\vspace{-2mm}

\initiumpsalmi{temporalia/ps101ii_xii-initium-i-d3-auto.gtex}

\input{temporalia/ps101ii_xii-i-d3.tex}

\vfill
\pagebreak

\pars{Psalmus 2.} \scriptura{Ps. 101, 13-23}

\vspace{-2mm}

\initiumpsalmi{temporalia/ps101xiii_xxiii-initium-i-d3-auto.gtex}

\input{temporalia/ps101xiii_xxiii-i-d3.tex}

\vfill
\pagebreak

\pars{Psalmus 3.} \scriptura{Ps. 101, 24-29}

%\vspace{-2mm}

\initiumpsalmi{temporalia/ps101iii-initium-i-d3-auto.gtex}

\input{temporalia/ps101iii-i-d3.tex}

\vfill

\antiphona{}{temporalia/ant-alleluia-auglx6.gtex}

\vfill
\pagebreak
\fi
\else
\matutinum
\fi

\pars{Versus.}

\ifx\matversus\undefined
\noindent \Vbardot{} Christus resúrgens ex mórtuis iam non móritur, allelúia.

\noindent \Rbardot{} Mors illi ultra non dominábitur, allelúia.
\else
\matversus
\fi

\vspace{5mm}

\sineinitiali{temporalia/oratiodominica-mat.gtex}

\vspace{5mm}

\pars{Absolutio.}

\cuminitiali{}{temporalia/absolutio-ipsius.gtex}

\vfill
\pagebreak

\cuminitiali{}{temporalia/benedictio-solemn-deus.gtex}

\vspace{7mm}

\lectioi

\noindent \Vbardot{} Tu autem, Dómine, miserére nobis.
\noindent \Rbardot{} Deo grátias.

\vfill
\pagebreak

\responsoriumi

\vfill
\pagebreak

\cuminitiali{}{temporalia/benedictio-solemn-christus.gtex}

\vspace{7mm}

\lectioii

\noindent \Vbardot{} Tu autem, Dómine, miserére nobis.
\noindent \Rbardot{} Deo grátias.

\vfill
\pagebreak

\responsoriumii

\vfill
\pagebreak

\cuminitiali{}{temporalia/benedictio-solemn-ignem.gtex}

\vspace{7mm}

\lectioiii

\noindent \Vbardot{} Tu autem, Dómine, miserére nobis.
\noindent \Rbardot{} Deo grátias.

\vfill
\pagebreak

\responsoriumiii

\vfill
\pagebreak

\rubrica{Reliqua omittuntur, nisi Laudes separandæ sint.}

\sineinitiali{temporalia/domineexaudi.gtex}

\vfill

\oratio

\vfill

\noindent \Vbardot{} Dómine, exáudi oratiónem meam.
\Rbardot{} Et clamor meus ad te véniat.

\vfill

\noindent \Vbardot{} Benedicámus Dómino.
\noindent \Rbardot{} Deo grátias.

\vfill

\noindent \Vbardot{} Fidélium ánimæ per misericórdiam Dei requiéscant in pace.
\Rbardot{} Amen.

\vfill
\pagebreak

\hora{Ad Laudes.} %%%%%%%%%%%%%%%%%%%%%%%%%%%%%%%%%%%%%%%%%%%%%%%%%%%%%

\cantusSineNeumas

\vspace{0.5cm}
\grechangedim{interwordspacetext}{0.18 cm plus 0.15 cm minus 0.05 cm}{scalable}%
\cuminitiali{}{temporalia/deusinadiutorium-communis.gtex}
\grechangedim{interwordspacetext}{0.22 cm plus 0.15 cm minus 0.05 cm}{scalable}%

\vfill
\pagebreak

\ifx\hymnuslaudes\undefined
\ifx\laudac\undefined
\else
\pars{Hymnus}

\cuminitiali{I}{temporalia/hym-ChorusNovae-praglia.gtex}
\fi
\ifx\laudbd\undefined
\else
\pars{Hymnus}

\cuminitiali{I}{temporalia/hym-ChorusNovae.gtex}
\fi
\else
\hymnuslaudes
\fi

\vfill
\pagebreak

\ifx\laudes\undefined
\ifx\lauda\undefined
\else
\pars{Psalmus 1.}

\vspace{-4mm}

\antiphona{IV* e}{temporalia/ant-alleluia-turco9.gtex}

\scriptura{Psalmus 23.}

\initiumpsalmi{temporalia/ps23-initium-iv_-e-auto.gtex}

\input{temporalia/ps23-iv_-e.tex} \Abardot{}

\vfill
\pagebreak

\pars{Psalmus 2.} \scriptura{Tob. 13, 10}

\vspace{-4mm}

\antiphona{VIII G}{temporalia/ant-benedicitedominumomneselecti.gtex}

\scriptura{Canticum Tobiæ, Tob. 13, 2-8}

\initiumpsalmi{temporalia/tobiae-initium-viii-g-auto.gtex}

\input{temporalia/tobiae-viii-g.tex} \Abardot{}

\vfill
\pagebreak

\pars{Psalmus 3.}

\vspace{-4mm}

\antiphona{E}{temporalia/ant-alleluia-praglia-e2.gtex}

%\vspace{-4mm}

\scriptura{Psalmus 32.}

%\vspace{-2mm}

\initiumpsalmi{temporalia/ps32-initium-e-auto.gtex}

\input{temporalia/ps32-e.tex}

\vfill

\antiphona{}{temporalia/ant-alleluia-praglia-e2.gtex}

\vfill
\pagebreak
\fi
\ifx\laudb\undefined
\else
\pars{Psalmus 1.}

\vspace{-4mm}

\antiphona{E}{temporalia/ant-alleluia-praglia-e.gtex}

\scriptura{Psalmus 42.}

\initiumpsalmi{temporalia/ps42-initium-e-e-auto.gtex}

\input{temporalia/ps42-e-e.tex} \Abardot{}

\vfill
\pagebreak

\pars{Psalmus 2.} \scriptura{Is. 38, 17}

\vspace{-4mm}

\antiphona{I g}{temporalia/ant-eruistidomine-tp.gtex}

%\vspace{-2mm}

\scriptura{Canticum Ezechiæ, Is. 38, 10-20}

%\vspace{-2mm}

\initiumpsalmi{temporalia/ezechiae-initium-i-g-auto.gtex}

%\vspace{-1.5mm}

\input{temporalia/ezechiae-i-g.tex}

\vfill

\antiphona{}{temporalia/ant-eruistidomine-tp.gtex}

\vfill
\pagebreak

\pars{Psalmus 3.}

\vspace{-4mm}

\antiphona{VIII c}{temporalia/ant-alleluia-turco16.gtex}

\vspace{-2mm}

\scriptura{Psalmus 64.}

\vspace{-2mm}

\initiumpsalmi{temporalia/ps64-initium-viii-C-auto.gtex}

\input{temporalia/ps64-viii-C.tex} \Abardot{}

\vfill
\pagebreak
\fi
\ifx\laudc\undefined
\else
\pars{Psalmus 1.}

\vspace{-4mm}

\antiphona{VI F}{temporalia/ant-alleluia-turco5.gtex}

\vspace{-2mm}

\scriptura{Psalmus 84.}

\vspace{-2mm}

\initiumpsalmi{temporalia/ps84-initium-vi-F-auto.gtex}

\input{temporalia/ps84-vi-F.tex} \Abardot{}

\vfill
\pagebreak

\pars{Psalmus 2.}

\vspace{-4mm}

\antiphona{VII d}{temporalia/ant-denoctespiritusmeus-tp.gtex}

\vspace{-2mm}

\scriptura{Canticum Isaiæ, Is. 26, 1-12}

\vspace{-2mm}

\initiumpsalmi{temporalia/isaiae3-initium-vii-d.gtex}

\input{temporalia/isaiae3-vii-d.tex} \Abardot{}

\vfill
\pagebreak

\pars{Psalmus 3.}

\vspace{-4mm}

\antiphona{E}{temporalia/ant-alleluia-praglia-e2.gtex}

%\vspace{-2mm}

\scriptura{Psalmus 66.}

%\vspace{-2mm}

\initiumpsalmi{temporalia/ps66-initium-e-auto.gtex}

\input{temporalia/ps66-e.tex} \Abardot{}

\vfill
\pagebreak
\fi
\ifx\laudd\undefined
\else
\pars{Psalmus 1.}

\vspace{-4mm}

\antiphona{VIII G}{temporalia/ant-alleluia-turco12.gtex}

\vspace{-2mm}

\scriptura{Psalmus 100.}

\vspace{-2mm}

\initiumpsalmi{temporalia/ps100-initium-viii-G-auto.gtex}

\input{temporalia/ps100-viii-G.tex} \Abardot{}

\vfill
\pagebreak

\pars{Psalmus 2.} \scriptura{Ps. 50, 19}

\vspace{-4mm}

\antiphona{I f}{temporalia/ant-sacrificiumdeo-tp.gtex}

%\vspace{-2mm}

\scriptura{Canticum Danielis, Dan. 3, 26.27.29.34-41}

%\vspace{-2mm}

\initiumpsalmi{temporalia/dan32-initium-i-f-auto.gtex}

\input{temporalia/dan32-i-f.tex} \Abardot{}

\vfill
\pagebreak

\pars{Psalmus 3.}

\vspace{-4mm}

\antiphona{VI F}{temporalia/ant-alleluia-turco5.gtex}

%\vspace{-2mm}

\scriptura{Psalmus 143, 1-10.}

%\vspace{-2mm}

\initiumpsalmi{temporalia/ps143i_x-initium-vi-F-auto.gtex}

\input{temporalia/ps143i_x-vi-F.tex} \Abardot{}

\vfill
\pagebreak
\fi
\else
\laudes
\fi

\ifx\lectiobrevis\undefined
\pars{Lectio Brevis.} \scriptura{Ac. 13, 30-33}

\noindent Deus suscitávit Iesum a mórtuis; qui visus est per dies multos his, qui simul ascénderant cum eo de Galilǽa in Ierúsalem, qui nunc sunt testes eius ad plebem. Et nos vobis evangelizámus eam, quæ ad patres promíssio facta est, quóniam hanc Deus adimplévit fíliis eórum, nobis resúscitans Iesum, sicut et in Psalmo secúndo scriptum est: Fílius meus es tu; ego hódie génui te.
\else
\lectiobrevis
\fi

\vfill

\ifx\responsoriumbreve\undefined
\pars{Responsorium breve.} \scriptura{Cf. Mt. 28, 6; Cf. Gal. 3, 13}

\cuminitiali{VI}{temporalia/resp-surrexitdominusdesepulcro.gtex}
\else
\responsoriumbreve
\fi

\vfill
\pagebreak

\benedictus

\vspace{-1cm}

\vfill
\pagebreak

\ifx\precestotum\undefined
\pars{Preces.}

\sineinitiali{}{temporalia/tonusprecum.gtex}

\ifx\preces\undefined
\ifx\lauda\undefined
\else
\noindent Exsultémus Christo, qui perémptum sui córporis templum sua virtúte restítuit,~\gredagger{} eíque supplicémus:

\Rbardot{} Fructus resurrectiónis tuæ, Dómine, nobis concéde.

\noindent Christe salvátor, qui in resurrectióne tua muliéribus et Apóstolis gáudium nuntiásti, totum orbem salvíficans,~\gredagger{} testes tuos nos éffice.

\Rbardot{} Fructus resurrectiónis tuæ, Dómine, nobis concéde.

\noindent Qui resurrectiónem ómnibus promisísti, qua ad vitam novam resurgerémus,~\gredagger{} Evangélii tui nos redde præcónes.

\Rbardot{} Fructus resurrectiónis tuæ, Dómine, nobis concéde.

\noindent Tu, qui Apóstolis sǽpius apparuísti et Sanctum eis Spíritum insufflásti,~\gredagger{} creatórem Spíritum rénova in nobis.

\Rbardot{} Fructus resurrectiónis tuæ, Dómine, nobis concéde.

\noindent Tu, qui discípulis tuis promisísti te cum eis mansúrum usque ad consummatiónem sǽculi,~\gredagger{} mane nobíscum hódie sempérque nobis adésto.

\Rbardot{} Fructus resurrectiónis tuæ, Dómine, nobis concéde.
\fi
\ifx\laudb\undefined
\else
\noindent Deum Patrem, cuius Agnus immaculátus tollit peccáta mundi nosque vivíficat,~\gredagger{} grati rogémus:

\Rbardot{} Auctor vitæ, vivífica nos.

\noindent Deus, auctor vitæ, meménto passiónis et resurrectiónis Agni, in cruce occísi,~\gredagger{} eúmque audi, semper interpellántem pro nobis.

\Rbardot{} Auctor vitæ, vivífica nos.

\noindent Expurgáto vétere ferménto malítiæ et nequítiæ,~\gredagger{} fac nos vívere in ázymis sinceritátis et veritátis Christi.

\Rbardot{} Auctor vitæ, vivífica nos.

\noindent Da, ut hódie reiciámus peccátum discórdiæ atque invídiæ,~\gredagger{} nosque redde fratrum necessitátibus magis inténtos.

\Rbardot{} Auctor vitæ, vivífica nos.

\noindent Spíritum evangélicum pone in médio nostri,~\gredagger{} ut hódie et semper in præcéptis tuis ambulémus.

\Rbardot{} Auctor vitæ, vivífica nos.
\fi
\ifx\laudc\undefined
\else
\noindent Exsultémus Christo, qui perémptum sui córporis templum sua virtúte restítuit,~\gredagger{} eíque supplicémus:

\Rbardot{} Fructus resurrectiónis tuæ, Dómine, nobis concéde.

\noindent Christe salvátor, qui in resurrectióne tua muliéribus et Apóstolis gáudium nuntiásti, totum orbem salvíficans,~\gredagger{} testes tuos nos éffice.

\Rbardot{} Fructus resurrectiónis tuæ, Dómine, nobis concéde.

\noindent Qui resurrectiónem ómnibus promisísti, qua ad vitam novam resurgerémus,~\gredagger{} Evangélii tui nos redde præcónes.

\Rbardot{} Fructus resurrectiónis tuæ, Dómine, nobis concéde.

\noindent Tu, qui Apóstolis sǽpius apparuísti et Sanctum eis Spíritum insufflásti,~\gredagger{} creatórem Spíritum rénova in nobis.

\Rbardot{} Fructus resurrectiónis tuæ, Dómine, nobis concéde.

\noindent Tu, qui discípulis tuis promisísti te cum eis mansúrum usque ad consummatiónem sǽculi,~\gredagger{} mane nobíscum hódie sempérque nobis adésto.

\Rbardot{} Fructus resurrectiónis tuæ, Dómine, nobis concéde.
\fi
\ifx\laudd\undefined
\else
\noindent Deum Patrem, cuius Agnus immaculátus tollit peccáta mundi nosque vivíficat,~\gredagger{} grati rogémus:

\Rbardot{} Auctor vitæ, vivífica nos.

\noindent Deus, auctor vitæ, meménto passiónis et resurrectiónis Agni, in cruce occísi,~\gredagger{} eúmque audi, semper interpellántem pro nobis.

\Rbardot{} Auctor vitæ, vivífica nos.

\noindent Expurgáto vétere ferménto malítiæ et nequítiæ,~\gredagger{} fac nos vívere in ázymis sinceritátis et veritátis Christi.

\Rbardot{} Auctor vitæ, vivífica nos.

\noindent Da, ut hódie reiciámus peccátum discórdiæ atque invídiæ,~\gredagger{} nosque redde fratrum necessitátibus magis inténtos.

\Rbardot{} Auctor vitæ, vivífica nos.

\noindent Spíritum evangélicum pone in médio nostri,~\gredagger{} ut hódie et semper in præcéptis tuis ambulémus.

\Rbardot{} Auctor vitæ, vivífica nos.
\fi
\else
\preces
\fi

\vfill

\pars{Oratio Dominica.}

\cuminitiali{}{temporalia/oratiodominicaalt.gtex}

\vfill
\pagebreak

\rubrica{vel:}

\pars{Supplicatio Litaniæ.}

\cuminitiali{}{temporalia/supplicatiolitaniae.gtex}

\vfill

\pars{Oratio Dominica.}

\cuminitiali{}{temporalia/oratiodominica.gtex}
\else
\precestotum
\fi

\vfill
\pagebreak

% Oratio. %%%
\oratio

\vspace{-1mm}

\vfill

\rubrica{Hebdomadarius dicit Dominus vobiscum, vel, absente sacerdote vel diacono, sic concluditur:}

\vspace{2mm}

\ifx\dominusnosbenedicat\undefined
\antiphona{C}{temporalia/dominusnosbenedicat.gtex}
\else
\dominusnosbenedicat
\fi

\rubrica{Postea cantatur a cantore:}

\vspace{2mm}

\ifx\benedicamuslaudes\undefined
\cuminitiali{VII}{temporalia/benedicamus-tempore-paschali.gtex}
\else
\benedicamuslaudes
\fi

\vspace{1mm}

\vfill
\pagebreak

\end{document}

