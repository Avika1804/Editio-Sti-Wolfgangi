\newcommand{\titulus}{\nomenFesti{Dominica VI per Annum.}}
\newcommand{\tedeumsimplex}{Simplex}
\newcommand{\oratio}{\pars{Oratio.}

\noindent Deus, qui te in rectis et sincéris manére pectóribus ásseris, da nobis tua grátia tales exsístere, in quibus habitáre dignéris.

\pars{Pro pace in universo mundo.} \scriptura{Sir. 50, 25; 2 Esdr. 4, 20; \textbf{H416}}

\vspace{-4mm}

\antiphona{II D}{temporalia/ant-dapacemdomine.gtex}

\vfill

\noindent Deus, a quo sancta desidéria, recta consília et iusta sunt ópera: da servis tuis illam, quam mundus dare non potest, pacem; ut et corda nostra mandátis tuis dédita, et hóstium subláta formídine, témpora sint tua protectióne tranquílla.

\noindent Per Dóminum nostrum Iesum Christum, Fílium tuum, qui tecum vivit et regnat in unitáte Spíritus Sancti, Deus, per ómnia sǽcula sæculórum.

\noindent \Rbardot{} Amen.}
\newcommand{\nocturnoii}{\vspace{-4mm}

\pars{Psalmus 4.}

\vspace{-4mm}

\antiphona{III a}{temporalia/ant-confessionemetdecorem.gtex}

%\vspace{-2mm}

\scriptura{Ps. 103, 1-12}

%\vspace{-2mm}

\initiumpsalmi{temporalia/ps103i-initium-iii-a-auto.gtex}

\input{temporalia/ps103i-iii-a.tex} \Abardot{}

\vfill
\pagebreak

\pars{Psalmus 5.}

\vspace{-4mm}

\antiphona{II D}{temporalia/ant-dominumdeumadoremus.gtex}

%\vspace{-2mm}

\scriptura{Ps. 103, 13-23}

\initiumpsalmi{temporalia/ps103ii-initium-ii-D-auto.gtex}

\input{temporalia/ps103ii-ii-D.tex} \Abardot{}

\vfill
\pagebreak

\pars{Psalmus 6.} \scriptura{Ps. 103, 24}

\vspace{-4mm}

\antiphona{E}{temporalia/ant-quammagnificatasunt.gtex}

\vspace{-4mm}

\scriptura{Ps. 103, 24-35}

%\vspace{-2mm}

\initiumpsalmi{temporalia/ps103iii-initium-e.gtex}

\vspace{-1.5mm}

\input{temporalia/ps103iii-e.tex} \Abardot{}

\vfill
\pagebreak}
\newcommand{\nocturnoiii}{\pars{Cantica.} \scriptura{Ps. 62, 2; \textbf{H138}}

\vspace{-4mm}

\antiphona{I f}{temporalia/ant-deusdeusmeus.gtex}

%\vspace{-2mm}

\scriptura{Canticum Isaiæ, Is. 33, 2-10}

%\vspace{-2mm}

\initiumpsalmi{temporalia/isaiae7-initium-i-f-auto.gtex}

\input{temporalia/isaiae7-i-f.tex} \hfill \rubrica{Hic non dicitur antiphona.}

\vfill
\pagebreak

\scriptura{Canticum Isaiæ, Is. 33, 13-17}

%\vspace{-2mm}

\initiumpsalmi{temporalia/isaiae8-initium-i-f-auto.gtex}

\input{temporalia/isaiae8-i-f.tex}

\vfill
\pagebreak

\scriptura{Canticum Ecclesiastici, Sir. 36, 14-19}

%\vspace{-2mm}

\initiumpsalmi{temporalia/ecclesiasticus36-initium-i-f-auto.gtex}

\input{temporalia/ecclesiasticus36-i-f.tex}

\vfill

\antiphona{}{temporalia/ant-deusdeusmeus.gtex}

\vfill
\pagebreak}
\newcommand{\lectioi}{\pars{Lectio I.} \scriptura{Gn. 15, 1-6}

\noindent De libro Génesis.

\noindent Factus est sermo Dómini ad Abram per visiónem dicens: “Noli timére, Abram! Ego protéctor tuus sum, et merces tua magna erit nimis”.

\noindent Dixítque Abram: “Dómine Deus, quid dabis mihi? Ego vadam absque líberis, et heres domus meæ erit Damascénus Elíezer”. Addidítque Abram: “En mihi non dedísti semen, et ecce vernáculus meus heres meus erit”.

\noindent Sed ecce sermo Dómini factus est ad eum: “Non erit hic heres tuus, sed qui egrediétur de viscéribus tuis, ipsum habébis herédem”. Eduxítque eum foras et ait illi: “Súspice cælum et númera stellas, si potes”. Et dixit ei: “Sic erit semen tuum”.

\noindent Crédidit Dómino, et reputátum est ei ad iustítiam.}
\newcommand{\responsoriumi}{\pars{Responsorium 1.} \scriptura{\Rbar{} Gn. 15, 1 \Vbar{} ibid., 7; Ps. 80, 11}

\vspace{-5mm}

\responsorium{VII}{temporalia/resp-factusestsermodomini.gtex}{}

\rubrica{vel ad libitum:}

\vspace{3mm}

\pars{Responsorium 1.} \scriptura{\Rbar{} Gn. 15, 1 \Vbar{} ibid., 7; Ps. 80, 11}

\vspace{-5mm}

\responsorium{VIII}{temporalia/resp-factusestsermodomini-CROCHU.gtex}{}}
\newcommand{\lectioii}{\pars{Lectio II.} \scriptura{Gn. 15, 7-11}

\noindent Dixítque ad eum: “Ego Dóminus, qui edúxi te de Ur Chaldæórum, ut darem tibi terram istam, et possidéres eam”.

\noindent Et ille ait: “Dómine Deus, unde scire possum quod possessúrus sim eam?”.

\noindent Respóndens Dóminus: “Sume, inquit, mihi vítulam triénnem et capram trimam et aríetem annórum trium, túrturem quoque et colúmbam”.

\noindent Qui tollens univérsa hæc divísit ea per médium et utrásque partes contra se altrínsecus pósuit; aves autem non divísit.

\noindent Descenderúntque vólucres super cadávera, et abigébat eas Abram.}
\newcommand{\responsoriumii}{\pars{Responsorium 2.} \scriptura{\Rbardot{} Gn. 22, 15-17 \Vbardot{} ibid., 18; \textbf{H141}}

\vspace{-5mm}

\responsorium{VIII}{temporalia/resp-vocavitangelusdomini-CROCHU.gtex}{}}
\newcommand{\lectioiii}{\pars{Lectio III.} \scriptura{Gn. 15, 12-18}

\noindent Cumque sol occúmberet, sopor írruit super Abram, et ecce horror magnus et tenebrósus invásit eum.

\noindent Dictúmque est ad eum: “Scito prænóscens quod peregrínum futúrum sit semen tuum in terra non sua, et subícient eos servitúti et afflígent quadringéntis annis. Verúmtamen et gentem, cui servitúri sunt, ego iudicábo, et post hæc egrediéntur cum magna substántia. Tu autem ibis ad patres tuos in pace, sepúltus in senectúte bona. Generatióne autem quarta reverténtur huc; necdum enim complétæ sunt iniquitátes Amorræórum usque ad præsens tempus”.

\noindent Cum ergo occubuísset sol, facta est calígo tenebrósa, et appáruit clíbanus fumans et lampas ignis tránsiens inter divisiónes illas. In illo die pépigit Dóminus cum Abram fœdus dicens: “Sémini tuo dabo terram hanc a flúvio Ægýpti usque ad magnum flúvium Euphráten.}
\newcommand{\responsoriumiii}{\pars{Responsorium 3.} \scriptura{\Rbar{} Gn. 15, 6}

\vspace{-5mm}

\responsorium{VIII}{temporalia/resp-crediditabrahamdeo-cumdox.gtex}{}

\rubrica{vel ad libitum:}

\vspace{3mm}

\pars{Responsorium 3.} \scriptura{\Rbar{} Gn. 15, 6}

\vspace{-5mm}

\responsorium{VIII}{temporalia/resp-crediditabrahamdeo-CROCHU-cumdox.gtex}{}}
\newcommand{\lectioiv}{\pars{Lectio IV.} \scriptura{Nn. 17-21 : CSEL 32, 514-517}

\noindent Ex Tractátu sancti Ambrósii epíscopi \emph{De Abraham}.

\noindent Quantum illud quod de præda victóriæ nihil vóluit Abraham contíngere nec oblátum súmere! Mínuit enim fructum triúmphi mercédis suscéptio et benefícii ádimit grátiam. Ideóque quóniam sibi mercédem ab hómine non quæsívit, a Deo accépit, sicut légimus scriptum quia \emph{post hæc verba factum est Dómini verbum ad Abraham in visu dicens: «Noli timére Abraham, ego prótegam te. Merces tua multa erit valde.»}

\noindent Ab ipso quoque Dómino mercédem quam póstulet considerémus. Non divítias ut avárus expóscit, non longævitátem istíus vitæ ut meticulósus mortis, non poténtiam, sed dignum quærit sui herédem labóris.}
\newcommand{\responsoriumiv}{\pars{Responsorium 4.} \scriptura{\Rbar{} Gn. 12, 1-2; \textbf{H140}}

\vspace{-4mm}

\responsorium{II}{temporalia/resp-locutusestdominusadabraham-CROCHU.gtex}{}}
\newcommand{\lectiov}{\pars{Lectio V.}

\noindent \emph{Non erit,} inquit, \emph{heres tuus hic, sed alter qui exíerit de te, ille erit heres tuus.} Per Isaac legítimum fílium illum verum legítimum póssumus intellégere Dóminum Iesum, quem in princípio evangélii secúndum Matthǽum \emph{Abrahæ fílium} légimus, qui verum se Abrahæ gessit herédem, auctóris illúminans successiónem, per quem Abraham respéxit in cælum et splendórem suæ posteritátis agnóvit non minus illústrem quam stellárum cæléstium fulget cláritas.}
\newcommand{\responsoriumv}{\pars{Responsorium 5.} \scriptura{\Rbar{} Gn. 13, 18 \Vbar{} ibid., 14.15}

\vspace{-5mm}

\responsorium{VI}{temporalia/resp-movensigiturabram.gtex}{}

\rubrica{vel ad libitum:}

\vspace{3mm}

\pars{Responsorium 5.} \scriptura{\Rbar{} Gn. 13, 18 \Vbar{} ibid., 14.15}

\vspace{-5mm}

\responsorium{VI}{temporalia/resp-movensigiturabraham-CROCHU.gtex}{}}
\newcommand{\lectiovi}{\pars{Lectio VI.}

\noindent Quómodo autem Abrahæ propágo diffúsa est nisi per fídei hereditátem, per quam cælo comparámur, conférimur ángelis, æquámur stellis? Ideo ait: \emph{Sic erit semen tuum. Et crédidit,} inquit, \emph{Abraham Deo.} Quid crédidit? Christum sibi per susceptiónem córporis herédem futúrum. Ut scias quia hoc crédidit, Dóminus ait: \emph{Abraham diem meum vidit et gavísus est. Ideo reputátum est illi ad iustítiam,} quia ratiónem non quæsívit, sed promptíssima fide crédidit. Bonum est ut ratiónem prævéniat fides, ne tamquam ab hómine ita a Dómino Deo nostro ratiónem videámur exígere. Etenim quam indígnum ut humánis testimóniis de álio credámus, Dei oráculis de se non credámus! Imitémur ergo Abraham, ut herédes simus terræ per iustítiam fídei, per quam ille mundi heres factus est.}
\newcommand{\responsoriumvi}{\pars{Responsorium 6.} \scriptura{\textbf{H141}}

\vspace{-5mm}

\responsorium{I}{temporalia/resp-dumstaretabrahamadradicem-CROCHU-cumdox.gtex}{}}
\newcommand{\evangelium}{
\pars{Versus.} \scriptura{Ps. 118, 148}

% Versus. %%%
\sineinitiali{temporalia/versus-praevenerunt.gtex}

\vspace{5mm}

\sineinitiali{temporalia/oratiodominica-mat.gtex}

\vspace{5mm}

\pars{Absolutio.}

\cuminitiali{}{temporalia/absolutio-avinculis.gtex}

\vfill
\pagebreak

\cuminitiali{}{temporalia/benedictio-solemn-evangelica.gtex}

\vspace{7mm}

\pars{Evangelium} \scriptura{Mt. 5, 20-37}

\noindent Léctio sancti Evangélii secúndum Matthǽum.

\noindent In illo témpore: Dixit Iesus discípulis suis:

\noindent «Dico vobis: Nisi abundáverit iustítia vestra plus quam scribárum et pharisæórum, non intrábitis in regnum cælórum.

\noindent Audístis quia dictum est antíquis: “Non occídes; qui autem occíderit, reus erit iudício”. Ego autem dico vobis: Omnis, qui iráscitur fratri suo, reus erit iudício.

\noindent Audístis quia dictum est: “Non mœcháberis”. Ego autem dico vobis: Omnis, qui víderit mulíerem ad concupiscéndum eam, iam mœchátus est eam in corde suo.

\noindent Iterum audístis quia dictum est antíquis: “Non periurábis; reddes autem Dómino iuraménta tua”. Ego autem dico vobis: Non iuráre omníno.

\noindent Sit autem sermo vester: “Est, est”, “Non, non”; quod autem his abundántius est, a Malo est».

\scriptura{Tract. 21: CCL 9A, 295}

\noindent Ex Tractátibus sancti Chromátii Aquileiénsis epíscopi in Evangélium Matthǽi.

\noindent {\color{gray} \emph{Non veni sólvere legem, sed adimplére.} Id est, quod minus erat, áddere: præcépta scílicet Legis in mélius reformáre. Unde et sanctus Apóstolus ait: \emph{Legem ergo destrúimus per fidem? Absit: sed Legem constitúimus.} Rudi et duro pópulo Legis trádita sunt mandáta iustítiæ, perfécto autem et fidéli pópulo evangélica tradúntur praecépta consummátae fídei et cæléstis iustítias. Lex præcépit non occidéndum. Evangélium vero non irascéndum sine causa ut omnem radícem peccáti auférret de córdibus nostris, quia per iracúndiam potest étiam usque ad homicídium perveníri.

\noindent Unde non immérito beátus Iob in libro suo ita testátus est dicens: \emph{Stultum intérficit iracúndia, sedúctum autem occídit æmulátio;} David quoque ita ait: \emph{Irascímini et nolíte peccáre, dícite in córdibus vestris et in cubílibus vestris compungímini.} Quod testimónium sanctus Apóstolus interpretátus est dicens: \emph{Sol non óccidat super iracúndiam vestram, neque detis locum diábolo.} Cum ergo irásci sine causa non líceat, multo magis homicídii crimen admíttere. Et cum iracúndia rea teneátur in futúro iudício, quam putámus pœnam habitúrum ipsíus scéleris factórem?}

\noindent \emph{Qui autem díxerit, inquit, fratri suo: «Raca», reus erit concílio. Qui autem díxerit: «Fátue», reus erit gehénnæ ignis.} Ita docet nos Dóminus per ómnia esse perféctos, ut ne lévibus quidem vel vanis sermónibus futúro iudício rei teneámur. Vetat namque dici fratri: \emph{Raca,} id est vácue et inánis; non enim debet dici vácuus et inánis, qui fide et Sancto Spíritu plenus est. Neque enim debet frater \emph{fátuus} appellári, qui credéndo in Christo, divínæ sapiéntiæ grátiam consecútus est. Unde et cum per Salomónem Spíritus Sanctus de viro evangélico loquerétur, ita testátus est dicens: \emph{Beátus qui non est lapsus verbo oris sui et non est stimulátus tristítia delícti.}

\noindent Quaprópter si vólumus múnera nostra Deo placére, exclúdere iracúndiam de corde debémus, interfícere malítiam contra fratrem suscéptam, tenére vero pacem fratérnam, serváre caritátem, dilígere unanimitátem, custodíre concórdiam, ut placére Dómino mereámur, qui est benedíctus in sǽcula. Amen.

\vfill
\pagebreak

\pars{Responsorium 7.} \scriptura{\Rbardot{} Ps. 118, 173 \Vbardot{} ibid., 176; \textbf{H85}}

\vspace{-5mm}

\responsorium{IV}{temporalia/resp-fiatmanustua-CROCHU-cumdox.gtex}{}

\vfill
\pagebreak
}
\newcommand{\hymnuslaudes}{\pars{Hymnus} \scriptura{Alcuinus (\olddag{} 804)}

\cuminitiali{IV}{temporalia/hym-EcceIam.gtex}}
\newcommand{\laudes}{\pars{Psalmus 1.} \scriptura{Ps. 117, 28; \textbf{H142}}

\vspace{-4mm}

\antiphona{VIII c}{temporalia/ant-deusmeusestu.gtex}

\scriptura{Psalmus 117.}

\initiumpsalmi{temporalia/ps117-initium-viii-C-auto.gtex}

\input{temporalia/ps117-viii-C.tex}

\vfill

\antiphona{}{temporalia/ant-deusmeusestu.gtex}

\vfill
\pagebreak

\pars{Psalmus 2.} \scriptura{Cf. Dn. 3, 57; \textbf{H142}}

\vspace{-4mm}

\antiphona{IV E}{temporalia/ant-hymnumdicite.gtex}

\scriptura{Canticum Danielis, Dan. 3, 52-57}

%\vspace{-3mm}

\initiumpsalmi{temporalia/dan33-initium-iv-E-auto.gtex}

\input{temporalia/dan33-iv-E.tex} \Abardot{}

\vfill
\pagebreak

\pars{Psalmus 3.} \scriptura{Ps. 150, 4; \textbf{H99}}

\vspace{-4mm}

\antiphona{I f}{temporalia/ant-intympanoetchoro.gtex}

\scriptura{Psalmus 150}

\initiumpsalmi{temporalia/ps150-initium-i-f-auto.gtex}

\input{temporalia/ps150-i-f.tex} \Abardot{}

\vfill
\pagebreak}
\newcommand{\benedictus}{\pars{Canticum Zachariæ.} \scriptura{Mt. 5, 21; \textbf{H428}}

\vspace{-4mm}

\antiphona{VIII G}{temporalia/ant-audistisquiadictum.gtex}

\vspace{-2mm}

\scriptura{Lc. 1, 68-79}

\vspace{-2mm}

\cantusSineNeumas
\initiumpsalmi{temporalia/benedictus-initium-viiisoll-g-auto.gtex}

%\vspace{-1.5mm}

\input{temporalia/benedictus-viiisoll-g.tex} \Abardot{}}
\include{hebdomadavi}
\include{dominica}
