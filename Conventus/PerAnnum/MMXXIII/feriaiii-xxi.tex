\newcommand{\titulus}{\nomenFesti{In Passione S. Ioannis Baptistæ.}
\dies{Die 29. Augusti.}}
\newcommand{\oratio}{\pars{Oratio.}

\noindent Deus, qui beátum Ioánnem Baptístam et nascéntis et moriéntis Fílii tui Præcursórem esse voluísti, concéde, ut, sicut ille veritátis et iustítiæ martyr occúbuit, ita et nos pro tuæ confessióne doctrínæ strénue certémus.

\pars{Pro pace in universo mundo.} \scriptura{Sir. 50, 25; 2 Esdr. 4, 20; \textbf{H416}}

\vspace{-4mm}

\antiphona{II D}{temporalia/ant-dapacemdomine.gtex}

\vfill

\noindent Deus, a quo sancta desidéria, recta consília et iusta sunt ópera: da servis tuis illam, quam mundus dare non potest, pacem; ut et corda nostra mandátis tuis dédita, et hóstium subláta formídine, témpora sint tua protectióne tranquílla.

\noindent Per Dóminum nostrum Iesum Christum, Fílium tuum, qui tecum vivit et regnat in unitáte Spíritus Sancti, Deus, per ómnia sǽcula sæculórum.

\noindent \Rbardot{} Amen.}
\newcommand{\invitatorium}{\pars{Invitatorium.}

\vspace{-4mm}

\antiphona{VII}{temporalia/inv-agnumdeiquemioannem.gtex}}
\newcommand{\hymnusmatutinum}{\pars{Hymnus.} \scriptura{Beda Venerabilis (\olddag{} 735)}

\vspace{-5mm}

\antiphona{IV}{temporalia/hym-PraecessorAlmus.gtex}}
\newcommand{\lectioi}{\pars{Lectio I.} \scriptura{Ier. 1, 1-10}

\noindent Incipit liber Ieremíæ prophétæ.

\noindent Verba Ieremíæ fílii Helcíæ de sacerdótibus, qui fuérunt in Anathoth in terra Béniamin. Quod factum est verbum Dómini ad eum in diébus Iosíæ fílii Amon regis Iudæ, in tértio décimo anno regni eius. Et factum est in diébus Ióachim fílii Iosíæ regis Iudæ, usque ad consummatiónem undécimi anni Sedecíæ fílii Iosíæ regis Iudæ, usque ad transmigratiónem Ierúsalem in mense quinto.

\noindent Et factum est verbum Dómini ad me dicens: «Priúsquam te formárem in útero, novi te et, ántequam exíres de vulva, sanctificávi te et prophétam géntibus dedi te».

\noindent Et dixi: «Heu, Dómine Deus! Ecce néscio loqui, quia puer ego sum».

\noindent Et dixit Dóminus ad me: «Noli dícere: “Puer sum”, quóniam, ad quoscúmque mittam te, ibis et univérsa, quæcúmque mandávero tibi, loquéris. Ne tímeas a fácie eórum, quia tecum ego sum, ut éruam te», dicit Dóminus.

\noindent Et misit Dóminus manum suam et tétigit os meum et dixit Dóminus ad me:

\noindent «Ecce dedi verba mea in ore tuo; ecce constítui te hódie super gentes et super regna, ut evéllas et déstruas et dispérdas et díssipes et ædífices et plantes».

\noindent {\color{gray} Et factum est verbum Dómini ad me dicens: «Quid tu vides, Ieremía?». Et dixi: «Virgam amýgdali vigilántis ego vídeo».

\noindent Et dixit Dóminus ad me: «Bene vidísti, quia vígilo ego super verbo meo, ut fáciam illud».

\noindent Et factum est verbum Dómini secúndo ad me dicens: «Quid tu vides?».

\noindent Et dixi: «Ollam succénsam ego vídeo; et fácies eius a fácie aquilónis».

\noindent Et dixit Dóminus ad me: «Ab aquilóne pandétur malum super omnes habitatóres terræ; quia ecce ego convocábo ómnia regna aquilónis, ait Dóminus, et vénient et ponent unusquísque sólium suum in intróitu portárum Ierúsalem et contra omnes muros eius in circúitu et contra univérsas urbes Iudæ; et loquar iudícia mea cum eis super omnem malítiam eórum, qui dereliquérunt me et incénsum obtulérunt diis aliénis et adoravérunt opus mánuum suárum.}

\noindent Tu ergo accínge lumbos tuos et surge et lóquere ad eos ómnia, quæ ego præcípio tibi: ne tímeas a fácie eórum, alióquin timére te fáciam vultum eórum.

\noindent Ego quippe dedi te hódie in civitátem munítam et in colúmnam férream et in murum ǽreum contra omnem terram régibus Iudæ, princípibus eius et sacerdótibus et pópulo terræ; et bellábunt advérsum te et non prævalébunt, quia tecum ego sum, ait Dóminus, ut erípiam te».}
\newcommand{\responsoriumi}{\pars{Responsorium 1.} \scriptura{\Rbardot{} Ier. 1, 5 \Vbardot{} Ier. 1, 7; \textbf{H275}}

\vspace{-5mm}

\responsorium{VII}{temporalia/resp-priusquamteformarem-FKP-IER.gtex}{}

\rubrica{vel ad libitum:}

\vspace{3mm}

\pars{Responsorium 1.} \scriptura{\Rbardot{} Dan. 13, 22.23 \Vbardot{} ibidem; \textbf{H417}}

\vspace{-5mm}

\responsorium{VIII}{temporalia/resp-angustiaemihiundique-CROCHU.gtex}{}}
\newcommand{\lectioii}{\pars{Lectio II.} \scriptura{Hom. 23: CCL 122, 354, 356-357}

\noindent Ex Homíliis sancti Bedæ Venerábilis presbýteri.

\noindent Dignam supérnis aspéctibus virtútem sui certáminis beátus præcúrsor domínicæ nativitátis, prædicatiónis et mortis osténdit qui ut Scriptúra ait: \emph{Et si coram homínibus torménta} passus est, \emph{spes} illíus \emph{immortalitáte plena est.} Cuius diem natálem mérito festa celebritáte repétimus quem ipse própria passióne nobis sollémnem réddidit, quem róseo sánguinis sui fulgóre decorávit; mérito memóriam illíus cum spiritáli gáudio venerámur, qui testimónium quod pro Dómino perhíbuit martýrii signáculo conclúsit.

\noindent Neque enim dubitándum est quia beátus Ioánnes pro Redemptóris nostri, quem præcurrébat testimónio, cárcerem et víncula sustínuit pro ipso et ánimam pósuit, cui non est dictum a persecutóre ut Christum negáret sed ut veritátem reticéret; et tamen pro Christo occúbuit.

\noindent Quia enim Christus ipse ait: \emph{Ego sum véritas,} ídeo útique pro Christo, quia pro veritáte sánguinem fudit; et cui nascitúro, prædicatúro, baptizatúro prius nascéndo, prædicándo ac baptizándo testimónium perhibébat, hunc étiam passúrum prior ipse patiéndo signávit.}
\newcommand{\responsoriumii}{\pars{Responsorium 2.} \scriptura{\Rbardot{} Mc. 6, 20; \textbf{H301}}

\vspace{-5mm}

\responsorium{IV}{temporalia/resp-metuebatherodes-CROCHU.gtex}{}}
\newcommand{\lectioiii}{\pars{Lectio III.} \scriptura{Hom. 23: CCL 122, 354, 356-357}

\noindent Talis ígitur ne tantus vir præséntis vitæ términum post longam vinculórum afflictiónem sánguinis effusióne suscépit. Ille qui libertátem supérnæ pacis evangelizábat, ab ímpiis in víncula conícitur; cláuditur obscuritáte cárceris, qui venit testimónium perhibére de lúmine quique ab ipsa luce, quæ Christus est, lucérna ardens et lucens appellári méruit; et ipse próprio cruóre baptizátur, cui Redemptórem mundi baptizáre, cui vocem Patris super illum audíre, cui Spíritus Sancti grátiam in eum descendéntis vidére donátum est. Sed non erat grave, immo leve ac desiderábile erat tálibus torménta pro veritáte temporália pérpeti, quæ perpétuis nóverat remuneránda gáudiis.

\noindent Optábile mortem habébant, quæ natúræ necessitáte inevitábilis imminébat, confésso Christi nómine cum palma vitæ perénnis excípere. Unde bene dicit Apóstolus: \emph{Quia vobis datum est a Christo non solum ut in eum credátis, sed étiam ut pro illo patiámini.} Qui ídeo donum esse dicit Christi ut pro illo patiántur elécti, quia sicut item dicit: \emph{Non sunt condígnæ passiónes huius témporis ad superventúram glóriam quæ revelábitur in nobis.}}
\newcommand{\responsoriumiii}{\pars{Responsorium 3.} \scriptura{\Rbardot{} Mt. 14, 12 \Vbardot{} Mc. 6, 27; \textbf{H301}}

\vspace{-5mm}

\responsorium{I}{temporalia/resp-accedentesdiscipuli-CROCHU-cumdox.gtex}{}}
\newcommand{\hymnuslaudes}{\pars{Hymnus.}

\cuminitiali{IV}{temporalia/hym-ONimisFelix.gtex}}
\newcommand{\laudes}{\pars{Psalmus 1.} \scriptura{Ier. 1, 9.5; \textbf{H274}}

\vspace{-4mm}

\antiphona{VII c}{temporalia/ant-misitdominusmanum-FKP.gtex}

%\vspace{-2mm}

\scriptura{Psalmus 62.}

%\vspace{-1mm}

\initiumpsalmi{temporalia/ps62-initium-vii-c-auto.gtex}

%\vspace{-1.5mm}

\input{temporalia/ps62-vii-c.tex} \Abardot{}

\vfill
\pagebreak

\pars{Psalmus 2.} \scriptura{Mc. 6, 20}

\vspace{-4mm}

\antiphona{II D}{temporalia/ant-metuebatherodes.gtex}

\vspace{-2mm}

%\trAntIV

\scriptura{Canticum trium puerorum, Dan. 3, 57-88 et 56}

\vspace{-2mm}

\initiumpsalmi{temporalia/dan3-initium-ii-D-auto.gtex}

\input{temporalia/dan3-ii-D-sinedox.tex}

\rubrica{Hic non dicitur Gloria Patri, neque Amen.}

\vfill

\antiphona{}{temporalia/ant-metuebatherodes.gtex}

\vfill
\pagebreak

\pars{Psalmus 3.} \scriptura{Mc. 6, 20}

\vspace{-4mm}

\antiphona{VIII G}{temporalia/ant-auditoeo.gtex}

%\vspace{-2mm}

\scriptura{Psalmus 149}

%\vspace{-2mm}

\initiumpsalmi{temporalia/ps149-initium-viii-G-auto.gtex}

\input{temporalia/ps149-viii-G.tex} \Abardot{}

\vfill
\pagebreak}
\newcommand{\lectiobrevis}{\pars{Lectio Brevis.} \scriptura{Is. 49, 1-2}

\noindent Dóminus ab útero vocávit me, de ventre matris meæ recordátus est nóminis mei; et pósuit os meum quasi gládium acútum, in umbra manus suæ protéxit me et pósuit me sicut sagíttam eléctam, in pháretra sua abscóndit me.}
\newcommand{\responsoriumbreve}{\pars{Responsorium breve.} \scriptura{Io. 5, 33.35}

\cuminitiali{VI}{temporalia/resp-vosmisistis.gtex}}
\newcommand{\benedictus}{\pars{Canticum Zachariæ.} \scriptura{Mc. 6, 17.18; \textbf{H302}}

\vspace{-4mm}

\antiphona{I d}{temporalia/ant-arguebatherodemioannes.gtex}

\vspace{-2mm}

\scriptura{Lc. 1, 68-79}

\vspace{-2mm}

\cantusSineNeumas
\initiumpsalmi{temporalia/benedictus-initium-i-d-auto.gtex}

%\vspace{-1.5mm}

\input{temporalia/benedictus-i-d.tex} \Abardot{}}
\newcommand{\preces}{\noindent Christum, qui Ioánnem præcursórem ante fáciem suam misit,~\gredagger{} ut viam Dómini paráret,~\grestar{} fidénter deprecémur:

\Rbardot{} Vísita nos, Oriens ex alto.

\noindent Tu, qui Ioánnem exsultáre fecísti in sinu Elísabeth,~\grestar{} de tuo advéntu in hunc mundum fac nos semper gaudére.

\Rbardot{} Vísita nos, Oriens ex alto.

\noindent Tu, qui Baptístæ ore et vita viam pæniténtiæ nobis indicásti,~\grestar{} convérte corda nostra ad mandáta regni tui.

\Rbardot{} Vísita nos, Oriens ex alto.

\noindent Tu, qui ore hóminis te prædicári voluísti~\grestar{} mitte in orbem totum Evangélii tui præcónes.

\Rbardot{} Vísita nos, Oriens ex alto.

\noindent Tu, qui in Iordáne a Ioánne baptizári voluísti~\gredagger{} ut omnis iustítia implerétur,~\grestar{} fac nos iustítiæ regni tui adlaboráre.

\Rbardot{} Vísita nos, Oriens ex alto.}
\newcommand{\benedicamuslaudes}{\cuminitiali{}{temporalia/benedicamus-memoria-laudes.gtex}}
\newcommand{\hebdomada}{infra Hebdom. XXI per Annum.}
\newcommand{\matua}{Matutinum Hebdomadae A}
\newcommand{\matuac}{Matutinum Hebdomadae A vel C}
\newcommand{\lauda}{Laudes Hebdomadae A}
\newcommand{\laudac}{Laudes Hebdomadae A vel C}

% LuaLaTeX

\documentclass[a4paper, twoside, 12pt]{article}
\usepackage[latin]{babel}
%\usepackage[landscape, left=3cm, right=1.5cm, top=2cm, bottom=1cm]{geometry} % okraje stranky
%\usepackage[landscape, a4paper, mag=1166, truedimen, left=2cm, right=1.5cm, top=1.6cm, bottom=0.95cm]{geometry} % okraje stranky
\usepackage[landscape, a4paper, mag=1400, truedimen, left=0.5cm, right=0.5cm, top=0.5cm, bottom=0.5cm]{geometry} % okraje stranky

\usepackage{fontspec}
\setmainfont[FeatureFile={junicode.fea}, Ligatures={Common, TeX}, RawFeature=+fixi]{Junicode}
%\setmainfont{Junicode}

% shortcut for Junicode without ligatures (for the Czech texts)
\newfontfamily\nlfont[FeatureFile={junicode.fea}, Ligatures={Common, TeX}, RawFeature=+fixi]{Junicode}

\usepackage{multicol}
\usepackage{color}
\usepackage{lettrine}
\usepackage{fancyhdr}

% usual packages loading:
\usepackage{luatextra}
\usepackage{graphicx} % support the \includegraphics command and options
\usepackage{gregoriotex} % for gregorio score inclusion
\usepackage{gregoriosyms}
\usepackage{wrapfig} % figures wrapped by the text
\usepackage{parcolumns}
\usepackage[contents={},opacity=1,scale=1,color=black]{background}
\usepackage{tikzpagenodes}
\usepackage{calc}
\usepackage{longtable}
\usetikzlibrary{calc}

\setlength{\headheight}{14.5pt}

% Commands used to produce a typical "Conventus" booklet

\newenvironment{titulusOfficii}{\begin{center}}{\end{center}}
\newcommand{\dies}[1]{#1

}
\newcommand{\nomenFesti}[1]{\textbf{\Large #1}

}
\newcommand{\celebratio}[1]{#1

}

\newcommand{\hora}[1]{%
\vspace{0.5cm}{\large \textbf{#1}}

\fancyhead[LE]{\thepage\ / #1}
\fancyhead[RO]{#1 / \thepage}
\addcontentsline{toc}{subsection}{#1}
}

% larger unit than a hora
\newcommand{\divisio}[1]{%
\begin{center}
{\Large \textsc{#1}}
\end{center}
\fancyhead[CO,CE]{#1}
\addcontentsline{toc}{section}{#1}
}

% a part of a hora, larger than pars
\newcommand{\subhora}[1]{
\begin{center}
{\large \textit{#1}}
\end{center}
%\fancyhead[CO,CE]{#1}
\addcontentsline{toc}{subsubsection}{#1}
}

% rubricated inline text
\newcommand{\rubricatum}[1]{\textit{#1}}

% standalone rubric
\newcommand{\rubrica}[1]{\vspace{3mm}\rubricatum{#1}}

\newcommand{\notitia}[1]{\textcolor{red}{#1}}

\newcommand{\scriptura}[1]{\hfill \small\textit{#1}}

\newcommand{\translatioCantus}[1]{\vspace{1mm}%
{\noindent\footnotesize \nlfont{#1}}}

% pruznejsi varianta nasledujiciho - umoznuje nastavit sirku sloupce
% s prekladem
\newcommand{\psalmusEtTranslatioB}[3]{
  \vspace{0.5cm}
  \begin{parcolumns}[colwidths={2=#3}, nofirstindent=true]{2}
    \colchunk{
      \input{#1}
    }

    \colchunk{
      \vspace{-0.5cm}
      {\footnotesize \nlfont
        \input{#2}
      }
    }
  \end{parcolumns}
}

\newcommand{\psalmusEtTranslatio}[2]{
  \psalmusEtTranslatioB{#1}{#2}{8.5cm}
}


\newcommand{\canticumMagnificatEtTranslatio}[1]{
  \psalmusEtTranslatioB{#1}{temporalia/extra-adventum-vespers/magnificat-boh.tex}{12cm}
}
\newcommand{\canticumBenedictusEtTranslatio}[1]{
  \psalmusEtTranslatioB{#1}{temporalia/extra-adventum-laudes/benedictus-boh.tex}{10.5cm}
}

% volne misto nad antifonami, kam si zpevaci dokresli neumy
\newcommand{\hicSuntNeumae}{\vspace{0.5cm}}

% prepinani mista mezi notovymi osnovami: pro neumovane a neneumovane zpevy
\newcommand{\cantusCumNeumis}{
  \setgrefactor{17}
  \global\advance\grespaceabovelines by 5mm%
}
\newcommand{\cantusSineNeumas}{
  \setgrefactor{17}
  \global\advance\grespaceabovelines by -5mm%
}

% znaky k umisteni nad inicialu zpevu
\newcommand{\superInitialam}[1]{\gresetfirstlineaboveinitial{\small {\textbf{#1}}}{\small {\textbf{#1}}}}

% pars officii, i.e. "oratio", ...
\newcommand{\pars}[1]{\textbf{#1}}

\newenvironment{psalmus}{
  \setlength{\parindent}{0pt}
  \setlength{\parskip}{5pt}
}{
  \setlength{\parindent}{10pt}
  \setlength{\parskip}{10pt}
}

%%%% Prejmenovat na latinske:
\newcommand{\nadpisZalmu}[1]{
  \hspace{2cm}\textbf{#1}\vspace{2mm}%
  \nopagebreak%

}

% mode, score, translation
\newcommand{\antiphona}[3]{%
\hicSuntNeumae
\superInitialam{#1}
\includescore{#2}

#3
}
 % Often used macros

\newcommand{\annusEditionis}{2021}

%%%% Vicekrat opakovane kousky

\newcommand{\anteOrationem}{
  \rubrica{Ante Orationem, cantatur a Superiore:}

  \pars{Supplicatio Litaniæ.}

  \cuminitiali{}{temporalia/supplicatiolitaniae.gtex}

  \pars{Oratio Dominica.}

  \cuminitiali{}{temporalia/oratiodominica.gtex}

  \rubrica{Deinde dicitur ab Hebdomadario:}

  \cuminitiali{}{temporalia/dominusvobiscum-solemnis.gtex}

  \rubrica{In choro monialium loco Dominus vobiscum dicitur:}

  \sineinitiali{temporalia/domineexaudi.gtex}
}

\setlength{\columnsep}{30pt} % prostor mezi sloupci

%%%%%%%%%%%%%%%%%%%%%%%%%%%%%%%%%%%%%%%%%%%%%%%%%%%%%%%%%%%%%%%%%%%%%%%%%%%%%%%%%%%%%%%%%%%%%%%%%%%%%%%%%%%%%
\begin{document}

% Here we set the space around the initial.
% Please report to http://home.gna.org/gregorio/gregoriotex/details for more details and options
\grechangedim{afterinitialshift}{2.2mm}{scalable}
\grechangedim{beforeinitialshift}{2.2mm}{scalable}
\grechangedim{interwordspacetext}{0.22 cm plus 0.15 cm minus 0.05 cm}{scalable}%
\grechangedim{annotationraise}{-0.2cm}{scalable}

% Here we set the initial font. Change 38 if you want a bigger initial.
% Emit the initials in red.
\grechangestyle{initial}{\color{red}\fontsize{38}{38}\selectfont}

\pagestyle{empty}

%%%% Titulni stranka
\begin{titulusOfficii}
\ifx\titulus\undefined
\nomenFesti{Feria III \hebdomada{}}
\else
\titulus
\fi
\end{titulusOfficii}

\vfill

\begin{center}
%Ad usum et secundum consuetudines chori \guillemotright{}Conventus Choralis\guillemotleft.

%Editio Sancti Wolfgangi \annusEditionis
\end{center}

\scriptura{}

\pars{}

\pagebreak

\renewcommand{\headrulewidth}{0pt} % no horiz. rule at the header
\fancyhf{}
\pagestyle{fancy}

\cantusSineNeumas

\hora{Ad Matutinum.} %%%%%%%%%%%%%%%%%%%%%%%%%%%%%%%%%%%%%%%%%%%%%%%%%%%%%

\vspace{2mm}

\cuminitiali{}{temporalia/dominelabiamea.gtex}

\vfill
%\pagebreak

\vspace{2mm}

\ifx\invitatorium\undefined
\pars{Invitatorium.} \scriptura{Lc. 24, 34; Psalmus 94; \textbf{H232}}

\vspace{-6mm}

\antiphona{VI}{temporalia/inv-surrexitdominusvere.gtex}
\else
\invitatorium
\fi

\vfill
\pagebreak

\ifx\hymnusmatutinum\undefined
\pars{Hymnus}

\cuminitiali{VIII}{temporalia/hym-LaetareCaelum.gtex}
\else
\hymnusmatutinum
\fi

\vspace{-3mm}

\vfill
\pagebreak

\ifx\matutinum\undefined
\ifx\matua\undefined
\else
% MAT A
\pars{Psalmus 1.}

\vspace{-4mm}

\antiphona{II D}{temporalia/ant-alleluia-turco7.gtex}

%\vspace{-2mm}

\scriptura{Ps. 9, 22-32}

%\vspace{-2mm}

\initiumpsalmi{temporalia/ps9xxii_xxxii-initium-ii-D-auto.gtex}

\input{temporalia/ps9xxii_xxxii-ii-D.tex}

\vfill
\pagebreak

\pars{Psalmus 2.} \scriptura{Ps. 9, 33-39}

%\vspace{-2mm}

\initiumpsalmi{temporalia/ps9xxxiii_xxxix-initium-ii-D-auto.gtex}

\input{temporalia/ps9xxxiii_xxxix-ii-D.tex}

\vfill
\pagebreak

\pars{Psalmus 3.} \scriptura{Ps. 11}

%\vspace{-2mm}

\initiumpsalmi{temporalia/ps11-initium-ii-D-auto.gtex}

\input{temporalia/ps11-ii-D.tex}

\vfill

\antiphona{}{temporalia/ant-alleluia-turco7.gtex}

\vfill
\pagebreak
\fi
\ifx\matub\undefined
\else
% MAT B
\pars{Psalmus 1.}

\vspace{-4mm}

\antiphona{VI F}{temporalia/ant-alleluia-turco6.gtex}

%\vspace{-2mm}

\scriptura{Ps. 36, 1-11}

%\vspace{-2mm}

\initiumpsalmi{temporalia/ps36i_xi-initium-vi-F-auto.gtex}

\input{temporalia/ps36i_xi-vi-F.tex}

\vfill
\pagebreak

\pars{Psalmus 2.}

\scriptura{Ps. 36, 12-29}

\vspace{-2mm}

\initiumpsalmi{temporalia/ps36xii_xxix-initium-vi-F-auto.gtex}

\input{temporalia/ps36xii_xxix-vi-F.tex}

\vfill
\pagebreak

\pars{Psalmus 3.}

\scriptura{Ps. 36, 30-40}

%\vspace{-2mm}

\initiumpsalmi{temporalia/ps36iii-initium-vi-F-auto.gtex}

\input{temporalia/ps36iii-vi-F.tex}

\antiphona{}{temporalia/ant-alleluia-turco6.gtex}

\vfill
\pagebreak
\fi
\ifx\matuc\undefined
\else
% MAT C
\pars{Psalmus 1.}

\vspace{-4mm}

\antiphona{I g\textsuperscript{5}}{temporalia/ant-alleluia-auglx2.gtex}

%\vspace{-2mm}

\scriptura{Ps. 67, 2-11}

\initiumpsalmi{temporalia/ps67i-initium-i-g5.gtex}

\input{temporalia/ps67i-i-g.tex}

\vfill
\pagebreak

\pars{Psalmus 2.}

\scriptura{Ps. 67, 12-24}

%\vspace{-2mm}

\initiumpsalmi{temporalia/ps67ii-initium-i-g5.gtex}

\input{temporalia/ps67ii-i-g.tex}

\vfill
\pagebreak

\pars{Psalmus 3.}

\scriptura{Ps. 67, 25-36}

\initiumpsalmi{temporalia/ps67iii-initium-i-g5.gtex}

\input{temporalia/ps67iii-i-g.tex}

\vfill

\antiphona{}{temporalia/ant-alleluia-auglx2.gtex}

\vfill
\pagebreak
\fi
\ifx\matud\undefined
\else
% MAT D
\pars{Psalmus 1.}

\vspace{-4mm}

\antiphona{I d\textsuperscript{3}}{temporalia/ant-alleluia-auglx6.gtex}

%\vspace{-2mm}

\scriptura{Ps. 101, 2-12}

%\vspace{-2mm}

\initiumpsalmi{temporalia/ps101ii_xii-initium-i-d3-auto.gtex}

\input{temporalia/ps101ii_xii-i-d3.tex}

\vfill
\pagebreak

\pars{Psalmus 2.} \scriptura{Ps. 101, 13-23}

\vspace{-2mm}

\initiumpsalmi{temporalia/ps101xiii_xxiii-initium-i-d3-auto.gtex}

\input{temporalia/ps101xiii_xxiii-i-d3.tex}

\vfill
\pagebreak

\pars{Psalmus 3.} \scriptura{Ps. 101, 24-29}

%\vspace{-2mm}

\initiumpsalmi{temporalia/ps101iii-initium-i-d3-auto.gtex}

\input{temporalia/ps101iii-i-d3.tex}

\vfill

\antiphona{}{temporalia/ant-alleluia-auglx6.gtex}

\vfill
\pagebreak
\fi
\else
\matutinum
\fi

\pars{Versus.}

\ifx\matversus\undefined
\noindent \Vbardot{} Christus resúrgens ex mórtuis iam non móritur, allelúia.

\noindent \Rbardot{} Mors illi ultra non dominábitur, allelúia.
\else
\matversus
\fi

\vspace{5mm}

\sineinitiali{temporalia/oratiodominica-mat.gtex}

\vspace{5mm}

\pars{Absolutio.}

\cuminitiali{}{temporalia/absolutio-ipsius.gtex}

\vfill
\pagebreak

\cuminitiali{}{temporalia/benedictio-solemn-deus.gtex}

\vspace{7mm}

\lectioi

\noindent \Vbardot{} Tu autem, Dómine, miserére nobis.
\noindent \Rbardot{} Deo grátias.

\vfill
\pagebreak

\responsoriumi

\vfill
\pagebreak

\cuminitiali{}{temporalia/benedictio-solemn-christus.gtex}

\vspace{7mm}

\lectioii

\noindent \Vbardot{} Tu autem, Dómine, miserére nobis.
\noindent \Rbardot{} Deo grátias.

\vfill
\pagebreak

\responsoriumii

\vfill
\pagebreak

\cuminitiali{}{temporalia/benedictio-solemn-ignem.gtex}

\vspace{7mm}

\lectioiii

\noindent \Vbardot{} Tu autem, Dómine, miserére nobis.
\noindent \Rbardot{} Deo grátias.

\vfill
\pagebreak

\responsoriumiii

\vfill
\pagebreak

\rubrica{Reliqua omittuntur, nisi Laudes separandæ sint.}

\sineinitiali{temporalia/domineexaudi.gtex}

\vfill

\oratio

\vfill

\noindent \Vbardot{} Dómine, exáudi oratiónem meam.
\Rbardot{} Et clamor meus ad te véniat.

\vfill

\noindent \Vbardot{} Benedicámus Dómino.
\noindent \Rbardot{} Deo grátias.

\vfill

\noindent \Vbardot{} Fidélium ánimæ per misericórdiam Dei requiéscant in pace.
\Rbardot{} Amen.

\vfill
\pagebreak

\hora{Ad Laudes.} %%%%%%%%%%%%%%%%%%%%%%%%%%%%%%%%%%%%%%%%%%%%%%%%%%%%%

\cantusSineNeumas

\vspace{0.5cm}
\grechangedim{interwordspacetext}{0.18 cm plus 0.15 cm minus 0.05 cm}{scalable}%
\cuminitiali{}{temporalia/deusinadiutorium-communis.gtex}
\grechangedim{interwordspacetext}{0.22 cm plus 0.15 cm minus 0.05 cm}{scalable}%

\vfill
\pagebreak

\ifx\hymnuslaudes\undefined
\ifx\laudac\undefined
\else
\pars{Hymnus}

\cuminitiali{I}{temporalia/hym-ChorusNovae-praglia.gtex}
\fi
\ifx\laudbd\undefined
\else
\pars{Hymnus}

\cuminitiali{I}{temporalia/hym-ChorusNovae.gtex}
\fi
\else
\hymnuslaudes
\fi

\vfill
\pagebreak

\ifx\laudes\undefined
\ifx\lauda\undefined
\else
\pars{Psalmus 1.}

\vspace{-4mm}

\antiphona{IV* e}{temporalia/ant-alleluia-turco9.gtex}

\scriptura{Psalmus 23.}

\initiumpsalmi{temporalia/ps23-initium-iv_-e-auto.gtex}

\input{temporalia/ps23-iv_-e.tex} \Abardot{}

\vfill
\pagebreak

\pars{Psalmus 2.} \scriptura{Tob. 13, 10}

\vspace{-4mm}

\antiphona{VIII G}{temporalia/ant-benedicitedominumomneselecti.gtex}

\scriptura{Canticum Tobiæ, Tob. 13, 2-8}

\initiumpsalmi{temporalia/tobiae-initium-viii-g-auto.gtex}

\input{temporalia/tobiae-viii-g.tex} \Abardot{}

\vfill
\pagebreak

\pars{Psalmus 3.}

\vspace{-4mm}

\antiphona{E}{temporalia/ant-alleluia-praglia-e2.gtex}

%\vspace{-4mm}

\scriptura{Psalmus 32.}

%\vspace{-2mm}

\initiumpsalmi{temporalia/ps32-initium-e-auto.gtex}

\input{temporalia/ps32-e.tex}

\vfill

\antiphona{}{temporalia/ant-alleluia-praglia-e2.gtex}

\vfill
\pagebreak
\fi
\ifx\laudb\undefined
\else
\pars{Psalmus 1.}

\vspace{-4mm}

\antiphona{E}{temporalia/ant-alleluia-praglia-e.gtex}

\scriptura{Psalmus 42.}

\initiumpsalmi{temporalia/ps42-initium-e-e-auto.gtex}

\input{temporalia/ps42-e-e.tex} \Abardot{}

\vfill
\pagebreak

\pars{Psalmus 2.} \scriptura{Is. 38, 17}

\vspace{-4mm}

\antiphona{I g}{temporalia/ant-eruistidomine-tp.gtex}

%\vspace{-2mm}

\scriptura{Canticum Ezechiæ, Is. 38, 10-20}

%\vspace{-2mm}

\initiumpsalmi{temporalia/ezechiae-initium-i-g-auto.gtex}

%\vspace{-1.5mm}

\input{temporalia/ezechiae-i-g.tex}

\vfill

\antiphona{}{temporalia/ant-eruistidomine-tp.gtex}

\vfill
\pagebreak

\pars{Psalmus 3.}

\vspace{-4mm}

\antiphona{VIII c}{temporalia/ant-alleluia-turco16.gtex}

\vspace{-2mm}

\scriptura{Psalmus 64.}

\vspace{-2mm}

\initiumpsalmi{temporalia/ps64-initium-viii-C-auto.gtex}

\input{temporalia/ps64-viii-C.tex} \Abardot{}

\vfill
\pagebreak
\fi
\ifx\laudc\undefined
\else
\pars{Psalmus 1.}

\vspace{-4mm}

\antiphona{VI F}{temporalia/ant-alleluia-turco5.gtex}

\vspace{-2mm}

\scriptura{Psalmus 84.}

\vspace{-2mm}

\initiumpsalmi{temporalia/ps84-initium-vi-F-auto.gtex}

\input{temporalia/ps84-vi-F.tex} \Abardot{}

\vfill
\pagebreak

\pars{Psalmus 2.}

\vspace{-4mm}

\antiphona{VII d}{temporalia/ant-denoctespiritusmeus-tp.gtex}

\vspace{-2mm}

\scriptura{Canticum Isaiæ, Is. 26, 1-12}

\vspace{-2mm}

\initiumpsalmi{temporalia/isaiae3-initium-vii-d.gtex}

\input{temporalia/isaiae3-vii-d.tex} \Abardot{}

\vfill
\pagebreak

\pars{Psalmus 3.}

\vspace{-4mm}

\antiphona{E}{temporalia/ant-alleluia-praglia-e2.gtex}

%\vspace{-2mm}

\scriptura{Psalmus 66.}

%\vspace{-2mm}

\initiumpsalmi{temporalia/ps66-initium-e-auto.gtex}

\input{temporalia/ps66-e.tex} \Abardot{}

\vfill
\pagebreak
\fi
\ifx\laudd\undefined
\else
\pars{Psalmus 1.}

\vspace{-4mm}

\antiphona{VIII G}{temporalia/ant-alleluia-turco12.gtex}

\vspace{-2mm}

\scriptura{Psalmus 100.}

\vspace{-2mm}

\initiumpsalmi{temporalia/ps100-initium-viii-G-auto.gtex}

\input{temporalia/ps100-viii-G.tex} \Abardot{}

\vfill
\pagebreak

\pars{Psalmus 2.} \scriptura{Ps. 50, 19}

\vspace{-4mm}

\antiphona{I f}{temporalia/ant-sacrificiumdeo-tp.gtex}

%\vspace{-2mm}

\scriptura{Canticum Danielis, Dan. 3, 26.27.29.34-41}

%\vspace{-2mm}

\initiumpsalmi{temporalia/dan32-initium-i-f-auto.gtex}

\input{temporalia/dan32-i-f.tex} \Abardot{}

\vfill
\pagebreak

\pars{Psalmus 3.}

\vspace{-4mm}

\antiphona{VI F}{temporalia/ant-alleluia-turco5.gtex}

%\vspace{-2mm}

\scriptura{Psalmus 143, 1-10.}

%\vspace{-2mm}

\initiumpsalmi{temporalia/ps143i_x-initium-vi-F-auto.gtex}

\input{temporalia/ps143i_x-vi-F.tex} \Abardot{}

\vfill
\pagebreak
\fi
\else
\laudes
\fi

\ifx\lectiobrevis\undefined
\pars{Lectio Brevis.} \scriptura{Ac. 13, 30-33}

\noindent Deus suscitávit Iesum a mórtuis; qui visus est per dies multos his, qui simul ascénderant cum eo de Galilǽa in Ierúsalem, qui nunc sunt testes eius ad plebem. Et nos vobis evangelizámus eam, quæ ad patres promíssio facta est, quóniam hanc Deus adimplévit fíliis eórum, nobis resúscitans Iesum, sicut et in Psalmo secúndo scriptum est: Fílius meus es tu; ego hódie génui te.
\else
\lectiobrevis
\fi

\vfill

\ifx\responsoriumbreve\undefined
\pars{Responsorium breve.} \scriptura{Cf. Mt. 28, 6; Cf. Gal. 3, 13}

\cuminitiali{VI}{temporalia/resp-surrexitdominusdesepulcro.gtex}
\else
\responsoriumbreve
\fi

\vfill
\pagebreak

\benedictus

\vspace{-1cm}

\vfill
\pagebreak

\ifx\precestotum\undefined
\pars{Preces.}

\sineinitiali{}{temporalia/tonusprecum.gtex}

\ifx\preces\undefined
\ifx\lauda\undefined
\else
\noindent Exsultémus Christo, qui perémptum sui córporis templum sua virtúte restítuit,~\gredagger{} eíque supplicémus:

\Rbardot{} Fructus resurrectiónis tuæ, Dómine, nobis concéde.

\noindent Christe salvátor, qui in resurrectióne tua muliéribus et Apóstolis gáudium nuntiásti, totum orbem salvíficans,~\gredagger{} testes tuos nos éffice.

\Rbardot{} Fructus resurrectiónis tuæ, Dómine, nobis concéde.

\noindent Qui resurrectiónem ómnibus promisísti, qua ad vitam novam resurgerémus,~\gredagger{} Evangélii tui nos redde præcónes.

\Rbardot{} Fructus resurrectiónis tuæ, Dómine, nobis concéde.

\noindent Tu, qui Apóstolis sǽpius apparuísti et Sanctum eis Spíritum insufflásti,~\gredagger{} creatórem Spíritum rénova in nobis.

\Rbardot{} Fructus resurrectiónis tuæ, Dómine, nobis concéde.

\noindent Tu, qui discípulis tuis promisísti te cum eis mansúrum usque ad consummatiónem sǽculi,~\gredagger{} mane nobíscum hódie sempérque nobis adésto.

\Rbardot{} Fructus resurrectiónis tuæ, Dómine, nobis concéde.
\fi
\ifx\laudb\undefined
\else
\noindent Deum Patrem, cuius Agnus immaculátus tollit peccáta mundi nosque vivíficat,~\gredagger{} grati rogémus:

\Rbardot{} Auctor vitæ, vivífica nos.

\noindent Deus, auctor vitæ, meménto passiónis et resurrectiónis Agni, in cruce occísi,~\gredagger{} eúmque audi, semper interpellántem pro nobis.

\Rbardot{} Auctor vitæ, vivífica nos.

\noindent Expurgáto vétere ferménto malítiæ et nequítiæ,~\gredagger{} fac nos vívere in ázymis sinceritátis et veritátis Christi.

\Rbardot{} Auctor vitæ, vivífica nos.

\noindent Da, ut hódie reiciámus peccátum discórdiæ atque invídiæ,~\gredagger{} nosque redde fratrum necessitátibus magis inténtos.

\Rbardot{} Auctor vitæ, vivífica nos.

\noindent Spíritum evangélicum pone in médio nostri,~\gredagger{} ut hódie et semper in præcéptis tuis ambulémus.

\Rbardot{} Auctor vitæ, vivífica nos.
\fi
\ifx\laudc\undefined
\else
\noindent Exsultémus Christo, qui perémptum sui córporis templum sua virtúte restítuit,~\gredagger{} eíque supplicémus:

\Rbardot{} Fructus resurrectiónis tuæ, Dómine, nobis concéde.

\noindent Christe salvátor, qui in resurrectióne tua muliéribus et Apóstolis gáudium nuntiásti, totum orbem salvíficans,~\gredagger{} testes tuos nos éffice.

\Rbardot{} Fructus resurrectiónis tuæ, Dómine, nobis concéde.

\noindent Qui resurrectiónem ómnibus promisísti, qua ad vitam novam resurgerémus,~\gredagger{} Evangélii tui nos redde præcónes.

\Rbardot{} Fructus resurrectiónis tuæ, Dómine, nobis concéde.

\noindent Tu, qui Apóstolis sǽpius apparuísti et Sanctum eis Spíritum insufflásti,~\gredagger{} creatórem Spíritum rénova in nobis.

\Rbardot{} Fructus resurrectiónis tuæ, Dómine, nobis concéde.

\noindent Tu, qui discípulis tuis promisísti te cum eis mansúrum usque ad consummatiónem sǽculi,~\gredagger{} mane nobíscum hódie sempérque nobis adésto.

\Rbardot{} Fructus resurrectiónis tuæ, Dómine, nobis concéde.
\fi
\ifx\laudd\undefined
\else
\noindent Deum Patrem, cuius Agnus immaculátus tollit peccáta mundi nosque vivíficat,~\gredagger{} grati rogémus:

\Rbardot{} Auctor vitæ, vivífica nos.

\noindent Deus, auctor vitæ, meménto passiónis et resurrectiónis Agni, in cruce occísi,~\gredagger{} eúmque audi, semper interpellántem pro nobis.

\Rbardot{} Auctor vitæ, vivífica nos.

\noindent Expurgáto vétere ferménto malítiæ et nequítiæ,~\gredagger{} fac nos vívere in ázymis sinceritátis et veritátis Christi.

\Rbardot{} Auctor vitæ, vivífica nos.

\noindent Da, ut hódie reiciámus peccátum discórdiæ atque invídiæ,~\gredagger{} nosque redde fratrum necessitátibus magis inténtos.

\Rbardot{} Auctor vitæ, vivífica nos.

\noindent Spíritum evangélicum pone in médio nostri,~\gredagger{} ut hódie et semper in præcéptis tuis ambulémus.

\Rbardot{} Auctor vitæ, vivífica nos.
\fi
\else
\preces
\fi

\vfill

\pars{Oratio Dominica.}

\cuminitiali{}{temporalia/oratiodominicaalt.gtex}

\vfill
\pagebreak

\rubrica{vel:}

\pars{Supplicatio Litaniæ.}

\cuminitiali{}{temporalia/supplicatiolitaniae.gtex}

\vfill

\pars{Oratio Dominica.}

\cuminitiali{}{temporalia/oratiodominica.gtex}
\else
\precestotum
\fi

\vfill
\pagebreak

% Oratio. %%%
\oratio

\vspace{-1mm}

\vfill

\rubrica{Hebdomadarius dicit Dominus vobiscum, vel, absente sacerdote vel diacono, sic concluditur:}

\vspace{2mm}

\ifx\dominusnosbenedicat\undefined
\antiphona{C}{temporalia/dominusnosbenedicat.gtex}
\else
\dominusnosbenedicat
\fi

\rubrica{Postea cantatur a cantore:}

\vspace{2mm}

\ifx\benedicamuslaudes\undefined
\cuminitiali{VII}{temporalia/benedicamus-tempore-paschali.gtex}
\else
\benedicamuslaudes
\fi

\vspace{1mm}

\vfill
\pagebreak

\end{document}

