\newcommand{\lectioi}{\pars{Lectio I.} \scriptura{Is. 30, 1-11}

\noindent De libro Isaíæ prophétæ.

\noindent «Væ, fílii desertóres, dicit Dóminus, eo quod fácitis consílium et non ex me, et pactum statúitis et non per spíritum meum, ut addátis peccátum super peccátum! Qui ambulátis, ut descendátis in Ægýptum, et os meum non interrogástis, sperántes auxílium in fortitúdine pharaónis et habéntes fidúciam in umbra Ægýpti. Et erit vobis fortitúdo pharaónis in confusiónem, et fidúcia sub umbra Ægýpti in ignomíniam. Cum fúerint enim in Tani príncipes tui, et núntii tui usque ad Hanes pervénerint, omnes confundéntur super pópulo, qui eis prodésse non potest; non erit in auxílium et in utilitátem, sed in confusiónem et oppróbrium».

\noindent Oráculum iumentórum Nageb. In terra tribulatiónis et angústiæ, leǽnæ et leónis rugiéntis, víperæ et dracónis volántis portant super úmeros iumentórum divítias suas, et super gibbum camelórum thesáuros suos ad pópulum, qui eis prodésse non póterit. Ægýptus enim frustra et vane auxiliábitur; ídeo vocávi Rahab otiósam. Nunc ingrédere, scribe coram eis super buxum et in libro diligénter éxara illud, et erit in pósterum in testimónium usque in ætérnum. Pópulus enim rebéllis est, et fílii mendáces, fílii noléntes audíre legem Dómini; qui dicunt vidéntibus: «Nolíte vidére» et aspiciéntibus: «Nolíte aspícere nobis ea, quæ recta sunt; loquímini nobis placéntia, aspícite nobis illusiónes. Recédite a via, declináte a sémita, tóllite a fácie nostra Sanctum Israel».}
\newcommand{\responsoriumi}{\pars{Responsorium 1.} \scriptura{\Rbardot{} Ier. 4, 26.24.27 \Vbardot{} Ps. 8, 2; \textbf{H418}}

\vspace{-5mm}

\responsorium{II}{temporalia/resp-afaciefuroristui-CROCHU.gtex}{}}
\newcommand{\lectioii}{\pars{Lectio II.} \scriptura{Is. 30, 12-18}

\noindent Proptérea hæc dicit Sanctus Israel: «Pro eo quod reprobástis verbum hoc et sperástis in perversitátem et in perfídiam et inníxi estis super eis, proptérea erit vobis iníquitas hæc sicut interrúptio cadens, locus tumens in muro excélso, cuius confráctio súbito, dum non sperátur, venit improvíso; et comminuétur, sicut contéritur lagœ́na fíguli contritióne absque misericórdia, et non inveniétur de fragméntis eius testa, in qua capiátur ignículus de incéndio, aut hauriátur aqua de fóvea». Quia hæc dixit Dóminus Deus, Sanctus Israel: «In conversióne et quiéte salvi éritis; in siléntio et in spe erit fortitúdo vestra». Et noluístis et dixístis: «Nequáquam, sed super equis fugiémus», ídeo fugiétis; et: «Super velóces ascendémus», ídeo velóces erunt, qui persequéntur vos. Mille pavébunt a fácie terróris uníus, et a fácie terróris quinque fugiétis, donec relinquámini quasi malus in vértice montis et quasi signum super collem. Proptérea exspéctat Dóminus, ut misereátur vestri, et ídeo exaltábitur parcens vobis, quia Deus iudícii Dóminus; beáti omnes, qui exspéctant eum.}
\newcommand{\responsoriumii}{\pars{Responsorium 2.} \scriptura{\Rbar{} Cf. Is. 35, 2 \Vbar{} Is. 40, 10; \textbf{H35}}

\vspace{-5mm}

\responsorium{I}{temporalia/resp-germinaveruntcampi-CROCHU.gtex}{}}
\newcommand{\lectioiii}{\pars{Lectio III.} \scriptura{Gap. 15 : PL 44, 899-900}

\noindent Ex Libro sancti Augustíni epíscopi De grátia et líbero arbítrio.

\noindent Ne putétur nihil fácere ipsos hómines per líberum arbítrium, ídeo in psalmo dícitur: \emph{Nolíte obduráre corda vestra.} Et per Ezechiélem: \emph{Proícite a vobis omnes impietátes vestras, quas ímpie egístis in me, et fácite vobis cor novum et spíritum novum, et fácite ómnia mandáta mea. Et convertímini et vivétis.} Meminérimus eum dícere: \emph{Et convertímini et vivétis;} cui dícitur: \emph{Convérte nos, Deus.} Meminérimus eum dícere: \emph{Proícite a vobis impietátes vestras;} cum ipse \emph{iustíficet ímpium.} Meminérimus ipsum dícere: \emph{Fácite vobis cor novum et spíritum novum; qui dicit: Dabo vobis cor novum, et spíritum novum dabo vobis.}

\noindent Quómodo ergo qui dicit: \emph{Fácite vobis;} hoc dicit: \emph{Dabo vobis?} Quare iubet, si ipse datúrus est? Quare dat, si homo factúrus est; nisi quia dat quod iubet, cum ádiuvat ut fáciat cui iubet? Semper est autem in nobis volúntas líbera, sed non semper est bona. Aut enim a iustítia líbera est, quando servit peccáto, et tunc est mala; aut a peccáto líbera est, quando servit iustítiæ, et tunc est bona. Grátia vero Dei semper est bona, et per hanc fit ut sit homo bonæ voluntátis, qui prius fuit voluntátis malæ. Per hanc étiam fit ut ipsa bona volúntas, quæ iam esse cœpit, augeátur, et tam magna fiat, ut possit implére divína mandáta quæ volúerit, cum valde perfectéque volúerit.

\noindent Ad hoc enim valet quod scriptum est: \emph{Si volúeris, servábis mandáta;} ut homo qui volúerit et non potúerit, nondum se plene velle cognóscat, et oret ut hábeat tantam voluntátem, quanta súfficit ad implénda mandáta. Sic quippe adiuvátur, ut fáciat quod iubétur. Tunc enim útile est velle, cum póssumus; et tunc útile est posse, cum vólumus; nam quid prodest, si quod non póssumus vólumus, aut quod póssumus nólumus?}
\newcommand{\responsoriumiii}{\pars{Responsorium 3.} \scriptura{\Rbardot{} Ier. 14, 19.20 \Vbardot{} Bar. 2, 12; \textbf{H417}}

\vspace{-5mm}

\responsorium{VIII}{temporalia/resp-sustinuimuspacemetnonvenit-CROCHU-cumdox.gtex}{}}
\newcommand{\hebdomada}{infra Hebdom. XXX per Annum.}
\newcommand{\hiemalis}{Hiemalis}
\newcommand{\matub}{Matutinum Hebdomadae B}
\newcommand{\matubd}{Matutinum Hebdomadae B vel D}
\newcommand{\laudb}{Laudes Hebdomadae B}
\newcommand{\laudbd}{Laudes Hebdomadae B vel D}

% LuaLaTeX

\documentclass[a4paper, twoside, 12pt]{article}
\usepackage[latin]{babel}
%\usepackage[landscape, left=3cm, right=1.5cm, top=2cm, bottom=1cm]{geometry} % okraje stranky
%\usepackage[landscape, a4paper, mag=1166, truedimen, left=2cm, right=1.5cm, top=1.6cm, bottom=0.95cm]{geometry} % okraje stranky
\usepackage[landscape, a4paper, mag=1400, truedimen, left=0.5cm, right=0.5cm, top=0.5cm, bottom=0.5cm]{geometry} % okraje stranky

\usepackage{fontspec}
\setmainfont[FeatureFile={junicode.fea}, Ligatures={Common, TeX}, RawFeature=+fixi]{Junicode}
%\setmainfont{Junicode}

% shortcut for Junicode without ligatures (for the Czech texts)
\newfontfamily\nlfont[FeatureFile={junicode.fea}, Ligatures={Common, TeX}, RawFeature=+fixi]{Junicode}

\usepackage{multicol}
\usepackage{color}
\usepackage{lettrine}
\usepackage{fancyhdr}

% usual packages loading:
\usepackage{luatextra}
\usepackage{graphicx} % support the \includegraphics command and options
\usepackage{gregoriotex} % for gregorio score inclusion
\usepackage{gregoriosyms}
\usepackage{wrapfig} % figures wrapped by the text
\usepackage{parcolumns}
\usepackage[contents={},opacity=1,scale=1,color=black]{background}
\usepackage{tikzpagenodes}
\usepackage{calc}
\usepackage{longtable}
\usetikzlibrary{calc}

\setlength{\headheight}{14.5pt}

% Commands used to produce a typical "Conventus" booklet

\newenvironment{titulusOfficii}{\begin{center}}{\end{center}}
\newcommand{\dies}[1]{#1

}
\newcommand{\nomenFesti}[1]{\textbf{\Large #1}

}
\newcommand{\celebratio}[1]{#1

}

\newcommand{\hora}[1]{%
\vspace{0.5cm}{\large \textbf{#1}}

\fancyhead[LE]{\thepage\ / #1}
\fancyhead[RO]{#1 / \thepage}
\addcontentsline{toc}{subsection}{#1}
}

% larger unit than a hora
\newcommand{\divisio}[1]{%
\begin{center}
{\Large \textsc{#1}}
\end{center}
\fancyhead[CO,CE]{#1}
\addcontentsline{toc}{section}{#1}
}

% a part of a hora, larger than pars
\newcommand{\subhora}[1]{
\begin{center}
{\large \textit{#1}}
\end{center}
%\fancyhead[CO,CE]{#1}
\addcontentsline{toc}{subsubsection}{#1}
}

% rubricated inline text
\newcommand{\rubricatum}[1]{\textit{#1}}

% standalone rubric
\newcommand{\rubrica}[1]{\vspace{3mm}\rubricatum{#1}}

\newcommand{\notitia}[1]{\textcolor{red}{#1}}

\newcommand{\scriptura}[1]{\hfill \small\textit{#1}}

\newcommand{\translatioCantus}[1]{\vspace{1mm}%
{\noindent\footnotesize \nlfont{#1}}}

% pruznejsi varianta nasledujiciho - umoznuje nastavit sirku sloupce
% s prekladem
\newcommand{\psalmusEtTranslatioB}[3]{
  \vspace{0.5cm}
  \begin{parcolumns}[colwidths={2=#3}, nofirstindent=true]{2}
    \colchunk{
      \input{#1}
    }

    \colchunk{
      \vspace{-0.5cm}
      {\footnotesize \nlfont
        \input{#2}
      }
    }
  \end{parcolumns}
}

\newcommand{\psalmusEtTranslatio}[2]{
  \psalmusEtTranslatioB{#1}{#2}{8.5cm}
}


\newcommand{\canticumMagnificatEtTranslatio}[1]{
  \psalmusEtTranslatioB{#1}{temporalia/extra-adventum-vespers/magnificat-boh.tex}{12cm}
}
\newcommand{\canticumBenedictusEtTranslatio}[1]{
  \psalmusEtTranslatioB{#1}{temporalia/extra-adventum-laudes/benedictus-boh.tex}{10.5cm}
}

% volne misto nad antifonami, kam si zpevaci dokresli neumy
\newcommand{\hicSuntNeumae}{\vspace{0.5cm}}

% prepinani mista mezi notovymi osnovami: pro neumovane a neneumovane zpevy
\newcommand{\cantusCumNeumis}{
  \setgrefactor{17}
  \global\advance\grespaceabovelines by 5mm%
}
\newcommand{\cantusSineNeumas}{
  \setgrefactor{17}
  \global\advance\grespaceabovelines by -5mm%
}

% znaky k umisteni nad inicialu zpevu
\newcommand{\superInitialam}[1]{\gresetfirstlineaboveinitial{\small {\textbf{#1}}}{\small {\textbf{#1}}}}

% pars officii, i.e. "oratio", ...
\newcommand{\pars}[1]{\textbf{#1}}

\newenvironment{psalmus}{
  \setlength{\parindent}{0pt}
  \setlength{\parskip}{5pt}
}{
  \setlength{\parindent}{10pt}
  \setlength{\parskip}{10pt}
}

%%%% Prejmenovat na latinske:
\newcommand{\nadpisZalmu}[1]{
  \hspace{2cm}\textbf{#1}\vspace{2mm}%
  \nopagebreak%

}

% mode, score, translation
\newcommand{\antiphona}[3]{%
\hicSuntNeumae
\superInitialam{#1}
\includescore{#2}

#3
}
 % Often used macros

\newcommand{\annusEditionis}{2021}

%%%% Vicekrat opakovane kousky

\newcommand{\anteOrationem}{
  \rubrica{Ante Orationem, cantatur a Superiore:}

  \pars{Supplicatio Litaniæ.}

  \cuminitiali{}{temporalia/supplicatiolitaniae.gtex}

  \pars{Oratio Dominica.}

  \cuminitiali{}{temporalia/oratiodominica.gtex}

  \rubrica{Deinde dicitur ab Hebdomadario:}

  \cuminitiali{}{temporalia/dominusvobiscum-solemnis.gtex}

  \rubrica{In choro monialium loco Dominus vobiscum dicitur:}

  \sineinitiali{temporalia/domineexaudi.gtex}
}

\setlength{\columnsep}{30pt} % prostor mezi sloupci

%%%%%%%%%%%%%%%%%%%%%%%%%%%%%%%%%%%%%%%%%%%%%%%%%%%%%%%%%%%%%%%%%%%%%%%%%%%%%%%%%%%%%%%%%%%%%%%%%%%%%%%%%%%%%
\begin{document}

% Here we set the space around the initial.
% Please report to http://home.gna.org/gregorio/gregoriotex/details for more details and options
\grechangedim{afterinitialshift}{2.2mm}{scalable}
\grechangedim{beforeinitialshift}{2.2mm}{scalable}
\grechangedim{interwordspacetext}{0.22 cm plus 0.15 cm minus 0.05 cm}{scalable}%
\grechangedim{annotationraise}{-0.2cm}{scalable}

% Here we set the initial font. Change 38 if you want a bigger initial.
% Emit the initials in red.
\grechangestyle{initial}{\color{red}\fontsize{38}{38}\selectfont}

\pagestyle{empty}

%%%% Titulni stranka
\begin{titulusOfficii}
\ifx\titulus\undefined
\nomenFesti{Feria VI \hebdomada{}}
\else
\titulus
\fi
\end{titulusOfficii}

\vfill

\begin{center}
%Ad usum et secundum consuetudines chori \guillemotright{}Conventus Choralis\guillemotleft.

%Editio Sancti Wolfgangi \annusEditionis
\end{center}

\scriptura{}

\pars{}

\pagebreak

\renewcommand{\headrulewidth}{0pt} % no horiz. rule at the header
\fancyhf{}
\pagestyle{fancy}

\cantusSineNeumas

\ifx\oratio\undefined
\ifx\lauda\undefined
\else
\newcommand{\oratio}{\pars{Oratio.}

\noindent Deus, qui ténebras ignorántiæ Verbi tui luce depéllis, auge in córdibus nostris virtútem fídei quam dedísti, ut ignis, quem grátia tua fecit accéndi, nullis tentatiónibus exstinguátur.

\noindent Per Dóminum nostrum Iesum Christum, Fílium tuum, qui tecum vivit et regnat in unitáte Spíritus Sancti, Deus, per ómnia sǽcula sæculórum.

\noindent \Rbardot{} Amen.}
\fi
\ifx\laudb\undefined
\else
\newcommand{\oratio}{\pars{Oratio.}

\noindent Præsta, quǽsumus, omnípotens Deus, ut laudes, quas nunc tibi persólvimus, in ætérnum cum sanctis tuis ubérius decantáre valeámus.

\noindent Per Dóminum nostrum Iesum Christum, Fílium tuum, qui tecum vivit et regnat in unitáte Spíritus Sancti, Deus, per ómnia sǽcula sæculórum.

\noindent \Rbardot{} Amen.}
\fi
\ifx\laudc\undefined
\else
\newcommand{\oratio}{\pars{Oratio.}

\noindent Illábere sénsibus nostris, omnípotens Pater, ut, in præceptórum tuórum lúmine gradiéntes, te ducem semper sequámur et príncipem.

\noindent Per Dóminum nostrum Iesum Christum, Fílium tuum, qui tecum vivit et regnat in unitáte Spíritus Sancti, Deus, per ómnia sǽcula sæculórum.

\noindent \Rbardot{} Amen.}
\fi
\fi

\hora{Ad Matutinum.} %%%%%%%%%%%%%%%%%%%%%%%%%%%%%%%%%%%%%%%%%%%%%%%%%%%%%

\vspace{2mm}

\cuminitiali{}{temporalia/dominelabiamea.gtex}

\vfill
%\pagebreak

\vspace{2mm}

\ifx\invitatorium\undefined
\pars{Invitatorium.} \scriptura{Ps. 94, 6.7; Psalmus 94}

\antiphona{E}{temporalia/inv-dominumdeum.gtex}
\else
\invitatorium
\fi

\vfill
\pagebreak

\ifx\hymnusmatutinum\undefined
\ifx\matuac\undefined
\else
\pars{Hymnus.} \scriptura{Gregorius Magnus (+604)}

{
\grechangedim{interwordspacetext}{0.10 cm plus 0.15 cm minus 0.05 cm}{scalable}%
\antiphona{IV}{temporalia/hym-TuTrinitatis.gtex}
\grechangedim{interwordspacetext}{0.22 cm plus 0.15 cm minus 0.05 cm}{scalable}%
}
\fi
\ifx\matubd\undefined
\else
\pars{Hymnus.}

{
\grechangedim{interwordspacetext}{0.10 cm plus 0.15 cm minus 0.05 cm}{scalable}%
\antiphona{II}{temporalia/hym-GalliCantu.gtex}
\grechangedim{interwordspacetext}{0.22 cm plus 0.15 cm minus 0.05 cm}{scalable}%
}
\fi
\else
\hymnusmatutinum
\fi

\vspace{-3mm}

\vfill
\pagebreak

\ifx\matua\undefined
\else
% MAT A
\pars{Psalmus 1.} \scriptura{Ps. 34, 1; \textbf{H93}}

\vspace{-4mm}

\antiphona{IV g}{temporalia/ant-expugnaimpugnantes.gtex}

%\vspace{-2mm}

\scriptura{Ps. 34, 1-10}

%\vspace{-2mm}

\initiumpsalmi{temporalia/ps34i-initium-iv-g-auto.gtex}

\input{temporalia/ps34i-iv-g.tex} \Abardot{}

\vfill
\pagebreak

\pars{Psalmus 2.} \scriptura{Ps. 118, 154; \textbf{H174}}

\vspace{-4mm}

\antiphona{VIII G}{temporalia/ant-iudicacausam.gtex}

%\vspace{-2mm}

\scriptura{Ps. 34, 11-17}

%\vspace{-2mm}

\initiumpsalmi{temporalia/ps34ii-initium-viii-G-auto.gtex}

\input{temporalia/ps34ii-viii-G.tex} \Abardot{}

\vfill
\pagebreak

\pars{Psalmus 3.} \scriptura{Ps. 50, 16; \textbf{H177}}

\vspace{-4mm}

\antiphona{VIII G}{temporalia/ant-liberame.gtex}

%\vspace{-2mm}

\scriptura{Ps. 34, 18-28}

\vspace{-2mm}

\initiumpsalmi{temporalia/ps34iii-initium-viii-G-auto.gtex}

\input{temporalia/ps34iii-viii-G.tex} \Abardot{}

\vfill
\pagebreak
\fi
\ifx\matub\undefined
\else
% MAT B
\pars{Psalmus 1.} \scriptura{Ps. 37, 2}

\vspace{-4mm}

\antiphona{VIII c}{temporalia/ant-neiniratua.gtex}

%\vspace{-2mm}

\scriptura{Ps. 37, 2-5}

%\vspace{-2mm}

\initiumpsalmi{temporalia/ps37ii_v-initium-viii-C-auto.gtex}

\input{temporalia/ps37ii_v-viii-C.tex} \Abardot{}

\vfill
\pagebreak

\pars{Psalmus 2.} \scriptura{Ps. 34, 4; \textbf{H218}}

\vspace{-4mm}

\antiphona{II* a}{temporalia/ant-confundantur.gtex}

%\vspace{-2mm}

\scriptura{Ps. 37, 6-13}

%\vspace{-2mm}

\initiumpsalmi{temporalia/ps37vi_xiii-initium-ii_-a-auto.gtex}

\input{temporalia/ps37vi_xiii-ii_-a.tex} \Abardot{}

\vfill
\pagebreak

\pars{Psalmus 3.} \scriptura{Ps. 139, 9.8}

\vspace{-4mm}

\antiphona{II A}{temporalia/ant-nederelinquasme.gtex}

%\vspace{-2mm}

\scriptura{Ps. 37, 14-23}

%\vspace{-2mm}

\initiumpsalmi{temporalia/ps37xiv_xxiii-initium-ii-A-auto.gtex}

\input{temporalia/ps37xiv_xxiii-ii-A.tex} \Abardot{}

\vfill
\pagebreak
\fi
\ifx\matuc\undefined
\else
% MAT C
\pars{Psalmus 1.} \scriptura{Ps. 68, 10; \textbf{H178}}

\vspace{-4mm}

\antiphona{VIII c}{temporalia/ant-zelusdomus.gtex}

%\vspace{-3mm}

\scriptura{Ps. 68, 2-13}

%\vspace{-2mm}

\initiumpsalmi{temporalia/ps68ii_xiii-initium-viii-c-auto.gtex}

%\vspace{-1.5mm}

\input{temporalia/ps68ii_xiii-viii-c.tex}

\vfill

\antiphona{}{temporalia/ant-zelusdomus.gtex}

\vfill
\pagebreak

\pars{Psalmus 2.}

\vspace{-4mm}

\antiphona{VIII c}{temporalia/ant-consolantemme.gtex}

%\vspace{-2mm}

\scriptura{Ps. 68, 14-22}

%\vspace{-2mm}

\initiumpsalmi{temporalia/ps68xiv_xxii-initium-viii-c-auto.gtex}

\input{temporalia/ps68xiv_xxii-viii-c.tex} \Abardot{}

\vfill
\pagebreak

\pars{Psalmus 3.} \scriptura{Ps. 68, 33; \textbf{H96}}

\vspace{-4mm}

\antiphona{VIII G}{temporalia/ant-quaeritedominumet.gtex}

%\vspace{-2mm}

\scriptura{Ps. 68, 30-37}

%\vspace{-2mm}

\initiumpsalmi{temporalia/ps68iii-initium-viii-G-auto.gtex}

\input{temporalia/ps68iii-viii-G.tex} \Abardot{}

\vfill
\pagebreak
\fi
\ifx\matud\undefined
\else
% MAT D
\pars{Psalmus 1.} \scriptura{Ps. 72, 8; \textbf{H179}}

\vspace{-4mm}

\antiphona{VIII c}{temporalia/ant-cogitaveruntimpii.gtex}

%\vspace{-3mm}

\scriptura{Ps. 54, 2-8}

%\vspace{-2mm}

\initiumpsalmi{temporalia/ps54i-initium-viii-c-auto.gtex}

%\vspace{-1.5mm}

\input{temporalia/ps54i-viii-c.tex} \Abardot{}

\vfill
\pagebreak

\pars{Psalmus 2.} \scriptura{Ps. 34, 4; \textbf{H178}}

\vspace{-4mm}

\antiphona{VII c\textsuperscript{2}}{temporalia/ant-avertanturretrorsum.gtex}

%\vspace{-2mm}

\scriptura{Ps. 54, 9-16}

%\vspace{-2mm}

\initiumpsalmi{temporalia/ps54ii-initium-vii-c2-auto.gtex}

\input{temporalia/ps54ii-vii-c2.tex} \Abardot{}

\vfill
\pagebreak

\pars{Psalmus 3.}

\vspace{-4mm}

\antiphona{VIII G}{temporalia/ant-iustusnonconturbabitur.gtex}

%\vspace{-2mm}

\scriptura{Ps. 54, 17-24}

%\vspace{-2mm}

\initiumpsalmi{temporalia/ps54iii-initium-viii-G-auto.gtex}

\input{temporalia/ps54iii-viii-G.tex} \Abardot{}

\vfill
\pagebreak
\fi

\pars{Versus.}

\ifx\matversus\undefined
\ifx\matua\undefined
\else
\noindent \Vbardot{} Fili mi, custódi sermónes meos.

\noindent \Rbardot{} Serva mandáta mea et vives.
\fi
\ifx\matub\undefined
\else
\noindent \Vbardot{} Oculi mei defecérunt in desidério salutáris tui.

\noindent \Rbardot{} Et elóquii iustítiæ tuæ.
\fi
\ifx\matuc\undefined
\else
\noindent \Vbardot{} Dóminus vias suas docébit nos.

\noindent \Rbardot{} Et ambulábimus in sémitis eius.
\fi
\ifx\matud\undefined
\else
\noindent \Vbardot{} Tribulátio et angústia invenérunt me.

\noindent \Rbardot{} Mandáta tua meditátio mea est.
\fi
\else
\matversus
\fi

\vspace{5mm}

\sineinitiali{temporalia/oratiodominica-mat.gtex}

\vspace{5mm}

\pars{Absolutio.}

\cuminitiali{}{temporalia/absolutio-ipsius.gtex}

\vfill
\pagebreak

\cuminitiali{}{temporalia/benedictio-solemn-deus.gtex}

\vspace{7mm}

\lectioi

\noindent \Vbardot{} Tu autem, Dómine, miserére nobis.
\noindent \Rbardot{} Deo grátias.

\vfill
\pagebreak

\responsoriumi

\vfill
\pagebreak

\cuminitiali{}{temporalia/benedictio-solemn-christus.gtex}

\vspace{7mm}

\lectioii

\noindent \Vbardot{} Tu autem, Dómine, miserére nobis.
\noindent \Rbardot{} Deo grátias.

\vfill
\pagebreak

\responsoriumii

\vfill
\pagebreak

\cuminitiali{}{temporalia/benedictio-solemn-ignem.gtex}

\vspace{7mm}

\lectioiii

\noindent \Vbardot{} Tu autem, Dómine, miserére nobis.
\noindent \Rbardot{} Deo grátias.

\vfill
\pagebreak

\responsoriumiii

\vfill
\pagebreak

\rubrica{Reliqua omittuntur, nisi Laudes separandæ sint.}

\sineinitiali{temporalia/domineexaudi.gtex}

\vfill

\oratio

\vfill

\noindent \Vbardot{} Dómine, exáudi oratiónem meam.
\Rbardot{} Et clamor meus ad te véniat.

\vfill

\noindent \Vbardot{} Benedicámus Dómino.
\noindent \Rbardot{} Deo grátias.

\vfill

\noindent \Vbardot{} Fidélium ánimæ per misericórdiam Dei requiéscant in pace.
\Rbardot{} Amen.

\vfill
\pagebreak

\hora{Ad Laudes.} %%%%%%%%%%%%%%%%%%%%%%%%%%%%%%%%%%%%%%%%%%%%%%%%%%%%%

\cantusSineNeumas

\vspace{0.5cm}
\grechangedim{interwordspacetext}{0.18 cm plus 0.15 cm minus 0.05 cm}{scalable}%
\cuminitiali{}{temporalia/deusinadiutorium-communis-quad.gtex}
\grechangedim{interwordspacetext}{0.22 cm plus 0.15 cm minus 0.05 cm}{scalable}%

\vfill
\pagebreak

\ifx\hymnuslaudes\undefined
\ifx\laudac\undefined
\else
\pars{Hymnus}

\grechangedim{interwordspacetext}{0.16 cm plus 0.15 cm minus 0.05 cm}{scalable}%
\cuminitiali{I}{temporalia/hym-AEternaCaeli.gtex}
\grechangedim{interwordspacetext}{0.22 cm plus 0.15 cm minus 0.05 cm}{scalable}%
\vspace{-3mm}
\fi
\ifx\laudbd\undefined
\else
\pars{Hymnus}

\grechangedim{interwordspacetext}{0.16 cm plus 0.15 cm minus 0.05 cm}{scalable}%
\cuminitiali{IV}{temporalia/hym-DeusQui.gtex}
\grechangedim{interwordspacetext}{0.22 cm plus 0.15 cm minus 0.05 cm}{scalable}%
\vspace{-3mm}
\fi
\else
\hymnuslaudes
\fi

\vfill
\pagebreak

\ifx\lauda\undefined
\else
\pars{Psalmus 1.} \scriptura{Ps. 50, 3; \textbf{H93}}

\vspace{-4mm}

\antiphona{VI F}{temporalia/ant-misereremeideus.gtex}

\scriptura{Psalmus 50.}

\initiumpsalmi{temporalia/ps50-initium-vi-F-auto.gtex}

\input{temporalia/ps50-vi-F.tex}

\vfill

\antiphona{}{temporalia/ant-misereremeideus.gtex}

\vfill
\pagebreak

\pars{Psalmus 2.} \scriptura{Is. 45, 25}

\vspace{-4mm}

\antiphona{V a}{temporalia/ant-indominoiustificabitur.gtex}

\scriptura{Canticum Isaiæ, Is. 45, 15-30}

%\vspace{-2mm}

\initiumpsalmi{temporalia/isaiae2-initium-v-a-auto.gtex}

\input{temporalia/isaiae2-v-a.tex}

\vfill

\antiphona{}{temporalia/ant-indominoiustificabitur.gtex}

\vfill
\pagebreak

\pars{Psalmus 3.} \scriptura{Ps. 99, 1; \textbf{H98}}

\vspace{-4mm}

\antiphona{IV* e}{temporalia/ant-iubilatedeo.gtex}

\scriptura{Psalmus 99.}

\initiumpsalmi{temporalia/ps99-initium-iv_-e-auto.gtex}

\input{temporalia/ps99-iv_-e.tex} \Abardot{}

\vfill
\pagebreak
\fi
\ifx\laudb\undefined
\else
\pars{Psalmus 1.} \scriptura{Ps. 50, 4; \textbf{H95}}

\vspace{-4mm}

\antiphona{VII a}{temporalia/ant-ampliuslavame.gtex}

\scriptura{Psalmus 50.}

\initiumpsalmi{temporalia/ps50-initium-vii-a-auto.gtex}

\input{temporalia/ps50-vii-a.tex}

\vfill

\antiphona{}{temporalia/ant-ampliuslavame.gtex}

\vfill
\pagebreak

\pars{Psalmus 2.} \scriptura{Hab. 3, 2; \textbf{H99}}

\vspace{-6mm}

\antiphona{IV* e}{temporalia/ant-domineaudivi.gtex}

\vspace{-2mm}

\scriptura{Canticum Habacuc, Hab. 3, 2-19}

%\vspace{-2mm}

%\initiumpsalmi{temporalia/habacuc-initium-iv_-e-auto.gtex}
\initiumpsalmi{temporalia/habacuc-initium-iv_-e.gtex}

\input{temporalia/habacuc-iv_-e.tex}

\vfill

\antiphona{}{temporalia/ant-domineaudivi.gtex}

\vfill
\pagebreak

\pars{Psalmus 3.} \scriptura{Ps. 147, 12}

\vspace{-4mm}

\antiphona{E}{temporalia/ant-laudaierusalem.gtex}

\vspace{-2mm}

\scriptura{Psalmus 147.}

%\vspace{-3mm}

%\initiumpsalmi{temporalia/ps147-initium-e-auto.gtex}
\initiumpsalmi{temporalia/ps147-initium-e.gtex}

\input{temporalia/ps147-e.tex} \Abardot{}

\vfill
\pagebreak
\fi
\ifx\laudc\undefined
\else
\pars{Psalmus 1.} \scriptura{Ps. 50, 6.3; \textbf{H96}}

\vspace{-4mm}

\antiphona{VIII G\textsuperscript{2}}{temporalia/ant-tibisoli.gtex}

\scriptura{Psalmus 50.}

\initiumpsalmi{temporalia/ps50-initium-viii-G2-auto.gtex}

\input{temporalia/ps50-viii-G2.tex}

\vfill

\antiphona{}{temporalia/ant-tibisoli.gtex}

\vfill
\pagebreak

\pars{Psalmus 2.}

\vspace{-4mm}

\antiphona{VIII G}{temporalia/ant-nosnosderelinquas.gtex}

%\vspace{-2mm}

\scriptura{Canticum Ieremiæ, Ier. 14, 17-31}

%\vspace{-2mm}

\initiumpsalmi{temporalia/jeremiae2-initium-viii-G.gtex}

\input{temporalia/jeremiae2-viii-G.tex} \Abardot{}

\vfill
\pagebreak

\pars{Psalmus 3.}

\vspace{-4mm}

\antiphona{E}{temporalia/ant-servitedominoinlaetitia.gtex}

\vspace{-2mm}

\scriptura{Psalmus 99.}

%\vspace{-2mm}

\initiumpsalmi{temporalia/ps99-initium-e.gtex}

\input{temporalia/ps99-e.tex} \Abardot{}

\vfill
\pagebreak
\fi
\ifx\laudd\undefined
\else
\pars{Psalmus 1.} \scriptura{Ps. 50, 12}

\vspace{-4mm}

\antiphona{I a\textsuperscript{2}}{temporalia/ant-cormundumcrea.gtex}

\scriptura{Psalmus 50.}

\initiumpsalmi{temporalia/ps50-initium-i-a2-auto.gtex}

\input{temporalia/ps50-i-a2.tex}

\vfill

\antiphona{}{temporalia/ant-cormundumcrea.gtex}

\vfill
\pagebreak

\pars{Psalmus 2.}

\vspace{-4mm}

\antiphona{II D}{temporalia/ant-aedificansierusalem.gtex}

%\vspace{-2mm}

\scriptura{Canticum Tobiæ, Tob. 13, 10-18}

%\vspace{-2mm}

\initiumpsalmi{temporalia/tobiae2-initium-ii-D-auto.gtex}

\input{temporalia/tobiae2-ii-D.tex} \Abardot{}

\vfill
\pagebreak

\pars{Psalmus 3.} \scriptura{Ps. 147, 13; \textbf{H101}}

\vspace{-4mm}

\antiphona{VI F}{temporalia/ant-benedixitfiliistuis.gtex}

\vspace{-2mm}

\scriptura{Psalmus 147.}

%\vspace{-2mm}

\initiumpsalmi{temporalia/ps147-initium-vi-F-auto.gtex}

\input{temporalia/ps147-vi-F.tex} \Abardot{}

\vfill
\pagebreak
\fi

\ifx\lectiobrevis\undefined
\ifx\lauda\undefined
\else
\pars{Lectio Brevis.} \scriptura{Eph. 4, 29-32}

\noindent Omnis sermo malus ex ore vestro non procédat, sed si quis bonus ad ædificatiónem opportunitátis, ut det grátiam audiéntibus. Et nolíte contristáre Spíritum Sanctum Dei, in quo signáti estis in diem redemptiónis. Omnis amaritúdo et ira et indignátio et clamor et blasphémia tollátur a vobis cum omni malítia. Estóte autem ínvicem benígni, misericórdes, donántes ínvicem, sicut et Deus in Christo donávit vobis.
\fi
\ifx\laudb\undefined
\else
\pars{Lectio Brevis.} \scriptura{Eph. 2, 13-16}

\noindent Nunc in Christo Iesu vos, qui aliquándo erátis longe, facti estis prope in sánguine Christi. Ipse est enim pax nostra, qui fecit utráque unum et médium paríetem macériæ solvit, inimicítiam, in carne sua, legem mandatórum in decrétis evácuans, ut duos condat in semetípso in unum novum hóminem, fáciens pacem, et reconcíliet ambos in uno córpore Deo per crucem interfíciens inimicítiam in semetípso.
\fi
\ifx\laudc\undefined
\else
\pars{Lectio Brevis.} \scriptura{2 Cor. 12, 9-10}

\noindent Libentíssime gloriábor in infirmitátibus meis, ut inhábitet in me virtus Christi. Propter quod pláceo mihi in infirmitátibus, in contuméliis, in necessitátibus, in persecutiónibus et in angústiis, pro Christo: cum enim infírmor, tunc potens sum.
\fi
\ifx\laudd\undefined
\else
\pars{Lectio Brevis.} \scriptura{2 Cor. 1, 3-5}

\noindent Benedíctus Deus et Pater Dómini nostri Iesu Christi, Pater misericordiárum et Deus totíus consolatiónis, qui consolátur nos in omni tribulatióne nostra, ut possímus et ipsi consolári eos, qui in omni pressúra sunt, per exhortatiónem, qua exhortámur et ipsi a Deo; quóniam, sicut abúndant passiónes Christi in nobis, ita per Christum abúndat et consolátio nostra.
\fi
\else
\lectiobrevis
\fi

\vfill

\ifx\responsoriumbreve\undefined
\ifx\laudac\undefined
\else
\pars{Responsorium breve.} \scriptura{Ps. 142, 8}

\cuminitiali{VI}{temporalia/resp-auditamfacmihi.gtex}
\fi
\ifx\laudbd\undefined
\else
\pars{Responsorium breve.} \scriptura{Ps. 56, 3-4}

\cuminitiali{VI}{temporalia/resp-clamaboaddeum.gtex}
\fi
\else
\responsoriumbreve
\fi

\vfill
\pagebreak

\ifx\benedictus\undefined
\ifx\laudac\undefined
\else
\pars{Canticum Zachariæ.} \scriptura{Lc. 1, 68; \textbf{H422}}

%\vspace{-4mm}

{
\grechangedim{interwordspacetext}{0.18 cm plus 0.15 cm minus 0.05 cm}{scalable}%
\antiphona{V a}{temporalia/ant-visitavitetfecit.gtex}
\grechangedim{interwordspacetext}{0.22 cm plus 0.15 cm minus 0.05 cm}{scalable}%
}

%\vspace{-3mm}

\scriptura{Lc. 1, 68-79}

%\vspace{-2mm}

\cantusSineNeumas
\initiumpsalmi{temporalia/benedictus-initium-v-a-auto.gtex}

%\vspace{-1.5mm}

\input{temporalia/benedictus-v-a.tex} \Abardot{}
\fi
\ifx\laudbd\undefined
\else
\pars{Canticum Zachariæ.} \scriptura{Lc. 1, 78; \textbf{H423}}

%\vspace{-4mm}

{
\grechangedim{interwordspacetext}{0.18 cm plus 0.15 cm minus 0.05 cm}{scalable}%
\antiphona{VIII G}{temporalia/ant-pervisceramisericordiae.gtex}
\grechangedim{interwordspacetext}{0.22 cm plus 0.15 cm minus 0.05 cm}{scalable}%
}

%\vspace{-3mm}

\scriptura{Lc. 1, 68-79}

%\vspace{-1mm}

\initiumpsalmi{temporalia/benedictus-initium-viii-G-auto.gtex}

\input{temporalia/benedictus-viii-G.tex} \Abardot{}
\fi
\else
\benedictus
\fi

\vspace{-1cm}

\vfill
\pagebreak

\pars{Preces.}

\sineinitiali{}{temporalia/tonusprecum.gtex}

\ifx\preces\undefined
\ifx\lauda\undefined
\else
\noindent Christum, qui per crucem suam salútem géneri cóntulit humáno, adorémus, \gredagger{} et pie clamémus:

\Rbardot{} Misericórdiam tuam nobis largíre, Dómine.

\noindent Christe, sol et dies noster, illúmina nos rádiis tuis, \gredagger{} et omnes sensus malos iam mane compésce.

\Rbardot{} Misericórdiam tuam nobis largíre, Dómine.

\noindent Custódi cogitatiónes, sermónes et ópera nostra, \gredagger{} ut hódie in conspéctu tuo placére possímus.

\Rbardot{} Misericórdiam tuam nobis largíre, Dómine.

\noindent Avérte fáciem tuam a peccátis nostris, \gredagger{} et omnes iniquitátes nostras dele.

\Rbardot{} Misericórdiam tuam nobis largíre, Dómine.

\noindent Per crucem et resurrectiónem tuam, \gredagger{} reple nos consolatióne Spíritus Sancti.

\Rbardot{} Misericórdiam tuam nobis largíre, Dómine.
\fi
\ifx\laudb\undefined
\else
\noindent Christum, qui sánguine suo per Spíritum Sanctum semetípsum óbtulit Patri ad emundándam consciéntiam nostram ab opéribus mórtuis, \gredagger{} adorémus et sincéro corde profiteámur:

\Rbardot{} In tua voluntáte pax nostra, Dómine.

\noindent Diéi exórdium a tua benignitáte suscépimus, \gredagger{} nobis páriter vitæ novæ concéde inítium.

\Rbardot{} In tua voluntáte pax nostra, Dómine.

\noindent Qui ómnia creásti providúsque consérvas, \gredagger{} fac ut inspiciámus perénne tui vestígium in creátis.

\Rbardot{} In tua voluntáte pax nostra, Dómine.

\noindent Qui sánguine tuo novum et ætérnum testaméntum sanxísti, \gredagger{} da ut, quæ prǽcipis faciéntes, tuo fidéles fœ́deri maneámus.

\Rbardot{} In tua voluntáte pax nostra, Dómine.

\noindent Qui, in cruce pendens, una cum sánguine aquam de látere effudísti, \gredagger{} hoc salutári flúmine áblue peccáta nostra et civitátem Dei lætífica.

\Rbardot{} In tua voluntáte pax nostra, Dómine.
\fi
\ifx\laudc\undefined
\else
\noindent Ad Christum óculos levémus, qui pro pópulo suo natus et mórtuus est ac resurréxit. \gredagger{} Itaque eum fidénter deprecémur:

\Rbardot{} Salva, Dómine, quos tuo sánguine redemísti.

\noindent Benedíctus es, Iesu hóminum salvátor, qui passiónem et crucem pro nobis subíre non dubitásti, \gredagger{} et sánguine tuo pretióso nos redemísti.

\Rbardot{} Salva, Dómine, quos tuo sánguine redemísti.

\noindent Qui promisísti te aquam esse datúrum saliéntem in vitam ætérnam, \gredagger{} Spíritum tuum effúnde super omnes hómines.

\Rbardot{} Salva, Dómine, quos tuo sánguine redemísti.

\noindent Qui discípulos misísti ad Evangélium géntibus prædicándum, \gredagger{} eos ádiuva, ut victóriam tuæ crucis exténdant.

\Rbardot{} Salva, Dómine, quos tuo sánguine redemísti.

\noindent Infírmis et míseris quos cruci tuæ sociásti, \gredagger{} virtútem et patiéntiam concéde.

\Rbardot{} Salva, Dómine, quos tuo sánguine redemísti.
\fi
\ifx\laudd\undefined
\else
\noindent Fratres, Salvatórem nostrum, testem fidélem, per mártyres interféctos propter verbum Dei, \gredagger{} celebrémus, clamántes:

\Rbardot{} Redemísti nos Deo in sánguine tuo.

\noindent Per mártyres tuos, qui líbere mortem in testimónium fídei sunt ampléxi, \gredagger{} da nobis, Dómine, veram spíritus libertátem.

\Rbardot{} Redemísti nos Deo in sánguine tuo.

\noindent Per mártyres tuos, qui fidem usque ad sánguinem sunt conféssi, \gredagger{} da nobis, Dómine, puritátem fideíque constántiam.

\Rbardot{} Redemísti nos Deo in sánguine tuo.

\noindent Per mártyres tuos, qui, sustinéntes crucem, tua vestígia sunt secúti, \gredagger{} da nobis, Dómine, ærúmnas vitæ fórtiter sustinére.

\Rbardot{} Redemísti nos Deo in sánguine tuo.

\noindent Per mártyres tuos, qui stolas suas lavérunt in sánguine Agni, \gredagger{} da nobis, Dómine, omnes insídias carnis mundíque devíncere.

\Rbardot{} Redemísti nos Deo in sánguine tuo.
\fi 
\else
\preces
\fi

\vfill

\pars{Oratio Dominica.}

\cuminitiali{}{temporalia/oratiodominicaalt.gtex}

\vfill
\pagebreak

\rubrica{vel:}

\pars{Supplicatio Litaniæ.}

\cuminitiali{}{temporalia/supplicatiolitaniae.gtex}

\vfill

\pars{Oratio Dominica.}

\cuminitiali{}{temporalia/oratiodominica.gtex}

\vfill
\pagebreak

% Oratio. %%%
\oratio

\vspace{-1mm}

\vfill

\rubrica{Hebdomadarius dicit Dominus vobiscum, vel, absente sacerdote vel diacono, sic concluditur:}

\vspace{2mm}

\antiphona{C}{temporalia/dominusnosbenedicat.gtex}

\rubrica{Postea cantatur a cantore:}

\vspace{2mm}

\cuminitiali{IV}{temporalia/benedicamus-feria-advequad.gtex}

\vspace{1mm}

\vfill
\pagebreak

\hora{Ad Vesperas.} %%%%%%%%%%%%%%%%%%%%%%%%%%%%%%%%%%%%%%%%%%%%%%%%%%%%%

\cantusSineNeumas

%\vspace{0.5cm}
\grechangedim{interwordspacetext}{0.18 cm plus 0.15 cm minus 0.05 cm}{scalable}%
\cuminitiali{}{temporalia/deusinadiutorium-communis-tq.gtex}
\grechangedim{interwordspacetext}{0.22 cm plus 0.15 cm minus 0.05 cm}{scalable}%

\vfill
%\pagebreak

\vspace{4mm}

\pars{Psalmus 1.} \scriptura{Ps. 141, 6; \textbf{H99}}

\vspace{-4mm}

\antiphona{VIII a}{temporalia/ant-portiomeadomine.gtex}

%\vspace{-4mm}

\scriptura{Psalmus 141.}

\initiumpsalmi{temporalia/ps141-initium-viii-A-auto.gtex}

\input{temporalia/ps141-viii-A.tex}

\vfill

\antiphona{}{temporalia/ant-portiomeadomine.gtex}

\vfill
\pagebreak

\pars{Psalmus 2.} \scriptura{Ps. 143, 1; \textbf{H99}}

\vspace{-4mm}

\antiphona{VI F}{temporalia/ant-benedictusdominus.gtex}

\scriptura{Psalmus 143, 1-8}

\initiumpsalmi{temporalia/ps143i-initium-vi-F-auto.gtex}

\input{temporalia/ps143i-vi-F.tex}

\vspace{4mm}

\rubrica{Hic non dicitur antiphonam.}

\vfill
\pagebreak

\pars{Psalmus 3.} \scriptura{Psalmus 143, 9-15}

\initiumpsalmi{temporalia/ps143ii-initium-vi-F-auto.gtex}

\input{temporalia/ps143ii-vi-F.tex}

\vfill

\antiphona{}{temporalia/ant-benedictusdominus.gtex}

\vfill
\pagebreak

\pars{Psalmus 4.} \scriptura{Ps. 144, 2; \textbf{H99}}

\vspace{-4mm}

\antiphona{VIII a}{temporalia/ant-persingulosdies.gtex}

\scriptura{Psalmus 144, 1-9}

\initiumpsalmi{temporalia/ps144i-initium-viii-A-auto.gtex}

\input{temporalia/ps144i-viii-A.tex} \Abardot{}

\vfill
\pagebreak

\pars{Capitulum.} \scriptura{Sir. 24, 14}

\grechangedim{interwordspacetext}{0.12 cm plus 0.15 cm minus 0.05 cm}{scalable}%
\cuminitiali{}{temporalia/capitulum-AbInitio.gtex}
\grechangedim{interwordspacetext}{0.22 cm plus 0.15 cm minus 0.05 cm}{scalable}%

\vfill

\pars{Responsorium breve.} \scriptura{Lc. 1, 28}

\cuminitiali{VI}{temporalia/resp-avemaria.gtex}

\vfill
\pagebreak

\pars{Hymnus}

\cuminitiali{I}{temporalia/hym-AveMarisStella.gtex}
\vspace{-3mm}

\vfill
%\pagebreak

\pars{Versus.} \scriptura{Ps. 43, 3}

\sineinitiali{temporalia/versus-diffusa-tq.gtex}

\vfill
\pagebreak

\ifx\magnificat\undefined
\pars{Canticum B. Mariæ V.} \scriptura{\textbf{H300}}

\vspace{-4mm}

{
\grechangedim{interwordspacetext}{0.18 cm plus 0.15 cm minus 0.05 cm}{scalable}%
\antiphona{II D}{temporalia/ant-beatamater.gtex}
\grechangedim{interwordspacetext}{0.22 cm plus 0.15 cm minus 0.05 cm}{scalable}%
}

\vspace{-3mm}

\scriptura{Lc. 1, 46-55}

\cantusSineNeumas
\initiumpsalmi{temporalia/magnificat-initium-ii-D.gtex}

%\vspace{-3mm}

\input{temporalia/magnificat-ii-D.tex} \Abardot{}
\else
\magnificat
\fi

\vspace{-1cm}

\vfill
\pagebreak

\anteOrationem

\pagebreak

% Oratio. %%%
\cuminitiali{}{temporalia/oratiobmv.gtex}

\vspace{-1mm}

\vfill

\rubrica{Hebdomadarius dicit iterum Dominus vobiscum, vel cantor dicit:}

\vspace{2mm}

\sineinitiali{temporalia/domineexaudi.gtex}

\rubrica{Postea cantatur a cantore:}

\vspace{2mm}

\cuminitiali{VIII}{temporalia/benedicamus-officium-bmv.gtex}

\vspace{1mm}

\vfill

\end{document}

