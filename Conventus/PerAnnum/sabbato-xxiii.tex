\newcommand{\titulus}{\nomenFesti{In Exaltatione Sanctæ Crucis.}
\dies{Die 14. Septembris.}}
\newcommand{\sineobmv}{Sine officium B.M.V.}
\newcommand{\oratio}{\pars{Oratio.}

\noindent Deus, qui Unigénitum tuum crucem subíre voluísti, ut salvum fáceret genus humánum, præsta, quǽsumus, ut, cuius mystérium in terra cognóvimus, eius redemptiónis prǽmia in cælo cónsequi mereámur.

\pars{Pro pace in universo mundo.} \scriptura{Sir. 50, 25; 2 Esdr. 4, 20; \textbf{H416}}

\vspace{-4mm}

\antiphona{II D}{temporalia/ant-dapacemdomine.gtex}

\vfill

\noindent Deus, a quo sancta desidéria, recta consília et iusta sunt ópera: da servis tuis illam, quam mundus dare non potest, pacem; ut et corda nostra mandátis tuis dédita, et hóstium subláta formídine, témpora sint tua protectióne tranquílla.

\noindent Per Dóminum nostrum Iesum Christum, Fílium tuum, qui tecum vivit et regnat in unitáte Spíritus Sancti, Deus, per ómnia sǽcula sæculórum.

\noindent \Rbardot{} Amen.}
\newcommand{\invitatorium}{\pars{Invitatorium.}

\vspace{-2mm}

\antiphona{IV*}{temporalia/inv-christumregem.gtex}}
\newcommand{\hymnusmatutinum}{\pars{Hymnus.}

\vspace{-5mm}

\antiphona{I}{temporalia/hym-SalveCruxSancta.gtex}}
\newcommand{\matutinum}{\pars{Psalmus 1.} \scriptura{Cf. Ap. 5, 5; \textbf{H257}}

\antiphona{I f}{temporalia/ant-eccecrucem.gtex}

%\vspace{-2mm}

\scriptura{Ps. 2}

%\vspace{-2mm}

\initiumpsalmi{temporalia/ps2-initium-i-f-auto.gtex}

%\vspace{-1.5mm}

\input{temporalia/ps2-i-f.tex} \Abardot{}

\vfill
\pagebreak

\pars{Psalmus 2.}

\vspace{-4mm}

\antiphona{VIII G}{temporalia/ant-propterlignumservi.gtex}

%\vspace{-2mm}

\scriptura{Ps. 8}

%\vspace{-2mm}

\initiumpsalmi{temporalia/ps8-initium-viii-G-auto.gtex}

%\vspace{-1.5mm}

\input{temporalia/ps8-viii-G.tex} \Abardot{}

\vfill
\pagebreak

\pars{Psalmus 3.} \scriptura{\textbf{H258}}

\vspace{-4mm}

\antiphona{I d}{temporalia/ant-ocruxbenedicta.gtex}

%\vspace{-2mm}

\scriptura{Ps. 95}

%\vspace{-2mm}

\initiumpsalmi{temporalia/ps95-initium-i-d-auto.gtex}

\input{temporalia/ps95-i-d.tex}

\antiphona{}{temporalia/ant-ocruxbenedicta.gtex}

\vfill
\pagebreak}
\newcommand{\matversus}{\noindent \Vbardot{} Sicut Móyses exaltávit serpéntem in desérto.

\noindent \Rbardot{} Ita exaltári opórtet Fílium hóminis.}
\newcommand{\lectioi}{\pars{Lectio I.} \scriptura{Gal. 2, 19-21; 3, 7.13-14; 6, 14-16}

\noindent De Epístola beáti Pauli apóstoli ad Gálatas.

\noindent Fratres: Ego Paulus per legem legi mórtuus sum, ut Deo vivam. Christo confíxus sum cruci: vivo autem iam non ego, vivit vero in me Christus; quod autem nunc vivo in carne, in fide vivo Fílii Dei, qui diléxit me et trádidit seípsum pro me. Non írritam fácio grátiam Dei; si enim per legem iustítia, ergo Christus gratis mórtuus est.

\noindent O insensáti Gálatæ, quis vos fascinávit, ante quorum óculos Iesus Christus descríptus est crucifíxus? Hoc solum volo a vobis díscere: Ex opéribus legis Spíritum accepístis an ex audítu fídei? Sic stulti estis? Cum Spíritu cœpéritis, nunc carne consummámini? Tanta passi estis sine causa? Si tamen et sine causa! Qui ergo tríbuit vobis Spíritum et operátur virtútes in vobis, ex opéribus legis an ex audítu fídei?

\noindent Sicut Abraham \emph{crédidit Deo et reputátum est ei ad iustítiam.} Cognóscitis ergo quia qui ex fide sunt, hi sunt fílii Abrahæ.

\noindent Christus nos redémit de maledícto legis factus pro nobis maledíctum, quia scriptum est: \emph{«Maledíctus omnis, qui pendet in ligno»,} ut in gentes benedíctio Abrahæ fíeret in Christo Iesu, ut promissiónem Spíritus accipiámus per fidem.

\noindent Mihi autem absit gloriári nisi in cruce Dómini nostri Iesu Christi, per quem mihi mundus crucifíxus est et ego mundo.

\noindent Neque enim circumcísio áliquid est, neque præpútium, sed nova creatúra. Et quicúmque hanc régulam secúti fúerint, pax super illos et misericórdia et super Israel Dei.}
\newcommand{\responsoriumi}{\pars{Responsorium 1.} \scriptura{\textbf{H255}}

\vspace{-5mm}

\responsorium{IV}{temporalia/resp-hocsignumcrucis-CROCHU.gtex}{}

\rubrica{vel ad libitum:}

\vspace{3mm}

\pars{Responsorium 1.} 

\scriptura{\Vbardot{} Gal. 6, 14; \textbf{H256}}

\vspace{-5mm}

\responsorium{II}{temporalia/resp-ocruxbenedicta-CROCHU.gtex}{}

\vfill
\pagebreak

\rubrica{vel ad libitum:}

\vspace{3mm}

\pars{Responsorium 1.} \scriptura{Venantius Fortunatus (sæc. VI); \textbf{H256}}

\vspace{-5mm}

\responsorium{IV}{temporalia/resp-cruxbenedictanitet-CROCHU-sinedox.gtex}{}}
\newcommand{\lectioii}{\pars{Lectio II.} \scriptura{Oratio 10 in Exaltatione sanctæ crucis: PG 97, 1018-1019. 1022-1023}

\noindent Ex Oratiónibus sancti Andréæ Creténsis epíscopi.

\noindent Crucis festum celebrámus, per quam ténebræ pulsæ sunt et lumen rédditum. Crucis festum celebrámus, et una cum Crucifíxo in sublíme tóllimur, ut, terra cum peccáto infra relícta, supérna comparémus. Talis est tantáque crucis posséssio et qui hanc póssidet, póssidet thesáurum. Ego vero id quod ómnium bonórum pulchérrimum est re ac nómine, thesáurum iure appelláverim; in quo et per quem et in quem salútis nostræ summa repósita et prístino státui restitúta est.

\noindent Si enim crux non foret, Christus crucifíxus non esset. Si crux non esset, vita ligno clavis suffíxa non esset. Si ea clavis suffíxa non esset, ex látere fontes immortalitátis, sánguinem et aquam, quæ mundum éxpiant, non fudíssent; peccáti chirógraphum disrúptum non foret, in libertátem assérti non essémus, ligno vitæ non fruerémur, paradísus non patéret. Si crux non esset, mors prostráta non esset, inférnus non spoliátus.}
\newcommand{\responsoriumii}{\pars{Responsorium 2.} \scriptura{Venantius Fortunatus (sæc. VI); \textbf{H256}}

\vspace{-5mm}

\responsorium{II}{temporalia/resp-cruxfidelis-CROCHU.gtex}{}

\rubrica{vel ad libitum:}

\vspace{3mm}

\pars{Responsorium 2.} \scriptura{Venantius Fortunatus; \textbf{H255}}

\vspace{-5mm}

\responsorium{VIII}{temporalia/resp-dulcelignum-CROCHU-sinedox.gtex}{}}
\newcommand{\lectioiii}{\pars{Lectio III.}

\noindent Magna ígitur et pretiósa res crux est. Magna quidem, quia multa per ipsam bona effécta sunt; et tanto plura, quanto magis Christi miráculis et cruciátibus potióres partes tribuéndæ sunt. Pretiósa vero, quia Dei pássio et tropǽum, crux est: pássio quidem, ob spontáneam in ipsa passiónis mortem; tropǽum autem, quia in ipsa diábolus sauciátus et cum eo mors devícta est atque inferórum claustra contríta et commúnis univérsi orbis salus crux facta est.

\noindent Hæc et Christi glória appellátur et Christi exaltátio dícitur. Hæc et calix desiderábilis intellégitur et cruciátuum, quos pro nobis Christus perpéssus est, conclúsio. Quod vero Christi glória sit crux, audi ipsum dicéntem: \emph{Nunc clarificátus est Fílius hóminis et Deus clarificátus est in eo, et contínuo clarificábit eum.} Et rursus: \emph{Clarífica me, tu, Pater, claritáte quam hábui, ántequam mundus esset, apud te.} Et íterum: \emph{Pater, clarífica nomen tuum. Venit ergo vox de cælo: Et clarificávi et íterum clarificábo,} illam signíficans, quæ tunc in cruce consecúta est.

\noindent Quod autem exaltátio quoque Christi, crux sit, áccipe quid ípsemet ait: \emph{Quando ego exaltátus fúero, tunc omnes traham ad meípsum.} Vides quod glória et exaltátio Christi, crux sit.}
\newcommand{\responsoriumiii}{\pars{Responsorium 3.} \scriptura{\Vbardot{} Cf. Gal. 6, 14; \textbf{H256}}

\vspace{-5mm}

\responsorium{VII}{temporalia/resp-ocruxgloriosa-CROCHU.gtex}{}

\vfill
\pagebreak

\pars{Hymnus Ambrosianus} \scriptura{Alio modo, iuxta morem Romanum}

\vspace{-2mm}

{
\grechangedim{interwordspacetext}{0.26 cm plus 0.15 cm minus 0.05 cm}{scalable}%
\cuminitiali{III}{temporalia/tedeum-romanum-gn.gtex}
\grechangedim{interwordspacetext}{0.22 cm plus 0.15 cm minus 0.05 cm}{scalable}%
}}
\newcommand{\deusinadiutorium}{\grechangedim{interwordspacetext}{0.18 cm plus 0.15 cm minus 0.05 cm}{scalable}%
\cuminitiali{}{temporalia/deusinadiutorium-alter.gtex}
\grechangedim{interwordspacetext}{0.22 cm plus 0.15 cm minus 0.05 cm}{scalable}}
\newcommand{\hymnuslaudes}{\pars{Hymnus}

\cuminitiali{IV}{temporalia/hym-SignumCrucis.gtex}}
\newcommand{\laudes}{\pars{Psalmus 1.}

\vspace{-4mm}

\antiphona{II D}{temporalia/ant-crucemsanctam.gtex}

%\vspace{-2mm}

\scriptura{Psalmus 62}

%\vspace{-2mm}

\initiumpsalmi{temporalia/ps62-initium-ii-D-auto.gtex}

%\vspace{-1.5mm}

\input{temporalia/ps62-ii-D.tex} \Abardot{}

\vfill
\pagebreak

\pars{Psalmus 2.} \scriptura{S. Venantius Fortunatus; \textbf{H258}}

\vspace{-4mm}

\antiphona{VIII G}{temporalia/ant-cruxbenedictanitetdominus.gtex}

%\vspace{-2mm}

\scriptura{Canticum trium puerorum, Dan. 3, 57-88 et 56}

\initiumpsalmi{temporalia/dan3-initium-viii-G-auto.gtex}

\input{temporalia/dan3-viii-G-sinedox.tex}

\rubrica{Hic non dicitur Gloria Patri, neque Amen.}

\vfill

\antiphona{}{temporalia/ant-cruxbenedictanitetdominus.gtex}

\vfill
\pagebreak

\pars{Psalmus 3.} \scriptura{\textbf{H259}}

\vspace{-4mm}

\antiphona{VII a}{temporalia/ant-cruxalmafulget.gtex}

%\vspace{-2mm}

\scriptura{Psalmus 149}

%\vspace{-2mm}

\initiumpsalmi{temporalia/ps149-initium-vii-a-auto.gtex}

\input{temporalia/ps149-vii-a.tex} \Abardot{}

\vfill
\pagebreak}
\newcommand{\lectiobrevis}{\pars{Lectio Brevis.} \scriptura{Hebr. 2, 9-10}

\noindent Vidémus Iesum propter passiónem mortis glória et honóre coronátum, ut grátia Dei pro ómnibus gustáverit mortem. Decébat enim eum, propter quem ómnia et per quem ómnia, qui multos fílios in glóriam addúxit, auctórem salútis eórum per passiónes consummáre.}
\newcommand{\responsoriumbreve}{\pars{Responsorium breve.}

\antiphona{VI}{temporalia/resp-adoramustechriste.gtex}}
\newcommand{\preces}{\noindent \noindent Redemptórem nostrum,~\gredagger{} qui per crucem suam nos redémit,~\grestar{} fidénter deprecémur:

\Rbardot{} Per crucem tuam salva nos, Dómine.

\noindent Fili Dei,~\gredagger{} qui per signum serpéntis ǽnei sanásti pópulum Israel,~\grestar{} prótege nos hódie a morsu peccáti.

\Rbardot{} Per crucem tuam salva nos, Dómine.

\noindent Fili hóminis,~\gredagger{} qui exaltátus es in cruce sicut a Móyse serpens in desérto,~\grestar{} exálta nos ad gáudia regni tui.

\Rbardot{} Per crucem tuam salva nos, Dómine.

\noindent Fili unigénite Patris,~\gredagger{} qui datus es mundo, ut omnis qui credit in te, non péreat,~\grestar{} concéde vitam ætérnam quæréntibus fáciem tuam.

\Rbardot{} Per crucem tuam salva nos, Dómine.

\noindent Fili dilécte Patris,~\gredagger{} qui non missus es in mundum ut iúdices mundum, sed ut per te salvétur mundus,~\grestar{} da fidem propínquis nostris, ne péreant.

\Rbardot{} Per crucem tuam salva nos, Dómine.

\noindent Fili ætérne Patris,~\gredagger{} qui ignem venísti míttere in terram et voluísti ut accendátur,~\grestar{} præsta, ut faciámus veritátem et veniámus ad lucem.

\Rbardot{} Per crucem tuam salva nos, Dómine.}
\newcommand{\benedictus}{\pars{Canticum Zachariæ.} \scriptura{\textbf{H258}}

\vspace{-4mm}

\antiphona{I d\textsuperscript{3}}{temporalia/ant-superomnialigna.gtex}

%\vspace{-3mm}

\scriptura{Lc. 1, 68-79}

%\vspace{-2mm}

\cantusSineNeumas
\initiumpsalmi{temporalia/benedictus-initium-isoll-d3-auto.gtex}

%\vspace{-1.5mm}

\input{temporalia/benedictus-isoll-d3.tex}

\antiphona{}{temporalia/ant-superomnialigna.gtex}}
\newcommand{\benedicamuslaudes}{\cuminitiali{VIII}{temporalia/benedicamus-duplexmajus-laudes.gtex}}
\newcommand{\hebdomada}{infra Hebdom. XXIII post Pentecosten.}
\newcommand{\oratioLaudes}{\cuminitiali{}{temporalia/oratio23.gtex}}
\newcommand{\hiemalis}{Hiemalis.}

% LuaLaTeX

\documentclass[a4paper, twoside, 12pt]{article}
\usepackage[latin]{babel}
%\usepackage[landscape, left=3cm, right=1.5cm, top=2cm, bottom=1cm]{geometry} % okraje stranky
%\usepackage[landscape, a4paper, mag=1166, truedimen, left=2cm, right=1.5cm, top=1.6cm, bottom=0.95cm]{geometry} % okraje stranky
\usepackage[landscape, a4paper, mag=1400, truedimen, left=0.5cm, right=0.5cm, top=0.5cm, bottom=0.5cm]{geometry} % okraje stranky

\usepackage{fontspec}
\setmainfont[FeatureFile={junicode.fea}, Ligatures={Common, TeX}, RawFeature=+fixi]{Junicode}
%\setmainfont{Junicode}

% shortcut for Junicode without ligatures (for the Czech texts)
\newfontfamily\nlfont[FeatureFile={junicode.fea}, Ligatures={Common, TeX}, RawFeature=+fixi]{Junicode}

% Hebrew font:
% http://scripts.sil.org/cms/scripts/page.php?site_id=nrsi&id=SILHebrUnic2
\newfontfamily\hebfont[Scale=1]{Ezra SIL}

\usepackage{multicol}
\usepackage{color}
\usepackage{lettrine}
\usepackage{fancyhdr}

% usual packages loading:
\usepackage{luatextra}
\usepackage{graphicx} % support the \includegraphics command and options
\usepackage{gregoriotex} % for gregorio score inclusion
\usepackage{gregoriosyms}
\usepackage{wrapfig} % figures wrapped by the text
\usepackage{parcolumns}
\usepackage[contents={},opacity=1,scale=1,color=black]{background}
\usepackage{tikzpagenodes}
\usepackage{calc}
\usepackage{longtable}
\usetikzlibrary{calc}

\setlength{\headheight}{14.5pt}

% Commands used to produce a typical "Conventus" booklet

\newenvironment{titulusOfficii}{\begin{center}}{\end{center}}
\newcommand{\dies}[1]{#1

}
\newcommand{\nomenFesti}[1]{\textbf{\Large #1}

}
\newcommand{\celebratio}[1]{#1

}

\newcommand{\hora}[1]{%
\vspace{0.5cm}{\large \textbf{#1}}

\fancyhead[LE]{\thepage\ / #1}
\fancyhead[RO]{#1 / \thepage}
\addcontentsline{toc}{subsection}{#1}
}

% larger unit than a hora
\newcommand{\divisio}[1]{%
\begin{center}
{\Large \textsc{#1}}
\end{center}
\fancyhead[CO,CE]{#1}
\addcontentsline{toc}{section}{#1}
}

% a part of a hora, larger than pars
\newcommand{\subhora}[1]{
\begin{center}
{\large \textit{#1}}
\end{center}
%\fancyhead[CO,CE]{#1}
\addcontentsline{toc}{subsubsection}{#1}
}

% rubricated inline text
\newcommand{\rubricatum}[1]{\textit{#1}}

% standalone rubric
\newcommand{\rubrica}[1]{\vspace{3mm}\rubricatum{#1}}

\newcommand{\notitia}[1]{\textcolor{red}{#1}}

\newcommand{\scriptura}[1]{\hfill \small\textit{#1}}

\newcommand{\translatioCantus}[1]{\vspace{1mm}%
{\noindent\footnotesize \nlfont{#1}}}

% pruznejsi varianta nasledujiciho - umoznuje nastavit sirku sloupce
% s prekladem
\newcommand{\psalmusEtTranslatioB}[3]{
  \vspace{0.5cm}
  \begin{parcolumns}[colwidths={2=#3}, nofirstindent=true]{2}
    \colchunk{
      \input{#1}
    }

    \colchunk{
      \vspace{-0.5cm}
      {\footnotesize \nlfont
        \input{#2}
      }
    }
  \end{parcolumns}
}

\newcommand{\psalmusEtTranslatio}[2]{
  \psalmusEtTranslatioB{#1}{#2}{8.5cm}
}


\newcommand{\canticumMagnificatEtTranslatio}[1]{
  \psalmusEtTranslatioB{#1}{temporalia/extra-adventum-vespers/magnificat-boh.tex}{12cm}
}
\newcommand{\canticumBenedictusEtTranslatio}[1]{
  \psalmusEtTranslatioB{#1}{temporalia/extra-adventum-laudes/benedictus-boh.tex}{10.5cm}
}

% volne misto nad antifonami, kam si zpevaci dokresli neumy
\newcommand{\hicSuntNeumae}{\vspace{0.5cm}}

% prepinani mista mezi notovymi osnovami: pro neumovane a neneumovane zpevy
\newcommand{\cantusCumNeumis}{
  \setgrefactor{17}
  \global\advance\grespaceabovelines by 5mm%
}
\newcommand{\cantusSineNeumas}{
  \setgrefactor{17}
  \global\advance\grespaceabovelines by -5mm%
}

% znaky k umisteni nad inicialu zpevu
\newcommand{\superInitialam}[1]{\gresetfirstlineaboveinitial{\small {\textbf{#1}}}{\small {\textbf{#1}}}}

% pars officii, i.e. "oratio", ...
\newcommand{\pars}[1]{\textbf{#1}}

\newenvironment{psalmus}{
  \setlength{\parindent}{0pt}
  \setlength{\parskip}{5pt}
}{
  \setlength{\parindent}{10pt}
  \setlength{\parskip}{10pt}
}

%%%% Prejmenovat na latinske:
\newcommand{\nadpisZalmu}[1]{
  \hspace{2cm}\textbf{#1}\vspace{2mm}%
  \nopagebreak%

}

% mode, score, translation
\newcommand{\antiphona}[3]{%
\hicSuntNeumae
\superInitialam{#1}
\includescore{#2}

#3
}
 % Often used macros

\newcommand{\annusEditionis}{2021}

\def\hebinitial#1{%
\leavevmode{\newbox\hebbox\setbox\hebbox\hbox{\hebfont{#1}\hskip 1mm}\kern -\wd\hebbox\hbox{\hebfont{#1}\hskip 1mm}}%
}

%%%% Vicekrat opakovane kousky

\newcommand{\anteOrationem}{
  \rubrica{Ante Orationem, cantatur a Superiore:}

  \pars{Supplicatio Litaniæ.}

  \cuminitiali{}{temporalia/supplicatiolitaniae.gtex}

  \pars{Oratio Dominica.}

  \cuminitiali{}{temporalia/oratiodominica.gtex}
}

\setlength{\columnsep}{30pt} % prostor mezi sloupci

%%%%%%%%%%%%%%%%%%%%%%%%%%%%%%%%%%%%%%%%%%%%%%%%%%%%%%%%%%%%%%%%%%%%%%%%%%%%%%%%%%%%%%%%%%%%%%%%%%%%%%%%%%%%%
\begin{document}

% Here we set the space around the initial.
% Please report to http://home.gna.org/gregorio/gregoriotex/details for more details and options
\grechangedim{afterinitialshift}{2.2mm}{scalable}
\grechangedim{beforeinitialshift}{2.2mm}{scalable}

\grechangedim{interwordspacetext}{0.22 cm plus 0.15 cm minus 0.05 cm}{scalable}%
\grechangedim{annotationraise}{-0.2cm}{scalable}

% Here we set the initial font. Change 38 if you want a bigger initial.
% Emit the initials in red.
\grechangestyle{initial}{\color{red}\fontsize{38}{38}\selectfont}

\pagestyle{empty}

%%%% Titulni stranka
\begin{titulusOfficii}
\ifx\titulus\undefined
\nomenFesti{Sabbato \hebdomada{}}
\else
\titulus
\fi
\end{titulusOfficii}

\vfill

\pars{}

\scriptura{}

\pagebreak

% graphic
\renewcommand{\headrulewidth}{0pt} % no horiz. rule at the header
\fancyhf{}
\pagestyle{fancy}

\cantusSineNeumas

\hora{Ad Matutinum.}

\vspace{2mm}

\cuminitiali{}{temporalia/dominelabiamea.gtex}

\vspace{2mm}

\ifx\invitatorium\undefined
\pars{Invitatorium.} \scriptura{\textbf{H14}}

\vspace{-6mm}

\antiphona{VI}{temporalia/inv-regemventurumsimplex.gtex}
\else
\invitatorium
\fi

\vfill
\pagebreak

\ifx\hymnusmatutinum\undefined
\pars{Hymnus.}

\vspace{-5mm}

\antiphona{II}{temporalia/hym-VerbumSupernum.gtex}
\else
\hymnusmatutinum
\fi

\vfill
\pagebreak

\ifx\matutinum\undefined
\ifx\matua\undefined
\else
% MAT A
\pars{Psalmus 1.} \scriptura{Ps. 104, 3; \textbf{H99}}

\vspace{-6mm}

\antiphona{D}{temporalia/ant-laeteturcor.gtex}

\vspace{-4mm}

\scriptura{Ps. 104, 1-15}

\vspace{-2mm}

\initiumpsalmi{temporalia/ps104i-initium-d-g-auto.gtex}

\vspace{-1.5mm}

\input{temporalia/ps104i-d-g.tex} \Abardot{}

\vfill
\pagebreak

\pars{Psalmus 2.} \scriptura{Ps. 113, 1; \textbf{H94}}

\vspace{-4mm}

\antiphona{VIII a}{temporalia/ant-domusiacob.gtex}

%\vspace{-2mm}

\scriptura{Ps. 104, 16-27}

%\vspace{-2mm}

\initiumpsalmi{temporalia/ps104ii-initium-viii-a-auto.gtex}

\input{temporalia/ps104ii-viii-a.tex} \Abardot{}

\vfill
\pagebreak

\pars{Psalmus 3.} \scriptura{Ps. 104, 43}

\vspace{-4mm}

\antiphona{IV E}{temporalia/ant-eduxitdeus.gtex}

%\vspace{-2mm}

\scriptura{Ps. 104, 28-45}

%\vspace{-2mm}

\initiumpsalmi{temporalia/ps104iii-initium-iv-E-auto.gtex}

\input{temporalia/ps104iii-iv-E.tex}

\vfill

\antiphona{}{temporalia/ant-eduxitdeus.gtex}

\vfill
\pagebreak\fi
\ifx\matub\undefined
\else
% MAT B
\pars{Psalmus 1.} \scriptura{Ps. 105, 4; \textbf{H100}}

\vspace{-4mm}

\antiphona{E}{temporalia/ant-visitanos.gtex}

%\vspace{-2mm}

\scriptura{Ps. 105, 1-15}

%\vspace{-2mm}

\initiumpsalmi{temporalia/ps105i-initium-e.gtex}

\input{temporalia/ps105i-e.tex}

\vfill

\antiphona{}{temporalia/ant-visitanos.gtex}

\vfill
\pagebreak

\pars{Psalmus 2.} \scriptura{Ps. 117, 6; \textbf{H156}}

\vspace{-8mm}

\antiphona{VIII G}{temporalia/ant-dominusmihi.gtex}

\vspace{-3mm}

\scriptura{Ps. 105, 16-31}

\vspace{-2.5mm}

\initiumpsalmi{temporalia/ps105ii-initium-viii-G-auto.gtex}

\vspace{-1.5mm}

\input{temporalia/ps105ii-viii-G.tex} \Abardot{}

\vspace{-5mm}

\vfill
\pagebreak

\pars{Psalmus 3.} \scriptura{Ps. 105, 44}

\vspace{-4mm}

\antiphona{VII a}{temporalia/ant-cumtribularentur.gtex}

%\vspace{-2mm}

\scriptura{Ps. 105, 32-48}

%\vspace{-2mm}

\initiumpsalmi{temporalia/ps105iii-initium-vii-a-auto.gtex}

\input{temporalia/ps105iii-vii-a.tex}

\vfill

\antiphona{}{temporalia/ant-cumtribularentur.gtex}

\vfill
\pagebreak
\fi
\ifx\matuc\undefined
\else
% MAT C
\pars{Psalmus 1.} \scriptura{Ps. 106, 8}

\vspace{-4mm}

\antiphona{IV* e}{temporalia/ant-confiteanturdomino.gtex}

%\vspace{-2mm}

\scriptura{Ps. 106, 1-14}

%\vspace{-2mm}

\initiumpsalmi{temporalia/ps106i-initium-iv_-e-auto.gtex}

\input{temporalia/ps106i-iv_-e.tex} \Abardot{}

\vfill
\pagebreak

\pars{Psalmus 2.} \scriptura{Ps. 24, 17; \textbf{H100}}

\vspace{-4mm}

\antiphona{C}{temporalia/ant-denecessitatibus.gtex}

%\vspace{-2mm}

\scriptura{Ps. 106, 15-30}

%\vspace{-2mm}

\initiumpsalmi{temporalia/ps106ii-initium-c-c2-auto.gtex}

\input{temporalia/ps106ii-c-c2.tex}

\vfill

\antiphona{}{temporalia/ant-denecessitatibus.gtex}

\vfill
\pagebreak

\pars{Psalmus 3.} \scriptura{Ps. 106, 24}

\vspace{-4mm}

\antiphona{III a\textsuperscript{2}}{temporalia/ant-ipsividerunt.gtex}

%\vspace{-2mm}

\scriptura{Ps. 106, 31-43}

%\vspace{-2mm}

\initiumpsalmi{temporalia/ps106iii-initium-iii-a2-auto.gtex}

\input{temporalia/ps106iii-iii-a2.tex} \Abardot{}

\vfill
\pagebreak
\fi
\ifx\matud\undefined
\else
% MAT D
\pars{Psalmus 1.} \scriptura{1 Sam. 2, 10; \textbf{H96}}

\vspace{-4mm}

\antiphona{I g\textsuperscript{2}}{temporalia/ant-dominusjudicabit.gtex}

%\vspace{-2mm}

\scriptura{Ps. 49, 1-6}

%\vspace{-2mm}

\initiumpsalmi{temporalia/ps49i_vi-initium-i-g2-auto.gtex}

\input{temporalia/ps49i_vi-i-g2.tex} \Abardot{}

\vfill
\pagebreak

\pars{Psalmus 2.}

\vspace{-4mm}

\antiphona{VIII G}{temporalia/ant-attenditepopulemeus.gtex}

%\vspace{-2mm}

\scriptura{Ps. 49, 7-15}

%\vspace{-2mm}

\initiumpsalmi{temporalia/ps49vii_xv-initium-viii-G-auto.gtex}

\input{temporalia/ps49vii_xv-viii-G.tex} \Abardot{}

\vfill
\pagebreak

\pars{Psalmus 3.} \scriptura{Ps. 49, 14; \textbf{H94}}

\vspace{-4mm}

\antiphona{E}{temporalia/ant-immoladeo.gtex}

%\vspace{-2mm}

\scriptura{Ps. 49, 16-23}

%\vspace{-2mm}

\initiumpsalmi{temporalia/ps49xvi_xxiii-initium-e-auto.gtex}

\input{temporalia/ps49xvi_xxiii-e.tex} \Abardot{}

\vfill
\pagebreak
\fi
\else
\matutinum
\fi

\ifx\matversus\undefined
\pars{Versus} \scriptura{Mc. 1, 3; Is. 40, 3}

% Versus. %%%
\sineinitiali{temporalia/versus-voxclamantis-simplex.gtex}
\else
\matversus
\fi

\vspace{5mm}

\sineinitiali{temporalia/oratiodominica-mat.gtex}

\vspace{5mm}

\pars{Absolutio.}

\cuminitiali{}{temporalia/absolutio-avinculis.gtex}

\vfill
\pagebreak

\cuminitiali{}{temporalia/benedictio-solemn-ille.gtex}

\vspace{7mm}

\lectioi

\noindent \Vbardot{} Tu autem, Dómine, miserére nobis.
\noindent \Rbardot{} Deo grátias.

\vfill
\pagebreak

\responsoriumi

\vfill
\pagebreak

\cuminitiali{}{temporalia/benedictio-solemn-divinum.gtex}

\vspace{7mm}

\lectioii

\noindent \Vbardot{} Tu autem, Dómine, miserére nobis.
\noindent \Rbardot{} Deo grátias.

\vfill
\pagebreak

\responsoriumii

\vfill
\pagebreak

\cuminitiali{}{temporalia/benedictio-solemn-adsocietatem.gtex}

\vspace{7mm}

\lectioiii

\noindent \Vbardot{} Tu autem, Dómine, miserére nobis.
\noindent \Rbardot{} Deo grátias.

\vfill
\pagebreak

\responsoriumiii

\vfill
\pagebreak

\rubrica{Reliqua omittuntur, nisi Laudes separandæ sint.}

\sineinitiali{temporalia/domineexaudi.gtex}

\vfill

\oratio

\vfill

\noindent \Vbardot{} Dómine, exáudi oratiónem meam.

\noindent \Rbardot{} Et clamor meus ad te véniat.

\noindent \Vbardot{} Benedicámus Dómino, allelúia, allelúia.

\noindent \Rbardot{} Deo grátias, allelúia, allelúia.

\noindent \Vbardot{} Fidélium ánimæ per misericórdiam Dei requiéscant in pace.

\noindent \Rbardot{} Amen.

\vfill
\pagebreak

\hora{Ad Laudes.} %%%%%%%%%%%%%%%%%%%%%%%%%%%%%%%%%%%%%%%%%%%%%%%%%%%%%

\cantusSineNeumas

\vspace{0.5cm}
\ifx\deusinadiutorium\undefined
\grechangedim{interwordspacetext}{0.18 cm plus 0.15 cm minus 0.05 cm}{scalable}%
\cuminitiali{}{temporalia/deusinadiutorium-communis.gtex}
\grechangedim{interwordspacetext}{0.22 cm plus 0.15 cm minus 0.05 cm}{scalable}%
\else
\deusinadiutorium
\fi

\vfill
\pagebreak

\ifx\hymnuslaudes\undefined
\pars{Hymnus} \scriptura{Ambrosius (\olddag{} 397)}

\cuminitiali{I}{temporalia/hym-VoxClara-aromi.gtex}
\vspace{-3mm}
\else
\hymnuslaudes
\fi

\vfill
\pagebreak

\ifx\laudes\undefined
\ifx\lauda\undefined
\else
\pars{Psalmus 1.} \scriptura{Ps. 62, 2.3; \textbf{H142}}

\vspace{-4mm}

\antiphona{VII a}{temporalia/ant-adtedeluce.gtex}

\scriptura{Psalmus 118, 145-152; \hspace{5mm} \hebinitial{ק}}

\initiumpsalmi{temporalia/ps118xix-initium-vii-a-auto.gtex}

\input{temporalia/ps118xix-vii-a.tex} \Abardot{}

\vfill
\pagebreak

\pars{Psalmus 2.} \scriptura{Ex. 15, 1; \textbf{H98}}

\vspace{-4mm}

\antiphona{E}{temporalia/ant-cantemusdomino.gtex}

\scriptura{Canticum Moysis, Ex. 15, 1-19}

\initiumpsalmi{temporalia/moysis-initium-e-auto.gtex}

\input{temporalia/moysis-e.tex}

\antiphona{}{temporalia/ant-cantemusdomino.gtex}

\vfill
\pagebreak

\pars{Psalmus 3.} \scriptura{Ps. 116, 1; \textbf{H94}}

\vspace{-4mm}

\antiphona{E}{temporalia/ant-laudatedominumomnes.gtex}

\scriptura{Psalmus 116.}

\initiumpsalmi{temporalia/ps116-initium-e.gtex}

\input{temporalia/ps116-e.tex} \Abardot{}

\vfill
\pagebreak
\fi
\ifx\laudb\undefined
\else
\pars{Psalmus 1.} \scriptura{Ps. 91, 6}

\vspace{-4.5mm}

\antiphona{E}{temporalia/ant-quammagnificatasunt.gtex}

\vspace{-3mm}

\scriptura{Psalmus 91.}

\vspace{-2mm}

\initiumpsalmi{temporalia/ps91-initium-e.gtex}

\vspace{-1.5mm}

\input{temporalia/ps91-e.tex} \Abardot{}

\vfill
\pagebreak

\pars{Psalmus 2.} \scriptura{Dt. 32, 3}

%\vspace{-4mm}

\antiphona{VI F}{temporalia/ant-datemagnitudinem.gtex}

\vspace{-4mm}

\scriptura{Canticum Moysi, Dt. 32, 1-32}

\initiumpsalmi{temporalia/moysis2i_xii-initium-vi-F-auto.gtex}

\input{temporalia/moysis2i_xii-vi-F.tex}

\vfill

\antiphona{}{temporalia/ant-datemagnitudinem.gtex}

\vfill
\pagebreak

\pars{Psalmus 3.} \scriptura{Ps. 8, 2}

\vspace{-4mm}

\antiphona{I g}{temporalia/ant-quamadmirabileest.gtex}

%\vspace{-2mm}

\scriptura{Ps. 8}

%\vspace{-2mm}

\initiumpsalmi{temporalia/ps8-initium-i-g-auto.gtex}

\input{temporalia/ps8-i-g.tex} \Abardot{}

\vfill
\pagebreak
\fi
\ifx\laudc\undefined
\else
\pars{Psalmus 1.} \scriptura{Ps. 62, 7}

\vspace{-4mm}

\antiphona{E}{temporalia/ant-inmatutinis.gtex}

%\vspace{-2mm}

\scriptura{Psalmus 118, 145-152.}

%\vspace{-2mm}

\initiumpsalmi{temporalia/ps118xix-initium-e-auto.gtex}

%\vspace{-1.5mm}

\input{temporalia/ps118xix-e.tex} \Abardot{}

\vfill
\pagebreak

\pars{Psalmus 2.}

\vspace{-4mm}

\antiphona{V a}{temporalia/ant-mecumsitdomine.gtex}

%\vspace{-2mm}

\scriptura{Canticum Sapientiæ, Sap. 9, 1-6.9-11}

\initiumpsalmi{temporalia/sapientia-initium-v-a-auto.gtex}

\input{temporalia/sapientia-v-a.tex} \Abardot{}

\vfill
\pagebreak

\pars{Psalmus 3.}

\vspace{-4mm}

\antiphona{II* b}{temporalia/ant-veritasdomini.gtex}

%\vspace{-2mm}

\scriptura{Ps. 116}

%\vspace{-2mm}

\initiumpsalmi{temporalia/ps116-initium-ii_-B-auto.gtex}

\input{temporalia/ps116-ii_-B.tex} \Abardot{}

\vfill
\pagebreak
\fi
\ifx\laudd\undefined
\else
\pars{Psalmus 1.} \scriptura{Ps. 91, 2; \textbf{H99}}

\vspace{-4mm}

\antiphona{VIII G}{temporalia/ant-bonumestconfiteri.gtex}

%\vspace{-2mm}

\scriptura{Psalmus 91.}

%\vspace{-2mm}

\initiumpsalmi{temporalia/ps91-initium-viii-g-auto.gtex}

%\vspace{-1.5mm}

\input{temporalia/ps91-viii-g.tex}

\vfill

\antiphona{}{temporalia/ant-bonumestconfiteri.gtex}

\vfill
\pagebreak

\pars{Psalmus 2.}

\vspace{-4mm}

\antiphona{IV* e}{temporalia/ant-dabovobiscor.gtex}

%\vspace{-2mm}

\scriptura{Canticum Habacuc, Hab. 3, 2-19}

\initiumpsalmi{temporalia/habacuc-initium-iv_-e.gtex}

\input{temporalia/habacuc-iv_-e.tex}

\vfill

\antiphona{}{temporalia/ant-dabovobiscor.gtex}

\vfill
\pagebreak

\pars{Psalmus 3.}

\vspace{-4mm}

\antiphona{I f}{temporalia/ant-exoreinfantium.gtex}

%\vspace{-2mm}

\scriptura{Ps. 8}

%\vspace{-2mm}

\initiumpsalmi{temporalia/ps8-initium-i-f-auto.gtex}

\input{temporalia/ps8-i-f.tex} \Abardot{}

\vfill
\pagebreak
\fi
\else
\laudes
\fi

\ifx\lectiobrevis\undefined
\pars{Lectio Brevis.} \scriptura{Is. 11, 1-3}

\noindent Egrediétur virga de stirpe Iesse, et flos de radíce eius ascéndet; et requiéscet super eum spíritus Dómini: spíritus sapiéntiæ et intelléctus, spíritus consílii et fortitúdinis, spíritus sciéntiæ et timóris Dómini; et delíciæ eius in timóre Dómini.
\else
\lectiobrevis
\fi

\vfill

\ifx\responsoriumbreve\undefined
\pars{Responsorium breve.} \scriptura{Is. 60, 2; \textbf{H20}}

\cuminitiali{IV}{temporalia/resp-superte.gtex}
\else
\responsoriumbreve
\fi

\vfill
\pagebreak

\benedictus

\vfill
\pagebreak

\pars{Preces.}

\sineinitiali{}{temporalia/tonusprecum.gtex}

\ifx\preces\undefined
\noindent Deum Patrem, qui antíqua dispositióne pópulum suum salváre státuit, \gredagger{} orémus dicéntes:

\Rbardot{} Custódi plebem tuam, Dómine.

\noindent Deus, qui pópulo tuo germen iustítiæ promisísti, \gredagger{} custódi sanctitátem Ecclésiæ tuæ.

\Rbardot{} Custódi plebem tuam, Dómine.

\noindent Inclína cor hóminum, Deus, in verbum tuum \gredagger{} et confírma fidéles tuos sine queréla in sanctitáte.

\Rbardot{} Custódi plebem tuam, Dómine.

\noindent Consérva nos in dilectióne Spíritus tui, \gredagger{} ut Fílii tui, qui ventúrus est, misericórdiam suscipiámus.

\Rbardot{} Custódi plebem tuam, Dómine.

\noindent Confírma nos, Deus clementíssime, usque in finem, \gredagger{} in diem advéntus Dómini Iesu Christi.

\Rbardot{} Custódi plebem tuam, Dómine.
\else
\preces
\fi

\vfill

\pars{Oratio Dominica.}

\cuminitiali{}{temporalia/oratiodominicaalt.gtex}

\vfill
\pagebreak

\rubrica{vel:}

\pars{Supplicatio Litaniæ.}

\cuminitiali{}{temporalia/supplicatiolitaniae.gtex}

\vfill

\pars{Oratio Dominica.}

\cuminitiali{}{temporalia/oratiodominica.gtex}

\vfill
\pagebreak

% Oratio. %%%
\oratio

\vspace{-1mm}

\vfill

\rubrica{Hebdomadarius dicit Dominus vobiscum, vel, absente sacerdote vel diacono, sic concluditur:}

\vspace{2mm}

\antiphona{C}{temporalia/dominusnosbenedicat.gtex}

\rubrica{Postea cantatur a cantore:}

\vspace{2mm}

\ifx\benedicamuslaudes\undefined
\cuminitiali{IV}{temporalia/benedicamus-feria-advequad.gtex}
\else
\benedicamuslaudes
\fi

\vfill

\vspace{1mm}

\end{document}

