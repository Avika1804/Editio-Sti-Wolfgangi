\newcommand{\titulus}{\nomenFesti{Ss. Apostolorum Petri et Pauli.}
\dies{Die 29. Iunii.}}
\newcommand{\sineobmv}{Sine officium B.M.V.}
\newcommand{\oratio}{\pars{Oratio.}

\noindent Deus, qui huius diéi venerándam sanctámque lætítiam in apostolórum Petri et Pauli sollemnitáte tribuísti, da Ecclésiæ tuæ eórum in ómnibus sequi præcéptum, per quos religiónis sumpsit exórdium.

\pars{Pro pace in universo mundo.} \scriptura{Sir. 50, 25; 2 Esdr. 4, 20; \textbf{H416}}

\vspace{-4mm}

\antiphona{II D}{temporalia/ant-dapacemdomine.gtex}

\vfill

\noindent Deus, a quo sancta desidéria, recta consília et iusta sunt ópera: da servis tuis illam, quam mundus dare non potest, pacem; ut et corda nostra mandátis tuis dédita, et hóstium subláta formídine, témpora sint tua protectióne tranquílla.

\noindent Per Dóminum nostrum Iesum Christum, Fílium tuum, qui tecum vivit et regnat in unitáte Spíritus Sancti, Deus, per ómnia sǽcula sæculórum.

\noindent \Rbardot{} Amen.}
\newcommand{\invitatorium}{\pars{Invitatorium.} \scriptura{Cantor; \textbf{H443}}

\vspace{-4mm}

\antiphona{III}{temporalia/inv-regemapostolorum.gtex}}
\newcommand{\hymnusmatutinum}{\pars{Hymnus}

\cuminitiali{IV}{temporalia/hym-FelixPerOmnes.gtex}}
\newcommand{\matutinum}{\subhora{In I. Nocturno}

\pars{Psalmus 1.} \scriptura{Io. 21, 15; \textbf{H289}}

\vspace{-4mm}

\antiphona{VIII G}{temporalia/ant-simonioannisdiligis.gtex}

%\vspace{-2mm}

\scriptura{Ps. 18, 1-7}

%\vspace{-2mm}

\initiumpsalmi{temporalia/ps18i-initium-viii-g-auto.gtex}

\input{temporalia/ps18i-viii-g.tex} \Abardot{}

\vfill
\pagebreak

\pars{Psalmus 2.} \scriptura{Phil. 1, 1, 21; Gal. 6, 14; \textbf{H284}}

\vspace{-4mm}

\antiphona{I g\textsuperscript{2}}{temporalia/ant-mihivivere.gtex}

%\vspace{-2mm}

\scriptura{Ps. 63}

\initiumpsalmi{temporalia/ps63-initium-i-g6-auto.gtex}

\input{temporalia/ps63-i-g6.tex} \Abardot{}

\vfill
\pagebreak

\pars{Psalmus 3.} \scriptura{Io. 21, 19; \textbf{H289}}

\vspace{-4mm}

\antiphona{I g}{temporalia/ant-significavit.gtex}

%\vspace{-2mm}

\scriptura{Ps. 96}

%\vspace{-2mm}

\initiumpsalmi{temporalia/ps96-initium-i-g-auto.gtex}

\input{temporalia/ps96-i-g.tex} \Abardot{}

\vfill
\pagebreak}
\newcommand{\matversus}{\noindent \Vbardot{} In omnem terram exívit sonus eórum.

\noindent \Rbardot{} Et in fines orbis terræ verba eórum.}
\newcommand{\lectioi}{\pars{Lectio I.} \scriptura{Gal. 1, 15-2,10}

\noindent De Epístola beáti Pauli apóstoli ad Gálatas.

\noindent Fratres: Cum plácuit Deo, qui me segregávit de útero matris meæ et vocávit per grátiam suam, ut reveláret Fílium suum in me, ut evangelizárem illum in géntibus, contínuo non cóntuli cum carne et sánguine neque ascéndi Hierosólymam ad antecessóres meos apóstolos, sed ábii in Arábiam et íterum revérsus sum Damáscum.

\noindent Deínde post annos tres ascéndi Hierosólymam vidére Cepham et mansi apud eum diébus quíndecim; álium autem apostolórum non vidi nisi Iacóbum fratrem Dómini. Quæ autem scribo vobis, ecce coram Deo quia non méntior. Deínde veni in partes Sýriæ et Cilíciæ. Eram autem ignótus fácie ecclésiis Iudǽæ, quæ sunt in Christo, tantum autem audítum habébant: «Qui persequebátur nos aliquándo, nunc evangelízat fidem, quam aliquándo expugnábat», et in me glorificábant Deum.

\noindent Deínde post annos quattuórdecim íterum ascéndi Hierosólymam cum Bárnaba, assúmpto et Tito; ascéndi autem secúndum revelatiónem; et cóntuli cum illis evangélium, quod prǽdico in géntibus, seórsum autem his, qui observabántur, ne forte in vácuum cúrrerem aut cucurríssem. Sed neque Titus, qui mecum erat, cum esset Græcus, compúlsus est circumcídi. Sed propter subintrodúctos falsos fratres, qui subintroiérunt exploráre libertátem nostram, quam habémus in Christo Iesu, ut nos in servitútem redígerent; quibus neque ad horam céssimus subiciéntes nos, ut véritas evangélii permáneat apud vos.

\noindent Ab his autem, qui videbántur esse áliquid —quales aliquándo fúerint, nihil mea ínterest; Deus persónam hóminis non áccipit— mihi enim, qui observabántur, nihil contulérunt, sed e contra, cum vidíssent quod créditum est mihi evangélium præpútii, sicut Petro circumcisiónis —qui enim operátus est Petro in apostolátum circumcisiónis operátus est et mihi inter gentes— et cum cognovíssent grátiam, quæ data est mihi, Iacóbus et Cephas et Ioánnes, qui videbántur colúmnæ esse, déxteras dedérunt mihi et Bárnabæ communiónis, ut nos in gentes, ipsi autem in circumcisiónem; tantum ut páuperum mémores essémus, quod étiam sollícitus fui hoc ipsum fácere.}
\newcommand{\responsoriumi}{\pars{Responsorium 1.} \scriptura{\textbf{H287}}

\vspace{-5mm}

\responsorium{V}{temporalia/resp-sanctepauleapostole-CROCHU.gtex}{}}
\newcommand{\lectioii}{\pars{Lectio II.} \scriptura{Sermo 295, 1-2. 4. 7-8: PL 38, 1348-1352}

\noindent Ex Sermónibus sancti Augustíni epíscopi.

\noindent Istum nobis diem beatissimórum apostolórum Petri et Pauli pássio consecrávit. Non de obscúris alíquibus martýribus lóquimur. \emph{In omnem terram éxiit sonus eórum, et in fines orbis terræ verba eórum.} Isti mártyres vidérunt quod prædicavérunt, secúti æquitátem, confiténdo veritátem, moriéndo pro veritáte.

\noindent Beátus Petrus, primus Apostolórum, véhemens Christi amátor, qui méruit audíre: \emph{Et ego dico tibi quia tu es Petrus.} Díxerat enim ipse: \emph{Tu es Christus Fílius Dei vivi.} Christus illi: \emph{Et ego dico tibi quia tu es Petrus, et super hanc petram ædificábo Ecclésiam meam.} Super hanc petram ædificábo fidem, quam confitéris. Super hoc quod dixísti: \emph{Tu es Christus Fílius Dei vivi}, ædificábo Ecclésiam meam. Tu enim Petrus. A petra Petrus, non a Petro petra. Sic a petra Petrus, quómodo a Christo christiánus.

\noindent Dóminus Iesus discípulos suos ante passiónem suam, sicut nostis, elégit, quos Apóstolos appellávit. Inter hos pæne ubíque solus Petrus totíus Ecclésiæ méruit gestáre persónam. Propter ipsam persónam, quam totíus Ecclésiæ solus gestábat, audíre méruit: \emph{Tibi dabo claves regni cælórum.} Has enim claves non homo unus, sed únitas accépit Ecclésiæ. Hinc ergo Petri excelléntia prædicátur, quia ipsíus universitátis et unitátis Ecclésiæ figúram gessit, quando ei dictum est: \emph{Tibi trado,} quod ómnibus tráditum est. Nam ut novéritis Ecclésiam accepísse claves regni cælórum, audíte in álio loco quid Dóminus dicat ómnibus Apóstolis suis: \emph{Accípite Spíritum sanctum.} Et contínuo: \emph{Si cui dimiséritis peccáta, dimitténtur ei; si cuius tenuéritis, tenebúntur.}}
\newcommand{\responsoriumii}{\pars{Responsorium 2.} \scriptura{\Rbardot{} Io. 21, 15.17 \Vbardot{} Mt. 26, 35; \textbf{H280}}

\vspace{-5mm}

\responsorium{VI}{temporalia/resp-sidiligisme-CROCHU.gtex}{}

\vfill

\rubrica{vel ad libitum:}

\vspace{3mm}

\pars{Responsorium 2.} \scriptura{\Rbardot{} Lc. 22, 32 \Vbardot{} Mt. 16, 17; \textbf{H282}}

\vspace{-5mm}

\responsorium{III}{temporalia/resp-egoproterogavi-CROCHU.gtex}{}}
\newcommand{\lectioiii}{\pars{Lectio III.}

\noindent Mérito étiam post resurrectiónem Dóminus ipsi Petro oves suas commendávit pascéndas. Non enim inter discípulos solus méruit páscere domínicas oves; sed quando Christus ad unum lóquitur, únitas commendátur; et Petro prímitus, quia in Apóstolis Petrus est primus. Noli tristis esse, Apóstole; respónde semel, respónde íterum, respónde tértio. Ter vincat in amóre conféssio, quia ter victa est in timóre præsúmptio. Solvéndum est ter, quod ligáveras ter. Solve per amórem, quod ligáveras per timórem. Et tamen Dóminus semel et íterum et tértio oves suas commendávit Petro.

\noindent Unus dies passiónis duóbus apóstolis. Sed et illi duo unum erant; quamquam divérsis diébus pateréntur, unum erant. Præcéssit Petrus, secútus est Paulus. Celebrámus diem festum apostolórum nobis sánguine consecrátum. Amémus fidem, vitam, labóres, passiónes, confessiónes, prædicatiónes.}
\newcommand{\responsoriumiii}{\pars{Responsorium 3.} \scriptura{\Rbardot{} Mt. 16, 18.19 \Vbardot{} ibid; \textbf{H280}}

\vspace{-5mm}

\responsorium{VII}{temporalia/resp-tuespetrus-CROCHU.gtex}{}

\vfill
\pagebreak

\subhora{In II. Nocturno}

\pars{Cantica.}

\vspace{-4mm}

\antiphona{III a}{temporalia/ant-inseparabilisfides.gtex}

%\vspace{-2mm}

\scriptura{Canticum Isaiaæ, Is. 61, 6-9}

\vspace{-3mm}

\initiumpsalmi{temporalia/isaiae16-initium-iii-a-auto.gtex}

\input{temporalia/isaiae16-iii-a.tex} \hfill \rubrica{Hic non dicitur antiphona.}

\vfill
\pagebreak

\scriptura{Canticum Sapientiæ, Sap. 3, 7-9}

%\vspace{-3mm}

\initiumpsalmi{temporalia/sapientia3-initium-iii-a.gtex}

\input{temporalia/sapientia3-iii-a.tex}

\vfill
\pagebreak

\scriptura{Canticum Sapientiæ, Sap. 10, 17-21}

%\vspace{-2mm}

\initiumpsalmi{temporalia/sapientia4-initium-iii-a-auto.gtex}

\input{temporalia/sapientia4-iii-a.tex}

\vfill

\antiphona{}{temporalia/ant-inseparabilisfides.gtex}

\vfill
\pagebreak

\pars{Versus.}

\noindent \Vbardot{} Constítues eos príncipes super omnem terram.

\noindent \Rbardot{} Mémores erunt nóminis tui, Dómine.

\vspace{5mm}

\sineinitiali{temporalia/oratiodominica-mat.gtex}

\vspace{5mm}

\pars{Absolutio.}

\cuminitiali{}{temporalia/absolutio-ipsius.gtex}

\vfill
\pagebreak

\cuminitiali{}{temporalia/benedictio-solemn-evangelica.gtex}

\vspace{7mm}

\pars{Evangelium} \scriptura{Mt. 16, 13-19}

\noindent Léctio sancti Evangélii secúndum Matthǽum.

\noindent In illo témpore:

\noindent Venit Iesus in partes Cæsaréæ Philíppi et interrogábat discípulos suos dicens: «Quem dicunt hómines esse Fílium hóminis?».

\noindent At illi dixérunt: «Alii Ioánnem Baptístam, álii autem Elíam, álii vero Ieremíam, aut unum ex prophétis».

\noindent Dicit illis: «Vos autem quem me esse dícitis?».

\noindent Respóndens Simon Petrus dixit: «Tu es Christus, Fílius Dei vivi».

\noindent Respóndens autem Iesus dixit ei: «Beátus es, Simon Barióna, quia caro et sanguis non revelávit tibi sed Pater meus, qui in cælis est. Et ego dico tibi: Tu es Petrus, et super hanc petram ædificábo Ecclésiam meam; et portæ ínferi non prævalébunt advérsum eam. Tibi dabo claves regni cælórum; et quodcúmque ligáveris super terram, erit ligátum in cælis, et quodcúmque sólveris super terram, erit solútum in cælis».

\vfill
\pagebreak

\pars{Responsorium 4.} \scriptura{\Rbardot{} Mt. 16, 13.16.18 \Vbardot{} ibid. 16, 17; \textbf{H282}}

\vspace{-5mm}

\responsorium{I}{temporalia/resp-quemdicunthomines-CROCHU.gtex}{}

\vfill
\pagebreak

{
\pars{Hymnus Ambrosianus} \scriptura{Tonus Solemnis}

\vspace{-2mm}

\grechangedim{interwordspacetext}{0.26 cm plus 0.15 cm minus 0.05 cm}{scalable}%
\cuminitiali{III}{temporalia/tedeum-solemnis-gn.gtex}
\grechangedim{interwordspacetext}{0.22 cm plus 0.15 cm minus 0.05 cm}{scalable}%
}}
\newcommand{\deusinadiutorium}{\grechangedim{interwordspacetext}{0.18 cm plus 0.15 cm minus 0.05 cm}{scalable}%
\cuminitiali{}{temporalia/deusinadiutorium-alter.gtex}}
\newcommand{\hymnuslaudes}{\pars{Hymnus}

\grechangedim{interwordspacetext}{0.16 cm plus 0.15 cm minus 0.05 cm}{scalable}%
\cuminitiali{IV}{temporalia/hym-ApostolorumPassio.gtex}
\grechangedim{interwordspacetext}{0.22 cm plus 0.15 cm minus 0.05 cm}{scalable}}
\newcommand{\laudes}{\pars{Psalmus 1.} \scriptura{2 Tim. 1, 12; 2 Tim. 4, 8; \textbf{H284}}

\vspace{-4mm}

\antiphona{I a}{temporalia/ant-sciocuicredidiet.gtex}

\scriptura{Ps. 62}

\initiumpsalmi{temporalia/ps62-initium-i-a-auto.gtex}

\input{temporalia/ps62-i-a.tex} \Abardot{}

\vfill
\pagebreak

\pars{Psalmus 2.} \scriptura{1 Cor. 15, 10; \textbf{H288}}

\vspace{-4mm}

\antiphona{II* a}{temporalia/ant-gratiadeiinme.gtex}

%\vspace{-2mm}

\scriptura{Canticum trium puerorum, Dan. 3, 57-88 et 56}

\vspace{-2mm}

\initiumpsalmi{temporalia/dan3-initium-ii_-a-auto.gtex}

\input{temporalia/dan3-ii_-a-sinedox.tex}

\rubrica{Hic non dicitur Gloria Patri, neque Amen.}
\vspace{1cm}

\antiphona{}{temporalia/ant-gratiadeiinme.gtex} % repeat the antiphon - new page

\vfill
\pagebreak

\pars{Psalmus 3.} \scriptura{2 Tim. 4, 7; \textbf{H285}}

\vspace{-4mm}

\antiphona{I g}{temporalia/ant-bonumcertamen.gtex}

%\vspace{-2mm}

\scriptura{Ps. 149}

\initiumpsalmi{temporalia/ps149-initium-i-g-auto.gtex}

\input{temporalia/ps149-i-g.tex} \Abardot{}

\vfill
\pagebreak}
\newcommand{\lectiobrevis}{\pars{Lectio Brevis.} \scriptura{1 Pet. 4, 13-14}

\noindent Sed, quemádmodum communicátis Christi passiónibus, gaudéte, ut et in revelatióne glóriæ eius gaudeátis exsultántes. Si exprobrámini in nómine Christi, beáti, quóniam Spíritus glóriæ et Dei super vos requiéscit.}
\newcommand{\responsoriumbreve}{\pars{Responsorium breve.} \scriptura{Ps. 18, 5}

\cuminitiali{VI}{temporalia/resp-inomnemterram.gtex}}
\newcommand{\benedictus}{\pars{Canticum Zachariæ.} \scriptura{\textbf{H284}}

\vspace{-4mm}

{
\grechangedim{interwordspacetext}{0.18 cm plus 0.15 cm minus 0.05 cm}{scalable}%
\antiphona{VI F}{temporalia/ant-gloriosiprincipes.gtex}
\grechangedim{interwordspacetext}{0.22 cm plus 0.15 cm minus 0.05 cm}{scalable}%
}

%\vspace{-2mm}

\scriptura{Lc. 1, 68-79}

\vspace{-1mm}

\cantusSineNeumas
\initiumpsalmi{temporalia/benedictus-initium-visoll-F-auto.gtex}

%\vspace{-1.5mm}

\input{temporalia/benedictus-visoll-F.tex} \Abardot{}}
\newcommand{\preces}{\noindent Christum, qui Ecclésiam suam super fundaméntum Apostolórum et prophetárum ædificávit,~\grestar{} fidénter deprecémur:

\Rbardot{} Bénefac, Dómine, Ecclésiæ tuæ.

\noindent Tu, qui pro Petro rogásti, ne fides eius defíceret,~\grestar{} confírma fidem Ecclésiæ tuæ.

\Rbardot{} Bénefac, Dómine, Ecclésiæ tuæ.

\noindent Tu, qui post resurrectiónem tuam Simóni Petro apparuísti~\gredagger{} et Saulo te revelásti,~\grestar{} illúmina mentes nostras, ut te vivéntem confiteámur.

\Rbardot{} Bénefac, Dómine, Ecclésiæ tuæ.

\noindent Tu, qui Paulum apóstolum elegísti,~\gredagger{} ut nomen tuum géntibus annuntiáret,~\grestar{} fac nos veros præcónes Evangélii tui.

\Rbardot{} Bénefac, Dómine, Ecclésiæ tuæ.

\noindent Tu, qui negatiónibus Petri misericórditer ignovísti,~\grestar{} dimítte nobis ómnia nostra débita.

\Rbardot{} Bénefac, Dómine, Ecclésiæ tuæ.}
\newcommand{\benedicamuslaudes}{\cuminitiali{II}{temporalia/benedicamus-solemnism-laud.gtex}}
\newcommand{\hebdomada}{infra Hebdom. XII per Annum.}
\newcommand{\matud}{Matutinum Hebdomadae D}
\newcommand{\matubd}{Matutinum Hebdomadae B vel D}
\newcommand{\laudd}{Laudes Hebdomadae D}
\newcommand{\laudbd}{Laudes Hebdomadae B vel D}

% LuaLaTeX

\documentclass[a4paper, twoside, 12pt]{article}
\usepackage[latin]{babel}
%\usepackage[landscape, left=3cm, right=1.5cm, top=2cm, bottom=1cm]{geometry} % okraje stranky
%\usepackage[landscape, a4paper, mag=1166, truedimen, left=2cm, right=1.5cm, top=1.6cm, bottom=0.95cm]{geometry} % okraje stranky
\usepackage[landscape, a4paper, mag=1400, truedimen, left=0.5cm, right=0.5cm, top=0.5cm, bottom=0.5cm]{geometry} % okraje stranky

\usepackage{fontspec}
\setmainfont[FeatureFile={junicode.fea}, Ligatures={Common, TeX}, RawFeature=+fixi]{Junicode}
%\setmainfont{Junicode}

% shortcut for Junicode without ligatures (for the Czech texts)
\newfontfamily\nlfont[FeatureFile={junicode.fea}, Ligatures={Common, TeX}, RawFeature=+fixi]{Junicode}

% Hebrew font:
% http://scripts.sil.org/cms/scripts/page.php?site_id=nrsi&id=SILHebrUnic2
\newfontfamily\hebfont[Scale=1]{Ezra SIL}

\usepackage{multicol}
\usepackage{color}
\usepackage{lettrine}
\usepackage{fancyhdr}

% usual packages loading:
\usepackage{luatextra}
\usepackage{graphicx} % support the \includegraphics command and options
\usepackage{gregoriotex} % for gregorio score inclusion
\usepackage{gregoriosyms}
\usepackage{wrapfig} % figures wrapped by the text
\usepackage{parcolumns}
\usepackage[contents={},opacity=1,scale=1,color=black]{background}
\usepackage{tikzpagenodes}
\usepackage{calc}
\usepackage{longtable}
\usetikzlibrary{calc}

\setlength{\headheight}{14.5pt}

% Commands used to produce a typical "Conventus" booklet

\newenvironment{titulusOfficii}{\begin{center}}{\end{center}}
\newcommand{\dies}[1]{#1

}
\newcommand{\nomenFesti}[1]{\textbf{\Large #1}

}
\newcommand{\celebratio}[1]{#1

}

\newcommand{\hora}[1]{%
\vspace{0.5cm}{\large \textbf{#1}}

\fancyhead[LE]{\thepage\ / #1}
\fancyhead[RO]{#1 / \thepage}
\addcontentsline{toc}{subsection}{#1}
}

% larger unit than a hora
\newcommand{\divisio}[1]{%
\begin{center}
{\Large \textsc{#1}}
\end{center}
\fancyhead[CO,CE]{#1}
\addcontentsline{toc}{section}{#1}
}

% a part of a hora, larger than pars
\newcommand{\subhora}[1]{
\begin{center}
{\large \textit{#1}}
\end{center}
%\fancyhead[CO,CE]{#1}
\addcontentsline{toc}{subsubsection}{#1}
}

% rubricated inline text
\newcommand{\rubricatum}[1]{\textit{#1}}

% standalone rubric
\newcommand{\rubrica}[1]{\vspace{3mm}\rubricatum{#1}}

\newcommand{\notitia}[1]{\textcolor{red}{#1}}

\newcommand{\scriptura}[1]{\hfill \small\textit{#1}}

\newcommand{\translatioCantus}[1]{\vspace{1mm}%
{\noindent\footnotesize \nlfont{#1}}}

% pruznejsi varianta nasledujiciho - umoznuje nastavit sirku sloupce
% s prekladem
\newcommand{\psalmusEtTranslatioB}[3]{
  \vspace{0.5cm}
  \begin{parcolumns}[colwidths={2=#3}, nofirstindent=true]{2}
    \colchunk{
      \input{#1}
    }

    \colchunk{
      \vspace{-0.5cm}
      {\footnotesize \nlfont
        \input{#2}
      }
    }
  \end{parcolumns}
}

\newcommand{\psalmusEtTranslatio}[2]{
  \psalmusEtTranslatioB{#1}{#2}{8.5cm}
}


\newcommand{\canticumMagnificatEtTranslatio}[1]{
  \psalmusEtTranslatioB{#1}{temporalia/extra-adventum-vespers/magnificat-boh.tex}{12cm}
}
\newcommand{\canticumBenedictusEtTranslatio}[1]{
  \psalmusEtTranslatioB{#1}{temporalia/extra-adventum-laudes/benedictus-boh.tex}{10.5cm}
}

% volne misto nad antifonami, kam si zpevaci dokresli neumy
\newcommand{\hicSuntNeumae}{\vspace{0.5cm}}

% prepinani mista mezi notovymi osnovami: pro neumovane a neneumovane zpevy
\newcommand{\cantusCumNeumis}{
  \setgrefactor{17}
  \global\advance\grespaceabovelines by 5mm%
}
\newcommand{\cantusSineNeumas}{
  \setgrefactor{17}
  \global\advance\grespaceabovelines by -5mm%
}

% znaky k umisteni nad inicialu zpevu
\newcommand{\superInitialam}[1]{\gresetfirstlineaboveinitial{\small {\textbf{#1}}}{\small {\textbf{#1}}}}

% pars officii, i.e. "oratio", ...
\newcommand{\pars}[1]{\textbf{#1}}

\newenvironment{psalmus}{
  \setlength{\parindent}{0pt}
  \setlength{\parskip}{5pt}
}{
  \setlength{\parindent}{10pt}
  \setlength{\parskip}{10pt}
}

%%%% Prejmenovat na latinske:
\newcommand{\nadpisZalmu}[1]{
  \hspace{2cm}\textbf{#1}\vspace{2mm}%
  \nopagebreak%

}

% mode, score, translation
\newcommand{\antiphona}[3]{%
\hicSuntNeumae
\superInitialam{#1}
\includescore{#2}

#3
}
 % Often used macros

\newcommand{\annusEditionis}{2021}

\def\hebinitial#1{%
\leavevmode{\newbox\hebbox\setbox\hebbox\hbox{\hebfont{#1}\hskip 1mm}\kern -\wd\hebbox\hbox{\hebfont{#1}\hskip 1mm}}%
}

%%%% Vicekrat opakovane kousky

\newcommand{\anteOrationem}{
  \rubrica{Ante Orationem, cantatur a Superiore:}

  \pars{Supplicatio Litaniæ.}

  \cuminitiali{}{temporalia/supplicatiolitaniae.gtex}

  \pars{Oratio Dominica.}

  \cuminitiali{}{temporalia/oratiodominica.gtex}
}

\setlength{\columnsep}{30pt} % prostor mezi sloupci

%%%%%%%%%%%%%%%%%%%%%%%%%%%%%%%%%%%%%%%%%%%%%%%%%%%%%%%%%%%%%%%%%%%%%%%%%%%%%%%%%%%%%%%%%%%%%%%%%%%%%%%%%%%%%
\begin{document}

% Here we set the space around the initial.
% Please report to http://home.gna.org/gregorio/gregoriotex/details for more details and options
\grechangedim{afterinitialshift}{2.2mm}{scalable}
\grechangedim{beforeinitialshift}{2.2mm}{scalable}

\grechangedim{interwordspacetext}{0.22 cm plus 0.15 cm minus 0.05 cm}{scalable}%
\grechangedim{annotationraise}{-0.2cm}{scalable}

% Here we set the initial font. Change 38 if you want a bigger initial.
% Emit the initials in red.
\grechangestyle{initial}{\color{red}\fontsize{38}{38}\selectfont}

\pagestyle{empty}

%%%% Titulni stranka
\begin{titulusOfficii}
\ifx\titulus\undefined
\nomenFesti{Sabbato \hebdomada{}}
\else
\titulus
\fi
\end{titulusOfficii}

\vfill

\pars{}

\scriptura{}

\pagebreak

% graphic
\renewcommand{\headrulewidth}{0pt} % no horiz. rule at the header
\fancyhf{}
\pagestyle{fancy}

\cantusSineNeumas

\hora{Ad Matutinum.}

\vspace{2mm}

\cuminitiali{}{temporalia/dominelabiamea.gtex}

\vspace{2mm}

\ifx\invitatorium\undefined
\pars{Invitatorium.} \scriptura{\textbf{H14}}

\vspace{-6mm}

\antiphona{VI}{temporalia/inv-regemventurumsimplex.gtex}
\else
\invitatorium
\fi

\vfill
\pagebreak

\ifx\hymnusmatutinum\undefined
\pars{Hymnus.}

\vspace{-5mm}

\antiphona{II}{temporalia/hym-VerbumSupernum.gtex}
\else
\hymnusmatutinum
\fi

\vfill
\pagebreak

\ifx\matutinum\undefined
\ifx\matua\undefined
\else
% MAT A
\pars{Psalmus 1.} \scriptura{Ps. 104, 3; \textbf{H99}}

\vspace{-6mm}

\antiphona{D}{temporalia/ant-laeteturcor.gtex}

\vspace{-4mm}

\scriptura{Ps. 104, 1-15}

\vspace{-2mm}

\initiumpsalmi{temporalia/ps104i-initium-d-g-auto.gtex}

\vspace{-1.5mm}

\input{temporalia/ps104i-d-g.tex} \Abardot{}

\vfill
\pagebreak

\pars{Psalmus 2.} \scriptura{Ps. 113, 1; \textbf{H94}}

\vspace{-4mm}

\antiphona{VIII a}{temporalia/ant-domusiacob.gtex}

%\vspace{-2mm}

\scriptura{Ps. 104, 16-27}

%\vspace{-2mm}

\initiumpsalmi{temporalia/ps104ii-initium-viii-a-auto.gtex}

\input{temporalia/ps104ii-viii-a.tex} \Abardot{}

\vfill
\pagebreak

\pars{Psalmus 3.} \scriptura{Ps. 104, 43}

\vspace{-4mm}

\antiphona{IV E}{temporalia/ant-eduxitdeus.gtex}

%\vspace{-2mm}

\scriptura{Ps. 104, 28-45}

%\vspace{-2mm}

\initiumpsalmi{temporalia/ps104iii-initium-iv-E-auto.gtex}

\input{temporalia/ps104iii-iv-E.tex}

\vfill

\antiphona{}{temporalia/ant-eduxitdeus.gtex}

\vfill
\pagebreak\fi
\ifx\matub\undefined
\else
% MAT B
\pars{Psalmus 1.} \scriptura{Ps. 105, 4; \textbf{H100}}

\vspace{-4mm}

\antiphona{E}{temporalia/ant-visitanos.gtex}

%\vspace{-2mm}

\scriptura{Ps. 105, 1-15}

%\vspace{-2mm}

\initiumpsalmi{temporalia/ps105i-initium-e.gtex}

\input{temporalia/ps105i-e.tex}

\vfill

\antiphona{}{temporalia/ant-visitanos.gtex}

\vfill
\pagebreak

\pars{Psalmus 2.} \scriptura{Ps. 117, 6; \textbf{H156}}

\vspace{-8mm}

\antiphona{VIII G}{temporalia/ant-dominusmihi.gtex}

\vspace{-3mm}

\scriptura{Ps. 105, 16-31}

\vspace{-2.5mm}

\initiumpsalmi{temporalia/ps105ii-initium-viii-G-auto.gtex}

\vspace{-1.5mm}

\input{temporalia/ps105ii-viii-G.tex} \Abardot{}

\vspace{-5mm}

\vfill
\pagebreak

\pars{Psalmus 3.} \scriptura{Ps. 105, 44}

\vspace{-4mm}

\antiphona{VII a}{temporalia/ant-cumtribularentur.gtex}

%\vspace{-2mm}

\scriptura{Ps. 105, 32-48}

%\vspace{-2mm}

\initiumpsalmi{temporalia/ps105iii-initium-vii-a-auto.gtex}

\input{temporalia/ps105iii-vii-a.tex}

\vfill

\antiphona{}{temporalia/ant-cumtribularentur.gtex}

\vfill
\pagebreak
\fi
\ifx\matuc\undefined
\else
% MAT C
\pars{Psalmus 1.} \scriptura{Ps. 106, 8}

\vspace{-4mm}

\antiphona{IV* e}{temporalia/ant-confiteanturdomino.gtex}

%\vspace{-2mm}

\scriptura{Ps. 106, 1-14}

%\vspace{-2mm}

\initiumpsalmi{temporalia/ps106i-initium-iv_-e-auto.gtex}

\input{temporalia/ps106i-iv_-e.tex} \Abardot{}

\vfill
\pagebreak

\pars{Psalmus 2.} \scriptura{Ps. 24, 17; \textbf{H100}}

\vspace{-4mm}

\antiphona{C}{temporalia/ant-denecessitatibus.gtex}

%\vspace{-2mm}

\scriptura{Ps. 106, 15-30}

%\vspace{-2mm}

\initiumpsalmi{temporalia/ps106ii-initium-c-c2-auto.gtex}

\input{temporalia/ps106ii-c-c2.tex}

\vfill

\antiphona{}{temporalia/ant-denecessitatibus.gtex}

\vfill
\pagebreak

\pars{Psalmus 3.} \scriptura{Ps. 106, 24}

\vspace{-4mm}

\antiphona{III a\textsuperscript{2}}{temporalia/ant-ipsividerunt.gtex}

%\vspace{-2mm}

\scriptura{Ps. 106, 31-43}

%\vspace{-2mm}

\initiumpsalmi{temporalia/ps106iii-initium-iii-a2-auto.gtex}

\input{temporalia/ps106iii-iii-a2.tex} \Abardot{}

\vfill
\pagebreak
\fi
\ifx\matud\undefined
\else
% MAT D
\pars{Psalmus 1.} \scriptura{1 Sam. 2, 10; \textbf{H96}}

\vspace{-4mm}

\antiphona{I g\textsuperscript{2}}{temporalia/ant-dominusjudicabit.gtex}

%\vspace{-2mm}

\scriptura{Ps. 49, 1-6}

%\vspace{-2mm}

\initiumpsalmi{temporalia/ps49i_vi-initium-i-g2-auto.gtex}

\input{temporalia/ps49i_vi-i-g2.tex} \Abardot{}

\vfill
\pagebreak

\pars{Psalmus 2.}

\vspace{-4mm}

\antiphona{VIII G}{temporalia/ant-attenditepopulemeus.gtex}

%\vspace{-2mm}

\scriptura{Ps. 49, 7-15}

%\vspace{-2mm}

\initiumpsalmi{temporalia/ps49vii_xv-initium-viii-G-auto.gtex}

\input{temporalia/ps49vii_xv-viii-G.tex} \Abardot{}

\vfill
\pagebreak

\pars{Psalmus 3.} \scriptura{Ps. 49, 14; \textbf{H94}}

\vspace{-4mm}

\antiphona{E}{temporalia/ant-immoladeo.gtex}

%\vspace{-2mm}

\scriptura{Ps. 49, 16-23}

%\vspace{-2mm}

\initiumpsalmi{temporalia/ps49xvi_xxiii-initium-e-auto.gtex}

\input{temporalia/ps49xvi_xxiii-e.tex} \Abardot{}

\vfill
\pagebreak
\fi
\else
\matutinum
\fi

\ifx\matversus\undefined
\pars{Versus} \scriptura{Mc. 1, 3; Is. 40, 3}

% Versus. %%%
\sineinitiali{temporalia/versus-voxclamantis-simplex.gtex}
\else
\matversus
\fi

\vspace{5mm}

\sineinitiali{temporalia/oratiodominica-mat.gtex}

\vspace{5mm}

\pars{Absolutio.}

\cuminitiali{}{temporalia/absolutio-avinculis.gtex}

\vfill
\pagebreak

\cuminitiali{}{temporalia/benedictio-solemn-ille.gtex}

\vspace{7mm}

\lectioi

\noindent \Vbardot{} Tu autem, Dómine, miserére nobis.
\noindent \Rbardot{} Deo grátias.

\vfill
\pagebreak

\responsoriumi

\vfill
\pagebreak

\cuminitiali{}{temporalia/benedictio-solemn-divinum.gtex}

\vspace{7mm}

\lectioii

\noindent \Vbardot{} Tu autem, Dómine, miserére nobis.
\noindent \Rbardot{} Deo grátias.

\vfill
\pagebreak

\responsoriumii

\vfill
\pagebreak

\cuminitiali{}{temporalia/benedictio-solemn-adsocietatem.gtex}

\vspace{7mm}

\lectioiii

\noindent \Vbardot{} Tu autem, Dómine, miserére nobis.
\noindent \Rbardot{} Deo grátias.

\vfill
\pagebreak

\responsoriumiii

\vfill
\pagebreak

\rubrica{Reliqua omittuntur, nisi Laudes separandæ sint.}

\sineinitiali{temporalia/domineexaudi.gtex}

\vfill

\oratio

\vfill

\noindent \Vbardot{} Dómine, exáudi oratiónem meam.

\noindent \Rbardot{} Et clamor meus ad te véniat.

\noindent \Vbardot{} Benedicámus Dómino, allelúia, allelúia.

\noindent \Rbardot{} Deo grátias, allelúia, allelúia.

\noindent \Vbardot{} Fidélium ánimæ per misericórdiam Dei requiéscant in pace.

\noindent \Rbardot{} Amen.

\vfill
\pagebreak

\hora{Ad Laudes.} %%%%%%%%%%%%%%%%%%%%%%%%%%%%%%%%%%%%%%%%%%%%%%%%%%%%%

\cantusSineNeumas

\vspace{0.5cm}
\ifx\deusinadiutorium\undefined
\grechangedim{interwordspacetext}{0.18 cm plus 0.15 cm minus 0.05 cm}{scalable}%
\cuminitiali{}{temporalia/deusinadiutorium-communis.gtex}
\grechangedim{interwordspacetext}{0.22 cm plus 0.15 cm minus 0.05 cm}{scalable}%
\else
\deusinadiutorium
\fi

\vfill
\pagebreak

\ifx\hymnuslaudes\undefined
\pars{Hymnus} \scriptura{Ambrosius (\olddag{} 397)}

\cuminitiali{I}{temporalia/hym-VoxClara-aromi.gtex}
\vspace{-3mm}
\else
\hymnuslaudes
\fi

\vfill
\pagebreak

\ifx\laudes\undefined
\ifx\lauda\undefined
\else
\pars{Psalmus 1.} \scriptura{Ps. 62, 2.3; \textbf{H142}}

\vspace{-4mm}

\antiphona{VII a}{temporalia/ant-adtedeluce.gtex}

\scriptura{Psalmus 118, 145-152; \hspace{5mm} \hebinitial{ק}}

\initiumpsalmi{temporalia/ps118xix-initium-vii-a-auto.gtex}

\input{temporalia/ps118xix-vii-a.tex} \Abardot{}

\vfill
\pagebreak

\pars{Psalmus 2.} \scriptura{Ex. 15, 1; \textbf{H98}}

\vspace{-4mm}

\antiphona{E}{temporalia/ant-cantemusdomino.gtex}

\scriptura{Canticum Moysis, Ex. 15, 1-19}

\initiumpsalmi{temporalia/moysis-initium-e-auto.gtex}

\input{temporalia/moysis-e.tex}

\antiphona{}{temporalia/ant-cantemusdomino.gtex}

\vfill
\pagebreak

\pars{Psalmus 3.} \scriptura{Ps. 116, 1; \textbf{H94}}

\vspace{-4mm}

\antiphona{E}{temporalia/ant-laudatedominumomnes.gtex}

\scriptura{Psalmus 116.}

\initiumpsalmi{temporalia/ps116-initium-e.gtex}

\input{temporalia/ps116-e.tex} \Abardot{}

\vfill
\pagebreak
\fi
\ifx\laudb\undefined
\else
\pars{Psalmus 1.} \scriptura{Ps. 91, 6}

\vspace{-4.5mm}

\antiphona{E}{temporalia/ant-quammagnificatasunt.gtex}

\vspace{-3mm}

\scriptura{Psalmus 91.}

\vspace{-2mm}

\initiumpsalmi{temporalia/ps91-initium-e.gtex}

\vspace{-1.5mm}

\input{temporalia/ps91-e.tex} \Abardot{}

\vfill
\pagebreak

\pars{Psalmus 2.} \scriptura{Dt. 32, 3}

%\vspace{-4mm}

\antiphona{VI F}{temporalia/ant-datemagnitudinem.gtex}

\vspace{-4mm}

\scriptura{Canticum Moysi, Dt. 32, 1-32}

\initiumpsalmi{temporalia/moysis2i_xii-initium-vi-F-auto.gtex}

\input{temporalia/moysis2i_xii-vi-F.tex}

\vfill

\antiphona{}{temporalia/ant-datemagnitudinem.gtex}

\vfill
\pagebreak

\pars{Psalmus 3.} \scriptura{Ps. 8, 2}

\vspace{-4mm}

\antiphona{I g}{temporalia/ant-quamadmirabileest.gtex}

%\vspace{-2mm}

\scriptura{Ps. 8}

%\vspace{-2mm}

\initiumpsalmi{temporalia/ps8-initium-i-g-auto.gtex}

\input{temporalia/ps8-i-g.tex} \Abardot{}

\vfill
\pagebreak
\fi
\ifx\laudc\undefined
\else
\pars{Psalmus 1.} \scriptura{Ps. 62, 7}

\vspace{-4mm}

\antiphona{E}{temporalia/ant-inmatutinis.gtex}

%\vspace{-2mm}

\scriptura{Psalmus 118, 145-152.}

%\vspace{-2mm}

\initiumpsalmi{temporalia/ps118xix-initium-e-auto.gtex}

%\vspace{-1.5mm}

\input{temporalia/ps118xix-e.tex} \Abardot{}

\vfill
\pagebreak

\pars{Psalmus 2.}

\vspace{-4mm}

\antiphona{V a}{temporalia/ant-mecumsitdomine.gtex}

%\vspace{-2mm}

\scriptura{Canticum Sapientiæ, Sap. 9, 1-6.9-11}

\initiumpsalmi{temporalia/sapientia-initium-v-a-auto.gtex}

\input{temporalia/sapientia-v-a.tex} \Abardot{}

\vfill
\pagebreak

\pars{Psalmus 3.}

\vspace{-4mm}

\antiphona{II* b}{temporalia/ant-veritasdomini.gtex}

%\vspace{-2mm}

\scriptura{Ps. 116}

%\vspace{-2mm}

\initiumpsalmi{temporalia/ps116-initium-ii_-B-auto.gtex}

\input{temporalia/ps116-ii_-B.tex} \Abardot{}

\vfill
\pagebreak
\fi
\ifx\laudd\undefined
\else
\pars{Psalmus 1.} \scriptura{Ps. 91, 2; \textbf{H99}}

\vspace{-4mm}

\antiphona{VIII G}{temporalia/ant-bonumestconfiteri.gtex}

%\vspace{-2mm}

\scriptura{Psalmus 91.}

%\vspace{-2mm}

\initiumpsalmi{temporalia/ps91-initium-viii-g-auto.gtex}

%\vspace{-1.5mm}

\input{temporalia/ps91-viii-g.tex}

\vfill

\antiphona{}{temporalia/ant-bonumestconfiteri.gtex}

\vfill
\pagebreak

\pars{Psalmus 2.}

\vspace{-4mm}

\antiphona{IV* e}{temporalia/ant-dabovobiscor.gtex}

%\vspace{-2mm}

\scriptura{Canticum Habacuc, Hab. 3, 2-19}

\initiumpsalmi{temporalia/habacuc-initium-iv_-e.gtex}

\input{temporalia/habacuc-iv_-e.tex}

\vfill

\antiphona{}{temporalia/ant-dabovobiscor.gtex}

\vfill
\pagebreak

\pars{Psalmus 3.}

\vspace{-4mm}

\antiphona{I f}{temporalia/ant-exoreinfantium.gtex}

%\vspace{-2mm}

\scriptura{Ps. 8}

%\vspace{-2mm}

\initiumpsalmi{temporalia/ps8-initium-i-f-auto.gtex}

\input{temporalia/ps8-i-f.tex} \Abardot{}

\vfill
\pagebreak
\fi
\else
\laudes
\fi

\ifx\lectiobrevis\undefined
\pars{Lectio Brevis.} \scriptura{Is. 11, 1-3}

\noindent Egrediétur virga de stirpe Iesse, et flos de radíce eius ascéndet; et requiéscet super eum spíritus Dómini: spíritus sapiéntiæ et intelléctus, spíritus consílii et fortitúdinis, spíritus sciéntiæ et timóris Dómini; et delíciæ eius in timóre Dómini.
\else
\lectiobrevis
\fi

\vfill

\ifx\responsoriumbreve\undefined
\pars{Responsorium breve.} \scriptura{Is. 60, 2; \textbf{H20}}

\cuminitiali{IV}{temporalia/resp-superte.gtex}
\else
\responsoriumbreve
\fi

\vfill
\pagebreak

\benedictus

\vfill
\pagebreak

\pars{Preces.}

\sineinitiali{}{temporalia/tonusprecum.gtex}

\ifx\preces\undefined
\noindent Deum Patrem, qui antíqua dispositióne pópulum suum salváre státuit, \gredagger{} orémus dicéntes:

\Rbardot{} Custódi plebem tuam, Dómine.

\noindent Deus, qui pópulo tuo germen iustítiæ promisísti, \gredagger{} custódi sanctitátem Ecclésiæ tuæ.

\Rbardot{} Custódi plebem tuam, Dómine.

\noindent Inclína cor hóminum, Deus, in verbum tuum \gredagger{} et confírma fidéles tuos sine queréla in sanctitáte.

\Rbardot{} Custódi plebem tuam, Dómine.

\noindent Consérva nos in dilectióne Spíritus tui, \gredagger{} ut Fílii tui, qui ventúrus est, misericórdiam suscipiámus.

\Rbardot{} Custódi plebem tuam, Dómine.

\noindent Confírma nos, Deus clementíssime, usque in finem, \gredagger{} in diem advéntus Dómini Iesu Christi.

\Rbardot{} Custódi plebem tuam, Dómine.
\else
\preces
\fi

\vfill

\pars{Oratio Dominica.}

\cuminitiali{}{temporalia/oratiodominicaalt.gtex}

\vfill
\pagebreak

\rubrica{vel:}

\pars{Supplicatio Litaniæ.}

\cuminitiali{}{temporalia/supplicatiolitaniae.gtex}

\vfill

\pars{Oratio Dominica.}

\cuminitiali{}{temporalia/oratiodominica.gtex}

\vfill
\pagebreak

% Oratio. %%%
\oratio

\vspace{-1mm}

\vfill

\rubrica{Hebdomadarius dicit Dominus vobiscum, vel, absente sacerdote vel diacono, sic concluditur:}

\vspace{2mm}

\antiphona{C}{temporalia/dominusnosbenedicat.gtex}

\rubrica{Postea cantatur a cantore:}

\vspace{2mm}

\ifx\benedicamuslaudes\undefined
\cuminitiali{IV}{temporalia/benedicamus-feria-advequad.gtex}
\else
\benedicamuslaudes
\fi

\vfill

\vspace{1mm}

\end{document}

