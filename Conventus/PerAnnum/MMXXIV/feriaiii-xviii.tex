\newcommand{\titulus}{\nomenFesti{Transfiguratio D.N.I.C.}
\dies{Die 6. Augusti.}}
\newcommand{\oratio}{\pars{Oratio.}

\noindent Deus, qui fídei sacraménta in Unigéniti tui gloriósa Transfiguratióne patrum testimónio roborásti et adoptiónem filiórum perféctam mirabíliter præsignásti, concéde nobis fámulis tuis, ut, ipsíus dilécti Fílii tui vocem audiéntes, eiúsdem coherédes éffici mereámur.

\pars{Pro pace in universo mundo.} \scriptura{Sir. 50, 25; 2 Esdr. 4, 20; \textbf{H416}}

\vspace{-4mm}

\antiphona{II D}{temporalia/ant-dapacemdomine.gtex}

\vfill

\noindent Deus, a quo sancta desidéria, recta consília et iusta sunt ópera: da servis tuis illam, quam mundus dare non potest, pacem; ut et corda nostra mandátis tuis dédita, et hóstium subláta formídine, témpora sint tua protectióne tranquílla.

\noindent Per Dóminum nostrum Iesum Christum, Fílium tuum, qui tecum vivit et regnat in unitáte Spíritus Sancti, Deus, per ómnia sǽcula sæculórum.

\noindent \Rbardot{} Amen.}
\newcommand{\invitatorium}{\pars{Invitatorium.} \scriptura{Cantor; Psalmus 94; \textbf{H447}}

\vspace{-6mm}

\antiphona{V}{temporalia/inv-summumregem.gtex}}
\newcommand{\hymnusmatutinum}{\pars{Hymnus.}

\cuminitiali{IV}{temporalia/hym-CaelestisFormam.gtex}}
\newcommand{\matutinum}{
\iffalse
\subhora{In I. Nocturno}
\fi

\pars{Psalmus 1.} \scriptura{Mt. 17, 4; \textbf{H149}}

\vspace{-4mm}

\antiphona{I g\textsuperscript{3}}{temporalia/ant-dominebonumest.gtex}

%\vspace{-2mm}

\scriptura{Ps. 83}

%\vspace{-2mm}

\initiumpsalmi{temporalia/ps83-initium-i-g3-auto.gtex}

%\vspace{-1.5mm}

\input{temporalia/ps83-i-g3.tex}

\vfill

\antiphona{}{temporalia/ant-dominebonumest.gtex}

\vfill
\pagebreak

\pars{Psalmus 2.} \scriptura{Mt. 17, 5}

\vspace{-4mm}

\antiphona{VI F}{temporalia/ant-adhuceoloquente.gtex}

%\vspace{-2mm}

\scriptura{Ps. 96}

%\vspace{-2mm}

\initiumpsalmi{temporalia/ps96-initium-vi-F-auto.gtex}

%\vspace{-1.5mm}

\input{temporalia/ps96-vi-F.tex} \Abardot{}

\vfill
\pagebreak

\pars{Psalmus 3.} \scriptura{Cf. Mt. 17, 5; \textbf{H78}}

\vspace{-4mm}

\antiphona{IV e}{temporalia/ant-voxdecaelissonuit.gtex}

%\vspace{-2mm}

\scriptura{Ps. 98}

%\vspace{-2mm}

\initiumpsalmi{temporalia/ps98-initium-iv-e2-auto.gtex}

\input{temporalia/ps98-iv-e2.tex} \Abardot{}

\vfill
\pagebreak}
\newcommand{\matversus}{\noindent \Vbardot{} In colúmna nubis loquebátur ad eos.

\noindent \Rbardot{} Custodiébant testimónia eius.}
\newcommand{\lectioi}{\pars{Lectio I.} \scriptura{2 Cor. 3, 7-18; 4, 1-6}

\noindent De Epístola secúnda beáti Pauli apóstoli ad Corínthios.

\noindent Fratres: Si ministrátio mortis lítteris deformáta in lapídibus fuit in glória, ita ut non possent inténdere fílii Israel in fáciem Móysis propter glóriam vultus eius, quæ evacuátur, quómodo non magis ministrátio Spíritus erit in glória? Nam si ministérium damnatiónis glória est, multo magis abúndat ministérium iustítiæ in glória. Nam nec glorificátum est, quod cláruit in hac parte, propter excelléntem glóriam; si enim, quod evacuátur, per glóriam est, multo magis, quod manet, in glória est.

\noindent Habéntes ígitur talem spem multa fidúcia útimur, et non sicut Móyses: ponébat velámen super fáciem suam, ut non inténderent fílii Israel in finem illíus, quod evacuátur. Sed obtúsi sunt sensus eórum. Usque in hodiérnum enim diem idípsum velámen in lectióne Véteris Testaménti manet non revelátum, quóniam in Christo evacuátur; sed usque in hodiérnum diem, cum légitur Móyses, velámen est pósitum super cor eórum. \emph{Quando autem convérsus fúerit ad Dóminum, aufértur velámen.} Dóminus autem Spíritus est; ubi autem Spíritus Dómini, ibi libértas. Nos vero omnes reveláta fácie glóriam Dómini speculántes, in eándem imáginem transformámur a claritáte in claritátem tamquam a Dómini Spíritu.

\noindent Ideo habéntes hanc ministratiónem, iuxta quod misericórdiam consecúti sumus, non defícimus, sed abdicávimus occúlta dedécoris non ambulántes in astútia neque adulterántes verbum Dei, sed in manifestatióne veritátis commendántes nosmetípsos ad omnem consciéntiam hóminum coram Deo.

\noindent Quod si étiam velátum est evangélium nostrum, in his, qui péreunt, est velátum, in quibus deus huius sǽculi excæcávit mentes infidélium, ut non fúlgeat illuminátio evangélii glóriæ Christi, qui est imágo Dei. Non enim nosmetípsos prædicámus sed Iesum Christum Dóminum; nos autem servos vestros per Iesum. Quóniam Deus qui dixit: «De ténebris lux splendéscat», ipse illúxit in córdibus nostris ad illuminatiónem sciéntiæ claritátis Dei in fácie Iesu Christi.}
\newcommand{\responsoriumi}{\pars{Responsorium 1.} \scriptura{\Rbardot{} Cf. Mt. 17, 2.5.6 \Vbardot{} ibid. 17, 3; \textbf{H160}}

\vspace{-5mm}

\responsorium{VIII}{temporalia/resp-splendidafactaest.gtex}{}}
\newcommand{\lectioii}{\pars{Lectio II.} \scriptura{Nn. 6-10: Mélanges d’archéologie et d’histoire 67 [1955], 241-244}

\noindent Ex Sermóne Anastásii Sinaítæ epíscopi in die Transfiguratiónis Dómini.

\noindent Mystérium hoc manifestávit Iesus discípulis suis in monte Thabor. Postquam enim inter eos ámbulans sermónes de regno deque suo áltero in glória advéntu díxerat, ut, qui fortásse non satis certi erant de iis quæ circa regnum nuntiáverat, firmíssime tandem in cordis íntimis convinceréntur, utque étiam ex præséntibus futúra créderent, divínam manifestatiónem in monte Thabor mirabíliter exhíbuit tamquam præfiguratívam imáginem regni cælórum. Ita prorsus ac si díceret: «Ne témporis intervállum incredulitátem in vobis gignat, statim, in præsénti, \emph{amen dico vobis, sunt quidam de hic stántibus, qui non gustábunt mortem, donec vídeant Fílium hóminis veniéntem} in glória Patris sui».

\noindent Osténdens autem Evangelísta poténtiam Christi cum ipsíus voluntáte conveníre, addit: \emph{Et post dies sex assúmit Iesus Petrum et Iacóbum et Ioánnem, et ducit illos in montem excélsum seórsum. Et transfigurátus est ante eos, et resplénduit fácies eius sicut sol, vestiménta autem eius facta sunt sicut nix. Et ecce apparuérunt Móyses et Elías cum eo loquéntes.}

\noindent Hæc sunt præséntis sollemnitátis mirácula, hoc est nobis salutáre mystérium quod in monte nunc est adimplétum; simul enim nos modo cóngregat et mors et festívitas Christi. Ut ígitur íntima ineffabílium horum sacrorúmque mysteriórum una cum eléctis inter discípulos a Deo inspirátos penetrémus, vocem divínam sacrámque audiámus, quæ ex alto, e vértice montis nos instánter cónvocat.}
\newcommand{\responsoriumii}{\pars{Responsorium 2.} \scriptura{\Rbardot{} Cf. Mt. 17, 1.2}

\vspace{-5mm}

\responsorium{III}{temporalia/resp-assumensiesus-sinedox.gtex}{}}
\newcommand{\lectioiii}{\pars{Lectio III.}

\noindent Illuc nos opórtet festináre —audénter dico— sicut Iesus, qui hic in cælis dux noster est ac præcúrsor, quocum fulgébimus óculis spiritálibus, lineaméntis quodam modo ánimæ nostræ renováti, ad eius conformáti imáginem, ac sicut ipse sine intermissióne transfiguráti divinǽque natúræ facti consórtes et ad superióra paráti.

\noindent Illuc currámus, animósi ac lætántes, et intrémus in íntimam nubem, facti tamquam Móyses et Elías, vel Iacóbus et Ioánnes. Esto sicut Petrus, in divínam visiónem et apparitiónem raptus, pulchra hac Transfiguratióne transfigurátus, elátus e mundo, abstráctus a terra; relínque carnem, désere creatiónem et convértere ad Creatórem, cui Petrus a se abréptus: \emph{Dómine,} inquit, \emph{bonum est nos hic esse}.

\noindent Equidem, Petre, vere \emph{bonum est nos hic esse} cum Iesu atque hic in sǽcula manére. Quid felícius, quid áltius, quid est præstántius quam esse cum Deo, ipsi conformári, in luce inveníri? Certe unusquísque nostrum, cum Deum in se hábeat et sit in divínam eius imáginem transfigurátus, cum lætítia exclámet: \emph{Bonum est nos hic esse}, ubi ómnia sunt lúcida, ubi gáudium est et beatitúdo et iucúnditas, ubi ómnia in corde tranquílla sunt et seréna et dúlcia, ubi (Christus) Deus conspícitur; ubi mansiónem ipse cum Patre facit et advéniens ait: \emph{Hódie salus dómui huic facta est}; ubi cum Christo thesáuri exstant et cumulántur bonórum æternórum; ubi primítiæ et imágines futurórum sæculórum velut in spéculo describúntur.}
\newcommand{\responsoriumiii}{\pars{Responsorium 3.} \scriptura{\Rbardot{} Cf. Mt. 17, 2.3.4}

\vspace{-5mm}

\responsorium{IV}{temporalia/resp-videnspetrusmoysen-cumdox.gtex}{}

\vfill
\pagebreak

\iffalse
\subhora{In II. Nocturno}

\pars{Cantica.}

\scriptura{Cf. Mt. 17, 7}

\vspace{-2mm}

\antiphona{VIII G}{temporalia/ant-accessitiesus.gtex}

\vspace{3mm}

\scriptura{Canticum ex Liber Chronicorum, 1 Chr. 29, 10-13}

%\vspace{-2mm}

\initiumpsalmi{temporalia/chronicorum-initium-viii-G-auto.gtex}

\input{temporalia/chronicorum-viii-G.tex}

\vfill
\pagebreak

\scriptura{Canticum Isaiæ Prophetæ, Is. 12, 1-7}

%\vspace{-2mm}

\initiumpsalmi{temporalia/isaiae-initium-viii-G-auto.gtex}

\vfill

\input{temporalia/isaiae-viii-G.tex}

\vfill
\pagebreak

\scriptura{Canticum Isaiaæ, Is. 61, 10-11; 62, 1-7}

\vspace{-3mm}

\initiumpsalmi{temporalia/isaiae4-initium-viii-G-auto.gtex}

\vspace{-1.5mm}

\input{temporalia/isaiae4-viii-G.tex}

\antiphona{}{temporalia/ant-accessitiesus.gtex}

\vfill
\pagebreak

\pars{Versus.}

\noindent \Vbardot{} Magna est glória eius in salutári tuo.

\noindent \Rbardot{} Glóriam et magnum decórem impónes super eum.

\vspace{5mm}

\sineinitiali{temporalia/oratiodominica-mat.gtex}

\vspace{5mm}

\pars{Absolutio.}

\cuminitiali{}{temporalia/absolutio-avinculis.gtex}

\vfill
\pagebreak

\cuminitiali{}{temporalia/benedictio-solemn-evangelica.gtex}

\vspace{7mm}

\pars{Evangelium} \scriptura{Mt. 17, 1-9}

\noindent Léctio sancti Evangélii secúndum Matthǽum.

\noindent In illo témpore: Assúmit Iesus Petrum et Iacóbum et Ioánnem fratrem eius et ducit illos in montem excélsum seórsum. Et transfigurátus est ante eos; et resplénduit fácies eius sicut sol, vestiménta autem eius facta sunt alba sicut lux. Et ecce appáruit illis Móyses et Elías cum eo loquéntes.

\noindent Respóndens autem Petrus dixit ad Iesum: «Dómine, bonum est nos hic esse. Si vis, fáciam hic tria tabernácula: tibi unum et Móysi unum et Elíæ unum».

\noindent Adhuc eo loquénte, ecce nubes lúcida obumbrávit eos; et ecce vox de nube dicens: «Hic est Fílius meus diléctus, in quo mihi bene complácui; ipsum audíte». Et audiéntes discípuli cecidérunt in fáciem suam et timuérunt valde.

\noindent Et accéssit Iesus et tétigit eos dixítque eis: «Súrgite et nolíte timére». Levántes autem óculos suos, néminem vidérunt nisi solum Iesum.

\noindent Et descendéntibus illis de monte, præcépit eis Iesus dicens: «Némini dixéritis visiónem, donec Fílius hóminis a mórtuis resúrgat».

\vspace{5mm}

\scriptura{Sermo 51, 3-4. 8: PL 54, 310.311. 313}

\noindent Ex Sermónibus sancti Leónis Magni papæ.

\noindent Aperit Dóminus coram eléctis téstibus glóriam suam, et commúnem illam cum céteris córporis formam tanto splendóre claríficat, ut et fácies eius solis fulgóri símilis, et vestítus candóri nívium esset æquális. In qua transfiguratióne illud quidem principáliter agebátur, ut de córdibus discipulórum crucis scándalum tollerétur; nec conturbáret eórum fidem voluntáriæ humílitas passiónis, quibus reveláta esset abscónditæ excelléntia dignitátis.

\noindent Sed non minóre providéntia spes sanctæ Ecclésiæ fundabátur, ut totum corpus Christi agnósceret quali esset commutatióne donándum, et eius sibi honóris consórtium membra promítterent, qui in cápite præfulsísset.

\noindent De quo idem Dóminus díxerat, cum de advéntus sui maiestáte loquerétur: \emph{Tunc iusti fulgébunt sicut sol in regno Patris sui;} protestánte hoc ipsum beáto Paulo apóstolo et dicénte: \emph{Exístimo enim quod non sunt condígnæ passiónes huius témporis ad futúram glóriam, quæ revelábitur in nobis;} et íterum: \emph{Mórtui enim estis, et vita vestra abscóndita est cum Christo in Deo. Cum enim Christus apparúerit vita vestra, tunc et vos apparébitis cum ipso in glória.}

\noindent {\color{gray} Confirmándis vero apóstolis et ad omnem sciéntiam provehéndis, ália quoque in illo miráculo accéssit instrúctio. Móyses enim et Elías, lex scílicet et prophétæ, apparuérunt cum Dómino loquéntes, ut veríssime in illa quinque virórum præséntia complerétur quod dictum est: \emph{In duóbus vel tribus téstibus stat omne verbum.}

\noindent Quid hoc stabílius, quid fírmius verbo, in cuius prædicatióne véteris et novi testaménti cóncinit tuba et cum evangélica doctrína antiquárum protestatiónum instruménta concúrrunt?

\noindent Astipulántur enim sibi ínvicem utriúsque fœ́deris páginæ; et, quem sub velámine mysteriórum præcedéntia signa promíserant, maniféstum atque perspícuum præséntis glóriæ splendor osténdit; quia, sicut ait beátus Ioánnes, \emph{lex per Móysen data est, grátia autem et véritas per Iesum Christum facta est}; in quo et propheticárum promíssio impléta est figurárum et legálium rátio præceptórum, dum et veram docet prophetíam per sui præséntiam, et possibília facit mandáta per grátiam.}

\noindent Confirmétur ergo secúndum prædicatiónem sacratíssimi Evangélii ómnium fides, et nemo de Christi cruce, per quam mundus redémptus est, erubéscat.

\noindent Nec ídeo quisquam aut pati pro iustítia tímeat, aut de promissórum retributióne diffídat, quia per labórem ad réquiem, et per mortem transítur ad vitam; cum omnem humilitátis nostræ infirmitátem ille suscéperit, in quo, si in confessióne et in dilectióne ipsíus permaneámus, et quod vicit víncimus, et quod promísit accípimus.

\noindent Quia sive ad faciénda mandáta, sive ad toleránda advérsa, præmíssa Patris vox debet semper áuribus nostris insonáre, dicéntis: \emph{Hic est Fílius meus diléctus, in quo mihi bene complácui: ipsum audíte.}

\vfill
\pagebreak

\cuminitiali{I}{temporalia/tedecetlaus.gtex}

\vfill

\rubrica{vel ad libitum:}

\vspace{3mm}

\cuminitiali{II}{temporalia/tedecetlausii.gtex}

\vfill
\pagebreak
\fi

\pars{Hymnus Ambrosianus} \scriptura{Alio modo, iuxta morem Romanum}

\vspace{-2mm}

{
\grechangedim{interwordspacetext}{0.26 cm plus 0.15 cm minus 0.05 cm}{scalable}%
\cuminitiali{III}{temporalia/tedeum-romanum-gn.gtex}
\grechangedim{interwordspacetext}{0.22 cm plus 0.15 cm minus 0.05 cm}{scalable}%
}}
\newcommand{\deusinadiutorium}{\grechangedim{interwordspacetext}{0.18 cm plus 0.15 cm minus 0.05 cm}{scalable}%
\cuminitiali{}{temporalia/deusinadiutorium-alter.gtex}
\grechangedim{interwordspacetext}{0.22 cm plus 0.15 cm minus 0.05 cm}{scalable}}
\newcommand{\hymnuslaudes}{\pars{Hymnus}

\cuminitiali{D}{temporalia/hym-DulcisIesuMemoria.gtex}}
\newcommand{\laudes}{\pars{Psalmus 1.} \scriptura{Cf. Mt. 17, 2}

\vspace{-4mm}

\antiphona{VIII G\textsuperscript{5}}{temporalia/ant-hodiedominusiesus.gtex}

%\vspace{-2mm}

\scriptura{Psalmus 62}

%\vspace{-2mm}

\initiumpsalmi{temporalia/ps62-initium-viii-G6-auto.gtex}

%\vspace{-1.5mm}

\input{temporalia/ps62-viii-G6.tex} \Abardot{}

\vfill
\pagebreak

\pars{Psalmus 2.}

\vspace{-4mm}

\antiphona{I d}{temporalia/ant-hodietransfigurato.gtex}

%\vspace{-2mm}

\scriptura{Canticum trium puerorum, Dan. 3, 57-88 et 56}

\initiumpsalmi{temporalia/dan3-initium-i-d-auto.gtex}

\input{temporalia/dan3-i-d-sinedox.tex}

\rubrica{Hic non dicitur Gloria Patri, neque Amen.}

\vfill

\antiphona{}{temporalia/ant-hodietransfigurato.gtex}

\vfill
\pagebreak

\pars{Psalmus 3.}

\vspace{-4mm}

\antiphona{I g}{temporalia/ant-lexpermoysen.gtex}

%\vspace{-2mm}

\scriptura{Psalmus 149}

%\vspace{-2mm}

\initiumpsalmi{temporalia/ps149-initium-i-g-auto.gtex}

\input{temporalia/ps149-i-g.tex} \Abardot{}

\vfill
\pagebreak}
\newcommand{\lectiobrevis}{\pars{Lectio Brevis.} \scriptura{Ap. 21, 10.23}

\noindent Sústulit me ángelus in spíritu super montem magnum et altum et osténdit mihi civitátem sanctam Ierúsalem descendéntem de cælo a Deo. Et cívitas non eget sole neque luna, ut lúceant ei, nam cláritas Dei illuminávit eam, et lucérna eius est Agnus.}
\newcommand{\responsoriumbreve}{\pars{Responsorium breve.} \scriptura{Ps. 8, 6-7}

\cuminitiali{VI}{temporalia/resp-gloriaethonore.gtex}}
\newcommand{\preces}{\noindent Deum Patrem Dómini et Salvatóris nostri Iesu Christi,~\gredagger{} qui in monte ante discípulos mirabíliter transfigurátus est,~\grestar{} fidénter deprecémur:

\Rbardot{} In lúmine tuo, Dómine, lumen videámus.

\noindent Pater clementíssime, qui Fílium tuum diléctum transfigurásti~\gredagger{} et in nube lúcida teípsum manifestásti,~\grestar{} fac ut verbum Christi fidéliter audiámus.

\Rbardot{} In lúmine tuo, Dómine, lumen videámus.

\noindent Deus, qui eléctos inebriásti ab ubertáte domus tuæ~\gredagger{} et torrénte voluptátis tuæ illos potásti,~\grestar{} concéde, ut in córpore Christi fontem vitæ nostræ inveniámus.

\Rbardot{} In lúmine tuo, Dómine, lumen videámus.

\noindent Deus, qui fecísti de ténebris lumen splendéscere~\gredagger{} et illuxísti in córdibus nostris ad contemplándam claritátem tuam in fácie Christi Iesu,~\grestar{} fove in nobis spíritum contemplatiónis Fílii tui dilécti.

\Rbardot{} In lúmine tuo, Dómine, lumen videámus.

\noindent Deus, qui nos vocásti vocatióne tua sancta,~\gredagger{} secúndum grátiam tuam nunc manifestátam per illuminatiónem salvatóris nostri Iesu Christi,~\grestar{} illústra per Evangélium inter hómines vitam incorruptíbilem.

\Rbardot{} In lúmine tuo, Dómine, lumen videámus.

\noindent Pater amantíssime,~\gredagger{} qui talem caritátem dedísti nobis ut fílii Dei nominémur et simus,~\grestar{} præsta, ut, cum apparúerit Christus, símiles ei fiámus.

\Rbardot{} In lúmine tuo, Dómine, lumen videámus.}
\newcommand{\benedictus}{\pars{Canticum Zachariæ.} \scriptura{Mt. 17, 5}

\vspace{-4mm}

\antiphona{VII a}{temporalia/ant-eteccevoxdenube.gtex}


\vspace{-2mm}

\scriptura{Lc. 1, 68-79}

\vspace{-2mm}

\cantusSineNeumas
\initiumpsalmi{temporalia/benedictus-initium-viisoll-a.gtex}

%\vspace{-1.5mm}

\input{temporalia/benedictus-viisoll-a.tex} \Abardot{}}
\newcommand{\benedicamuslaudes}{\cuminitiali{II}{temporalia/benedicamus-solemnism-laud.gtex}}
\newcommand{\hebdomada}{infra Hebdom. XVIII post Pentecosten.}
\newcommand{\oratioLaudes}{\cuminitiali{}{temporalia/oratio18.gtex}}

% LuaLaTeX

\documentclass[a4paper, twoside, 12pt]{article}
\usepackage[latin]{babel}
%\usepackage[landscape, left=3cm, right=1.5cm, top=2cm, bottom=1cm]{geometry} % okraje stranky
%\usepackage[landscape, a4paper, mag=1166, truedimen, left=2cm, right=1.5cm, top=1.6cm, bottom=0.95cm]{geometry} % okraje stranky
\usepackage[landscape, a4paper, mag=1400, truedimen, left=0.5cm, right=0.5cm, top=0.5cm, bottom=0.5cm]{geometry} % okraje stranky

\usepackage{fontspec}
\setmainfont[FeatureFile={junicode.fea}, Ligatures={Common, TeX}, RawFeature=+fixi]{Junicode}
%\setmainfont{Junicode}

% shortcut for Junicode without ligatures (for the Czech texts)
\newfontfamily\nlfont[FeatureFile={junicode.fea}, Ligatures={Common, TeX}, RawFeature=+fixi]{Junicode}

\usepackage{multicol}
\usepackage{color}
\usepackage{lettrine}
\usepackage{fancyhdr}

% usual packages loading:
\usepackage{luatextra}
\usepackage{graphicx} % support the \includegraphics command and options
\usepackage{gregoriotex} % for gregorio score inclusion
\usepackage{gregoriosyms}
\usepackage{wrapfig} % figures wrapped by the text
\usepackage{parcolumns}
\usepackage[contents={},opacity=1,scale=1,color=black]{background}
\usepackage{tikzpagenodes}
\usepackage{calc}
\usepackage{longtable}
\usetikzlibrary{calc}

\setlength{\headheight}{14.5pt}

% Commands used to produce a typical "Conventus" booklet

\newenvironment{titulusOfficii}{\begin{center}}{\end{center}}
\newcommand{\dies}[1]{#1

}
\newcommand{\nomenFesti}[1]{\textbf{\Large #1}

}
\newcommand{\celebratio}[1]{#1

}

\newcommand{\hora}[1]{%
\vspace{0.5cm}{\large \textbf{#1}}

\fancyhead[LE]{\thepage\ / #1}
\fancyhead[RO]{#1 / \thepage}
\addcontentsline{toc}{subsection}{#1}
}

% larger unit than a hora
\newcommand{\divisio}[1]{%
\begin{center}
{\Large \textsc{#1}}
\end{center}
\fancyhead[CO,CE]{#1}
\addcontentsline{toc}{section}{#1}
}

% a part of a hora, larger than pars
\newcommand{\subhora}[1]{
\begin{center}
{\large \textit{#1}}
\end{center}
%\fancyhead[CO,CE]{#1}
\addcontentsline{toc}{subsubsection}{#1}
}

% rubricated inline text
\newcommand{\rubricatum}[1]{\textit{#1}}

% standalone rubric
\newcommand{\rubrica}[1]{\vspace{3mm}\rubricatum{#1}}

\newcommand{\notitia}[1]{\textcolor{red}{#1}}

\newcommand{\scriptura}[1]{\hfill \small\textit{#1}}

\newcommand{\translatioCantus}[1]{\vspace{1mm}%
{\noindent\footnotesize \nlfont{#1}}}

% pruznejsi varianta nasledujiciho - umoznuje nastavit sirku sloupce
% s prekladem
\newcommand{\psalmusEtTranslatioB}[3]{
  \vspace{0.5cm}
  \begin{parcolumns}[colwidths={2=#3}, nofirstindent=true]{2}
    \colchunk{
      \input{#1}
    }

    \colchunk{
      \vspace{-0.5cm}
      {\footnotesize \nlfont
        \input{#2}
      }
    }
  \end{parcolumns}
}

\newcommand{\psalmusEtTranslatio}[2]{
  \psalmusEtTranslatioB{#1}{#2}{8.5cm}
}


\newcommand{\canticumMagnificatEtTranslatio}[1]{
  \psalmusEtTranslatioB{#1}{temporalia/extra-adventum-vespers/magnificat-boh.tex}{12cm}
}
\newcommand{\canticumBenedictusEtTranslatio}[1]{
  \psalmusEtTranslatioB{#1}{temporalia/extra-adventum-laudes/benedictus-boh.tex}{10.5cm}
}

% volne misto nad antifonami, kam si zpevaci dokresli neumy
\newcommand{\hicSuntNeumae}{\vspace{0.5cm}}

% prepinani mista mezi notovymi osnovami: pro neumovane a neneumovane zpevy
\newcommand{\cantusCumNeumis}{
  \setgrefactor{17}
  \global\advance\grespaceabovelines by 5mm%
}
\newcommand{\cantusSineNeumas}{
  \setgrefactor{17}
  \global\advance\grespaceabovelines by -5mm%
}

% znaky k umisteni nad inicialu zpevu
\newcommand{\superInitialam}[1]{\gresetfirstlineaboveinitial{\small {\textbf{#1}}}{\small {\textbf{#1}}}}

% pars officii, i.e. "oratio", ...
\newcommand{\pars}[1]{\textbf{#1}}

\newenvironment{psalmus}{
  \setlength{\parindent}{0pt}
  \setlength{\parskip}{5pt}
}{
  \setlength{\parindent}{10pt}
  \setlength{\parskip}{10pt}
}

%%%% Prejmenovat na latinske:
\newcommand{\nadpisZalmu}[1]{
  \hspace{2cm}\textbf{#1}\vspace{2mm}%
  \nopagebreak%

}

% mode, score, translation
\newcommand{\antiphona}[3]{%
\hicSuntNeumae
\superInitialam{#1}
\includescore{#2}

#3
}
 % Often used macros

\newcommand{\annusEditionis}{2021}

%%%% Vicekrat opakovane kousky

\newcommand{\anteOrationem}{
  \rubrica{Ante Orationem, cantatur a Superiore:}

  \pars{Supplicatio Litaniæ.}

  \cuminitiali{}{temporalia/supplicatiolitaniae.gtex}

  \pars{Oratio Dominica.}

  \cuminitiali{}{temporalia/oratiodominica.gtex}

  \rubrica{Deinde dicitur ab Hebdomadario:}

  \cuminitiali{}{temporalia/dominusvobiscum-solemnis.gtex}

  \rubrica{In choro monialium loco Dominus vobiscum dicitur:}

  \sineinitiali{temporalia/domineexaudi.gtex}
}

\setlength{\columnsep}{30pt} % prostor mezi sloupci

%%%%%%%%%%%%%%%%%%%%%%%%%%%%%%%%%%%%%%%%%%%%%%%%%%%%%%%%%%%%%%%%%%%%%%%%%%%%%%%%%%%%%%%%%%%%%%%%%%%%%%%%%%%%%
\begin{document}

% Here we set the space around the initial.
% Please report to http://home.gna.org/gregorio/gregoriotex/details for more details and options
\grechangedim{afterinitialshift}{2.2mm}{scalable}
\grechangedim{beforeinitialshift}{2.2mm}{scalable}
\grechangedim{interwordspacetext}{0.22 cm plus 0.15 cm minus 0.05 cm}{scalable}%
\grechangedim{annotationraise}{-0.2cm}{scalable}

% Here we set the initial font. Change 38 if you want a bigger initial.
% Emit the initials in red.
\grechangestyle{initial}{\color{red}\fontsize{38}{38}\selectfont}

\pagestyle{empty}

%%%% Titulni stranka
\begin{titulusOfficii}
\ifx\titulus\undefined
\nomenFesti{Feria III \hebdomada{}}
\else
\titulus
\fi
\end{titulusOfficii}

\vfill

\begin{center}
%Ad usum et secundum consuetudines chori \guillemotright{}Conventus Choralis\guillemotleft.

%Editio Sancti Wolfgangi \annusEditionis
\end{center}

\scriptura{}

\pars{}

\pagebreak

\renewcommand{\headrulewidth}{0pt} % no horiz. rule at the header
\fancyhf{}
\pagestyle{fancy}

\cantusSineNeumas

\hora{Ad Matutinum.} %%%%%%%%%%%%%%%%%%%%%%%%%%%%%%%%%%%%%%%%%%%%%%%%%%%%%

\vspace{2mm}

\cuminitiali{}{temporalia/dominelabiamea.gtex}

\vfill
%\pagebreak

\vspace{2mm}

\ifx\invitatorium\undefined
\pars{Invitatorium.} \scriptura{Lc. 24, 34; Psalmus 94; \textbf{H232}}

\vspace{-6mm}

\antiphona{VI}{temporalia/inv-surrexitdominusvere.gtex}
\else
\invitatorium
\fi

\vfill
\pagebreak

\ifx\hymnusmatutinum\undefined
\pars{Hymnus}

\cuminitiali{VIII}{temporalia/hym-LaetareCaelum.gtex}
\else
\hymnusmatutinum
\fi

\vspace{-3mm}

\vfill
\pagebreak

\ifx\matutinum\undefined
\ifx\matua\undefined
\else
% MAT A
\pars{Psalmus 1.}

\vspace{-4mm}

\antiphona{II D}{temporalia/ant-alleluia-turco7.gtex}

%\vspace{-2mm}

\scriptura{Ps. 9, 22-32}

%\vspace{-2mm}

\initiumpsalmi{temporalia/ps9xxii_xxxii-initium-ii-D-auto.gtex}

\input{temporalia/ps9xxii_xxxii-ii-D.tex}

\vfill
\pagebreak

\pars{Psalmus 2.} \scriptura{Ps. 9, 33-39}

%\vspace{-2mm}

\initiumpsalmi{temporalia/ps9xxxiii_xxxix-initium-ii-D-auto.gtex}

\input{temporalia/ps9xxxiii_xxxix-ii-D.tex}

\vfill
\pagebreak

\pars{Psalmus 3.} \scriptura{Ps. 11}

%\vspace{-2mm}

\initiumpsalmi{temporalia/ps11-initium-ii-D-auto.gtex}

\input{temporalia/ps11-ii-D.tex}

\vfill

\antiphona{}{temporalia/ant-alleluia-turco7.gtex}

\vfill
\pagebreak
\fi
\ifx\matub\undefined
\else
% MAT B
\pars{Psalmus 1.}

\vspace{-4mm}

\antiphona{VI F}{temporalia/ant-alleluia-turco6.gtex}

%\vspace{-2mm}

\scriptura{Ps. 36, 1-11}

%\vspace{-2mm}

\initiumpsalmi{temporalia/ps36i_xi-initium-vi-F-auto.gtex}

\input{temporalia/ps36i_xi-vi-F.tex}

\vfill
\pagebreak

\pars{Psalmus 2.}

\scriptura{Ps. 36, 12-29}

\vspace{-2mm}

\initiumpsalmi{temporalia/ps36xii_xxix-initium-vi-F-auto.gtex}

\input{temporalia/ps36xii_xxix-vi-F.tex}

\vfill
\pagebreak

\pars{Psalmus 3.}

\scriptura{Ps. 36, 30-40}

%\vspace{-2mm}

\initiumpsalmi{temporalia/ps36iii-initium-vi-F-auto.gtex}

\input{temporalia/ps36iii-vi-F.tex}

\antiphona{}{temporalia/ant-alleluia-turco6.gtex}

\vfill
\pagebreak
\fi
\ifx\matuc\undefined
\else
% MAT C
\pars{Psalmus 1.}

\vspace{-4mm}

\antiphona{I g\textsuperscript{5}}{temporalia/ant-alleluia-auglx2.gtex}

%\vspace{-2mm}

\scriptura{Ps. 67, 2-11}

\initiumpsalmi{temporalia/ps67i-initium-i-g5.gtex}

\input{temporalia/ps67i-i-g.tex}

\vfill
\pagebreak

\pars{Psalmus 2.}

\scriptura{Ps. 67, 12-24}

%\vspace{-2mm}

\initiumpsalmi{temporalia/ps67ii-initium-i-g5.gtex}

\input{temporalia/ps67ii-i-g.tex}

\vfill
\pagebreak

\pars{Psalmus 3.}

\scriptura{Ps. 67, 25-36}

\initiumpsalmi{temporalia/ps67iii-initium-i-g5.gtex}

\input{temporalia/ps67iii-i-g.tex}

\vfill

\antiphona{}{temporalia/ant-alleluia-auglx2.gtex}

\vfill
\pagebreak
\fi
\ifx\matud\undefined
\else
% MAT D
\pars{Psalmus 1.}

\vspace{-4mm}

\antiphona{I d\textsuperscript{3}}{temporalia/ant-alleluia-auglx6.gtex}

%\vspace{-2mm}

\scriptura{Ps. 101, 2-12}

%\vspace{-2mm}

\initiumpsalmi{temporalia/ps101ii_xii-initium-i-d3-auto.gtex}

\input{temporalia/ps101ii_xii-i-d3.tex}

\vfill
\pagebreak

\pars{Psalmus 2.} \scriptura{Ps. 101, 13-23}

\vspace{-2mm}

\initiumpsalmi{temporalia/ps101xiii_xxiii-initium-i-d3-auto.gtex}

\input{temporalia/ps101xiii_xxiii-i-d3.tex}

\vfill
\pagebreak

\pars{Psalmus 3.} \scriptura{Ps. 101, 24-29}

%\vspace{-2mm}

\initiumpsalmi{temporalia/ps101iii-initium-i-d3-auto.gtex}

\input{temporalia/ps101iii-i-d3.tex}

\vfill

\antiphona{}{temporalia/ant-alleluia-auglx6.gtex}

\vfill
\pagebreak
\fi
\else
\matutinum
\fi

\pars{Versus.}

\ifx\matversus\undefined
\noindent \Vbardot{} Christus resúrgens ex mórtuis iam non móritur, allelúia.

\noindent \Rbardot{} Mors illi ultra non dominábitur, allelúia.
\else
\matversus
\fi

\vspace{5mm}

\sineinitiali{temporalia/oratiodominica-mat.gtex}

\vspace{5mm}

\pars{Absolutio.}

\cuminitiali{}{temporalia/absolutio-ipsius.gtex}

\vfill
\pagebreak

\cuminitiali{}{temporalia/benedictio-solemn-deus.gtex}

\vspace{7mm}

\lectioi

\noindent \Vbardot{} Tu autem, Dómine, miserére nobis.
\noindent \Rbardot{} Deo grátias.

\vfill
\pagebreak

\responsoriumi

\vfill
\pagebreak

\cuminitiali{}{temporalia/benedictio-solemn-christus.gtex}

\vspace{7mm}

\lectioii

\noindent \Vbardot{} Tu autem, Dómine, miserére nobis.
\noindent \Rbardot{} Deo grátias.

\vfill
\pagebreak

\responsoriumii

\vfill
\pagebreak

\cuminitiali{}{temporalia/benedictio-solemn-ignem.gtex}

\vspace{7mm}

\lectioiii

\noindent \Vbardot{} Tu autem, Dómine, miserére nobis.
\noindent \Rbardot{} Deo grátias.

\vfill
\pagebreak

\responsoriumiii

\vfill
\pagebreak

\rubrica{Reliqua omittuntur, nisi Laudes separandæ sint.}

\sineinitiali{temporalia/domineexaudi.gtex}

\vfill

\oratio

\vfill

\noindent \Vbardot{} Dómine, exáudi oratiónem meam.
\Rbardot{} Et clamor meus ad te véniat.

\vfill

\noindent \Vbardot{} Benedicámus Dómino.
\noindent \Rbardot{} Deo grátias.

\vfill

\noindent \Vbardot{} Fidélium ánimæ per misericórdiam Dei requiéscant in pace.
\Rbardot{} Amen.

\vfill
\pagebreak

\hora{Ad Laudes.} %%%%%%%%%%%%%%%%%%%%%%%%%%%%%%%%%%%%%%%%%%%%%%%%%%%%%

\cantusSineNeumas

\vspace{0.5cm}
\grechangedim{interwordspacetext}{0.18 cm plus 0.15 cm minus 0.05 cm}{scalable}%
\cuminitiali{}{temporalia/deusinadiutorium-communis.gtex}
\grechangedim{interwordspacetext}{0.22 cm plus 0.15 cm minus 0.05 cm}{scalable}%

\vfill
\pagebreak

\ifx\hymnuslaudes\undefined
\ifx\laudac\undefined
\else
\pars{Hymnus}

\cuminitiali{I}{temporalia/hym-ChorusNovae-praglia.gtex}
\fi
\ifx\laudbd\undefined
\else
\pars{Hymnus}

\cuminitiali{I}{temporalia/hym-ChorusNovae.gtex}
\fi
\else
\hymnuslaudes
\fi

\vfill
\pagebreak

\ifx\laudes\undefined
\ifx\lauda\undefined
\else
\pars{Psalmus 1.}

\vspace{-4mm}

\antiphona{IV* e}{temporalia/ant-alleluia-turco9.gtex}

\scriptura{Psalmus 23.}

\initiumpsalmi{temporalia/ps23-initium-iv_-e-auto.gtex}

\input{temporalia/ps23-iv_-e.tex} \Abardot{}

\vfill
\pagebreak

\pars{Psalmus 2.} \scriptura{Tob. 13, 10}

\vspace{-4mm}

\antiphona{VIII G}{temporalia/ant-benedicitedominumomneselecti.gtex}

\scriptura{Canticum Tobiæ, Tob. 13, 2-8}

\initiumpsalmi{temporalia/tobiae-initium-viii-g-auto.gtex}

\input{temporalia/tobiae-viii-g.tex} \Abardot{}

\vfill
\pagebreak

\pars{Psalmus 3.}

\vspace{-4mm}

\antiphona{E}{temporalia/ant-alleluia-praglia-e2.gtex}

%\vspace{-4mm}

\scriptura{Psalmus 32.}

%\vspace{-2mm}

\initiumpsalmi{temporalia/ps32-initium-e-auto.gtex}

\input{temporalia/ps32-e.tex}

\vfill

\antiphona{}{temporalia/ant-alleluia-praglia-e2.gtex}

\vfill
\pagebreak
\fi
\ifx\laudb\undefined
\else
\pars{Psalmus 1.}

\vspace{-4mm}

\antiphona{E}{temporalia/ant-alleluia-praglia-e.gtex}

\scriptura{Psalmus 42.}

\initiumpsalmi{temporalia/ps42-initium-e-e-auto.gtex}

\input{temporalia/ps42-e-e.tex} \Abardot{}

\vfill
\pagebreak

\pars{Psalmus 2.} \scriptura{Is. 38, 17}

\vspace{-4mm}

\antiphona{I g}{temporalia/ant-eruistidomine-tp.gtex}

%\vspace{-2mm}

\scriptura{Canticum Ezechiæ, Is. 38, 10-20}

%\vspace{-2mm}

\initiumpsalmi{temporalia/ezechiae-initium-i-g-auto.gtex}

%\vspace{-1.5mm}

\input{temporalia/ezechiae-i-g.tex}

\vfill

\antiphona{}{temporalia/ant-eruistidomine-tp.gtex}

\vfill
\pagebreak

\pars{Psalmus 3.}

\vspace{-4mm}

\antiphona{VIII c}{temporalia/ant-alleluia-turco16.gtex}

\vspace{-2mm}

\scriptura{Psalmus 64.}

\vspace{-2mm}

\initiumpsalmi{temporalia/ps64-initium-viii-C-auto.gtex}

\input{temporalia/ps64-viii-C.tex} \Abardot{}

\vfill
\pagebreak
\fi
\ifx\laudc\undefined
\else
\pars{Psalmus 1.}

\vspace{-4mm}

\antiphona{VI F}{temporalia/ant-alleluia-turco5.gtex}

\vspace{-2mm}

\scriptura{Psalmus 84.}

\vspace{-2mm}

\initiumpsalmi{temporalia/ps84-initium-vi-F-auto.gtex}

\input{temporalia/ps84-vi-F.tex} \Abardot{}

\vfill
\pagebreak

\pars{Psalmus 2.}

\vspace{-4mm}

\antiphona{VII d}{temporalia/ant-denoctespiritusmeus-tp.gtex}

\vspace{-2mm}

\scriptura{Canticum Isaiæ, Is. 26, 1-12}

\vspace{-2mm}

\initiumpsalmi{temporalia/isaiae3-initium-vii-d.gtex}

\input{temporalia/isaiae3-vii-d.tex} \Abardot{}

\vfill
\pagebreak

\pars{Psalmus 3.}

\vspace{-4mm}

\antiphona{E}{temporalia/ant-alleluia-praglia-e2.gtex}

%\vspace{-2mm}

\scriptura{Psalmus 66.}

%\vspace{-2mm}

\initiumpsalmi{temporalia/ps66-initium-e-auto.gtex}

\input{temporalia/ps66-e.tex} \Abardot{}

\vfill
\pagebreak
\fi
\ifx\laudd\undefined
\else
\pars{Psalmus 1.}

\vspace{-4mm}

\antiphona{VIII G}{temporalia/ant-alleluia-turco12.gtex}

\vspace{-2mm}

\scriptura{Psalmus 100.}

\vspace{-2mm}

\initiumpsalmi{temporalia/ps100-initium-viii-G-auto.gtex}

\input{temporalia/ps100-viii-G.tex} \Abardot{}

\vfill
\pagebreak

\pars{Psalmus 2.} \scriptura{Ps. 50, 19}

\vspace{-4mm}

\antiphona{I f}{temporalia/ant-sacrificiumdeo-tp.gtex}

%\vspace{-2mm}

\scriptura{Canticum Danielis, Dan. 3, 26.27.29.34-41}

%\vspace{-2mm}

\initiumpsalmi{temporalia/dan32-initium-i-f-auto.gtex}

\input{temporalia/dan32-i-f.tex} \Abardot{}

\vfill
\pagebreak

\pars{Psalmus 3.}

\vspace{-4mm}

\antiphona{VI F}{temporalia/ant-alleluia-turco5.gtex}

%\vspace{-2mm}

\scriptura{Psalmus 143, 1-10.}

%\vspace{-2mm}

\initiumpsalmi{temporalia/ps143i_x-initium-vi-F-auto.gtex}

\input{temporalia/ps143i_x-vi-F.tex} \Abardot{}

\vfill
\pagebreak
\fi
\else
\laudes
\fi

\ifx\lectiobrevis\undefined
\pars{Lectio Brevis.} \scriptura{Ac. 13, 30-33}

\noindent Deus suscitávit Iesum a mórtuis; qui visus est per dies multos his, qui simul ascénderant cum eo de Galilǽa in Ierúsalem, qui nunc sunt testes eius ad plebem. Et nos vobis evangelizámus eam, quæ ad patres promíssio facta est, quóniam hanc Deus adimplévit fíliis eórum, nobis resúscitans Iesum, sicut et in Psalmo secúndo scriptum est: Fílius meus es tu; ego hódie génui te.
\else
\lectiobrevis
\fi

\vfill

\ifx\responsoriumbreve\undefined
\pars{Responsorium breve.} \scriptura{Cf. Mt. 28, 6; Cf. Gal. 3, 13}

\cuminitiali{VI}{temporalia/resp-surrexitdominusdesepulcro.gtex}
\else
\responsoriumbreve
\fi

\vfill
\pagebreak

\benedictus

\vspace{-1cm}

\vfill
\pagebreak

\ifx\precestotum\undefined
\pars{Preces.}

\sineinitiali{}{temporalia/tonusprecum.gtex}

\ifx\preces\undefined
\ifx\lauda\undefined
\else
\noindent Exsultémus Christo, qui perémptum sui córporis templum sua virtúte restítuit,~\gredagger{} eíque supplicémus:

\Rbardot{} Fructus resurrectiónis tuæ, Dómine, nobis concéde.

\noindent Christe salvátor, qui in resurrectióne tua muliéribus et Apóstolis gáudium nuntiásti, totum orbem salvíficans,~\gredagger{} testes tuos nos éffice.

\Rbardot{} Fructus resurrectiónis tuæ, Dómine, nobis concéde.

\noindent Qui resurrectiónem ómnibus promisísti, qua ad vitam novam resurgerémus,~\gredagger{} Evangélii tui nos redde præcónes.

\Rbardot{} Fructus resurrectiónis tuæ, Dómine, nobis concéde.

\noindent Tu, qui Apóstolis sǽpius apparuísti et Sanctum eis Spíritum insufflásti,~\gredagger{} creatórem Spíritum rénova in nobis.

\Rbardot{} Fructus resurrectiónis tuæ, Dómine, nobis concéde.

\noindent Tu, qui discípulis tuis promisísti te cum eis mansúrum usque ad consummatiónem sǽculi,~\gredagger{} mane nobíscum hódie sempérque nobis adésto.

\Rbardot{} Fructus resurrectiónis tuæ, Dómine, nobis concéde.
\fi
\ifx\laudb\undefined
\else
\noindent Deum Patrem, cuius Agnus immaculátus tollit peccáta mundi nosque vivíficat,~\gredagger{} grati rogémus:

\Rbardot{} Auctor vitæ, vivífica nos.

\noindent Deus, auctor vitæ, meménto passiónis et resurrectiónis Agni, in cruce occísi,~\gredagger{} eúmque audi, semper interpellántem pro nobis.

\Rbardot{} Auctor vitæ, vivífica nos.

\noindent Expurgáto vétere ferménto malítiæ et nequítiæ,~\gredagger{} fac nos vívere in ázymis sinceritátis et veritátis Christi.

\Rbardot{} Auctor vitæ, vivífica nos.

\noindent Da, ut hódie reiciámus peccátum discórdiæ atque invídiæ,~\gredagger{} nosque redde fratrum necessitátibus magis inténtos.

\Rbardot{} Auctor vitæ, vivífica nos.

\noindent Spíritum evangélicum pone in médio nostri,~\gredagger{} ut hódie et semper in præcéptis tuis ambulémus.

\Rbardot{} Auctor vitæ, vivífica nos.
\fi
\ifx\laudc\undefined
\else
\noindent Exsultémus Christo, qui perémptum sui córporis templum sua virtúte restítuit,~\gredagger{} eíque supplicémus:

\Rbardot{} Fructus resurrectiónis tuæ, Dómine, nobis concéde.

\noindent Christe salvátor, qui in resurrectióne tua muliéribus et Apóstolis gáudium nuntiásti, totum orbem salvíficans,~\gredagger{} testes tuos nos éffice.

\Rbardot{} Fructus resurrectiónis tuæ, Dómine, nobis concéde.

\noindent Qui resurrectiónem ómnibus promisísti, qua ad vitam novam resurgerémus,~\gredagger{} Evangélii tui nos redde præcónes.

\Rbardot{} Fructus resurrectiónis tuæ, Dómine, nobis concéde.

\noindent Tu, qui Apóstolis sǽpius apparuísti et Sanctum eis Spíritum insufflásti,~\gredagger{} creatórem Spíritum rénova in nobis.

\Rbardot{} Fructus resurrectiónis tuæ, Dómine, nobis concéde.

\noindent Tu, qui discípulis tuis promisísti te cum eis mansúrum usque ad consummatiónem sǽculi,~\gredagger{} mane nobíscum hódie sempérque nobis adésto.

\Rbardot{} Fructus resurrectiónis tuæ, Dómine, nobis concéde.
\fi
\ifx\laudd\undefined
\else
\noindent Deum Patrem, cuius Agnus immaculátus tollit peccáta mundi nosque vivíficat,~\gredagger{} grati rogémus:

\Rbardot{} Auctor vitæ, vivífica nos.

\noindent Deus, auctor vitæ, meménto passiónis et resurrectiónis Agni, in cruce occísi,~\gredagger{} eúmque audi, semper interpellántem pro nobis.

\Rbardot{} Auctor vitæ, vivífica nos.

\noindent Expurgáto vétere ferménto malítiæ et nequítiæ,~\gredagger{} fac nos vívere in ázymis sinceritátis et veritátis Christi.

\Rbardot{} Auctor vitæ, vivífica nos.

\noindent Da, ut hódie reiciámus peccátum discórdiæ atque invídiæ,~\gredagger{} nosque redde fratrum necessitátibus magis inténtos.

\Rbardot{} Auctor vitæ, vivífica nos.

\noindent Spíritum evangélicum pone in médio nostri,~\gredagger{} ut hódie et semper in præcéptis tuis ambulémus.

\Rbardot{} Auctor vitæ, vivífica nos.
\fi
\else
\preces
\fi

\vfill

\pars{Oratio Dominica.}

\cuminitiali{}{temporalia/oratiodominicaalt.gtex}

\vfill
\pagebreak

\rubrica{vel:}

\pars{Supplicatio Litaniæ.}

\cuminitiali{}{temporalia/supplicatiolitaniae.gtex}

\vfill

\pars{Oratio Dominica.}

\cuminitiali{}{temporalia/oratiodominica.gtex}
\else
\precestotum
\fi

\vfill
\pagebreak

% Oratio. %%%
\oratio

\vspace{-1mm}

\vfill

\rubrica{Hebdomadarius dicit Dominus vobiscum, vel, absente sacerdote vel diacono, sic concluditur:}

\vspace{2mm}

\ifx\dominusnosbenedicat\undefined
\antiphona{C}{temporalia/dominusnosbenedicat.gtex}
\else
\dominusnosbenedicat
\fi

\rubrica{Postea cantatur a cantore:}

\vspace{2mm}

\ifx\benedicamuslaudes\undefined
\cuminitiali{VII}{temporalia/benedicamus-tempore-paschali.gtex}
\else
\benedicamuslaudes
\fi

\vspace{1mm}

\vfill
\pagebreak

\end{document}

