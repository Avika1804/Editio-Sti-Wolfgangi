\newcommand{\titulus}{\nomenFesti{S. Francisci de Sales, Episcopi \& Ecclesiæ Doctoris.}
\dies{Die 24. Ianuarii.}}
\newcommand{\oratio}{\pars{Oratio.}

\noindent Deus, qui ad animárum salútem beátum Francíscum, epíscopum, ómnibus ómnia factum esse voluísti, concéde propítius, ut, eius exémplo, tuæ mansuetúdinem caritátis in fratrum servítio semper ostendámus.

\pars{Pro pace in universo mundo.} \scriptura{Sir. 50, 25; 2 Esdr. 4, 20; \textbf{H416}}

\vspace{-4mm}

\antiphona{II D}{temporalia/ant-dapacemdomine.gtex}

\vfill

\noindent Deus, a quo sancta desidéria, recta consília et iusta sunt ópera: da servis tuis illam, quam mundus dare non potest, pacem; ut et corda nostra mandátis tuis dédita, et hóstium subláta formídine, témpora sint tua protectióne tranquílla.

\noindent Per Dóminum nostrum Iesum Christum, Fílium tuum, qui tecum vivit et regnat in unitáte Spíritus Sancti, Deus, per ómnia sǽcula sæculórum.

\noindent \Rbardot{} Amen.}
\newcommand{\invitatorium}{\pars{Invitatorium.}

\vspace{-4mm}

\antiphona{III}{temporalia/inv-christumpastorumprincipem.gtex}}
\newcommand{\hymnusmatutinum}{\pars{Hymnus}

\cuminitiali{III}{temporalia/hym-ChristePastorum.gtex}}
\newcommand{\matversus}{\noindent \Vbardot{} Audi, pópule meus, et loquar. 

\noindent \Rbardot{} Deus, Deus tuus ego sum.}
\newcommand{\lectioi}{\pars{Lectio I.} \scriptura{Dt. 26, 1-19}

\noindent De libro Deuteronómii.

\noindent In diébus illis: Locútus est Móyses pópulo dicens:

\noindent «Cum intráveris terram, quam Dóminus Deus tuus tibi datúrus est possidéndam, et obtinúeris eam atque habitáveris in illa, tolles primítias de cunctis frúgibus agri, quas collégeris de terra tua, quam Dóminus Deus tuus dabit tibi, et pones in cartállo pergésque ad locum, quem Dóminus Deus tuus elégerit, ut ibi hábitet nomen eius, accedésque ad sacerdótem, qui fúerit in diébus illis, et dices ad eum: “Profíteor hódie coram Dómino Deo tuo quod ingréssus sim terram, pro qua iurávit pátribus nostris, ut daret eam nobis”.

\noindent {\color{gray} Suscipiénsque sacérdos cartállum de manu tua ponet ante altáre Dómini Dei tui, et loquéris in conspéctu Dómini Dei tui: “Syrus vagus erat pater meus et descéndit in Ægýptum et ibi peregrinátus est in paucíssimo número; crevítque in gentem magnam ac robústam et infinítæ multitúdinis. Afflixerúntque nos Ægýptii et persecúti sunt imponéntes ónera gravíssima. Et clamávimus ad Dóminum, Deum patrum nostrórum, qui exaudívit nos et respéxit humilitátem nostram et labórem atque angústias, et edúxit nos Dóminus de Ægýpto in manu forti et bráchio exténto, in ingénti pavóre, in signis atque porténtis, et introdúxit ad locum istum et trádidit nobis terram hanc lacte et melle manántem. Et ecce nunc áttuli primítias frugum terræ, quam dedísti mihi, Dómine”. Et dimíttes eas in conspéctu Dómini Dei tui et adoráto Dómino Deo tuo. Et epuláberis in ómnibus bonis, quæ Dóminus Deus tuus déderit tibi et dómui tuæ, tu et Levítes et ádvena, qui tecum est.}

\noindent Quando compléveris décimam cunctárum frugum tuárum, anno tértio, anno decimárum, et déderis Levítæ et ádvenæ et pupíllo et víduæ, ut cómedant intra portas tuas et saturéntur, loquéris in conspéctu Dómini Dei tui: “Abstuli, quod sanctificátum est, de domo mea et dedi illud Levítæ et ádvenæ, pupíllo ac víduæ, sicut iussísti mihi; non præterívi mandáta tua, nec sum oblítus impérii tui, non comédi ex eis in luctu meo nec separávi ex eis in quálibet immundítia nec expéndi ex his quidquam mórtuo: obœdívi voci Dómini Dei mei et feci ómnia, sicut præcepísti mihi. Réspice de habitáculo sancto tuo, de cælo, et bénedic pópulo tuo Israel et terræ, quam dedísti nobis, sicut iurásti pátribus nostris, terræ lacte et melle manánti”.

\noindent Hódie Dóminus Deus tuus mandávit tibi, ut fácias præcépta hæc atque iudícia et custódias et ímpleas illa ex toto corde tuo et ex tota ánima tua. Dóminum elegísti hódie, ut sit tibi Deus, et ámbules in viis eius et custódias præcépta illíus et mandáta atque iudícia et obœ́dias eius império; et Dóminus elégit te hódie, ut sis ei pópulus peculiáris, sicut locútus est tibi, et custódias ómnia mandáta illíus, et fáciat te excelsiórem cunctis géntibus, quas creávit in laudem et nomen et glóriam suam, ut sis pópulus sanctus Dómini Dei tui, sicut locútus est».}
\newcommand{\responsoriumi}{\pars{Responsorium 1.} \scriptura{\Rbardot{} 1 Reg. 8, 28.29 \Vbardot{} Dt. 27, 15; \textbf{H396}}

\vspace{-5mm}

\responsorium{I}{temporalia/resp-audidominehymnum-CROCHU.gtex}{}}
\newcommand{\lectioii}{\pars{Lectio II.} \scriptura{Pars 1, cap. 3}

\noindent Ex Introductióne ad vitam devótam sancti Francísci de Sales epíscopi.

\noindent In ipsa rerum creatióne plantis mandávit Creátor Deus, ut unaquǽque fructum ferret secúndum genus suum: ita christiánis, qui vivæ Ecclésiæ suæ plantæ sunt, imperávit ut quisque devotiónis fructum pro qualitáte, statu et vocatióne sua proférret.

\noindent Diversímode, inquam, devótio est exercénda a nóbili vidélicet et ab opífice, a fámulo et a príncipe, a vídua et a puélla innúpta, ac coniugáta. Immo ne hoc quidem satis est; sed et ipsa devotiónis praxis, víribus, negótiis et offíciis cuiúsque particulátim est accommodánda.

\noindent Dic mihi, óbsecro, Philóthea mea, num consentáneum sit, ut epíscopi, Carthusianórum instar, solitúdinem consectári velint; coniugáti non magis de pecúlio suo adaugéndo sollíciti sint quam Capuccíni; religiósi in morem, ópifex totos dies in templo agat; et religiósus, ex contrário, ad omnes casus et evéntus, qui próximi necessitátem consequúntur, epíscopi exémplo, semper sit expósitus; an non devótio hæc ridícula, inordináta et intolerábilis esset?

\noindent Hic tamen error et absúrditas sæpíssime occúrrit. Nequáquam, Philóthea mea: étenim devótio, dúmmodo ea vera et sincéra sit, nihil omníno déstruit, sed cuncta pérficit et consúmmat, et si quando legítime alicúius vocatióni ac státui repúgnat et adversátur, haud dúbie falsa est.}
\newcommand{\responsoriumii}{\pars{Responsorium 2.} \scriptura{\Rbardot{} Ps. 88, 20.22 \Vbardot{} ibid., 21; \textbf{H378}}

\vspace{-5mm}

\responsorium{I}{temporalia/resp-posuiadiutorium-CROCHU.gtex}{}}
\newcommand{\lectioiii}{\pars{Lectio III.} \scriptura{Cap. 2-4: PG 26, 842-846}

\noindent Apis ita suum ex flóribus mel cólligit, ut eos mínime lædat aut déstruat, tam eos íntegros, intaminátos et recéntes relínquens atque eos répperit. Mélius vero id præstat vera devótio: nam ádeo nullum vocatiónis aut negotiórum genus déstruit, ut ea étiam exórnet ac condécoret.

\noindent Nam ut quǽlibet gemmæ, melli impósitæ, pro suo quæque colóre fulgentióres et rutilantióres sunt, sic in sua quisque vocatióne accéptior fit et perféctior, si devotióni illam coniúngat. Per hanc tibi enim et famíliæ cura longe rédditur quiétior, mútuus inter marítum et uxórem amor sincérior, obséquium príncipi débitum fidélius, et omnes quotquot sunt, occupatiónes suavióres et melióres.

\noindent Error proínde est, quin et hǽresis, devotiónis exercítium a cohórtibus militáribus, ab officína opíficum, ab aulis príncipum, a família coniugatórum velle exclúdere; fáteor quidem, caríssima Philóthea, devotiónem, quæ pure contemplatíva, monástica et religiósa est, hisce in offíciis et státibus mínime exercéri posse; at præter triplex hoc devotiónis genus, sunt et ália multa ad eos qui in státibus sæculáribus vivunt perficiéndos peraccommodáta.

\noindent Ubicúmque locórum ergo sumus, ad vitam perféctam aspiráre et debémus et póssumus.}
\newcommand{\responsoriumiii}{\pars{Responsorium 3.} \scriptura{\Rbardot{} Sir. 15, 3 \Vbardot{} Ps. 131, 18; \textbf{H62}}

\vspace{-5mm}

\responsorium{VII}{temporalia/resp-cibavitillum-CROCHU-cumdox.gtex}{}}
\newcommand{\hymnuslaudes}{\pars{Hymnus}

\cuminitiali{VIII}{temporalia/hym-InclitusPastor-doctorum.gtex}}
\newcommand{\lectiobrevis}{\pars{Lectio Brevis.} \scriptura{Sap. 7, 13-14}

\noindent Sapiéntiam sine fictióne dídici et sine invídia commúnico; divítias illíus non abscóndo. Infinítus enim thesáurus est homínibus; quem qui acquisiérunt, ad amicítiam in Deum se paravérunt propter disciplínæ dona commendáti.}
\newcommand{\responsoriumbreve}{\pars{Responsorium breve.} \scriptura{Eccli. 44, 15}

\cuminitiali{VI}{temporalia/resp-sapientiamsanctorum.gtex}}
\newcommand{\preces}{\noindent Christo, bono pastóri,~\gredagger{} qui pro suis óvibus ánimam pósuit,~\grestar{} laudes grati exsolvámus et supplicémus, dicéntes:

\Rbardot{} Pasce pópulum tuum, Dómine.

\noindent Christe, qui in sanctis pastóribus misericórdiam et dilectiónem tuam dignátus es osténdere,~\grestar{} numquam désinas per eos nobíscum misericórditer ágere.

\Rbardot{} Pasce pópulum tuum, Dómine.

\noindent Qui múnere pastóris animárum fungi per tuos vicários pergis,~\grestar{} ne destíteris nos ipse per rectóres nostros dirígere.

\Rbardot{} Pasce pópulum tuum, Dómine.

\noindent Qui in sanctis tuis, populórum dúcibus, córporum animarúmque médicus exstitísti,~\grestar{} numquam cesses ministérium in nos vitæ et sanctitátis perágere.

\Rbardot{} Pasce pópulum tuum, Dómine.

\noindent Qui, prudéntia et caritáte sanctórum, tuum gregem erudísti,~\grestar{} nos in sanctitáte iúgiter per pastóres nostros ædífica.

\Rbardot{} Pasce pópulum tuum, Dómine.}
\newcommand{\benedictus}{\pars{Canticum Zachariæ.} \scriptura{Ier. 3, 15}

\vspace{-4mm}

\antiphona{VIII G}{temporalia/ant-dabovobispastores.gtex}

\vspace{-1mm}

\scriptura{Lc. 1, 68-79}

\vspace{-2mm}

\cantusSineNeumas
\initiumpsalmi{temporalia/benedictus-initium-viii-G-auto.gtex}

%\vspace{-1.5mm}

\input{temporalia/benedictus-viii-G.tex} \Abardot{}}
\newcommand{\benedicamuslaudes}{\cuminitiali{}{temporalia/benedicamus-memoria-laudes.gtex}}
\newcommand{\hebdomada}{infra Hebdom. III post Oct. Paschæ.}
\newcommand{\oratioMatutinum}{\noindent Deus, qui errántibus, ut in viam possint redíre justítiæ, veritátis tuæ lumen osténdis:~\gredagger{} da cunctis qui christiána professióne censéntur, et illa respúere quæ huic inimíca sunt nómini;~\grestar{} et ea quæ sunt apta sectári. Per Dóminum.}
\newcommand{\oratioLaudes}{\cuminitiali{}{temporalia/oratio3.gtex}}

% LuaLaTeX

\documentclass[a4paper, twoside, 12pt]{article}
\usepackage[latin]{babel}
%\usepackage[landscape, left=3cm, right=1.5cm, top=2cm, bottom=1cm]{geometry} % okraje stranky
%\usepackage[landscape, a4paper, mag=1166, truedimen, left=2cm, right=1.5cm, top=1.6cm, bottom=0.95cm]{geometry} % okraje stranky
\usepackage[landscape, a4paper, mag=1400, truedimen, left=0.5cm, right=0.5cm, top=0.5cm, bottom=0.5cm]{geometry} % okraje stranky

\usepackage{fontspec}
\setmainfont[FeatureFile={junicode.fea}, Ligatures={Common, TeX}, RawFeature=+fixi]{Junicode}
%\setmainfont{Junicode}

% shortcut for Junicode without ligatures (for the Czech texts)
\newfontfamily\nlfont[FeatureFile={junicode.fea}, Ligatures={Common, TeX}, RawFeature=+fixi]{Junicode}

\usepackage{multicol}
\usepackage{color}
\usepackage{lettrine}
\usepackage{fancyhdr}

% usual packages loading:
\usepackage{luatextra}
\usepackage{graphicx} % support the \includegraphics command and options
\usepackage{gregoriotex} % for gregorio score inclusion
\usepackage{gregoriosyms}
\usepackage{wrapfig} % figures wrapped by the text
\usepackage{parcolumns}
\usepackage[contents={},opacity=1,scale=1,color=black]{background}
\usepackage{tikzpagenodes}
\usepackage{calc}
\usepackage{longtable}
\usetikzlibrary{calc}

\setlength{\headheight}{14.5pt}

% Commands used to produce a typical "Conventus" booklet

\newenvironment{titulusOfficii}{\begin{center}}{\end{center}}
\newcommand{\dies}[1]{#1

}
\newcommand{\nomenFesti}[1]{\textbf{\Large #1}

}
\newcommand{\celebratio}[1]{#1

}

\newcommand{\hora}[1]{%
\vspace{0.5cm}{\large \textbf{#1}}

\fancyhead[LE]{\thepage\ / #1}
\fancyhead[RO]{#1 / \thepage}
\addcontentsline{toc}{subsection}{#1}
}

% larger unit than a hora
\newcommand{\divisio}[1]{%
\begin{center}
{\Large \textsc{#1}}
\end{center}
\fancyhead[CO,CE]{#1}
\addcontentsline{toc}{section}{#1}
}

% a part of a hora, larger than pars
\newcommand{\subhora}[1]{
\begin{center}
{\large \textit{#1}}
\end{center}
%\fancyhead[CO,CE]{#1}
\addcontentsline{toc}{subsubsection}{#1}
}

% rubricated inline text
\newcommand{\rubricatum}[1]{\textit{#1}}

% standalone rubric
\newcommand{\rubrica}[1]{\vspace{3mm}\rubricatum{#1}}

\newcommand{\notitia}[1]{\textcolor{red}{#1}}

\newcommand{\scriptura}[1]{\hfill \small\textit{#1}}

\newcommand{\translatioCantus}[1]{\vspace{1mm}%
{\noindent\footnotesize \nlfont{#1}}}

% pruznejsi varianta nasledujiciho - umoznuje nastavit sirku sloupce
% s prekladem
\newcommand{\psalmusEtTranslatioB}[3]{
  \vspace{0.5cm}
  \begin{parcolumns}[colwidths={2=#3}, nofirstindent=true]{2}
    \colchunk{
      \input{#1}
    }

    \colchunk{
      \vspace{-0.5cm}
      {\footnotesize \nlfont
        \input{#2}
      }
    }
  \end{parcolumns}
}

\newcommand{\psalmusEtTranslatio}[2]{
  \psalmusEtTranslatioB{#1}{#2}{8.5cm}
}


\newcommand{\canticumMagnificatEtTranslatio}[1]{
  \psalmusEtTranslatioB{#1}{temporalia/extra-adventum-vespers/magnificat-boh.tex}{12cm}
}
\newcommand{\canticumBenedictusEtTranslatio}[1]{
  \psalmusEtTranslatioB{#1}{temporalia/extra-adventum-laudes/benedictus-boh.tex}{10.5cm}
}

% volne misto nad antifonami, kam si zpevaci dokresli neumy
\newcommand{\hicSuntNeumae}{\vspace{0.5cm}}

% prepinani mista mezi notovymi osnovami: pro neumovane a neneumovane zpevy
\newcommand{\cantusCumNeumis}{
  \setgrefactor{17}
  \global\advance\grespaceabovelines by 5mm%
}
\newcommand{\cantusSineNeumas}{
  \setgrefactor{17}
  \global\advance\grespaceabovelines by -5mm%
}

% znaky k umisteni nad inicialu zpevu
\newcommand{\superInitialam}[1]{\gresetfirstlineaboveinitial{\small {\textbf{#1}}}{\small {\textbf{#1}}}}

% pars officii, i.e. "oratio", ...
\newcommand{\pars}[1]{\textbf{#1}}

\newenvironment{psalmus}{
  \setlength{\parindent}{0pt}
  \setlength{\parskip}{5pt}
}{
  \setlength{\parindent}{10pt}
  \setlength{\parskip}{10pt}
}

%%%% Prejmenovat na latinske:
\newcommand{\nadpisZalmu}[1]{
  \hspace{2cm}\textbf{#1}\vspace{2mm}%
  \nopagebreak%

}

% mode, score, translation
\newcommand{\antiphona}[3]{%
\hicSuntNeumae
\superInitialam{#1}
\includescore{#2}

#3
}
 % Often used macros

\newcommand{\annusEditionis}{2021}

%%%% Vicekrat opakovane kousky

\newcommand{\anteOrationem}{
  \rubrica{Ante Orationem, cantatur a Superiore:}

  \pars{Supplicatio Litaniæ.}

  \cuminitiali{}{temporalia/supplicatiolitaniae.gtex}

  \pars{Oratio Dominica.}

  \cuminitiali{}{temporalia/oratiodominica.gtex}

  \rubrica{Deinde dicitur ab Hebdomadario:}

  \cuminitiali{}{temporalia/dominusvobiscum-solemnis.gtex}

  \rubrica{In choro monialium loco Dominus vobiscum dicitur:}

  \sineinitiali{temporalia/domineexaudi.gtex}
}

\setlength{\columnsep}{30pt} % prostor mezi sloupci

%%%%%%%%%%%%%%%%%%%%%%%%%%%%%%%%%%%%%%%%%%%%%%%%%%%%%%%%%%%%%%%%%%%%%%%%%%%%%%%%%%%%%%%%%%%%%%%%%%%%%%%%%%%%%
\begin{document}

% Here we set the space around the initial.
% Please report to http://home.gna.org/gregorio/gregoriotex/details for more details and options
\grechangedim{afterinitialshift}{2.2mm}{scalable}
\grechangedim{beforeinitialshift}{2.2mm}{scalable}
\grechangedim{interwordspacetext}{0.22 cm plus 0.15 cm minus 0.05 cm}{scalable}%
\grechangedim{annotationraise}{-0.2cm}{scalable}

% Here we set the initial font. Change 38 if you want a bigger initial.
% Emit the initials in red.
\grechangestyle{initial}{\color{red}\fontsize{38}{38}\selectfont}

\pagestyle{empty}

%%%% Titulni stranka
\begin{titulusOfficii}
\ifx\titulus\undefined
\nomenFesti{Feria III \hebdomada{}}
\else
\titulus
\fi
\end{titulusOfficii}

\vfill

\begin{center}
%Ad usum et secundum consuetudines chori \guillemotright{}Conventus Choralis\guillemotleft.

%Editio Sancti Wolfgangi \annusEditionis
\end{center}

\scriptura{}

\pars{}

\pagebreak

\renewcommand{\headrulewidth}{0pt} % no horiz. rule at the header
\fancyhf{}
\pagestyle{fancy}

\cantusSineNeumas

\hora{Ad Matutinum.} %%%%%%%%%%%%%%%%%%%%%%%%%%%%%%%%%%%%%%%%%%%%%%%%%%%%%

\vspace{2mm}

\cuminitiali{}{temporalia/dominelabiamea.gtex}

\vfill
%\pagebreak

\vspace{2mm}

\ifx\invitatorium\undefined
\pars{Invitatorium.} \scriptura{Lc. 24, 34; Psalmus 94; \textbf{H232}}

\vspace{-6mm}

\antiphona{VI}{temporalia/inv-surrexitdominusvere.gtex}
\else
\invitatorium
\fi

\vfill
\pagebreak

\ifx\hymnusmatutinum\undefined
\pars{Hymnus}

\cuminitiali{VIII}{temporalia/hym-LaetareCaelum.gtex}
\else
\hymnusmatutinum
\fi

\vspace{-3mm}

\vfill
\pagebreak

\ifx\matutinum\undefined
\ifx\matua\undefined
\else
% MAT A
\pars{Psalmus 1.}

\vspace{-4mm}

\antiphona{II D}{temporalia/ant-alleluia-turco7.gtex}

%\vspace{-2mm}

\scriptura{Ps. 9, 22-32}

%\vspace{-2mm}

\initiumpsalmi{temporalia/ps9xxii_xxxii-initium-ii-D-auto.gtex}

\input{temporalia/ps9xxii_xxxii-ii-D.tex}

\vfill
\pagebreak

\pars{Psalmus 2.} \scriptura{Ps. 9, 33-39}

%\vspace{-2mm}

\initiumpsalmi{temporalia/ps9xxxiii_xxxix-initium-ii-D-auto.gtex}

\input{temporalia/ps9xxxiii_xxxix-ii-D.tex}

\vfill
\pagebreak

\pars{Psalmus 3.} \scriptura{Ps. 11}

%\vspace{-2mm}

\initiumpsalmi{temporalia/ps11-initium-ii-D-auto.gtex}

\input{temporalia/ps11-ii-D.tex}

\vfill

\antiphona{}{temporalia/ant-alleluia-turco7.gtex}

\vfill
\pagebreak
\fi
\ifx\matub\undefined
\else
% MAT B
\pars{Psalmus 1.}

\vspace{-4mm}

\antiphona{VI F}{temporalia/ant-alleluia-turco6.gtex}

%\vspace{-2mm}

\scriptura{Ps. 36, 1-11}

%\vspace{-2mm}

\initiumpsalmi{temporalia/ps36i_xi-initium-vi-F-auto.gtex}

\input{temporalia/ps36i_xi-vi-F.tex}

\vfill
\pagebreak

\pars{Psalmus 2.}

\scriptura{Ps. 36, 12-29}

\vspace{-2mm}

\initiumpsalmi{temporalia/ps36xii_xxix-initium-vi-F-auto.gtex}

\input{temporalia/ps36xii_xxix-vi-F.tex}

\vfill
\pagebreak

\pars{Psalmus 3.}

\scriptura{Ps. 36, 30-40}

%\vspace{-2mm}

\initiumpsalmi{temporalia/ps36iii-initium-vi-F-auto.gtex}

\input{temporalia/ps36iii-vi-F.tex}

\antiphona{}{temporalia/ant-alleluia-turco6.gtex}

\vfill
\pagebreak
\fi
\ifx\matuc\undefined
\else
% MAT C
\pars{Psalmus 1.}

\vspace{-4mm}

\antiphona{I g\textsuperscript{5}}{temporalia/ant-alleluia-auglx2.gtex}

%\vspace{-2mm}

\scriptura{Ps. 67, 2-11}

\initiumpsalmi{temporalia/ps67i-initium-i-g5.gtex}

\input{temporalia/ps67i-i-g.tex}

\vfill
\pagebreak

\pars{Psalmus 2.}

\scriptura{Ps. 67, 12-24}

%\vspace{-2mm}

\initiumpsalmi{temporalia/ps67ii-initium-i-g5.gtex}

\input{temporalia/ps67ii-i-g.tex}

\vfill
\pagebreak

\pars{Psalmus 3.}

\scriptura{Ps. 67, 25-36}

\initiumpsalmi{temporalia/ps67iii-initium-i-g5.gtex}

\input{temporalia/ps67iii-i-g.tex}

\vfill

\antiphona{}{temporalia/ant-alleluia-auglx2.gtex}

\vfill
\pagebreak
\fi
\ifx\matud\undefined
\else
% MAT D
\pars{Psalmus 1.}

\vspace{-4mm}

\antiphona{I d\textsuperscript{3}}{temporalia/ant-alleluia-auglx6.gtex}

%\vspace{-2mm}

\scriptura{Ps. 101, 2-12}

%\vspace{-2mm}

\initiumpsalmi{temporalia/ps101ii_xii-initium-i-d3-auto.gtex}

\input{temporalia/ps101ii_xii-i-d3.tex}

\vfill
\pagebreak

\pars{Psalmus 2.} \scriptura{Ps. 101, 13-23}

\vspace{-2mm}

\initiumpsalmi{temporalia/ps101xiii_xxiii-initium-i-d3-auto.gtex}

\input{temporalia/ps101xiii_xxiii-i-d3.tex}

\vfill
\pagebreak

\pars{Psalmus 3.} \scriptura{Ps. 101, 24-29}

%\vspace{-2mm}

\initiumpsalmi{temporalia/ps101iii-initium-i-d3-auto.gtex}

\input{temporalia/ps101iii-i-d3.tex}

\vfill

\antiphona{}{temporalia/ant-alleluia-auglx6.gtex}

\vfill
\pagebreak
\fi
\else
\matutinum
\fi

\pars{Versus.}

\ifx\matversus\undefined
\noindent \Vbardot{} Christus resúrgens ex mórtuis iam non móritur, allelúia.

\noindent \Rbardot{} Mors illi ultra non dominábitur, allelúia.
\else
\matversus
\fi

\vspace{5mm}

\sineinitiali{temporalia/oratiodominica-mat.gtex}

\vspace{5mm}

\pars{Absolutio.}

\cuminitiali{}{temporalia/absolutio-ipsius.gtex}

\vfill
\pagebreak

\cuminitiali{}{temporalia/benedictio-solemn-deus.gtex}

\vspace{7mm}

\lectioi

\noindent \Vbardot{} Tu autem, Dómine, miserére nobis.
\noindent \Rbardot{} Deo grátias.

\vfill
\pagebreak

\responsoriumi

\vfill
\pagebreak

\cuminitiali{}{temporalia/benedictio-solemn-christus.gtex}

\vspace{7mm}

\lectioii

\noindent \Vbardot{} Tu autem, Dómine, miserére nobis.
\noindent \Rbardot{} Deo grátias.

\vfill
\pagebreak

\responsoriumii

\vfill
\pagebreak

\cuminitiali{}{temporalia/benedictio-solemn-ignem.gtex}

\vspace{7mm}

\lectioiii

\noindent \Vbardot{} Tu autem, Dómine, miserére nobis.
\noindent \Rbardot{} Deo grátias.

\vfill
\pagebreak

\responsoriumiii

\vfill
\pagebreak

\rubrica{Reliqua omittuntur, nisi Laudes separandæ sint.}

\sineinitiali{temporalia/domineexaudi.gtex}

\vfill

\oratio

\vfill

\noindent \Vbardot{} Dómine, exáudi oratiónem meam.
\Rbardot{} Et clamor meus ad te véniat.

\vfill

\noindent \Vbardot{} Benedicámus Dómino.
\noindent \Rbardot{} Deo grátias.

\vfill

\noindent \Vbardot{} Fidélium ánimæ per misericórdiam Dei requiéscant in pace.
\Rbardot{} Amen.

\vfill
\pagebreak

\hora{Ad Laudes.} %%%%%%%%%%%%%%%%%%%%%%%%%%%%%%%%%%%%%%%%%%%%%%%%%%%%%

\cantusSineNeumas

\vspace{0.5cm}
\grechangedim{interwordspacetext}{0.18 cm plus 0.15 cm minus 0.05 cm}{scalable}%
\cuminitiali{}{temporalia/deusinadiutorium-communis.gtex}
\grechangedim{interwordspacetext}{0.22 cm plus 0.15 cm minus 0.05 cm}{scalable}%

\vfill
\pagebreak

\ifx\hymnuslaudes\undefined
\ifx\laudac\undefined
\else
\pars{Hymnus}

\cuminitiali{I}{temporalia/hym-ChorusNovae-praglia.gtex}
\fi
\ifx\laudbd\undefined
\else
\pars{Hymnus}

\cuminitiali{I}{temporalia/hym-ChorusNovae.gtex}
\fi
\else
\hymnuslaudes
\fi

\vfill
\pagebreak

\ifx\laudes\undefined
\ifx\lauda\undefined
\else
\pars{Psalmus 1.}

\vspace{-4mm}

\antiphona{IV* e}{temporalia/ant-alleluia-turco9.gtex}

\scriptura{Psalmus 23.}

\initiumpsalmi{temporalia/ps23-initium-iv_-e-auto.gtex}

\input{temporalia/ps23-iv_-e.tex} \Abardot{}

\vfill
\pagebreak

\pars{Psalmus 2.} \scriptura{Tob. 13, 10}

\vspace{-4mm}

\antiphona{VIII G}{temporalia/ant-benedicitedominumomneselecti.gtex}

\scriptura{Canticum Tobiæ, Tob. 13, 2-8}

\initiumpsalmi{temporalia/tobiae-initium-viii-g-auto.gtex}

\input{temporalia/tobiae-viii-g.tex} \Abardot{}

\vfill
\pagebreak

\pars{Psalmus 3.}

\vspace{-4mm}

\antiphona{E}{temporalia/ant-alleluia-praglia-e2.gtex}

%\vspace{-4mm}

\scriptura{Psalmus 32.}

%\vspace{-2mm}

\initiumpsalmi{temporalia/ps32-initium-e-auto.gtex}

\input{temporalia/ps32-e.tex}

\vfill

\antiphona{}{temporalia/ant-alleluia-praglia-e2.gtex}

\vfill
\pagebreak
\fi
\ifx\laudb\undefined
\else
\pars{Psalmus 1.}

\vspace{-4mm}

\antiphona{E}{temporalia/ant-alleluia-praglia-e.gtex}

\scriptura{Psalmus 42.}

\initiumpsalmi{temporalia/ps42-initium-e-e-auto.gtex}

\input{temporalia/ps42-e-e.tex} \Abardot{}

\vfill
\pagebreak

\pars{Psalmus 2.} \scriptura{Is. 38, 17}

\vspace{-4mm}

\antiphona{I g}{temporalia/ant-eruistidomine-tp.gtex}

%\vspace{-2mm}

\scriptura{Canticum Ezechiæ, Is. 38, 10-20}

%\vspace{-2mm}

\initiumpsalmi{temporalia/ezechiae-initium-i-g-auto.gtex}

%\vspace{-1.5mm}

\input{temporalia/ezechiae-i-g.tex}

\vfill

\antiphona{}{temporalia/ant-eruistidomine-tp.gtex}

\vfill
\pagebreak

\pars{Psalmus 3.}

\vspace{-4mm}

\antiphona{VIII c}{temporalia/ant-alleluia-turco16.gtex}

\vspace{-2mm}

\scriptura{Psalmus 64.}

\vspace{-2mm}

\initiumpsalmi{temporalia/ps64-initium-viii-C-auto.gtex}

\input{temporalia/ps64-viii-C.tex} \Abardot{}

\vfill
\pagebreak
\fi
\ifx\laudc\undefined
\else
\pars{Psalmus 1.}

\vspace{-4mm}

\antiphona{VI F}{temporalia/ant-alleluia-turco5.gtex}

\vspace{-2mm}

\scriptura{Psalmus 84.}

\vspace{-2mm}

\initiumpsalmi{temporalia/ps84-initium-vi-F-auto.gtex}

\input{temporalia/ps84-vi-F.tex} \Abardot{}

\vfill
\pagebreak

\pars{Psalmus 2.}

\vspace{-4mm}

\antiphona{VII d}{temporalia/ant-denoctespiritusmeus-tp.gtex}

\vspace{-2mm}

\scriptura{Canticum Isaiæ, Is. 26, 1-12}

\vspace{-2mm}

\initiumpsalmi{temporalia/isaiae3-initium-vii-d.gtex}

\input{temporalia/isaiae3-vii-d.tex} \Abardot{}

\vfill
\pagebreak

\pars{Psalmus 3.}

\vspace{-4mm}

\antiphona{E}{temporalia/ant-alleluia-praglia-e2.gtex}

%\vspace{-2mm}

\scriptura{Psalmus 66.}

%\vspace{-2mm}

\initiumpsalmi{temporalia/ps66-initium-e-auto.gtex}

\input{temporalia/ps66-e.tex} \Abardot{}

\vfill
\pagebreak
\fi
\ifx\laudd\undefined
\else
\pars{Psalmus 1.}

\vspace{-4mm}

\antiphona{VIII G}{temporalia/ant-alleluia-turco12.gtex}

\vspace{-2mm}

\scriptura{Psalmus 100.}

\vspace{-2mm}

\initiumpsalmi{temporalia/ps100-initium-viii-G-auto.gtex}

\input{temporalia/ps100-viii-G.tex} \Abardot{}

\vfill
\pagebreak

\pars{Psalmus 2.} \scriptura{Ps. 50, 19}

\vspace{-4mm}

\antiphona{I f}{temporalia/ant-sacrificiumdeo-tp.gtex}

%\vspace{-2mm}

\scriptura{Canticum Danielis, Dan. 3, 26.27.29.34-41}

%\vspace{-2mm}

\initiumpsalmi{temporalia/dan32-initium-i-f-auto.gtex}

\input{temporalia/dan32-i-f.tex} \Abardot{}

\vfill
\pagebreak

\pars{Psalmus 3.}

\vspace{-4mm}

\antiphona{VI F}{temporalia/ant-alleluia-turco5.gtex}

%\vspace{-2mm}

\scriptura{Psalmus 143, 1-10.}

%\vspace{-2mm}

\initiumpsalmi{temporalia/ps143i_x-initium-vi-F-auto.gtex}

\input{temporalia/ps143i_x-vi-F.tex} \Abardot{}

\vfill
\pagebreak
\fi
\else
\laudes
\fi

\ifx\lectiobrevis\undefined
\pars{Lectio Brevis.} \scriptura{Ac. 13, 30-33}

\noindent Deus suscitávit Iesum a mórtuis; qui visus est per dies multos his, qui simul ascénderant cum eo de Galilǽa in Ierúsalem, qui nunc sunt testes eius ad plebem. Et nos vobis evangelizámus eam, quæ ad patres promíssio facta est, quóniam hanc Deus adimplévit fíliis eórum, nobis resúscitans Iesum, sicut et in Psalmo secúndo scriptum est: Fílius meus es tu; ego hódie génui te.
\else
\lectiobrevis
\fi

\vfill

\ifx\responsoriumbreve\undefined
\pars{Responsorium breve.} \scriptura{Cf. Mt. 28, 6; Cf. Gal. 3, 13}

\cuminitiali{VI}{temporalia/resp-surrexitdominusdesepulcro.gtex}
\else
\responsoriumbreve
\fi

\vfill
\pagebreak

\benedictus

\vspace{-1cm}

\vfill
\pagebreak

\ifx\precestotum\undefined
\pars{Preces.}

\sineinitiali{}{temporalia/tonusprecum.gtex}

\ifx\preces\undefined
\ifx\lauda\undefined
\else
\noindent Exsultémus Christo, qui perémptum sui córporis templum sua virtúte restítuit,~\gredagger{} eíque supplicémus:

\Rbardot{} Fructus resurrectiónis tuæ, Dómine, nobis concéde.

\noindent Christe salvátor, qui in resurrectióne tua muliéribus et Apóstolis gáudium nuntiásti, totum orbem salvíficans,~\gredagger{} testes tuos nos éffice.

\Rbardot{} Fructus resurrectiónis tuæ, Dómine, nobis concéde.

\noindent Qui resurrectiónem ómnibus promisísti, qua ad vitam novam resurgerémus,~\gredagger{} Evangélii tui nos redde præcónes.

\Rbardot{} Fructus resurrectiónis tuæ, Dómine, nobis concéde.

\noindent Tu, qui Apóstolis sǽpius apparuísti et Sanctum eis Spíritum insufflásti,~\gredagger{} creatórem Spíritum rénova in nobis.

\Rbardot{} Fructus resurrectiónis tuæ, Dómine, nobis concéde.

\noindent Tu, qui discípulis tuis promisísti te cum eis mansúrum usque ad consummatiónem sǽculi,~\gredagger{} mane nobíscum hódie sempérque nobis adésto.

\Rbardot{} Fructus resurrectiónis tuæ, Dómine, nobis concéde.
\fi
\ifx\laudb\undefined
\else
\noindent Deum Patrem, cuius Agnus immaculátus tollit peccáta mundi nosque vivíficat,~\gredagger{} grati rogémus:

\Rbardot{} Auctor vitæ, vivífica nos.

\noindent Deus, auctor vitæ, meménto passiónis et resurrectiónis Agni, in cruce occísi,~\gredagger{} eúmque audi, semper interpellántem pro nobis.

\Rbardot{} Auctor vitæ, vivífica nos.

\noindent Expurgáto vétere ferménto malítiæ et nequítiæ,~\gredagger{} fac nos vívere in ázymis sinceritátis et veritátis Christi.

\Rbardot{} Auctor vitæ, vivífica nos.

\noindent Da, ut hódie reiciámus peccátum discórdiæ atque invídiæ,~\gredagger{} nosque redde fratrum necessitátibus magis inténtos.

\Rbardot{} Auctor vitæ, vivífica nos.

\noindent Spíritum evangélicum pone in médio nostri,~\gredagger{} ut hódie et semper in præcéptis tuis ambulémus.

\Rbardot{} Auctor vitæ, vivífica nos.
\fi
\ifx\laudc\undefined
\else
\noindent Exsultémus Christo, qui perémptum sui córporis templum sua virtúte restítuit,~\gredagger{} eíque supplicémus:

\Rbardot{} Fructus resurrectiónis tuæ, Dómine, nobis concéde.

\noindent Christe salvátor, qui in resurrectióne tua muliéribus et Apóstolis gáudium nuntiásti, totum orbem salvíficans,~\gredagger{} testes tuos nos éffice.

\Rbardot{} Fructus resurrectiónis tuæ, Dómine, nobis concéde.

\noindent Qui resurrectiónem ómnibus promisísti, qua ad vitam novam resurgerémus,~\gredagger{} Evangélii tui nos redde præcónes.

\Rbardot{} Fructus resurrectiónis tuæ, Dómine, nobis concéde.

\noindent Tu, qui Apóstolis sǽpius apparuísti et Sanctum eis Spíritum insufflásti,~\gredagger{} creatórem Spíritum rénova in nobis.

\Rbardot{} Fructus resurrectiónis tuæ, Dómine, nobis concéde.

\noindent Tu, qui discípulis tuis promisísti te cum eis mansúrum usque ad consummatiónem sǽculi,~\gredagger{} mane nobíscum hódie sempérque nobis adésto.

\Rbardot{} Fructus resurrectiónis tuæ, Dómine, nobis concéde.
\fi
\ifx\laudd\undefined
\else
\noindent Deum Patrem, cuius Agnus immaculátus tollit peccáta mundi nosque vivíficat,~\gredagger{} grati rogémus:

\Rbardot{} Auctor vitæ, vivífica nos.

\noindent Deus, auctor vitæ, meménto passiónis et resurrectiónis Agni, in cruce occísi,~\gredagger{} eúmque audi, semper interpellántem pro nobis.

\Rbardot{} Auctor vitæ, vivífica nos.

\noindent Expurgáto vétere ferménto malítiæ et nequítiæ,~\gredagger{} fac nos vívere in ázymis sinceritátis et veritátis Christi.

\Rbardot{} Auctor vitæ, vivífica nos.

\noindent Da, ut hódie reiciámus peccátum discórdiæ atque invídiæ,~\gredagger{} nosque redde fratrum necessitátibus magis inténtos.

\Rbardot{} Auctor vitæ, vivífica nos.

\noindent Spíritum evangélicum pone in médio nostri,~\gredagger{} ut hódie et semper in præcéptis tuis ambulémus.

\Rbardot{} Auctor vitæ, vivífica nos.
\fi
\else
\preces
\fi

\vfill

\pars{Oratio Dominica.}

\cuminitiali{}{temporalia/oratiodominicaalt.gtex}

\vfill
\pagebreak

\rubrica{vel:}

\pars{Supplicatio Litaniæ.}

\cuminitiali{}{temporalia/supplicatiolitaniae.gtex}

\vfill

\pars{Oratio Dominica.}

\cuminitiali{}{temporalia/oratiodominica.gtex}
\else
\precestotum
\fi

\vfill
\pagebreak

% Oratio. %%%
\oratio

\vspace{-1mm}

\vfill

\rubrica{Hebdomadarius dicit Dominus vobiscum, vel, absente sacerdote vel diacono, sic concluditur:}

\vspace{2mm}

\ifx\dominusnosbenedicat\undefined
\antiphona{C}{temporalia/dominusnosbenedicat.gtex}
\else
\dominusnosbenedicat
\fi

\rubrica{Postea cantatur a cantore:}

\vspace{2mm}

\ifx\benedicamuslaudes\undefined
\cuminitiali{VII}{temporalia/benedicamus-tempore-paschali.gtex}
\else
\benedicamuslaudes
\fi

\vspace{1mm}

\vfill
\pagebreak

\end{document}

