\newcommand{\titulus}{\dies{11. Februarii.}
\nomenFesti{B. Mariæ Virginis de Lourdes.}}
\newcommand{\oratio}{\pars{Oratio.}

\noindent Concéde, miséricors Deus, fragilitáti nostræ præsídium, ut, qui Immaculátæ Dei Genetrícis memóriam ágimus, intercessiónis eius auxílio, a nostris iniquitátibus resurgámus.

\pars{Pro pace in Ucraina.} \scriptura{Sir. 50, 25; 2 Esdr. 4, 20; \textbf{H416}}

\vspace{-4mm}

\antiphona{II D}{temporalia/ant-dapacemdomine.gtex}

\vfill

\noindent Deus, a quo sancta desidéria, recta consília et iusta sunt ópera: da servis tuis illam, quam mundus dare non potest, pacem; ut et corda nostra mandátis tuis dédita, et hóstium subláta formídine, témpora sint tua protectióne tranquílla.

\noindent Per Dóminum nostrum Iesum Christum, Fílium tuum, qui tecum vivit et regnat in unitáte Spíritus Sancti, Deus, per ómnia sǽcula sæculórum.

\noindent \Rbardot{} Amen.}
\newcommand{\invitatorium}{\pars{Invitatorium.}

\vspace{-4mm}

\antiphona{V}{temporalia/inv-christummariaefilium.gtex}}
\newcommand{\hymnusmatutinum}{\pars{Hymnus.}

\vspace{-5mm}

{
\grechangedim{interwordspacetext}{0.30 cm plus 0.15 cm minus 0.05 cm}{scalable}%
\antiphona{VIII}{temporalia/hym-OVirgoMater.gtex}
\grechangedim{interwordspacetext}{0.22 cm plus 0.15 cm minus 0.05 cm}{scalable}%
}}
\newcommand{\lectioi}{\pars{Lectio I.} \scriptura{Gn. 8, 1-22}

\noindent De libro Génesis.

\noindent Recordátus autem Deus Noe cunctorúmque animántium et ómnium iumentórum, quæ erant cum eo in arca, addúxit spíritum super terram, et imminútæ sunt aquæ. Et clausi sunt fontes abýssi et cataráctæ cæli, et prohíbitæ sunt plúviæ de cælo. Reversǽque sunt aquæ de terra eúntes et redeúntes et cœpérunt mínui post centum quinquagínta dies. Requievítque arca mense séptimo, décima séptima die mensis super montes Ararat. At vero aquæ ibant et decrescébant usque ad décimum mensem; décimo enim mense, prima die mensis, apparuérunt cacúmina móntium.

\noindent Cumque transíssent quadragínta dies, apériens Noe fenéstram arcæ, quam fécerat, dimísit corvum; qui egrediebátur éxiens et rédiens, donec siccaréntur aquæ super terram.

\noindent Emísit quoque colúmbam a se, ut vidéret si iam cessássent aquæ super fáciem terræ. Quæ, cum non invenísset, ubi requiésceret pes eius, revérsa est ad eum in arcam; aquæ enim erant super univérsam terram. Extendítque manum et apprehénsam íntulit in arcam.

\noindent Exspectátis autem ultra septem diébus áliis, rursum dimísit colúmbam ex arca. At illa venit ad eum ad vésperam portans ramum olívæ viréntibus fóliis in ore suo. Intelléxit ergo Noe quod cessassent aquæ super terram. Exspectavítque nihilóminus septem álios dies; et emísit colúmbam, quæ non est revérsa ultra ad eum.

\noindent {\color{gray} Igitur sescentésimo primo anno, primo mense, prima die mensis, siccátæ sunt aquæ super terram; et apériens Noe tectum arcæ, et ecce aspéxit vidítque quod exsiccáta erat superfícies terræ. Mense secúndo, séptima et vicésima die mensis, arefácta est terra.}

\noindent Locútus est autem Deus ad Noe dicens: “Egrédere de arca tu et uxor tua, fílii tui et uxóres filiórum tuórum tecum. Cuncta animántia, quæ sunt apud te ex omni carne, tam in volatílibus quam in pecóribus et in univérsis reptílibus, quæ reptant super terram, educ tecum, ut púllulent super terram et crescant et multiplicéntur super eam”.

\noindent {\color{gray} Egréssus est ergo Noe et fílii eius, uxor illíus et uxóres filiórum eius cum eo. Sed et ómnia animántia, iuménta, volatília et reptília, quæ reptant super terram, secúndum genus suum egréssa sunt de arca.}

\noindent Ædificávit autem Noe altáre Dómino; et tollens de cunctis pecóribus mundis et volúcribus mundis óbtulit holocáusta super altáre.

\noindent Odoratúsque est Dóminus odórem suavitátis et locútus est Dóminus ad cor suum: “Nequáquam ultra maledícam terræ propter hómines, quia cogitátio humáni cordis in malum prona est ab adulescéntia sua. Non ígitur ultra percútiam omnem ánimam vivéntem, sicut feci. Cunctis diébus terræ, seméntis et messis, frigus et æstus, æstas et hiems, dies et nox non requiéscent”.}
\newcommand{\responsoriumi}{\pars{Responsorium 1.}

\vspace{-5mm}

\responsorium{VIII}{temporalia/resp-emisitnoecolumbam.gtex}{}

\rubrica{vel ad libitum:} \pars{Responsorium 1.} \scriptura{\Rbardot{} Gn. 8, 20 \Vbardot{} ibid. 9, 20; \textbf{H140}}

\vspace{-5mm}

\responsorium{III}{temporalia/resp-aedificavitnoealtare-CROCHU.gtex}{}}
\newcommand{\lectioii}{\pars{Lectio II.} \scriptura{Ep. ad P. Gondrand, a. 1861: cf. A. Ravier, Les écrits de sainte Bernadette, Paris 1961, pp. 53-59}

\noindent Ex Epístola sanctæ Maríæ Bernárdæ Soubirous vírginis.

\noindent Quadam die, cum me contulíssem ad ripam flúminis Gavi ut ligna collígerem cum duábus puéllis, rumórem quendam audívi. Me verti ad pratum, sed árbores vidi mínime agitári. Unde caput levávi et antrum aspéxi. Dóminam autem vidi véstibus albis indútam: cándido enim hábitu erat amícta zonáque cærúlea cincta, et gilvam super utróque pede rosam habébat, quæ eiúsdem colóris erat ac coróna eius rosárii. Quæ cum vidi, óculos perfrícui, putans me falli; manus autem in vestis sinu insérui, ubi meam invéni corónam rosárii. Vólui étiam frontem cruce signáre, sed manum illuc attóllere non válui, quæ décidit. Cum vero Dómina illa signum fecísset crucis, ego quoque, treménte licet manu, conáta sum, et tandem pótui. Simul rosárium recitáre cœpi, ipsa quoque Dómina corónæ rosárii sui volvénte gránula nec tamen lábia movénte. Cum rosário finem dedi, vísio statim evánuit. Quæsívi ígitur a duábus puéllis num quidquam conspexíssent: quod illæ negárunt; quin étiam interrogavérunt quid habérem sibi revelándum. Quas certióres feci vidísse me Dóminam albis vestiméntis indútam, nescíre autem quæ esset; sed ut hoc tacérent admónui. Hortátæ sunt me dein et illæ, ne illuc redírem; quod ego recusávi. Revérsa sum ígitur die domínico, cum intérius me ciéri sentírem.}
\newcommand{\responsoriumii}{\pars{Responsorium 2.} \scriptura{\textbf{H306}}

\vspace{-5mm}

\responsorium{I}{temporalia/resp-conceptiogloriosae-CROCHU.gtex}{}}
\newcommand{\lectioiii}{\pars{Lectio III.}

\noindent Dómina illa nónnisi tértium mihi locúta est, atque rogávit num ire ad se per dies quíndecim vellem. Quod me velle respóndi. Adiécit autem illa debére a me presbýteros admonéri ut sacéllum ibídem ædificándum curárent; deínde iussit ut e fonte bíberem. Cum nullum conspícerem fontem, ibam ad flúvium Gavum; at ipsa significávit non de illo se loqui, et dígito fontem monstrávit. Cumque ad hunc adiíssem, non invéni nisi parum lutuléntæ aquæ. Admóta manu, nihil cápere pótui; unde scálpere cœpi, ac tandem paulum aquæ hauríre valens, ter proiéci, quarta autem vice bíbere pótui. Vísio dein dilápsa est et ego recéssi. Per dies vero quíndecim illuc rédii, atque Dómina síngulis diébus, præter quandam fériam secúndam et fériam sextam, mihi appáruit, idéntidem mandans debére me presbýteros monére de sacéllo ibídem erigéndo, et fontem ad me lavándam pétere, et pro peccatórum conversióne deprecári. Plúries quidem eam interrogávi quæ esset, at illa léniter arridébat; demum suspénsa tenens bráchia oculósque in cælum élevans, dixit mihi se esse Immaculátam Conceptiónem. Intra quíndecim dies illos tria quoque mihi secréta patefécit, quæ omníno ne cuíquam pánderem interdíxit; quod fidéliter hucúsque servávi.}
\newcommand{\responsoriumiii}{\pars{Responsorium 3.} \scriptura{\Rbardot{} Cantor \Vbardot{} Lc. 1, 42; \textbf{H307}}

\vspace{-5mm}

\responsorium{I}{temporalia/resp-conceptiotua-CROCHU-cumdox.gtex}{}}
\newcommand{\matversus}{\noindent \Vbardot{} María conservábat ómnia verba hæc.

\noindent \Rbardot{} Cónferens in corde suo.}
\newcommand{\hymnuslaudes}{\pars{Hymnus}

\cuminitiali{I}{temporalia/hym-AveMarisStella.gtex}}
\newcommand{\lectiobrevis}{\pars{Lectio Brevis.} \scriptura{Ap. 12, 1}

\noindent Signum magnum appáruit in cælo: múlier amícta sole, et luna sub pédibus eius, et super caput eius coróna stellárum duódecim.}
\newcommand{\responsoriumbreve}{\pars{Responsorium breve.} \scriptura{Lc. 1, 28}

\cuminitiali{VI}{temporalia/resp-avemaria-alt.gtex}}
\newcommand{\benedictus}{\pars{Canticum Zachariæ.} \scriptura{Cf. Mal. 4, 2; Lc. 1, 78}

\vspace{-4mm}

{
\grechangedim{interwordspacetext}{0.18 cm plus 0.15 cm minus 0.05 cm}{scalable}%
\antiphona{VIII G}{temporalia/ant-praeclarasalutisaurora.gtex}
\grechangedim{interwordspacetext}{0.22 cm plus 0.15 cm minus 0.05 cm}{scalable}%
}

%\vspace{-3mm}

\scriptura{Lc. 1, 68-79}

%\vspace{-2mm}

\cantusSineNeumas
\initiumpsalmi{temporalia/benedictus-initium-viii-G-auto.gtex}

%\vspace{-1.5mm}

\input{temporalia/benedictus-viii-G.tex} \Abardot{}}
\newcommand{\preces}{\noindent Salvatórem nostrum celebrántes, \gredagger{} qui ex María Vírgine nasci dignátus est, \grestar{} exorémus dicéntes:

\Rbardot{} Intercédat pro nobis mater tua, Dómine.

\noindent O sol iustítiæ, \gredagger{} quem Immaculáta Virgo ut lucens auróra præcéssit, \grestar{} tríbue ut in lúmine visitatiónis tuæ semper ambulémus.

\Rbardot{} Intercédat pro nobis mater tua, Dómine.

\noindent Verbum ætérnum, \gredagger{} quod Maríam habitatiónis tuæ arcam incorruptíbilem elegísti, \grestar{} líbera nos a corruptióne peccáti.

\Rbardot{} Intercédat pro nobis mater tua, Dómine.

\noindent Salvátor noster, qui iuxta crucem matrem tuam habuísti, \gredagger{} præsta ut, ipsa intercedénte, \grestar{} communicántes tuis passiónibus gaudeámus.

\Rbardot{} Intercédat pro nobis mater tua, Dómine.

\noindent Benigníssime Iesu, \gredagger{} qui pendens in cruce, Maríam Ioánni matrem dedísti, \grestar{} da nobis ita vívere ut eius fílii agnoscámur.

\Rbardot{} Intercédat pro nobis mater tua, Dómine.}
\newcommand{\benedicamuslaudes}{\cuminitiali{I}{temporalia/benedicamus-festis-bmv.gtex}}
\newcommand{\hebdomada}{in Rogationibus.}
\newcommand{\oratioMatutinum}{\noindent Deus, a quo bona cuncta procédunt, largíre supplícibus tuis:~\gredagger{} ut cogitémus, te inspiránte, quæ recta sunt;~\grestar{} et, te gubernánte, éadem faciámus. Per Dóminum.}
\newcommand{\oratioLaudes}{\cuminitiali{}{temporalia/oratio5.gtex}}

% LuaLaTeX

\documentclass[a4paper, twoside, 12pt]{article}
\usepackage[latin]{babel}
%\usepackage[landscape, left=3cm, right=1.5cm, top=2cm, bottom=1cm]{geometry} % okraje stranky
%\usepackage[landscape, a4paper, mag=1166, truedimen, left=2cm, right=1.5cm, top=1.6cm, bottom=0.95cm]{geometry} % okraje stranky
\usepackage[landscape, a4paper, mag=1400, truedimen, left=0.5cm, right=0.5cm, top=0.5cm, bottom=0.5cm]{geometry} % okraje stranky

\usepackage{fontspec}
\setmainfont[FeatureFile={junicode.fea}, Ligatures={Common, TeX}, RawFeature=+fixi]{Junicode}
%\setmainfont{Junicode}

% shortcut for Junicode without ligatures (for the Czech texts)
\newfontfamily\nlfont[FeatureFile={junicode.fea}, Ligatures={Common, TeX}, RawFeature=+fixi]{Junicode}

% Hebrew font:
% http://scripts.sil.org/cms/scripts/page.php?site_id=nrsi&id=SILHebrUnic2
\newfontfamily\hebfont[Scale=1]{Ezra SIL}

\usepackage{multicol}
\usepackage{color}
\usepackage{lettrine}
\usepackage{fancyhdr}

% usual packages loading:
\usepackage{luatextra}
\usepackage{graphicx} % support the \includegraphics command and options
\usepackage{gregoriotex} % for gregorio score inclusion
\usepackage{gregoriosyms}
\usepackage{wrapfig} % figures wrapped by the text
\usepackage{parcolumns}
\usepackage[contents={},opacity=1,scale=1,color=black]{background}
\usepackage{tikzpagenodes}
\usepackage{calc}
\usepackage{longtable}
\usetikzlibrary{calc}

\setlength{\headheight}{14.5pt}

% Commands used to produce a typical "Conventus" booklet

\newenvironment{titulusOfficii}{\begin{center}}{\end{center}}
\newcommand{\dies}[1]{#1

}
\newcommand{\nomenFesti}[1]{\textbf{\Large #1}

}
\newcommand{\celebratio}[1]{#1

}

\newcommand{\hora}[1]{%
\vspace{0.5cm}{\large \textbf{#1}}

\fancyhead[LE]{\thepage\ / #1}
\fancyhead[RO]{#1 / \thepage}
\addcontentsline{toc}{subsection}{#1}
}

% larger unit than a hora
\newcommand{\divisio}[1]{%
\begin{center}
{\Large \textsc{#1}}
\end{center}
\fancyhead[CO,CE]{#1}
\addcontentsline{toc}{section}{#1}
}

% a part of a hora, larger than pars
\newcommand{\subhora}[1]{
\begin{center}
{\large \textit{#1}}
\end{center}
%\fancyhead[CO,CE]{#1}
\addcontentsline{toc}{subsubsection}{#1}
}

% rubricated inline text
\newcommand{\rubricatum}[1]{\textit{#1}}

% standalone rubric
\newcommand{\rubrica}[1]{\vspace{3mm}\rubricatum{#1}}

\newcommand{\notitia}[1]{\textcolor{red}{#1}}

\newcommand{\scriptura}[1]{\hfill \small\textit{#1}}

\newcommand{\translatioCantus}[1]{\vspace{1mm}%
{\noindent\footnotesize \nlfont{#1}}}

% pruznejsi varianta nasledujiciho - umoznuje nastavit sirku sloupce
% s prekladem
\newcommand{\psalmusEtTranslatioB}[3]{
  \vspace{0.5cm}
  \begin{parcolumns}[colwidths={2=#3}, nofirstindent=true]{2}
    \colchunk{
      \input{#1}
    }

    \colchunk{
      \vspace{-0.5cm}
      {\footnotesize \nlfont
        \input{#2}
      }
    }
  \end{parcolumns}
}

\newcommand{\psalmusEtTranslatio}[2]{
  \psalmusEtTranslatioB{#1}{#2}{8.5cm}
}


\newcommand{\canticumMagnificatEtTranslatio}[1]{
  \psalmusEtTranslatioB{#1}{temporalia/extra-adventum-vespers/magnificat-boh.tex}{12cm}
}
\newcommand{\canticumBenedictusEtTranslatio}[1]{
  \psalmusEtTranslatioB{#1}{temporalia/extra-adventum-laudes/benedictus-boh.tex}{10.5cm}
}

% volne misto nad antifonami, kam si zpevaci dokresli neumy
\newcommand{\hicSuntNeumae}{\vspace{0.5cm}}

% prepinani mista mezi notovymi osnovami: pro neumovane a neneumovane zpevy
\newcommand{\cantusCumNeumis}{
  \setgrefactor{17}
  \global\advance\grespaceabovelines by 5mm%
}
\newcommand{\cantusSineNeumas}{
  \setgrefactor{17}
  \global\advance\grespaceabovelines by -5mm%
}

% znaky k umisteni nad inicialu zpevu
\newcommand{\superInitialam}[1]{\gresetfirstlineaboveinitial{\small {\textbf{#1}}}{\small {\textbf{#1}}}}

% pars officii, i.e. "oratio", ...
\newcommand{\pars}[1]{\textbf{#1}}

\newenvironment{psalmus}{
  \setlength{\parindent}{0pt}
  \setlength{\parskip}{5pt}
}{
  \setlength{\parindent}{10pt}
  \setlength{\parskip}{10pt}
}

%%%% Prejmenovat na latinske:
\newcommand{\nadpisZalmu}[1]{
  \hspace{2cm}\textbf{#1}\vspace{2mm}%
  \nopagebreak%

}

% mode, score, translation
\newcommand{\antiphona}[3]{%
\hicSuntNeumae
\superInitialam{#1}
\includescore{#2}

#3
}
 % Often used macros

\newcommand{\annusEditionis}{2021}

\def\hebinitial#1{%
\leavevmode{\newbox\hebbox\setbox\hebbox\hbox{\hebfont{#1}\hskip 1mm}\kern -\wd\hebbox\hbox{\hebfont{#1}\hskip 1mm}}%
}

%%%% Vicekrat opakovane kousky

\newcommand{\anteOrationem}{
  \rubrica{Ante Orationem, cantatur a Superiore:}

  \pars{Supplicatio Litaniæ.}

  \cuminitiali{}{temporalia/supplicatiolitaniae.gtex}

  \pars{Oratio Dominica.}

  \cuminitiali{}{temporalia/oratiodominica.gtex}
}

\setlength{\columnsep}{30pt} % prostor mezi sloupci

%%%%%%%%%%%%%%%%%%%%%%%%%%%%%%%%%%%%%%%%%%%%%%%%%%%%%%%%%%%%%%%%%%%%%%%%%%%%%%%%%%%%%%%%%%%%%%%%%%%%%%%%%%%%%
\begin{document}

% Here we set the space around the initial.
% Please report to http://home.gna.org/gregorio/gregoriotex/details for more details and options
\grechangedim{afterinitialshift}{2.2mm}{scalable}
\grechangedim{beforeinitialshift}{2.2mm}{scalable}

\grechangedim{interwordspacetext}{0.22 cm plus 0.15 cm minus 0.05 cm}{scalable}%
\grechangedim{annotationraise}{-0.2cm}{scalable}

% Here we set the initial font. Change 38 if you want a bigger initial.
% Emit the initials in red.
\grechangestyle{initial}{\color{red}\fontsize{38}{38}\selectfont}

\pagestyle{empty}

%%%% Titulni stranka
\begin{titulusOfficii}
\ifx\titulus\undefined
\nomenFesti{Sabbato \hebdomada{}}
\else
\titulus
\fi
\end{titulusOfficii}

\vfill

\pars{}

\scriptura{}

\pagebreak

% graphic
\renewcommand{\headrulewidth}{0pt} % no horiz. rule at the header
\fancyhf{}
\pagestyle{fancy}

\cantusSineNeumas

\hora{Ad Matutinum.}

\vspace{2mm}

\cuminitiali{}{temporalia/dominelabiamea.gtex}

\vspace{2mm}

\ifx\invitatorium\undefined
\pars{Invitatorium.} \scriptura{\textbf{H14}}

\vspace{-6mm}

\antiphona{VI}{temporalia/inv-regemventurumsimplex.gtex}
\else
\invitatorium
\fi

\vfill
\pagebreak

\ifx\hymnusmatutinum\undefined
\pars{Hymnus.}

\vspace{-5mm}

\antiphona{II}{temporalia/hym-VerbumSupernum.gtex}
\else
\hymnusmatutinum
\fi

\vfill
\pagebreak

\ifx\matutinum\undefined
\ifx\matua\undefined
\else
% MAT A
\pars{Psalmus 1.} \scriptura{Ps. 104, 3; \textbf{H99}}

\vspace{-6mm}

\antiphona{D}{temporalia/ant-laeteturcor.gtex}

\vspace{-4mm}

\scriptura{Ps. 104, 1-15}

\vspace{-2mm}

\initiumpsalmi{temporalia/ps104i-initium-d-g-auto.gtex}

\vspace{-1.5mm}

\input{temporalia/ps104i-d-g.tex} \Abardot{}

\vfill
\pagebreak

\pars{Psalmus 2.} \scriptura{Ps. 113, 1; \textbf{H94}}

\vspace{-4mm}

\antiphona{VIII a}{temporalia/ant-domusiacob.gtex}

%\vspace{-2mm}

\scriptura{Ps. 104, 16-27}

%\vspace{-2mm}

\initiumpsalmi{temporalia/ps104ii-initium-viii-a-auto.gtex}

\input{temporalia/ps104ii-viii-a.tex} \Abardot{}

\vfill
\pagebreak

\pars{Psalmus 3.} \scriptura{Ps. 104, 43}

\vspace{-4mm}

\antiphona{IV E}{temporalia/ant-eduxitdeus.gtex}

%\vspace{-2mm}

\scriptura{Ps. 104, 28-45}

%\vspace{-2mm}

\initiumpsalmi{temporalia/ps104iii-initium-iv-E-auto.gtex}

\input{temporalia/ps104iii-iv-E.tex}

\vfill

\antiphona{}{temporalia/ant-eduxitdeus.gtex}

\vfill
\pagebreak\fi
\ifx\matub\undefined
\else
% MAT B
\pars{Psalmus 1.} \scriptura{Ps. 105, 4; \textbf{H100}}

\vspace{-4mm}

\antiphona{E}{temporalia/ant-visitanos.gtex}

%\vspace{-2mm}

\scriptura{Ps. 105, 1-15}

%\vspace{-2mm}

\initiumpsalmi{temporalia/ps105i-initium-e.gtex}

\input{temporalia/ps105i-e.tex}

\vfill

\antiphona{}{temporalia/ant-visitanos.gtex}

\vfill
\pagebreak

\pars{Psalmus 2.} \scriptura{Ps. 117, 6; \textbf{H156}}

\vspace{-8mm}

\antiphona{VIII G}{temporalia/ant-dominusmihi.gtex}

\vspace{-3mm}

\scriptura{Ps. 105, 16-31}

\vspace{-2.5mm}

\initiumpsalmi{temporalia/ps105ii-initium-viii-G-auto.gtex}

\vspace{-1.5mm}

\input{temporalia/ps105ii-viii-G.tex} \Abardot{}

\vspace{-5mm}

\vfill
\pagebreak

\pars{Psalmus 3.} \scriptura{Ps. 105, 44}

\vspace{-4mm}

\antiphona{VII a}{temporalia/ant-cumtribularentur.gtex}

%\vspace{-2mm}

\scriptura{Ps. 105, 32-48}

%\vspace{-2mm}

\initiumpsalmi{temporalia/ps105iii-initium-vii-a-auto.gtex}

\input{temporalia/ps105iii-vii-a.tex}

\vfill

\antiphona{}{temporalia/ant-cumtribularentur.gtex}

\vfill
\pagebreak
\fi
\ifx\matuc\undefined
\else
% MAT C
\pars{Psalmus 1.} \scriptura{Ps. 106, 8}

\vspace{-4mm}

\antiphona{IV* e}{temporalia/ant-confiteanturdomino.gtex}

%\vspace{-2mm}

\scriptura{Ps. 106, 1-14}

%\vspace{-2mm}

\initiumpsalmi{temporalia/ps106i-initium-iv_-e-auto.gtex}

\input{temporalia/ps106i-iv_-e.tex} \Abardot{}

\vfill
\pagebreak

\pars{Psalmus 2.} \scriptura{Ps. 24, 17; \textbf{H100}}

\vspace{-4mm}

\antiphona{C}{temporalia/ant-denecessitatibus.gtex}

%\vspace{-2mm}

\scriptura{Ps. 106, 15-30}

%\vspace{-2mm}

\initiumpsalmi{temporalia/ps106ii-initium-c-c2-auto.gtex}

\input{temporalia/ps106ii-c-c2.tex}

\vfill

\antiphona{}{temporalia/ant-denecessitatibus.gtex}

\vfill
\pagebreak

\pars{Psalmus 3.} \scriptura{Ps. 106, 24}

\vspace{-4mm}

\antiphona{III a\textsuperscript{2}}{temporalia/ant-ipsividerunt.gtex}

%\vspace{-2mm}

\scriptura{Ps. 106, 31-43}

%\vspace{-2mm}

\initiumpsalmi{temporalia/ps106iii-initium-iii-a2-auto.gtex}

\input{temporalia/ps106iii-iii-a2.tex} \Abardot{}

\vfill
\pagebreak
\fi
\ifx\matud\undefined
\else
% MAT D
\pars{Psalmus 1.} \scriptura{1 Sam. 2, 10; \textbf{H96}}

\vspace{-4mm}

\antiphona{I g\textsuperscript{2}}{temporalia/ant-dominusjudicabit.gtex}

%\vspace{-2mm}

\scriptura{Ps. 49, 1-6}

%\vspace{-2mm}

\initiumpsalmi{temporalia/ps49i_vi-initium-i-g2-auto.gtex}

\input{temporalia/ps49i_vi-i-g2.tex} \Abardot{}

\vfill
\pagebreak

\pars{Psalmus 2.}

\vspace{-4mm}

\antiphona{VIII G}{temporalia/ant-attenditepopulemeus.gtex}

%\vspace{-2mm}

\scriptura{Ps. 49, 7-15}

%\vspace{-2mm}

\initiumpsalmi{temporalia/ps49vii_xv-initium-viii-G-auto.gtex}

\input{temporalia/ps49vii_xv-viii-G.tex} \Abardot{}

\vfill
\pagebreak

\pars{Psalmus 3.} \scriptura{Ps. 49, 14; \textbf{H94}}

\vspace{-4mm}

\antiphona{E}{temporalia/ant-immoladeo.gtex}

%\vspace{-2mm}

\scriptura{Ps. 49, 16-23}

%\vspace{-2mm}

\initiumpsalmi{temporalia/ps49xvi_xxiii-initium-e-auto.gtex}

\input{temporalia/ps49xvi_xxiii-e.tex} \Abardot{}

\vfill
\pagebreak
\fi
\else
\matutinum
\fi

\ifx\matversus\undefined
\pars{Versus} \scriptura{Mc. 1, 3; Is. 40, 3}

% Versus. %%%
\sineinitiali{temporalia/versus-voxclamantis-simplex.gtex}
\else
\matversus
\fi

\vspace{5mm}

\sineinitiali{temporalia/oratiodominica-mat.gtex}

\vspace{5mm}

\pars{Absolutio.}

\cuminitiali{}{temporalia/absolutio-avinculis.gtex}

\vfill
\pagebreak

\cuminitiali{}{temporalia/benedictio-solemn-ille.gtex}

\vspace{7mm}

\lectioi

\noindent \Vbardot{} Tu autem, Dómine, miserére nobis.
\noindent \Rbardot{} Deo grátias.

\vfill
\pagebreak

\responsoriumi

\vfill
\pagebreak

\cuminitiali{}{temporalia/benedictio-solemn-divinum.gtex}

\vspace{7mm}

\lectioii

\noindent \Vbardot{} Tu autem, Dómine, miserére nobis.
\noindent \Rbardot{} Deo grátias.

\vfill
\pagebreak

\responsoriumii

\vfill
\pagebreak

\cuminitiali{}{temporalia/benedictio-solemn-adsocietatem.gtex}

\vspace{7mm}

\lectioiii

\noindent \Vbardot{} Tu autem, Dómine, miserére nobis.
\noindent \Rbardot{} Deo grátias.

\vfill
\pagebreak

\responsoriumiii

\vfill
\pagebreak

\rubrica{Reliqua omittuntur, nisi Laudes separandæ sint.}

\sineinitiali{temporalia/domineexaudi.gtex}

\vfill

\oratio

\vfill

\noindent \Vbardot{} Dómine, exáudi oratiónem meam.

\noindent \Rbardot{} Et clamor meus ad te véniat.

\noindent \Vbardot{} Benedicámus Dómino, allelúia, allelúia.

\noindent \Rbardot{} Deo grátias, allelúia, allelúia.

\noindent \Vbardot{} Fidélium ánimæ per misericórdiam Dei requiéscant in pace.

\noindent \Rbardot{} Amen.

\vfill
\pagebreak

\hora{Ad Laudes.} %%%%%%%%%%%%%%%%%%%%%%%%%%%%%%%%%%%%%%%%%%%%%%%%%%%%%

\cantusSineNeumas

\vspace{0.5cm}
\ifx\deusinadiutorium\undefined
\grechangedim{interwordspacetext}{0.18 cm plus 0.15 cm minus 0.05 cm}{scalable}%
\cuminitiali{}{temporalia/deusinadiutorium-communis.gtex}
\grechangedim{interwordspacetext}{0.22 cm plus 0.15 cm minus 0.05 cm}{scalable}%
\else
\deusinadiutorium
\fi

\vfill
\pagebreak

\ifx\hymnuslaudes\undefined
\pars{Hymnus} \scriptura{Ambrosius (\olddag{} 397)}

\cuminitiali{I}{temporalia/hym-VoxClara-aromi.gtex}
\vspace{-3mm}
\else
\hymnuslaudes
\fi

\vfill
\pagebreak

\ifx\laudes\undefined
\ifx\lauda\undefined
\else
\pars{Psalmus 1.} \scriptura{Ps. 62, 2.3; \textbf{H142}}

\vspace{-4mm}

\antiphona{VII a}{temporalia/ant-adtedeluce.gtex}

\scriptura{Psalmus 118, 145-152; \hspace{5mm} \hebinitial{ק}}

\initiumpsalmi{temporalia/ps118xix-initium-vii-a-auto.gtex}

\input{temporalia/ps118xix-vii-a.tex} \Abardot{}

\vfill
\pagebreak

\pars{Psalmus 2.} \scriptura{Ex. 15, 1; \textbf{H98}}

\vspace{-4mm}

\antiphona{E}{temporalia/ant-cantemusdomino.gtex}

\scriptura{Canticum Moysis, Ex. 15, 1-19}

\initiumpsalmi{temporalia/moysis-initium-e-auto.gtex}

\input{temporalia/moysis-e.tex}

\antiphona{}{temporalia/ant-cantemusdomino.gtex}

\vfill
\pagebreak

\pars{Psalmus 3.} \scriptura{Ps. 116, 1; \textbf{H94}}

\vspace{-4mm}

\antiphona{E}{temporalia/ant-laudatedominumomnes.gtex}

\scriptura{Psalmus 116.}

\initiumpsalmi{temporalia/ps116-initium-e.gtex}

\input{temporalia/ps116-e.tex} \Abardot{}

\vfill
\pagebreak
\fi
\ifx\laudb\undefined
\else
\pars{Psalmus 1.} \scriptura{Ps. 91, 6}

\vspace{-4.5mm}

\antiphona{E}{temporalia/ant-quammagnificatasunt.gtex}

\vspace{-3mm}

\scriptura{Psalmus 91.}

\vspace{-2mm}

\initiumpsalmi{temporalia/ps91-initium-e.gtex}

\vspace{-1.5mm}

\input{temporalia/ps91-e.tex} \Abardot{}

\vfill
\pagebreak

\pars{Psalmus 2.} \scriptura{Dt. 32, 3}

%\vspace{-4mm}

\antiphona{VI F}{temporalia/ant-datemagnitudinem.gtex}

\vspace{-4mm}

\scriptura{Canticum Moysi, Dt. 32, 1-32}

\initiumpsalmi{temporalia/moysis2i_xii-initium-vi-F-auto.gtex}

\input{temporalia/moysis2i_xii-vi-F.tex}

\vfill

\antiphona{}{temporalia/ant-datemagnitudinem.gtex}

\vfill
\pagebreak

\pars{Psalmus 3.} \scriptura{Ps. 8, 2}

\vspace{-4mm}

\antiphona{I g}{temporalia/ant-quamadmirabileest.gtex}

%\vspace{-2mm}

\scriptura{Ps. 8}

%\vspace{-2mm}

\initiumpsalmi{temporalia/ps8-initium-i-g-auto.gtex}

\input{temporalia/ps8-i-g.tex} \Abardot{}

\vfill
\pagebreak
\fi
\ifx\laudc\undefined
\else
\pars{Psalmus 1.} \scriptura{Ps. 62, 7}

\vspace{-4mm}

\antiphona{E}{temporalia/ant-inmatutinis.gtex}

%\vspace{-2mm}

\scriptura{Psalmus 118, 145-152.}

%\vspace{-2mm}

\initiumpsalmi{temporalia/ps118xix-initium-e-auto.gtex}

%\vspace{-1.5mm}

\input{temporalia/ps118xix-e.tex} \Abardot{}

\vfill
\pagebreak

\pars{Psalmus 2.}

\vspace{-4mm}

\antiphona{V a}{temporalia/ant-mecumsitdomine.gtex}

%\vspace{-2mm}

\scriptura{Canticum Sapientiæ, Sap. 9, 1-6.9-11}

\initiumpsalmi{temporalia/sapientia-initium-v-a-auto.gtex}

\input{temporalia/sapientia-v-a.tex} \Abardot{}

\vfill
\pagebreak

\pars{Psalmus 3.}

\vspace{-4mm}

\antiphona{II* b}{temporalia/ant-veritasdomini.gtex}

%\vspace{-2mm}

\scriptura{Ps. 116}

%\vspace{-2mm}

\initiumpsalmi{temporalia/ps116-initium-ii_-B-auto.gtex}

\input{temporalia/ps116-ii_-B.tex} \Abardot{}

\vfill
\pagebreak
\fi
\ifx\laudd\undefined
\else
\pars{Psalmus 1.} \scriptura{Ps. 91, 2; \textbf{H99}}

\vspace{-4mm}

\antiphona{VIII G}{temporalia/ant-bonumestconfiteri.gtex}

%\vspace{-2mm}

\scriptura{Psalmus 91.}

%\vspace{-2mm}

\initiumpsalmi{temporalia/ps91-initium-viii-g-auto.gtex}

%\vspace{-1.5mm}

\input{temporalia/ps91-viii-g.tex}

\vfill

\antiphona{}{temporalia/ant-bonumestconfiteri.gtex}

\vfill
\pagebreak

\pars{Psalmus 2.}

\vspace{-4mm}

\antiphona{IV* e}{temporalia/ant-dabovobiscor.gtex}

%\vspace{-2mm}

\scriptura{Canticum Habacuc, Hab. 3, 2-19}

\initiumpsalmi{temporalia/habacuc-initium-iv_-e.gtex}

\input{temporalia/habacuc-iv_-e.tex}

\vfill

\antiphona{}{temporalia/ant-dabovobiscor.gtex}

\vfill
\pagebreak

\pars{Psalmus 3.}

\vspace{-4mm}

\antiphona{I f}{temporalia/ant-exoreinfantium.gtex}

%\vspace{-2mm}

\scriptura{Ps. 8}

%\vspace{-2mm}

\initiumpsalmi{temporalia/ps8-initium-i-f-auto.gtex}

\input{temporalia/ps8-i-f.tex} \Abardot{}

\vfill
\pagebreak
\fi
\else
\laudes
\fi

\ifx\lectiobrevis\undefined
\pars{Lectio Brevis.} \scriptura{Is. 11, 1-3}

\noindent Egrediétur virga de stirpe Iesse, et flos de radíce eius ascéndet; et requiéscet super eum spíritus Dómini: spíritus sapiéntiæ et intelléctus, spíritus consílii et fortitúdinis, spíritus sciéntiæ et timóris Dómini; et delíciæ eius in timóre Dómini.
\else
\lectiobrevis
\fi

\vfill

\ifx\responsoriumbreve\undefined
\pars{Responsorium breve.} \scriptura{Is. 60, 2; \textbf{H20}}

\cuminitiali{IV}{temporalia/resp-superte.gtex}
\else
\responsoriumbreve
\fi

\vfill
\pagebreak

\benedictus

\vfill
\pagebreak

\pars{Preces.}

\sineinitiali{}{temporalia/tonusprecum.gtex}

\ifx\preces\undefined
\noindent Deum Patrem, qui antíqua dispositióne pópulum suum salváre státuit, \gredagger{} orémus dicéntes:

\Rbardot{} Custódi plebem tuam, Dómine.

\noindent Deus, qui pópulo tuo germen iustítiæ promisísti, \gredagger{} custódi sanctitátem Ecclésiæ tuæ.

\Rbardot{} Custódi plebem tuam, Dómine.

\noindent Inclína cor hóminum, Deus, in verbum tuum \gredagger{} et confírma fidéles tuos sine queréla in sanctitáte.

\Rbardot{} Custódi plebem tuam, Dómine.

\noindent Consérva nos in dilectióne Spíritus tui, \gredagger{} ut Fílii tui, qui ventúrus est, misericórdiam suscipiámus.

\Rbardot{} Custódi plebem tuam, Dómine.

\noindent Confírma nos, Deus clementíssime, usque in finem, \gredagger{} in diem advéntus Dómini Iesu Christi.

\Rbardot{} Custódi plebem tuam, Dómine.
\else
\preces
\fi

\vfill

\pars{Oratio Dominica.}

\cuminitiali{}{temporalia/oratiodominicaalt.gtex}

\vfill
\pagebreak

\rubrica{vel:}

\pars{Supplicatio Litaniæ.}

\cuminitiali{}{temporalia/supplicatiolitaniae.gtex}

\vfill

\pars{Oratio Dominica.}

\cuminitiali{}{temporalia/oratiodominica.gtex}

\vfill
\pagebreak

% Oratio. %%%
\oratio

\vspace{-1mm}

\vfill

\rubrica{Hebdomadarius dicit Dominus vobiscum, vel, absente sacerdote vel diacono, sic concluditur:}

\vspace{2mm}

\antiphona{C}{temporalia/dominusnosbenedicat.gtex}

\rubrica{Postea cantatur a cantore:}

\vspace{2mm}

\ifx\benedicamuslaudes\undefined
\cuminitiali{IV}{temporalia/benedicamus-feria-advequad.gtex}
\else
\benedicamuslaudes
\fi

\vfill

\vspace{1mm}

\end{document}

