\newcommand{\titulus}{\dies{11. Februarii.}
\nomenFesti{B. Mariæ Virginis de Lourdes.}}
\newcommand{\oratio}{\pars{Oratio.}

\noindent Concéde, miséricors Deus, fragilitáti nostræ præsídium, ut, qui Immaculátæ Dei Genetrícis memóriam ágimus, intercessiónis eius auxílio, a nostris iniquitátibus resurgámus.

\pars{Pro pace in universo mundo.} \scriptura{Sir. 50, 25; 2 Esdr. 4, 20; \textbf{H416}}

\vspace{-4mm}

\antiphona{II D}{temporalia/ant-dapacemdomine.gtex}

\vfill

\noindent Deus, a quo sancta desidéria, recta consília et iusta sunt ópera: da servis tuis illam, quam mundus dare non potest, pacem; ut et corda nostra mandátis tuis dédita, et hóstium subláta formídine, témpora sint tua protectióne tranquílla.

\noindent Per Dóminum nostrum Iesum Christum, Fílium tuum, qui tecum vivit et regnat in unitáte Spíritus Sancti, Deus, per ómnia sǽcula sæculórum.

\noindent \Rbardot{} Amen.}
\newcommand{\invitatorium}{\pars{Invitatorium.}

\vspace{-4mm}

\antiphona{V}{temporalia/inv-christummariaefilium.gtex}}
\newcommand{\hymnusmatutinum}{\pars{Hymnus.}

\vspace{-5mm}

{
\grechangedim{interwordspacetext}{0.30 cm plus 0.15 cm minus 0.05 cm}{scalable}%
\antiphona{VIII}{temporalia/hym-OVirgoMater.gtex}
\grechangedim{interwordspacetext}{0.22 cm plus 0.15 cm minus 0.05 cm}{scalable}%
}}
\newcommand{\lectioi}{\pars{Lectio I.} \scriptura{Gn. 12, 1-9}

\noindent De libro Génesis.

\noindent Dixit autem Dóminus ad Abram:

\noindent “Egrédere de terra tua et de cognatióne tua et de domo patris tui in terram, quam monstrábo tibi.

\noindent Faciámque te in gentem magnam et benedícam tibi et magnificábo nomen tuum, erísque in benedictiónem.

\noindent Benedícam benedicéntibus tibi et maledicéntibus tibi maledícam, atque in te benedicéntur univérsæ cognatiónes terræ!”.

\noindent Egréssus est ítaque Abram, sicut præcéperat ei Dóminus, et ivit cum eo Lot. Septuagínta quinque annórum erat Abram, cum egrederétur de Charran.

\noindent Tulítque Sárai uxórem suam et Lot fílium fratris sui universámque substántiam, quam acquisíverant, et ánimas, quas fécerant in Charran, et egréssi sunt, ut irent in terram Chánaan; et venérunt in terram Chánaan.

\noindent Pertransívit Abram terram usque ad locum Sichem, usque ad Quercum Moreh. Chananǽus autem tunc erat in terra.

\noindent Appáruit autem Dóminus Abram et dixit ei: “Sémini tuo dabo terram hanc”. Qui ædificávit ibi altáre Dómino, qui apparúerat ei.

\noindent Et inde transgrédiens ad montem, qui erat contra oriéntem Bethel, teténdit ibi tabernáculum suum ab occidénte habens Bethel et ab oriénte Hai; ædificávit quoque ibi altáre Dómino et invocávit nomen Dómini.

\noindent Perrexítque Abram de mansióne in mansiónem usque ad Nageb.}
\newcommand{\responsoriumi}{\pars{Responsorium 1.} \scriptura{\Rbar{} Gn. 12, 1-2; \textbf{H140}}

\vspace{-5mm}

\responsorium{II}{temporalia/resp-locutusestdominusadabraham-CROCHU.gtex}{}}
\newcommand{\lectioii}{\pars{Lectio II.} \scriptura{Ep. ad P. Gondrand, a. 1861: cf. A. Ravier, Les écrits de sainte Bernadette, Paris 1961, pp. 53-59}

\noindent Ex Epístola sanctæ Maríæ Bernárdæ Soubirous vírginis.

\noindent Quadam die, cum me contulíssem ad ripam flúminis Gavi ut ligna collígerem cum duábus puéllis, rumórem quendam audívi. Me verti ad pratum, sed árbores vidi mínime agitári. Unde caput levávi et antrum aspéxi. Dóminam autem vidi véstibus albis indútam: cándido enim hábitu erat amícta zonáque cærúlea cincta, et gilvam super utróque pede rosam habébat, quæ eiúsdem colóris erat ac coróna eius rosárii. Quæ cum vidi, óculos perfrícui, putans me falli; manus autem in vestis sinu insérui, ubi meam invéni corónam rosárii. Vólui étiam frontem cruce signáre, sed manum illuc attóllere non válui, quæ décidit. Cum vero Dómina illa signum fecísset crucis, ego quoque, treménte licet manu, conáta sum, et tandem pótui. Simul rosárium recitáre cœpi, ipsa quoque Dómina corónæ rosárii sui volvénte gránula nec tamen lábia movénte. Cum rosário finem dedi, vísio statim evánuit. Quæsívi ígitur a duábus puéllis num quidquam conspexíssent: quod illæ negárunt; quin étiam interrogavérunt quid habérem sibi revelándum. Quas certióres feci vidísse me Dóminam albis vestiméntis indútam, nescíre autem quæ esset; sed ut hoc tacérent admónui. Hortátæ sunt me dein et illæ, ne illuc redírem; quod ego recusávi. Revérsa sum ígitur die domínico, cum intérius me ciéri sentírem.}
\newcommand{\responsoriumii}{\pars{Responsorium 2.} \scriptura{\textbf{H306}}

\vspace{-5mm}

\responsorium{I}{temporalia/resp-conceptiogloriosae-CROCHU.gtex}{}}
\newcommand{\lectioiii}{\pars{Lectio III.}

\noindent Dómina illa nónnisi tértium mihi locúta est, atque rogávit num ire ad se per dies quíndecim vellem. Quod me velle respóndi. Adiécit autem illa debére a me presbýteros admonéri ut sacéllum ibídem ædificándum curárent; deínde iussit ut e fonte bíberem. Cum nullum conspícerem fontem, ibam ad flúvium Gavum; at ipsa significávit non de illo se loqui, et dígito fontem monstrávit. Cumque ad hunc adiíssem, non invéni nisi parum lutuléntæ aquæ. Admóta manu, nihil cápere pótui; unde scálpere cœpi, ac tandem paulum aquæ hauríre valens, ter proiéci, quarta autem vice bíbere pótui. Vísio dein dilápsa est et ego recéssi. Per dies vero quíndecim illuc rédii, atque Dómina síngulis diébus, præter quandam fériam secúndam et fériam sextam, mihi appáruit, idéntidem mandans debére me presbýteros monére de sacéllo ibídem erigéndo, et fontem ad me lavándam pétere, et pro peccatórum conversióne deprecári. Plúries quidem eam interrogávi quæ esset, at illa léniter arridébat; demum suspénsa tenens bráchia oculósque in cælum élevans, dixit mihi se esse Immaculátam Conceptiónem. Intra quíndecim dies illos tria quoque mihi secréta patefécit, quæ omníno ne cuíquam pánderem interdíxit; quod fidéliter hucúsque servávi.}
\newcommand{\responsoriumiii}{\pars{Responsorium 3.} \scriptura{\Rbardot{} Cantor \Vbardot{} Lc. 1, 42; \textbf{H307}}

\vspace{-5mm}

\responsorium{I}{temporalia/resp-conceptiotua-CROCHU-cumdox.gtex}{}}
\newcommand{\matversus}{\noindent \Vbardot{} María conservábat ómnia verba hæc.

\noindent \Rbardot{} Cónferens in corde suo.}
\newcommand{\hymnuslaudes}{\pars{Hymnus}

\cuminitiali{I}{temporalia/hym-AveMarisStella.gtex}}
\newcommand{\lectiobrevis}{\pars{Lectio Brevis.} \scriptura{Ap. 12, 1}

\noindent Signum magnum appáruit in cælo: múlier amícta sole, et luna sub pédibus eius, et super caput eius coróna stellárum duódecim.}
\newcommand{\responsoriumbreve}{\pars{Responsorium breve.} \scriptura{Lc. 1, 28}

\cuminitiali{VI}{temporalia/resp-avemaria-alt.gtex}}
\newcommand{\benedictus}{\pars{Canticum Zachariæ.} \scriptura{Cf. Mal. 4, 2; Lc. 1, 78}

\vspace{-4mm}

{
\grechangedim{interwordspacetext}{0.18 cm plus 0.15 cm minus 0.05 cm}{scalable}%
\antiphona{VIII G}{temporalia/ant-praeclarasalutisaurora.gtex}
\grechangedim{interwordspacetext}{0.22 cm plus 0.15 cm minus 0.05 cm}{scalable}%
}

%\vspace{-3mm}

\scriptura{Lc. 1, 68-79}

%\vspace{-2mm}

\cantusSineNeumas
\initiumpsalmi{temporalia/benedictus-initium-viii-G-auto.gtex}

%\vspace{-1.5mm}

\input{temporalia/benedictus-viii-G.tex} \Abardot{}}
\newcommand{\preces}{\noindent Salvatórem nostrum celebrántes,~\gredagger{} qui ex María Vírgine nasci dignátus est,~\grestar{} exorémus dicéntes:

\Rbardot{} Intercédat pro nobis mater tua, Dómine.

\noindent O sol iustítiæ,~\gredagger{} quem Immaculáta Virgo ut lucens auróra præcéssit,~\grestar{} tríbue ut in lúmine visitatiónis tuæ semper ambulémus.

\Rbardot{} Intercédat pro nobis mater tua, Dómine.

\noindent Verbum ætérnum,~\gredagger{} quod Maríam habitatiónis tuæ arcam incorruptíbilem elegísti,~\grestar{} líbera nos a corruptióne peccáti.

\Rbardot{} Intercédat pro nobis mater tua, Dómine.

\noindent Salvátor noster, qui iuxta crucem matrem tuam habuísti,~\gredagger{} præsta ut, ipsa intercedénte,~\grestar{} communicántes tuis passiónibus gaudeámus.

\Rbardot{} Intercédat pro nobis mater tua, Dómine.

\noindent Benigníssime Iesu,~\gredagger{} qui pendens in cruce, Maríam Ioánni matrem dedísti,~\grestar{} da nobis ita vívere ut eius fílii agnoscámur.

\Rbardot{} Intercédat pro nobis mater tua, Dómine.}
\newcommand{\benedicamuslaudes}{\cuminitiali{I}{temporalia/benedicamus-festis-bmv.gtex}}
\include{hebdomadav}
\include{feriaiv}
