\documentclass[options]{article}
\usepackage[T1]{fontenc}
\begin{document}
Ex Sermónibus sancti Martíni Legionénsis presbýteri 
	\begin{flushright}
	(Sermo 2 in Advéntu : PL 208,46-47)
	\end{flushright}
Isaías prophéta congregári omnes gentes prænúntiat, ut fídei disciplínam percípiant, ita: \emph{Et erit in novíssimis diébus præparátus mons domus Dómini in vértice móntium.} Hi sunt novíssimi dies in quibus Salvatóris resplénduit fides; præparátus autem mons super vérticem móntium Christus est; quia ipse caput apostolórum et prophetárum est. Domus Dómini vero Christi est Ecclésia super eúmdem stabilíta, ad quam géntium congregátur divérsitas in fide et concórdia. \emph{Lex autem de Sion éxiit, et verbum Dómini de Ierúsalem,} sive ut veníret in géntibus, relíctis ob incredulitátem Jud\'{æ}is, sive quia in eádem plebe pósitus, Iesus dixit discípulis suis: \emph{Ite, docéte omnes gentes, baptizántes eos in nómine Patris et Fílii et Spíritus Sancti.} Item, sic dicit Zacharías prophéta de congregatárum géntium Ecclésia: \emph{Lauda et lætáre, fília Sion; quia ecce véniam et habitábo in médio tui, dicit Dóminus.}\\
\\
Resp 2 Bethleem civitas VEL, ad libitum, Veni Domine\\
\\
Dei Fílius advéntus sui diem appropinquáre videns, et sanctæ Ecclésiæ afflictiónem subleváre volens, ad consolatiónem illíus, per Zacharíam prophétam lóquitur, dicens: \emph{Lauda et lætáre fília Sion,} ac si díceret: «Quia te obligátam conspício vínculis peccatórum, ego ut te rédimam, de sinu Patris vénio in hunc mundum, cum homínibus conversári non refúgiam; ut de tantis malis te erípiam, et ad societátem supernórum cívium perdúcam. Lauda ergo medúllis cordis, lauda voce iubilatiónis, dum in médio tui Deum habitáre cognóscis.»\\
\\
Quis est iste Dóminus a Deo exercítuum missus, nisi Dóminus noster Iesus Christus, quem Deus Pater misit ad liberándum humánum genus de duríssimo infelicitátis iugo, quo eum sibi subdíderat hostis antíquus? Iterum Isaías prophéta de magnitúdine Christi et glória lóquitur ita: \emph{Et erit exténsio alárum eius, implens latitúdinem terræ tuæ, Emmánuel.} Ac si díceret: «Usque ádeo se exténdet princípium vel principátus eius implens latitúdinem terræ tuæ, scílicet Iud\'{æ}æ, ut excédat omnes términos tuos magnitúdo illíus.» \emph{Emmánuel} ex hebr\'{æ}o in latínum signíficat \emph{nobíscum Deus} quia per Vírginem Deus natus, in carne appáruit homínibus, ut viam salútis aperíret fidélibus, qua ad cælum perveníre, et angélicis possent admiscéri c\'{œ}tibus.\\
\\
Resp 3 : Ecce Radix Iesse VEL ad libitum Docebit nos Dominus
\end{document}