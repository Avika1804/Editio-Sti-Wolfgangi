\documentclass[options]{article}
\usepackage[T1]{fontenc}
\begin{document}
	Léctio sancti Evangélii secúndum Lucam
	\begin{flushright}
	Lc 1, 26-38
	\end{flushright}
	In illo témpore:
	Missus est ángelus Gábriel a Deo in civitátem Galil\'{æ}æ, cui nomen Názareth, ad vírginem desponsátam viro, cui nomen erat Ioseph de domo David, et nomen vírginis María.
	Et ingréssus ad eam dixit: «Ave, grátia plena, Dóminus tecum». Ipsa autem turbáta est in sermóne eius et cogitábat qualis esset ista salutátio.
	Et ait ángelus ei: «Ne tímeas, María; invenísti enim grátiam apud Deum. Et ecce concípies in útero et páries fílium et vocábis nomen eius Iesum. Hic erit magnus et Fílius Altíssimi vocábitur, et dabit illi Dóminus Deus sedem David patris eius, et regnábit super domum Iacob in ætérnum, et regni eius non erit finis».
	Dixit autem María ad ángelum: «Quómodo fiet istud, quóniam virum non cognósco?».
	Et respóndens ángelus dixit ei: «Spíritus Sanctus supervéniet in te, et virtus Altíssimi obumbrábit tibi: ideóque et quod nascétur sanctum, vocábitur Fílius Dei. Et ecce Elísabeth cognáta tua et ipsa concépit fílium in senécta sua, et hic mensis est sextus illi, quæ vocátur stérilis, quia non erit impossíbile apud Deum omne verbum».
	Dixit autem María: «Ecce ancílla Dómini; fiat mihi secúndum verbum tuum». Et discéssit ab illa ángelus.\\
	\\
	Ex Sermónibus sancti Ambrósii Autpérti abbátis
	\begin{flushright}
			(Sermo 194,1-2 : PL 39, 2104-2105)
	\end{flushright}
	Adest nobis, dilectíssimi, optátus dies beátæ ac venerábilis semper Vírginis Maríæ, ídeo cum summa exsultatióne gáudeat terra nostra, tantæ Vírginis illustráta die sollémni. Hæc est enim flos campi, de qua ortum est pretiósum lílium convállium, per cuius partum mutátur natúra, protoplastorúmque delétur et culpa. Præcísum est in ea illud Evæ infelicitátis elógium quo dícitur : \emph{In tristítia páries fílios:} quia ista in lætítia Dóminum parturívit. Eva enim luxit, ista exsultávit; Eva lácrimas, María gáudium in ventre portávit: quia illa peccatórem, ista édidit innocéntem.\\
	\\
	Virgo quippe génuit, quia virgo concépit; invioláta péperit, quia in concéptu libído non fuit. Utrobíque miráculum, et sine corruptióne grávida, et in partu virgo puérpera. Descéndit ángelus de cælo missus a Patre Deo in nostræ redemptiónis exórdium, ad beátam salutándam Maríam: \emph{Ave,} inquit ángelus ad eam, \emph{grátia plena, Dóminus tecum.} Impléta est ergo María grátia, et Eva vacuáta est a culpa. Maledíctio Evæ in benedictiónem mutátur Maríæ: \emph{Ave, grátia plena, Dóminus tecum.} Tecum Dóminus in corde, tecum in ventre, tecum in útero, tecum in auxílio.\\
	\\
	Gratuláre, beáta Virgo: Christus rex e cælo suo venit in úterum tuum, ex sinu Patris in úterum dignátur descéndere matris. \emph{Benedícta,} inquit, \emph{tu in muliéribus,} quæ vitam et viris et muliéribus peperísti. Mater géneris nostri pœnam íntulit mundo; Génetrix Dómini nostri salútem áttulit mundo. Auctrix peccáti Eva; auctrix mériti María; Eva occidéndo óbfuit; María vivificándo prófuit. Illa percússit; ista sanávit. Pro inobœdiéntia enim obœdiéntia commutátur, fides pro perfídia compensátur.\\
	\\
	Læta ígitur María gestat infántem, exsúltans amplexátur fílium, portat a quo portabátur. Plaudat nunc órganis María, et inter velóces artículos týmpana puérperæ concrepent. Cóncinant lætántes chori, et alternántibus módulis dulcísona cármina misceántur. Audíte ígitur quemádmodum tympanístria nostra cantáverit: ait enim : \emph{Magníficat ánima mea Dóminum.} Causa ígitur tantæ invalescéntis lætítiæ erat miráculum novum. Novus Maríæ partus partum Evæ evícit et Evæ planctum Maríæ cantus exclúsit.	
\end{document}