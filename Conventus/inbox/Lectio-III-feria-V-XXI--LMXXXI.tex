\documentclass[options]{article}
\begin{document}
	Ex Tractátu sancti Cypriáni epíscopi et mártyris De domínica oratióne \begin{flushright}
		(Nn. 29-30 : CCL 3A, 108-109)
	\end{flushright}
Non verbis tantum sed et factis Dóminus oráre nos dócuit ipse orans frequénter et déprecans et quid nos fácere oportéret exémpli sui contestatióne demónstrans, sicut scriptum est: 
\emph{Ipse autem fuit secédens in solitúdines et adórans,}
 et íterum: 
 \emph{ Exívit in montem oráre et fuit pernóctans in oratióne Dei.}
Quodsi ille orábat qui sine peccáto erat, quanto magis peccatóres opórtet oráre, et si ille per totam noctem iúgiter vígilans contínuis précibus orábat, quanto nos magis in frequentánda oratióne debémus nocte vigiláre.


Orábat autem Dóminus et rogábat non pro se — quid enim pro se ínnocens precarétur ? — sed pro delíctis nostris, sicut ipse declárat cum dicit ad Petrum: 
\emph{Ecce sátanas postulávit ut vos vexáret quómodo tríticum. Ego autem rogávi pro te, ne defíciat fides tua.}
 Et póstmodum pro ómnibus Patrem deprecátur dicens: 
 \emph{Non pro his autem rogo solis, sed et pro illis qui creditúri sunt per verbum ipsórum in me, ut omnes unum sint, sicut et tu, Pater, in me et ego in te, ut et ipsi in nobis sint.}

Magna Dómini propter salútem nostram benígnitas páriter et píetas, ut non conténtus quod nos sánguine suo redímeret adhuc pro nobis ámplius et rogáret. Rogántis autem desidérium vidéte quod fúerit, ut quómodo unum sunt Pater et Fílius, sic et nos in ipsa unitáte maneámus: ut hinc quoque possit intéllegi quantum delínquat qui unitátem scindit et pacem, cum pro hoc et rogáverit Dóminus volens scílicet plebem suam vívere, cum sciret ad regnum Dei discórdiam non veníre.
	
	
\end{document}