\documentclass[options]{article}
\begin{document}
	Ex Commentáriis Eusébii Cæsariensis epíscopi in Isaíam
	\begin{flushright}
		(Cap. 40: PG 24, 3766-367)
	\end{flushright}
\emph{Vox clamántis in desérto, paráte viam Dómini, rectas fácite sémitas Dei nostri.}
Apérte declárat ea, quæ in vaticínio ferúntur, non Hierosólymæ, sed in desérto gerénda esse; nempe quod futúrum sit, ut glória Dómini appáreat, et salutáre Dei in omnis carnis notítiam véniat.\\
Et hæc quidem secúndum históriam et ad verbum, tunc impléta sunt, cum Ioánnes Baptísta salutárem Dei advéntum prædicávit in desérto Iordánis, ubi salutáre Dei visum fuit. Nam tunc Christus eiúsque glória ómnibus innótuit, cum, ipso baptizáto, apérti sunt cæli, et Spíritus Sanctus, in colúmbæ spécie descéndens, super eo insédit, patérnaque vox deláta est, Fílio testimónium reddens, \emph{Hic est Fílius meus diléctus, ipsum audíte.}\\
 Hæc quippe dicebántur, quia Deus in desértum, a s\'{æ}culo impérvium et inaccéssum, adventúrus erat. Erant porro gentes omnes Dei cognitióne vácuæ, a quarum áditu omnes iusti Dei ac prophétæ arcebántur.\\
 \\
 Resp 4 -- Civitas Ierusalem\\
 \\
 Quámobrem iubet vox illa viam paráre Dei Verbo, et ínviam asperámque complanáre, ut en advéniens Deus noster prógredi váleat. \emph{Paráte viam Dómini:} ea est evangélica prædicátio nóvaque consolátio, quæ salutáre Dei in ómnium hóminum notítiam veníre exóptat.\\
 \emph{Super montem excélsum ascénde, qui evangelízas Sion. Exálta in fortitúdine vocem tuam, qui evangelízas Ierúsalem.}
 Hæc præmissórum senténtiæ appríme convéniunt, atque opportúne evangelistárum mentiónem fáciunt, et advéntum Dei ad hómines annúntiant, postquam de voce in desérto clamánte sermo hábitus est. Etenim prophetíam de Ioánne Baptísta evangelistárum méntio congruénter sequebátur.\\
 \\
 Resp 5 -- Ecce veniet Dominus protector\\
 \\
 Quænam ígitur hæc Sion est, nisi quæ ántea Ierúsalem vocabátur? Nam et ipsa mons erat, quod declárat Scriptúra illa quæ dicit: \emph{Mons Sion hic, in quo habitásti}; et Apóstolus:\emph{Accessístis ad Sion montem.} Num forte chorus apostólicus, ex prisco pópulo ex circumcisióne deléctus, hac ratióne significátur?\\
 Hæc enim Sion et Ierúsalem est, quæ salutáre Dei accépit, quæ et ipsa monti Dei, vidélicet unigénito Verbo eius, sublímis impónitur: quam iubet, conscénso monte sublími, salutáre verbum annuntiáre. Quis autem ille est, qui evangelízat, nisi evangélicus chorus? Quid est evangelizáre? univérsis homínibus, et ante omnes, civitátibus Iuda, Christi in terram advéntum prædicáre.\\
 \\
 Resp 6 Sicut mater
\end{document}