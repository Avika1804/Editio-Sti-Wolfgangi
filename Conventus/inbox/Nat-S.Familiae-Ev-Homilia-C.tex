\documentclass[options]{article}
	\usepackage[T1]{fontenc}
\begin{document}
		Lectio sancti Evangélii secúndum Lucam
		\begin{flushright}
			(Lc 2, 41-52)
		\end{flushright}
In illo témpore:
Ibant paréntes Iesu per omnes annos in Ierúsalem in die festo Paschæ. Et cum factus esset annórum duódecim, ascendéntibus illis secúndum consuetúdinem diéi festi, consummatísque diébus, cum redírent, remánsit puer Iesus in Ierúsalem, et non cognovérunt paréntes eius. Existimántes autem illum esse in comitátu, venérunt iter diéi et requirébant eum inter cognátos et notos; et non inveniéntes regréssi sunt in Ierúsalem requiréntes eum.\\
Et factum est, post tríduum invenérunt illum in templo sedéntem in médio doctórum, audiéntem illos et interrogántem eos; stupébant autem omnes, qui eum audiébant, super prudéntia et respónsis eius.\\
Et vidéntes eum admiráti sunt, et dixit mater eius ad illum: «Fili, quid fecísti nobis sic? Ecce pater tuus et ego doléntes quærebámus te».\\
Et ait ad illos: «Quid est quod me quærebátis? Nesciebátis quia in his, quæ Patris mei sunt, opórtet me esse?».
Et ipsi non intellexérunt verbum, quod locútus est ad illos.
Et descéndit cum eis et venit Názareth et erat súbditus illis. Et mater eius conservábat ómnia verba in corde suo.
Et Iesus proficiébat sapiéntia et ætáte et grátia apud Deum et hómines.\
		\\	
Ex Homíliis Orígenis in Lucam
		\begin{flushright}
(Hom. 18,2-5; 19, 5 : SC 87, 268.276)
		\end{flushright}
In multórum comitátu Iesus meus non potest inveníri. Disce ubi eum quæréntes repérerint, ut et tu quærens cum Ioseph Mariáque repérias. Et quæréntes, inquit, \emph{invenérunt illum in templo.} Non ubicúmque in álio loco, sed in templo, neque in templo simplíciter, sed in médio doctórum, \emph{audiéntem et interrogántem eos.} Et tu ergo quære Iesum in templo Dei, quære in Ecclésia, quære eum apud magístros, qui in templo sunt et non egrediúntur ex eo; si enim ita quæsíeris, invénies eum.\\
\\	
\emph{In médio doctórum sedéntem} invéniunt eum, et non solum sedéntem, sed et \emph{sciscitántem et audiéntem eos.} Et nunc præsens est Iesus, intérrogat nos et audit loquéntes. \emph{Et mirabántur,} inquit, \emph{omnes.} Super quo \emph{mirabántur?} Non super interrogatiónibus eius, licet et ipsæ mirábiles erant, sed \emph{super responsiónibus.} Iesus interrogábat magístros et, quia intérdum respondére non póterant, ipse illis, de quibus interrogáverat, respondébat. \emph{Móyses loquebátur, Deus autem respondébat ei voce.} Respónsio illa eórum erat, super quibus ignorántem Móysen Dóminus instruébat.\\
\\
Ut ígitur et nos audiámus eum et propónat nobis quæstiónes, quas ipse dissólvat, obsecrémus illum et cum labóre nímio et dolóre quærámus, et tunc potérimus inveníre, quem qu\'{æ}rimus. Non enim frustra scriptum est: \emph{Ego et pater tuus doléntes quærebámus te.} Opórtet eum, qui quærit Iesum, non neglegénter, non dissolúte, non transitórie qu\'{æ}rere, sicut quærunt nonnúlli et ídeo inveníre non possunt. Nos autem dicámus: \emph{Doléntes qu\'{æ}rimus te.}\\
Ubi ígitur invéniunt eum? In templo; ibi enim invenítur Fílius Dei. Si quando et tu quæsíeris Fílium Dei, quære primum templum, illuc própera: ibi Christum, Sermónem atque Sapiéntiam, id est Fílium Dei, repéries.\\
\\
Resp 4 resp-beataviscera-CROCHU-cumdox.gabc\\
\\	
\end{document}