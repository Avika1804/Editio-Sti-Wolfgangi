\documentclass[options]{article}
\usepackage[T1]{fontenc}
\begin{document}
	Léctio sancti Evangélii secundum Ioánnem 
	\begin{flushright}
		 (Io 2, 1-12)
	\end{flushright}
	In illo témpore:
	Núptiæ factæ sunt in Cana Galil\'{æ}æ, et erat mater Iesu ibi; vocátus est autem et Iesus et discípuli eius ad núptias.\\
	Et deficiénte vino, dicit mater Iesu ad eum: «Vinum non habent».\\
	Et dicit ei Iesus: «Quid mihi et tibi, múlier? Nondum venit hora mea».\\
	Dicit mater eius minístris: «Quodcúmque díxerit vobis, fácite».\\
	Erant autem ibi lapídeæ hýdriæ sex pósitæ secúndum purificatiónem Iudæórum, capiéntes síngulæ metrétas binas vel ternas.\\
	Dicit eis Iesus: «Impléte hýdrias aqua». Et implevérunt eas usque ad summum. Et dicit eis: «Hauríte nunc et ferte architriclíno». Illi autem tulérunt.
	Ut autem gustávit architriclínus aquam vinum factam et non sciébat unde esset, minístri autem sciébant, qui hauríerant aquam, vocat sponsum architriclínus et dicit ei: «Omnis homo primum bonum vinum ponit et, cum inebriáti fúerint, id quod detérius est; tu servásti bonum vinum usque adhuc».\\
	Hoc fecit inítium signórum Iesus in Cana Galil\'{æ}æ et manifestávit glóriam suam, et credidérunt in eum discípuli eius.\\
	\\
	Ex Epístulis sancti Cypriáni epíscopi et mártyris
	\begin{flushright}
		(Ep. 63,12-13 : CSEL 3, 710-712)	
	\end{flushright}
	Quam pervérsum est quamque contrárium, ut cum Dóminus in núptiis de aqua vinum fécerit, nos de vino aquam faciámus, cum sacraméntum quoque rei illíus admonére et instrúere nos débeat, ut in sacrifíciis domínicis vinum pótius offerámus. Nam quia nos omnes portábat Christus qui et peccáta nostra portábat, vidémus in aqua pópulum intéllegi, in vino vero osténdi sánguinem Christi. Quando autem in cálice vino aqua miscétur, Christo pópulus adunátur et credéntium plebs ei in quem crédidit copulátur et iúngitur.\\
	
	Quæ copulátio et coniúnctio aquæ et vini sic miscétur in cálice Dómini ut commíxtio illa non possit ab ínvicem separári. Unde Ecclésiam, id est plebem in Ecclésia constitútam, fidéliter et fírmiter in eo quod crédidit perseverántem nulla res separáre póterit a Christo quo minus h\'{æ}reat semper et máneat indivídua diléctio.\\
	
	Sic autem in sanctificándo cálice Dómini offérri aqua sola non potest quómodo nec vinum solum potest. Nam si vinum tantum quis ófferat, sanguis Christi íncipit esse sine nobis. Si vero aqua sit sola, plebs íncipit esse sine Christo. Quando autem utrúmque miscétur et adunatióne confúsa sibi ínvicem copulátur, tunc sacraméntum spiritále et cæléste perfícitur. Sic vero calix Dómini non est aqua sola aut vinum solum, nisi utrúmque sibi misceátur, quómodo nec corpus Dómini potest esse farína sola aut aqua sola, nisi utrúmque adunátum fúerit et copulátum, et panis uníus compáge solidátum.\\
	
	Quo et ipso sacraménto pópulus noster osténditur adunátus, ut quemádmodum grana multa in unum collécta et commólita et commíxta panem unum fáciunt, sic in Christo, qui est panis cæléstis, unum sciámus esse corpus, cui coniúnctus sit noster númerus et adunátus.\\
	\\
	resp-quammagnamultitudo-cumdox.gabc
\end{document}