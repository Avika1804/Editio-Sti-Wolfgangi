\documentclass[options]{article}
\title{Lectio IV-VI Dominica XXIII}
\begin{document}
	\textbf{Ex Confessiónum libris sancti Augustíni epíscopi}
	
	\textbf{(Lib. 1, 1. 1 – 2. 2; 5. 5: CCL 27, 1-3)}
	
	
	\textit{Magnus es, Dómine, et laudábilis valde: magna virtus tua, et sapiéntiæ tuæ non est númerus.}
	Et laudáre te vult homo, áliqua pórtio creatúræ tuæ, et homo circúmferens mortalitátem suam, circúmferens testimónium peccáti sui et testimónium quia supérbis resístis: et tamen laudáre te vult homo, áliqua pórtio creatúræ tuæ. Tu éxcitas, ut laudáre te deléctet, quia fecísti nos ad te et inquiétum est cor nostrum, donec requiéscat in te.
	
	
	\textbf{Resp 4 Qui consolabatur me}
	
	
	\textbf{Lectio V}
	
	
 Da mihi, Dómine, scire et intellégere, utrum sit prius invocáre te an laudáre te et scire te prius sit an invocáre te. Sed quis te ínvocat, nésciens te? Aliud enim pro álio potest invocáre nésciens. An pótius invocáris, ut sciáris? 
	\textit{Quómodo autem invocábunt, in quem non credidérunt?}
	Aut
	\textit{quómodo credent sine prædicánte?}
	
	\textit{Et laudábunt Dóminum qui requírunt eum.}
	Quæréntes enim invéniunt eum et inveniéntes laudábunt eum. Quæram te, Dómine, ínvocans te et ínvocem te credens in te: prædicátus enim es nobis. Invocat te, Dómine, fides mea, quam dedísti mihi, quam inspirásti mihi per humanitátem Fílii tui, per ministérium prædicatóris tui.
	
	Et quómodo invocábo Deum meum, Deum et Dóminum meum quóniam útique in me ipsum eum vocábo, cum invocábo eum? Et quis locus est in me, quo véniat in me Deus meus? Quo Deus véniat in me, Deus qui fecit cælum et terram? Itane, Dómine Deus meus, est quidquam in me quod cápiat te? An vero cælum et terra, quæ fecísti et in quibus me fecísti, cápiunt te? An quia sine te non esset quidquid est, fit, ut quidquid est cápiat te?
		
	\textbf{Responsorium V    Nocte os meum}
	
	\textbf{Lectio VI}
	
 Quóniam ítaque et ego sum, quid peto, ut vénias in me, qui non essem, nisi esses in me? Non enim ego iam in ínferis, et tamen étiam ibi es. Nam  
	\textit{etsi descéndero in inférnum, ades.}
	Non ergo essem, Deus meus, non omníno essem, nisi esses in me. An pótius non essem, nisi essem in te, 
	\textit{ex quo ómnia, per quem ómnia, in quo ómnia?}
	Etiam sic, Dómine, étiam sic. Quo te ínvoco, cum in te sim? aut unde vénias in me? Quo enim recédam extra cælum et terram, ut inde in me véniat Deus meus, qui dixit: 
	\textit{Cælum et terram ego ímpleo?}
	
	Quis mihi dabit acquiéscere in te? Quis dabit mihi ut vénias in cor meum, et inébries illud, ut oblivíscar mala mea et unum bonum meum ampléctar, te? Quid mihi es? Miserére, ut loquar. Quid tibi sum ipse, ut amári te iúbeas a me et, nisi fáciam irascáris mihi et minéris ingéntes misérias? Párvane ipsa est, si non amem te?
	
	  Ei mihi! Dic mihi per miseratiónes tuas, Dómine Deus meus, quid sis mihi.
	  \textit{Dic ánimæ meæ: Salus tua ego sum.} 
	  Sic dic, ut áudiam. Ecce aures cordis mei ante te, Dómine; áperi eas, et 
	  \textit{dic ánimæ meæ: Salus tua ego sum. } 
	  Curram post vocem hanc et apprehéndam te. Noli abscóndere a me fáciem tuam: móriar, ne móriar, ut eam vídeam.
	
	\textbf{Responsorium 6 Scio Domine quia morti}
	
\end{document}
