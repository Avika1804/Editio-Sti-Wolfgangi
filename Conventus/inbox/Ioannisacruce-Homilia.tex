\documentclass[options]{article}
\begin{document}
	Ex Cántico spiritáli sancti Ioánnis a Cruce presbýteri 
	\begin{flushright}
		(Red. B, str. 36-37, Edit. E. Pacho, S. Juan de la Cruz, Obras completas, Burgos, 1982, pp. 1124-1135)	
	\end{flushright}
	Licet plura mystéria et mirabília sancti doctóres detéxerint et ánimæ devótæ in huiúsmodi vitæ statu cognóverint, adhuc tamen illis pótior pars enuntiánda, immo et intellegénda restat.\\
	Quare profúnde fodiéndum est in Christo, qui est instar fodínæ abundántis, plúrimos thesaurórum sinus habéntis, ubi quantumcúmque alte quis fodit, eórum finem vel términum numquam invéniet. Immo, in quólibet sinu novæ novárum divitiárum venæ hic et illic reperiúntur.\\
	Hanc ob causam de eódem Christo apóstolus Paulus dixit: \emph{In quo sunt omnes thesáuri sapiéntiæ et sciéntiæ Dei abscónditi.} Quos thesáuros íngredi nequit ánima, nec ad illos pertíngere valet, nisi prius densitátem labórum transíerit et ingréssa fúerit, intérius exteriúsque patiéndo, nísique prius plúrima a Deo benefícia intellectuália et sensibília recéperit diutúrnaque spirituális præcésserit exercitátio.\\
	\\
	Resp 2  Ecce Radix Iesse\\
	\\
	Hæc ómnia quippe inferióra sunt et dispositiónes ad sublímia penetrália cognitiónis mysteriórum Christi, quæ est ómnium altíssima sapiéntia quæ in hac vita obtinéri possit.\\
	O útinam tandem hómines agnóscerent esse omníno impossíbile perveníre ad densitátem divitiárum et sapiéntiæ Dei nisi ingrediéndo prius densitátem labórum, multiplíciter patiéndo, ita ut ánima solámen ac desidérium suum repónat. Quam vero ánima quæ concupíscit divínam sapiéntiam, vere prius optat intráre densitátem crucis.\\
	Quaprópter sanctus Paulus Ephésios adhortabátur \emph{ne defícerent in tribulatiónibus, et ut fortíssimi essent et in caritáte radicáti et fundáti, ut possent comprehéndere cum ómnibus sanctis, quæ sit latitúdo et longitúdo et sublímitas et profúndum: scire étiam supereminéntem sciéntiæ caritátem Christi, ut impleréntur in omnem plenitúdinem Dei.}
	Quia porta per quam in huiúsmodi divítias eius sapiéntiæ íngredi potest, crux est, eadémque porta angústa est, et dum plures delícias áppetunt quæ per ipsam pertíngi valent, paucórum ádmodum est ipsam íngredi desideráre.
	
	Resp 3 Amavit eum
	
\end{document}