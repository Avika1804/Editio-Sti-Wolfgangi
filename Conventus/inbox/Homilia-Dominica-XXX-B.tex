\documentclass[options]{article}

\begin{document}
	Ex Homíliis sancti Gregórii Magni papæ in Evangélia 
	\begin{flushright}
	(Hom. 2,5. 7 : PL 76,1084-1085)
	\end{flushright}
Cum in oratióne nostra veheménter insístimus, transeúntem Iesum menti fígimus. Unde dícitur : 
\emph{Stans autem lesus, iussit cæcum addúci ad se.}
Ecce stat qui ante transíbat, quia dum adhuc turbas phantásmatum in oratióne pátimur, lesum aliquátenus transeúntem sentímus. Cum vero oratióni veheménter insístimus, stat lesus ut lucem restítuat, quia Deus in corde fígitur, et lux amíssa reparátur.\\
\\
Et notándum quid cæco veniénti dicat:
\emph{Quid tibi vis fáciam?}
 Numquid qui lumen réddere póterat quid vellet cæcus ignorábat? Sed peti vult id quod et nos pétere et se concédere prænóscit. Importúne namque ad oratiónem nos ádmonet, et tamen dicit:\\
 \emph{Scit namque Pater vester cæléstis quid opus sit vobis ántequam petátis eum.}
 Ad hoc ergo requírit, ut petátur; ad hoc requírit, ut cor ad oratiónem éxcitet. Unde et cæcus prótinus adiúnxit:
  \emph{Dómine, ut vídeam.}
  Ecce cæcus a Dómino non aurum sed lucem quærit. Parvipéndit extra lucem áliquid qu\'{æ}rere, quia etsi habére cæcus quódlibet potest, sine luce vidére non potest quod habet.\\
  \\
  Imitémur ergo, fratres caríssimi, eum quem et córpore audívimus et mente salvátum. Non falsas divítias, non terréna dona, non fugitívos honóres a Dómino, sed lucem quærámus; nec lucem quæ loco cláuditur, quæ témpore finítur, quæ nóctium interruptióne variátur, quæ a nobis commúniter cum pecóribus cérnitur, sed lucem quærámus quam vidére cum solis ángelis possímus, quam nec inítium ínchoat, nec finis angústat.\\
  \\
  Ad quam profécto lucem via fides est. Unde recte et illuminándo cæco prótinus respondétur : 
  \emph{Réspice, fides tua te salvum fecit.}
  Sed ad hæc cogitátio carnális dicit: "Quómodo possum lucem spiritálem qu\'{æ}rere, quam vidére non possum? Unde mihi certum est si sit, quæ corpóreis óculis non infúlget?" Cui scílicet cogitatióni est quod bréviter quisque respóndeat, quia et hæc ipsa quæ sentit, non per corpus, sed per ánimam cógitat. Et nemo suam ánimam videt, nec tamen dúbitat se ánimam habére, quam non videt.\\
  \\
  Resp- 7 Diligam te Domine
  

\end{document}