\documentclass[options]{article}
\begin{document}
Ex Oratiónibus sancti Gregórii Nazianzéni epíscopi
\begin{flushright}
	(Orat. 37, 6-7 : PG 36,290-29)
\end{flushright}
Christus respóndens, dixit pharis\'{æ}is:
\textit{Non legístis, quia is qui fecit hóminem ab inítio másculum et féminam fecit eos.} 
Qu\'{æ}stio hæc, inquit, quam proposuísti, ad pudicítiæ cultum et honórem spectáre mihi vidétur, responsionémque humánam et cómmodam postuláre. Circa hanc enim plerósque male afféctos cerno, ac legem eórum iníquam, nec sibi constántem. Quid enim causæ fuit, cur hómines mulíerem coercérent, maríto contra indulgérent, eúmque líberum relínquerent ? Et múlier quidem, quæ ímprobum consílium advérsus viri sui cubíle suscéperit, adultérii piáculo constringátur, acerbissimísque legum pœnis excruciétur; vir autem, qui fidem uxóri datam per adultérium violáverit, nulli supplício obnóxius sit ? Hanc legem haudquáquam probo, hanc consuetúdinem mínime laudo.\\
\\
Viri erant, qui hanc legem sanxérunt, ac proptérea advérsus mulíeres lata est; quandóquidem et fílios patérnæ potestáti subiecérunt, infirmiórem sexum incúltum atque incurátum reliquérunt. At Deus non sic. Divínæ legis æquabilitátem vidétis: unus viri et mulíeris creátor, pulvis unus utérque, imágo una, lex una, mors una, resurréctio una. Teque ex viro et mulíere procreáti sumus; unum idémque paréntibus débitum fílii persolvéndum habent.\\
\\
Qua ígitur fronte pudicítiam éxigis, quam ipse vicíssim non præstas ? Quómodo, quod non das, petis ? Quómodo córpori pari honóre prǽdito, ímparem legem státuis ? Si deterióra expéndis, peccávit múlier; eódem quoque modo Adam peccávit; utrúmque serpens decépit atque in fraudem ímpulit. Non áltera infírmior invénta est, alter fórtior. At melióra consíderas ? Utrúmque Christus passióne sua salúte donávit. Pro viro caro factus est ? Pro mulíere item. Pro viro mortem súbiit ? Ipsíus item morte mulíeri salus parta est.\\
\\
At ex sémine David nominátur, atque hinc forsan viros honóre præférri cólligis ? Audio, sed ex Vírgine quoque náscitur, quod et pro muliéribus valet.\\
\textit{Erunt ígitur,}
inquit,
\textit{duo in carne una:}
proínde caro, quæ una est, æquálem honórem hábeat. lam vero Paulus exémplo quoque castitátem velut lege præscríbit. Quo tandem modo et qua ratióne?
\textit{Sacraméntum hoc magnum est: ego autem dico in Christo et in Ecclésia.}
Pulchrum est mulíeri Christum per virum reveréri; pulchrum quoque viro Ecclésiam per uxórem non aspernári.
\textit{Múlier,}
inquit,
\textit{ut virum tímeat,}
quippe et Christum.
\textit{At vir}
étiam,
\textit{ut uxórem fóveat atque complectátur;}
nam et Christus Ecclésiam.\\
\\

Resp A solis ortu
   

\end{document}