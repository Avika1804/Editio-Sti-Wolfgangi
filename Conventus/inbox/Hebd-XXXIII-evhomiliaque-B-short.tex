\documentclass[options]{article}
 \usepackage[T1]{fontenc}
\begin{document}
	Léctio sancti Evangélii secúndum Marcum 
	\begin{flushright}
		Mc 13, 24-32
	\end{flushright} 
	In illo témpore: Dixit Iesus discípulis suis:
	«In illis diébus post magnam tribulatiónem sol contenebrábitur, et luna non dabit splendórem suum, et erunt stellæ de cælo decidéntes, et virtútes, quæ sunt in cælis, movebúntur.
	Et tunc vidébunt Fílium hóminis veniéntem in núbibus cum virtúte multa et glória. Et tunc mittet ángelos et congregábit eléctos suos a quáttuor ventis, a summo terræ usque ad summum cæli.
	A ficu autem díscite parábolam: cum iam ramus eius tener fúerit et germináverit fólia, cognóscitis quia in próximo sit æstas. Sic et vos, cum vidéritis hæc fíeri, scitóte quod in próximo sit in óstiis.
	Amen dico vobis: Non tránsiet generátio hæc, donec ómnia ista fiant. Cælum et terra transíbunt, verba autem mea non transíbunt. De die autem illo vel hora nemo scit, neque ángeli in cælo neque Fílius, nisi Pater».\\
	\\
	Ex Sermónibus sancti Augustíni epíscopi
	\begin{flushright}
		(Sermo 97,1.4 : PL 38, 589. 591)
	\end{flushright}
Fratres, quod audístis modo monéntem Scriptúram atque dicéntem, ut propter diem novíssimum vigilémus, unusquísque de novíssimo suo die cógitet; ne forte cum senséritis vel putavéritis longe esse novíssimum s\'{æ}culi diem, dormitétis ad novíssimum vestrum diem. De die novíssimo s\'{æ}culi huius audístis quid díxerit: Quia nésciunt \emph{neque ángeli cælórum, neque Fílius, nisi Pater.} Ubi quidem magna qu\'{æ}stio est, ne carnáliter sapiéntes putémus áliquid Patrem scire, quod nésciat Fílius.\\
\\
Nam útique cum dixit: \emph{Pater scit,} ídeo hoc dixit quia in Patre et Fílius scit. Quid enim est in die, quod non in Verbo factum est, per quem factus est dies? Nemo ergo quærat novíssimum diem, quando futúrus sit; sed vigilémus omnes bene vivéndo, ne novíssimus dies cuiuscúmque nostrum nos invéniat imparátos, et qualis quisque hinc exíerit suo novíssimo die, talis inveniátur in novíssimo s\'{æ}culi die. Nihil te adiuvábit quod hic non féceris. Unumquémque ópera sua iuvábunt, aut ópera sua pressúra sunt.\\
\\
Mundum non amémus. Premit amatóres suos, non eos ad bonum addúcit. Laborándum est in eo pótius ne cápiat, quam timéndum ne cadat. Ecce cadit mundus; stat Christiánus quia non cadit Christus. Nam quare dicit Dóminus: \emph{Gaudéte, quia ego vici mundum?} Respondeámus ei, si placet: « Gaude, sed tu. Si tu vicísti, tu gaude.» Quare nos? Quare nobis dicit: \emph{Gaudéte,} nisi quia nobis vicit, nobis pugnávit? Ubi enim pugnávit? Quia hóminem suscépit.\\
\\
Resp-supermurostuos-CROCHU.gabc


\end{document}