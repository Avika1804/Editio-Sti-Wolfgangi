\documentclass[options]{article}
\usepackage[T1]{fontenc}
\begin{document}
	Léctio sancti Evangélii secúndum Lucam 
\begin{flushright}
Lc 6, 39-45	
\end{flushright}
	In illo témpore:
	Dixit Iesus discípulis similitúdinem: «Numquid potest cæcus cæcum dúcere? Nonne ambo in fóveam cadent?
	Non est discípulus super magístrum; perféctus autem omnis erit sicut magíster eius.
	Quid autem vides festúcam in óculo fratris tui, trabem autem, quæ in óculo tuo est, non consíderas? Quómodo potes dícere fratri tuo: “Frater, sine eíciam festúcam, quæ est in óculo tuo”, ipse in óculo tuo trabem non videns? Hypócrita, éice primum trabem de óculo tuo et tunc perspícies, ut edúcas festúcam, quæ est in óculo fratris tui.
	Non est enim arbor bona fáciens fructum malum, neque íterum arbor mala fáciens fructum bonum. Unaqu\'{æ}que enim arbor de fructu suo cognóscitur; neque enim de spinis cólligunt ficus, neque de rubo vindémiant uvam. Bonus homo de bono thesáuro cordis profert bonum, et malus homo de malo profert malum: ex abundántia enim cordis os eius lóquitur».\\
	\\
	Ex Sermónibus sancti Cæsárii Arelaténsis epíscopi
	\begin{flushright}
	(Sermo Morin 160A : PLS 4, 405)	
	\end{flushright}
	Audívimus, fratres caríssimi, cum evangélium legerétur dixísse Dóminum ad turbas vel ad discípulos suos: \emph{Bonus homo de bono thesáuro cordis sui profert bona; malus homo de malo thesáuro cordis sui profert mala: ex abundántia enim cordis os lóquitur.} Si diligénter considerátis, fratres, duo génera vel duo loca thesaurórum Christus Dóminus demonstrávit: thesáurum bonum in corde bono, thesáurum malum in corde malo.\\
	\\
	Itaque, fratres, considerémus consciéntias nostras, considerémus interióres arcéllas ánimæ nostræ, et videámus cuius ibi recónditum thesáurum habémus: tunc enim scire potérimus ad cuius domínium nos aut thesáurus ipse pertíneat. Si enim, Deo adiuvánte, semper quæ bona sunt cogitémus, et quæ honésta sunt agámus, non solum thesáurum Christi in corde servámus, sed étiam nos ipsi Christi thesáurus sumus; si vero cogitatióne sórdida vel malígna ánimus noster fúerit occupátus, ad quem pertíneat thesáurus cordis nostri non opus est dícere.\\
	\\
	Unusquísque enim qualem habúerit thesáurum, talem habébit et Dóminum. Nam quia in bonis homínibus Christus hábitat et Apóstolus testis est, \emph{in interióri hómine per fidem habitáre Christum}; quod vero diábolus in malis hábitat, de Iuda in evangélio légimus: Cum iam, inquit, \emph{ascendísset diábolus in corde Iudæ ut tráderet Dóminum.} Agnóscite ergo, fratres, quia secúndum quod se unusquísque cum Dei adiutório præparáre volúerit, aut Christi aut adversárii posséssio erit.\\
	\\
	Et quia, quod iam díximus, \emph{ex abundántia cordis os lóquitur}, si vis scire quid ab hómine habeátur in corde, atténde quid proferátur ex ore: si enim quæ sancta, quæ iusta, qua pie sunt loquátur, Christus , qui hábitat in corde, ipse sonat in ore; si vero turpilóquia, convícia, maledícta, oppróbria, detractiónes, murmuratiónes vídeas hóminem ex ore proférre, quis intus hábitet póteris evidénter agnóscere. Si Christum, suscéperis, cum ipso eris regnatúrus in cælo.\\
	\\
	Resp 7 resp-fiatmanustua-CROCHU-cumdox.gabc
\end{document}