\documentclass[options]{article}
\usepackage[T1]{fontenc}
\begin{document}
	Léctio sancti Evangélii secúndum Ioánnem 
	\begin{flushright}
		Io 1,1-18	
	\end{flushright}	
In princípio erat Verbum, et Verbum erat apud Deum, et Deus erat Verbum.\\
Hoc erat in princípio apud Deum. Omnia per ipsum facta sunt, et sine ipso factum est nihil, quod factum est; in ipso vita erat, et vita erat lux hóminum, et lux in ténebris lucet, et ténebræ eam non comprehendérunt.\\
Fuit homo missus a Deo, cui nomen erat Ioánnes; hic venit in testimónium, ut testimónium perhibéret de lúmine, ut omnes créderent per illum. Non erat ille lux, sed ut testimónium perhibéret de lúmine.\\
Erat lux vera, quæ illúminat omnem hóminem, véniens in mundum. In mundo erat, et mundus per ipsum factus est, et mundus eum non cognóvit. In própria venit, et sui eum non recepérunt.\\
Quotquot autem accepérunt eum, dedit eis potestátem fílios Dei fíeri, his, qui credunt in nómine eius, qui non ex sanguínibus neque ex voluntáte carnis neque ex voluntáte viri, sed ex Deo nati sunt.\\
Et Verbum caro factum est et habitávit in nobis; et vídimus glóriam eius, glóriam quasi Unigéniti a Patre, plenum grátiæ et veritátis.\\
Ioánnes testimónium pérhibet de ipso et clamat dicens: «Hic erat, quem dixi: Qui post me ventúrus est, ante me factus est, quia prior me erat».\\
Et de plenitúdine eius nos omnes accépimus, et grátiam pro grátia; quia lex per Móysen data est, grátia et véritas per Iesum Christum facta est.\\
Deum nemo vidit umquam; unigénitus Deus, qui est in sinum Patris, ipse enarrávit.\\
	\\
Ex Tractátu sancti Hilárii epíscopi De Trinitáte
	\begin{flushright}
		(Lib. 2,11. 21: CCL 62,48-49.56-57)	
	\end{flushright}
Est Fílius ab eo Patre qui est, unigénitus ab ingénito, progénies a parénte, vivus a vivo. Ut Patri vita in semetípso, ita et Fílio data est vita in semetípso. Perféctus a perfécto, quia totus a toto; non divísio aut discíssio, quia alter in áltero, et plenitúdo divinitátis in Fílio est. Incomprehensíbilis ab incomprehensíbili; novit enim nemo, nisi ínvicem. Invisíbilis ab invisíbili, quia \emph{imágo Dei invisíbilis est;} et quia \emph{qui vidit Fílium, vidit et Patrem.}\\
\\
Alius ab álio; quia Pater et Fílius: non natúra divinitátis ália et ália, quia ambo unum. Deus a Deo, ab uno ingénito Deo unus unigénitus Deus; non dii duo, sed unus ab uno; non ingéniti duo, quia natus est ab innáto; alter ab áltero nihil dífferens, quia vita vivéntis in vivo est. Hæc de natúra divinitátis attígimus, non summam intellegéntiæ comprehendéntes, sed intellegéntes esse incomprehensibília quæ loquámur. «Nullum ergo dicis, offícium est fídei, si nihil póterit comprehéndi.» Immo hoc offícium fides profiteátur, id unde qu\'{æ}ritur incomprehensíbile sibi esse se scire.\\
\\
Hæc \emph{vita lux hóminum est,} hæc lux ténebras illúminans. Et ut impossibilitátem generatiónis eius enarrándæ secúndum Prophétam piscátor consolarétur, adiécit: \emph{Et ténebræ eam non comprehendérunt.} Cessit sermo natúræ, et non habet quo excúrrat; et tamen hoc piscátor iste récubans in Dómini pectus accépit.\\
\\
Non est iste s\'{æ}culi sermo; quia de qua ágitur, non s\'{æ}culi res est. Edátur áliquid, si in significatióne verbórum reperíri potest ultra quam dictum sit: et si qua sunt ália expósitæ a nobis natúræ nómina, proferántur. Quæ si non sunt, immo quia non sunt; mirémur hanc in piscatóre doctrínam, et in eo elóquia Dei sentiámus; confessionémque Patris et Fílii, ingéniti et unigéniti inenarrábilem et excedéntem compléxum omnem et sermónis et sensus, teneámus atque adorémus; et in Dómino Iesu, exémplo Ioánnis, ut hæc possímus sentíre et cólloqui, accubémus.\\
\\	
	Resp : resp-verbumcarofactumest-CROCHU-cumdox.gabc
\end{document}