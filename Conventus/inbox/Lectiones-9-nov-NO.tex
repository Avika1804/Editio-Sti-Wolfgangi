\documentclass[options]{article}
 \usepackage[T1]{fontenc}
 \begin{document}
 	De Epístola prima beáti Petri apóstoli 	
\begin{flushright}
 1 P. 2, 1-17
\end{flushright}
Caríssimi: Deponéntes omnem malítiam et omnem dolum et simulatiónes et invídias et omnes detractiónes, sicut modo géniti infántes, rationále sine dolo lac concupíscite, ut in eo crescátis in salútem, si
\emph{gustástis quóniam dulcis Dóminus.}
Ad quem accedéntes, lápidem vivum, ab homínibus quidem reprobátum, coram Deo autem eléctum, pretiósum, et ipsi tamquam lápides vivi ædificámini domus spiritális in sacerdótium sanctum offérre spiritáles hóstias acceptábiles Deo per Iesum Christum. Propter quod cóntinet Scriptúra:
  \emph{«Ecce pono in Sion lápidem angulárem, eléctum, pretiósum;  et, qui credit in eo, non confundétur».}\\
  \\
  Vobis ígitur honor credéntibus; non credéntibus autem
  \emph{«Lapis, quem reprobavérunt ædificántes, hic factus est in caput ánguli»} et
  \emph{ «lapis offensiónis et petra scándali»};
  qui offéndunt verbo non credéntes, in quod et pósiti sunt. Vos autem 
  \emph{genus eléctum, regále sacerdótium, gens sancta, pópulus in acquisitiónem, ut virtútes annuntiétis}
  eius, qui de ténebris vos vocávit in admirábile lumen suum; qui aliquándo
  \emph{non pópulus,}
  nunc autem
  \emph{pópulus Dei; qui non consecúti misericórdiam,}
  nunc autem  
  \emph{ misericórdiam consecúti.}\\
  \\
  Caríssimi, óbsecro tamquam ádvenas et peregrínos abstinére vos a carnálibus desidériis, quæ mílitant advérsus ánimam; conversatiónem vestram inter gentes habéntes bonam, ut in eo, quod detréctant de vobis tamquam de malefactóribus, ex bonis opéribus considerántes gloríficent Deum in die visitatiónis.\\
  \\
  Subiécti estóte omni humánæ creatúræ propter Dóminum: sive regi quasi præcellénti sive dúcibus tamquam ab eo missis ad vindíctam malefactórum, laudem vero bonórum; quia sic est volúntas Dei, ut benefaciéntes obmutéscere faciátis imprudéntium hóminum ignorántiam, quasi líberi, et non quasi velámen habéntes malítiæ libertátem, sed sicut servi Dei. Omnes honoráte, fraternitátem dilígite, Deum timéte, regem honorificáte.\\
  \\
  resp-indedicationetempli-CROCHU.gabc\\
  \\\\
  
  Ex Sermónibus sancti Cæsárii Arelaténsis epíscopi 
  \begin{flushright}
  	  (Sermo 229, 1-3: CCL 104, 905-908)
    \end{flushright}
  	  Natálem templi huius diem, fratres dilectíssimi, Christo propítio cum exsultatióne et gáudio hódie celebrámus; sed templum Dei verum et vivum nos esse debémus. Mérito tamen sollemnitátem matris Ecclésiæ christiáni pópuli fidéliter colunt, per quam se spiritáliter renátos esse cognóscunt. Nam qui per primam nativitátem vasa iræ Dei fúimus, per secúndam vasa misericórdiæ fíeri merúimus. Prima enim natívitas prodúxit nos ad mortem; secúnda revocávit ad vitam.\\
  	   Omnes enim nos, caríssimi, ante baptísmum fana diáboli fúimus, post baptísmum templa Christi esse merúimus; et si de salúte ánimæ nostræ atténtius cogitémus, templum Dei verum et vivum nos esse cognóscimus. 
  	   \emph{Non} solum \emph{in manufáctis hábitat} Deus, nec in domo de lignis et lapídibus facta; sed præcípue in ánima ad imáginem Dei facta, et manu ipsíus artíficis cóndita. Sic enim beátus Paulus apóstolus dixit: \emph{Templum Dei sanctum est, quod estis vos.}\\
  	   \\
  	   resp-fundataest-CROCHU.gabc\\
  	   \\
  	   Et quia Christus véniens proiécit diábolum de córdibus nostris, ut sibi templum præparáret in nobis, quantum póssumus cum ipsíus adiutório laborémus, ne in nobis per mala ópera nostra patiátur iniúriam. Omnis enim qui male agit, Christo iniúriam facit. Sicut enim dixi, prius quam nos redímeret Christus, diáboli domus fúimus; póstea domus Dei esse merúimus: Deus enim de nobis dignátus est sibi fácere domum.\\
  	   Unde nos, caríssimi, si natálem templi cum gáudio celebráre vólumus, templa Dei vivéntia malis opéribus in nobis destrúere non debémus. Et hoc dicam, quod omnes intellégere possunt: quóties ad ecclésiam venímus, qualem illam inveníre vólumus, tales et ánimas nostras præparáre debémus.\\
  	     Vis basílicam nítidam inveníre? Noli tuam ánimam peccatórum sórdibus inquináre. Si tu vis ut basílica luminósa sit, et Deus hoc vult, ut ánima tua tenebrósa non sit, sed fiat quod Dóminus dicit, ut lúceat lux in nobis bonórum óperum et ille glorificétur qui in cælis est. Quómodo tu intras in ecclésiam istam, sic Deus vult intráre in ánimam tuam, sicut ipse promísit: \emph{Et habitábo in illis et inambulábo.}\\
  	     \\
  	     resp-benedicdominedomum-CROCHU.gabc 
  	     
  	      \end{document}
  	   
  	   


  
  
  
  
  
  
   