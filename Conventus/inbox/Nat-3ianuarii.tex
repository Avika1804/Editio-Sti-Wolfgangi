\documentclass[options]{article}
\begin{document}
	Ex Tractátibus sancti Augustíni epíscopi in Ioánnem 
	\begin{flushright}
			(Tract. 17, 7-9: CCL 36, 174-175)
	\end{flushright}	
	Venit ipse Dóminus, caritátis doctor, caritáte plenus, \emph{brévians,} sicut de illo prædíctum est, \emph{verbum super terram,} et osténdit in duóbus præcéptis caritátis pendére legem et prophétas.\\
	Quæ sunt illa duo præcépta, fratres, recólite mecum. Notíssima enim esse debent, nec modo tantum veníre in mentem, cum commemorántur a nobis, sed deléri numquam debent de córdibus vestris. Semper omníno cogitáte diligéndum esse Deum et próximum: \emph{Deum ex toto corde, ex tota ánima et ex tota mente; et próximum tamquam seípsum.}\\
	Hæc semper cogitánda, hæc meditánda, hæc retinénda, hæc agénda, hæc implénda sunt. Dei diléctio prior est órdine præcipiéndi, próximi autem diléctio prior est órdine faciéndi. Neque enim qui tibi præcíperet dilectiónem istam in duóbus præcéptis, prius tibi commendáret próximum, et póstea Deum, sed prius Deum, póstea próximum.\\
	Tu autem, quia Deum nondum vides, diligéndo próximum promeréris quem vídeas; diligéndo próximum purgas óculum ad vidéndum Deum, evidénter Ioánne dicénte: \emph{Si fratrem quem vides non díligis, Deum quem non vides quómodo dilígere póteris?}\\
	\\
	resp-hicquiadvenit-CROCHU.gabc
	\\\\
	Ecce dícitur tibi: Dílige Deum. Si dicas mihi: Osténde mihi quem díligam, quid respondébo, nisi quod ait ipse Ioánnes: \emph{Deum nemo vidit umquam?} Et ne te aliénum omníno a Deo vidéndo esse arbitréris: \emph{Deus,} inquit, \emph{cáritas est; et qui manet in caritáte, in Deo manet.} Dílige ergo próximum, et intuére in te unde díligis próximum; ibi vidébis, ut póteris, Deum.\\
	Incipe ergo dilígere próximum. \emph{Frange esuriénti panem tuum, et egénum sine tecto induc in domum tuam; si víderis nudum, vesti et domésticos séminis tui ne despéxeris.}\\
	Fáciens autem ista, quid consequéris? \emph{Tunc erúmpet velut matutína lux tua.} Lux tua Deus tuus est tibi \emph{matutína,} quia post noctem s\'{æ}culi tibi véniet; nam ille nec óritur, nec óccidit, quia semper manet.\\
	Diligéndo próximum, et curam habéndo de próximo tuo, iter agis. Quo iter agis, nisi ad Dóminum Deum, ad eum, quem dilígere debémus ex toto corde, ex tota ánima, ex tota mente? Ad Dóminum enim nondum pervénimus, sed próximum nobíscum habémus. Porta ergo eum, cum quo ámbulas, ut ad eum pervénias, cum quo manére desíderas.\\
	\\
resp-confirmatumest-CROCHU-cumdox.gabc
\end{document}