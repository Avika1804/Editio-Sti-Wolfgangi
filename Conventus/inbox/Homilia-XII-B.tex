\documentclass[options]{article}
\title{Homilía dominica XII B}
\begin{document}
	\textbf{Ex sermone sancti Augustini Episcopi }
	
	\textbf{(Sermo 63, 1-3; PL 38, 424-425)}
	
De lectióne recentíssima sancti evangélii, donánte dómino, álloquor vos, et in illo exhórtor, ut contra tempestátes et fluctus s\'{æ}culi huius non dórmiat fides in córdibus vestris. Non enim revéra dóminus Christus mortem hábuit in potestáte, somnum non hábuit in potestáte; et forte Omnipoténtem navigántem somnus pressit invítum. hoc si credidéritis, dormit in vobis: si autem in vobis vígilat Christus, vígilat fides vestra. Apóstolus dicit, 
\textit{habitáre Christum per fidem in córdibus vestris.} 
ergo et somnus Christi signum est sacraménti. Navigántes sunt ánimæ in ligno s\'{æ}culum transeúntes. Etiam navis illa ecclésiam figurábat. Et sínguli quippe templa sunt Dei, et unusquísque in corde suo návigat: nec facit naufrágium, si bona cógitat. 

Audísti convícium, ventus est: irátus es, fluctus est. Vento ígitur flante, fluctu surgénte, periclitátur navis, periclitátur cor tuum, flúctuat cor tuum. Audíto convício vindicári desíderas: et ecce vindicátus es, et malo aliéno cedens, fecísti naufrágium. Et quare hoc? Quia dormit in te Christus. Quid est, dormit in te Christus? Oblítus es Christum. Excita ergo Christum, recordáre Christum, evígilet in te Christus: consídera illum. Quid volébas? Vindicári. Excídit tibi, quia ipse cum crucifigerétur dixit:

\textit{Pater, ignósce illis, quia nésciunt quid fáciunt?} 
Qui dormiébat in corde tuo, nóluit vindicári. Excita illum, recóle illum. Memória ipsíus, verbum ipsíus: memória ipsíus, iússio ipsíus. Et dices apud te, si vígilat in te Christus: Qualis ego homo, qui volo vindicári? Qui sum ego, qui in hóminem exséro comminatiónes? Mórior forte ántequam víndicer. Et cum anhélans, ira inflammátus, et sítiens vindíctam, exíero de córpore, non me súscipit ille qui nóluit vindicári: non me súscipit ille qui dixit, 
\textit{Date, et dábitur vobis; dimíttite, et dimittétur vobis.} 
Ergo compéscam me ab iracúndia mea, et redíbo ad quiétem cordis mei. Imperávit Christus mari, facta est tranquíllitas. 

Quod autem dixi ad iracúndiam, hoc tenéte reguláriter in ómnibus tentatiónibus vestris. Nata est tentátio, ventus est: turbátus es, fluctus est. Excita Christum, loquátur tecum. 
\textit{quis est hic, quando et venti et mare obédiunt ei?} 
Quis est hic, cui obáudit mare? 
\textit{Ipsíus est mare, et ipse fecit illud}. 
Omnia per ipsum facta sunt. Magis imitáre ventos et mare: obtémpera Creatóri. Sub iussióne Christi mare audit, et tu surdus es? Mare audit, et ventus cessat, et tu sufflas? Quid? Dico, fácio, fingo: quid est áliud nisi suffláre, et sub verbo Christi nolle cessáre? Non vos vincat fluctus in perturbatióne cordis vestri. Sed tamen quia hómines sumus, si ventus impúlerit, si afféctum ánimæ nostræ móverit, non desperémus: Christum excitémus, ut in tranquíllo navigémus, et ad pátriam veniámus.

\end{document}
