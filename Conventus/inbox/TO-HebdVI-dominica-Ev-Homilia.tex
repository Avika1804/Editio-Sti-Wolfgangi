\documentclass[options]{article}
\usepackage[T1]{fontenc}
\begin{document}
	Léctio sancti Evangélii secúndum Lucam 
	\begin{flushright}
		Lc 6,17.20-26
	\end{flushright}
In illo témpore:
Descéndens Iesus cum Duódecim stetit in loco campéstri, et turba multa discipulórum eius, et multitúdo copiósa plebis ab omni Iud\'{æ}a et Ierúsalem et marítima Tyri et Sidónis.
Et ipse, elevátis óculis suis in discípulos suos, dicébat:
«Beáti páuperes, quia vestrum est regnum Dei.\\
Beáti, qui nunc esurítis, quia saturabímini.\\
Beáti, qui nunc fletis, quia ridébitis.\\
Beáti éritis, cum vos óderint hómines et cum separáverint vos et exprobráverint et eiécerint nomen vestrum tamquam malum propter Fílium hóminis. Gaudéte in illa die et exsultáte, ecce enim merces vestra multa in cælo; secúndum hæc enim faciébant prophétis patres eórum.\\
Verúmtamen væ vobis divítibus, quia habétis consolatiónem vestram!\\
Væ vobis, qui saturáti estis nunc, quia esuriétis!\\
Væ vobis, qui ridétis nunc, quia lugébitis et flébitis!\\
Væ, cum bene vobis díxerint omnes hómines! Secúndum hæc enim faciébant pseudoprophétis patres eórum».\\
\\
Ex Enarratiónibus sancti Augustíni epíscopi in psalmos
\begin{flushright}
(En. in ps. 137,10 : CCL 40,1986)	
\end{flushright}
\emph{Si ambulávero in médio tribulatiónis, vivificábis me.} Verum est; in quacúmque tribulatióne fúeris, confitére, ínvoca ; líberat te, vivíficat te. Sed tamen áliquid hic debémus mélius intellégere, quo iam familiárius inhæreámus Deo, dicamúsque illi: \emph{Cito exáudi me.}\\

Díxerat enim: \emph{Excélsa a longe cognóscit}: et excélsa supérba non norunt tribulatiónem. Non norunt, inquam, tribulatiónem de qua dícitur álio loco: \emph{Tribulatiónem et dolórem invéni, et nomen Dómini invocávi.} Quid enim magnum, si tribulátio te invéniat ? Si áliquid potes, tu ínveni tribulatiónem. «Et quis est, inquis, qui invéniat tribulatiónem ? Aut quis illam vel quærat?» — «In médio tribulatiónis es, et nescis? Vita ista parva tribulátio est? Si non est tribulátio, non est peregrinátio; si autem peregrinátio est, aut parum pátriam díligis, aut sine dúbio tribuláris.»\\
\\	
Quis enim non tribulétur, non se esse cum eo quod desíderat? Sed unde tibi non vidétur ista tribulátio? Quia non amas. Ama álteram vitam, et vidébis quia ista vita tribulátio est; quacúmque prosperitáte fúlgeat, quibúslibet delíciis abúndet atque circúmfluat, quando nondum est illud gáudium sine ulla tentatióne certíssimum, quod nobis in fine servat Deus, sine dúbio tribulátio est.»\\
\\	
Ergo intellegámus et huius tribulatiónem, fratres. \emph{Si ambulávero in médio tribulatiónis, vivificábis me.} Non sic ait, tamquam díceret: «Si forte evénerit mihi tribulátio áliqua, liberábis inde me.» Sed quómodo? \emph{Si ambulávero in médio tribulatiónis, vivificábis me,} id est: «Aliter non vivificábis me, nisi in médio tribulatiónis ambulávero.» Væ ridéntibus; beáti lugéntes: \emph{Si ambulávero in médio tribulatiónis, vivificábis me.}\\
\\
resp-beatipauperesspiritu-CROCHU.gabc
\end{document}