\documentclass[options]{article}
\begin{document}
	\textbf{E Régula pastoráli sancti Gregórii Magni papæ}\\
	\begin{flushright}
		(Lib. 2, 4: PL 77, 30-31)
	\end{flushright}
	Sit rector discrétus in siléntio, útilis in verbo, ne aut tacénda próferat aut proferénda reticéscat. Nam sicut incáuta locútio in errórem pértrahit, ita indiscrétum siléntium hos qui erudíri póterant, in erróre derelínquit. Sæpe namque rectóres impróvidi humánam amíttere grátiam formidántes, loqui líbere recta pertiméscunt; et iuxta Veritátis vocem, nequáquam iam gregis custódiæ pastórum stúdio, sed mercenariórum vice desérviunt, quia veniénte lupo fúgiunt, dum se sub siléntio abscóndunt.\\
	Hinc namque eos per Prophétam Dóminus íncrepat, dicens:
	\textit{Canes muti non valéntes latráre.}
	Hinc rursum quéritur dicens: 
	\textit{Non ascendístis ex advérso, nec opposuístis murum pro domo Israel, ut starétis in prœlio in die Dómini.}
	 Ex advérso quippe ascéndere, est pro defensióne gregis voce líbera huius mundi potestátibus contraíre. Et in die Dómini in pr\'{œ}lio stare, est pravis decertántibus ex iustítiæ amóre resístere.\\
	 \\
	 Resp 4 domineneiniratua \\
	 \\
	  Pastóri enim recta timuísse dícere, quid est áliud quam tacéndo terga præbuísse? qui nimírum si pro grege se óbicit, murum pro domo Israel hóstibus oppónit. Hinc rursum delinquénti pópulo dícitur: 
	  \textit{Prophétæ tui vidérunt tibi falsa et stulta, nec aperiébant iniquitátem tuam, ut te ad pæniténtiam provocárent.}
	  Prophétæ quippe in sacro elóquio nonnúmquam doctóres vocántur, qui dum fugitíva esse præséntia índicant, quæ sunt ventúra maniféstant. Quos divínus sermo falsa vidére redárguit, quia dum corrípere culpas métuunt, incássum delinquéntibus promíssa securitáte blandiúntur; qui iniquitátem peccántium nequáquam apériunt quia ab increpatiónis voce conticéscunt.\\
	  \\
	  Resp 5 audiamdomine\\
	  \\
	  Clavis quippe apertiónis est sermo correptiónis, quia increpándo culpam détegit, quam sæpe nescit ipse étiam qui perpetrávit. Hinc Paulus ait: 
	  \textit{Ut potens sit exhortári in doctrína sana, et eos qui contradícunt argúere.}
	  Hinc per Malachíam dícitur:
	  \textit{Lábia sacerdótis custódient sciéntiam et legem requírent ex ore eius, quia ángelus Dómini exercítuum est.}
	  Hinc per Isaíam Dóminus ádmonet, dicens: 
	  \textit{Clama, ne cesses, quasi tuba exálta vocem tuam.}\\
	  Præcónis quippe offícium súscipit quisquis ad sacerdótium accédit, ut ante advéntum iúdicis qui terribíliter séquitur, ipse scílicet clamándo gradiátur. Sacérdos ergo si prædicatiónis est néscius, quam clamóris vocem datúrus est præco mutus? Hinc est enim quod super pastóres primos in linguárum spécie Spíritus Sanctus insédit: quia nimírum quos repléverit, de se prótinus loquéntes facit.\\
	  \\
	  
	  Resp 6 fiat manus tua
	  
\end{document}