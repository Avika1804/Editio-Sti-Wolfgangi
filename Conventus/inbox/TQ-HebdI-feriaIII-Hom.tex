\documentclass[options]{article}
\begin{document}
	Ex Tractátu sancti Cypriáni epíscopi et mártyris De domínica oratióne 
	\begin{flushright}
	(Cap. 1-3: CSEL 3, 267-268)	
	\end{flushright}
	
	Evangélica præcépta, fratres dilectíssimi, nihil sunt áliud quam magistéria divína, fundaménta ædificándæ spei, firmaménta corroborándæ fídei, nutriménta fovéndi cordis, gubernácula dirigéndi itíneris, præsídia obtinéndæ salútis, quæ, dum docíbiles credéntium mentes in terris ínstruunt, ad cæléstia regna perdúcunt.
	
	Multa et per prophétas servos suos dici Deus vóluit et audíri; sed quanto maióra sunt quæ Fílius lóquitur, quæ Dei sermo, qui in prophétis fuit, própria voce testátur, non iam mandans ut parétur veniénti via, sed ipse véniens et viam nobis apériens et osténdens, ut, qui in ténebris mortis errántes impróvidi et cæci prius fúimus, luce grátiæ lumináti iter vitæ duce et rectóre Dómino tenerémus.
	Qui inter cétera salutária sua mónita et præcépta divína, quibus pópulo suo cónsulit ad salútem, étiam orándi ipse formam dedit, ipse quid precarémur mónuit et instrúxit. Qui fecit vívere, dócuit et oráre, benignitáte ea scílicet, qua et cétera dare et conférre dignátus est, ut, dum prece et oratióne quam Fílius dócuit apud Patrem lóquimur, facílius audiámur.
	
	Iam prædíxerat horam veníre, quando veri adoratóres adorárent Patrem in spíritu et veritáte, et implévit quod ante promísit, ut, qui spíritum et veritátem de eius sanctificatióne percépimus, de traditióne quoque eius vere et spiritáliter adorémus.\\
	\\
	resp-paradisiportas-CROCHU.gabc
	\\
	Quæ enim potest esse magis spiritális orátio quam quæ a Christo nobis data est, a quo nobis et Spíritus Sanctus missus est; quæ vera magis apud Patrem precátio quam quæ a Fílio, qui est véritas, de eius ore proláta est? Ut áliter oráre quam dócuit non ignorántia sola sit, sed et culpa, quando ipse posúerit et díxerit: \emph{Reícitis mandátum Dei, ut traditiónem vestram statuátis.}
	
	Orémus ítaque, fratres dilectíssimi, sicut magíster Deus dócuit. Amíca et familiáris orátio est Deum de suo rogáre, ad aures eius ascéndere Christi oratiónem.
	
	Agnóscat Pater Fílii sui verba, cum precem fácimus: qui hábitat intus in péctore, ipse et in voce; et cum ipsum habeámus apud Patrem advocátum pro peccátis nostris, quando peccatóres pro delíctis nostris pétimus, advocáti nostri verba promámus. Nam cum dicat: \emph{Quia quodcúmque petiérimus a Patre in nómine eius dabit nobis,} quanto efficácius impetrámus quod pétimus Christi nómine, si petámus ipsíus oratióne?\\
	\\
	resp-ductusestiesusindesertum-CROCHU-cumdox.gabc
\end{document}