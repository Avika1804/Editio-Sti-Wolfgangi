\documentclass[options]{article}
\usepackage[T1]{fontenc}
\begin{document}
	Léctio Sancti Evangélii secúndum Lucam
	\begin{flushright}
		Lc 3. 15-16, 21-22.
	\end{flushright}
In illo témpore:
Existimánte pópulo et cogitántibus ómnibus in córdibus suis de Ioánne, ne forte ipse esset Christus, respóndit Ioánnes dicens ómnibus: «Ego quidem aqua baptízo vos. Venit autem fórtior me, cuius non sum dignus sólvere corrígiam calceamentórum eius: ipse vos baptizábit in Spíritu Sancto et igni».
Factum est autem, cum baptizarétur omnis pópulus, et Iesu baptizáto et oránte, apértum est cælum, et descéndit Spíritus Sanctus corporáli spécie sicut colúmba super ipsum; et vox de cælo facta est: «Tu es Fílius meus diléctus; in te complácui mihi».\\
\\
Ex Sermónibus sancti Gregórii Antiochéni epíscopi
\begin{flushright}
	(Sermo 2, 2.5.7 : PG 88,1871.1875.1878)
\end{flushright}
\emph{Hic est Fílius meus diléctus.} Hic est qui mecum Spíritum Sanctum misit supra se ipsum; vicissímque quem misit Spíritum in se recépit. Hic ille est ante ómnia s\'{æ}cula ex me génitus; génitus, inquam, non creátus; génitus non ascítus; Fílius ex me uno génitus, unigénitus única genitúra, prout ipse solus scio, et prout ipse quoque solus novit. Hic est perfectiónis meæ perfécta expréssio; hic est qui deitátem meam in se éxprimit; hic est qui meam substántiam maniféste repræséntat; Fílius meus est, non aliénus; Fílius meus, mihi consubstantiális, non ex aliéna substántia constans, consubstantiális mihi ceu lumen de lúmine, vita ex vita, fons ex fonte, véritas ex veritáte, vis ex vi, Deus de Deo.\\
\\
Hic est qui a meo sinu non separátus, Maríæ sinum occupávit, qui et in me mansit inseparabíliter, et in illa éxstitit incircumscríptus; qui et in cælis est indivisibíliter, et in Vírginis útero habitávit intemeráte: quam enim creándo non fúerat ipse pollútus, eámdem incoléndo non est contaminátus. Hic est qui de beáto ventre procéssit, tamquam ex thálamo virgináli sponsus hílaris; qui generatióne sua generatiónem terrigenárum honorávit, partúque virginitátem paréntis suas obsignávit; qui per virginitátis iánuam in mundum ingréssus, a quo numquam afúerat, integritátis claustra non fregit.\\
\\
Etiámsi vero vidéritis Fílium meum esuriéntem, aut sitiéntem, aut dormiéntem, aut ambulántem, aut fatigátum, aut flagéllis cæsum, aut sponte crucifíxum, vel clavis haud citra consénsum reténtum, aut non sine própria voluntáte mórtuum, aut tamquam mórtuum in sepúlcro custodítum: hæc ómnia in carnem eius conférte. Prætérea si hunc Fílium meum vidéritis lepram verbo purgántem, oculórum cæcitáti luto medéntem, natúram nutu reformántem, quinque pánibus quinque hóminum mília saturántem, hæc ómnia divinitáti eius attribúite.\\
\\
Cavéte quóminus altérius cuiúsdam sublímia ópera, et item altérius cuiúsdam humília esse existimétis; sed uni eidémque et hæc et illa imputáte. Eius enim sunt divína ómnia, eiúsque item sunt humána ómnia : eius sunt creatiónes, eius item passiónes. Unus quippe idémque est Fílius meus, qui et suæ divinitátis ópera facit, et ómnia quæ carnis suæ sunt sibi própria facit.\\
\\
Resp resp-incolumbaespecie-CROCHU-cumdox.gabc
\end{document}