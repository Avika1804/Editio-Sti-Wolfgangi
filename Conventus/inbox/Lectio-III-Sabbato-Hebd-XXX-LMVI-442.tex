\documentclass[options]{article}
\begin{document}
	Ex Sermónibus sancti Columbáni abbátis
	\begin{flushright}
	(Sermo 13, 2-3 : Op. ómnia 1957,118-120)	
	\end{flushright}
Fratres, vocatiónem hanc sequámur, qua ad vitæ fontem vocámur a Vita, qui est Fons, non solum aquæ vivæ, sed et fons ætérnæ vitæ, fons lucis, idem et fons lúminis; ab illo enim hæc ómnia sunt, sapiéntia et vita, et lux ætérna. Auctor vitæ fons vitæ est, lucis creátor, fons lúminis est; et ídeo contémptis his quæ vidéntur, transcénso s\'{æ}culo, in superióribus cælórum fontem lúminis, fontem vitæ, fontem aquæ vivæ, ut rationábiles et sagacíssimi pisces quærámus, ut ibi bibámus
\emph{aquam}
 vivam et
 \emph{saliéntem in vitam ætérnam.}\\
 \\
 Utinam me illuc dignáres adscíscere ad illum fontem, Deus miséricors, pie Dómine, ut ibi et ego cum sitiéntibus tuis vivam undam vivi fontis aquæ vivæ bíberem, cuius nímia dulcédine delectátus sursum semper ei hærérem et dícerem: "Quam dulcis est fons aquæ vivæ, cuius non déficit
 \emph{aqua sáliens in vitam ætérnam."}\\
 \\
 O Dómine, tu es ipse iste fons semper et semper desiderándus, semper licet et semper hauriéndus. Nobis semper 
 \emph{da, Dómine}
 Christe,
 \emph{hanc aquam}
 ut sit in nobis quoque fons
 \emph{aquæ}
 vivæ et
 \emph{saliéntis in vitam ætérnam.}
 Magna quidem posco, quis nésciat? Sed tu, Rex glóriæ, magna donáre nosti et magna promisísti; nihil te maius, et te nobis donásti, te pro nobis dedísti. Unde te rogámus, ut sciámus quod amámus, quia nihil áliud præter te nobis dari postulámus; tu es enim ómnia nostra, vita nostra, lux nostra, salus nostra, cibus noster, potus noster, Deus noster. Inspíra corda nostra, rogo, Iesu noster, illa tui Spíritus aura, et vúlnera nostras tua caritáte ánimas, ut possit uniuscuiúsque nostrum ánima in veritáte dícere:
  \emph{Indica mihi quem diléxit ánima mea}
  quóniam vulneráta caritáte ego sum.
 \\
 \\ 
   resp-civitatemistamtucircumda-CROCHU.gabc
\end{document}