\documentclass[options]{article}
\begin{document}
	Ex Catechésibus sancti Cyrílli Hierosolymitáni epíscopi.
	\begin{flushright}
		(Cat. 15, 1-3: PG 33, 870-874)
	\end{flushright}
Christi advéntum annuntiámus, non unum dumtáxat, sed et álterum prióre multo speciosiórem. Ille enim patiéntiæ significatiónem pr\'{æ}tulit; iste vero divíni regni diadéma feret.\\
Nam ut plúrimum duplícia sunt ómnia apud Dóminum nostrum Iesum Christum. Duplex natívitas, áltera ex Deo ante s\'{æ}cula, áltera ex Vírgine in consummatióne sæculórum; duplex descénsus: unus obscúrus, qui tamquam plúviæ in vellus, alter vero conspícuus, nimírum futúrus.\\
In primo advéntu fásciis involútus est in præsépio: in secúndo amicítur lúmine quasi vestiménto. In prióre sustínuit crucem, ignomíniam aspernátus; in áltero véniet angelórum exércitu stipátus, glorificátus.\\
Non ígitur in primo tantum advéntu acquiéscimus, sed et secúndum exspectámus. Cumque in prióre dixérimus: \emph{Benedíctus qui venit in nómine Dómini} idem rursum dicémus in posterióre: ut cum ángelis occurréntes Dómino, adorántes clamémus: \emph{Benedíctus qui venit in nómine Dómini.}\\
\\
Resp 4  Ave Maria\\
\\
  Véniet Salvátor non ut íterum iudicétur, sed ut in ius vocet eos a quibus in iudícium est vocátus. Qui prius cum iudicarétur tácuit, conscelerátis qui immánia in illum, quando in crucem egére ausi sunt, in memóriam révocans dicet: \emph{Hæc fecísti et tácui.}\\
Tunc, piæ dispensatiónis causa, hómines leni persuasióne docens venit: eo vero témpore, velint, nolint, regno illíus necessário subiciéntur.\\
De utróque illo advéntu lóquitur Malachías prophéta: \emph{Et statim véniet ad templum suum Dóminus, quem vos qu\'{æ}ritis. Ecce unus advéntus.}\\
Et rursus de áltero ita infit: \emph{Et Angelus testaménti quem vos qu\'{æ}ritis. Ecce véniet Dóminus omnípotens: et quis sustinébit diem ingréssus eius, aut quis subsístet in visióne eius? eo quod ipse accédit tamquam ignis conflatórii, tamquam herba lavántum: et sedébit conflans et emúndans.}\\
\\
Resp 5 Salvatorem exspectamus\\
\\
Duos quoque illos advéntus signíficat Paulus ad Titum scribens his verbis: \emph{Appáruit grátia Dei salvatóris ómnibus homínibus, erúdiens nos, ut abnegántes impietátem et mundána desidéria, temperánter et pie et iuste vivámus in præsénti s\'{æ}culo, exspectántes beátam spem et apparitiónem glóriæ magni Dei et salvatóris nostri Iesu Christi.} Viden, quómodo primum expréssit advéntum, de quo grátias agit; álterum vero, quem exspectámus.\\
Proptérea fídei quam profitémur tenor nunc ita est tráditus, ut credámus in eum \emph{qui et ascéndit in cælos, et consédit a dextris Patris. Et ventúrus est in glória iudicáre vivos et mórtuos; cuius regni non erit finis.}\\
Véniet ígitur Dóminus noster Iesus Christus e cælis. Véniet vero circa finem mundi huius cum glória in postréma die. Fiet enim mundi huius consummátio, et factus iste mundus rursum renovábitur.\\
\\
Resp 6 Obsecro, Domine\\
\\

\end{document}