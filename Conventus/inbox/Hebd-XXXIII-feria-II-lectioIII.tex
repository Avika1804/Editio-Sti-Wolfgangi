\documentclass[options]{article}
 \usepackage[T1]{fontenc}
\begin{document}
	Ex Tractátu sancti Fulgéntii Ruspénsis epíscopi \emph{De remissióne}
	\begin{flushright}
		(Liber 2, 11 2 — 12, 1. 3-4: CCL 91 A, 693-695)
	\end{flushright}
\emph{In moménto, in ictu óculi, in novíssima tuba, canet enim tuba, et mórtui resúrgent incorrúpti et nos immutábimur.} Cum dicit «nos», osténdit Paulus illos secum futúræ mutatiónis potitúros dono, quos in hoc témpore cum ipso sociísque eius tenet ecclesiástica rectéque vivéndi commúnio. Ipsíus vero mutatiónis insínuans qualitátem dicit: \emph{Opórtet enim corruptíbile hoc indúere incorruptiónem, et mortále hoc indúere immortalitátem.} Ut ergo in tálibus tunc sequátur mutátio iustæ retributiónis, nunc præcédit mutátio gratuítæ largitátis.\\
 His ergo qui in præsénti vita de malo in bonum fúerint immutáti, futúræ mutatiónis retribútio promíttitur.\\
 \\
 Resp 2  resp-murotuoinexpugnabili-CROCHU.gabc\\
 \\
  Hoc ergo in eis ágitur per grátiam, ut primum hic in eis per iustificatiónem, in qua spiritáliter resúrgitur, mutátio divíni múneris inchoétur et póstmodum in córporis resurrectióne, qua iustificatórum immutátio perfícitur, in ætérnum manens perfécta glorificátio non mutétur. Ad hoc enim illos primum grátia iustificatiónis, deínde glorificatiónis mutat, ut in eis ipsa glorificátio incommutábilis æternáque permáneat.\\
   Mutántur hic enim per primam resurrectiónem, qua illuminántur ut convertántur; qua scílicet tránseunt de morte ad vitam, de iniquitáte ad iustítiam, de infidelitáte ad fidem, de malis áctibus ad sanctam conversatiónem. Ideo in illis secúnda mors non habet potestátem. De tálibus in Apocalýpsi dícitur: \emph{Beátus qui habet partem in resurrectióne prima; in his secúnda mors non habet potestátem.} In eódem rursus libro dícitur: \emph{Qui vícerit, a morte secúnda non lædétur.} Sicut ergo in conversióne cordis prima consístit resurréctio, sic mors secúnda in supplício sempitérno.\\
    Festínet ítaque hic primæ resurrectiónis párticeps fíeri omnis qui non vult secúndæ mortis ætérna punitióne damnári. Si qui enim, in præsénti vita timóre divíno mutáti, tránseunt a vita mala ad vitam bonam, ipsi tránseunt de morte ad vitam, qui étiam póstmodum de ignobilitáte mutabúntur in glóriam.\\
    \\
    Resp 3 resp-misitdominusangelumsuum-CROCHU-cumdox.gabc
 
\end{document}