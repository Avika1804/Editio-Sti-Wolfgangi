\documentclass[options]{article}
\begin{document}
	Ex Tractátibus sancti Augustíni epíscopi in Ioánnem
	\begin{flushright}
		 (Tract. 26, 4-6: CCL 36, 261-263)
	\end{flushright}
	\emph{Nemo venit ad me, nisi quem Pater attráxerit.}
	Noli te cogitáre invítum trahi; tráhitur ánimus et amóre. Nec timére debémus ne ab homínibus qui verba perpéndunt, et a rebus máxime divínis intellegéndis longe remóti sunt, in hoc Scripturárum sanctárum evangélico verbo fórsitan reprehendámur, et dicátur nobis: "Quómodo voluntáte credo, si trahor?". Ego dico: "Parum est voluntáte, étiam voluptáte tráheris".\\
	 Quid est trahi voluptáte?
	 \emph{Delectáre in Dómino et dabit tibi petitiónes cordis tui.}
	 Est quædam volúptas cordis, cui panis dulcis est ille cæléstis. Porro si poétæ dícere lícuit: "Trahit sua quemque volúptas", non necéssitas sed volúptas, non obligátio sed delectátio, quanto fórtius nos dícere debémus trahi hóminem ad Christum, qui delectátur veritáte, delectátur beatitúdine, delectátur iustítia, delectátur sempitérna vita, quod totum Christus est?\\
	   An vero habent córporis sensus voluptátes suas, et ánimus deséritur a voluptátibus suis? Si ánimus non habet voluptátes suas, unde dícitur:
	   \emph{Fílii autem hóminum sub tégmine alárum tuárum sperábunt, inebriabúntur ab ubertáte domus tuæ, et torrénte voluptátis tuæ potábis eos, quóniam apud te est fons vitæ, et in lúmine tuo vidébimus lumen?}\\
	   \\
	    resp-delectareindomino-CROCHU.gabc\\
	    \\
	    Da amántem, et sentit quod dico. Da desiderántem, da esuriéntem, da in ista solitúdine peregrinántem atque sitiéntem et fontem ætérnæ pátriæ suspirántem, da talem, et scit quid dicam. Si autem frígido loquor, nescit quid loquor.\\
	     Ramum víridem osténdis ovi, et trahis illam. Nuces púero demonstrántur, et tráhitur; et quo currit tráhitur, amándo tráhitur, sine læsióne córporis tráhitur, cordis vínculo tráhitur. Si ergo ista, quæ inter delícias et voluptátes terrénas revelántur amántibus, trahunt, quóniam verum est "Trahit sua quemque volúptas", non trahit revelátus Christus a Patre? Quid enim fórtius desíderat ánima quam veritátem? Quo ávidas fauces habére debet, unde optáre ut sanum sit intus palátum vera iudicándi, nisi ut mandúcet et bibat sapiéntiam, iustítiam, veritátem, æternitátem?\\
	     \emph{Beáti}
	     enim, inquit,
	     \emph{qui esúriunt et sítiunt iustítiam,}
	     sed hic! 
	      \emph{quóniam saturabúntur,}
	     sed ibi! Reddo illi quod amat, reddo quod sperat; vidébit quod adhuc non vidéndo crédidit; manducábit quod ésurit, saturábitur eo quod sitit. Ubi? In resurrectióne mortuórum, quia 
	      \emph{ego resuscitábo eum in novíssimo die.}\\
	      \\
	      resp-laetenturcaeli-CROCHU-cumdox.gabc\\
	      
	       
	     
	    
	 
	
	
\end{document}