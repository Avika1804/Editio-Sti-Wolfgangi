\documentclass[options]{article}
\begin{document}
	Ex Libris Exercítium sanctæ Gertrudis vírginis
\begin{flushright}
	(Exerc. 6 : SC 127,222-224)	
\end{flushright}	
	\emph{Cor meum et caro mea exsultavérunt in} te \emph{Deum vivum,} et ánima mea lætáta est in te, verum salutáre meum. Concupíscit et déficit virtus ánimæ meæ, super intróitu glórias tuæ, \emph{Deus, Deus meus,} cordis mei amor et iúbilus, refúgium et virtus, glória mea et laus mea Deus, o quando laudábit te ánima mea in ecclésia sanctórum? O quando óculi mei vidébunt te, Deum meum, Deum deórum? O quando tríbues mihi desidérium ánimæ meæ, in manifestatióne glóriæ tuæ? Deus meus, pórtio mea electíssima, fortitúdo et glória mea. O quando índues me pállio laudis pro spíritu mæróris, ut simul cum ángelis ómnia membra mea tibi reddant \emph{hóstiam vociferatiónis?}\\
	 Deus vitæ meæ, o quando introíbo in tabernáculum glóriæ tuæ, ut et ego proclámem tibi Allelúia splendidíssimum, et coram ómnibus sanctis tuis confiteátur tibi ánima mea et cor meum, quia magnificásti misericórdias tuas mecum? Deus meus, præclára heréditas mea, o quando contríto láqueo mortis huius, sine médio vidébit et laudábit te única mea ? O quando inhabitábo in tabernáculo tuo in s\'{æ}cula, ut laudem nomem tuum assídue, et hymnum novum dicam magnificéntiæ tuæ super multitúdine misericórdiæ tuæ?\\
	\emph{Non est símilis tui in diis, Dómine} mi, et non est comparátio altitúdinis divitiárum admirábilis glóriæ tuæ. Quis investigábit abýssum sapiéntiæ tuæ, et quis dinumerábit infinítos thesáuros copiosíssimæ misericórdiæ tuæ? Vere non est tantus, non est talis, ut tu, Deus meus rex immortális. Quis explicábit glóriam tuæ maiestátis? Quis salutári póterit visu tuæ claritátis? Quómodo suffíciet óculos visu, vel auris audítu super admiratióne glóriæ tui vultus? \emph{Deus, Deus meus,}  tu solus admirábilis es et gloriósus.
\end{document}