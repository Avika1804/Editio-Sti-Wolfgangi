\documentclass[options]{article}
\begin{document}
	Ex Sermónibus sancti Bernárdi abbátis
	\begin{flushright}
			(Sermo 19 de divérsis, 2-3 : EC 6,162-163)
	\end{flushright}
Quid est regnum Dei ? \emph{Iustítia et pax et gáudium in Spíritu Sancto.} Atténdis et inténdis quia gáudium in fine est? Sic fátui fílii Adam, et præcípiti saltu iustítiam transiliéntes et pacem, rem finálem in princípium convértere et pervértere vultis? Nemo enim est qui gaudére non velit. Non stabit et non erit istud, quia, sicut \emph{non est pax ímpiis,} sic nec gaudére ímpiis, \emph{dicit Dóminus. Non sic ímpii, non sic}. 

Prius est iustítiam fácere, inquírere pacem et pérsequi eam, et sic demum apprehéndere gáudium, immo a gáudio comprehéndi. Sic angélicus ille convéntus prius iustítiam fecit, cum stetit in veritáte et veritátis deséruit desertórem. Post hæc, illa pace firmáti sunt quæ exsúperat omnem sensum, quia, cum divérsis honórum primátibus ambiántur, nullus qui múrmuret, qui invídeat nullus.

\emph{Lauda} tu, \emph{Ierúsalem, Dóminum, lauda Deum tuum, Sion, quóniam confortávit seras portárum tuárum, benedíxit fíliis tuis in te, qui pósuit fines tuos pacem}. Iam non est timor in fínibus tuis, \emph{quia pósuit fines tuos pacem.}  Nullæ ibi tentatiónes, nulla sese cogitatiónum turma confúndit. Ille qui idem est, ómnia in identitáte consólidat atque coniúngit, \emph{cuius participátio,} inquit,\emph{in idípsum.} Hoc iam tértio háuriunt \emph{aquas in gáudio de fóntibus Salvatóris,} et nudis, ut ita dicam, óculis deitátis intuéntur esséntiam, nulla corporeórum phantásmatum imaginatióne decépti. Ecce gáudium in fine, sed sine fine.


\end{document}