\documentclass[options]{article}
\begin{document}
	Ex Tractátibus Balduíni Cantuariénsis epíscopi 
	\begin{flushright}
		(Tract. 10: PL 204, 513-514, 516)
	\end{flushright}
Fortis est mors, quæ nos múnere vitæ valet destitúere. Fortis est diléctio, quæ ad usum melióris vitæ valet restitúere.\\ 
Fortis est mors, quæ nos exúviis huius córporis potens est spoliáre. Fortis est diléctio, quæ mortis spólia potens est dirípere nobísque resignáre.\\ 
Fortis est mors, cui non valet omnis homo resístere. Fortis est diléctio, ut ipsam váleat triumpháre, acúleum eius obtúndere, contentiónem sedáre, victóriam confúndere. Erit enim quando insultábitur, quando dicétur: 
\emph{Ubi est mors acúleus tuus? Ubi est mors conténtio tua?}\\
\emph{Fortis est ut mors diléctio,}
quia mors mortis est Christi diléctio. Propter quod dicit: 
\emph{Ero mors tua, o mors; morsus tuus ero, inférne.}
Diléctio quoque, qua a nobis dilígitur Christus, et ipsa fortis est ut mors, cum sit ipsa quasi quædam mors, útpote véteris vitæ exstínctio et vitiórum abolítio et mortuórum óperum deposítio. \\
\\
resp-fluctustuisuperme-CROCHU.gabc\\
\\
Hæc autem diléctio nostra ad Christum, quædam vicissitúdo est, quamvis impar dilectiónis ipsíus quoad nos, et coimagináta similitúdo. 
\emph{Ipse enim prior diléxit nos}
et per exémplum amóris quod nobis propósuit, factus est nobis signáculum, quo efficiámur confórmes imáginis ipsíus, imáginem terréni deponéntes et imáginem cæléstis portántes; sicut dilécti sumus, sic et eum diligéntes. In hoc enim 
\emph{nobis relínquit exémplum, ut sequámur vestígia eius.}\\
  Propter quod dicit: 
  \emph{Pone me ut signáculum super cor tuum.}
  Ac si díceret: "Ama me, sicut amo te. Habe me in mente tua, in memória tua, in desidério tuo, in suspírio tuo, in gémitu et singúltu tuo. Meménto, homo, qualem te fécerim, quantum creatúris céteris te prætúlerim, quali dignitáte te nobilitáverim, quómodo glória et honóre te coronáverim, quómodo paulo minus ab ángelis te minoráverim, quómodo sub pédibus tuis ómnia subiécerim. Meménto, non solum quanta tibi fécerim, sed quam dura, quam indígna pro te pertúlerim; et vide, si non iníque agis contra me, si non amas me. Quis enim te sic amat, ut ego? quis te creávit, nisi ego? quis te redémit, nisi ego?". 
  Aufer a me, Dómine, cor lapídeum, aufer cor coagulátum, aufer cor incircumcísum; da mihi cor novum, cor cárneum, cor mundum! Tu cordis mundátor et mundi cordis amátor, pósside cor meum et inhábita, cóntinens et implens, supérior summo meo et intérior íntimo meo! Tu forma pulchritúdinis et signáculum sanctitátis, signa cor meum in imágine tua; signa cor meum sub misericórdia tua,
  \emph{Deus cordis mei, et pars mea Deus in ætérnum.}
  Amen.\\
  \\
  
Resp 3  resp-confitebortibidomine-CROCHU-cumdox.gabc
  

\end{document}