\documentclass[options]{article}

\begin{document}
	
Ex Constitutióne Apostólica "Divíno afflátu" sancti Pii papæ Décimi 
	
	\textbf{(AAS 3 [1911], 633-635)}
	
	Divíno afflátu compósitos psalmos, quorum est in sacris lítteris colléctio, inde ab Ecclésiæ exórdiis non modo mirífice valuísse constat ad fovéndam fidélium pietátem, qui offerébant 
	\textit{hóstiam laudis semper Deo, id est, fructum labiórum confiténtium nómini eius;}
	verum étiam ex more iam in vétere Lege recépto in ipsa sacra Liturgía divinóque Offício conspícuam habuísse partem. Hinc illa, quam dicit Basilíus, nata  
	\textit{Ecclésiæ vox,}
	atque psalmódia, eius 
	\textit{hymnódiæ fília,}
	ut a decessóre nostro Urbáno octávo appellátur, 
	\textit{quæ cánitur assídue ante sedem Dei et Agni,}
	quæque hómines in primis divíno cúltui addíctos docet, ex Athanásii senténtia, 
	\textit{qua ratióne Deum laudáre opórteat quibúsque verbis decénter}
	confiteántur. Pulchre ad rem Augustínus: 
	\textit{Ut bene ab hómine laudétur Deus, laudávit se ipse Deus; et quia dignátus est laudáre se, ídeo invénit homo, quemádmodum laudet eum.}
	
	Accédit quod in psalmis mirábilis quædam vis inest ad excitánda in ánimis ómnium stúdia virtútum. 
	\textit{Etsi}
	enim 
	\textit{omnis nostra Scriptúra, cum vetus tum nova, divínitus inspiráta utilísque ad doctrínam est, ut scriptum habétur; at psalmórum liber, quasi paradísus ómnium reliquórum } 
	(librórum fructus) 
	\textit{in se cóntinens, cantus edit, et próprios ínsuper cum ipsis inter psalléndum éxhibet.}
	Hæc íterum Athanásius, qui recte ibídem addit: 
	\textit{Mihi quidem vidétur psallénti psalmos esse instar spéculi, ut et seípsum et próprii ánimi motus in ipsis contemplétur, atque ita afféctus eos récitet.}
	
	Itaque Augustínus in Confessiónibus:
	\textit{Quantum,}
	inquit,
	\textit{flevi in hymnis et cánticis tuis, suáve sonántis Ecclésiæ tuæ vócibus commótus ácriter! Voces illæ influébant áuribus meis et eliquabátur véritas in cor meum et exæstuábat inde afféctus pietátis et currébant lácrimæ et bene mihi erat cum eis.}
	
\end{document}