% LuaLaTeX

\documentclass[a4paper, twoside, 12pt]{article}
\usepackage[latin]{babel}
%\usepackage[landscape, left=3cm, right=1.5cm, top=2cm, bottom=1cm]{geometry} % okraje stranky
%\usepackage[landscape, a4paper, mag=1166, truedimen, left=2cm, right=1.5cm, top=1.6cm, bottom=0.95cm]{geometry} % okraje stranky
\usepackage[landscape, a4paper, mag=1400, truedimen, left=0.5cm, right=0.5cm, top=0.5cm, bottom=0.5cm]{geometry} % okraje stranky

\usepackage{fontspec}
\setmainfont[FeatureFile={junicode.fea}, Ligatures={Common, TeX}, RawFeature=+fixi]{Junicode}
%\setmainfont{Junicode}

% shortcut for Junicode without ligatures (for the Czech texts)
\newfontfamily\nlfont[FeatureFile={junicode.fea}, Ligatures={Common, TeX}, RawFeature=+fixi]{Junicode}

\usepackage{multicol}
\usepackage{color}
\usepackage{lettrine}
\usepackage{fancyhdr}

% usual packages loading:
\usepackage{luatextra}
\usepackage{graphicx} % support the \includegraphics command and options
\usepackage{gregoriotex} % for gregorio score inclusion
\usepackage{gregoriosyms}
\usepackage{wrapfig} % figures wrapped by the text
\usepackage{parcolumns}
\usepackage[contents={},opacity=1,scale=1,color=black]{background}
\usepackage{tikzpagenodes}
\usepackage{calc}
\usepackage{longtable}
\usetikzlibrary{calc}

\setlength{\headheight}{14.5pt}

% Commands used to produce a typical "Conventus" booklet

\newenvironment{titulusOfficii}{\begin{center}}{\end{center}}
\newcommand{\dies}[1]{#1

}
\newcommand{\nomenFesti}[1]{\textbf{\Large #1}

}
\newcommand{\celebratio}[1]{#1

}

\newcommand{\hora}[1]{%
\vspace{0.5cm}{\large \textbf{#1}}

\fancyhead[LE]{\thepage\ / #1}
\fancyhead[RO]{#1 / \thepage}
\addcontentsline{toc}{subsection}{#1}
}

% larger unit than a hora
\newcommand{\divisio}[1]{%
\begin{center}
{\Large \textsc{#1}}
\end{center}
\fancyhead[CO,CE]{#1}
\addcontentsline{toc}{section}{#1}
}

% a part of a hora, larger than pars
\newcommand{\subhora}[1]{
\begin{center}
{\large \textit{#1}}
\end{center}
%\fancyhead[CO,CE]{#1}
\addcontentsline{toc}{subsubsection}{#1}
}

% rubricated inline text
\newcommand{\rubricatum}[1]{\textit{#1}}

% standalone rubric
\newcommand{\rubrica}[1]{\vspace{3mm}\rubricatum{#1}}

\newcommand{\notitia}[1]{\textcolor{red}{#1}}

\newcommand{\scriptura}[1]{\hfill \small\textit{#1}}

\newcommand{\translatioCantus}[1]{\vspace{1mm}%
{\noindent\footnotesize \nlfont{#1}}}

% pruznejsi varianta nasledujiciho - umoznuje nastavit sirku sloupce
% s prekladem
\newcommand{\psalmusEtTranslatioB}[3]{
  \vspace{0.5cm}
  \begin{parcolumns}[colwidths={2=#3}, nofirstindent=true]{2}
    \colchunk{
      \input{#1}
    }

    \colchunk{
      \vspace{-0.5cm}
      {\footnotesize \nlfont
        \input{#2}
      }
    }
  \end{parcolumns}
}

\newcommand{\psalmusEtTranslatio}[2]{
  \psalmusEtTranslatioB{#1}{#2}{8.5cm}
}


\newcommand{\canticumMagnificatEtTranslatio}[1]{
  \psalmusEtTranslatioB{#1}{temporalia/extra-adventum-vespers/magnificat-boh.tex}{12cm}
}
\newcommand{\canticumBenedictusEtTranslatio}[1]{
  \psalmusEtTranslatioB{#1}{temporalia/extra-adventum-laudes/benedictus-boh.tex}{10.5cm}
}

% volne misto nad antifonami, kam si zpevaci dokresli neumy
\newcommand{\hicSuntNeumae}{\vspace{0.5cm}}

% prepinani mista mezi notovymi osnovami: pro neumovane a neneumovane zpevy
\newcommand{\cantusCumNeumis}{
  \setgrefactor{17}
  \global\advance\grespaceabovelines by 5mm%
}
\newcommand{\cantusSineNeumas}{
  \setgrefactor{17}
  \global\advance\grespaceabovelines by -5mm%
}

% znaky k umisteni nad inicialu zpevu
\newcommand{\superInitialam}[1]{\gresetfirstlineaboveinitial{\small {\textbf{#1}}}{\small {\textbf{#1}}}}

% pars officii, i.e. "oratio", ...
\newcommand{\pars}[1]{\textbf{#1}}

\newenvironment{psalmus}{
  \setlength{\parindent}{0pt}
  \setlength{\parskip}{5pt}
}{
  \setlength{\parindent}{10pt}
  \setlength{\parskip}{10pt}
}

%%%% Prejmenovat na latinske:
\newcommand{\nadpisZalmu}[1]{
  \hspace{2cm}\textbf{#1}\vspace{2mm}%
  \nopagebreak%

}

% mode, score, translation
\newcommand{\antiphona}[3]{%
\hicSuntNeumae
\superInitialam{#1}
\includescore{#2}

#3
}
 % Often used macros

\newcommand{\annusEditionis}{2022}

%%%% Vicekrat opakovane kousky

\newcommand{\anteOrationem}{
  \rubrica{Ante Orationem, cantatur a Superiore:}

  \pars{Supplicatio Litaniæ.}

  \cuminitiali{}{temporalia/supplicatiolitaniae.gtex}

  \pars{Oratio Dominica.}

  \cuminitiali{}{temporalia/oratiodominica.gtex}

  \rubrica{Deinde dicitur ab Hebdomadario:}

  \cuminitiali{}{temporalia/dominusvobiscum-solemnis.gtex}

  \rubrica{In choro monialium loco Dominus vobiscum dicitur:}

  \sineinitiali{temporalia/domineexaudi.gtex}
}

\setlength{\columnsep}{30pt} % prostor mezi sloupci

\newcommand{\titulus}{\nomenFesti{Feria V in Cœna Domini.}
\celebratio{Duplex 1. Classis.}}
\newcommand{\tempquad}{Tempore Passionis}
\newcommand{\lectioi}{\pars{Lectio I.} \scriptura{Lam. 1, 1-5}

\noindent Incipit Lamentátio Ieremíæ Prophétæ.

\noindent Aleph. Quómodo sedet sola cívitas plena pópulo: facta est quasi vídua dómina géntium: princeps provinciárum facta est sub tribúto.

\noindent Beth. Plorans plorávit in nocte, et lácrimæ eius in maxíllis eius: non est qui consolétur eam ex ómnibus caris eius: omnes amíci eius sprevérunt eam, et facti sunt ei inimíci.

\noindent Ghimel. Migrávit Iudas propter afflictiónem, et multitúdinem servitútis: habitávit inter gentes, nec invénit réquiem: omnes persecutóres eius apprehendérunt eam inter angústias.

\noindent Daleth. Viæ Sion lugent eo quod non sint qui véniant ad solemnitátem: omnes portæ eius destrúctæ: sacerdótes eius geméntes: vírgines eius squálidæ, et ipsa oppréssa amaritúdine.

\noindent He. Facti sunt hostes eius in cápite, inimíci eius locupletáti sunt: quia Dóminus locútus est super eam propter multitúdinem iniquitátum eius: párvuli eius ducti sunt in captivitátem, ante fáciem tribulántis.

\noindent \Vbardot{} Ierúsalem, Ierúsalem, convértere ad Dóminum Deum tuum.}
\newcommand{\responsoriumi}{\pars{Responsorium 1.} \scriptura{\Rbardot{} Mt. 26, 39 \Vbardot{} Mc. 14, 38; \textbf{H178}}

\vspace{-5mm}

\grechangedim{spacebeneathtext}{5mm}{scalable}

\responsorium{VIII}{temporalia/resp-inmonteoliveti.gtex}{}

\grechangedim{spacebeneathtext}{0mm}{scalable}}
\newcommand{\lectioii}{\pars{Lectio II.} \scriptura{Lam. 1, 6-9}

\noindent Vau. Et egréssus est a fília Sion omnis decor eius: facti sunt príncipes eius velut aríetes non inveniéntes páscua: et abiérunt absque fortitúdine ante fáciem subsequéntis.

\noindent Zain. Recordáta est Ierúsalem diérum afflictiónis suæ, et prævaricatiónis ómnium desiderabílium suórum, quæ habúerat a diébus antíquis, cum cáderet pópulus eius in manu hostíli, et non esset auxiliátor: vidérunt eam hostes, et derisérunt sábbata eius.

\noindent Heth. Peccátum peccávit Ierúsalem, proptérea instábilis facta est: omnes, qui glorificábant eam, sprevérunt illam, quia vidérunt ignomíniam eius: ipsa autem gemens convérsa est retrórsum.

\noindent Teth. Sordes eius in pédibus eius, nec recordáta est finis sui: depósita est veheménter, non habens consolatórem: vide, Dómine, afflictiónem meam, quóniam eréctus est inimícus.

\noindent \Vbardot{} Ierúsalem, Ierúsalem, convértere ad Dóminum Deum tuum.}
\newcommand{\responsoriumii}{\pars{Responsorium 2.} \scriptura{\Rbardot{} Mc. 14, 34 \Vbardot{} Is. 53, 4; \textbf{H178}}

\vspace{-5mm}

\grechangedim{spacebeneathtext}{5mm}{scalable}
\grechangedim{interwordspacetext}{0.18 cm plus 0.15 cm minus 0.05 cm}{scalable}%

\responsorium{VIII}{temporalia/resp-tristisest.gtex}{}

\grechangedim{spacebeneathtext}{0mm}{scalable}
\grechangedim{interwordspacetext}{0.22 cm plus 0.15 cm minus 0.05 cm}{scalable}%
}
\newcommand{\lectioiii}{\pars{Lectio III.} \scriptura{Lam. 1, 10-14}

\noindent Iod. Manum suam misit hostis ad ómnia desiderabília eius: quia vidit gentes ingréssas sanctuárium suum, de quibus præcéperas ne intrárent in ecclésiam tuam.

\noindent Caph. Omnis pópulus eius gemens, et quærens panem: dedérunt pretiósa quæque pro cibo ad refocillándam ánimam. Vide, Dómine, et consídera, quóniam facta sum vilis.

\noindent Lamed. O vos omnes, qui transítis per viam, atténdite, et vidéte, si est dolor sicut dolor meus: quóniam vindemiávit me, ut locútus est Dóminus in die iræ furóris sui.

\noindent Mem. De excélso misit ignem in óssibus meis, et erudívit me: expándit rete pédibus meis, convértit me retrórsum: pósuit me desolátam, tota die mæróre conféctam.

\noindent Nun. Vigilávit iugum iniquitátum meárum: in manu eius convolútæ sunt, et impósitæ collo meo: infirmáta est virtus mea: dedit me Dóminus in manu, de qua non pótero súrgere.

\noindent \Vbardot{} Ierúsalem, Ierúsalem, convértere ad Dóminum Deum tuum.}
\newcommand{\responsoriumiii}{\pars{Responsorium 3.} \scriptura{\Rbardot{} Is. 53, 2.11.4 \Vbardot{} ibid. 53, 4; \textbf{H178}}

\vspace{-5mm}

\grechangedim{spacebeneathtext}{5mm}{scalable}

\responsorium{V}{temporalia/resp-eccevidimus.gtex}{}

\grechangedim{spacebeneathtext}{0mm}{scalable}

\rubrica{Omittitur Versus \textnormal{Gloria Patri}, repetitur integrum Responsorium usque ad Versum.}}
\newcommand{\lectioiv}{\pars{Lectio IV.} \scriptura{In Psalmum 54 ad 1 versum}

\noindent Ex tractátu sancti Augustíni Epíscopi super Psalmos.

\noindent Exáudi, Deus, oratiónem meam, et ne despéxeris deprecatiónem meam: inténde mihi, et exáudi me. Satagéntis, sollíciti, in tribulatióne pósiti, verba sunt ista. Orat multa pátiens, de malo liberári desíderans. Súperest ut videámus in quo malo sit: et cum dícere cœ́perit, agnoscámus ibi nos esse: ut communicáta tribulatióne, coniungámus oratiónem. Contristátus sum, inquit, in exercitatióne mea, et conturbátus sum. Ubi contristátus? ubi conturbátus? In exercitatióne mea, inquit. Hómines malos, quos pátitur, commemorátus est: eandémque passiónem malórum hóminum exercitatiónem suam dixit. Ne putétis gratis esse malos in hoc mundo, et nihil boni de illis ágere Deum. Omnis malus aut ídeo vivit, ut corrigátur; aut ídeo vivit, ut per illum bonus exerceátur.}
\newcommand{\responsoriumiv}{\pars{Responsorium 4.} \scriptura{\Rbardot{} Cantor super Mt. 26, 48 \& Cantor \Vbardot{} Mt. 26, 24; \textbf{H179}}

\vspace{-5mm}

%\grechangedim{spacebeneathtext}{5mm}{scalable}
\grechangedim{interwordspacetext}{0.16 cm plus 0.15 cm minus 0.05 cm}{scalable}%

\responsorium{VIII}{temporalia/resp-amicusmeus.gtex}{}

%\grechangedim{spacebeneathtext}{0mm}{scalable}
\grechangedim{interwordspacetext}{0.22 cm plus 0.15 cm minus 0.05 cm}{scalable}%
}
\newcommand{\lectiov}{\pars{Lectio V.}

\noindent Utinam ergo qui nos modo exércent, convertántur, et nobíscum exerceántur: tamen quámdiu ita sunt ut exérceant, non eos odérimus: quia in eo quod malus est quis eórum, utrum usque in finem perseveratúrus sit, ignorámus. Et plerúmque cum tibi vidéris odísse inimícum, fratrem odísti, et nescis. Diábolus, et ángeli eius in Scriptúris sanctis manifestáti sunt nobis, quod ad ignem ætérnum sint destináti. Ipsórum tantum desperánda est corréctio, contra quos habémus occúltam luctam: ad quam luctam nos armat Apóstolus, dicens: Non est nobis colluctátio advérsus carnem et sánguinem: id est, non advérsus hómines, quos vidétis, sed advérsus príncipes, et potestátes, et rectóres mundi, tenebrárum harum. Ne forte cum dixísset, mundi, intellégeres dǽmones esse rectóres cæli et terræ. Mundi dixit, tenebrárum harum: mundi dixit, amatórum mundi: mundi dixit, impiórum et iniquórum: mundi dixit, de quo dicit Evangélium: Et mundus eum non cognóvit.}
\newcommand{\responsoriumv}{\pars{Responsorium 5.} \scriptura{\Rbardot{} Cantor \Vbardot{} ibidem; \textbf{H179}}

\vspace{-5mm}

\grechangedim{spacebeneathtext}{5mm}{scalable}

\responsorium{II}{temporalia/resp-judasmercator.gtex}{}

\grechangedim{spacebeneathtext}{0mm}{scalable}}
\newcommand{\lectiovi}{\pars{Lectio VI.}

\noindent Quóniam vidi iniquitátem, et contradictiónem in civitáte. Atténde glóriam crucis ipsíus. Iam in fronte regum crux illa fixa est, cui inimíci insultavérunt. Efféctus probávit virtútem: dómuit orbem non ferro, sed ligno. Lignum crucis contuméliis dignum visum est inimícis, et ante ipsum lignum stantes caput agitábant, et dicébant: Si Fílius Dei est, descéndat de cruce. Extendébat ille manus suas ad pópulum non credéntem, et contradicéntem. Si enim iustus est, qui ex fide vivit; iníquus est, qui non habet fidem. Quod ergo hic ait, iniquitátem: perfídiam intéllege. Vidébat ergo Dóminus in civitáte iniquitátem et contradictiónem, et extendébat manus suas ad pópulum non credéntem et contradicéntem: et tamen et ipsos exspéctans dicébat: Pater, ignósce illis, quia nésciunt quid fáciunt.}
\newcommand{\responsoriumvi}{\pars{Responsorium 6.} \scriptura{\Rbardot{} Cantor super Mt. 26, 24-26 \Vbardot{} ibid. 26, 23; \textbf{H179}}

\vspace{-5mm}

\grechangedim{spacebeneathtext}{5mm}{scalable}

\responsorium{VIII}{temporalia/resp-unusexdiscipulis.gtex}{}

\grechangedim{spacebeneathtext}{0mm}{scalable}

\rubrica{Omittitur Versus \textnormal{Gloria Patri}, repetitur integrum Responsorium usque ad Versum.}}
\newcommand{\lectiovii}{\pars{Lectio VII.} \scriptura{1 Cor. 11, 17-22}

\noindent De Epístola prima beáti Pauli Apóstoli ad Corínthios.

\noindent Hoc autem præcípio: non laudans quod non in mélius, sed in detérius convenítis. Primum quidem conveniéntibus vobis in Ecclésiam, áudio scissúras esse inter vos, et ex parte credo. Nam opórtet et hǽreses esse, ut et qui probáti sunt, manifésti fiant in vobis. Conveniéntibus ergo vobis in unum, iam non est Domínicam cenam manducáre. Unusquísque enim suam cenam præsúmit ad manducándum. Et álius quidem ésurit, álius autem ébrius est. Numquid domos non habétis ad manducándum et bibéndum? aut Ecclésiam Dei contémnitis, et confúnditis eos, qui non habent? Quid dicam vobis? Laudo vos? In hoc non laudo.}
\newcommand{\responsoriumvii}{\pars{Responsorium 7.} \scriptura{\Rbardot{} Ier. 11, 19 \Vbardot{} Ps. 40, 8.9; \textbf{H180}}

\vspace{-5mm}

\grechangedim{spacebeneathtext}{5mm}{scalable}

\responsorium{VII}{temporalia/resp-eramquasiagnus.gtex}{}

\grechangedim{spacebeneathtext}{0mm}{scalable}}
\newcommand{\lectioviii}{\pars{Lectio VIII.} \scriptura{1 Cor. 11, 23-26}

\noindent Ego enim accépi a Dómino quod et trádidi vobis, quóniam Dóminus Iesus, in qua nocte tradebátur, accépit panem, et grátias agens fregit, et dixit: «\textsc{Accípite, et manducáte: hoc est corpus meum, quod pro vobis tradétur: hoc fácite in meam commemoratiónem}». Simíliter et cálicem, postquam cœnávit, dicens: «\textsc{Hic calix novum testaméntum est in meo sánguine: hoc fácite, quotiescúmque bibétis, in meam commemoratiónem}». Quotiescúmque enim manducábitis panem hunc, et cálicem bibétis, mortem Dómini annuntiábitis donec véniat.}
\newcommand{\responsoriumviii}{\pars{Responsorium 8.} \scriptura{\Rbardot{} Mt. 26, 40 \& Cantor \Vbardot{} Lc. 22, 46; \textbf{H180}}

\vspace{-5mm}

%\grechangedim{spacebeneathtext}{5mm}{scalable}
\grechangedim{interwordspacetext}{0.16 cm plus 0.15 cm minus 0.05 cm}{scalable}%

\responsorium{VII}{temporalia/resp-unahora.gtex}{}

%\grechangedim{spacebeneathtext}{0mm}{scalable}
\grechangedim{interwordspacetext}{0.22 cm plus 0.15 cm minus 0.05 cm}{scalable}%
}
\newcommand{\lectioix}{\pars{Lectio IX.} \scriptura{1 Cor. 11, 27-34}

\noindent Itaque quicúmque manducáverit panem hunc, vel bíberit cálicem Dómini indígne, reus erit córporis et sánguinis Dómini. Probet autem seípsum homo: et sic de pane illo edat, et de cálice bibat. Qui enim mandúcat et bibit indígne, iudícium sibi mandúcat et bibit, non diiúdicans corpus Dómini. Ideo inter vos multi infírmi et imbecílles, et dórmiunt multi. Quod, si nosmetípsos diiudicarémus, non útique iudicarémur. Dum iudicámur autem, a Dómino corrípimur, ut non cum hoc mundo damnémur. Itaque, fratres mei, cum convenítis ad manducándum, ínvicem exspectáte. Si quis ésurit, domi mandúcet: ut non in iudícium conveniátis. Cétera autem, cum vénero, dispónam.}
\newcommand{\responsoriumix}{\pars{Responsorium 9.} \scriptura{\Rbardot{} Mt. 26, 3.4.55 \Vbardot{} Io. 11, 47; \textbf{H180}}

\vspace{-5mm}

\grechangedim{spacebeneathtext}{5mm}{scalable}

\responsorium{I}{temporalia/resp-seniorespopuli.gtex}{}

\grechangedim{spacebeneathtext}{0mm}{scalable}

\rubrica{Omittitur Versus \textnormal{Gloria Patri}, repetitur integrum Responsorium usque ad Versum.}}
\newcommand{\hebdomada}{infra Hebdom. Passionis.}
\newcommand{\oratioLaudes}{\cuminitiali{}{temporalia/oratio.gtex}}
\newcommand{\hiemalis}{Hiemalis.}
\newcommand{\tempuspassionis}{Tempore Passionis}

%%%%%%%%%%%%%%%%%%%%%%%%%%%%%%%%%%%%%%%%%%%%%%%%%%%%%%%%%%%%%%%%%%%%%%%%%%%%%%%%%%%%%%%%%%%%%%%%%%%%%%%%%%%%%
\begin{document}

% Here we set the space around the initial.
% Please report to http://home.gna.org/gregorio/gregoriotex/details for more details and options
\grechangedim{afterinitialshift}{2.2mm}{scalable}
\grechangedim{beforeinitialshift}{2.2mm}{scalable}
\grechangedim{interwordspacetext}{0.22 cm plus 0.15 cm minus 0.05 cm}{scalable}%
\grechangedim{annotationraise}{-0.2cm}{scalable}

% Here we set the initial font. Change 38 if you want a bigger initial.
% Emit the initials in red.
\grechangestyle{initial}{\color{red}\fontsize{38}{38}\selectfont}

\pagestyle{empty}

%%%% Titulni stranka
\begin{titulusOfficii}
\titulus{}
\end{titulusOfficii}

\scriptura{}

\pars{}

\vfill

\begin{center}
%Ad usum et secundum consuetudines chori \guillemotright{}Conventus Choralis\guillemotleft.

%Editio Sancti Wolfgangi \annusEditionis
\end{center}

\pagebreak

\renewcommand{\headrulewidth}{0pt} % no horiz. rule at the header
\fancyhf{}
\pagestyle{fancy}

\pars{Oratio ante divinum Officium.}

\lettrine{{\color{red}A}}{peri,} Dómine, os meum ad benedicéndum nomen sanctum tuum:
munda quoque cor meum ab ómnibus vanis, pervérsis, et aliénis
cogitatiónibus:
intelléctum illúmina, afféctum inflámma,
ut digne, atténte ac devóte hoc Offícium recitáre váleam,
et exaudíri mérear ante conspéctum Divínæ Maiestátis tuæ.
Per Christum, Dóminum nostrum.
\Rbardot{} Amen.

Dómine, in unióne illíus divínæ intentiónis,
qua ipse in terris laudes Deo persolvísti,
has tibi Horas \rubricatum{(vel \textnormal{hanc tibi Horam})} persólvo.

\vfill

\pars{Oratio post divinum Officium.} \scriptura{}

\rubrica{
  Orationem sequentem devote post Officium recitantibus
  Leo Papa X. defectus, et culpas in eo persolvendo ex humana
  fragilitate contractas, indulsit, et dicitur flexis genibus.
}

\lettrine{{\color{red}S}}{acrosánctæ} et indivíduæ Trinitáti,
crucifíxi Dómini nostri Iesu Christi humanitáti,
beatíssimæ et gloriosíssimæ sempérque Vírginis Maríæ
fecúndæ integritáti, 
et ómnium Sanctórum universitáti
sit sempitérna laus, honor, virtus et glória
ab omni creatúra,
nobísque remíssio ómnium peccatórum,
per infiníta sǽcula sæculórum.
\Rbardot{} Amen.

\noindent \Vbardot{} Beáta víscera Maríæ Virginis, quæ portavérunt
ætérni Patris Fílium.\\
\Rbardot{} Et beáta úbera, quæ lactavérunt Christum Dominum.

\rubrica{Et dicitur secreto \textnormal{Pater noster.} et \textnormal{Ave María.}}

\cantusSineNeumas

\vfill
\pagebreak

%\vspace{-5mm}

\hora{Ad Matutinum.} %%%%%%%%%%%%%%%%%%%%%%%%%%%%%%%%%%%%%%%%%%%%%%%%%%%%%

\rubrica{Absolute incipitur.}

\vspace{2mm}

\subhora{In I. Nocturno}

\pars{Psalmus 1.} \scriptura{Ps. 68, 10; \textbf{H178}}

\vspace{-4mm}

\antiphona{VIII c}{temporalia/ant-zelusdomus-cgp.gtex}

\scriptura{Ps. 68}

\vspace{-2mm}

\initiumpsalmi{temporalia/ps68-initium-viii-c-auto.gtex}

\input{temporalia/ps68-viii-c-sinedox.tex}

\rubrica{Hic non dicitur Gloria Patri, neque Amen.}

\vfill

\antiphona{}{temporalia/ant-zelusdomus-cgp.gtex}

\vfill
\pagebreak

\pars{Psalmus 2.} \scriptura{Ps. 34, 4; \textbf{H178}}

\vspace{-4mm}

\antiphona{VIII c}{temporalia/ant-avertanturretrorsum-cgp.gtex}

\vspace{-1mm}

\scriptura{Ps. 69}

\vspace{-2mm}

\initiumpsalmi{temporalia/ps69-initium-viii-c-auto.gtex}

\input{temporalia/ps69-viii-c-sinedox.tex}

\noindent \Abardot{}

\vfill
\pagebreak

\pars{Psalmus 3.} \scriptura{Ps. 70, 4; \textbf{H178}}

\vspace{-4mm}

\antiphona{VIII c}{temporalia/ant-deusmeuseripeme-cgp.gtex}

%\vspace{-5mm}

\scriptura{Ps. 70}

\initiumpsalmi{temporalia/ps70-initium-viii-c-auto.gtex}

\input{temporalia/ps70-viii-c-sinedox.tex}

\vfill

\antiphona{}{temporalia/ant-deusmeuseripeme-cgp.gtex}

\vfill
\pagebreak

\pars{Versus.} \scriptura{Ps. 69, 4}

% Versus. %%%
\sineinitiali{temporalia/versus-avertantur.gtex}

\vspace{5mm}

\sineinitiali{temporalia/oratiodominica-mat.gtex}

\vfill
\pagebreak

\lectioi

\vfill
\pagebreak

\responsoriumi

\vfill
\pagebreak

\lectioii

\vfill
\pagebreak

\responsoriumii

\vfill
\pagebreak

\lectioiii

\vfill
\pagebreak

\responsoriumiii

\vfill
\pagebreak

\subhora{In II. Nocturno}

\pars{Psalmus 4.} \scriptura{Ps. 71, 12; \textbf{H179}}

\vspace{-4mm}

\antiphona{VII c trans.}{temporalia/ant-liberavitdominus-cgp.gtex}

%\vspace{-2mm}

\scriptura{Ps. 71}

%A\vspace{-2mm}

\initiumpsalmi{temporalia/ps71-initium-vii-c-trans.gtex}

\input{temporalia/ps71-vii-c-sinedox.tex}

\vfill

\antiphona{}{temporalia/ant-liberavitdominus-cgp.gtex}

\vfill
\pagebreak

\pars{Psalmus 5.} \scriptura{Ps. 72, 8; \textbf{H179}}

\vspace{-4mm}

\antiphona{VIII c}{temporalia/ant-cogitaveruntimpii-cgp.gtex}

%\vspace{-2mm}

\scriptura{Ps. 72}

\initiumpsalmi{temporalia/ps72-initium-viii-c-auto.gtex}

\input{temporalia/ps72-viii-c-sinedox.tex}

\vfill

\antiphona{}{temporalia/ant-cogitaveruntimpii-cgp.gtex}

\vfill
\pagebreak

\pars{Psalmus 6.} \scriptura{Ps. 73, 22; \textbf{H179}}

\vspace{-4mm}

\antiphona{I g}{temporalia/ant-exsurgedomineetjudica-cgp.gtex}

%\vspace{-5mm}

\scriptura{Ps. 73}

\initiumpsalmi{temporalia/ps73-initium-i-g-auto.gtex}

\input{temporalia/ps73-i-g-sinedox.tex}

\vfill

\antiphona{}{temporalia/ant-exsurgedomineetjudica-cgp.gtex}

\vfill
\pagebreak

\pars{Versus.} \scriptura{Ps. 70, 4}

% Versus. %%%
\sineinitiali{temporalia/versus-deusmeus.gtex}

\vspace{5mm}

\sineinitiali{temporalia/oratiodominica-mat.gtex}

\vfill
\pagebreak

\lectioiv

\vfill
\pagebreak

\responsoriumiv

\vfill
\pagebreak

\lectiov

\vfill
\pagebreak

\responsoriumv

\vfill
\pagebreak

\lectiovi

\vfill
\pagebreak

\responsoriumvi

\vfill
\pagebreak

\subhora{In III. Nocturno}

\pars{Psalmus 7.} \scriptura{Ps. 74, 5; \textbf{H180}}

\vspace{-4mm}

\antiphona{VII c}{temporalia/ant-dixiiniquis-cgp.gtex}

\scriptura{Ps. 74}

%\vspace{-2mm}

\initiumpsalmi{temporalia/ps74-initium-vii-c-auto.gtex}

\input{temporalia/ps74-vii-c-sinedox.tex}

\noindent \Abardot{}

\vfill
\pagebreak

\pars{Psalmus 8.}\scriptura{Ps. 75, 9-10; \textbf{H180}}

\vspace{-4mm}

\antiphona{VIII c}{temporalia/ant-terratremuit-cgp.gtex}

%\vspace{-4mm}

\scriptura{Ps. 75}

\initiumpsalmi{temporalia/ps75-initium-viii-c-auto.gtex}

\input{temporalia/ps75-viii-c-sinedox.tex}

\noindent \Abardot{}

\vfill
\pagebreak

\pars{Psalmus 9.} \scriptura{Ps. 76, 3; \textbf{H180}}

\vspace{-4mm}

\antiphona{VII a}{temporalia/ant-indietribulationis-cgp.gtex}

%\vspace{-4mm}

\scriptura{Ps. 76}

\initiumpsalmi{temporalia/ps76-initium-vii-a-auto.gtex}

\input{temporalia/ps76-vii-a-sinedox.tex}

\vfill

\antiphona{}{temporalia/ant-indietribulationis-cgp.gtex}

\vfill
\pagebreak

\pars{Versus.} \scriptura{Ps. 73, 22}

% Versus. %%%
\sineinitiali{temporalia/versus-exsurge.gtex}

\vspace{5mm}

\sineinitiali{temporalia/oratiodominica-mat.gtex}

\vfill
\pagebreak

\lectiovii

\vfill
\pagebreak

\responsoriumvii

\vfill
\pagebreak

\lectioviii

\vfill
\pagebreak

\responsoriumviii

\vfill
\pagebreak

\lectioix

\vfill
\pagebreak

\responsoriumix

\vfill
\pagebreak

\scriptura{Phil. 2, 8; \textbf{C96}}

\antiphona{V}{temporalia/gr-christusfactus-feriav.gtex}

\vspace{5mm}

\pars{Oratio}

\noindent Réspice, quǽsumus, Dómine, super hanc famíliam tuam, pro qua Dóminus noster Iesus Christus non dubitávit mánibus tradi nocéntium, et crucis subíre torméntum:

\rubrica{Et sub silentio concluditur.}

\noindent Qui tecum vivit et regnat in unitáte Spíritus Santi, Deus, per ómnia sǽcula sæculórum.

\noindent \Rbardot{} Amen.

\vfill
\pagebreak

\hora{Ad Laudes.} %%%%%%%%%%%%%%%%%%%%%%%%%%%%%%%%%%%%%%%%%%%%%%%%%%%%%

\rubrica{Absolute incipitur.}

\pars{Psalmus 1.} \scriptura{Ps. 50, 6; \textbf{H181}}

\vspace{-4mm}

\antiphona{VIII G}{temporalia/ant-justificerisdomine-cgp.gtex}

\scriptura{Psalmus 50.}

\initiumpsalmi{temporalia/ps50-initium-viii-G-auto.gtex}

\input{temporalia/ps50-viii-G-sinedox.tex}

\rubrica{Hic non dicitur Gloria Patri, neque Amen.}

\antiphona{}{temporalia/ant-justificerisdomine-cgp.gtex}

\vfill
\pagebreak

\pars{Psalmus 2.} \scriptura{Is. 53, 7; Ac. 8, 32; \textbf{H181}}

\vspace{-4mm}

\antiphona{II D}{temporalia/ant-dominustamquamovis-cgp.gtex}

\scriptura{Psalmus 89.}

\initiumpsalmi{temporalia/ps89-initium-ii-D-auto.gtex}

\input{temporalia/ps89-ii-D-sinedox.tex}

\vfill

\antiphona{}{temporalia/ant-dominustamquamovis-cgp.gtex}

\vfill
\pagebreak

\pars{Psalmus 3.} \scriptura{Ier. 23, 9; \textbf{H181}}

\vspace{-4mm}

\antiphona{VIII G}{temporalia/ant-contritumest-cgp.gtex}

\scriptura{Psalmus 35.}

\initiumpsalmi{temporalia/ps35-initium-viii-G-auto.gtex}

\input{temporalia/ps35-viii-G-sinedox.tex}

\noindent \Abardot{}

\vfill
\pagebreak

\pars{Psalmus 4.} \scriptura{Ps. 69, 4; \textbf{H181}}

\vspace{-4mm}

\antiphona{\textit{IV A*}}{temporalia/ant-exhortatuses-cgp.gtex}

\scriptura{Canticum Moysis, Ex. 15, 1-19}

\initiumpsalmi{temporalia/moysis-initium-iv-A_-auto.gtex}

\input{temporalia/moysis-iv-A_-sinedox.tex}

\vfill

\antiphona{}{temporalia/ant-exhortatuses-cgp.gtex}

\vfill
\pagebreak

\pars{Psalmus 5.} \scriptura{Is. 53, 7.11; \textbf{H181}}

\vspace{-4mm}

\antiphona{II D}{temporalia/ant-oblatusest-cgp.gtex}

\scriptura{Psalmus 146.}

\initiumpsalmi{temporalia/ps146-initium-ii-D-auto.gtex}

\input{temporalia/ps148-ii-D-sinedox.tex}

\noindent \Abardot{}

\vfill
\pagebreak

\pars{Versus.} \scriptura{Ps. 40, 10}

\sineinitiali{temporalia/versus-homopacis.gtex}

\vfill
\pagebreak

\pars{Canticum Zachariæ.} \scriptura{Mt. 26, 48; Mc. 14, 44; \textbf{H181}}

\vspace{-4mm}

{
\grechangedim{interwordspacetext}{0.18 cm plus 0.15 cm minus 0.05 cm}{scalable}%
\antiphona{I g}{temporalia/ant-traditorautem-cgp.gtex}
\grechangedim{interwordspacetext}{0.22 cm plus 0.15 cm minus 0.05 cm}{scalable}%
}

\vspace{-2mm}

\scriptura{Lc. 1, 68-79}

%\vspace{-2mm}

\initiumpsalmi{temporalia/benedictus-initium-isoll-g-auto.gtex}

%\vspace{-1.5mm}

\input{temporalia/benedictus-isoll-g-sinedox.tex}

\noindent \Abardot{}

\vspace{-2mm}

\vfill
\pagebreak

\scriptura{Phil. 2, 8; \textbf{C96}}

\antiphona{V}{temporalia/gr-christusfactus-feriav.gtex}

\vspace{5mm}

\pars{Oratio}

\noindent Réspice, quǽsumus, Dómine, super hanc famíliam tuam, pro qua Dóminus noster Iesus Christus non dubitávit mánibus tradi nocéntium, et crucis subíre torméntum:

\rubrica{Et sub silentio concluditur.}

\noindent Qui tecum vivit et regnat in unitáte Spíritus Santi, Deus, per ómnia sǽcula sæculórum.

\noindent \Rbardot{} Amen.

\vfill
\pagebreak

\end{document}
