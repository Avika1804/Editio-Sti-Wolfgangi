% LuaLaTeX

\documentclass[a4paper, twoside, 12pt]{article}
\usepackage[latin]{babel}
%\usepackage[landscape, left=3cm, right=1.5cm, top=2cm, bottom=1cm]{geometry} % okraje stranky
%\usepackage[landscape, a4paper, mag=1166, truedimen, left=2cm, right=1.5cm, top=1.6cm, bottom=0.95cm]{geometry} % okraje stranky
\usepackage[landscape, a4paper, mag=1400, truedimen, left=0.5cm, right=0.5cm, top=0.5cm, bottom=0.5cm]{geometry} % okraje stranky

\usepackage{fontspec}
\setmainfont[FeatureFile={junicode.fea}, Ligatures={Common, TeX}, RawFeature=+fixi]{Junicode}
%\setmainfont{Junicode}

% shortcut for Junicode without ligatures (for the Czech texts)
\newfontfamily\nlfont[FeatureFile={junicode.fea}, Ligatures={Common, TeX}, RawFeature=+fixi]{Junicode}

\usepackage{multicol}
\usepackage{color}
\usepackage{lettrine}
\usepackage{fancyhdr}

% usual packages loading:
\usepackage{luatextra}
\usepackage{graphicx} % support the \includegraphics command and options
\usepackage{gregoriotex} % for gregorio score inclusion
\usepackage{gregoriosyms}
\usepackage{wrapfig} % figures wrapped by the text
\usepackage{parcolumns}
\usepackage[contents={},opacity=1,scale=1,color=black]{background}
\usepackage{tikzpagenodes}
\usepackage{calc}
\usepackage{longtable}
\usetikzlibrary{calc}

\setlength{\headheight}{14.5pt}

% Commands used to produce a typical "Conventus" booklet

\newenvironment{titulusOfficii}{\begin{center}}{\end{center}}
\newcommand{\dies}[1]{#1

}
\newcommand{\nomenFesti}[1]{\textbf{\Large #1}

}
\newcommand{\celebratio}[1]{#1

}

\newcommand{\hora}[1]{%
\vspace{0.5cm}{\large \textbf{#1}}

\fancyhead[LE]{\thepage\ / #1}
\fancyhead[RO]{#1 / \thepage}
\addcontentsline{toc}{subsection}{#1}
}

% larger unit than a hora
\newcommand{\divisio}[1]{%
\begin{center}
{\Large \textsc{#1}}
\end{center}
\fancyhead[CO,CE]{#1}
\addcontentsline{toc}{section}{#1}
}

% a part of a hora, larger than pars
\newcommand{\subhora}[1]{
\begin{center}
{\large \textit{#1}}
\end{center}
%\fancyhead[CO,CE]{#1}
\addcontentsline{toc}{subsubsection}{#1}
}

% rubricated inline text
\newcommand{\rubricatum}[1]{\textit{#1}}

% standalone rubric
\newcommand{\rubrica}[1]{\vspace{3mm}\rubricatum{#1}}

\newcommand{\notitia}[1]{\textcolor{red}{#1}}

\newcommand{\scriptura}[1]{\hfill \small\textit{#1}}

\newcommand{\translatioCantus}[1]{\vspace{1mm}%
{\noindent\footnotesize \nlfont{#1}}}

% pruznejsi varianta nasledujiciho - umoznuje nastavit sirku sloupce
% s prekladem
\newcommand{\psalmusEtTranslatioB}[3]{
  \vspace{0.5cm}
  \begin{parcolumns}[colwidths={2=#3}, nofirstindent=true]{2}
    \colchunk{
      \input{#1}
    }

    \colchunk{
      \vspace{-0.5cm}
      {\footnotesize \nlfont
        \input{#2}
      }
    }
  \end{parcolumns}
}

\newcommand{\psalmusEtTranslatio}[2]{
  \psalmusEtTranslatioB{#1}{#2}{8.5cm}
}


\newcommand{\canticumMagnificatEtTranslatio}[1]{
  \psalmusEtTranslatioB{#1}{temporalia/extra-adventum-vespers/magnificat-boh.tex}{12cm}
}
\newcommand{\canticumBenedictusEtTranslatio}[1]{
  \psalmusEtTranslatioB{#1}{temporalia/extra-adventum-laudes/benedictus-boh.tex}{10.5cm}
}

% volne misto nad antifonami, kam si zpevaci dokresli neumy
\newcommand{\hicSuntNeumae}{\vspace{0.5cm}}

% prepinani mista mezi notovymi osnovami: pro neumovane a neneumovane zpevy
\newcommand{\cantusCumNeumis}{
  \setgrefactor{17}
  \global\advance\grespaceabovelines by 5mm%
}
\newcommand{\cantusSineNeumas}{
  \setgrefactor{17}
  \global\advance\grespaceabovelines by -5mm%
}

% znaky k umisteni nad inicialu zpevu
\newcommand{\superInitialam}[1]{\gresetfirstlineaboveinitial{\small {\textbf{#1}}}{\small {\textbf{#1}}}}

% pars officii, i.e. "oratio", ...
\newcommand{\pars}[1]{\textbf{#1}}

\newenvironment{psalmus}{
  \setlength{\parindent}{0pt}
  \setlength{\parskip}{5pt}
}{
  \setlength{\parindent}{10pt}
  \setlength{\parskip}{10pt}
}

%%%% Prejmenovat na latinske:
\newcommand{\nadpisZalmu}[1]{
  \hspace{2cm}\textbf{#1}\vspace{2mm}%
  \nopagebreak%

}

% mode, score, translation
\newcommand{\antiphona}[3]{%
\hicSuntNeumae
\superInitialam{#1}
\includescore{#2}

#3
}
 % Often used macros

\newcommand{\annusEditionis}{2021}

%%%% Vicekrat opakovane kousky

\newcommand{\anteOrationem}{
  \rubrica{Ante Orationem, cantatur a Superiore:}

  \pars{Supplicatio Litaniæ.}

  \cuminitiali{}{temporalia/supplicatiolitaniae.gtex}

  \pars{Oratio Dominica.}

  \cuminitiali{}{temporalia/oratiodominica.gtex}

  \rubrica{Deinde dicitur ab Hebdomadario:}

  \cuminitiali{}{temporalia/dominusvobiscum-solemnis.gtex}

  \rubrica{In choro monialium loco Dominus vobiscum dicitur:}

  \sineinitiali{temporalia/domineexaudi.gtex}
}

\setlength{\columnsep}{30pt} % prostor mezi sloupci

\newcommand{\titulus}{\nomenFesti{Feria II Maioris Hebdomadæ.}
\celebratio{1. Classis. Duplex}}
\newcommand{\tempquad}{Tempore Passionis}
\newcommand{\invitatorium}{\pars{Invitatorium.}

\vspace{-4mm}

\antiphona{IV*}{temporalia/inv-christumdominum.gtex}}
\newcommand{\hymnusmatutinum}{\pars{Hymnus.} \scriptura{Venantius Fortunatus (sæc. VI)}

\vspace{-5mm}

\antiphona{I}{temporalia/hym-PangeLingua.gtex}}
\newcommand{\matversus}{\pars{Versus.}

\noindent \Vbardot{} Erue a frámea, Deus, ánimam meam.

\noindent \Rbardot{} Et de manu canis únicam meam.}
\newcommand{\lectioi}{\pars{Lectio I.} \scriptura{Io. 12, 1-9}

\noindent Léctio sancti Evangélii secúndum Ioánnem.

\noindent Ante sex dies Paschæ venit Iesus Bethániam, ubi Lázarus fúerat mórtuus, quem suscitávit Iesus. Et réliqua.

\scriptura{Tractus 50 in Ioann., post initium}

\noindent Homilía sancti Augustíni Epíscopi.

\noindent Ne putárent hómines phantásma esse factum, quia mórtuus resurréxit, Lázarus unus erat ex recumbéntibus: vivébat, loquebátur, epulabátur, véritas ostendebátur, infidélitas Iudæórum confundebátur. Discumbébat ergo Iesus cum Lázaro, et céteris: ministrábat Martha, una ex soróribus Lázari. María vero, áltera soror Lázari, accépit libram unguénti nardi pístici pretiósi, et unxit pedes Iesu, et extérsit capíllis suis pedes eius, et domus impléta est ex odóre unguénti. Factum audívimus: mystérium requirámus.}
\newcommand{\responsoriumi}{\pars{Responsorium 1.} \scriptura{\Rbardot{} Sap. 2, 1.10 \Vbardot{} Sap. 2, 21; \textbf{H173}}

\vspace{-5mm}

\grechangedim{spacebeneathtext}{4mm}{scalable}

\responsorium{I}{temporalia/resp-viriimpiidixerunt.gtex}{}

\grechangedim{spacebeneathtext}{0mm}{scalable}}
\newcommand{\lectioii}{\pars{Lectio II.}

\noindent Quæcúmque ánima fidélis vis esse, cum María unge pedes Dómini pretióso unguénto. Unguéntum illud iustítia fuit, ídeo libra fuit: erat autem unguéntum nardi pístici pretiósi. Quod ait, pístici, locum áliquem crédere debémus, unde hoc erat unguéntum pretiósum: nec tamen hoc vacat, et sacraménto óptime cónsonat. Pistis Græce, fides Latíne dícitur. Quærébas operári iustítiam. Iustus ex fide vivit. Unge pedes Iesu bene vivéndo: Domínica sectáre vestígia. Capíllis terge: si habes supérflua, da paupéribus, et Dómini pedes tersísti: capílli enim supérflua córporis vidéntur. Habes quod agas de supérfluis tuis: tibi supérflua sunt, sed Dómini pédibus necessária sunt. Forte in terra Dómini pedes índigent.}
\newcommand{\responsoriumii}{\pars{Responsorium 2.} \scriptura{\Rbardot{} Ps. 30, 12 \Vbardot{} Ps. 108, 3; \textbf{H172}}

\vspace{-5mm}

\grechangedim{spacebeneathtext}{5mm}{scalable}

\responsorium{II}{temporalia/resp-opprobriumfactussum.gtex}{}

\grechangedim{spacebeneathtext}{0mm}{scalable}}
\newcommand{\lectioiii}{\pars{Lectio III.}

\noindent De quibus enim, nisi de membris suis in fine dictúrus est: Cum uni ex mínimis meis fecístis, mihi fecístis? Supérflua vestra impendístis: sed pédibus meis obsecúti estis. Domus autem impléta est odóre: mundus implétus est fama bona: nam odor bonus, fama bona est. Qui male vivunt, et Christiáni vocántur, iniúriam Christo fáciunt: de quálibus dictum est, quod per eos nomen Dómini blasphemátur. Si per tales nomen Dei blasphemátur, per bonos nomen Dómini laudátur. Audi Apóstolum: Christi bonus odor sumus, inquit, in omni loco.}
\newcommand{\responsoriumiii}{\pars{Responsorium 3.} \scriptura{\Rbardot{} Ps. 85, 14 \Vbardot{} Cantor; \textbf{H173}}

\vspace{-5mm}

\grechangedim{spacebeneathtext}{5mm}{scalable}

\responsorium{III}{temporalia/resp-insurrexeruntinme.gtex}{}

\grechangedim{spacebeneathtext}{0mm}{scalable}

\rubrica{Omittitur Versus \textnormal{Gloria Patri}, repetitur integrum Responsorium usque ad Versum.}}
\newcommand{\hebdomada}{Maioris Hebdomadæ.}
\newcommand{\oratioLaudes}{\cuminitiali{}{temporalia/oratio2.gtex}}
\newcommand{\hiemalis}{Hiemalis.}
\newcommand{\tempuspassionis}{Tempore Passionis}

%%%%%%%%%%%%%%%%%%%%%%%%%%%%%%%%%%%%%%%%%%%%%%%%%%%%%%%%%%%%%%%%%%%%%%%%%%%%%%%%%%%%%%%%%%%%%%%%%%%%%%%%%%%%%
\begin{document}

% Here we set the space around the initial.
% Please report to http://home.gna.org/gregorio/gregoriotex/details for more details and options
\grechangedim{afterinitialshift}{2.2mm}{scalable}
\grechangedim{beforeinitialshift}{2.2mm}{scalable}
\grechangedim{interwordspacetext}{0.22 cm plus 0.15 cm minus 0.05 cm}{scalable}%
\grechangedim{annotationraise}{-0.2cm}{scalable}

% Here we set the initial font. Change 38 if you want a bigger initial.
% Emit the initials in red.
\grechangestyle{initial}{\color{red}\fontsize{38}{38}\selectfont}

\pagestyle{empty}

%%%% Titulni stranka
\begin{titulusOfficii}
\titulus{}
\end{titulusOfficii}

\scriptura{}

\pars{}

\vfill

\begin{center}
%Ad usum et secundum consuetudines chori \guillemotright{}Conventus Choralis\guillemotleft.

%Editio Sancti Wolfgangi \annusEditionis
\end{center}

\pagebreak

\renewcommand{\headrulewidth}{0pt} % no horiz. rule at the header
\fancyhf{}
\pagestyle{fancy}

\cantusSineNeumas

\hora{Ad Matutinum.} %%%%%%%%%%%%%%%%%%%%%%%%%%%%%%%%%%%%%%%%%%%%%%%%%%%%%

\vspace{2mm}

\cuminitiali{}{temporalia/dominelabiamea.gtex}

%\vspace{2mm}
\vfill
%\pagebreak

\ifx\invitatorium\undefined
\pars{Invitatorium.} \scriptura{Ps. 94, 1; Psalmus 94}

\vspace{-6mm}

\antiphona{E}{temporalia/inv-veniteexsultemus.gtex}
\else
\invitatorium
\fi

\vfill
\pagebreak

\ifx\hymnusmatutinum\undefined
\pars{Hymnus.} \scriptura{Adamus Sancti Victoris (\olddag 1146)}

\vspace{-5mm}

\antiphona{VII}{temporalia/hym-SalveDies.gtex}

\scriptura{Non dicitur \textnormal{Amen} in fine.}
\else
\hymnusmatutinum
\fi

\vfill
\pagebreak

\pars{Psalmus 1.} \scriptura{Ps. 13, 2}

\vspace{-4mm}

\antiphona{II D}{temporalia/ant-dominusdecaelo-cgp.gtex}

%\vspace{-2mm}

\scriptura{Ps. 13}

%\vspace{-2mm}

\initiumpsalmi{temporalia/ps13-initium-ii-D-auto.gtex}

\input{temporalia/ps13-ii-D.tex} \Abardot{}

\vfill
\pagebreak

\pars{Psalmus 2.} \scriptura{Ps. 14, 2.1}

\vspace{-4mm}

\antiphona{IV E}{temporalia/ant-quioperatur-cgp.gtex}

%\vspace{-2mm}

\scriptura{Ps. 14}

\initiumpsalmi{temporalia/ps14-initium-iv-E-auto.gtex}

\input{temporalia/ps14-iv-E.tex} \Abardot{}

\vfill
\pagebreak

\pars{Psalmus 3.} \scriptura{Ps. 16, 6; \textbf{H99}}

\vspace{-4mm}

\antiphona{VII c}{temporalia/ant-inclinadomine-cgp.gtex}

%\vspace{-5mm}

\scriptura{Ps. 16}

\vspace{-1mm}

\initiumpsalmi{temporalia/ps16-initium-vii-c-auto.gtex}

\input{temporalia/ps16-vii-c.tex}

\vfill

\antiphona{}{temporalia/ant-inclinadomine-cgp.gtex}

\vfill
\pagebreak

\pars{Psalmus 4.} \scriptura{Ps. 17, 2}

\vspace{-4mm}

\antiphona{VI F}{temporalia/ant-diligamtedomine-cgp.gtex}

%\vspace{-2mm}

\scriptura{Ps. 17, 2-16}

\vspace{-1mm}

\initiumpsalmi{temporalia/ps17i-initium-vi-F-auto.gtex}

\input{temporalia/ps17i-vi-F.tex}

\vfill

\antiphona{}{temporalia/ant-diligamtedomine-cgp.gtex}

\vfill
\pagebreak

\pars{Psalmus 5.} \scriptura{Ps. 17, 21}

\vspace{-4mm}

\antiphona{IV E}{temporalia/ant-retribuetmihi-cgp.gtex}

%\vspace{-2mm}

\scriptura{Ps. 17, 17-35}

\vspace{-1mm}

\initiumpsalmi{temporalia/ps17ii-initium-iv-E-auto.gtex}

\input{temporalia/ps17ii-iv-E.tex}

\vfill

\antiphona{}{temporalia/ant-retribuetmihi-cgp.gtex}

\vfill
\pagebreak

\pars{Psalmus 6.} \scriptura{Ps. 17, 47; \textbf{H100}}

\vspace{-4mm}

\antiphona{VII c\textsuperscript{2}}{temporalia/ant-vivitdominus.gtex}

%\vspace{-2mm}

\scriptura{Ps. 17, 36-51}

\vspace{-1mm}

\initiumpsalmi{temporalia/ps17iii-initium-vii-c2-auto.gtex}

\input{temporalia/ps17iii-vii-c2.tex}

\vfill

\antiphona{}{temporalia/ant-vivitdominus.gtex}

\vfill
\pagebreak

\pars{Psalmus 7.} \scriptura{Ps. 19, 2; \textbf{H100}}

\vspace{-4mm}

\antiphona{VIII G}{temporalia/ant-exaudiatte.gtex}

%\vspace{-2mm}

\scriptura{Ps. 19}

\vspace{-1mm}

\initiumpsalmi{temporalia/ps19-initium-viii-G-auto.gtex}

\input{temporalia/ps19-viii-G.tex} \Abardot{}

\vfill
\pagebreak

\pars{Psalmus 8.} \scriptura{Ps. 20, 1; \textbf{H89}}

\vspace{-4mm}

\antiphona{VIII G}{temporalia/ant-domineinvirtute.gtex}

%\vspace{-2mm}

\scriptura{Ps. 20}

\vspace{-1mm}

\initiumpsalmi{temporalia/ps20-initium-viii-G-auto.gtex}

\input{temporalia/ps20-viii-G.tex} \Abardot{}

\vfill
\pagebreak

\pars{Psalmus 9.} \scriptura{Ps. 29, 2; \textbf{H262}}

\vspace{-7mm}

\antiphona{VIII G}{temporalia/ant-exaltabote-cgp.gtex}

\vspace{-3mm}

\scriptura{Ps. 29}

\vspace{-2mm}

\initiumpsalmi{temporalia/ps29-initium-viii-G-auto.gtex}

\vspace{-1.5mm}

\input{temporalia/ps29-viii-G.tex} \Abardot{}

\vfill
\pagebreak

\ifx\matversus\undefined
\pars{Versus.} \scriptura{Ps. 118, 55}

% Versus. %%%
\sineinitiali{temporalia/versus-memorfui.gtex}
\else
\matversus
\fi

\vspace{5mm}

\sineinitiali{temporalia/oratiodominica-mat.gtex}

\vspace{5mm}

\pars{Absolutio.}

\cuminitiali{}{temporalia/absolutio-exaudi.gtex}

\vfill
\pagebreak

\cuminitiali{}{temporalia/benedictio-solemn-benedictione.gtex}

\vspace{7mm}

\lectioi

\noindent \Vbardot{} Tu autem, Dómine, miserére nobis.
\noindent \Rbardot{} Deo grátias.

\vfill
\pagebreak

\responsoriumi

\vfill
\pagebreak

\cuminitiali{}{temporalia/benedictio-solemn-unigenitus.gtex}

\vspace{7mm}

\lectioii

\noindent \Vbardot{} Tu autem, Dómine, miserére nobis.
\noindent \Rbardot{} Deo grátias.

\vfill
\pagebreak

\responsoriumii

\vfill
\pagebreak

\cuminitiali{}{temporalia/benedictio-solemn-spiritus.gtex}

\vspace{7mm}

\lectioiii

\noindent \Vbardot{} Tu autem, Dómine, miserére nobis.
\noindent \Rbardot{} Deo grátias.

\vfill
\pagebreak

\responsoriumiii

\vfill
\pagebreak

\rubrica{Reliqua omittuntur, nisi Laudes separandæ sint.}

\pars{Oratio}

\noindent \Vbardot{} Dómine, exáudi oratiónem meam.

\noindent \Rbardot{} Et clamor meus ad te véniat.

Orémus:

\oratioLaudes

\vspace{7mm}

\pars{Conclusio}

\noindent \Vbardot{} Dómine, exáudi oratiónem meam.

\noindent \Rbardot{} Et clamor meus ad te véniat.

\noindent \Vbardot{} Benedicámus Dómino.

\noindent \Rbardot{} Deo grátias.

\noindent \Vbardot{} Fidélium ánimæ per misericórdiam Dei requiéscant in pace.

\noindent \Rbardot{} Amen.

\vfill
\pagebreak

\end{document}
