\newcommand{\titulus}{\dies{Dominica infra octavam Nativitatis.}
\nomenFesti{Sanctæ Familiæ Iesu Mariæ \& Ioseph.}}
%\newcommand{\impar}{Impar}
\newcommand{\invitatorium}{\pars{Invitatorium.}

\vspace{-6mm}

\antiphona{IV *}{temporalia/inv-christumdeifilium.gtex}}
\newcommand{\hymnusmatutinum}{\pars{Hymnus.}

\vspace{-5mm}

\antiphona{VIII}{temporalia/hym-DulceFit.gtex}}
\newcommand{\matutinum}{\subhora{In I. Nocturno}

\vspace{-6mm}

\pars{Psalmus 1.} \scriptura{Cf. Lc. 2, 28; \textbf{H119}}

\vspace{-4mm}

\antiphona{III b}{temporalia/ant-accipienssimeonpuerum.gtex}

\vspace{-4mm}

\scriptura{Ps. 23}

%\vspace{-2mm}

\initiumpsalmi{temporalia/ps23-initium-iii-b-auto.gtex}

\input{temporalia/ps23-iii-b.tex} \Abardot{}

\vfill
\pagebreak

\pars{Psalmus 2.} \scriptura{Mt. 2, 11}

\vspace{-4mm}

\antiphona{III a}{temporalia/ant-intrantesmagi.gtex}

%\vspace{-5mm}

\scriptura{Ps. 45}

\initiumpsalmi{temporalia/ps45-initium-iii-a-auto.gtex}

\input{temporalia/ps45-iii-a.tex} \Abardot{}

\vfill
\pagebreak

\pars{Psalmus 3.} \scriptura{Mt. 2, 20-21; \textbf{H120}}

\vspace{-4mm}

\antiphona{III a}{temporalia/ant-tollepuerum.gtex}

%\vspace{-2mm}

\scriptura{Ps. 86}

%\vspace{-2mm}

\initiumpsalmi{temporalia/ps86-initium-iii-a-auto.gtex}

\input{temporalia/ps86-iii-a.tex} \Abardot{}

\vfill
\pagebreak

\pars{Versus.} 

\noindent \Vbardot{} Ponam univérsos fílios tuos doctos a Dómino.

\noindent \Rbardot{} Et multitúdinem pacis fíliis tuis.

\vspace{5mm}

\sineinitiali{temporalia/oratiodominica-mat.gtex}

\vspace{5mm}

\pars{Absolutio.}

\cuminitiali{}{temporalia/absolutio-exaudi.gtex}

\vfill
\pagebreak

\cuminitiali{}{temporalia/benedictio-solemn-benedictione.gtex}

\vspace{7mm}

\pars{Lectio I.} \scriptura{Eph. 5, 21-33; 6, 1-4}

\noindent De Epístola beáti Pauli apóstoli ad Ephésios.

\noindent Fratres: Subiécti ínvicem estóte in timóre Christi. Mulíeres viris suis sicut Dómino, quóniam vir caput est mulíeris, sicut et Christus caput est ecclésiæ, ipse salvátor córporis. Sed ut ecclésia subiécta est Christo, ita et mulíeres viris in ómnibus.

\noindent Viri, dilígite uxóres, sicut et Christus diléxit ecclésiam et seípsum trádidit pro ea, ut illam sanctificáret mundans lavácro aquæ in verbo, ut exhibéret ipse sibi gloriósam ecclésiam non habéntem máculam aut rugam aut áliquid eiúsmodi, sed ut sit sancta et immaculáta. Ita et viri debent dilígere uxóres suas ut córpora sua. Qui suam uxórem díligit, seípsum díligit; nemo enim umquam carnem suam ódio hábuit, sed nutrit et fovet eam sicut et Christus ecclésiam, quia membra sumus córporis eius.

\noindent \emph{Propter hoc relínquet homo patrem et matrem et adhærébit uxóri suæ et erunt duo in carne una.} Mystérium hoc magnum est; ego autem dico de Christo et ecclésia! Verúmtamen et vos sínguli unusquísque suam uxórem sicut seípsum díligat; uxor autem tímeat virum.

\noindent Fílii, obœdíte paréntibus vestris in Dómino, hoc enim est iustum. \emph{Honóra patrem tuum et matrem,} quod est mandátum primum cum promissióne, \emph{ut bene sit tibi et sis longǽvus super terram.} Et, patres, nolíte ad iracúndiam provocáre fílios vestros, sed educáte illos in disciplína et correptióne Dómini.

\noindent \Vbardot{} Tu autem, Dómine, miserére nobis.
\noindent \Rbardot{} Deo grátias.

\vfill
\pagebreak

\pars{Responsorium 1.} \scriptura{\Rbardot{} Cf. Ap. 2, 17 \Vbardot{} Cf. Mal. 3, 1; \textbf{H49}}

\vspace{-5mm}

\responsorium{III}{temporalia/resp-hicquiadvenit-CROCHU.gtex}{}

\vfill
\pagebreak

\cuminitiali{}{temporalia/benedictio-solemn-unigenitus.gtex}

\vspace{7mm}

\pars{Lectio II.} \scriptura{Sermo 50, 7-10: SC 339, 184-186}

\noindent Ex Sermónibus beáti Isaac abbátis monastérii de Stella.

\noindent Diábolus obœdiéntiæ præcéptum prior discútere cœpit: \emph{Quare}, inquit, \emph{præcépit vobis Deus non comédere de ligno sciéntiæ boni et mali?} Antehac homo simplex simplíciter obœdíerat, non tam propter præcépti ratiónem, quam propter præcipiéntis auctoritátem. Sicut enim \emph{fides non habet méritum, cui rátio humána præbet experiméntum,} sic nimírum obœdiéntia a virtúte humilitátis eo evacuátur, quo ei rátio præcépti astipulátur.

\noindent Vis tamen audíre quare ad aliénum aut laborámus aut pausámus arbítrium et impérium? Quia in hoc nimírum imitatóres sumus Christi, sicut fílii caríssimi, et ambulámus in dilectióne qua diléxit nos, qui ad ómnia factus est obœ́diens propter nos, non solum ad remédium sed étiam ad exémplum, ut quemádmodum ille fuit, sic et nos simus in hoc mundo. \emph{In hoc enim}, sicut beátus ait Ioánnes, \emph{est fidúcia}. Factus est ergo obœ́diens per ómnia, non solum Patri usque ad mortem, sed Maríæ et loseph usque ad prælatiónem.

\noindent \Vbardot{} Tu autem, Dómine, miserére nobis.
\noindent \Rbardot{} Deo grátias.

\vfill
\pagebreak

\pars{Responsorium 2.} \scriptura{\Rbardot{} Cf. Lc. 2, 12 \Vbardot{} Cf. Hab. 3, 2; \textbf{H45}}

\vspace{-5mm}

\responsorium{I}{temporalia/resp-oregemcaeli-CROCHU.gtex}{}

\vfill
\pagebreak

\cuminitiali{}{temporalia/benedictio-solemn-spiritus.gtex}

\vspace{7mm}

\pars{Lectio III.}

\noindent Cum enim — patérna vocatióne dicénte: \emph{Hic est Fílius meus diléctus in quo mihi bene complácui: ipsum audíte} — ad prælatiónem vocarétur, tunc primum cœpit áliis præésse, qui diu didícerat áliis subésse; cœpit iubére, qui dídicit obœdíre. Prætérea recompensátio iusta vidétur, ut qui in paradíso dedignátus est regnáre dóminus sub Dómino, in exsílio iam sérviat servus sub consérvo.

\noindent Condítio enim natúræ státuit hóminem sub Dómino; transgréssio obœdiéntiæ subiugávit eum inimíco; reconciliátio vero grátiæ suppósuit fratri consérvo. Natúra súbdidit eum Deo, culpa diábolo, reconciliátio vero hómini amíco.

\noindent \Vbardot{} Tu autem, Dómine, miserére nobis.
\noindent \Rbardot{} Deo grátias.

\vfill
\pagebreak

\pars{Responsorium 3.} \scriptura{\Vbardot{} Sedulius; \textbf{H48}}

\vspace{-5mm}

\responsorium{VII}{temporalia/resp-congratulaminiquiacum-CROCHU-cumdox.gtex}{}

\vfill
\pagebreak

\subhora{In II. Nocturno}

\vspace{-6mm}

\pars{Cantica.} \scriptura{Lc. 2, 7.13; \textbf{H54}}

\vspace{-4mm}

\antiphona{IV E}{temporalia/ant-oregemcaeli.gtex}

\vspace{-1mm}

\scriptura{Canticum Isaiæ, Is. 26, 1-12.}

\vspace{-2mm}

\initiumpsalmi{temporalia/isaiae3-initium-iv-E-auto.gtex}

\vspace{-1.5mm}

\input{temporalia/isaiae3-iv-E.tex}

\vfill
\pagebreak

\scriptura{Canticum Isaiæ, Is. 40, 1-8.}

\vspace{-2mm}

\initiumpsalmi{temporalia/isaiae6-initium-iv-E-auto.gtex}

\input{temporalia/isaiae6-iv-E.tex}

\vfill
\pagebreak

\scriptura{Canticum Isaiæ, Is. 66, 10-14b.}

\vspace{-2mm}

\initiumpsalmi{temporalia/isaiae5-initium-iv-E-auto.gtex}

\input{temporalia/isaiae5-iv-E.tex}

\vfill

\antiphona{}{temporalia/ant-oregemcaeli.gtex}

\vfill
\pagebreak

\pars{Versus.} \scriptura{Ps. 44, 3}

% Versus. %%%
\noindent \Vbardot{} Speciósus forma præ fíliis hóminum.

\noindent \Rbardot{} Diffúsa est grátia in lábiis tuis.

\vspace{5mm}

\sineinitiali{temporalia/oratiodominica-mat.gtex}

\vspace{5mm}

\pars{Absolutio.}

\cuminitiali{}{temporalia/absolutio-ipsius.gtex}

\vfill
\pagebreak

\cuminitiali{}{temporalia/benedictio-solemn-evangelica.gtex}

\vspace{7mm}

\pars{Evangelium} \scriptura{Mt. 2,13-15.19-23}

\noindent Léctio sancti Evangélii secúndum Matthǽum 

\noindent Cum recessíssent Magi, ecce ángelus Dómini appáret in somnis Ioseph dicens: «Surge et áccipe púerum et matrem eius et fuge in Ægýptum et esto ibi, usque dum dicam tibi; futúrum est enim ut Heródes quærat púerum ad perdéndum eum». Qui consúrgens accépit púerum et matrem eius nocte et recéssit in Ægýptum et erat ibi usque ad óbitum Heródis, ut adimplerétur, quod dictum est a Dómino per prophétam dicéntem: «Ex Ægýpto vocávi fílium meum».

\noindent Defúncto autem Heróde, ecce appáret ángelus Dómini in somnis Ioseph in Ægýpto dicens: «Surge et áccipe púerum et matrem eius et vade in terram Israel; defúncti sunt enim, qui quærébant ánimam púeri». Qui surgens accépit púerum et matrem eius et venit in terram Israel.

\noindent Audiens autem quia Archeláus regnáret in Iudǽa pro Heróde patre suo, tímuit illuc ire; et admónitus in somnis, secéssit in partes Galilǽæ et véniens habitávit in civitáte, quæ vocátur Názareth, ut adimplerétur, quod dictum est per Prophétas: «Nazarǽus vocábitur».

\scriptura{Sermo in Natálem Christi diem : PG 56, 392. 396}

\noindent Ex Sermónibus sancti Ioánnis Chrysóstomi, epíscopi

\noindent Quid dicam aut quid loquar? Ecce infans fásciis invólvitur, et in præsépi iacet; adest autem et María quæ virgo et mater est; áderat autem et Ioseph qui pater appellátur. Hic vir dícitur, illa uxor vocátur: legítima sunt nómina copulatióne destitúta. Verbórum tenus hæc mihi intéllege, non rerum tenus. Hic solum desponsávit, et Spíritus Sanctus obumbrávit ei: unde dúbitans loseph, quid appelláret infántem nesciébat. Ex adultério natum dícere ipsum non audébat, probrósum in vírginem iácere sermónem non póterat, fílium ipsum dícere suum refugiébat: probe namque sciébat sibi ignótum esse, quo pacto, vel unde infans natus esset; quam ob causam illi de re dubitánti, de cælo oráculum ángeli voce delátum est: \emph{Noli timére, Ioseph, quod enim ex ea génitum est, de Spíritu Sancto est.} Spíritus enim Sanctus Vírgini obumbrávit.

\noindent {\color{gray} Quare vero ex vírgine náscitur, et virginitátem illibátam servat? Quia quondam vírginem Evam decépit idcírco ad Maríam, quæ virgo erat, felícem núntium Gábriel détulit. Sed decépta quidem Eva péperit verbum quod mortem íntulit; at felícem núntium accípiens María Verbum in carne génuit, quod vitam nobis ætérnam concíliat.

\noindent Verbum Evæ lignum indicávit, per quod lignum e paradíso Adámum éxpulit: Verbum autem quod ex Vírgine pródiit, crucem exhíbuit, per quod latrónem vice Adámi in paradísum introdúxit. Nam quóniam neque gentíles, neque Iudǽi, neque hærétici credébant Deum sine passióne ac deflúxu genuísse, proptérea hódie ex patíbili córpore progréssus, impatíbile conservávit corpus patíbile, ut osténderet, quemádmodum ex Vírgine natus virginitátem non solvit, ita Deum quoque, non deflúxa nec mutáta manénte sacra substántia, tamquam Deum, prout Deo conveniébat, Deum genuísse.

\noindent Postquam enim hómines eo derelícto státuas sibi humána prǽditas forma scúlpserant, quibus in creatóris contuméliam cultum deferébant, proptérea Dei Verbum, cum Deus esset, hódie in hóminis forma appáruit, ut mendácium dissólveret et occúlte in seípsum cultum omnem transférret. Huic ígitur, qui res impedítas ita réddidit expedítas, Christo Dómino glóriam offerámus et una Patri et Spirítui Sancto, nunc et semper et in sǽcula sæculórum. Amen.} 

\vfill
\pagebreak

\pars{Responsorium 4.} \scriptura{\Vbardot{} Lc. 1, 28; \textbf{H48}}

\vspace{-5mm}

\responsorium{VII}{temporalia/resp-beataviscera-CROCHU-cumdox.gtex}{}

\vfill
\pagebreak}
\newcommand{\hymnuslaudes}{\pars{Hymnus.}

\vspace{-5mm}

\antiphona{VIII}{temporalia/hym-ChristeSplendor.gtex}}
\newcommand{\laudes}{\pars{Psalmus 1.} \scriptura{Lc. 2, 41}

\vspace{-4mm}

\antiphona{VII d}{temporalia/ant-ibantparentesiesu.gtex}

\vspace{-2mm}

\scriptura{Psalmus 62.}

\vspace{-1mm}

\initiumpsalmi{temporalia/ps62-initium-vii-d-auto.gtex}

\input{temporalia/ps62-vii-d.tex} \Abardot{}

\vfill
\pagebreak

\pars{Psalmus 2.} \scriptura{Lc. 2, 52 \textbf{H50}}

\vspace{-4mm}

\antiphona{VI F}{temporalia/ant-pueriesusproficiebat.gtex}

\scriptura{Canticum trium puerorum, Dan. 3, 57-88 et 56}

\initiumpsalmi{temporalia/dan3-initium-vi-F-auto.gtex}

\input{temporalia/dan3-vi-F-sinedox.tex}

\rubrica{Hic non dicitur Gloria Patri, neque Amen.}

\vfill

\vspace{-6mm}

\antiphona{}{temporalia/ant-pueriesusproficiebat.gtex} % repeat the antiphon - new page

\vfill
\pagebreak

\pars{Psalmus 3.}

\vspace{-4mm}

\antiphona{II D}{temporalia/ant-erantioseph.gtex}

\scriptura{Psalmus 149.}

\initiumpsalmi{temporalia/ps149-initium-ii-D-auto.gtex}

\input{temporalia/ps149-ii-D.tex} \Abardot{}

\vfill
\pagebreak}
\newcommand{\lectiobrevis}{\pars{Lectio Brevis.} \scriptura{Dt. 5, 16}

\noindent Honóra patrem tuum et matrem, sicut præcépit tibi Dóminus Deus tuus, ut longo vivas témpore et bene sit tibi in terra, quam Dóminus Deus tuus datúrus est tibi.}
\newcommand{\responsoriumbreve}{\pars{Responsorium breve.} \scriptura{Cf. Mt. 16, 16; Lc. 2, 51}

\cuminitiali{VI}{temporalia/resp-christefilidei-familiae.gtex}}
\newcommand{\oratio}{\pars{Oratio.}

\noindent Deus, qui præclára nobis Sanctæ Famíliæ dignátus es exémpla præbére, concéde propítius, ut, domésticis virtútibus caritatísque vínculis illam sectántes, in lætítia domus tuæ prǽmiis fruámur ætérnis.

\pars{Pro commemoratione Sancti Stephani, Protomartyris.} \scriptura{\textbf{H59}}

\vspace{-4mm}

\antiphona{VIII G}{temporalia/ant-patefactaesunt.gtex}

\vfill

\noindent Da nobis, quǽsumus, Dómine, imitári quod cólimus, ut discámus et inimícos dilígere, quia eius natalícia celebrámus, qui novit étiam pro persecutóribus exoráre.

\noindent Per Dóminum nostrum Iesum Christum, Fílium tuum, qui tecum vivit et regnat in unitáte Spíritus Sancti, Deus, per ómnia sǽcula sæculórum.  Amen.

\noindent \Rbardot{} Amen.}
\newcommand{\benedictus}{\pars{Canticum Zachariæ.} \scriptura{Lc. 2, 39-40}

\vspace{-4mm}

{
\grechangedim{interwordspacetext}{0.18 cm plus 0.15 cm minus 0.05 cm}{scalable}%
\antiphona{I g}{temporalia/ant-etutperfecerunt.gtex}
\grechangedim{interwordspacetext}{0.22 cm plus 0.15 cm minus 0.05 cm}{scalable}%
}

\vspace{-1mm}

\scriptura{Lc. 1, 68-79}

\vspace{-2mm}

\cantusSineNeumas
\initiumpsalmi{temporalia/benedictus-initium-isoll-g-auto.gtex}

%\vspace{-1.5mm}

\input{temporalia/benedictus-isoll-g.tex} \Abardot{}}
\newcommand{\preces}{\noindent Fílium Dei vivi, \gredagger{} qui fílius humánæ famíliæ fíeri dignátus est, \grestar{} adorémus, ipsum exorántes:

\Rbardot{} Iesu, factus obœ́diens, sanctífica nos.

\noindent Iesu, Verbum ætérnum Patris, \gredagger{} qui súbditum Maríæ et Ioseph te fecísti, \grestar{} doce nos humilitátem.

\Rbardot{} Iesu, factus obœ́diens, sanctífica nos.

\noindent Magíster noster, \gredagger{} cuius facta et verba mater tua in corde suo servábat, \grestar{} præsta, ut audiámus verbum tuum et custodiámus illud in corde puro et bono.

\Rbardot{} Iesu, factus obœ́diens, sanctífica nos.

\noindent Christe, faber mundi, \gredagger{} qui fílium fabri te vocári voluísti, \grestar{} doce nos sedulitátem in ópere.

\Rbardot{} Iesu, factus obœ́diens, sanctífica nos.

\noindent Iesu, qui in família Názareth sapiéntia, ætáte et grátia apud Deum et hómines proficiébas, \grestar{} fac, ut crescámus per ómnia in te, qui es Caput nostrum.

\Rbardot{} Iesu, factus obœ́diens, sanctífica nos.}
\newcommand{\magnificat}{\pars{Canticum B. Mariæ V.} \scriptura{Lc. 2, 52 \textbf{H50}}

\vspace{-4mm}

\antiphona{VI F}{temporalia/ant-pueriesusproficiebat.gtex}

\vspace{-2mm}

\scriptura{Lc. 1, 46-55}

\vspace{-2mm}

\cantusSineNeumas
\initiumpsalmi{temporalia/magnificat-initium-visoll-F.gtex}

\vspace{-1.5mm}

\input{temporalia/magnificat-visoll-F.tex} \Abardot{}}
\newcommand{\capitulum}{\pars{Capitulum.} \scriptura{Gal. 4, 1-2}

\grechangedim{interwordspacetext}{0.12 cm plus 0.15 cm minus 0.05 cm}{scalable}%
\cuminitiali{}{temporalia/capitulum-FratresQuanto.gtex}
\grechangedim{interwordspacetext}{0.22 cm plus 0.15 cm minus 0.05 cm}{scalable}
}
\newcommand{\oratioVesperas}{\cuminitiali{}{temporalia/oratiodomi.gtex}}
% LuaLaTeX

\documentclass[a4paper, twoside, 12pt]{article}
\usepackage[latin]{babel}
%\usepackage[landscape, left=3cm, right=1.5cm, top=2cm, bottom=1cm]{geometry} % okraje stranky
%\usepackage[landscape, a4paper, mag=1166, truedimen, left=2cm, right=1.5cm, top=1.6cm, bottom=0.95cm]{geometry} % okraje stranky
\usepackage[landscape, a4paper, mag=1400, truedimen, left=0.5cm, right=0.5cm, top=0.5cm, bottom=0.5cm]{geometry} % okraje stranky

\usepackage{fontspec}
\setmainfont[FeatureFile={junicode.fea}, Ligatures={Common, TeX}, RawFeature=+fixi]{Junicode}
%\setmainfont{Junicode}

% shortcut for Junicode without ligatures (for the Czech texts)
\newfontfamily\nlfont[FeatureFile={junicode.fea}, Ligatures={Common, TeX}, RawFeature=+fixi]{Junicode}

% Hebrew font:
% http://scripts.sil.org/cms/scripts/page.php?site_id=nrsi&id=SILHebrUnic2
\newfontfamily\hebfont[Scale=1]{Ezra SIL}

\usepackage{multicol}
\usepackage{color}
\usepackage{lettrine}
\usepackage{fancyhdr}

% usual packages loading:
\usepackage{luatextra}
\usepackage{graphicx} % support the \includegraphics command and options
\usepackage{gregoriotex} % for gregorio score inclusion
\usepackage{gregoriosyms}
\usepackage{wrapfig} % figures wrapped by the text
\usepackage{parcolumns}
\usepackage[contents={},opacity=1,scale=1,color=black]{background}
\usepackage{tikzpagenodes}
\usepackage{calc}
\usepackage{longtable}
\usetikzlibrary{calc}

\setlength{\headheight}{14.5pt}

% Commands used to produce a typical "Conventus" booklet

\newenvironment{titulusOfficii}{\begin{center}}{\end{center}}
\newcommand{\dies}[1]{#1

}
\newcommand{\nomenFesti}[1]{\textbf{\Large #1}

}
\newcommand{\celebratio}[1]{#1

}

\newcommand{\hora}[1]{%
\vspace{0.5cm}{\large \textbf{#1}}

\fancyhead[LE]{\thepage\ / #1}
\fancyhead[RO]{#1 / \thepage}
\addcontentsline{toc}{subsection}{#1}
}

% larger unit than a hora
\newcommand{\divisio}[1]{%
\begin{center}
{\Large \textsc{#1}}
\end{center}
\fancyhead[CO,CE]{#1}
\addcontentsline{toc}{section}{#1}
}

% a part of a hora, larger than pars
\newcommand{\subhora}[1]{
\begin{center}
{\large \textit{#1}}
\end{center}
%\fancyhead[CO,CE]{#1}
\addcontentsline{toc}{subsubsection}{#1}
}

% rubricated inline text
\newcommand{\rubricatum}[1]{\textit{#1}}

% standalone rubric
\newcommand{\rubrica}[1]{\vspace{3mm}\rubricatum{#1}}

\newcommand{\notitia}[1]{\textcolor{red}{#1}}

\newcommand{\scriptura}[1]{\hfill \small\textit{#1}}

\newcommand{\translatioCantus}[1]{\vspace{1mm}%
{\noindent\footnotesize \nlfont{#1}}}

% pruznejsi varianta nasledujiciho - umoznuje nastavit sirku sloupce
% s prekladem
\newcommand{\psalmusEtTranslatioB}[3]{
  \vspace{0.5cm}
  \begin{parcolumns}[colwidths={2=#3}, nofirstindent=true]{2}
    \colchunk{
      \input{#1}
    }

    \colchunk{
      \vspace{-0.5cm}
      {\footnotesize \nlfont
        \input{#2}
      }
    }
  \end{parcolumns}
}

\newcommand{\psalmusEtTranslatio}[2]{
  \psalmusEtTranslatioB{#1}{#2}{8.5cm}
}


\newcommand{\canticumMagnificatEtTranslatio}[1]{
  \psalmusEtTranslatioB{#1}{temporalia/extra-adventum-vespers/magnificat-boh.tex}{12cm}
}
\newcommand{\canticumBenedictusEtTranslatio}[1]{
  \psalmusEtTranslatioB{#1}{temporalia/extra-adventum-laudes/benedictus-boh.tex}{10.5cm}
}

% volne misto nad antifonami, kam si zpevaci dokresli neumy
\newcommand{\hicSuntNeumae}{\vspace{0.5cm}}

% prepinani mista mezi notovymi osnovami: pro neumovane a neneumovane zpevy
\newcommand{\cantusCumNeumis}{
  \setgrefactor{17}
  \global\advance\grespaceabovelines by 5mm%
}
\newcommand{\cantusSineNeumas}{
  \setgrefactor{17}
  \global\advance\grespaceabovelines by -5mm%
}

% znaky k umisteni nad inicialu zpevu
\newcommand{\superInitialam}[1]{\gresetfirstlineaboveinitial{\small {\textbf{#1}}}{\small {\textbf{#1}}}}

% pars officii, i.e. "oratio", ...
\newcommand{\pars}[1]{\textbf{#1}}

\newenvironment{psalmus}{
  \setlength{\parindent}{0pt}
  \setlength{\parskip}{5pt}
}{
  \setlength{\parindent}{10pt}
  \setlength{\parskip}{10pt}
}

%%%% Prejmenovat na latinske:
\newcommand{\nadpisZalmu}[1]{
  \hspace{2cm}\textbf{#1}\vspace{2mm}%
  \nopagebreak%

}

% mode, score, translation
\newcommand{\antiphona}[3]{%
\hicSuntNeumae
\superInitialam{#1}
\includescore{#2}

#3
}
 % Often used macros

\newcommand{\annusEditionis}{2021}

\def\hebinitial#1{%
\leavevmode{\newbox\hebbox\setbox\hebbox\hbox{\hebfont{#1}\hskip 1mm}\kern -\wd\hebbox\hbox{\hebfont{#1}\hskip 1mm}}%
}

%%%% Vicekrat opakovane kousky

\newcommand{\anteOrationem}{
  \rubrica{Ante Orationem, cantatur a Superiore:}

  \pars{Supplicatio Litaniæ.}

  \cuminitiali{}{temporalia/supplicatiolitaniae.gtex}

  \pars{Oratio Dominica.}

  \cuminitiali{}{temporalia/oratiodominica.gtex}

  \rubrica{Deinde dicitur ab Hebdomadario:}

  \cuminitiali{}{temporalia/dominusvobiscum-solemnis.gtex}

  \rubrica{In choro monialium loco Dominus vobiscum dicitur:}

  \sineinitiali{temporalia/domineexaudi.gtex}
}

\setlength{\columnsep}{30pt} % prostor mezi sloupci

%%%%%%%%%%%%%%%%%%%%%%%%%%%%%%%%%%%%%%%%%%%%%%%%%%%%%%%%%%%%%%%%%%%%%%%%%%%%%%%%%%%%%%%%%%%%%%%%%%%%%%%%%%%%%
\begin{document}

% Here we set the space around the initial.
% Please report to http://home.gna.org/gregorio/gregoriotex/details for more details and options
\grechangedim{afterinitialshift}{2.2mm}{scalable}
\grechangedim{beforeinitialshift}{2.2mm}{scalable}
\grechangedim{interwordspacetext}{0.22 cm plus 0.15 cm minus 0.05 cm}{scalable}%
\grechangedim{annotationraise}{-0.2cm}{scalable}

% Here we set the initial font. Change 38 if you want a bigger initial.
% Emit the initials in red.
\grechangestyle{initial}{\color{red}\fontsize{38}{38}\selectfont}

\pagestyle{empty}

%%%% Titulni stranka
\begin{titulusOfficii}
\titulus
\end{titulusOfficii}

\vfill

\begin{center}
%Ad usum et secundum consuetudines chori \guillemotright{}Conventus Choralis\guillemotleft.

%Editio Sancti Wolfgangi \annusEditionis
\end{center}

\scriptura{}

\pars{}

\pagebreak

\renewcommand{\headrulewidth}{0pt} % no horiz. rule at the header
\fancyhf{}
\pagestyle{fancy}

\cantusSineNeumas

\pars{} \scriptura{}

\ifx\sinematutinum\undefined
\hora{Ad Matutinum.} %%%%%%%%%%%%%%%%%%%%%%%%%%%%%%%%%%%%%%%%%%%%%%%%%%%%%

\vspace{2mm}

\cuminitiali{}{temporalia/dominelabiamea.gtex}

\vfill
%\pagebreak

\vspace{2mm}

\ifx\invitatorium\undefined
\pars{Invitatorium.}

\vspace{-2mm}

\antiphona{E}{temporalia/inv-christusnatusest-simplex.gtex}
\else
\invitatorium
\fi

\vfill
\pagebreak

\ifx\hymnusmatutinum\undefined
\pars{Hymnus.}

{
\grechangedim{interwordspacetext}{0.10 cm plus 0.15 cm minus 0.05 cm}{scalable}%
\antiphona{IV}{temporalia/hym-CandorAEternae-simplex.gtex}
\grechangedim{interwordspacetext}{0.22 cm plus 0.15 cm minus 0.05 cm}{scalable}%
}

\vspace{-3mm}
\else
\hymnusmatutinum
\fi

\vfill
\pagebreak

\matutinum

\ifx\postoctavam\undefined
% Te Deum

\vspace{-5mm}

\ifx\tedeumsolemnis\undefined
\ifx\tedeumsimplex\undefined
\ifx\tedeummonasticum\undefined
{
\pars{Hymnus Ambrosianus} \scriptura{Alio modo, iuxta morem Romanum}

\vspace{-2mm}

\grechangedim{interwordspacetext}{0.26 cm plus 0.15 cm minus 0.05 cm}{scalable}%
\cuminitiali{III}{temporalia/tedeum-romanum-gn.gtex}
\grechangedim{interwordspacetext}{0.22 cm plus 0.15 cm minus 0.05 cm}{scalable}%
}
\else
{
\pars{Hymnus Ambrosianus} \scriptura{Tonus Monasticus}

\vspace{-2mm}

\grechangedim{interwordspacetext}{0.26 cm plus 0.15 cm minus 0.05 cm}{scalable}%
\cuminitiali{III}{temporalia/tedeum-monasticum-am34.gtex}
\grechangedim{interwordspacetext}{0.22 cm plus 0.15 cm minus 0.05 cm}{scalable}%
}
\fi
\else
{
\pars{Hymnus Ambrosianus} \scriptura{Tonus Simplex}

\vspace{-2mm}

\grechangedim{interwordspacetext}{0.30 cm plus 0.15 cm minus 0.05 cm}{scalable}%
\cuminitiali{III}{temporalia/tedeum-simplex-gn.gtex}
\grechangedim{interwordspacetext}{0.22 cm plus 0.15 cm minus 0.05 cm}{scalable}%
}
\fi
\else
{
\pars{Hymnus Ambrosianus} \scriptura{Tonus Solemnis}

\vspace{-2mm}

\grechangedim{interwordspacetext}{0.26 cm plus 0.15 cm minus 0.05 cm}{scalable}%
\cuminitiali{III}{temporalia/tedeum-solemnis-gn.gtex}
\grechangedim{interwordspacetext}{0.22 cm plus 0.15 cm minus 0.05 cm}{scalable}%
}
\fi

\vfill
\pagebreak
\fi

\rubrica{Reliqua omittuntur, nisi Laudes separandæ sint.}

\sineinitiali{temporalia/domineexaudi.gtex}

\vfill

\oratio

\vfill

\noindent \Vbardot{} Dómine, exáudi oratiónem meam.
\Rbardot{} Et clamor meus ad te véniat.

\vfill

% Nocturnale Romanum 2002, p. LXXVI Benedicamus Domino seems to match
% the one from Solemn Laudes.
\cuminitiali{V}{temporalia/benedicamus-solemnis-laud.gtex}

\vfill

\noindent \Vbardot{} Fidélium ánimæ per misericórdiam Dei requiéscant in pace.
\Rbardot{} Amen.

\vfill
\pagebreak
\fi

\ifx\sinelaudes\undefined
\hora{Ad Laudes.} %%%%%%%%%%%%%%%%%%%%%%%%%%%%%%%%%%%%%%%%%%%%%%%%%%%%%

\cantusSineNeumas

\vspace{0.5cm}
\grechangedim{interwordspacetext}{0.18 cm plus 0.15 cm minus 0.05 cm}{scalable}%
\ifx\postoctavam\undefined
\cuminitiali{}{temporalia/deusinadiutorium-alter.gtex}
\else
\cuminitiali{}{temporalia/deusinadiutorium-communis.gtex}
\fi
\grechangedim{interwordspacetext}{0.22 cm plus 0.15 cm minus 0.05 cm}{scalable}%

\vfill
%\pagebreak

\ifx\hymnuslaudes\undefined
\pagebreak
\pars{Hymnus} \scriptura{Sedulius}

\grechangedim{interwordspacetext}{0.16 cm plus 0.15 cm minus 0.05 cm}{scalable}%
\cuminitiali{III}{temporalia/hym-ASolisOrtus.gtex}
\grechangedim{interwordspacetext}{0.22 cm plus 0.15 cm minus 0.05 cm}{scalable}%
\vspace{-3mm}
\else
\hymnuslaudes
\fi

\vfill
\pagebreak

\ifx\laudes\undefined
\pars{Psalmus 1.} \scriptura{Lc. 2, 8.11.13.18; \textbf{H50}}

\vspace{-4mm}

\antiphona{II D}{temporalia/ant-quemvidistis.gtex}

\vspace{-2mm}

\scriptura{Psalmus 62.}

\vspace{-1mm}

\initiumpsalmi{temporalia/ps62-initium-ii-D-auto.gtex}

\input{temporalia/ps62-ii-D.tex} \Abardot{}

\vfill
\pagebreak

\pars{Psalmus 2.} \scriptura{Lc. 2, 10.11; \textbf{H50}}

\vspace{-4mm}

\antiphona{VII d}{temporalia/ant-angelusadpastores.gtex}

\scriptura{Canticum trium puerorum, Dan. 3, 57-88 et 56}

\initiumpsalmi{temporalia/dan3-initium-vii-d-auto.gtex}

\input{temporalia/dan3-vii-d-sinedox.tex}

\rubrica{Hic non dicitur Gloria Patri, neque Amen.}

\vfill

\vspace{-6mm}

\antiphona{}{temporalia/ant-angelusadpastores.gtex} % repeat the antiphon - new page

\vfill
\pagebreak

\pars{Psalmus 3.} \scriptura{Is. 9, 6; \textbf{H51}}

\vspace{-4mm}

\antiphona{VIII G\textsuperscript{2}}{temporalia/ant-parvulusfilius.gtex}

\scriptura{Psalmus 149.}

\initiumpsalmi{temporalia/ps149-initium-viii-G2-auto.gtex}

\input{temporalia/ps149-viii-G2.tex} \Abardot{}

\vfill
\pagebreak
\else
\laudes
\fi

\lectiobrevis

\vfill

\ifx\responsoriumbreve\undefined
\pars{Responsorium breve.} \scriptura{Ps. 97, 2}

\cuminitiali{VI}{temporalia/resp-notumfecit.gtex}
\else
\responsoriumbreve
\fi

\vfill
\pagebreak

\benedictus

\vspace{-1cm}

\vfill
\pagebreak

\pars{Preces.}

\sineinitiali{}{temporalia/tonusprecumnovum.gtex}

\preces

\vfill

\pars{Oratio Dominica.}

\cuminitiali{}{temporalia/oratiodominicaalt.gtex}

\vfill
\pagebreak

\rubrica{vel:}

\pars{Deprecatio Gelasii}

\vspace{-5mm}

\grechangedim{interwordspacetext}{0.16 cm plus 0.15 cm minus 0.05 cm}{scalable}%
\antiphona{D\textsuperscript{1}}{temporalia/deprecatio4-propace.gtex}
\grechangedim{interwordspacetext}{0.22 cm plus 0.15 cm minus 0.05 cm}{scalable}%

\vfill

\pars{Oratio Dominica.}

\cuminitiali{D}{temporalia/oratiodominica-d.gtex}

\vfill
\pagebreak

% Oratio. %%%
\oratio

\vspace{-1mm}

\vfill

\ifx\commemoratio\undefined
\else
\commemoratio
\fi

\rubrica{Hebdomadarius dicit Dominus vobiscum, vel, absente sacerdote vel diacono, sic concluditur:}

\vspace{2mm}

\antiphona{C}{temporalia/dominusnosbenedicat.gtex}

\rubrica{Postea cantatur a cantore:}

\vspace{2mm}

\ifx\benedicamuslaudes\undefined
\ifx\postoctavam\undefined
\cuminitiali{II}{temporalia/benedicamus-solemnism-laud.gtex}
\else
\cuminitiali{}{temporalia/benedicamus-tempore-nativitatis.gtex}
\fi
\else
\benedicamuslaudes
\fi

\vspace{1mm}

\vfill
\pagebreak
\fi

\ifx\sinevesperas\undefined
\hora{Ad Vesperas.} %%%%%%%%%%%%%%%%%%%%%%%%%%%%%%%%%%%%%%%%%%%%%%%%%%%%%

\cantusSineNeumas

%\vspace{-2mm}
\grechangedim{interwordspacetext}{0.18 cm plus 0.15 cm minus 0.05 cm}{scalable}%
\ifx\postoctavam\undefined
\cuminitiali{}{temporalia/deusinadiutorium-solemnis.gtex}
\else
\cuminitiali{}{temporalia/deusinadiutorium-communis.gtex}
\fi
\grechangedim{interwordspacetext}{0.22 cm plus 0.15 cm minus 0.05 cm}{scalable}%

\vfill
%\pagebreak

\vspace{-2mm}

\ifx\vesperas\undefined
\pars{Psalmus 1.} \scriptura{Ps. 109, 3; \textbf{H52}}

\vspace{-5mm}

\antiphona{I g}{temporalia/ant-tecumprincipium.gtex}

\scriptura{Psalmus 109.}

\initiumpsalmi{temporalia/ps109-initium-i-g-auto.gtex}

\vspace{-1.5mm}

\input{temporalia/ps109-i-g.tex} \Abardot{}

\vfill
\pagebreak

\pars{Psalmus 2.} \scriptura{Ps. 110, 9; \textbf{H52}}

\vspace{-4mm}

\antiphona{VII a}{temporalia/ant-redemptionemmisit.gtex}

\scriptura{Psalmus 110.}

\initiumpsalmi{temporalia/ps110-initium-vii-a-auto.gtex}

\input{temporalia/ps110-vii-a.tex} \Abardot{}

\vfill
\pagebreak

\pars{Psalmus 3.} \scriptura{Ps. 111, 4; \textbf{H52}}

\vspace{-4mm}

\antiphona{VII d}{temporalia/ant-exortumest.gtex}

\scriptura{Psalmus 111.}

\initiumpsalmi{temporalia/ps111-initium-vii-d-auto.gtex}

\input{temporalia/ps111-vii-d.tex} \Abardot{}

\vfill
\pagebreak

\ifx\impar\undefined
\pars{Psalmus 4.} \scriptura{Ps. 131, 11; \textbf{H52}}

\vspace{-4mm}

\antiphona{VIII G}{temporalia/ant-defructuventris.gtex}

\scriptura{Psalmus 131.}

\initiumpsalmi{temporalia/ps131-initium-viii-G-auto.gtex}

\input{temporalia/ps131-viii-G.tex}

\vfill

\antiphona{}{temporalia/ant-defructuventris.gtex}
\else
\pars{Psalmus 4.} \scriptura{Ps. 129, 7; \textbf{H52}}

\vspace{-4mm}

\antiphona{II* b}{temporalia/ant-apuddominum.gtex}

\scriptura{Psalmus 129.}

\initiumpsalmi{temporalia/ps129-initium-ii_-B-auto.gtex}

\input{temporalia/ps129-ii_-B.tex} \Abardot{}
\fi

\vfill
\pagebreak
\else
\vesperas
\fi

\ifx\capitulum\undefined
\pars{Capitulum.} \scriptura{Hebr. 1, 1-2}

\grechangedim{interwordspacetext}{0.12 cm plus 0.15 cm minus 0.05 cm}{scalable}%
\cuminitiali{}{temporalia/capitulum-Multifariam.gtex}
\grechangedim{interwordspacetext}{0.22 cm plus 0.15 cm minus 0.05 cm}{scalable}
\else
\capitulum
\fi

\vfill

\ifx\responsoriumbrevevesp\undefined
\pars{Responsorium breve.} \scriptura{Io. 1, 14}

\cuminitiali{VI}{temporalia/resp-verbumcaro-simplex.gtex}
\else
\responsoriumbrevevesp
\fi

\vfill
\pagebreak

\ifx\hymnusvesperas\undefined
\pars{Hymnus}

\cuminitiali{I}{temporalia/hym-ChristeRedemptor.gtex}
\else
\hymnusvesperas
\fi
\vspace{-3mm}

\vfill
%\pagebreak

\ifx\vespversus\undefined
\pars{Versus.} \scriptura{Ps. 97, 2}

% Versus. %%%
\sineinitiali{temporalia/versus-notumfecit-communis.gtex}
\else
\vespversus
\fi

\vfill
\pagebreak

\magnificat

\vfill
\pagebreak

\anteOrationem

\pagebreak

% Oratio. %%%
\ifx\oratioVesperas\undefined
\cuminitiali{}{temporalia/oratio.gtex}
\else
\oratioVesperas
\fi

\vspace{-1mm}

\vfill

\rubrica{Hebdomadarius dicit iterum Dominus vobiscum, vel cantor dicit:}

\vspace{2mm}

\sineinitiali{temporalia/domineexaudi.gtex}

\rubrica{Postea cantatur a cantore:}

\vspace{2mm}

\ifx\postoctavam\undefined
\cuminitiali{II}{temporalia/benedicamus-solemnism-2vesp.gtex}
\else
\cuminitiali{I}{temporalia/benedicamus-feria-vesperae.gtex}
\fi

\vspace{1mm}
\fi

\end{document}

