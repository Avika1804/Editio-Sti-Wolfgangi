\newcommand{\titulus}{\dies{27. Decembris.}
\nomenFesti{S. Ioannis, Apostoli \& Evangelistæ.}}
\newcommand{\impar}{Impar}
\newcommand{\invitatorium}{\pars{Invitatorium.}

\vspace{-2mm}

\antiphona{IV*}{temporalia/inv-regemapostolorum-iv.gtex}}
\newcommand{\hymnusmatutinum}{\pars{Hymnus.}

\vspace{-5mm}

\antiphona{IV}{temporalia/hym-VirginisVirgo.gtex}}
\newcommand{\matutinum}{\pars{Psalmus 1.} \scriptura{Io. 21, 24; \textbf{H60}}

\vspace{-4mm}

\antiphona{III b trans.}{temporalia/ant-hicestdiscipulusille.gtex}

\vspace{-4mm}

\scriptura{Ps. 18}

%\vspace{-2mm}

\initiumpsalmi{temporalia/ps18-initium-iii-b-trans.gtex}

\input{temporalia/ps18-iii-b.tex}

\vfill

\antiphona{}{temporalia/ant-hicestdiscipulusille.gtex}

\vfill
\pagebreak

\pars{Psalmus 2.} \scriptura{Io. 21, 24.22; \textbf{H64}}

\vspace{-4mm}

\antiphona{III b}{temporalia/ant-hicestdiscipulusmeus.gtex}

%\vspace{-5mm}

\scriptura{Ps. 63}

\initiumpsalmi{temporalia/ps63-initium-iii-b-auto.gtex}

\input{temporalia/ps63-iii-b.tex} \Abardot{}

\vfill
\pagebreak

\pars{Psalmus 3.} \scriptura{\textbf{H64}}

\vspace{-4mm}

\antiphona{I g}{temporalia/ant-isteestioannesquisupra.gtex}

%\vspace{-2mm}

\scriptura{Ps. 98}

%\vspace{-2mm}

\initiumpsalmi{temporalia/ps98-initium-i-g-auto.gtex}

\input{temporalia/ps98-i-g.tex} \Abardot{}

\vfill
\pagebreak

\pars{Versus.} 

\noindent \Vbardot{} Narravérunt laudes Dómini et virtútes eius.

\noindent \Rbardot{} Et mirabília eius quæ fecit.

\vspace{5mm}

\sineinitiali{temporalia/oratiodominica-mat.gtex}

\vspace{5mm}

\pars{Absolutio.}

\cuminitiali{}{temporalia/absolutio-exaudi.gtex}

\vfill
\pagebreak

\cuminitiali{}{temporalia/benedictio-solemn-benedictione.gtex}

\vspace{7mm}

\pars{Lectio I.} \scriptura{1 Io. 1, 1-10; 2, 1-3}

\noindent De Epístola prima beáti Ioánnis apóstoli.

\noindent Quod fuit ab inítio, quod audívimus, quod vídimus óculis nostris, quod perspéximus, et manus nostræ contrectavérunt de verbo vitæ — et vita appáruit, et vídimus et testámur et annuntiámus vobis vitam ætérnam, quæ erat coram Patre et appáruit nobis — quod vídimus et audívimus, annuntiámus et vobis, ut et vos communiónem habeátis nobíscum. Commúnio autem nostra est cum Patre et cum Fílio eius Iesu Christo. Et hæc scríbimus nos, ut gáudium nostrum sit plenum. Et hæc est annuntiátio, quam audívimus ab eo et annuntiámus vobis, quóniam Deus lux est, et ténebræ in eo non sunt ullæ.

\noindent Si dixérimus quóniam communiónem habémus cum eo et in ténebris ambulámus, mentímur et non fácimus veritátem; si autem in luce ambulémus, sicut ipse est in luce, communiónem habémus ad ínvicem, et sanguis Iesu Fílii eius mundat nos ab omni peccáto.

\noindent Si dixérimus quóniam peccátum non habémus, nosmetípsos sedúcimus, et véritas in nobis non est. Si confiteámur peccáta nostra, fidélis est et iustus, ut remíttat nobis peccáta et emúndet nos ab omni iniustítia. Si dixérimus quóniam non peccávimus, mendácem fácimus eum, et verbum eius non est in nobis.

\noindent Filíoli mei, hæc scribo vobis, ut non peccétis. Sed si quis peccáverit, advocátum habémus ad Patrem, Iesum Christum iustum; et ipse est propitiátio pro peccátis nostris, non pro nostris autem tantum sed étiam pro totíus mundi.

\noindent Et in hoc cognóscimus quóniam nóvimus eum: si mandáta eius servémus.

\noindent \Vbardot{} Tu autem, Dómine, miserére nobis.
\noindent \Rbardot{} Deo grátias.

\vfill
\pagebreak

\pars{Responsorium 1.} \scriptura{\Vbardot{} Io. 13, 25; \textbf{H63}}

\vspace{-5mm}

\responsorium{I}{temporalia/resp-isteestioannesquisuprapectus-CROCHU.gtex}{}

\vfill
\pagebreak

\cuminitiali{}{temporalia/benedictio-solemn-unigenitus.gtex}

\vspace{7mm}

\pars{Lectio II.} \scriptura{Tract. 1, 1. 3: PL 35, 1978. 1980}

\noindent Ex Tractátibus sancti Augustíni epíscopi in Epístolam Ioánnis.

\noindent \emph{Quod erat ab inítio, quod audívimus et quod vídimus óculis nostris et manus nostræ tractavérunt de Verbo vitæ.} Quis est qui mánibus tractat Verbum, nisi quia \emph{Verbum caro factum est et habitávit in nobis?}

\noindent Hoc autem Verbum, quod caro factum est, ut mánibus tractarétur, cœpit esse caro ex vírgine María: sed non tunc cœpit Verbum, quia \emph{quod erat ab inítio} dixit. Vidéte si non attestátur Epístola sua Evangélio suo, ubi modo audístis. \emph{In princípio erat Verbum et Verbum erat apud Deum.}

\noindent Forte de Verbo vitæ sic quisque accípiat quasi locutiónem quandam de Christo, non ipsum corpus Christi, quod mánibus tractátum est. Vidéte quid sequátur: \emph{Et ipsa vita manifestáta est.} Christus ergo Verbum vitæ.

\noindent Et unde manifestáta est? Erat enim ab inítio, sed non erat manifestáta homínibus: manifestáta autem erat ángelis vidéntibus et tamquam pane suo cibántibus. Sed quid ait Scriptúra? \emph{Panem angelórum manducávit homo.}

\noindent Ergo manifestáta est ipsa Vita in carne; quia in manifestatióne pósita est, ut res, quæ solo corde vidéri potest, viderétur et óculis, ut corda sanáret. Solo enim corde vidétur Verbum: caro autem et óculis corporálibus vidétur. Erat unde viderémus carnem, sed non erat unde viderémus Verbum: factum est Verbum caro, quam vidére possémus, et sanarétur in nobis unde Verbum viderémus.

\noindent \Vbardot{} Tu autem, Dómine, miserére nobis.
\noindent \Rbardot{} Deo grátias.

\vfill
\pagebreak

\pars{Responsorium 2.} \scriptura{\Rbardot{} Sir. 15, 5 \Vbardot{} Ier. 1, 9; \textbf{H63}}

\vspace{-5mm}

\responsorium{I}{temporalia/resp-inmedioecclesiae-CROCHU.gtex}{}

\vfill
\pagebreak

\cuminitiali{}{temporalia/benedictio-solemn-spiritus.gtex}

\vspace{7mm}

\pars{Lectio III.}

\noindent \emph{Et testes,} inquit, \emph{sumus; et annuntiámus vobis vitam ætérnam, quæ erat apud Patrem et manifestáta est in nobis,} hoc est: manifestáta est inter nos; quod apértius dicerétur, manifestáta est nobis.

\noindent \emph{Quæ ergo vídimus et audívimus, nuntiámus vobis.} Inténdat cáritas vestra: \emph{quæ ergo vídimus et audívimus, nuntiámus vobis.} Illi vidérunt ipsum Dóminum præséntem in carne, et audiérunt verba ex ore Dómini, et annuntiavérunt nobis. Et nos ergo audívimus, sed non vídimus.

\noindent Minus ergo sumus felíces quam illi qui vidérunt et audiérunt? Et quómodo adiúngit: \emph{Ut et vos societátem habeátis nobíscum?} Illi vidérunt, nos non vídimus, et tamen sócii sumus; quia fidem commúnem tenémus.

\noindent \emph{Et socíetas nostra sit cum Deo Patre et Iesu Christo Fílio eius. Et hæc,} inquit, \emph{scríbimus vobis, ut gáudium vestrum sit plenum.} Plenum gáudium dicit in ipsa societáte, in ipsa caritáte, in ipsa unitáte.

\noindent \Vbardot{} Tu autem, Dómine, miserére nobis.
\noindent \Rbardot{} Deo grátias.

\vfill
\pagebreak

\pars{Responsorium 3.} \scriptura{\Rbardot{} Mt. 12, 18 \Vbardot{} Io. 21, 20; \textbf{H62}}

\vspace{-5mm}

\responsorium{VII}{temporalia/resp-eccepuermeus-CROCHU-cumdox.gtex}{}

\vfill
\pagebreak}
\newcommand{\hymnuslaudes}{\pars{Hymnus.}

\vspace{-5mm}

\antiphona{IV}{temporalia/hym-CohorsBeata.gtex}}
\newcommand{\laudes}{\pars{Psalmus 1.}

\vspace{-4mm}

\antiphona{I d}{temporalia/ant-ioannesapostolusetevangelista.gtex}

%\vspace{-2mm}

\scriptura{Psalmus 62.}

\vspace{-1mm}

\initiumpsalmi{temporalia/ps62-initium-i-d-auto.gtex}

\input{temporalia/ps62-i-d.tex} \Abardot{}

\vfill
\pagebreak

\pars{Psalmus 2.} \scriptura{\textbf{H62}}

\vspace{-4mm}

\antiphona{I g}{temporalia/ant-isteestioannescuichristus.gtex}

\scriptura{Canticum trium puerorum, Dan. 3, 57-88 et 56}

\initiumpsalmi{temporalia/dan3-initium-i-g-auto.gtex}

\input{temporalia/dan3-i-g-sinedox.tex}

\rubrica{Hic non dicitur Gloria Patri, neque Amen.}

\vfill

\vspace{-6mm}

\antiphona{}{temporalia/ant-isteestioannescuichristus.gtex} % repeat the antiphon - new page

\vfill
\pagebreak

\pars{Psalmus 3.} \scriptura{Io. 21, 7}

\vspace{-4mm}

\antiphona{I g}{temporalia/ant-dixitdiscipulusille.gtex}

\scriptura{Psalmus 149.}

\initiumpsalmi{temporalia/ps149-initium-i-g-auto.gtex}

\input{temporalia/ps149-i-g.tex} \Abardot{}

\vfill
\pagebreak}
\newcommand{\lectiobrevis}{\pars{Lectio Brevis.} \scriptura{Ac. 4, 19-20}

\noindent Petrus et Ioánnes respondéntes dixérunt ad eos: «Si iustum est in conspéctu Dei vos pótius audíre quam Deum, iudicáte; non enim póssumus nos, quæ vídimus et audívimus, non loqui».}
\newcommand{\responsoriumbreve}{\pars{Responsorium breve.} \scriptura{Ps. 44, 17-18}

\cuminitiali{VI}{temporalia/resp-constitueseos.gtex}}
\newcommand{\oratio}{\pars{Oratio.}

\noindent Deus, qui per beátum apóstolum Ioánnem Verbi tui nobis arcána reserásti, præsta, quǽsumus, ut, quod ille nostris áuribus excellénter infúdit, intellegéntiæ competéntis eruditióne capiámus.

\noindent Per Dóminum nostrum Iesum Christum, Fílium tuum, qui tecum vivit et regnat in unitáte Spíritus Sancti, Deus, per ómnia sǽcula sæculórum.  Amen.

\noindent \Rbardot{} Amen.}
\newcommand{\benedictus}{\pars{Canticum Zachariæ.} \scriptura{Io. 1, 14}

\vspace{-4mm}

{
\grechangedim{interwordspacetext}{0.18 cm plus 0.15 cm minus 0.05 cm}{scalable}%
\antiphona{V g}{temporalia/ant-verbumcarofactumest.gtex}
\grechangedim{interwordspacetext}{0.22 cm plus 0.15 cm minus 0.05 cm}{scalable}%
}

\vspace{-1mm}

\scriptura{Lc. 1, 68-79}

\vspace{-2mm}

\cantusSineNeumas
\initiumpsalmi{temporalia/benedictus-initium-vsoll-g-auto.gtex}

%\vspace{-1.5mm}

\input{temporalia/benedictus-vsoll-g.tex} \Abardot{}}
\newcommand{\preces}{\noindent Superædificáti, fratres, \gredagger{} super fundaméntum Apostolórum, \grestar{} orémus Patrem omnipoténtem pro pópulo sancto eius dicéntes:

\Rbardot{} Recordáre Ecclésiæ tuæ, Dómine.

\noindent Pater, qui voluísti Fílium tuum, a mórtuis suscitátum, prius Apóstolis fíeri maniféstum, \grestar{} præsta, ut eius testes simus usque ad últimum terræ.

\Rbardot{} Recordáre Ecclésiæ tuæ, Dómine.

\noindent Qui Fílium tuum misísti in mundum evangelizáre paupéribus, \grestar{} da, ut omni creatúræ Evangélium prædicémus.

\Rbardot{} Recordáre Ecclésiæ tuæ, Dómine.

\noindent Qui tuum misísti Fílium semináre semen verbi, \grestar{} concéde, ut, verbum seminántes cum labóre, fruges metámus in gáudio.

\Rbardot{} Recordáre Ecclésiæ tuæ, Dómine.

\noindent Qui Fílium tuum misísti reconciliáre tibi mundum per sánguinem suum, \grestar{} tríbue, ut ad reconciliatiónem omnes cooperémur.

\Rbardot{} Recordáre Ecclésiæ tuæ, Dómine.}
\newcommand{\magnificat}{\pars{Canticum B. Mariæ V.} \scriptura{Io. 21, 23; \textbf{H62}}

\vspace{-4mm}

\antiphona{I g}{temporalia/ant-exiitsermointerfratres.gtex}

%\vspace{-2mm}

\scriptura{Lc. 1, 46-55}

\vspace{-2mm}

\cantusSineNeumas
\initiumpsalmi{temporalia/magnificat-initium-isoll-g.gtex}

%\vspace{-1.5mm}

\input{temporalia/magnificat-isoll-g.tex}

\vfill

\antiphona{}{temporalia/ant-exiitsermointerfratres.gtex}}
\newcommand{\capitulum}{\pars{Capitulum.} \scriptura{Eccli. 15, 1-2}

\grechangedim{interwordspacetext}{0.12 cm plus 0.15 cm minus 0.05 cm}{scalable}%
\cuminitiali{}{temporalia/capitulum-QuiTimet.gtex}
\grechangedim{interwordspacetext}{0.22 cm plus 0.15 cm minus 0.05 cm}{scalable}
}
\newcommand{\responsoriumbrevevesp}{\pars{Responsorium breve.} \scriptura{Ps. 44, 17-18}

\cuminitiali{VI}{temporalia/resp-constitueseos.gtex}}
\newcommand{\vespversus}{\pars{Versus.}

\noindent \Vbardot{} Valde honorándus est beátus Ioánnes.

\noindent \Rbardot{} Qui supra pectus Dómini in cœna recúbuit.}
\newcommand{\oratioVesperas}{\cuminitiali{}{temporalia/oratioioannes.gtex}}
% LuaLaTeX

\documentclass[a4paper, twoside, 12pt]{article}
\usepackage[latin]{babel}
%\usepackage[landscape, left=3cm, right=1.5cm, top=2cm, bottom=1cm]{geometry} % okraje stranky
%\usepackage[landscape, a4paper, mag=1166, truedimen, left=2cm, right=1.5cm, top=1.6cm, bottom=0.95cm]{geometry} % okraje stranky
\usepackage[landscape, a4paper, mag=1400, truedimen, left=0.5cm, right=0.5cm, top=0.5cm, bottom=0.5cm]{geometry} % okraje stranky

\usepackage{fontspec}
\setmainfont[FeatureFile={junicode.fea}, Ligatures={Common, TeX}, RawFeature=+fixi]{Junicode}
%\setmainfont{Junicode}

% shortcut for Junicode without ligatures (for the Czech texts)
\newfontfamily\nlfont[FeatureFile={junicode.fea}, Ligatures={Common, TeX}, RawFeature=+fixi]{Junicode}

% Hebrew font:
% http://scripts.sil.org/cms/scripts/page.php?site_id=nrsi&id=SILHebrUnic2
\newfontfamily\hebfont[Scale=1]{Ezra SIL}

\usepackage{multicol}
\usepackage{color}
\usepackage{lettrine}
\usepackage{fancyhdr}

% usual packages loading:
\usepackage{luatextra}
\usepackage{graphicx} % support the \includegraphics command and options
\usepackage{gregoriotex} % for gregorio score inclusion
\usepackage{gregoriosyms}
\usepackage{wrapfig} % figures wrapped by the text
\usepackage{parcolumns}
\usepackage[contents={},opacity=1,scale=1,color=black]{background}
\usepackage{tikzpagenodes}
\usepackage{calc}
\usepackage{longtable}
\usetikzlibrary{calc}

\setlength{\headheight}{14.5pt}

% Commands used to produce a typical "Conventus" booklet

\newenvironment{titulusOfficii}{\begin{center}}{\end{center}}
\newcommand{\dies}[1]{#1

}
\newcommand{\nomenFesti}[1]{\textbf{\Large #1}

}
\newcommand{\celebratio}[1]{#1

}

\newcommand{\hora}[1]{%
\vspace{0.5cm}{\large \textbf{#1}}

\fancyhead[LE]{\thepage\ / #1}
\fancyhead[RO]{#1 / \thepage}
\addcontentsline{toc}{subsection}{#1}
}

% larger unit than a hora
\newcommand{\divisio}[1]{%
\begin{center}
{\Large \textsc{#1}}
\end{center}
\fancyhead[CO,CE]{#1}
\addcontentsline{toc}{section}{#1}
}

% a part of a hora, larger than pars
\newcommand{\subhora}[1]{
\begin{center}
{\large \textit{#1}}
\end{center}
%\fancyhead[CO,CE]{#1}
\addcontentsline{toc}{subsubsection}{#1}
}

% rubricated inline text
\newcommand{\rubricatum}[1]{\textit{#1}}

% standalone rubric
\newcommand{\rubrica}[1]{\vspace{3mm}\rubricatum{#1}}

\newcommand{\notitia}[1]{\textcolor{red}{#1}}

\newcommand{\scriptura}[1]{\hfill \small\textit{#1}}

\newcommand{\translatioCantus}[1]{\vspace{1mm}%
{\noindent\footnotesize \nlfont{#1}}}

% pruznejsi varianta nasledujiciho - umoznuje nastavit sirku sloupce
% s prekladem
\newcommand{\psalmusEtTranslatioB}[3]{
  \vspace{0.5cm}
  \begin{parcolumns}[colwidths={2=#3}, nofirstindent=true]{2}
    \colchunk{
      \input{#1}
    }

    \colchunk{
      \vspace{-0.5cm}
      {\footnotesize \nlfont
        \input{#2}
      }
    }
  \end{parcolumns}
}

\newcommand{\psalmusEtTranslatio}[2]{
  \psalmusEtTranslatioB{#1}{#2}{8.5cm}
}


\newcommand{\canticumMagnificatEtTranslatio}[1]{
  \psalmusEtTranslatioB{#1}{temporalia/extra-adventum-vespers/magnificat-boh.tex}{12cm}
}
\newcommand{\canticumBenedictusEtTranslatio}[1]{
  \psalmusEtTranslatioB{#1}{temporalia/extra-adventum-laudes/benedictus-boh.tex}{10.5cm}
}

% volne misto nad antifonami, kam si zpevaci dokresli neumy
\newcommand{\hicSuntNeumae}{\vspace{0.5cm}}

% prepinani mista mezi notovymi osnovami: pro neumovane a neneumovane zpevy
\newcommand{\cantusCumNeumis}{
  \setgrefactor{17}
  \global\advance\grespaceabovelines by 5mm%
}
\newcommand{\cantusSineNeumas}{
  \setgrefactor{17}
  \global\advance\grespaceabovelines by -5mm%
}

% znaky k umisteni nad inicialu zpevu
\newcommand{\superInitialam}[1]{\gresetfirstlineaboveinitial{\small {\textbf{#1}}}{\small {\textbf{#1}}}}

% pars officii, i.e. "oratio", ...
\newcommand{\pars}[1]{\textbf{#1}}

\newenvironment{psalmus}{
  \setlength{\parindent}{0pt}
  \setlength{\parskip}{5pt}
}{
  \setlength{\parindent}{10pt}
  \setlength{\parskip}{10pt}
}

%%%% Prejmenovat na latinske:
\newcommand{\nadpisZalmu}[1]{
  \hspace{2cm}\textbf{#1}\vspace{2mm}%
  \nopagebreak%

}

% mode, score, translation
\newcommand{\antiphona}[3]{%
\hicSuntNeumae
\superInitialam{#1}
\includescore{#2}

#3
}
 % Often used macros

\newcommand{\annusEditionis}{2021}

\def\hebinitial#1{%
\leavevmode{\newbox\hebbox\setbox\hebbox\hbox{\hebfont{#1}\hskip 1mm}\kern -\wd\hebbox\hbox{\hebfont{#1}\hskip 1mm}}%
}

%%%% Vicekrat opakovane kousky

\newcommand{\anteOrationem}{
  \rubrica{Ante Orationem, cantatur a Superiore:}

  \pars{Supplicatio Litaniæ.}

  \cuminitiali{}{temporalia/supplicatiolitaniae.gtex}

  \pars{Oratio Dominica.}

  \cuminitiali{}{temporalia/oratiodominica.gtex}

  \rubrica{Deinde dicitur ab Hebdomadario:}

  \cuminitiali{}{temporalia/dominusvobiscum-solemnis.gtex}

  \rubrica{In choro monialium loco Dominus vobiscum dicitur:}

  \sineinitiali{temporalia/domineexaudi.gtex}
}

\setlength{\columnsep}{30pt} % prostor mezi sloupci

%%%%%%%%%%%%%%%%%%%%%%%%%%%%%%%%%%%%%%%%%%%%%%%%%%%%%%%%%%%%%%%%%%%%%%%%%%%%%%%%%%%%%%%%%%%%%%%%%%%%%%%%%%%%%
\begin{document}

% Here we set the space around the initial.
% Please report to http://home.gna.org/gregorio/gregoriotex/details for more details and options
\grechangedim{afterinitialshift}{2.2mm}{scalable}
\grechangedim{beforeinitialshift}{2.2mm}{scalable}
\grechangedim{interwordspacetext}{0.22 cm plus 0.15 cm minus 0.05 cm}{scalable}%
\grechangedim{annotationraise}{-0.2cm}{scalable}

% Here we set the initial font. Change 38 if you want a bigger initial.
% Emit the initials in red.
\grechangestyle{initial}{\color{red}\fontsize{38}{38}\selectfont}

\pagestyle{empty}

%%%% Titulni stranka
\begin{titulusOfficii}
\titulus
\end{titulusOfficii}

\vfill

\begin{center}
%Ad usum et secundum consuetudines chori \guillemotright{}Conventus Choralis\guillemotleft.

%Editio Sancti Wolfgangi \annusEditionis
\end{center}

\scriptura{}

\pars{}

\pagebreak

\renewcommand{\headrulewidth}{0pt} % no horiz. rule at the header
\fancyhf{}
\pagestyle{fancy}

\cantusSineNeumas

\pars{} \scriptura{}

\ifx\sinematutinum\undefined
\hora{Ad Matutinum.} %%%%%%%%%%%%%%%%%%%%%%%%%%%%%%%%%%%%%%%%%%%%%%%%%%%%%

\vspace{2mm}

\cuminitiali{}{temporalia/dominelabiamea.gtex}

\vfill
%\pagebreak

\vspace{2mm}

\ifx\invitatorium\undefined
\pars{Invitatorium.}

\vspace{-2mm}

\antiphona{E}{temporalia/inv-christusnatusest-simplex.gtex}
\else
\invitatorium
\fi

\vfill
\pagebreak

\ifx\hymnusmatutinum\undefined
\pars{Hymnus.}

{
\grechangedim{interwordspacetext}{0.10 cm plus 0.15 cm minus 0.05 cm}{scalable}%
\antiphona{IV}{temporalia/hym-CandorAEternae-simplex.gtex}
\grechangedim{interwordspacetext}{0.22 cm plus 0.15 cm minus 0.05 cm}{scalable}%
}

\vspace{-3mm}
\else
\hymnusmatutinum
\fi

\vfill
\pagebreak

\matutinum

\ifx\postoctavam\undefined
% Te Deum

\vspace{-5mm}

\ifx\tedeumsolemnis\undefined
\ifx\tedeumsimplex\undefined
\ifx\tedeummonasticum\undefined
{
\pars{Hymnus Ambrosianus} \scriptura{Alio modo, iuxta morem Romanum}

\vspace{-2mm}

\grechangedim{interwordspacetext}{0.26 cm plus 0.15 cm minus 0.05 cm}{scalable}%
\cuminitiali{III}{temporalia/tedeum-romanum-gn.gtex}
\grechangedim{interwordspacetext}{0.22 cm plus 0.15 cm minus 0.05 cm}{scalable}%
}
\else
{
\pars{Hymnus Ambrosianus} \scriptura{Tonus Monasticus}

\vspace{-2mm}

\grechangedim{interwordspacetext}{0.26 cm plus 0.15 cm minus 0.05 cm}{scalable}%
\cuminitiali{III}{temporalia/tedeum-monasticum-am34.gtex}
\grechangedim{interwordspacetext}{0.22 cm plus 0.15 cm minus 0.05 cm}{scalable}%
}
\fi
\else
{
\pars{Hymnus Ambrosianus} \scriptura{Tonus Simplex}

\vspace{-2mm}

\grechangedim{interwordspacetext}{0.30 cm plus 0.15 cm minus 0.05 cm}{scalable}%
\cuminitiali{III}{temporalia/tedeum-simplex-gn.gtex}
\grechangedim{interwordspacetext}{0.22 cm plus 0.15 cm minus 0.05 cm}{scalable}%
}
\fi
\else
{
\pars{Hymnus Ambrosianus} \scriptura{Tonus Solemnis}

\vspace{-2mm}

\grechangedim{interwordspacetext}{0.26 cm plus 0.15 cm minus 0.05 cm}{scalable}%
\cuminitiali{III}{temporalia/tedeum-solemnis-gn.gtex}
\grechangedim{interwordspacetext}{0.22 cm plus 0.15 cm minus 0.05 cm}{scalable}%
}
\fi

\vfill
\pagebreak
\fi

\rubrica{Reliqua omittuntur, nisi Laudes separandæ sint.}

\sineinitiali{temporalia/domineexaudi.gtex}

\vfill

\oratio

\vfill

\noindent \Vbardot{} Dómine, exáudi oratiónem meam.
\Rbardot{} Et clamor meus ad te véniat.

\vfill

% Nocturnale Romanum 2002, p. LXXVI Benedicamus Domino seems to match
% the one from Solemn Laudes.
\cuminitiali{V}{temporalia/benedicamus-solemnis-laud.gtex}

\vfill

\noindent \Vbardot{} Fidélium ánimæ per misericórdiam Dei requiéscant in pace.
\Rbardot{} Amen.

\vfill
\pagebreak
\fi

\ifx\sinelaudes\undefined
\hora{Ad Laudes.} %%%%%%%%%%%%%%%%%%%%%%%%%%%%%%%%%%%%%%%%%%%%%%%%%%%%%

\cantusSineNeumas

\vspace{0.5cm}
\grechangedim{interwordspacetext}{0.18 cm plus 0.15 cm minus 0.05 cm}{scalable}%
\ifx\postoctavam\undefined
\cuminitiali{}{temporalia/deusinadiutorium-alter.gtex}
\else
\cuminitiali{}{temporalia/deusinadiutorium-communis.gtex}
\fi
\grechangedim{interwordspacetext}{0.22 cm plus 0.15 cm minus 0.05 cm}{scalable}%

\vfill
%\pagebreak

\ifx\hymnuslaudes\undefined
\pagebreak
\pars{Hymnus} \scriptura{Sedulius}

\grechangedim{interwordspacetext}{0.16 cm plus 0.15 cm minus 0.05 cm}{scalable}%
\cuminitiali{III}{temporalia/hym-ASolisOrtus.gtex}
\grechangedim{interwordspacetext}{0.22 cm plus 0.15 cm minus 0.05 cm}{scalable}%
\vspace{-3mm}
\else
\hymnuslaudes
\fi

\vfill
\pagebreak

\ifx\laudes\undefined
\pars{Psalmus 1.} \scriptura{Lc. 2, 8.11.13.18; \textbf{H50}}

\vspace{-4mm}

\antiphona{II D}{temporalia/ant-quemvidistis.gtex}

\vspace{-2mm}

\scriptura{Psalmus 62.}

\vspace{-1mm}

\initiumpsalmi{temporalia/ps62-initium-ii-D-auto.gtex}

\input{temporalia/ps62-ii-D.tex} \Abardot{}

\vfill
\pagebreak

\pars{Psalmus 2.} \scriptura{Lc. 2, 10.11; \textbf{H50}}

\vspace{-4mm}

\antiphona{VII d}{temporalia/ant-angelusadpastores.gtex}

\scriptura{Canticum trium puerorum, Dan. 3, 57-88 et 56}

\initiumpsalmi{temporalia/dan3-initium-vii-d-auto.gtex}

\input{temporalia/dan3-vii-d-sinedox.tex}

\rubrica{Hic non dicitur Gloria Patri, neque Amen.}

\vfill

\vspace{-6mm}

\antiphona{}{temporalia/ant-angelusadpastores.gtex} % repeat the antiphon - new page

\vfill
\pagebreak

\pars{Psalmus 3.} \scriptura{Is. 9, 6; \textbf{H51}}

\vspace{-4mm}

\antiphona{VIII G\textsuperscript{2}}{temporalia/ant-parvulusfilius.gtex}

\scriptura{Psalmus 149.}

\initiumpsalmi{temporalia/ps149-initium-viii-G2-auto.gtex}

\input{temporalia/ps149-viii-G2.tex} \Abardot{}

\vfill
\pagebreak
\else
\laudes
\fi

\lectiobrevis

\vfill

\ifx\responsoriumbreve\undefined
\pars{Responsorium breve.} \scriptura{Ps. 97, 2}

\cuminitiali{VI}{temporalia/resp-notumfecit.gtex}
\else
\responsoriumbreve
\fi

\vfill
\pagebreak

\benedictus

\vspace{-1cm}

\vfill
\pagebreak

\pars{Preces.}

\sineinitiali{}{temporalia/tonusprecumnovum.gtex}

\preces

\vfill

\pars{Oratio Dominica.}

\cuminitiali{}{temporalia/oratiodominicaalt.gtex}

\vfill
\pagebreak

\rubrica{vel:}

\pars{Deprecatio Gelasii}

\vspace{-5mm}

\grechangedim{interwordspacetext}{0.16 cm plus 0.15 cm minus 0.05 cm}{scalable}%
\antiphona{D\textsuperscript{1}}{temporalia/deprecatio4-propace.gtex}
\grechangedim{interwordspacetext}{0.22 cm plus 0.15 cm minus 0.05 cm}{scalable}%

\vfill

\pars{Oratio Dominica.}

\cuminitiali{D}{temporalia/oratiodominica-d.gtex}

\vfill
\pagebreak

% Oratio. %%%
\oratio

\vspace{-1mm}

\vfill

\ifx\commemoratio\undefined
\else
\commemoratio
\fi

\rubrica{Hebdomadarius dicit Dominus vobiscum, vel, absente sacerdote vel diacono, sic concluditur:}

\vspace{2mm}

\antiphona{C}{temporalia/dominusnosbenedicat.gtex}

\rubrica{Postea cantatur a cantore:}

\vspace{2mm}

\ifx\benedicamuslaudes\undefined
\ifx\postoctavam\undefined
\cuminitiali{II}{temporalia/benedicamus-solemnism-laud.gtex}
\else
\cuminitiali{}{temporalia/benedicamus-tempore-nativitatis.gtex}
\fi
\else
\benedicamuslaudes
\fi

\vspace{1mm}

\vfill
\pagebreak
\fi

\ifx\sinevesperas\undefined
\hora{Ad Vesperas.} %%%%%%%%%%%%%%%%%%%%%%%%%%%%%%%%%%%%%%%%%%%%%%%%%%%%%

\cantusSineNeumas

%\vspace{-2mm}
\grechangedim{interwordspacetext}{0.18 cm plus 0.15 cm minus 0.05 cm}{scalable}%
\ifx\postoctavam\undefined
\cuminitiali{}{temporalia/deusinadiutorium-solemnis.gtex}
\else
\cuminitiali{}{temporalia/deusinadiutorium-communis.gtex}
\fi
\grechangedim{interwordspacetext}{0.22 cm plus 0.15 cm minus 0.05 cm}{scalable}%

\vfill
%\pagebreak

\vspace{-2mm}

\ifx\vesperas\undefined
\pars{Psalmus 1.} \scriptura{Ps. 109, 3; \textbf{H52}}

\vspace{-5mm}

\antiphona{I g}{temporalia/ant-tecumprincipium.gtex}

\scriptura{Psalmus 109.}

\initiumpsalmi{temporalia/ps109-initium-i-g-auto.gtex}

\vspace{-1.5mm}

\input{temporalia/ps109-i-g.tex} \Abardot{}

\vfill
\pagebreak

\pars{Psalmus 2.} \scriptura{Ps. 110, 9; \textbf{H52}}

\vspace{-4mm}

\antiphona{VII a}{temporalia/ant-redemptionemmisit.gtex}

\scriptura{Psalmus 110.}

\initiumpsalmi{temporalia/ps110-initium-vii-a-auto.gtex}

\input{temporalia/ps110-vii-a.tex} \Abardot{}

\vfill
\pagebreak

\pars{Psalmus 3.} \scriptura{Ps. 111, 4; \textbf{H52}}

\vspace{-4mm}

\antiphona{VII d}{temporalia/ant-exortumest.gtex}

\scriptura{Psalmus 111.}

\initiumpsalmi{temporalia/ps111-initium-vii-d-auto.gtex}

\input{temporalia/ps111-vii-d.tex} \Abardot{}

\vfill
\pagebreak

\ifx\impar\undefined
\pars{Psalmus 4.} \scriptura{Ps. 131, 11; \textbf{H52}}

\vspace{-4mm}

\antiphona{VIII G}{temporalia/ant-defructuventris.gtex}

\scriptura{Psalmus 131.}

\initiumpsalmi{temporalia/ps131-initium-viii-G-auto.gtex}

\input{temporalia/ps131-viii-G.tex}

\vfill

\antiphona{}{temporalia/ant-defructuventris.gtex}
\else
\pars{Psalmus 4.} \scriptura{Ps. 129, 7; \textbf{H52}}

\vspace{-4mm}

\antiphona{II* b}{temporalia/ant-apuddominum.gtex}

\scriptura{Psalmus 129.}

\initiumpsalmi{temporalia/ps129-initium-ii_-B-auto.gtex}

\input{temporalia/ps129-ii_-B.tex} \Abardot{}
\fi

\vfill
\pagebreak
\else
\vesperas
\fi

\ifx\capitulum\undefined
\pars{Capitulum.} \scriptura{Hebr. 1, 1-2}

\grechangedim{interwordspacetext}{0.12 cm plus 0.15 cm minus 0.05 cm}{scalable}%
\cuminitiali{}{temporalia/capitulum-Multifariam.gtex}
\grechangedim{interwordspacetext}{0.22 cm plus 0.15 cm minus 0.05 cm}{scalable}
\else
\capitulum
\fi

\vfill

\ifx\responsoriumbrevevesp\undefined
\pars{Responsorium breve.} \scriptura{Io. 1, 14}

\cuminitiali{VI}{temporalia/resp-verbumcaro-simplex.gtex}
\else
\responsoriumbrevevesp
\fi

\vfill
\pagebreak

\ifx\hymnusvesperas\undefined
\pars{Hymnus}

\cuminitiali{I}{temporalia/hym-ChristeRedemptor.gtex}
\else
\hymnusvesperas
\fi
\vspace{-3mm}

\vfill
%\pagebreak

\ifx\vespversus\undefined
\pars{Versus.} \scriptura{Ps. 97, 2}

% Versus. %%%
\sineinitiali{temporalia/versus-notumfecit-communis.gtex}
\else
\vespversus
\fi

\vfill
\pagebreak

\magnificat

\vfill
\pagebreak

\anteOrationem

\pagebreak

% Oratio. %%%
\ifx\oratioVesperas\undefined
\cuminitiali{}{temporalia/oratio.gtex}
\else
\oratioVesperas
\fi

\vspace{-1mm}

\vfill

\rubrica{Hebdomadarius dicit iterum Dominus vobiscum, vel cantor dicit:}

\vspace{2mm}

\sineinitiali{temporalia/domineexaudi.gtex}

\rubrica{Postea cantatur a cantore:}

\vspace{2mm}

\ifx\postoctavam\undefined
\cuminitiali{II}{temporalia/benedicamus-solemnism-2vesp.gtex}
\else
\cuminitiali{I}{temporalia/benedicamus-feria-vesperae.gtex}
\fi

\vspace{1mm}
\fi

\end{document}

