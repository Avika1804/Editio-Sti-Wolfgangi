\newcommand{\titulus}{\dies{Die 29. Decembris.}
\nomenFesti{Die Quinta post Nativitatem.}}
\newcommand{\impar}{Impar}
\newcommand{\tedeummonasticum}{Monasticum}
\newcommand{\matutinum}{\pars{Psalmus 1.}

\vspace{-4mm}

\antiphona{VIII G}{temporalia/ant-dominusvirtutum.gtex}

%\vspace{-5mm}

\scriptura{Ps. 45}

%\vspace{-2mm}

\initiumpsalmi{temporalia/ps45-initium-viii-G-auto.gtex}

\input{temporalia/ps45-viii-G.tex} \Abardot{}

\vfill
\pagebreak

\pars{Psalmus 2.} \scriptura{Ier. 27, 7}

\vspace{-4mm}

\antiphona{III b}{temporalia/ant-orieturdiebus.gtex}

%\vspace{-5mm}

\scriptura{Ps. 71, 1-11}

\initiumpsalmi{temporalia/ps71i-initium-iii-b-auto.gtex}

\input{temporalia/ps71i-iii-b.tex} \Abardot{}

\vfill
\pagebreak

\pars{Psalmus 3.}

\vspace{-4mm}

\antiphona{II* a}{temporalia/ant-benedicenturinipso.gtex}

%\vspace{-2mm}

\scriptura{Ps. 71, 12-19}

%\vspace{-2mm}

\initiumpsalmi{temporalia/ps71ii-initium-ii_-a-auto.gtex}

\input{temporalia/ps71ii-ii_-a.tex} \Abardot{}

\vfill
\pagebreak

\pars{Versus.}

\noindent \Vbardot{} Vidéntes pastóres cognovérunt de verbo.

\noindent \Rbardot{} Quod dictum erat illis de Púero.

\vspace{5mm}

\sineinitiali{temporalia/oratiodominica-mat.gtex}

\vspace{5mm}

\pars{Absolutio.}

\cuminitiali{}{temporalia/absolutio-exaudi.gtex}

\vfill
\pagebreak

\cuminitiali{}{temporalia/benedictio-solemn-benedictione.gtex}

\vspace{7mm}

\pars{Lectio I.} \scriptura{Col. 1, 1-14}

\noindent Incipit Epístola beáti Pauli apóstoli ad Colossénses.

\noindent Paulus apóstolus Christi Iesu per voluntátem Dei et Timótheus frater his, qui sunt Colóssis, sanctis et fidélibus frátribus in Christo: grátia vobis et pax a Deo Patre nostro.

\noindent Grátias ágimus Deo Patri Dómini nostri Iesu Christi semper pro vobis orántes, audiéntes fidem vestram in Christo Iesu et dilectiónem, quam habétis in sanctos omnes, propter spem, quæ repósita est vobis in cælis, quam ante audístis in verbo veritátis evangélii, quod pervénit ad vos, sicut et in univérso mundo est fructíficans et crescens sicut et in vobis ex ea die, qua audístis et cognovístis grátiam Dei in veritáte; sicut didicístis ab Epáphra caríssimo consérvo nostro, qui est fidélis pro nobis miníster Christi, qui étiam manifestávit nobis dilectiónem vestram in Spíritu.

\noindent Ideo et nos, ex qua die audívimus, non cessámus pro vobis orántes et postulántes, ut impleámini agnitióne voluntátis eius in omni sapiéntia et intelléctu spiritáli, ut ambulétis digne Dómino per ómnia placéntes, in omni ópere bono fructificántes et crescéntes in sciéntia Dei, in omni virtúte confortáti secúndum poténtiam claritátis eius in omnem patiéntiam et longanimitátem, cum gáudio grátias agéntes Patri, qui idóneos vos fecit in partem sortis sanctórum in lúmine; qui erípuit nos de potestáte tenebrárum et tránstulit in regnum Fílii dilectiónis suæ, in quo habémus redemptiónem, remissiónem peccatórum.

\noindent \Vbardot{} Tu autem, Dómine, miserére nobis.
\noindent \Rbardot{} Deo grátias.

\vfill
\pagebreak

\pars{Responsorium 1.} \scriptura{\Rbardot{} Ps. 117, 26-27 \Vbardot{} ibid., 22; \textbf{H49}}

\vspace{-5mm}

\responsorium{V}{temporalia/resp-benedictusquivenit-CROCHU.gtex}{}

\vfill
\pagebreak

\cuminitiali{}{temporalia/benedictio-solemn-unigenitus.gtex}

\vspace{7mm}

\pars{Lectio II.} \scriptura{Sermo 1 in Epiphania Domini, 1-2: PL 133, 141-143}

\noindent Ex Sermónibus sancti Bernárdi abbátis.

\noindent \emph{Appáruit benígnitas et humánitas Salvatóris nostri Dei.} Grátias Deo, per quem sic abúndat consolátio nostra in hac peregrinatióne, in hoc exsílio, in hac miséria.

\noindent Priúsquam apparéret humánitas, latébat benígnitas; síquidem et prius erat: nam et misericórdia Dómini ab ætérno est. Sed unde tanta agnósci póterat? Promittebátur, sed non sentiebátur; unde et a multis non credebátur. \emph{Multifárie} quippe \emph{multísque modis loquebátur Dóminus in prophétis. Ego,} ínquiens, \emph{cógito cogitatiónes pacis et non afflictiónis.} Sed quid respondébat homo, afflictiónem séntiens, pacem nésciens? Quoúsque dícitis, \emph{Pax, pax, et non est pax?} Propter hoc \emph{ángeli pacis amáre flebant, dicéntes: Dómine, quis crédidit audítui nostro?} Sed nunc credant hómines vel vísui suo, quia \emph{testimónia Dei credibília facta sunt nimis.} Ut enim nec turbátum quidem óculum láteat, \emph{in sole pósuit tabernáculum suum}.

\noindent Ecce pax non promíssa, sed missa; non diláta, sed data; non prophetáta, sed præsentáta. Ecce quasi saccum plenum misericórdia sua Deus Pater misit in terram; saccum, inquam, in passióne concidéndum, ut effundátur quod in eo latet prétium nostrum; saccum útique, etsi parvum, sed plenum.

\noindent \Vbardot{} Tu autem, Dómine, miserére nobis.
\noindent \Rbardot{} Deo grátias.

\vfill
\pagebreak

\pars{Responsorium 2.} \scriptura{\Vbardot{} Lc. 1, 28; \textbf{H47}}

\vspace{-5mm}

\responsorium{VII}{temporalia/resp-beatadeigenitrix-CROCHU.gtex}{}

\vfill
\pagebreak

\cuminitiali{}{temporalia/benedictio-solemn-spiritus.gtex}

\vspace{7mm}

\pars{Lectio III.}

\noindent \emph{Párvulus} síquidem \emph{datus est nobis,} sed \emph{in quo hábitat omnis plenitúdo divinitátis.} Postquam enim venit plenitúdo témporis, venit et plenitúdo divinitátis. Venit in carne, ut vel sic carnálibus exhiberétur, ei apparénte humanitáte benígnitas agnoscerétur. Ubi enim Dei innotéscit humánitas, iam benígnitas latére non potest. In quo enim magis commendáre póterat benignitátem suam, quam suscipiéndo carnem meam? Meam, inquam, non carnem Adam, id est non qualem ille hábuit ante culpam.

\noindent Quid tantópere decláret eius misericórdiam, quam quod ipsam suscépit misériam? Quid ita pietáte plenum, quam quod Dei Verbum propter nos factum est fenum? \emph{Dómine, quid est homo quia réputas eum? aut quid appónis erga eum cor tuum?} Hinc atténdat homo quanta sit cura eius Deo; hinc sciat quid de eo cógitet aut quid séntiat. Non intérroges, o homo, ea quæ páteris, sed quæ passus est ille. Quanti te fecit, ex his quæ pro te factus est, agnósce, ut appáreat tibi benígnitas eius ex humanitáte. Quanto enim minórem se fecit in humanitáte, tanto maiórem exhíbuit in bonitáte; et quanto pro me vílior, tanto mihi cárior est. \emph{Appáruit,} inquit Apóstolus, \emph{benígnitas et humánitas Salvatóris nostri Dei.} Magna plane et manifésta benígnitas Dei et humánitas! et magnum benignitátis indícium declarávit, qui humanitáti áddere nomen Dei curávit.

\noindent \Vbardot{} Tu autem, Dómine, miserére nobis.
\noindent \Rbardot{} Deo grátias.

\vfill
\pagebreak

\pars{Responsorium 3.} \scriptura{\Rbardot{} Io. 1, 29-30 \Vbardot{} ibid., 19; \textbf{H49}}

\vspace{-5mm}

\responsorium{VII}{temporalia/resp-ecceagnusdei-CROCHU-cumdox.gtex}{}

\vfill
\pagebreak}
\newcommand{\lectiobrevis}{\pars{Lectio Brevis.} \scriptura{Hebr. 1, 1-2}

\noindent Multifáriam et multis modis olim Deus locútus pátribus in prophétis, in novíssimis his diébus locútus est nobis in Fílio, quem constítuit herédem universórum, per quem fecit et sǽcula.}
\newcommand{\oratio}{\pars{Oratio.}

\noindent Omnípotens et invisíbilis Deus, qui tuæ lucis advéntu mundi ténebras effugásti, seréno vultu nos, quǽsumus, intuére, ut magnificéntiam nativitátis Unigéniti tui dignis præcóniis collaudémus.

\vfill

\pars{Pro commemoratione S. Thomæ Becket, Episcopi \& Martyris.} \scriptura{Io. 12, 25}

\vspace{-4mm}

\antiphona{III a}{temporalia/ant-quioditanimamsuam.gtex}

\vfill

\noindent Deus, qui beáto Thomæ, mártyri, pro iustítia magno ánimo vitam profúndere tribuísti, da nobis, eius intercessióne, nostram pro Christo vitam in hoc sǽculo abnegáre, ut eam in cælo inveníre possímus.

\noindent Qui tecum vivit et regnat in unitáte Spíritus Sancti, Deus, per ómnia sǽcula sæculórum.

\noindent \Rbardot{} Amen.}
\newcommand{\benedictus}{\pars{Canticum Zachariæ.} \scriptura{Lc. 2, 15; \textbf{H54}}

\vspace{-4mm}

{
\grechangedim{interwordspacetext}{0.18 cm plus 0.15 cm minus 0.05 cm}{scalable}%
\antiphona{VII d}{temporalia/ant-pastoresloquebantur.gtex}
\grechangedim{interwordspacetext}{0.22 cm plus 0.15 cm minus 0.05 cm}{scalable}%
}

%\vspace{-3mm}

\scriptura{Lc. 1, 68-79}

%\vspace{-2mm}

\cantusSineNeumas
\initiumpsalmi{temporalia/benedictus-initium-vii-d-auto.gtex}

\vspace{-1.5mm}

\input{temporalia/benedictus-vii-d.tex}

\vfill

{
\grechangedim{interwordspacetext}{0.18 cm plus 0.15 cm minus 0.05 cm}{scalable}%
\antiphona{}{temporalia/ant-pastoresloquebantur.gtex}
\grechangedim{interwordspacetext}{0.22 cm plus 0.15 cm minus 0.05 cm}{scalable}%
}}
\newcommand{\preces}{\noindent Quia Deus in sua misericórdia Christum,~\gredagger{} príncipem pacis, nobis misit,~\grestar{} cum fidúcia clamémus:

\Rbardot{} Pax homínibus bonæ voluntátis.

\noindent Deus omnípotens, Pater Dómini nostri Iesu Christi, quo témpore Ecclésia salvíficum tuum célebrat amórem,~\grestar{} laudem nostram suscípere dignáre propítius.

\Rbardot{} Pax homínibus bonæ voluntátis.

\noindent Qui ab inítio victóriam tuam per Christum salvatórem nostrum homínibus promisísti,~\grestar{} Evangélii lúmine omnes hómines illustrári concéde.

\Rbardot{} Pax homínibus bonæ voluntátis.

\noindent In laudem Fílii tui, cuius diem cum gáudio Abraham prævídit, patriárchæ speravérunt, prophétæ nuntiavérunt et gentes desideravérunt,~\grestar{} fac ut omnis Israel salvus fiat.

\Rbardot{} Pax homínibus bonæ voluntátis.

\noindent Qui voluísti nativitátem Fílii tui a cælórum spirítibus prædicári atque ab Apóstolis, martýribus et fidélibus sæculórum ómnium collaudári,~\grestar{} pacem in terris largíre, quam nuntiavérunt ángeli.

\Rbardot{} Pax homínibus bonæ voluntátis.}
\newcommand{\magnificat}{\pars{Canticum B. Mariæ V.} \scriptura{Lc. 2, 11; \textbf{H51}}

\vspace{-6mm}

{
\grechangedim{interwordspacetext}{0.18 cm plus 0.15 cm minus 0.05 cm}{scalable}%
\antiphona{VI F}{temporalia/ant-gaudeamusomnesfideles.gtex}
\grechangedim{interwordspacetext}{0.22 cm plus 0.15 cm minus 0.05 cm}{scalable}%
}

\vspace{-2mm}

\scriptura{Lc. 1, 46-55}

\vspace{-2mm}

\cantusSineNeumas
\initiumpsalmi{temporalia/magnificat-initium-vi-F.gtex}

\vspace{-1.5mm}

\input{temporalia/magnificat-vi-F.tex} \Abardot{}}
% LuaLaTeX

\documentclass[a4paper, twoside, 12pt]{article}
\usepackage[latin]{babel}
%\usepackage[landscape, left=3cm, right=1.5cm, top=2cm, bottom=1cm]{geometry} % okraje stranky
%\usepackage[landscape, a4paper, mag=1166, truedimen, left=2cm, right=1.5cm, top=1.6cm, bottom=0.95cm]{geometry} % okraje stranky
\usepackage[landscape, a4paper, mag=1400, truedimen, left=0.5cm, right=0.5cm, top=0.5cm, bottom=0.5cm]{geometry} % okraje stranky

\usepackage{fontspec}
\setmainfont[FeatureFile={junicode.fea}, Ligatures={Common, TeX}, RawFeature=+fixi]{Junicode}
%\setmainfont{Junicode}

% shortcut for Junicode without ligatures (for the Czech texts)
\newfontfamily\nlfont[FeatureFile={junicode.fea}, Ligatures={Common, TeX}, RawFeature=+fixi]{Junicode}

% Hebrew font:
% http://scripts.sil.org/cms/scripts/page.php?site_id=nrsi&id=SILHebrUnic2
\newfontfamily\hebfont[Scale=1]{Ezra SIL}

\usepackage{multicol}
\usepackage{color}
\usepackage{lettrine}
\usepackage{fancyhdr}

% usual packages loading:
\usepackage{luatextra}
\usepackage{graphicx} % support the \includegraphics command and options
\usepackage{gregoriotex} % for gregorio score inclusion
\usepackage{gregoriosyms}
\usepackage{wrapfig} % figures wrapped by the text
\usepackage{parcolumns}
\usepackage[contents={},opacity=1,scale=1,color=black]{background}
\usepackage{tikzpagenodes}
\usepackage{calc}
\usepackage{longtable}
\usetikzlibrary{calc}

\setlength{\headheight}{14.5pt}

% Commands used to produce a typical "Conventus" booklet

\newenvironment{titulusOfficii}{\begin{center}}{\end{center}}
\newcommand{\dies}[1]{#1

}
\newcommand{\nomenFesti}[1]{\textbf{\Large #1}

}
\newcommand{\celebratio}[1]{#1

}

\newcommand{\hora}[1]{%
\vspace{0.5cm}{\large \textbf{#1}}

\fancyhead[LE]{\thepage\ / #1}
\fancyhead[RO]{#1 / \thepage}
\addcontentsline{toc}{subsection}{#1}
}

% larger unit than a hora
\newcommand{\divisio}[1]{%
\begin{center}
{\Large \textsc{#1}}
\end{center}
\fancyhead[CO,CE]{#1}
\addcontentsline{toc}{section}{#1}
}

% a part of a hora, larger than pars
\newcommand{\subhora}[1]{
\begin{center}
{\large \textit{#1}}
\end{center}
%\fancyhead[CO,CE]{#1}
\addcontentsline{toc}{subsubsection}{#1}
}

% rubricated inline text
\newcommand{\rubricatum}[1]{\textit{#1}}

% standalone rubric
\newcommand{\rubrica}[1]{\vspace{3mm}\rubricatum{#1}}

\newcommand{\notitia}[1]{\textcolor{red}{#1}}

\newcommand{\scriptura}[1]{\hfill \small\textit{#1}}

\newcommand{\translatioCantus}[1]{\vspace{1mm}%
{\noindent\footnotesize \nlfont{#1}}}

% pruznejsi varianta nasledujiciho - umoznuje nastavit sirku sloupce
% s prekladem
\newcommand{\psalmusEtTranslatioB}[3]{
  \vspace{0.5cm}
  \begin{parcolumns}[colwidths={2=#3}, nofirstindent=true]{2}
    \colchunk{
      \input{#1}
    }

    \colchunk{
      \vspace{-0.5cm}
      {\footnotesize \nlfont
        \input{#2}
      }
    }
  \end{parcolumns}
}

\newcommand{\psalmusEtTranslatio}[2]{
  \psalmusEtTranslatioB{#1}{#2}{8.5cm}
}


\newcommand{\canticumMagnificatEtTranslatio}[1]{
  \psalmusEtTranslatioB{#1}{temporalia/extra-adventum-vespers/magnificat-boh.tex}{12cm}
}
\newcommand{\canticumBenedictusEtTranslatio}[1]{
  \psalmusEtTranslatioB{#1}{temporalia/extra-adventum-laudes/benedictus-boh.tex}{10.5cm}
}

% volne misto nad antifonami, kam si zpevaci dokresli neumy
\newcommand{\hicSuntNeumae}{\vspace{0.5cm}}

% prepinani mista mezi notovymi osnovami: pro neumovane a neneumovane zpevy
\newcommand{\cantusCumNeumis}{
  \setgrefactor{17}
  \global\advance\grespaceabovelines by 5mm%
}
\newcommand{\cantusSineNeumas}{
  \setgrefactor{17}
  \global\advance\grespaceabovelines by -5mm%
}

% znaky k umisteni nad inicialu zpevu
\newcommand{\superInitialam}[1]{\gresetfirstlineaboveinitial{\small {\textbf{#1}}}{\small {\textbf{#1}}}}

% pars officii, i.e. "oratio", ...
\newcommand{\pars}[1]{\textbf{#1}}

\newenvironment{psalmus}{
  \setlength{\parindent}{0pt}
  \setlength{\parskip}{5pt}
}{
  \setlength{\parindent}{10pt}
  \setlength{\parskip}{10pt}
}

%%%% Prejmenovat na latinske:
\newcommand{\nadpisZalmu}[1]{
  \hspace{2cm}\textbf{#1}\vspace{2mm}%
  \nopagebreak%

}

% mode, score, translation
\newcommand{\antiphona}[3]{%
\hicSuntNeumae
\superInitialam{#1}
\includescore{#2}

#3
}
 % Often used macros

\newcommand{\annusEditionis}{2021}

\def\hebinitial#1{%
\leavevmode{\newbox\hebbox\setbox\hebbox\hbox{\hebfont{#1}\hskip 1mm}\kern -\wd\hebbox\hbox{\hebfont{#1}\hskip 1mm}}%
}

%%%% Vicekrat opakovane kousky

\newcommand{\anteOrationem}{
  \rubrica{Ante Orationem, cantatur a Superiore:}

  \pars{Supplicatio Litaniæ.}

  \cuminitiali{}{temporalia/supplicatiolitaniae.gtex}

  \pars{Oratio Dominica.}

  \cuminitiali{}{temporalia/oratiodominica.gtex}

  \rubrica{Deinde dicitur ab Hebdomadario:}

  \cuminitiali{}{temporalia/dominusvobiscum-solemnis.gtex}

  \rubrica{In choro monialium loco Dominus vobiscum dicitur:}

  \sineinitiali{temporalia/domineexaudi.gtex}
}

\setlength{\columnsep}{30pt} % prostor mezi sloupci

%%%%%%%%%%%%%%%%%%%%%%%%%%%%%%%%%%%%%%%%%%%%%%%%%%%%%%%%%%%%%%%%%%%%%%%%%%%%%%%%%%%%%%%%%%%%%%%%%%%%%%%%%%%%%
\begin{document}

% Here we set the space around the initial.
% Please report to http://home.gna.org/gregorio/gregoriotex/details for more details and options
\grechangedim{afterinitialshift}{2.2mm}{scalable}
\grechangedim{beforeinitialshift}{2.2mm}{scalable}
\grechangedim{interwordspacetext}{0.22 cm plus 0.15 cm minus 0.05 cm}{scalable}%
\grechangedim{annotationraise}{-0.2cm}{scalable}

% Here we set the initial font. Change 38 if you want a bigger initial.
% Emit the initials in red.
\grechangestyle{initial}{\color{red}\fontsize{38}{38}\selectfont}

\pagestyle{empty}

%%%% Titulni stranka
\begin{titulusOfficii}
\titulus
\end{titulusOfficii}

\vfill

\begin{center}
%Ad usum et secundum consuetudines chori \guillemotright{}Conventus Choralis\guillemotleft.

%Editio Sancti Wolfgangi \annusEditionis
\end{center}

\scriptura{}

\pars{}

\pagebreak

\renewcommand{\headrulewidth}{0pt} % no horiz. rule at the header
\fancyhf{}
\pagestyle{fancy}

\cantusSineNeumas

\pars{} \scriptura{}

\ifx\sinematutinum\undefined
\hora{Ad Matutinum.} %%%%%%%%%%%%%%%%%%%%%%%%%%%%%%%%%%%%%%%%%%%%%%%%%%%%%

\vspace{2mm}

\cuminitiali{}{temporalia/dominelabiamea.gtex}

\vfill
%\pagebreak

\vspace{2mm}

\ifx\invitatorium\undefined
\pars{Invitatorium.}

\vspace{-2mm}

\antiphona{E}{temporalia/inv-christusnatusest-simplex.gtex}
\else
\invitatorium
\fi

\vfill
\pagebreak

\ifx\hymnusmatutinum\undefined
\pars{Hymnus.}

{
\grechangedim{interwordspacetext}{0.10 cm plus 0.15 cm minus 0.05 cm}{scalable}%
\antiphona{IV}{temporalia/hym-CandorAEternae-simplex.gtex}
\grechangedim{interwordspacetext}{0.22 cm plus 0.15 cm minus 0.05 cm}{scalable}%
}

\vspace{-3mm}
\else
\hymnusmatutinum
\fi

\vfill
\pagebreak

\matutinum

\ifx\postoctavam\undefined
% Te Deum

\vspace{-5mm}

\ifx\tedeumsolemnis\undefined
\ifx\tedeumsimplex\undefined
\ifx\tedeummonasticum\undefined
{
\pars{Hymnus Ambrosianus} \scriptura{Alio modo, iuxta morem Romanum}

\vspace{-2mm}

\grechangedim{interwordspacetext}{0.26 cm plus 0.15 cm minus 0.05 cm}{scalable}%
\cuminitiali{III}{temporalia/tedeum-romanum-gn.gtex}
\grechangedim{interwordspacetext}{0.22 cm plus 0.15 cm minus 0.05 cm}{scalable}%
}
\else
{
\pars{Hymnus Ambrosianus} \scriptura{Tonus Monasticus}

\vspace{-2mm}

\grechangedim{interwordspacetext}{0.26 cm plus 0.15 cm minus 0.05 cm}{scalable}%
\cuminitiali{III}{temporalia/tedeum-monasticum-am34.gtex}
\grechangedim{interwordspacetext}{0.22 cm plus 0.15 cm minus 0.05 cm}{scalable}%
}
\fi
\else
{
\pars{Hymnus Ambrosianus} \scriptura{Tonus Simplex}

\vspace{-2mm}

\grechangedim{interwordspacetext}{0.30 cm plus 0.15 cm minus 0.05 cm}{scalable}%
\cuminitiali{III}{temporalia/tedeum-simplex-gn.gtex}
\grechangedim{interwordspacetext}{0.22 cm plus 0.15 cm minus 0.05 cm}{scalable}%
}
\fi
\else
{
\pars{Hymnus Ambrosianus} \scriptura{Tonus Solemnis}

\vspace{-2mm}

\grechangedim{interwordspacetext}{0.26 cm plus 0.15 cm minus 0.05 cm}{scalable}%
\cuminitiali{III}{temporalia/tedeum-solemnis-gn.gtex}
\grechangedim{interwordspacetext}{0.22 cm plus 0.15 cm minus 0.05 cm}{scalable}%
}
\fi

\vfill
\pagebreak
\fi

\rubrica{Reliqua omittuntur, nisi Laudes separandæ sint.}

\sineinitiali{temporalia/domineexaudi.gtex}

\vfill

\oratio

\vfill

\noindent \Vbardot{} Dómine, exáudi oratiónem meam.
\Rbardot{} Et clamor meus ad te véniat.

\vfill

% Nocturnale Romanum 2002, p. LXXVI Benedicamus Domino seems to match
% the one from Solemn Laudes.
\cuminitiali{V}{temporalia/benedicamus-solemnis-laud.gtex}

\vfill

\noindent \Vbardot{} Fidélium ánimæ per misericórdiam Dei requiéscant in pace.
\Rbardot{} Amen.

\vfill
\pagebreak
\fi

\ifx\sinelaudes\undefined
\hora{Ad Laudes.} %%%%%%%%%%%%%%%%%%%%%%%%%%%%%%%%%%%%%%%%%%%%%%%%%%%%%

\cantusSineNeumas

\vspace{0.5cm}
\grechangedim{interwordspacetext}{0.18 cm plus 0.15 cm minus 0.05 cm}{scalable}%
\ifx\postoctavam\undefined
\cuminitiali{}{temporalia/deusinadiutorium-alter.gtex}
\else
\cuminitiali{}{temporalia/deusinadiutorium-communis.gtex}
\fi
\grechangedim{interwordspacetext}{0.22 cm plus 0.15 cm minus 0.05 cm}{scalable}%

\vfill
%\pagebreak

\ifx\hymnuslaudes\undefined
\pagebreak
\pars{Hymnus} \scriptura{Sedulius}

\grechangedim{interwordspacetext}{0.16 cm plus 0.15 cm minus 0.05 cm}{scalable}%
\cuminitiali{III}{temporalia/hym-ASolisOrtus.gtex}
\grechangedim{interwordspacetext}{0.22 cm plus 0.15 cm minus 0.05 cm}{scalable}%
\vspace{-3mm}
\else
\hymnuslaudes
\fi

\vfill
\pagebreak

\ifx\laudes\undefined
\pars{Psalmus 1.} \scriptura{Lc. 2, 8.11.13.18; \textbf{H50}}

\vspace{-4mm}

\antiphona{II D}{temporalia/ant-quemvidistis.gtex}

\vspace{-2mm}

\scriptura{Psalmus 62.}

\vspace{-1mm}

\initiumpsalmi{temporalia/ps62-initium-ii-D-auto.gtex}

\input{temporalia/ps62-ii-D.tex} \Abardot{}

\vfill
\pagebreak

\pars{Psalmus 2.} \scriptura{Lc. 2, 10.11; \textbf{H50}}

\vspace{-4mm}

\antiphona{VII d}{temporalia/ant-angelusadpastores.gtex}

\scriptura{Canticum trium puerorum, Dan. 3, 57-88 et 56}

\initiumpsalmi{temporalia/dan3-initium-vii-d-auto.gtex}

\input{temporalia/dan3-vii-d-sinedox.tex}

\rubrica{Hic non dicitur Gloria Patri, neque Amen.}

\vfill

\vspace{-6mm}

\antiphona{}{temporalia/ant-angelusadpastores.gtex} % repeat the antiphon - new page

\vfill
\pagebreak

\pars{Psalmus 3.} \scriptura{Is. 9, 6; \textbf{H51}}

\vspace{-4mm}

\antiphona{VIII G\textsuperscript{2}}{temporalia/ant-parvulusfilius.gtex}

\scriptura{Psalmus 149.}

\initiumpsalmi{temporalia/ps149-initium-viii-G2-auto.gtex}

\input{temporalia/ps149-viii-G2.tex} \Abardot{}

\vfill
\pagebreak
\else
\laudes
\fi

\lectiobrevis

\vfill

\ifx\responsoriumbreve\undefined
\pars{Responsorium breve.} \scriptura{Ps. 97, 2}

\cuminitiali{VI}{temporalia/resp-notumfecit.gtex}
\else
\responsoriumbreve
\fi

\vfill
\pagebreak

\benedictus

\vspace{-1cm}

\vfill
\pagebreak

\pars{Preces.}

\sineinitiali{}{temporalia/tonusprecumnovum.gtex}

\preces

\vfill

\pars{Oratio Dominica.}

\cuminitiali{}{temporalia/oratiodominicaalt.gtex}

\vfill
\pagebreak

\rubrica{vel:}

\pars{Deprecatio Gelasii}

\vspace{-5mm}

\grechangedim{interwordspacetext}{0.16 cm plus 0.15 cm minus 0.05 cm}{scalable}%
\antiphona{D\textsuperscript{1}}{temporalia/deprecatio4-propace.gtex}
\grechangedim{interwordspacetext}{0.22 cm plus 0.15 cm minus 0.05 cm}{scalable}%

\vfill

\pars{Oratio Dominica.}

\cuminitiali{D}{temporalia/oratiodominica-d.gtex}

\vfill
\pagebreak

% Oratio. %%%
\oratio

\vspace{-1mm}

\vfill

\ifx\commemoratio\undefined
\else
\commemoratio
\fi

\rubrica{Hebdomadarius dicit Dominus vobiscum, vel, absente sacerdote vel diacono, sic concluditur:}

\vspace{2mm}

\antiphona{C}{temporalia/dominusnosbenedicat.gtex}

\rubrica{Postea cantatur a cantore:}

\vspace{2mm}

\ifx\benedicamuslaudes\undefined
\ifx\postoctavam\undefined
\cuminitiali{II}{temporalia/benedicamus-solemnism-laud.gtex}
\else
\cuminitiali{}{temporalia/benedicamus-tempore-nativitatis.gtex}
\fi
\else
\benedicamuslaudes
\fi

\vspace{1mm}

\vfill
\pagebreak
\fi

\ifx\sinevesperas\undefined
\hora{Ad Vesperas.} %%%%%%%%%%%%%%%%%%%%%%%%%%%%%%%%%%%%%%%%%%%%%%%%%%%%%

\cantusSineNeumas

%\vspace{-2mm}
\grechangedim{interwordspacetext}{0.18 cm plus 0.15 cm minus 0.05 cm}{scalable}%
\ifx\postoctavam\undefined
\cuminitiali{}{temporalia/deusinadiutorium-solemnis.gtex}
\else
\cuminitiali{}{temporalia/deusinadiutorium-communis.gtex}
\fi
\grechangedim{interwordspacetext}{0.22 cm plus 0.15 cm minus 0.05 cm}{scalable}%

\vfill
%\pagebreak

\vspace{-2mm}

\ifx\vesperas\undefined
\pars{Psalmus 1.} \scriptura{Ps. 109, 3; \textbf{H52}}

\vspace{-5mm}

\antiphona{I g}{temporalia/ant-tecumprincipium.gtex}

\scriptura{Psalmus 109.}

\initiumpsalmi{temporalia/ps109-initium-i-g-auto.gtex}

\vspace{-1.5mm}

\input{temporalia/ps109-i-g.tex} \Abardot{}

\vfill
\pagebreak

\pars{Psalmus 2.} \scriptura{Ps. 110, 9; \textbf{H52}}

\vspace{-4mm}

\antiphona{VII a}{temporalia/ant-redemptionemmisit.gtex}

\scriptura{Psalmus 110.}

\initiumpsalmi{temporalia/ps110-initium-vii-a-auto.gtex}

\input{temporalia/ps110-vii-a.tex} \Abardot{}

\vfill
\pagebreak

\pars{Psalmus 3.} \scriptura{Ps. 111, 4; \textbf{H52}}

\vspace{-4mm}

\antiphona{VII d}{temporalia/ant-exortumest.gtex}

\scriptura{Psalmus 111.}

\initiumpsalmi{temporalia/ps111-initium-vii-d-auto.gtex}

\input{temporalia/ps111-vii-d.tex} \Abardot{}

\vfill
\pagebreak

\ifx\impar\undefined
\pars{Psalmus 4.} \scriptura{Ps. 131, 11; \textbf{H52}}

\vspace{-4mm}

\antiphona{VIII G}{temporalia/ant-defructuventris.gtex}

\scriptura{Psalmus 131.}

\initiumpsalmi{temporalia/ps131-initium-viii-G-auto.gtex}

\input{temporalia/ps131-viii-G.tex}

\vfill

\antiphona{}{temporalia/ant-defructuventris.gtex}
\else
\pars{Psalmus 4.} \scriptura{Ps. 129, 7; \textbf{H52}}

\vspace{-4mm}

\antiphona{II* b}{temporalia/ant-apuddominum.gtex}

\scriptura{Psalmus 129.}

\initiumpsalmi{temporalia/ps129-initium-ii_-B-auto.gtex}

\input{temporalia/ps129-ii_-B.tex} \Abardot{}
\fi

\vfill
\pagebreak
\else
\vesperas
\fi

\ifx\capitulum\undefined
\pars{Capitulum.} \scriptura{Hebr. 1, 1-2}

\grechangedim{interwordspacetext}{0.12 cm plus 0.15 cm minus 0.05 cm}{scalable}%
\cuminitiali{}{temporalia/capitulum-Multifariam.gtex}
\grechangedim{interwordspacetext}{0.22 cm plus 0.15 cm minus 0.05 cm}{scalable}
\else
\capitulum
\fi

\vfill

\ifx\responsoriumbrevevesp\undefined
\pars{Responsorium breve.} \scriptura{Io. 1, 14}

\cuminitiali{VI}{temporalia/resp-verbumcaro-simplex.gtex}
\else
\responsoriumbrevevesp
\fi

\vfill
\pagebreak

\ifx\hymnusvesperas\undefined
\pars{Hymnus}

\cuminitiali{I}{temporalia/hym-ChristeRedemptor.gtex}
\else
\hymnusvesperas
\fi
\vspace{-3mm}

\vfill
%\pagebreak

\ifx\vespversus\undefined
\pars{Versus.} \scriptura{Ps. 97, 2}

% Versus. %%%
\sineinitiali{temporalia/versus-notumfecit-communis.gtex}
\else
\vespversus
\fi

\vfill
\pagebreak

\magnificat

\vfill
\pagebreak

\anteOrationem

\pagebreak

% Oratio. %%%
\ifx\oratioVesperas\undefined
\cuminitiali{}{temporalia/oratio.gtex}
\else
\oratioVesperas
\fi

\vspace{-1mm}

\vfill

\rubrica{Hebdomadarius dicit iterum Dominus vobiscum, vel cantor dicit:}

\vspace{2mm}

\sineinitiali{temporalia/domineexaudi.gtex}

\rubrica{Postea cantatur a cantore:}

\vspace{2mm}

\ifx\postoctavam\undefined
\cuminitiali{II}{temporalia/benedicamus-solemnism-2vesp.gtex}
\else
\cuminitiali{I}{temporalia/benedicamus-feria-vesperae.gtex}
\fi

\vspace{1mm}
\fi

\end{document}

