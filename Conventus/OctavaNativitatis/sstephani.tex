\newcommand{\titulus}{\dies{26. Decembris.}
\nomenFesti{S. Stephani, Protomartyris.}}
%\newcommand{\impar}{Impar}
\newcommand{\invitatorium}{\pars{Invitatorium.}

\vspace{-2mm}

\antiphona{IV}{temporalia/inv-christumnatumquibeatumhodie.gtex}}
\newcommand{\hymnusmatutinum}{\pars{Hymnus.}

\vspace{-5mm}

\antiphona{IV}{temporalia/hym-FestumCelebre.gtex}}
\newcommand{\matutinum}{\pars{Psalmus 1.} \scriptura{Ac. 7, 55; \textbf{H59}}

\vspace{-4mm}

\antiphona{I g}{temporalia/ant-cumautemessetstephanus.gtex}

\vspace{-2mm}

\scriptura{Ps. 2}

\vspace{-2mm}

\initiumpsalmi{temporalia/ps2-initium-i-g-auto.gtex}

\input{temporalia/ps2-i-g.tex}

\vfill

\antiphona{}{temporalia/ant-cumautemessetstephanus.gtex}

\vfill
\pagebreak

\pars{Psalmus 2.} \scriptura{Ac. 7, 58.59; \textbf{H57}}

\vspace{-4mm}

\antiphona{VIII G}{temporalia/ant-lapidaverunt.gtex}

%\vspace{-5mm}

\scriptura{Ps. 10}

\initiumpsalmi{temporalia/ps10-initium-viii-G-auto.gtex}

\input{temporalia/ps10-viii-G.tex} \Abardot{}

\vfill
\pagebreak

\pars{Psalmus 3.} \scriptura{Ac. 6, 10; \textbf{H56}}

\vspace{-4mm}

\antiphona{VIII c}{temporalia/ant-nemopoteratresistere.gtex}

%\vspace{-2mm}

\scriptura{Ps. 16}

%\vspace{-2mm}

\initiumpsalmi{temporalia/ps16-initium-viii-c-auto.gtex}

\input{temporalia/ps16-viii-c.tex}

\vfill

\antiphona{}{temporalia/ant-nemopoteratresistere.gtex}

\vfill
\pagebreak

\pars{Versus.} 

\noindent \Vbardot{} Tribulátio et angústia invenérunt me.

\noindent \Rbardot{} Mandáta tua meditátio mea est.

\vspace{5mm}

\sineinitiali{temporalia/oratiodominica-mat.gtex}

\vspace{5mm}

\pars{Absolutio.}

\cuminitiali{}{temporalia/absolutio-exaudi.gtex}

\vfill
\pagebreak

\cuminitiali{}{temporalia/benedictio-solemn-benedictione.gtex}

\vspace{7mm}

\pars{Lectio I.} \scriptura{Ac. 6, 8-15; 7, 1-2.44-59; 8, 1}

\noindent De libro Actuum Apostolórum.

\noindent Stéphanus plenus grátia et virtúte faciébat prodígia et signa magna in pópulo. Surrexérunt autem quidam de synagóga, quæ appellátur Libertinórum et Cyrenénsium et Alexandrinórum et eórum, qui erant a Cilícia et Asia, disputántes cum Stéphano, et non póterant resístere sapiéntiæ et Spirítui, quo loquebátur. Tunc submisérunt viros, qui dícerent: «Audívimus eum dicéntem verba blasphéma in Móysen et Deum»; et commovérunt plebem et senióres et scribas, et concurréntes rapuérunt eum et adduxérunt in concílium {\color{gray} et statuérunt testes falsos dicéntes: «Homo iste non cessat loqui verba advérsus locum sanctum et legem; audívimus enim eum dicéntem quóniam Iesus Nazarénus hic déstruet locum istum et mutábit consuetúdines, quas trádidit nobis Móyses».

\noindent Et intuéntes eum omnes, qui sedébant in concílio, vidérunt fáciem eius tamquam fáciem ángeli.}

\noindent Dixit autem princeps sacerdótum: «Si hæc ita se habent?». Qui ait: «Viri fratres et patres, audíte.

\noindent {\color{gray} Tabernáculum testimónii erat pátribus nostris in desérto, sicut dispósuit, qui loquebátur ad Móysen, ut fáceret illud secúndum formam, quam víderat; quod et induxérunt suscipiéntes patres nostri cum Iesu in possessiónem géntium, quas éxpulit Deus a fácie patrum nostrórum, usque in diébus David, qui invénit grátiam ante Deum et pétiit, ut inveníret tabernáculum dómui Iacob. Sálomon autem ædificávit illi domum. Sed non Altíssimus in manufáctis hábitat, sicut prophéta dicit:

\noindent \emph{“Cælum mihi thronus est, terra autem scabéllum pedum meórum. Quam domum ædificábitis mihi, dicit Dóminus, aut quis locus requietiónis meæ? Nonne manus mea fecit hæc ómnia?”.} }

\noindent Duri cervíce et incircumcísi córdibus et áuribus, vos semper Spirítui Sancto resístitis, sicut patres vestri et vos. Quem prophetárum non sunt persecúti patres vestri? Et occidérunt eos, qui prænuntiábant de advéntu Iusti, cuius vos nunc proditóres et homicídæ fuístis, qui accepístis legem in dispositiónibus angelórum et non custodístis».

\noindent Audiéntes autem hæc, dissecabántur córdibus suis et stridébant déntibus in eum. Cum autem esset plenus Spíritu Sancto, inténdens in cælum vidit glóriam Dei et Iesum stantem a dextris Dei et ait: «Ecce vídeo cælos apértos et Fílium hóminis a dextris stantem Dei». Exclamántes autem voce magna continuérunt aures suas et ímpetum fecérunt unanímiter in eum et eiciéntes extra civitátem lapidábant.

\noindent Et testes deposuérunt vestiménta sua secus pedes adulescéntis, qui vocabátur Saulus. Et lapidábant Stéphanum invocántem et dicéntem: «Dómine Iesu, súscipe spíritum meum». Pósitis autem génibus clamávit voce magna: «Dómine, ne státuas illis hoc peccátum»; et cum hoc dixísset, obdormívit.

\noindent Saulus autem erat conséntiens neci eius.

\noindent \Vbardot{} Tu autem, Dómine, miserére nobis.
\noindent \Rbardot{} Deo grátias.

\vfill
\pagebreak

\pars{Responsorium 1.} \scriptura{\Rbardot{} Ac. 7, 55 \Vbardot{} Ac. 6, 8; \textbf{H58}}

\vspace{-5mm}

\responsorium{VIII}{temporalia/resp-stephanusservusdei-CROCHU.gtex}{}

\vfill
\pagebreak

\cuminitiali{}{temporalia/benedictio-solemn-unigenitus.gtex}

\vspace{7mm}

\pars{Lectio II.} \scriptura{Sermo 3, 1-3. 5-6: CCL 91 A, 905-909}

\noindent Ex Sermónibus sancti Fulgéntii Ruspénsis epíscopi.

\noindent Heri celebrávimus temporálem sempitérni regis nostri natálem; hódie celebrámus triumphálem mílitis passiónem.

\noindent Heri enim Rex noster, trábea carnis indútus, de aula úteri virginális egrédiens, visitáre dignátus est mundum; hódie miles, de tabernáculo córporis éxiens, triumphátor migrávit ad cælum.

\noindent Rex noster, cum sit altíssimus, pro nobis húmilis venit, sed inánis veníre non pótuit. Magnum quippe donatívum suis milítibus áttulit, quo eos non solum copióse ditávit, sed étiam ad certándum invictíssime confortávit. Attulit namque donum caritátis, quæ perdúceret hómines ad consórtium deitátis.

\noindent Quod ergo áttulit, erogávit; nec sibi áliquid mínuit, sed mirabíliter et suórum fidélium paupériem ditávit et indeficiéntibus thesáuris plenus ipse permánsit.

\noindent Cáritas ergo, quæ ad terram de cælo depósuit Christum, ipsa Stéphanum de terris elevávit ad cælum. Cáritas, quæ præcéssit in rege, ipsa subsequénter refúlsit in mílite.

\noindent Stéphanus ergo, ut nóminis sui corónam meruísset accípere, caritátem pro armis habébat et per ipsam ubíque vincébat.

\noindent Per caritátem Dei sæviéntibus Iudǽis non cessit, per caritátem próximi pro lapidántibus intercéssit.

\noindent Per caritátem arguébat errántes, ut corrigeréntur; per caritátem pro lapidántibus orábat, ne puniréntur.

\noindent \Vbardot{} Tu autem, Dómine, miserére nobis.
\noindent \Rbardot{} Deo grátias.

\vfill
\pagebreak

\pars{Responsorium 2.} \scriptura{\Rbardot{} Ac. 7, 58-59 \Vbardot{} ibid., 59; \textbf{H57}}

\vspace{-5mm}

\responsorium{IV}{temporalia/resp-lapidabantstephanum-CROCHU.gtex}{}

\rubrica{vel ad libitum:}

\vspace{3mm}

\pars{Responsorium 2.} \scriptura{\Rbardot{} Ac. 7, 55 \Vbardot{} ibid.; \textbf{H58}}

\vspace{-5mm}

\responsorium{IV}{temporalia/resp-intuensincaelum-CROCHU.gtex}{}

\vfill
\pagebreak

\cuminitiali{}{temporalia/benedictio-solemn-spiritus.gtex}

\vspace{7mm}

\pars{Lectio III.}

\noindent Caritátis virtúte subníxus, vicit Saulum crudéliter sæviéntem et, quem hábuit in terra persecutórem, in cælo méruit habére consórtem. Ipsa sancta et indeféssa cáritas desiderábat orándo acquírere, quos nequívit monéndo convértere.

\noindent Et ecce nunc Paulus cum Stéphano lætátur, cum Stéphano Christi claritáte perfrúitur, cum Stéphano exsúltat, cum Stéphano regnat. Quo enim præcéssit Stéphanus, trucidátus lapídibus Pauli, illuc secútus est Paulus, adiútus oratiónibus Stéphani.

\noindent Quam vera vita, fratres mei, ubi non Paulus de Stéphani occisióne confúnditur, sed Stéphanus de Pauli consórtio gratulátur, quóniam cáritas in utróque lætátur. Cáritas quippe in Stéphano superávit sævítiam Iudæórum, cáritas in Paulo coopéruit multitúdinem peccatórum, cáritas in utróque páriter regnum méruit possidére cælórum.

\noindent Cáritas est ígitur ómnium fons et orígo bonórum, munímen egrégium, via quæ ducit ad cælum. In caritáte qui ámbulat, nec erráre póterit, nec timére. Ipsa dírigit, ipsa prótegit, ipsa perdúcit.

\noindent Quocírca, fratres, quóniam scalam caritátis constítuit Christus, per quam ad cælum omnis possit conscéndere christiánus, puram caritátem fórtiter retinéte, ipsam vobis ínvicem exhibéte et in ea proficiéndo conscéndite.

\noindent \Vbardot{} Tu autem, Dómine, miserére nobis.
\noindent \Rbardot{} Deo grátias.

\vfill
\pagebreak

\pars{Responsorium 3.} \scriptura{\Vbardot{} Ac. 7, 55; \textbf{H56}}

\vspace{-5mm}

\responsorium{VII}{temporalia/resp-hesternadiedominusnatus-CROCHU-cumdox.gtex}{}

\vfill
\pagebreak}
\newcommand{\hymnuslaudes}{\pars{Hymnus.}

\vspace{-5mm}

\antiphona{IV}{temporalia/hym-CohorsBeata.gtex}}
\newcommand{\laudes}{\pars{Psalmus 1.}

\vspace{-4mm}

\antiphona{I d}{temporalia/ant-ioannesapostolusetevangelista.gtex}

%\vspace{-2mm}

\scriptura{Psalmus 62.}

\vspace{-1mm}

\initiumpsalmi{temporalia/ps62-initium-i-d-auto.gtex}

\input{temporalia/ps62-i-d.tex} \Abardot{}

\vfill
\pagebreak

\pars{Psalmus 2.} \scriptura{\textbf{H62}}

\vspace{-4mm}

\antiphona{I g}{temporalia/ant-isteestioannescuichristus.gtex}

\scriptura{Canticum trium puerorum, Dan. 3, 57-88 et 56}

\initiumpsalmi{temporalia/dan3-initium-i-g-auto.gtex}

\input{temporalia/dan3-i-g-sinedox.tex}

\rubrica{Hic non dicitur Gloria Patri, neque Amen.}

\vfill

\vspace{-6mm}

\antiphona{}{temporalia/ant-isteestioannescuichristus.gtex} % repeat the antiphon - new page

\vfill
\pagebreak

\pars{Psalmus 3.} \scriptura{Io. 21, 7}

\vspace{-4mm}

\antiphona{I g}{temporalia/ant-dixitdiscipulusille.gtex}

\scriptura{Psalmus 149.}

\initiumpsalmi{temporalia/ps149-initium-i-g-auto.gtex}

\input{temporalia/ps149-i-g.tex} \Abardot{}

\vfill
\pagebreak}
\newcommand{\lectiobrevis}{\pars{Lectio Brevis.} \scriptura{Ac. 6, 2-5}

\noindent Non est æquum nos derelinquéntes verbum Dei ministráre mensis; consideráte vero, fratres, viros ex vobis boni testimónii septem plenos Spíritu et sapiéntia, quos constituémus super hoc opus; nos vero oratióni et ministério verbi instántes érimus. Et plácuit sermo coram omni multitúdine.}
\newcommand{\responsoriumbreve}{\pars{Responsorium breve.} \scriptura{Ps. 44, 17-18}

\cuminitiali{VI}{temporalia/resp-constitueseos.gtex}}
\newcommand{\oratio}{\pars{Oratio.}

\noindent Da nobis, quǽsumus, Dómine, imitári quod cólimus, ut discámus et inimícos dilígere, quia eius natalícia celebrámus, qui novit étiam pro persecutóribus exoráre.

\noindent Per Dóminum nostrum Iesum Christum, Fílium tuum, qui tecum vivit et regnat in unitáte Spíritus Sancti, Deus, per ómnia sǽcula sæculórum.  Amen.

\noindent \Rbardot{} Amen.}
\newcommand{\benedictus}{\pars{Canticum Zachariæ.} \scriptura{Io. 1, 14}

\vspace{-4mm}

{
\grechangedim{interwordspacetext}{0.18 cm plus 0.15 cm minus 0.05 cm}{scalable}%
\antiphona{V g}{temporalia/ant-verbumcarofactumest.gtex}
\grechangedim{interwordspacetext}{0.22 cm plus 0.15 cm minus 0.05 cm}{scalable}%
}

\vspace{-1mm}

\scriptura{Lc. 1, 68-79}

\vspace{-2mm}

\cantusSineNeumas
\initiumpsalmi{temporalia/benedictus-initium-vsoll-g-auto.gtex}

%\vspace{-1.5mm}

\input{temporalia/benedictus-vsoll-g.tex} \Abardot{}}
\newcommand{\preces}{\noindent Fratres, Salvatórem nostrum, testem fidélem, \gredagger{} per mártyres interféctos propter verbum Dei celebrémus, \grestar{} clamántes:

\Rbardot{} Redemísti nos Deo in sánguine tuo.

\noindent Per mártyres tuos, \gredagger{} qui líbere mortem in testimónium fídei sunt ampléxi, \grestar{} da nobis, Dómine, veram spíritus libertátem.

\Rbardot{} Redemísti nos Deo in sánguine tuo.

\noindent Per mártyres tuos, \gredagger{} qui fidem usque ad sánguinem sunt conféssi, \grestar{} da nobis, Dómine, puritátem fideíque constántiam.

\Rbardot{} Redemísti nos Deo in sánguine tuo.

\noindent Per mártyres tuos, \gredagger{} qui, sustinéntes crucem, tua vestígia sunt secúti, \grestar{} da nobis, Dómine, ærúmnas vitæ fórtiter sustinére.

\Rbardot{} Redemísti nos Deo in sánguine tuo.

\noindent Per mártyres tuos, \gredagger{} qui stolas suas lavérunt in sánguine Agni, \grestar{} da nobis, Dómine, omnes insídias carnis mundíque devíncere.

\Rbardot{} Redemísti nos Deo in sánguine tuo.}
\newcommand{\magnificat}{\pars{Canticum B. Mariæ V.} \scriptura{Io. 21, 23; \textbf{H62}}

\vspace{-4mm}

\antiphona{I g}{temporalia/ant-exiitsermointerfratres.gtex}

%\vspace{-2mm}

\scriptura{Lc. 1, 46-55}

\vspace{-2mm}

\cantusSineNeumas
\initiumpsalmi{temporalia/magnificat-initium-isoll-g.gtex}

%\vspace{-1.5mm}

\input{temporalia/magnificat-isoll-g.tex}

\vfill

\antiphona{}{temporalia/ant-exiitsermointerfratres.gtex}}
\newcommand{\capitulum}{\pars{Capitulum.} \scriptura{Eccli. 15, 1-2}

\grechangedim{interwordspacetext}{0.12 cm plus 0.15 cm minus 0.05 cm}{scalable}%
\cuminitiali{}{temporalia/capitulum-QuiTimet.gtex}
\grechangedim{interwordspacetext}{0.22 cm plus 0.15 cm minus 0.05 cm}{scalable}
}
\newcommand{\responsoriumbrevevesp}{\pars{Responsorium breve.} \scriptura{Ps. 44, 17-18}

\cuminitiali{VI}{temporalia/resp-constitueseos.gtex}}
\newcommand{\vespversus}{\pars{Versus.}

\noindent \Vbardot{} Valde honorándus est beátus Ioánnes.

\noindent \Rbardot{} Qui supra pectus Dómini in cœna recúbuit.}
\newcommand{\oratioVesperas}{\cuminitiali{}{temporalia/oratioioannes.gtex}}
% LuaLaTeX

\documentclass[a4paper, twoside, 12pt]{article}
\usepackage[latin]{babel}
%\usepackage[landscape, left=3cm, right=1.5cm, top=2cm, bottom=1cm]{geometry} % okraje stranky
%\usepackage[landscape, a4paper, mag=1166, truedimen, left=2cm, right=1.5cm, top=1.6cm, bottom=0.95cm]{geometry} % okraje stranky
\usepackage[landscape, a4paper, mag=1400, truedimen, left=0.5cm, right=0.5cm, top=0.5cm, bottom=0.5cm]{geometry} % okraje stranky

\usepackage{fontspec}
\setmainfont[FeatureFile={junicode.fea}, Ligatures={Common, TeX}, RawFeature=+fixi]{Junicode}
%\setmainfont{Junicode}

% shortcut for Junicode without ligatures (for the Czech texts)
\newfontfamily\nlfont[FeatureFile={junicode.fea}, Ligatures={Common, TeX}, RawFeature=+fixi]{Junicode}

% Hebrew font:
% http://scripts.sil.org/cms/scripts/page.php?site_id=nrsi&id=SILHebrUnic2
\newfontfamily\hebfont[Scale=1]{Ezra SIL}

\usepackage{multicol}
\usepackage{color}
\usepackage{lettrine}
\usepackage{fancyhdr}

% usual packages loading:
\usepackage{luatextra}
\usepackage{graphicx} % support the \includegraphics command and options
\usepackage{gregoriotex} % for gregorio score inclusion
\usepackage{gregoriosyms}
\usepackage{wrapfig} % figures wrapped by the text
\usepackage{parcolumns}
\usepackage[contents={},opacity=1,scale=1,color=black]{background}
\usepackage{tikzpagenodes}
\usepackage{calc}
\usepackage{longtable}
\usetikzlibrary{calc}

\setlength{\headheight}{14.5pt}

% Commands used to produce a typical "Conventus" booklet

\newenvironment{titulusOfficii}{\begin{center}}{\end{center}}
\newcommand{\dies}[1]{#1

}
\newcommand{\nomenFesti}[1]{\textbf{\Large #1}

}
\newcommand{\celebratio}[1]{#1

}

\newcommand{\hora}[1]{%
\vspace{0.5cm}{\large \textbf{#1}}

\fancyhead[LE]{\thepage\ / #1}
\fancyhead[RO]{#1 / \thepage}
\addcontentsline{toc}{subsection}{#1}
}

% larger unit than a hora
\newcommand{\divisio}[1]{%
\begin{center}
{\Large \textsc{#1}}
\end{center}
\fancyhead[CO,CE]{#1}
\addcontentsline{toc}{section}{#1}
}

% a part of a hora, larger than pars
\newcommand{\subhora}[1]{
\begin{center}
{\large \textit{#1}}
\end{center}
%\fancyhead[CO,CE]{#1}
\addcontentsline{toc}{subsubsection}{#1}
}

% rubricated inline text
\newcommand{\rubricatum}[1]{\textit{#1}}

% standalone rubric
\newcommand{\rubrica}[1]{\vspace{3mm}\rubricatum{#1}}

\newcommand{\notitia}[1]{\textcolor{red}{#1}}

\newcommand{\scriptura}[1]{\hfill \small\textit{#1}}

\newcommand{\translatioCantus}[1]{\vspace{1mm}%
{\noindent\footnotesize \nlfont{#1}}}

% pruznejsi varianta nasledujiciho - umoznuje nastavit sirku sloupce
% s prekladem
\newcommand{\psalmusEtTranslatioB}[3]{
  \vspace{0.5cm}
  \begin{parcolumns}[colwidths={2=#3}, nofirstindent=true]{2}
    \colchunk{
      \input{#1}
    }

    \colchunk{
      \vspace{-0.5cm}
      {\footnotesize \nlfont
        \input{#2}
      }
    }
  \end{parcolumns}
}

\newcommand{\psalmusEtTranslatio}[2]{
  \psalmusEtTranslatioB{#1}{#2}{8.5cm}
}


\newcommand{\canticumMagnificatEtTranslatio}[1]{
  \psalmusEtTranslatioB{#1}{temporalia/extra-adventum-vespers/magnificat-boh.tex}{12cm}
}
\newcommand{\canticumBenedictusEtTranslatio}[1]{
  \psalmusEtTranslatioB{#1}{temporalia/extra-adventum-laudes/benedictus-boh.tex}{10.5cm}
}

% volne misto nad antifonami, kam si zpevaci dokresli neumy
\newcommand{\hicSuntNeumae}{\vspace{0.5cm}}

% prepinani mista mezi notovymi osnovami: pro neumovane a neneumovane zpevy
\newcommand{\cantusCumNeumis}{
  \setgrefactor{17}
  \global\advance\grespaceabovelines by 5mm%
}
\newcommand{\cantusSineNeumas}{
  \setgrefactor{17}
  \global\advance\grespaceabovelines by -5mm%
}

% znaky k umisteni nad inicialu zpevu
\newcommand{\superInitialam}[1]{\gresetfirstlineaboveinitial{\small {\textbf{#1}}}{\small {\textbf{#1}}}}

% pars officii, i.e. "oratio", ...
\newcommand{\pars}[1]{\textbf{#1}}

\newenvironment{psalmus}{
  \setlength{\parindent}{0pt}
  \setlength{\parskip}{5pt}
}{
  \setlength{\parindent}{10pt}
  \setlength{\parskip}{10pt}
}

%%%% Prejmenovat na latinske:
\newcommand{\nadpisZalmu}[1]{
  \hspace{2cm}\textbf{#1}\vspace{2mm}%
  \nopagebreak%

}

% mode, score, translation
\newcommand{\antiphona}[3]{%
\hicSuntNeumae
\superInitialam{#1}
\includescore{#2}

#3
}
 % Often used macros

\newcommand{\annusEditionis}{2021}

\def\hebinitial#1{%
\leavevmode{\newbox\hebbox\setbox\hebbox\hbox{\hebfont{#1}\hskip 1mm}\kern -\wd\hebbox\hbox{\hebfont{#1}\hskip 1mm}}%
}

%%%% Vicekrat opakovane kousky

\newcommand{\anteOrationem}{
  \rubrica{Ante Orationem, cantatur a Superiore:}

  \pars{Supplicatio Litaniæ.}

  \cuminitiali{}{temporalia/supplicatiolitaniae.gtex}

  \pars{Oratio Dominica.}

  \cuminitiali{}{temporalia/oratiodominica.gtex}

  \rubrica{Deinde dicitur ab Hebdomadario:}

  \cuminitiali{}{temporalia/dominusvobiscum-solemnis.gtex}

  \rubrica{In choro monialium loco Dominus vobiscum dicitur:}

  \sineinitiali{temporalia/domineexaudi.gtex}
}

\setlength{\columnsep}{30pt} % prostor mezi sloupci

%%%%%%%%%%%%%%%%%%%%%%%%%%%%%%%%%%%%%%%%%%%%%%%%%%%%%%%%%%%%%%%%%%%%%%%%%%%%%%%%%%%%%%%%%%%%%%%%%%%%%%%%%%%%%
\begin{document}

% Here we set the space around the initial.
% Please report to http://home.gna.org/gregorio/gregoriotex/details for more details and options
\grechangedim{afterinitialshift}{2.2mm}{scalable}
\grechangedim{beforeinitialshift}{2.2mm}{scalable}
\grechangedim{interwordspacetext}{0.22 cm plus 0.15 cm minus 0.05 cm}{scalable}%
\grechangedim{annotationraise}{-0.2cm}{scalable}

% Here we set the initial font. Change 38 if you want a bigger initial.
% Emit the initials in red.
\grechangestyle{initial}{\color{red}\fontsize{38}{38}\selectfont}

\pagestyle{empty}

%%%% Titulni stranka
\begin{titulusOfficii}
\titulus
\end{titulusOfficii}

\vfill

\begin{center}
%Ad usum et secundum consuetudines chori \guillemotright{}Conventus Choralis\guillemotleft.

%Editio Sancti Wolfgangi \annusEditionis
\end{center}

\scriptura{}

\pars{}

\pagebreak

\renewcommand{\headrulewidth}{0pt} % no horiz. rule at the header
\fancyhf{}
\pagestyle{fancy}

\cantusSineNeumas

\pars{} \scriptura{}

\ifx\sinematutinum\undefined
\hora{Ad Matutinum.} %%%%%%%%%%%%%%%%%%%%%%%%%%%%%%%%%%%%%%%%%%%%%%%%%%%%%

\vspace{2mm}

\cuminitiali{}{temporalia/dominelabiamea.gtex}

\vfill
%\pagebreak

\vspace{2mm}

\ifx\invitatorium\undefined
\pars{Invitatorium.}

\vspace{-2mm}

\antiphona{E}{temporalia/inv-christusnatusest-simplex.gtex}
\else
\invitatorium
\fi

\vfill
\pagebreak

\ifx\hymnusmatutinum\undefined
\pars{Hymnus.}

{
\grechangedim{interwordspacetext}{0.10 cm plus 0.15 cm minus 0.05 cm}{scalable}%
\antiphona{IV}{temporalia/hym-CandorAEternae-simplex.gtex}
\grechangedim{interwordspacetext}{0.22 cm plus 0.15 cm minus 0.05 cm}{scalable}%
}

\vspace{-3mm}
\else
\hymnusmatutinum
\fi

\vfill
\pagebreak

\matutinum

\ifx\postoctavam\undefined
% Te Deum

\vspace{-5mm}

\ifx\tedeumsolemnis\undefined
\ifx\tedeumsimplex\undefined
\ifx\tedeummonasticum\undefined
{
\pars{Hymnus Ambrosianus} \scriptura{Alio modo, iuxta morem Romanum}

\vspace{-2mm}

\grechangedim{interwordspacetext}{0.26 cm plus 0.15 cm minus 0.05 cm}{scalable}%
\cuminitiali{III}{temporalia/tedeum-romanum-gn.gtex}
\grechangedim{interwordspacetext}{0.22 cm plus 0.15 cm minus 0.05 cm}{scalable}%
}
\else
{
\pars{Hymnus Ambrosianus} \scriptura{Tonus Monasticus}

\vspace{-2mm}

\grechangedim{interwordspacetext}{0.26 cm plus 0.15 cm minus 0.05 cm}{scalable}%
\cuminitiali{III}{temporalia/tedeum-monasticum-am34.gtex}
\grechangedim{interwordspacetext}{0.22 cm plus 0.15 cm minus 0.05 cm}{scalable}%
}
\fi
\else
{
\pars{Hymnus Ambrosianus} \scriptura{Tonus Simplex}

\vspace{-2mm}

\grechangedim{interwordspacetext}{0.30 cm plus 0.15 cm minus 0.05 cm}{scalable}%
\cuminitiali{III}{temporalia/tedeum-simplex-gn.gtex}
\grechangedim{interwordspacetext}{0.22 cm plus 0.15 cm minus 0.05 cm}{scalable}%
}
\fi
\else
{
\pars{Hymnus Ambrosianus} \scriptura{Tonus Solemnis}

\vspace{-2mm}

\grechangedim{interwordspacetext}{0.26 cm plus 0.15 cm minus 0.05 cm}{scalable}%
\cuminitiali{III}{temporalia/tedeum-solemnis-gn.gtex}
\grechangedim{interwordspacetext}{0.22 cm plus 0.15 cm minus 0.05 cm}{scalable}%
}
\fi

\vfill
\pagebreak
\fi

\rubrica{Reliqua omittuntur, nisi Laudes separandæ sint.}

\sineinitiali{temporalia/domineexaudi.gtex}

\vfill

\oratio

\vfill

\noindent \Vbardot{} Dómine, exáudi oratiónem meam.
\Rbardot{} Et clamor meus ad te véniat.

\vfill

% Nocturnale Romanum 2002, p. LXXVI Benedicamus Domino seems to match
% the one from Solemn Laudes.
\cuminitiali{V}{temporalia/benedicamus-solemnis-laud.gtex}

\vfill

\noindent \Vbardot{} Fidélium ánimæ per misericórdiam Dei requiéscant in pace.
\Rbardot{} Amen.

\vfill
\pagebreak
\fi

\ifx\sinelaudes\undefined
\hora{Ad Laudes.} %%%%%%%%%%%%%%%%%%%%%%%%%%%%%%%%%%%%%%%%%%%%%%%%%%%%%

\cantusSineNeumas

\vspace{0.5cm}
\grechangedim{interwordspacetext}{0.18 cm plus 0.15 cm minus 0.05 cm}{scalable}%
\ifx\postoctavam\undefined
\cuminitiali{}{temporalia/deusinadiutorium-alter.gtex}
\else
\cuminitiali{}{temporalia/deusinadiutorium-communis.gtex}
\fi
\grechangedim{interwordspacetext}{0.22 cm plus 0.15 cm minus 0.05 cm}{scalable}%

\vfill
%\pagebreak

\ifx\hymnuslaudes\undefined
\pagebreak
\pars{Hymnus} \scriptura{Sedulius}

\grechangedim{interwordspacetext}{0.16 cm plus 0.15 cm minus 0.05 cm}{scalable}%
\cuminitiali{III}{temporalia/hym-ASolisOrtus.gtex}
\grechangedim{interwordspacetext}{0.22 cm plus 0.15 cm minus 0.05 cm}{scalable}%
\vspace{-3mm}
\else
\hymnuslaudes
\fi

\vfill
\pagebreak

\ifx\laudes\undefined
\pars{Psalmus 1.} \scriptura{Lc. 2, 8.11.13.18; \textbf{H50}}

\vspace{-4mm}

\antiphona{II D}{temporalia/ant-quemvidistis.gtex}

\vspace{-2mm}

\scriptura{Psalmus 62.}

\vspace{-1mm}

\initiumpsalmi{temporalia/ps62-initium-ii-D-auto.gtex}

\input{temporalia/ps62-ii-D.tex} \Abardot{}

\vfill
\pagebreak

\pars{Psalmus 2.} \scriptura{Lc. 2, 10.11; \textbf{H50}}

\vspace{-4mm}

\antiphona{VII d}{temporalia/ant-angelusadpastores.gtex}

\scriptura{Canticum trium puerorum, Dan. 3, 57-88 et 56}

\initiumpsalmi{temporalia/dan3-initium-vii-d-auto.gtex}

\input{temporalia/dan3-vii-d-sinedox.tex}

\rubrica{Hic non dicitur Gloria Patri, neque Amen.}

\vfill

\vspace{-6mm}

\antiphona{}{temporalia/ant-angelusadpastores.gtex} % repeat the antiphon - new page

\vfill
\pagebreak

\pars{Psalmus 3.} \scriptura{Is. 9, 6; \textbf{H51}}

\vspace{-4mm}

\antiphona{VIII G\textsuperscript{2}}{temporalia/ant-parvulusfilius.gtex}

\scriptura{Psalmus 149.}

\initiumpsalmi{temporalia/ps149-initium-viii-G2-auto.gtex}

\input{temporalia/ps149-viii-G2.tex} \Abardot{}

\vfill
\pagebreak
\else
\laudes
\fi

\lectiobrevis

\vfill

\ifx\responsoriumbreve\undefined
\pars{Responsorium breve.} \scriptura{Ps. 97, 2}

\cuminitiali{VI}{temporalia/resp-notumfecit.gtex}
\else
\responsoriumbreve
\fi

\vfill
\pagebreak

\benedictus

\vspace{-1cm}

\vfill
\pagebreak

\pars{Preces.}

\sineinitiali{}{temporalia/tonusprecumnovum.gtex}

\preces

\vfill

\pars{Oratio Dominica.}

\cuminitiali{}{temporalia/oratiodominicaalt.gtex}

\vfill
\pagebreak

\rubrica{vel:}

\pars{Deprecatio Gelasii}

\vspace{-5mm}

\grechangedim{interwordspacetext}{0.16 cm plus 0.15 cm minus 0.05 cm}{scalable}%
\antiphona{D\textsuperscript{1}}{temporalia/deprecatio4-propace.gtex}
\grechangedim{interwordspacetext}{0.22 cm plus 0.15 cm minus 0.05 cm}{scalable}%

\vfill

\pars{Oratio Dominica.}

\cuminitiali{D}{temporalia/oratiodominica-d.gtex}

\vfill
\pagebreak

% Oratio. %%%
\oratio

\vspace{-1mm}

\vfill

\ifx\commemoratio\undefined
\else
\commemoratio
\fi

\rubrica{Hebdomadarius dicit Dominus vobiscum, vel, absente sacerdote vel diacono, sic concluditur:}

\vspace{2mm}

\antiphona{C}{temporalia/dominusnosbenedicat.gtex}

\rubrica{Postea cantatur a cantore:}

\vspace{2mm}

\ifx\benedicamuslaudes\undefined
\ifx\postoctavam\undefined
\cuminitiali{II}{temporalia/benedicamus-solemnism-laud.gtex}
\else
\cuminitiali{}{temporalia/benedicamus-tempore-nativitatis.gtex}
\fi
\else
\benedicamuslaudes
\fi

\vspace{1mm}

\vfill
\pagebreak
\fi

\ifx\sinevesperas\undefined
\hora{Ad Vesperas.} %%%%%%%%%%%%%%%%%%%%%%%%%%%%%%%%%%%%%%%%%%%%%%%%%%%%%

\cantusSineNeumas

%\vspace{-2mm}
\grechangedim{interwordspacetext}{0.18 cm plus 0.15 cm minus 0.05 cm}{scalable}%
\ifx\postoctavam\undefined
\cuminitiali{}{temporalia/deusinadiutorium-solemnis.gtex}
\else
\cuminitiali{}{temporalia/deusinadiutorium-communis.gtex}
\fi
\grechangedim{interwordspacetext}{0.22 cm plus 0.15 cm minus 0.05 cm}{scalable}%

\vfill
%\pagebreak

\vspace{-2mm}

\ifx\vesperas\undefined
\pars{Psalmus 1.} \scriptura{Ps. 109, 3; \textbf{H52}}

\vspace{-5mm}

\antiphona{I g}{temporalia/ant-tecumprincipium.gtex}

\scriptura{Psalmus 109.}

\initiumpsalmi{temporalia/ps109-initium-i-g-auto.gtex}

\vspace{-1.5mm}

\input{temporalia/ps109-i-g.tex} \Abardot{}

\vfill
\pagebreak

\pars{Psalmus 2.} \scriptura{Ps. 110, 9; \textbf{H52}}

\vspace{-4mm}

\antiphona{VII a}{temporalia/ant-redemptionemmisit.gtex}

\scriptura{Psalmus 110.}

\initiumpsalmi{temporalia/ps110-initium-vii-a-auto.gtex}

\input{temporalia/ps110-vii-a.tex} \Abardot{}

\vfill
\pagebreak

\pars{Psalmus 3.} \scriptura{Ps. 111, 4; \textbf{H52}}

\vspace{-4mm}

\antiphona{VII d}{temporalia/ant-exortumest.gtex}

\scriptura{Psalmus 111.}

\initiumpsalmi{temporalia/ps111-initium-vii-d-auto.gtex}

\input{temporalia/ps111-vii-d.tex} \Abardot{}

\vfill
\pagebreak

\ifx\impar\undefined
\pars{Psalmus 4.} \scriptura{Ps. 131, 11; \textbf{H52}}

\vspace{-4mm}

\antiphona{VIII G}{temporalia/ant-defructuventris.gtex}

\scriptura{Psalmus 131.}

\initiumpsalmi{temporalia/ps131-initium-viii-G-auto.gtex}

\input{temporalia/ps131-viii-G.tex}

\vfill

\antiphona{}{temporalia/ant-defructuventris.gtex}
\else
\pars{Psalmus 4.} \scriptura{Ps. 129, 7; \textbf{H52}}

\vspace{-4mm}

\antiphona{II* b}{temporalia/ant-apuddominum.gtex}

\scriptura{Psalmus 129.}

\initiumpsalmi{temporalia/ps129-initium-ii_-B-auto.gtex}

\input{temporalia/ps129-ii_-B.tex} \Abardot{}
\fi

\vfill
\pagebreak
\else
\vesperas
\fi

\ifx\capitulum\undefined
\pars{Capitulum.} \scriptura{Hebr. 1, 1-2}

\grechangedim{interwordspacetext}{0.12 cm plus 0.15 cm minus 0.05 cm}{scalable}%
\cuminitiali{}{temporalia/capitulum-Multifariam.gtex}
\grechangedim{interwordspacetext}{0.22 cm plus 0.15 cm minus 0.05 cm}{scalable}
\else
\capitulum
\fi

\vfill

\ifx\responsoriumbrevevesp\undefined
\pars{Responsorium breve.} \scriptura{Io. 1, 14}

\cuminitiali{VI}{temporalia/resp-verbumcaro-simplex.gtex}
\else
\responsoriumbrevevesp
\fi

\vfill
\pagebreak

\ifx\hymnusvesperas\undefined
\pars{Hymnus}

\cuminitiali{I}{temporalia/hym-ChristeRedemptor.gtex}
\else
\hymnusvesperas
\fi
\vspace{-3mm}

\vfill
%\pagebreak

\ifx\vespversus\undefined
\pars{Versus.} \scriptura{Ps. 97, 2}

% Versus. %%%
\sineinitiali{temporalia/versus-notumfecit-communis.gtex}
\else
\vespversus
\fi

\vfill
\pagebreak

\magnificat

\vfill
\pagebreak

\anteOrationem

\pagebreak

% Oratio. %%%
\ifx\oratioVesperas\undefined
\cuminitiali{}{temporalia/oratio.gtex}
\else
\oratioVesperas
\fi

\vspace{-1mm}

\vfill

\rubrica{Hebdomadarius dicit iterum Dominus vobiscum, vel cantor dicit:}

\vspace{2mm}

\sineinitiali{temporalia/domineexaudi.gtex}

\rubrica{Postea cantatur a cantore:}

\vspace{2mm}

\ifx\postoctavam\undefined
\cuminitiali{II}{temporalia/benedicamus-solemnism-2vesp.gtex}
\else
\cuminitiali{I}{temporalia/benedicamus-feria-vesperae.gtex}
\fi

\vspace{1mm}
\fi

\end{document}

