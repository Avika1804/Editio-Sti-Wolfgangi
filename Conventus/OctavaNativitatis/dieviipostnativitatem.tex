\newcommand{\titulus}{\dies{Die 31. Decembris.}
\nomenFesti{Die Septima post Nativitatem.}}
\newcommand{\tedeummonasticum}{Monasticum}
\newcommand{\matutinum}{\pars{Psalmus 1.} \scriptura{Ps. 95, 11-13; \textbf{H46}}

\vspace{-4mm}

\antiphona{II* a}{temporalia/ant-laetenturcaeli.gtex}

%\vspace{-1mm}

\scriptura{Ps. 95}

%\vspace{-2mm}

\initiumpsalmi{temporalia/ps95-initium-ii_-a-auto.gtex}

%\vspace{-1.5mm}

\input{temporalia/ps95-ii_-a.tex} \Abardot{}

\vfill
\pagebreak

\pars{Psalmus 2.} \scriptura{Ps. 96, 11; \textbf{H361}}

\vspace{-4mm}

\antiphona{VI F}{temporalia/ant-luxortaestiustoet.gtex}

%\vspace{-5mm}

\scriptura{Ps. 96}

\initiumpsalmi{temporalia/ps96-initium-vi-F-auto.gtex}

\input{temporalia/ps96-vi-F.tex} \Abardot{}

\vfill
\pagebreak

\pars{Psalmus 3.} \scriptura{Ps. 97, 2}

\vspace{-4mm}

\antiphona{V d}{temporalia/ant-notumfecitdominus.gtex}

%\vspace{-2mm}

\scriptura{Ps. 97}

%\vspace{-2mm}

\initiumpsalmi{temporalia/ps97-initium-v-d-auto.gtex}

\input{temporalia/ps97-v-d.tex} \Abardot{}

\vfill
\pagebreak

\pars{Versus.}

\noindent \Vbardot{} Novíssime Deus locútus est nobis in Fílio.
\noindent \Rbardot{} Per quem fecit et sǽcula.

\vspace{5mm}

\sineinitiali{temporalia/oratiodominica-mat.gtex}

\vspace{5mm}

\pars{Absolutio.}

\cuminitiali{}{temporalia/absolutio-exaudi.gtex}

\vfill
\pagebreak

\cuminitiali{}{temporalia/benedictio-solemn-benedictione.gtex}

\vspace{7mm}

\pars{Lectio I.} \scriptura{Col. 2, 4-15}

\noindent De Epístola beáti Pauli apóstoli ad Colossénses.

\noindent Fratres: Hoc dico, ut nemo vos decípiat in subtilitáte sermónum. Nam etsi córpore absens sum, sed spíritu vobíscum sum, gaudens et videns órdinem vestrum et firmaméntum eius, quæ in Christum est, fídei vestræ.

\noindent Sicut ergo accepístis Christum Iesum Dóminum, in ipso ambuláte, radicáti et superædificáti in ipso et confirmáti fide, sicut didicístis, abundántes in gratiárum actióne. Vidéte, ne quis vos deprædétur per philosophíam et inánem falláciam secúndum traditiónem hóminum, secúndum eleménta mundi et non secúndum Christum; quia in ipso inhábitat omnis plenitúdo divinitátis corporáliter, et estis in illo repléti, qui est caput omnis principátus et potestátis, in quo et circumcísi estis circumcisióne non manufácta in exspoliatióne córporis carnis, in circumcisióne Christi, consepúlti ei in baptísmo, in quo et conresuscitáti estis per fidem operatiónis Dei, qui suscitávit illum a mórtuis; et vos, cum mórtui essétis in delíctis et præpútio carnis vestræ, convivificávit cum illo, donans nobis ómnia delícta, delens, quod advérsum nos erat, chirógraphum decrétis, quod erat contrárium nobis, et ipsum tulit de médio affígens illud cruci; exspólians principátus et potestátes tradúxit confidénter, triúmphans illos in semetípso.

\noindent \Vbardot{} Tu autem, Dómine, miserére nobis.
\noindent \Rbardot{} Deo grátias.

\vfill
\pagebreak

\pars{Responsorium 1.} \scriptura{\Rbardot{} Cf. Lc. 2, 12 \Vbardot{} Cf. Hab. 3, 2; \textbf{H45}}

\vspace{-5mm}

\responsorium{I}{temporalia/resp-oregemcaeli-CROCHU.gtex}{}

\vfill
\pagebreak

\cuminitiali{}{temporalia/benedictio-solemn-unigenitus.gtex}

\vspace{7mm}

\pars{Lectio II.} \scriptura{Sermo 6 in Nativitate Domini, 2-3, 5: PL 54, 213-216}

\noindent Ex Sermónibus sancti Leónis Magni papæ.

\noindent Quamvis illa infántia, quam Fílii Dei non est dedignáta maiéstas, in virum perféctum ætátis adiectióne provécta sit, et, consummáto passiónis et resurrectiónis triúmpho, omnes suscéptæ pro nobis humilitátis transíerint actiónes, rénovat tamen nobis hodiérna festívitas nati Iesu ex María Vírgine sacra primórdia; et, dum Salvatóris nostri adorámus ortum, invenímur nos nostrum celebráre princípium.

\noindent Generátio enim Christi orígo est pópuli christiáni, et natális cápitis natális est córporis.

\noindent Hábeant licet sínguli quique vocatórum órdinem suum, et omnes Ecclésiæ fílii témporum sint successióne distíncti, univérsa tamen summa fidélium, fonte orta baptísmatis, sicut cum Christo in passióne crucifíxi, in resurrectióne resuscitáti, in ascensióne ad déxteram Patris collocáti, ita cum ipso sunt in hac nativitáte congéniti.

\noindent Quisquis enim hóminum in quacúmque mundi parte credéntium regenerátur in Christo, intercíso originális trámite vetustátis, transit in novum hóminem renascéndo; nec iam in propágine habétur carnális patris, sed in gérmine Salvatóris, qui ídeo Fílius hóminis est factus, ut nos fílii Dei esse possímus.

\noindent \Vbardot{} Tu autem, Dómine, miserére nobis.
\noindent \Rbardot{} Deo grátias.

\vfill
\pagebreak

\pars{Responsorium 2.} \scriptura{\Vbardot{} Lc. 2, 14; \textbf{H45}}

\vspace{-5mm}

\responsorium{VIII}{temporalia/resp-hodienobis-CROCHU.gtex}

\vfill
\pagebreak

\cuminitiali{}{temporalia/benedictio-solemn-spiritus.gtex}

\vspace{7mm}

\pars{Lectio III.}

\noindent Nisi enim ille ad nos hac humilitáte descénderet, nemo ad illum ullis suis méritis perveníret.

\noindent Unde ipsa colláti múneris magnitúdo dignam a nobis éxigit suo splendóre reveréntiam. Ideo enim, sicut beátus Apóstolus docet, non spíritum huius mundi accépimus, sed Spíritum qui ex Deo est, ut sciámus quæ a Deo donáta sunt nobis; qui non áliter pie cólitur, nisi id ei, quod ipse tríbuit, offerátur.

\noindent Quid autem in thesáuro domínicæ largitátis ad honórem præséntis festi tam cóngruum póssumus inveníre, quam pacem, quæ in nativitáte Dómini primo est angélico prædicáta concéntu?

\noindent Ipsa enim est, quæ parit fílios Dei, nutrix dilectiónis et génetrix unitátis, réquies beatórum et æternitátis habitáculum; cuius hoc opus próprium et speciále benefícium est, ut iungat Deo quos secérnit de mundo.

\noindent Qui ergo non \emph{ex sanguínibus, neque ex voluntáte carnis, neque ex voluntáte viri, sed ex Deo nati sunt}, ófferant Patri pacificórum concórdiam filiórum, et in primogénitum novæ creatúræ, qui venit non suam, sed mitténtis, fácere voluntátem, univérsa adoptiónis membra concúrrant: quóniam grátia Patris non discórdes neque dissímiles, sed unum sentiéntes unúmque amántes adoptávit herédes. Ad unam reformátos imáginem, opórtet ánimam habére confórmem.

\noindent Natális Dómini, natális est pacis: sic enim ait Apóstolus: \emph{Ipse est pax nostra, qui fecit útraque unum}; quóniam sive Iudǽus, sive gentílis, \emph{per ipsum habémus accéssum in uno Spíritu ad Patrem}.

\noindent \Vbardot{} Tu autem, Dómine, miserére nobis.
\noindent \Rbardot{} Deo grátias.

\vfill
\pagebreak

\pars{Responsorium 3.} \scriptura{\Vbardot{} Lc. 2, 24; \textbf{H45}}

\vspace{-5mm}

\responsorium{V}{temporalia/resp-hodienobiscaelorum-CROCHU-cumdox.gtex}{}

\vfill
\pagebreak}
\newcommand{\lectiobrevis}{\pars{Lectio Brevis.} \scriptura{Is. 4, 2-3}

\noindent In die illa erit germen Dómini in splendórem et glóriam et fructus terræ sublímis et exsultátio his, qui salváti fúerint de Israel. Et erit: omnis qui relíctus fúerit in Sion et resíduus in Ierúsalem, sanctus vocábitur, omnis, qui scriptus est ad vitam in Ierúsalem.}
\newcommand{\oratio}{\pars{Oratio.}

\noindent Omnípotens sempitérne Deus, qui in Fílii tui nativitáte tribuísti totíus religiónis inítium perfectionémque constáre, \gredagger{} da nobis, quǽsumus, in eius portióne censéri, \grestar{} in quo totíus salútis humánæ summa consístit.

\vfill

\pars{Pro commemoratione S. Silvestri I, Papæ et Confessoris.} \scriptura{Mt. 10, 20}

\antiphona{VIII G}{temporalia/ant-nonenimvosestis.gtex}

\vfill

\noindent Auxiliáre, Dómine, pópulo tuo, beáti Sylvéstri, papæ, intercessióne suffúlto, ut, præséntem vitam sub tua gubernatióne transcúrrens, mereátur felíciter inveníre perpétuam.

\noindent Per Dóminum nostrum Iesum Christum, Fílium tuum, qui tecum vivit et regnat in unitáte Spíritus Sancti, Deus, per ómnia sǽcula sæculórum.

\noindent \Rbardot{} Amen.}
\newcommand{\benedictus}{\pars{Canticum Zachariæ.} \scriptura{Lc. 2, 13-14; \textbf{H50}}

\vspace{-4mm}

{
\grechangedim{interwordspacetext}{0.18 cm plus 0.15 cm minus 0.05 cm}{scalable}%
\antiphona{VII d}{temporalia/ant-factaestcumangelo.gtex}
\grechangedim{interwordspacetext}{0.22 cm plus 0.15 cm minus 0.05 cm}{scalable}%
}

%\vspace{-3mm}

\scriptura{Lc. 1, 68-79}

%\vspace{-2mm}

\cantusSineNeumas
\initiumpsalmi{temporalia/benedictus-initium-vii-d-auto.gtex}

%\vspace{-1.5mm}

\input{temporalia/benedictus-vii-d.tex}

\vfill

{
\grechangedim{interwordspacetext}{0.18 cm plus 0.15 cm minus 0.05 cm}{scalable}%
\antiphona{}{temporalia/ant-factaestcumangelo.gtex}
\grechangedim{interwordspacetext}{0.22 cm plus 0.15 cm minus 0.05 cm}{scalable}%
}}
\newcommand{\preces}{\noindent Christum Dóminum, \gredagger{} cuius grátia ómnibus homínibus appáruit, \grestar{} cum húmili fidúcia implorémus:

\Rbardot{} Dómine, miserére nostri.

\noindent Christe, a Patre génite ante ómnia sǽcula, \gredagger{} splendor glóriæ eius, imágo eius substántiæ, \gredagger{} qui verbo tuo univérsa susténtas, \grestar{} te rogámus, ut Evangélio tuo hunc diem vivífices.

\Rbardot{} Dómine, miserére nostri.

\noindent Christe, in hoc mundo quando venit plenitúdo témporis nate, \gredagger{} ad salvándum humánum genus et ad liberándam creatúram univérsam,\grestar{} te rogámus, ut præstes ómnibus libertátem.

\Rbardot{} Dómine, miserére nostri.

\noindent Christe, Fili consubstantiális Patris, ante lucíferum génite, \gredagger{} qui in Béthlehem natus es, ut adimpleréntur Scriptúræ, \grestar{} te rogámus, ut paupértas effúlgeat in Ecclésia tua.

\Rbardot{} Dómine, miserére nostri.

\noindent Christe, Deus et homo, \gredagger{} qui Dóminus es David et fílius eius, prophetías adímplens, \grestar{} te rogámus, ut Israel te Messíam agnóscat.

\Rbardot{} Dómine, miserére nostri.}
\newcommand{\sinevesperas}{Sine Vesperas}
% LuaLaTeX

\documentclass[a4paper, twoside, 12pt]{article}
\usepackage[latin]{babel}
%\usepackage[landscape, left=3cm, right=1.5cm, top=2cm, bottom=1cm]{geometry} % okraje stranky
%\usepackage[landscape, a4paper, mag=1166, truedimen, left=2cm, right=1.5cm, top=1.6cm, bottom=0.95cm]{geometry} % okraje stranky
\usepackage[landscape, a4paper, mag=1400, truedimen, left=0.5cm, right=0.5cm, top=0.5cm, bottom=0.5cm]{geometry} % okraje stranky

\usepackage{fontspec}
\setmainfont[FeatureFile={junicode.fea}, Ligatures={Common, TeX}, RawFeature=+fixi]{Junicode}
%\setmainfont{Junicode}

% shortcut for Junicode without ligatures (for the Czech texts)
\newfontfamily\nlfont[FeatureFile={junicode.fea}, Ligatures={Common, TeX}, RawFeature=+fixi]{Junicode}

% Hebrew font:
% http://scripts.sil.org/cms/scripts/page.php?site_id=nrsi&id=SILHebrUnic2
\newfontfamily\hebfont[Scale=1]{Ezra SIL}

\usepackage{multicol}
\usepackage{color}
\usepackage{lettrine}
\usepackage{fancyhdr}

% usual packages loading:
\usepackage{luatextra}
\usepackage{graphicx} % support the \includegraphics command and options
\usepackage{gregoriotex} % for gregorio score inclusion
\usepackage{gregoriosyms}
\usepackage{wrapfig} % figures wrapped by the text
\usepackage{parcolumns}
\usepackage[contents={},opacity=1,scale=1,color=black]{background}
\usepackage{tikzpagenodes}
\usepackage{calc}
\usepackage{longtable}
\usetikzlibrary{calc}

\setlength{\headheight}{14.5pt}

% Commands used to produce a typical "Conventus" booklet

\newenvironment{titulusOfficii}{\begin{center}}{\end{center}}
\newcommand{\dies}[1]{#1

}
\newcommand{\nomenFesti}[1]{\textbf{\Large #1}

}
\newcommand{\celebratio}[1]{#1

}

\newcommand{\hora}[1]{%
\vspace{0.5cm}{\large \textbf{#1}}

\fancyhead[LE]{\thepage\ / #1}
\fancyhead[RO]{#1 / \thepage}
\addcontentsline{toc}{subsection}{#1}
}

% larger unit than a hora
\newcommand{\divisio}[1]{%
\begin{center}
{\Large \textsc{#1}}
\end{center}
\fancyhead[CO,CE]{#1}
\addcontentsline{toc}{section}{#1}
}

% a part of a hora, larger than pars
\newcommand{\subhora}[1]{
\begin{center}
{\large \textit{#1}}
\end{center}
%\fancyhead[CO,CE]{#1}
\addcontentsline{toc}{subsubsection}{#1}
}

% rubricated inline text
\newcommand{\rubricatum}[1]{\textit{#1}}

% standalone rubric
\newcommand{\rubrica}[1]{\vspace{3mm}\rubricatum{#1}}

\newcommand{\notitia}[1]{\textcolor{red}{#1}}

\newcommand{\scriptura}[1]{\hfill \small\textit{#1}}

\newcommand{\translatioCantus}[1]{\vspace{1mm}%
{\noindent\footnotesize \nlfont{#1}}}

% pruznejsi varianta nasledujiciho - umoznuje nastavit sirku sloupce
% s prekladem
\newcommand{\psalmusEtTranslatioB}[3]{
  \vspace{0.5cm}
  \begin{parcolumns}[colwidths={2=#3}, nofirstindent=true]{2}
    \colchunk{
      \input{#1}
    }

    \colchunk{
      \vspace{-0.5cm}
      {\footnotesize \nlfont
        \input{#2}
      }
    }
  \end{parcolumns}
}

\newcommand{\psalmusEtTranslatio}[2]{
  \psalmusEtTranslatioB{#1}{#2}{8.5cm}
}


\newcommand{\canticumMagnificatEtTranslatio}[1]{
  \psalmusEtTranslatioB{#1}{temporalia/extra-adventum-vespers/magnificat-boh.tex}{12cm}
}
\newcommand{\canticumBenedictusEtTranslatio}[1]{
  \psalmusEtTranslatioB{#1}{temporalia/extra-adventum-laudes/benedictus-boh.tex}{10.5cm}
}

% volne misto nad antifonami, kam si zpevaci dokresli neumy
\newcommand{\hicSuntNeumae}{\vspace{0.5cm}}

% prepinani mista mezi notovymi osnovami: pro neumovane a neneumovane zpevy
\newcommand{\cantusCumNeumis}{
  \setgrefactor{17}
  \global\advance\grespaceabovelines by 5mm%
}
\newcommand{\cantusSineNeumas}{
  \setgrefactor{17}
  \global\advance\grespaceabovelines by -5mm%
}

% znaky k umisteni nad inicialu zpevu
\newcommand{\superInitialam}[1]{\gresetfirstlineaboveinitial{\small {\textbf{#1}}}{\small {\textbf{#1}}}}

% pars officii, i.e. "oratio", ...
\newcommand{\pars}[1]{\textbf{#1}}

\newenvironment{psalmus}{
  \setlength{\parindent}{0pt}
  \setlength{\parskip}{5pt}
}{
  \setlength{\parindent}{10pt}
  \setlength{\parskip}{10pt}
}

%%%% Prejmenovat na latinske:
\newcommand{\nadpisZalmu}[1]{
  \hspace{2cm}\textbf{#1}\vspace{2mm}%
  \nopagebreak%

}

% mode, score, translation
\newcommand{\antiphona}[3]{%
\hicSuntNeumae
\superInitialam{#1}
\includescore{#2}

#3
}
 % Often used macros

\newcommand{\annusEditionis}{2021}

\def\hebinitial#1{%
\leavevmode{\newbox\hebbox\setbox\hebbox\hbox{\hebfont{#1}\hskip 1mm}\kern -\wd\hebbox\hbox{\hebfont{#1}\hskip 1mm}}%
}

%%%% Vicekrat opakovane kousky

\newcommand{\anteOrationem}{
  \rubrica{Ante Orationem, cantatur a Superiore:}

  \pars{Supplicatio Litaniæ.}

  \cuminitiali{}{temporalia/supplicatiolitaniae.gtex}

  \pars{Oratio Dominica.}

  \cuminitiali{}{temporalia/oratiodominica.gtex}

  \rubrica{Deinde dicitur ab Hebdomadario:}

  \cuminitiali{}{temporalia/dominusvobiscum-solemnis.gtex}

  \rubrica{In choro monialium loco Dominus vobiscum dicitur:}

  \sineinitiali{temporalia/domineexaudi.gtex}
}

\setlength{\columnsep}{30pt} % prostor mezi sloupci

%%%%%%%%%%%%%%%%%%%%%%%%%%%%%%%%%%%%%%%%%%%%%%%%%%%%%%%%%%%%%%%%%%%%%%%%%%%%%%%%%%%%%%%%%%%%%%%%%%%%%%%%%%%%%
\begin{document}

% Here we set the space around the initial.
% Please report to http://home.gna.org/gregorio/gregoriotex/details for more details and options
\grechangedim{afterinitialshift}{2.2mm}{scalable}
\grechangedim{beforeinitialshift}{2.2mm}{scalable}
\grechangedim{interwordspacetext}{0.22 cm plus 0.15 cm minus 0.05 cm}{scalable}%
\grechangedim{annotationraise}{-0.2cm}{scalable}

% Here we set the initial font. Change 38 if you want a bigger initial.
% Emit the initials in red.
\grechangestyle{initial}{\color{red}\fontsize{38}{38}\selectfont}

\pagestyle{empty}

%%%% Titulni stranka
\begin{titulusOfficii}
\titulus
\end{titulusOfficii}

\vfill

\begin{center}
%Ad usum et secundum consuetudines chori \guillemotright{}Conventus Choralis\guillemotleft.

%Editio Sancti Wolfgangi \annusEditionis
\end{center}

\scriptura{}

\pars{}

\pagebreak

\renewcommand{\headrulewidth}{0pt} % no horiz. rule at the header
\fancyhf{}
\pagestyle{fancy}

\cantusSineNeumas

\pars{} \scriptura{}

\ifx\sinematutinum\undefined
\hora{Ad Matutinum.} %%%%%%%%%%%%%%%%%%%%%%%%%%%%%%%%%%%%%%%%%%%%%%%%%%%%%

\vspace{2mm}

\cuminitiali{}{temporalia/dominelabiamea.gtex}

\vfill
%\pagebreak

\vspace{2mm}

\ifx\invitatorium\undefined
\pars{Invitatorium.}

\vspace{-2mm}

\antiphona{E}{temporalia/inv-christusnatusest-simplex.gtex}
\else
\invitatorium
\fi

\vfill
\pagebreak

\ifx\hymnusmatutinum\undefined
\pars{Hymnus.}

{
\grechangedim{interwordspacetext}{0.10 cm plus 0.15 cm minus 0.05 cm}{scalable}%
\antiphona{IV}{temporalia/hym-CandorAEternae-simplex.gtex}
\grechangedim{interwordspacetext}{0.22 cm plus 0.15 cm minus 0.05 cm}{scalable}%
}

\vspace{-3mm}
\else
\hymnusmatutinum
\fi

\vfill
\pagebreak

\matutinum

\ifx\postoctavam\undefined
% Te Deum

\vspace{-5mm}

\ifx\tedeumsolemnis\undefined
\ifx\tedeumsimplex\undefined
\ifx\tedeummonasticum\undefined
{
\pars{Hymnus Ambrosianus} \scriptura{Alio modo, iuxta morem Romanum}

\vspace{-2mm}

\grechangedim{interwordspacetext}{0.26 cm plus 0.15 cm minus 0.05 cm}{scalable}%
\cuminitiali{III}{temporalia/tedeum-romanum-gn.gtex}
\grechangedim{interwordspacetext}{0.22 cm plus 0.15 cm minus 0.05 cm}{scalable}%
}
\else
{
\pars{Hymnus Ambrosianus} \scriptura{Tonus Monasticus}

\vspace{-2mm}

\grechangedim{interwordspacetext}{0.26 cm plus 0.15 cm minus 0.05 cm}{scalable}%
\cuminitiali{III}{temporalia/tedeum-monasticum-am34.gtex}
\grechangedim{interwordspacetext}{0.22 cm plus 0.15 cm minus 0.05 cm}{scalable}%
}
\fi
\else
{
\pars{Hymnus Ambrosianus} \scriptura{Tonus Simplex}

\vspace{-2mm}

\grechangedim{interwordspacetext}{0.30 cm plus 0.15 cm minus 0.05 cm}{scalable}%
\cuminitiali{III}{temporalia/tedeum-simplex-gn.gtex}
\grechangedim{interwordspacetext}{0.22 cm plus 0.15 cm minus 0.05 cm}{scalable}%
}
\fi
\else
{
\pars{Hymnus Ambrosianus} \scriptura{Tonus Solemnis}

\vspace{-2mm}

\grechangedim{interwordspacetext}{0.26 cm plus 0.15 cm minus 0.05 cm}{scalable}%
\cuminitiali{III}{temporalia/tedeum-solemnis-gn.gtex}
\grechangedim{interwordspacetext}{0.22 cm plus 0.15 cm minus 0.05 cm}{scalable}%
}
\fi

\vfill
\pagebreak
\fi

\rubrica{Reliqua omittuntur, nisi Laudes separandæ sint.}

\sineinitiali{temporalia/domineexaudi.gtex}

\vfill

\oratio

\vfill

\noindent \Vbardot{} Dómine, exáudi oratiónem meam.
\Rbardot{} Et clamor meus ad te véniat.

\vfill

% Nocturnale Romanum 2002, p. LXXVI Benedicamus Domino seems to match
% the one from Solemn Laudes.
\cuminitiali{V}{temporalia/benedicamus-solemnis-laud.gtex}

\vfill

\noindent \Vbardot{} Fidélium ánimæ per misericórdiam Dei requiéscant in pace.
\Rbardot{} Amen.

\vfill
\pagebreak
\fi

\ifx\sinelaudes\undefined
\hora{Ad Laudes.} %%%%%%%%%%%%%%%%%%%%%%%%%%%%%%%%%%%%%%%%%%%%%%%%%%%%%

\cantusSineNeumas

\vspace{0.5cm}
\grechangedim{interwordspacetext}{0.18 cm plus 0.15 cm minus 0.05 cm}{scalable}%
\ifx\postoctavam\undefined
\cuminitiali{}{temporalia/deusinadiutorium-alter.gtex}
\else
\cuminitiali{}{temporalia/deusinadiutorium-communis.gtex}
\fi
\grechangedim{interwordspacetext}{0.22 cm plus 0.15 cm minus 0.05 cm}{scalable}%

\vfill
%\pagebreak

\ifx\hymnuslaudes\undefined
\pagebreak
\pars{Hymnus} \scriptura{Sedulius}

\grechangedim{interwordspacetext}{0.16 cm plus 0.15 cm minus 0.05 cm}{scalable}%
\cuminitiali{III}{temporalia/hym-ASolisOrtus.gtex}
\grechangedim{interwordspacetext}{0.22 cm plus 0.15 cm minus 0.05 cm}{scalable}%
\vspace{-3mm}
\else
\hymnuslaudes
\fi

\vfill
\pagebreak

\ifx\laudes\undefined
\pars{Psalmus 1.} \scriptura{Lc. 2, 8.11.13.18; \textbf{H50}}

\vspace{-4mm}

\antiphona{II D}{temporalia/ant-quemvidistis.gtex}

\vspace{-2mm}

\scriptura{Psalmus 62.}

\vspace{-1mm}

\initiumpsalmi{temporalia/ps62-initium-ii-D-auto.gtex}

\input{temporalia/ps62-ii-D.tex} \Abardot{}

\vfill
\pagebreak

\pars{Psalmus 2.} \scriptura{Lc. 2, 10.11; \textbf{H50}}

\vspace{-4mm}

\antiphona{VII d}{temporalia/ant-angelusadpastores.gtex}

\scriptura{Canticum trium puerorum, Dan. 3, 57-88 et 56}

\initiumpsalmi{temporalia/dan3-initium-vii-d-auto.gtex}

\input{temporalia/dan3-vii-d-sinedox.tex}

\rubrica{Hic non dicitur Gloria Patri, neque Amen.}

\vfill

\vspace{-6mm}

\antiphona{}{temporalia/ant-angelusadpastores.gtex} % repeat the antiphon - new page

\vfill
\pagebreak

\pars{Psalmus 3.} \scriptura{Is. 9, 6; \textbf{H51}}

\vspace{-4mm}

\antiphona{VIII G\textsuperscript{2}}{temporalia/ant-parvulusfilius.gtex}

\scriptura{Psalmus 149.}

\initiumpsalmi{temporalia/ps149-initium-viii-G2-auto.gtex}

\input{temporalia/ps149-viii-G2.tex} \Abardot{}

\vfill
\pagebreak
\else
\laudes
\fi

\lectiobrevis

\vfill

\ifx\responsoriumbreve\undefined
\pars{Responsorium breve.} \scriptura{Ps. 97, 2}

\cuminitiali{VI}{temporalia/resp-notumfecit.gtex}
\else
\responsoriumbreve
\fi

\vfill
\pagebreak

\benedictus

\vspace{-1cm}

\vfill
\pagebreak

\pars{Preces.}

\sineinitiali{}{temporalia/tonusprecumnovum.gtex}

\preces

\vfill

\pars{Oratio Dominica.}

\cuminitiali{}{temporalia/oratiodominicaalt.gtex}

\vfill
\pagebreak

\rubrica{vel:}

\pars{Deprecatio Gelasii}

\vspace{-5mm}

\grechangedim{interwordspacetext}{0.16 cm plus 0.15 cm minus 0.05 cm}{scalable}%
\antiphona{D\textsuperscript{1}}{temporalia/deprecatio4-propace.gtex}
\grechangedim{interwordspacetext}{0.22 cm plus 0.15 cm minus 0.05 cm}{scalable}%

\vfill

\pars{Oratio Dominica.}

\cuminitiali{D}{temporalia/oratiodominica-d.gtex}

\vfill
\pagebreak

% Oratio. %%%
\oratio

\vspace{-1mm}

\vfill

\ifx\commemoratio\undefined
\else
\commemoratio
\fi

\rubrica{Hebdomadarius dicit Dominus vobiscum, vel, absente sacerdote vel diacono, sic concluditur:}

\vspace{2mm}

\antiphona{C}{temporalia/dominusnosbenedicat.gtex}

\rubrica{Postea cantatur a cantore:}

\vspace{2mm}

\ifx\benedicamuslaudes\undefined
\ifx\postoctavam\undefined
\cuminitiali{II}{temporalia/benedicamus-solemnism-laud.gtex}
\else
\cuminitiali{}{temporalia/benedicamus-tempore-nativitatis.gtex}
\fi
\else
\benedicamuslaudes
\fi

\vspace{1mm}

\vfill
\pagebreak
\fi

\ifx\sinevesperas\undefined
\hora{Ad Vesperas.} %%%%%%%%%%%%%%%%%%%%%%%%%%%%%%%%%%%%%%%%%%%%%%%%%%%%%

\cantusSineNeumas

%\vspace{-2mm}
\grechangedim{interwordspacetext}{0.18 cm plus 0.15 cm minus 0.05 cm}{scalable}%
\ifx\postoctavam\undefined
\cuminitiali{}{temporalia/deusinadiutorium-solemnis.gtex}
\else
\cuminitiali{}{temporalia/deusinadiutorium-communis.gtex}
\fi
\grechangedim{interwordspacetext}{0.22 cm plus 0.15 cm minus 0.05 cm}{scalable}%

\vfill
%\pagebreak

\vspace{-2mm}

\ifx\vesperas\undefined
\pars{Psalmus 1.} \scriptura{Ps. 109, 3; \textbf{H52}}

\vspace{-5mm}

\antiphona{I g}{temporalia/ant-tecumprincipium.gtex}

\scriptura{Psalmus 109.}

\initiumpsalmi{temporalia/ps109-initium-i-g-auto.gtex}

\vspace{-1.5mm}

\input{temporalia/ps109-i-g.tex} \Abardot{}

\vfill
\pagebreak

\pars{Psalmus 2.} \scriptura{Ps. 110, 9; \textbf{H52}}

\vspace{-4mm}

\antiphona{VII a}{temporalia/ant-redemptionemmisit.gtex}

\scriptura{Psalmus 110.}

\initiumpsalmi{temporalia/ps110-initium-vii-a-auto.gtex}

\input{temporalia/ps110-vii-a.tex} \Abardot{}

\vfill
\pagebreak

\pars{Psalmus 3.} \scriptura{Ps. 111, 4; \textbf{H52}}

\vspace{-4mm}

\antiphona{VII d}{temporalia/ant-exortumest.gtex}

\scriptura{Psalmus 111.}

\initiumpsalmi{temporalia/ps111-initium-vii-d-auto.gtex}

\input{temporalia/ps111-vii-d.tex} \Abardot{}

\vfill
\pagebreak

\ifx\impar\undefined
\pars{Psalmus 4.} \scriptura{Ps. 131, 11; \textbf{H52}}

\vspace{-4mm}

\antiphona{VIII G}{temporalia/ant-defructuventris.gtex}

\scriptura{Psalmus 131.}

\initiumpsalmi{temporalia/ps131-initium-viii-G-auto.gtex}

\input{temporalia/ps131-viii-G.tex}

\vfill

\antiphona{}{temporalia/ant-defructuventris.gtex}
\else
\pars{Psalmus 4.} \scriptura{Ps. 129, 7; \textbf{H52}}

\vspace{-4mm}

\antiphona{II* b}{temporalia/ant-apuddominum.gtex}

\scriptura{Psalmus 129.}

\initiumpsalmi{temporalia/ps129-initium-ii_-B-auto.gtex}

\input{temporalia/ps129-ii_-B.tex} \Abardot{}
\fi

\vfill
\pagebreak
\else
\vesperas
\fi

\ifx\capitulum\undefined
\pars{Capitulum.} \scriptura{Hebr. 1, 1-2}

\grechangedim{interwordspacetext}{0.12 cm plus 0.15 cm minus 0.05 cm}{scalable}%
\cuminitiali{}{temporalia/capitulum-Multifariam.gtex}
\grechangedim{interwordspacetext}{0.22 cm plus 0.15 cm minus 0.05 cm}{scalable}
\else
\capitulum
\fi

\vfill

\ifx\responsoriumbrevevesp\undefined
\pars{Responsorium breve.} \scriptura{Io. 1, 14}

\cuminitiali{VI}{temporalia/resp-verbumcaro-simplex.gtex}
\else
\responsoriumbrevevesp
\fi

\vfill
\pagebreak

\ifx\hymnusvesperas\undefined
\pars{Hymnus}

\cuminitiali{I}{temporalia/hym-ChristeRedemptor.gtex}
\else
\hymnusvesperas
\fi
\vspace{-3mm}

\vfill
%\pagebreak

\ifx\vespversus\undefined
\pars{Versus.} \scriptura{Ps. 97, 2}

% Versus. %%%
\sineinitiali{temporalia/versus-notumfecit-communis.gtex}
\else
\vespversus
\fi

\vfill
\pagebreak

\magnificat

\vfill
\pagebreak

\anteOrationem

\pagebreak

% Oratio. %%%
\ifx\oratioVesperas\undefined
\cuminitiali{}{temporalia/oratio.gtex}
\else
\oratioVesperas
\fi

\vspace{-1mm}

\vfill

\rubrica{Hebdomadarius dicit iterum Dominus vobiscum, vel cantor dicit:}

\vspace{2mm}

\sineinitiali{temporalia/domineexaudi.gtex}

\rubrica{Postea cantatur a cantore:}

\vspace{2mm}

\ifx\postoctavam\undefined
\cuminitiali{II}{temporalia/benedicamus-solemnism-2vesp.gtex}
\else
\cuminitiali{I}{temporalia/benedicamus-feria-vesperae.gtex}
\fi

\vspace{1mm}
\fi

\end{document}

