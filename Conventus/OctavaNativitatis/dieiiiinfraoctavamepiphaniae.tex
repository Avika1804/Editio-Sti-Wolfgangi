\newcommand{\titulus}{\dies{Die 8. Ianuarii.}
\nomenFesti{Die Tertia infra Octavam Epiphaniæ.}}
\newcommand{\postoctavam}{Post octavam}
\newcommand{\invitatorium}{\pars{Invitatorium.}

%\vspace{-6mm}

\antiphona{E}{temporalia/inv-christusapparuit.gtex}}
\newcommand{\hymnusmatutinum}{\pars{Hymnus.}

{
\grechangedim{interwordspacetext}{0.10 cm plus 0.15 cm minus 0.05 cm}{scalable}%
\antiphona{III}{temporalia/hym-MagiVidentes.gtex}
\grechangedim{interwordspacetext}{0.22 cm plus 0.15 cm minus 0.05 cm}{scalable}%
}}
\newcommand{\matutinum}{\pars{Psalmus 1.} \scriptura{Ps. 105, 4; \textbf{H100}}

\vspace{-4mm}

\antiphona{E}{temporalia/ant-visitanos.gtex}

%\vspace{-2mm}

\scriptura{Ps. 105, 1-15}

%\vspace{-2mm}

\initiumpsalmi{temporalia/ps105i-initium-e.gtex}

\input{temporalia/ps105i-e.tex}

\vfill

\antiphona{}{temporalia/ant-visitanos.gtex}

\vfill
\pagebreak

\pars{Psalmus 2.} \scriptura{Ps. 117, 6; \textbf{H156}}

\vspace{-8mm}

\antiphona{VIII G}{temporalia/ant-dominusmihi.gtex}

\vspace{-3mm}

\scriptura{Ps. 105, 16-31}

\vspace{-2.5mm}

\initiumpsalmi{temporalia/ps105ii-initium-viii-G-auto.gtex}

\vspace{-1.5mm}

\input{temporalia/ps105ii-viii-G.tex} \Abardot{}

\vspace{-5mm}

\vfill
\pagebreak

\pars{Psalmus 3.} \scriptura{Ps. 105, 44}

\vspace{-4mm}

\antiphona{VII a}{temporalia/ant-cumtribularentur.gtex}

%\vspace{-2mm}

\scriptura{Ps. 105, 32-48}

%\vspace{-2mm}

\initiumpsalmi{temporalia/ps105iii-initium-vii-a-auto.gtex}

\input{temporalia/ps105iii-vii-a.tex}

\vfill

\antiphona{}{temporalia/ant-cumtribularentur.gtex}

\vfill
\pagebreak

\pars{Versus.}

\noindent \Vbardot{} Lauda, Ierúsalem, Dóminum.

\noindent \Rbardot{} Qui emíttit elóquium suum in terra.

\vspace{5mm}

\sineinitiali{temporalia/oratiodominica-mat.gtex}

\vspace{5mm}

\pars{Absolutio.}

\cuminitiali{}{temporalia/absolutio-exaudi.gtex}

\vfill
\pagebreak

\cuminitiali{}{temporalia/benedictio-solemn-benedictione.gtex}

\vspace{7mm}

\pars{Lectio I.} \scriptura{Is. 62, 1-12}

\noindent De libro Isaíæ prophétæ.

\noindent Propter Sion non tacébo et propter Ierúsalem non quiéscam, donec egrediátur ut splendor iustítia eius, et salus eius ut lampas accendátur. Et vidébunt gentes iustítiam tuam et cuncti reges glóriam tuam; et vocáberis nómine novo, quod os Dómini nominábit. Et eris coróna glóriæ in manu Dómini et diadéma regni in manu Dei tui. Non vocáberis ultra Derelícta, et terra tua non vocábitur ámplius Desoláta; sed vocáberis Beneplácitum meum in ea, et terra tua Nupta, quia complácuit Dómino in te, et terra tua erit nupta. Nam ut iúvenis uxórem ducit vírginem, ita ducent te fílii tui; ut gaudet sponsus super sponsam, ita gaudébit super te Deus tuus. Super muros tuos Ierúsalem constítui custódes; tota die et tota nocte, in perpétuo non tacébunt. Qui commonétis Dóminum, ne taceátis et ne detis siléntium ei, donec stabíliat et donec ponat Ierúsalem laudem in terra. Iurávit Dóminus in déxtera sua et in bráchio fortitúdinis suæ: «Non dabo tríticum tuum ultra cibum inimícis tuis, neque bibent fílii aliéni vinum tuum, in quo laborásti. Quia, qui collígerint illud, cómedent et laudábunt Dóminum; et qui vindémiam fécerint, illud bibent in átriis sanctuárii mei. Transíte, transíte per portas, paráte viam pópulo. Stérnite, stérnite sémitam, elígite lápides, eleváte signum ad pópulos». Ecce Dóminus audítum fecit in extrémis terræ: «Dícite fíliæ Sion: Ecce salus tua venit, ecce merces eius cum eo et prǽmium eius coram illa. Et vocábunt eos Pópulus sanctus, Redémpti a Dómino; tu autem vocáberis Quæsíta, Cívitas non derelícta».

\noindent \Vbardot{} Tu autem, Dómine, miserére nobis.
\noindent \Rbardot{} Deo grátias.

\vfill
\pagebreak

\pars{Responsorium 1.} \scriptura{\Rbar{} Mt. 2, 9-10 \Vbar{} ibid. 2, 11; \textbf{H73}}

\vspace{-5mm}

\responsorium{VIII}{temporalia/resp-stellaquamviderunt-CROCHU.gtex}{}

\vfill
\pagebreak

\cuminitiali{}{temporalia/benedictio-solemn-unigenitus.gtex}

\vspace{7mm}

\pars{Lectio II.} \scriptura{Nn. 2. 6-8. 10: PG 10, 854. 858-859. 862}

\noindent Ex Sermóne in sancta Theophanía sancto Hippólyto presbýtero attribúto.

\noindent Ad Ioánnem venit Iesus, et baptísmum ab eo suscépit. O res digníssima admiratióne! Infinítum flumen, quod lætíficat civitátem Dei exígua ablúitur aqua. Fons incomprehensíbilis, qui vitam ómnibus homínibus progérminat et término caret, a parvis et temporáriis aquis obrúitur.

\noindent Qui ubíque præsens est nec usquam abest, incomprehensíbilis ángelis et ab hóminum conspéctu remótus, ad baptísmum accédit ut ipsi plácuit. Et ecce apérti sunt ei cæli et vox facta est dicens: Hic est Fílius meus diléctus, in quo complácui.

\noindent Amátus amórem génerat, et lux immateriális lucem inaccessíbilem. Hic est qui Ioséphi nominátus est fílius, et meus est Unigénitus secúndum divínam esséntiam.

\noindent Hic est Fílius meus diléctus: esúriens ille, qui innúmera alit mília; labórans, idémque récreans laborántes; non habens ubi caput reclínet suum, et ómnia gerens manu; qui pátitur, et ómnibus medétur passiónibus; qui cǽditur cólaphis, et mundum donat libertáte; qui in látere percútitur, et latus Adámi córrigit.

\noindent Sed mentem, quæso, mihi accuráte inténdite: volo enim recúrrere ad fontem vitæ, et fontem medélas scaturiéntem contemplári.

\noindent \Vbardot{} Tu autem, Dómine, miserére nobis.
\noindent \Rbardot{} Deo grátias.

\vfill
\pagebreak

\pars{Responsorium 2.} \scriptura{\Rbar{} Mt. 3, 16-17 \Vbar{} ibidem; \textbf{H75}}

\vspace{-5mm}

\responsorium{II}{temporalia/resp-incolumbaespecie-CROCHU.gtex}{}

\vfill
\pagebreak

\cuminitiali{}{temporalia/benedictio-solemn-spiritus.gtex}

\vspace{7mm}

\pars{Lectio III.}

\noindent Pater immortalitátis immortálem Fílium ac Verbum in mundum misit, qui venit ad hómines, lotúrus eos aqua et Spíritu: et regeneratúrus ad ánimæ corporísque incorruptibilitátem, inspirávit in nos spíritum vitæ, et incorruptíbili armatúra nos índuit.

\noindent Si ígitur homo factus immortális est, deus étiam erit. Si vero per aquam et Spíritum Sanctum a regeneratióne ex lavácro deus fit, comperítur étiam post resurrectiónem a mórtuis cohéres Christi esse.

\noindent Igitur præcónis voce proclámo: Veníte, omnes tribus géntium, ad baptísmatis immortalitátem. Hæc est aqua cum Spíritu coniúncta, qua paradísus rigátur, terra pinguéscit, increméntum plantæ cápiunt, génerant animália; atque ut ómnia compéndio ampléctar, per quam regenerátus homo vivificátur, qua Christus baptizátus est, in quam Spíritus Sanctus colúmbæ spécie descéndit.

\noindent Qui enim cum fide in hoc regeneratiónis lavácrum descéndit, renúntiat diábolo, et Christo se addícit; hostem ábnegat, at Christum Deum esse confitétur; servitútem éxuit, índuit adoptiónem; redit ex baptísmo spléndidus ut sol, rádios iustítiæ effúlgurans; quod vero máximum est, revértitur fílius Dei et Christi cohéres.

\noindent Ipsi glória et poténtia cum sanctíssimo, bono et vivífico eius Spíritu, nunc sit et semper et in ómnia sǽcula sæculórum. Amen.


\noindent \Vbardot{} Tu autem, Dómine, miserére nobis.
\noindent \Rbardot{} Deo grátias.

\vfill
\pagebreak

\pars{Responsorium 3.} \scriptura{\Vbardot{} Mt. 2, 11; \textbf{H74}}

\vspace{-5mm}

\responsorium{VII}{temporalia/resp-hicestdiespraeclarus-CROCHU-cumdox.gtex}{}

\vfill
\pagebreak}
\newcommand{\hymnuslaudes}{\pagebreak
\pars{Hymnus}

\grechangedim{interwordspacetext}{0.16 cm plus 0.15 cm minus 0.05 cm}{scalable}%
\cuminitiali{III}{temporalia/hym-QuicumqueChristum.gtex}
\grechangedim{interwordspacetext}{0.22 cm plus 0.15 cm minus 0.05 cm}{scalable}%
\vspace{-3mm}}
\newcommand{\laudes}{\pars{Psalmus 1.} \scriptura{Ps. 91, 6}

\vspace{-4.5mm}

\antiphona{E}{temporalia/ant-quammagnificatasunt.gtex}

\vspace{-3mm}

\scriptura{Psalmus 91.}

\vspace{-2mm}

\initiumpsalmi{temporalia/ps91-initium-e.gtex}

\vspace{-1.5mm}

\input{temporalia/ps91-e.tex} \Abardot{}

\vfill
\pagebreak

\pars{Psalmus 2.} \scriptura{Dt. 32, 3}

%\vspace{-4mm}

\antiphona{VI F}{temporalia/ant-datemagnitudinem.gtex}

\vspace{-4mm}

\scriptura{Canticum Moysi, Dt. 32, 1-32}

\initiumpsalmi{temporalia/moysis2i_xii-initium-vi-F-auto.gtex}

\input{temporalia/moysis2i_xii-vi-F.tex}

\vfill

\antiphona{}{temporalia/ant-datemagnitudinem.gtex}

\vfill
\pagebreak

\pars{Psalmus 3.} \scriptura{Ps. 8, 2}

\vspace{-4mm}

\antiphona{I g}{temporalia/ant-quamadmirabileest.gtex}

%\vspace{-2mm}

\scriptura{Ps. 8}

%\vspace{-2mm}

\initiumpsalmi{temporalia/ps8-initium-i-g-auto.gtex}

\input{temporalia/ps8-i-g.tex} \Abardot{}

\vfill
\pagebreak}
\newcommand{\lectiobrevis}{\pars{Lectio Brevis.} \scriptura{Is. 4, 2-3}

\noindent In die illa erit germen Dómini in splendórem et glóriam et fructus terræ sublímis et exsultátio his, qui salváti fúerint de Israel. Et erit: omnis qui relíctus fúerit in Sion et resíduus in Ierúsalem, sanctus vocábitur, omnis, qui scriptus est ad vitam in Ierúsalem.}
\newcommand{\responsoriumbreve}{\pars{Responsorium breve.} \scriptura{Ps. 71, 11}

\cuminitiali{VI}{temporalia/resp-adorabunteum.gtex}}
\newcommand{\oratio}{\pars{Oratio.}

\noindent Deus, cuius Unigénitus in substántia nostræ carnis appáruit, præsta, quǽsumus, ut per eum, quem símilem nobis foris agnóvimus, intus reformári mereámur.

\noindent Qui tecum vivit et regnat in unitáte Spíritus Sancti, Deus, per ómnia sǽcula sæculórum.

\noindent \Rbardot{} Amen.}
\newcommand{\benedictus}{\pars{Canticum Zachariæ.} \scriptura{Mt. 2, 11; \textbf{H74}}

\vspace{-4mm}

{
\grechangedim{interwordspacetext}{0.18 cm plus 0.15 cm minus 0.05 cm}{scalable}%
\antiphona{IV E}{temporalia/ant-triasuntmunera.gtex}
\grechangedim{interwordspacetext}{0.22 cm plus 0.15 cm minus 0.05 cm}{scalable}%
}

%\vspace{-3mm}

\scriptura{Lc. 1, 68-79}

%\vspace{-2mm}

\cantusSineNeumas
\initiumpsalmi{temporalia/benedictus-initium-iv-E-auto.gtex}

\vspace{-1.5mm}

\input{temporalia/benedictus-iv-E.tex} \Abardot{}}
\newcommand{\preces}{\noindent Misericórdiam Christi celebrémus,~\gredagger{} qui venit ut creatúra liberarétur a servitúte corruptiónis in libertátem filiórum Dei.~\grestar{} Hac divína freti pietáte, rogémus:

\Rbardot{} Per nativitátem tuam, líbera nos a malo.

\noindent Dómine, qui, ab ætérno exsístens, novam vitam ingréssus es,~\grestar{} rénova nos semper per mystérium natális tui.

\Rbardot{} Per nativitátem tuam, líbera nos a malo.

\noindent Qui divinitátem non amíttens, humanitátem mirabíliter assumpsísti,~\grestar{} præsta, ut vita nostra ad pleniórem divinitátis tuæ participatiónem nitátur.

\Rbardot{} Per nativitátem tuam, líbera nos a malo.

\noindent Qui véniens, lumen géntium et magíster sanctitátis factus es,~\grestar{} præsta, ut sermo tuus sit lucérna pédibus nostris.

\Rbardot{} Per nativitátem tuam, líbera nos a malo.

\noindent Verbum Dei, quod in sinu Maríæ Vírginis caro factum es et in hunc mundum venísti,~\grestar{} in córdibus nostris per fidem semper inhabitáre dignéris.

\Rbardot{} Per nativitátem tuam, líbera nos a malo.}
\newcommand{\sinevesperas}{Sine Vesperas}
% LuaLaTeX

\documentclass[a4paper, twoside, 12pt]{article}
\usepackage[latin]{babel}
%\usepackage[landscape, left=3cm, right=1.5cm, top=2cm, bottom=1cm]{geometry} % okraje stranky
%\usepackage[landscape, a4paper, mag=1166, truedimen, left=2cm, right=1.5cm, top=1.6cm, bottom=0.95cm]{geometry} % okraje stranky
\usepackage[landscape, a4paper, mag=1400, truedimen, left=0.5cm, right=0.5cm, top=0.5cm, bottom=0.5cm]{geometry} % okraje stranky

\usepackage{fontspec}
\setmainfont[FeatureFile={junicode.fea}, Ligatures={Common, TeX}, RawFeature=+fixi]{Junicode}
%\setmainfont{Junicode}

% shortcut for Junicode without ligatures (for the Czech texts)
\newfontfamily\nlfont[FeatureFile={junicode.fea}, Ligatures={Common, TeX}, RawFeature=+fixi]{Junicode}

% Hebrew font:
% http://scripts.sil.org/cms/scripts/page.php?site_id=nrsi&id=SILHebrUnic2
\newfontfamily\hebfont[Scale=1]{Ezra SIL}

\usepackage{multicol}
\usepackage{color}
\usepackage{lettrine}
\usepackage{fancyhdr}

% usual packages loading:
\usepackage{luatextra}
\usepackage{graphicx} % support the \includegraphics command and options
\usepackage{gregoriotex} % for gregorio score inclusion
\usepackage{gregoriosyms}
\usepackage{wrapfig} % figures wrapped by the text
\usepackage{parcolumns}
\usepackage[contents={},opacity=1,scale=1,color=black]{background}
\usepackage{tikzpagenodes}
\usepackage{calc}
\usepackage{longtable}
\usetikzlibrary{calc}

\setlength{\headheight}{14.5pt}

% Commands used to produce a typical "Conventus" booklet

\newenvironment{titulusOfficii}{\begin{center}}{\end{center}}
\newcommand{\dies}[1]{#1

}
\newcommand{\nomenFesti}[1]{\textbf{\Large #1}

}
\newcommand{\celebratio}[1]{#1

}

\newcommand{\hora}[1]{%
\vspace{0.5cm}{\large \textbf{#1}}

\fancyhead[LE]{\thepage\ / #1}
\fancyhead[RO]{#1 / \thepage}
\addcontentsline{toc}{subsection}{#1}
}

% larger unit than a hora
\newcommand{\divisio}[1]{%
\begin{center}
{\Large \textsc{#1}}
\end{center}
\fancyhead[CO,CE]{#1}
\addcontentsline{toc}{section}{#1}
}

% a part of a hora, larger than pars
\newcommand{\subhora}[1]{
\begin{center}
{\large \textit{#1}}
\end{center}
%\fancyhead[CO,CE]{#1}
\addcontentsline{toc}{subsubsection}{#1}
}

% rubricated inline text
\newcommand{\rubricatum}[1]{\textit{#1}}

% standalone rubric
\newcommand{\rubrica}[1]{\vspace{3mm}\rubricatum{#1}}

\newcommand{\notitia}[1]{\textcolor{red}{#1}}

\newcommand{\scriptura}[1]{\hfill \small\textit{#1}}

\newcommand{\translatioCantus}[1]{\vspace{1mm}%
{\noindent\footnotesize \nlfont{#1}}}

% pruznejsi varianta nasledujiciho - umoznuje nastavit sirku sloupce
% s prekladem
\newcommand{\psalmusEtTranslatioB}[3]{
  \vspace{0.5cm}
  \begin{parcolumns}[colwidths={2=#3}, nofirstindent=true]{2}
    \colchunk{
      \input{#1}
    }

    \colchunk{
      \vspace{-0.5cm}
      {\footnotesize \nlfont
        \input{#2}
      }
    }
  \end{parcolumns}
}

\newcommand{\psalmusEtTranslatio}[2]{
  \psalmusEtTranslatioB{#1}{#2}{8.5cm}
}


\newcommand{\canticumMagnificatEtTranslatio}[1]{
  \psalmusEtTranslatioB{#1}{temporalia/extra-adventum-vespers/magnificat-boh.tex}{12cm}
}
\newcommand{\canticumBenedictusEtTranslatio}[1]{
  \psalmusEtTranslatioB{#1}{temporalia/extra-adventum-laudes/benedictus-boh.tex}{10.5cm}
}

% volne misto nad antifonami, kam si zpevaci dokresli neumy
\newcommand{\hicSuntNeumae}{\vspace{0.5cm}}

% prepinani mista mezi notovymi osnovami: pro neumovane a neneumovane zpevy
\newcommand{\cantusCumNeumis}{
  \setgrefactor{17}
  \global\advance\grespaceabovelines by 5mm%
}
\newcommand{\cantusSineNeumas}{
  \setgrefactor{17}
  \global\advance\grespaceabovelines by -5mm%
}

% znaky k umisteni nad inicialu zpevu
\newcommand{\superInitialam}[1]{\gresetfirstlineaboveinitial{\small {\textbf{#1}}}{\small {\textbf{#1}}}}

% pars officii, i.e. "oratio", ...
\newcommand{\pars}[1]{\textbf{#1}}

\newenvironment{psalmus}{
  \setlength{\parindent}{0pt}
  \setlength{\parskip}{5pt}
}{
  \setlength{\parindent}{10pt}
  \setlength{\parskip}{10pt}
}

%%%% Prejmenovat na latinske:
\newcommand{\nadpisZalmu}[1]{
  \hspace{2cm}\textbf{#1}\vspace{2mm}%
  \nopagebreak%

}

% mode, score, translation
\newcommand{\antiphona}[3]{%
\hicSuntNeumae
\superInitialam{#1}
\includescore{#2}

#3
}
 % Often used macros

\newcommand{\annusEditionis}{2021}

\def\hebinitial#1{%
\leavevmode{\newbox\hebbox\setbox\hebbox\hbox{\hebfont{#1}\hskip 1mm}\kern -\wd\hebbox\hbox{\hebfont{#1}\hskip 1mm}}%
}

%%%% Vicekrat opakovane kousky

\newcommand{\anteOrationem}{
  \rubrica{Ante Orationem, cantatur a Superiore:}

  \pars{Supplicatio Litaniæ.}

  \cuminitiali{}{temporalia/supplicatiolitaniae.gtex}

  \pars{Oratio Dominica.}

  \cuminitiali{}{temporalia/oratiodominica.gtex}

  \rubrica{Deinde dicitur ab Hebdomadario:}

  \cuminitiali{}{temporalia/dominusvobiscum-solemnis.gtex}

  \rubrica{In choro monialium loco Dominus vobiscum dicitur:}

  \sineinitiali{temporalia/domineexaudi.gtex}
}

\setlength{\columnsep}{30pt} % prostor mezi sloupci

%%%%%%%%%%%%%%%%%%%%%%%%%%%%%%%%%%%%%%%%%%%%%%%%%%%%%%%%%%%%%%%%%%%%%%%%%%%%%%%%%%%%%%%%%%%%%%%%%%%%%%%%%%%%%
\begin{document}

% Here we set the space around the initial.
% Please report to http://home.gna.org/gregorio/gregoriotex/details for more details and options
\grechangedim{afterinitialshift}{2.2mm}{scalable}
\grechangedim{beforeinitialshift}{2.2mm}{scalable}
\grechangedim{interwordspacetext}{0.22 cm plus 0.15 cm minus 0.05 cm}{scalable}%
\grechangedim{annotationraise}{-0.2cm}{scalable}

% Here we set the initial font. Change 38 if you want a bigger initial.
% Emit the initials in red.
\grechangestyle{initial}{\color{red}\fontsize{38}{38}\selectfont}

\pagestyle{empty}

%%%% Titulni stranka
\begin{titulusOfficii}
\titulus
\end{titulusOfficii}

\vfill

\begin{center}
%Ad usum et secundum consuetudines chori \guillemotright{}Conventus Choralis\guillemotleft.

%Editio Sancti Wolfgangi \annusEditionis
\end{center}

\scriptura{}

\pars{}

\pagebreak

\renewcommand{\headrulewidth}{0pt} % no horiz. rule at the header
\fancyhf{}
\pagestyle{fancy}

\cantusSineNeumas

\pars{} \scriptura{}

\ifx\sinematutinum\undefined
\hora{Ad Matutinum.} %%%%%%%%%%%%%%%%%%%%%%%%%%%%%%%%%%%%%%%%%%%%%%%%%%%%%

\vspace{2mm}

\cuminitiali{}{temporalia/dominelabiamea.gtex}

\vfill
%\pagebreak

\vspace{2mm}

\ifx\invitatorium\undefined
\pars{Invitatorium.}

\vspace{-2mm}

\antiphona{E}{temporalia/inv-christusnatusest-simplex.gtex}
\else
\invitatorium
\fi

\vfill
\pagebreak

\ifx\hymnusmatutinum\undefined
\pars{Hymnus.}

{
\grechangedim{interwordspacetext}{0.10 cm plus 0.15 cm minus 0.05 cm}{scalable}%
\antiphona{IV}{temporalia/hym-CandorAEternae-simplex.gtex}
\grechangedim{interwordspacetext}{0.22 cm plus 0.15 cm minus 0.05 cm}{scalable}%
}

\vspace{-3mm}
\else
\hymnusmatutinum
\fi

\vfill
\pagebreak

\matutinum

\ifx\postoctavam\undefined
% Te Deum

\vspace{-5mm}

\ifx\tedeumsolemnis\undefined
\ifx\tedeumsimplex\undefined
\ifx\tedeummonasticum\undefined
{
\pars{Hymnus Ambrosianus} \scriptura{Alio modo, iuxta morem Romanum}

\vspace{-2mm}

\grechangedim{interwordspacetext}{0.26 cm plus 0.15 cm minus 0.05 cm}{scalable}%
\cuminitiali{III}{temporalia/tedeum-romanum-gn.gtex}
\grechangedim{interwordspacetext}{0.22 cm plus 0.15 cm minus 0.05 cm}{scalable}%
}
\else
{
\pars{Hymnus Ambrosianus} \scriptura{Tonus Monasticus}

\vspace{-2mm}

\grechangedim{interwordspacetext}{0.26 cm plus 0.15 cm minus 0.05 cm}{scalable}%
\cuminitiali{III}{temporalia/tedeum-monasticum-am34.gtex}
\grechangedim{interwordspacetext}{0.22 cm plus 0.15 cm minus 0.05 cm}{scalable}%
}
\fi
\else
{
\pars{Hymnus Ambrosianus} \scriptura{Tonus Simplex}

\vspace{-2mm}

\grechangedim{interwordspacetext}{0.30 cm plus 0.15 cm minus 0.05 cm}{scalable}%
\cuminitiali{III}{temporalia/tedeum-simplex-gn.gtex}
\grechangedim{interwordspacetext}{0.22 cm plus 0.15 cm minus 0.05 cm}{scalable}%
}
\fi
\else
{
\pars{Hymnus Ambrosianus} \scriptura{Tonus Solemnis}

\vspace{-2mm}

\grechangedim{interwordspacetext}{0.26 cm plus 0.15 cm minus 0.05 cm}{scalable}%
\cuminitiali{III}{temporalia/tedeum-solemnis-gn.gtex}
\grechangedim{interwordspacetext}{0.22 cm plus 0.15 cm minus 0.05 cm}{scalable}%
}
\fi

\vfill
\pagebreak
\fi

\rubrica{Reliqua omittuntur, nisi Laudes separandæ sint.}

\sineinitiali{temporalia/domineexaudi.gtex}

\vfill

\oratio

\vfill

\noindent \Vbardot{} Dómine, exáudi oratiónem meam.
\Rbardot{} Et clamor meus ad te véniat.

\vfill

% Nocturnale Romanum 2002, p. LXXVI Benedicamus Domino seems to match
% the one from Solemn Laudes.
\cuminitiali{V}{temporalia/benedicamus-solemnis-laud.gtex}

\vfill

\noindent \Vbardot{} Fidélium ánimæ per misericórdiam Dei requiéscant in pace.
\Rbardot{} Amen.

\vfill
\pagebreak
\fi

\ifx\sinelaudes\undefined
\hora{Ad Laudes.} %%%%%%%%%%%%%%%%%%%%%%%%%%%%%%%%%%%%%%%%%%%%%%%%%%%%%

\cantusSineNeumas

\vspace{0.5cm}
\grechangedim{interwordspacetext}{0.18 cm plus 0.15 cm minus 0.05 cm}{scalable}%
\ifx\postoctavam\undefined
\cuminitiali{}{temporalia/deusinadiutorium-alter.gtex}
\else
\cuminitiali{}{temporalia/deusinadiutorium-communis.gtex}
\fi
\grechangedim{interwordspacetext}{0.22 cm plus 0.15 cm minus 0.05 cm}{scalable}%

\vfill
%\pagebreak

\ifx\hymnuslaudes\undefined
\pagebreak
\pars{Hymnus} \scriptura{Sedulius}

\grechangedim{interwordspacetext}{0.16 cm plus 0.15 cm minus 0.05 cm}{scalable}%
\cuminitiali{III}{temporalia/hym-ASolisOrtus.gtex}
\grechangedim{interwordspacetext}{0.22 cm plus 0.15 cm minus 0.05 cm}{scalable}%
\vspace{-3mm}
\else
\hymnuslaudes
\fi

\vfill
\pagebreak

\ifx\laudes\undefined
\pars{Psalmus 1.} \scriptura{Lc. 2, 8.11.13.18; \textbf{H50}}

\vspace{-4mm}

\antiphona{II D}{temporalia/ant-quemvidistis.gtex}

\vspace{-2mm}

\scriptura{Psalmus 62.}

\vspace{-1mm}

\initiumpsalmi{temporalia/ps62-initium-ii-D-auto.gtex}

\input{temporalia/ps62-ii-D.tex} \Abardot{}

\vfill
\pagebreak

\pars{Psalmus 2.} \scriptura{Lc. 2, 10.11; \textbf{H50}}

\vspace{-4mm}

\antiphona{VII d}{temporalia/ant-angelusadpastores.gtex}

\scriptura{Canticum trium puerorum, Dan. 3, 57-88 et 56}

\initiumpsalmi{temporalia/dan3-initium-vii-d-auto.gtex}

\input{temporalia/dan3-vii-d-sinedox.tex}

\rubrica{Hic non dicitur Gloria Patri, neque Amen.}

\vfill

\vspace{-6mm}

\antiphona{}{temporalia/ant-angelusadpastores.gtex} % repeat the antiphon - new page

\vfill
\pagebreak

\pars{Psalmus 3.} \scriptura{Is. 9, 6; \textbf{H51}}

\vspace{-4mm}

\antiphona{VIII G\textsuperscript{2}}{temporalia/ant-parvulusfilius.gtex}

\scriptura{Psalmus 149.}

\initiumpsalmi{temporalia/ps149-initium-viii-G2-auto.gtex}

\input{temporalia/ps149-viii-G2.tex} \Abardot{}

\vfill
\pagebreak
\else
\laudes
\fi

\lectiobrevis

\vfill

\ifx\responsoriumbreve\undefined
\pars{Responsorium breve.} \scriptura{Ps. 97, 2}

\cuminitiali{VI}{temporalia/resp-notumfecit.gtex}
\else
\responsoriumbreve
\fi

\vfill
\pagebreak

\benedictus

\vspace{-1cm}

\vfill
\pagebreak

\pars{Preces.}

\sineinitiali{}{temporalia/tonusprecumnovum.gtex}

\preces

\vfill

\pars{Oratio Dominica.}

\cuminitiali{}{temporalia/oratiodominicaalt.gtex}

\vfill
\pagebreak

\rubrica{vel:}

\pars{Deprecatio Gelasii}

\vspace{-5mm}

\grechangedim{interwordspacetext}{0.16 cm plus 0.15 cm minus 0.05 cm}{scalable}%
\antiphona{D\textsuperscript{1}}{temporalia/deprecatio4-propace.gtex}
\grechangedim{interwordspacetext}{0.22 cm plus 0.15 cm minus 0.05 cm}{scalable}%

\vfill

\pars{Oratio Dominica.}

\cuminitiali{D}{temporalia/oratiodominica-d.gtex}

\vfill
\pagebreak

% Oratio. %%%
\oratio

\vspace{-1mm}

\vfill

\ifx\commemoratio\undefined
\else
\commemoratio
\fi

\rubrica{Hebdomadarius dicit Dominus vobiscum, vel, absente sacerdote vel diacono, sic concluditur:}

\vspace{2mm}

\antiphona{C}{temporalia/dominusnosbenedicat.gtex}

\rubrica{Postea cantatur a cantore:}

\vspace{2mm}

\ifx\benedicamuslaudes\undefined
\ifx\postoctavam\undefined
\cuminitiali{II}{temporalia/benedicamus-solemnism-laud.gtex}
\else
\cuminitiali{}{temporalia/benedicamus-tempore-nativitatis.gtex}
\fi
\else
\benedicamuslaudes
\fi

\vspace{1mm}

\vfill
\pagebreak
\fi

\ifx\sinevesperas\undefined
\hora{Ad Vesperas.} %%%%%%%%%%%%%%%%%%%%%%%%%%%%%%%%%%%%%%%%%%%%%%%%%%%%%

\cantusSineNeumas

%\vspace{-2mm}
\grechangedim{interwordspacetext}{0.18 cm plus 0.15 cm minus 0.05 cm}{scalable}%
\ifx\postoctavam\undefined
\cuminitiali{}{temporalia/deusinadiutorium-solemnis.gtex}
\else
\cuminitiali{}{temporalia/deusinadiutorium-communis.gtex}
\fi
\grechangedim{interwordspacetext}{0.22 cm plus 0.15 cm minus 0.05 cm}{scalable}%

\vfill
%\pagebreak

\vspace{-2mm}

\ifx\vesperas\undefined
\pars{Psalmus 1.} \scriptura{Ps. 109, 3; \textbf{H52}}

\vspace{-5mm}

\antiphona{I g}{temporalia/ant-tecumprincipium.gtex}

\scriptura{Psalmus 109.}

\initiumpsalmi{temporalia/ps109-initium-i-g-auto.gtex}

\vspace{-1.5mm}

\input{temporalia/ps109-i-g.tex} \Abardot{}

\vfill
\pagebreak

\pars{Psalmus 2.} \scriptura{Ps. 110, 9; \textbf{H52}}

\vspace{-4mm}

\antiphona{VII a}{temporalia/ant-redemptionemmisit.gtex}

\scriptura{Psalmus 110.}

\initiumpsalmi{temporalia/ps110-initium-vii-a-auto.gtex}

\input{temporalia/ps110-vii-a.tex} \Abardot{}

\vfill
\pagebreak

\pars{Psalmus 3.} \scriptura{Ps. 111, 4; \textbf{H52}}

\vspace{-4mm}

\antiphona{VII d}{temporalia/ant-exortumest.gtex}

\scriptura{Psalmus 111.}

\initiumpsalmi{temporalia/ps111-initium-vii-d-auto.gtex}

\input{temporalia/ps111-vii-d.tex} \Abardot{}

\vfill
\pagebreak

\ifx\impar\undefined
\pars{Psalmus 4.} \scriptura{Ps. 131, 11; \textbf{H52}}

\vspace{-4mm}

\antiphona{VIII G}{temporalia/ant-defructuventris.gtex}

\scriptura{Psalmus 131.}

\initiumpsalmi{temporalia/ps131-initium-viii-G-auto.gtex}

\input{temporalia/ps131-viii-G.tex}

\vfill

\antiphona{}{temporalia/ant-defructuventris.gtex}
\else
\pars{Psalmus 4.} \scriptura{Ps. 129, 7; \textbf{H52}}

\vspace{-4mm}

\antiphona{II* b}{temporalia/ant-apuddominum.gtex}

\scriptura{Psalmus 129.}

\initiumpsalmi{temporalia/ps129-initium-ii_-B-auto.gtex}

\input{temporalia/ps129-ii_-B.tex} \Abardot{}
\fi

\vfill
\pagebreak
\else
\vesperas
\fi

\ifx\capitulum\undefined
\pars{Capitulum.} \scriptura{Hebr. 1, 1-2}

\grechangedim{interwordspacetext}{0.12 cm plus 0.15 cm minus 0.05 cm}{scalable}%
\cuminitiali{}{temporalia/capitulum-Multifariam.gtex}
\grechangedim{interwordspacetext}{0.22 cm plus 0.15 cm minus 0.05 cm}{scalable}
\else
\capitulum
\fi

\vfill

\ifx\responsoriumbrevevesp\undefined
\pars{Responsorium breve.} \scriptura{Io. 1, 14}

\cuminitiali{VI}{temporalia/resp-verbumcaro-simplex.gtex}
\else
\responsoriumbrevevesp
\fi

\vfill
\pagebreak

\ifx\hymnusvesperas\undefined
\pars{Hymnus}

\cuminitiali{I}{temporalia/hym-ChristeRedemptor.gtex}
\else
\hymnusvesperas
\fi
\vspace{-3mm}

\vfill
%\pagebreak

\ifx\vespversus\undefined
\pars{Versus.} \scriptura{Ps. 97, 2}

% Versus. %%%
\sineinitiali{temporalia/versus-notumfecit-communis.gtex}
\else
\vespversus
\fi

\vfill
\pagebreak

\magnificat

\vfill
\pagebreak

\anteOrationem

\pagebreak

% Oratio. %%%
\ifx\oratioVesperas\undefined
\cuminitiali{}{temporalia/oratio.gtex}
\else
\oratioVesperas
\fi

\vspace{-1mm}

\vfill

\rubrica{Hebdomadarius dicit iterum Dominus vobiscum, vel cantor dicit:}

\vspace{2mm}

\sineinitiali{temporalia/domineexaudi.gtex}

\rubrica{Postea cantatur a cantore:}

\vspace{2mm}

\ifx\postoctavam\undefined
\cuminitiali{II}{temporalia/benedicamus-solemnism-2vesp.gtex}
\else
\cuminitiali{I}{temporalia/benedicamus-feria-vesperae.gtex}
\fi

\vspace{1mm}
\fi

\end{document}

