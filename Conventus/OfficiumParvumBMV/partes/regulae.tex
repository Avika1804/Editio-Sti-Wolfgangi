\hora{Regulæ generales}

\begin{multicols}{2}

1. Officium parvum B. Mariæ Virg. octo complectitur partes, quas Horas vocant;
Matutinum nempe, Laudes, Primam, Tertiam, Sextam, Nonam, Vesperas, denique
Completorium. Quamlibet autem Horam separatim ab alia recitare licet.

2. Horas hoc fere tempore recitare convenit, nempe Matutinum cum Laudibus 
pridie ab hora secunda pomeridiana, Primam, Tertiam, Sextam et Nonam mane,
Vesperas et Completorium post meridiem.
Tempore autem Quadragesimæ, id est a Sabbato~I in Quadragesima usque ad
Sabbatum sanctum inclusive, satius est Vesperas ante comestionem recitare.

3. Ad lucrandas indulgentias Officio parvo B.~Mariæ Virg. annexas, in publica
recitatione omnes lingua latina uti debent, in privata autem recitatione
qualibet versione fideli et probata uti valent.
Porro recitatio retinenda est adhuc privata, quamvis locum habeat in communi
intra sæpta domus religiosæ, immo et in ipsa Ecclesia vel publico Oratorio
prædictæ domui annexis, sed januis clausis (S.~C. Indulg. 13~Sept. 1888,
28~Aug. 1903, 18~Dec. 1906).

4. Quamvis Officium parvum B.~Mariæ Virg. infra annum sit unum idemque,
exceptis partibus propriis Tempore Adventus et post Nativitatem Domini,
communiter tamen in tria Officia distinguitur:
Primum ergo Officium est dicendum a Matutino diei 3 Februarii usque ad Nonam
Sabbati ante Dominicam I Adventus inclusive, præterquam in Festo Annuntiationis
B.~M.~V., in quo dicitur Officium ut in Adventu.
Secundum recitandum est a Vesperis Sabbati ante Dominicam~I Adventus usque ad
Nonam Vigiliæ Nativitatis Domini inclusive et in Festo Annuntiationis
B.~Mariæ Virg.
Tertium persolvendum est a Vesperis Diei 24 Decembris usque ad Completorium
diei 2 Februarii inclusive.

5. Hoc Officium quovis tempore persolvendum est plane, prout in Breviario
præscribitur. Quamobrem Tempore Passionis, incluso etiam ultimo Triduo 
sacro Majoris Hebdomadæ, non omittitur \rubricatum{Glória Patri} in Invitatorio
ac tertio Responsorio; nec Tempore Paschali numerus Antiphonarum imminuitur,
neque Invitatorio, Antiphonis, Versibus et Responsoriis additur in fine
\rubricatum{Allelúja} (Rubr. gen. Brev. tit.~37, num.~2. S.~R.~C. num. 1334 ad~6).

6. Tempore Paschali, idest a Vesperis Sabbati sancti usque ad Nonam Sabbati
infra Octavam Pentecostes inclusive, Officium parvum B.~Mariæ Virg.
dicitur sicuti per Annum; sed ad \rubricatum{Benedíctus}, ad \rubricatum{Magníficat}
et ad \rubricatum{Nunc dimíttis} dicitur Antiphona \rubricatum{Regína cæli}.

7. In Festo Annuntiationis (a Matutino usque ad Completorium inclusive)
idem Officium recitatur, quod præscribitur Tempore Adventus, et in fine
dicitur Antiphona \rubricatum{Ave, Regína cælórum} vel \rubricatum{Regina cæli}
juxta temporis diversitatem. Si hoc Festum celebretur Tempore Quadragesimæ,
loco \rubricatum{Allelúja} in principio omnium Horarum dicitur
\rubricatum{Laus tibi, Dómine, Rex ætérnæ glóriæ}.

8. Quando Officium parvum B.~Mariæ Virg. recitatur separatim ab Officio divino,
Hymnus \rubricatum{Te Deum} dicitur 
a Nativitate Domini usque ad Sabbatum ante Dominicam Septuagesimæ inclusive
et a Dominica Resurrectionis usque ad Sabbatum ante Dominicam I Adventus
pariter inclusive;
in Adventu autem et a Septuagesima usque ad Pascha nonnisi in Festis B.~Mariæ
Virg. (S.~R.~C. n.~3572 ad~1 et 3659), quæ in Ecclesia universali
celebrantur vel in Kalendario approbato respectivæ Dioecesis vel Instituti
religiosi vel Ecclesiæ sive Oratorii assignantur,
atque in Festo sancti Joseph.
Quando dicitur \rubricatum{Te Deum}, in fine secundi Responsorii adjungitur
℣. \rubricatum{Glória Patri, et Fílio, et Spirítui Sancto}, ac repetitur
altera pars Responsorii.

9. Ante Orationem, etiam quando quis solus recitat Officium,
semper dicitur Versus \rubricatum{Dominus vobiscum} et respondetur
\rubricatum{Et cum spiritu tuo}. Qui Versus non dicitur ab eo, qui non est saltem
in ordine Diaconatus. Si quis autem ad Diaconatus ordinem non pervenerit,
ejus loco dicat Versum \rubricatum{Dómine, exáudi oratiónem meam},
et ei respondetur \rubricatum{Et clamor meus ad te véniat}.
Deinde dicitur \rubricatum{Orémus}, postea Oratio.
Et post ultimam Orationem repetitur ℣.~\rubricatum{Dóminus vobíscum}
vel \rubricatum{Dómine exáudi}. (Rubr. gen. tit.~30, n.~3.)

10. Omnes manu extensa se signent signo crucis a fronte ad pectus et a sinistro
humero ad dexterum
ad \rubricatum{Benedíctus} et ad \rubricatum{Magníficat}. Ceterum servetur
consuetudo quoad sugnum crucis communiter faciendum ad alias Officii partes,
scilicet \rubricatum{Dómine, lábia mea; Convérte nos, Deus; Deus, in adjutórium;
Nunc dimíttis;} et ad benedictionem in fine Completorii.

\end{multicols}
