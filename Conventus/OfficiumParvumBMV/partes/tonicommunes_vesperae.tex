\quasiHora{In principio Horarum.}


\label{tc:deusinadiutorium}

\pars{Tonus festivus.}

\includescore{\ccommunesAR/deusinadiutorium-festivus.tex}

\rubrica{Hoc tono utendum est in 
  %Duplicibus, Semiduplicibus 
  Festis
  et Dominicis
  ad Matutinum, Laudes et Vesperas: et ad Tertiam ante Missam Pontificalem.}

\rubrica{A Completorio Sabbati ante Dominicam Septuagesimæ 
  usque ad Nonam Sabbati sancti inclusive dicitur loco Alleluia:}

% really the same as for the ferial tone
\includescore{\ccommunesAR/laustibi-ferialis.tex}



\pars{Tonus ferialis.}

\includescore{\ccommunesAR/deusinadiutorium-ferialis.tex}

\rubrica{A Completorio Sabbati ante Dominicam Septuagesimæ 
  usque ad Nonam Sabbati sancti inclusive dicitur loco Alleluia:}

\includescore{\ccommunesAR/laustibi-ferialis.tex}

\rubrica{Hoc tono utendum est supradictis diebus 
  ad Primam, Tertiam, Sextam, Nonam, et Completorium;
  et in 
  %Festis Simplicibus 
  Commemorationibus
  et Feriis ad omnes Horas.}


\pagebreak

\pars{Tonus solemnis.}

\includescore{\ccommunesAR/deusinadiutorium-solemnis.tex}

\rubrica{A Completorio Sabbati ante Dominicam Septuagesimæ 
  usque ad Nonam Sabbati sancti inclusive dicitur loco Alleluia:}

\includescore{\ccommunesAR/laustibi-solemnis.tex}

\rubrica{Adhiberi potest ad libitum in Vesperis Festorum quæ cum majori
  solemnitate celebrantur.}


\quasiHora{Toni Orationum.}

\pars{Tonus festivus.}

\rubrica{Hic tonus servatur quando Officium est 
  %Duplex, vel Semiduplex,
  de Festo
vel de Dominica, ad Orationes Missæ, in Matutinis, Laudibus, Vesperis,
et ad Tertiam ante Missam pontificalem.}

\includescore{cantus/arom12/oratio-exemplar.tex}

\rubrica{In ipsa Oratione fit primo metrum, deinde flexa.
  In conclusione vero prius flexa, deinde metrum.
  Metrum in Oratione fit plerumque ubi in textu habetur duplex punctum;
  flexa, ubi habetur punctum cum virgula, vel si non adsit,
  ad primam virgulam post metrum ubi permittit sensus; secus, omittitur.}

\rubrica{In conclusione \textnormal{Qui vivis} vel \textnormal{Qui tecum vivit,}
  fit solummodo metrum.}

\pars{Tonus ferialis.}

\rubrica{Diebus supra memoratis ad Horas minores,
  in 
  %Festis Simplicibus 
  Commemorationibus
  et Feriis ad omnes Horas et ad Missam,
  Orationes cantantur in tono, ut aiunt, Feriali, hoc est:
  recta voce a principio ad finem, solummodo sustentando tenorem
  ubi alias fieret metrum et flexa, et in fine.}




\quasiHora{In fine Horarum.}

\rubrica{[Notitia editoris: Rubricæ Antiphonalis atque Breviarii Romani
    tacent circa tonos \textnormal{Benedicámus Dómino,} quorum utendum est 
    in Officio parvo. Videtur autem opportunum in festis majoribus
    Beatæ Virginis toni solemnis uti, in diebus vero aliis ad libitum
    toni pro festis Beatæ Mariæ Virginis aut toni ferialis.]}

\quasiHora{Toni \textnormal{℣.} Benedicamus Domino.}

\label{tc:benedicamus:vesperae}

\rubrica{In fine Laudum (Matutini si separetur a Laudibus), et Vesperarum,
  \textnormal{℣. Benedicámus Dómino} cantatur a Cantore vel Cantoribus
  in uno ex tonis infra positis, prout qualitas diei vel Festi requirit.
  Eodem modo respondetur a Choro \textnormal{Deo grátias.}
}


\pars{In Festis Solemnibus.}

\rubrica{In Primis Vesperis.}

\superInitialam{II}
\includescore{\ccommunesAR/benedicamus-solemnis-1vesp.tex}

\rubrica{In Secundis Vesperis.}

\superInitialam{VI}
\includescore{\ccommunesAR/benedicamus-solemnis-2vesp.tex}

\pagebreak

\pars{In Festis B. Mariæ Virginis.}

\rubrica{(In festis B. Mariæ Virginis majoribus cantatur \textnormal{Benedicámus
    Dómino} ut in aliis solemnibus Festis.)}

\superInitialam{I}
\includescore{\ccommunesAR/benedicamus-maria.tex}

\vspace{1cm}

\pars{In Feriis.}

\superInitialam{IV}
\includescore{\ccommunesAR/benedicamus-feria.tex}
