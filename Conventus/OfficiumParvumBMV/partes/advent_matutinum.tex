\hora{Ad Matutinum.}

\rubrica{Matutinum fit ut Extra Adventum, pg. \pageref{pars:matutinum}.
Feria IV et Sabbato dicitur tertius psalmus cum antiphona speciali, 
\textnormal{Angelus Dómini}. Lectiones et Responsoria sunt propria,
Absolutiones atque Benedictiones ut Extra Adventum.}

\vfill

\pars{Lectio i}
\scriptura{Luc. 1, 26-28}

Missus est Angelus Gábriel a Deo in civitátem Galilǽæ,
cui nomen Názareth, ad Vírginem desponsátam viro, cui nomen erat Joseph,
de domo David, et nomen Vírginis María.
Et, ingréssus Angelus ad eam, dixit: Ave, grátia plena;
Dóminus tecum: benedícta tu in muliéribus.
\tuAutem

\vfill

\superInitialam{VII}
\includescore{cantus/nrom02/adv_matresp1}

\vfill

\pars{Lectio ii}
\scriptura{Luc 1, 29-33}

Quæ cum audísset, turbáta est in sermóne ejus, et cogitábat
quális esset ista salutátio. Et ait Angelus ei:
Ne tímeas, María; invenísti grátiam apud Deum: ecce, concípies
in útero et páries fílium, et vocábis nomen ejus Jesum.
Hic erit magnus, et Fílius Altíssimi vocábitur;
et dabit illi Dóminus Deus sedem David, patris ejus;
et regnábit in domo Jacob in ætérnum, et regni ejus non erit finis.
\tuAutem

\pagebreak

\rubrica{Quando dicitur \textnormal{Te Deum,} id est in Festis B. Mariæ
Virg., sequens responsorium dicitur cum \textnormal{Glória Patri.}
Cum autem \textnormal{Te Deum} non dicitur, \textnormal{Glória Patri}
et ultima repetitio \textnormal{Ave, María} omittitur.}

\superInitialam{VII}
\includescore{cantus/nrom02/adv_matresp2}

\vfill

\pars{Lectio iii}
\scriptura{Luc 1, 34-38}

Dixit autem María ad Angelum: Quómodo fiet istud, quóniam virum non
cognósco? Et respóndens Angelus dixit ei: Spíritus Sanctus supervéniet
in te, et virtus Altíssimi obumbrábit tibi. Ideóque et quod nascétur
ex te Sanctum, vocábitur Fílius Dei. Et ecce, Elísabeth, cognáta tua,
et ipsa concépit fílium en senectúte sua, et hic mensis sextus est illi,
quæ vocátur stérilis; quia non erit impossíbile apud Deum omne verbum.
Dixit autem María: Ecce ancílla Dómini: fiat mihi secúndum verbum tuum.
\tuAutem

\vfill

\rubrica{Extra Festa B. Mariæ Virg.:}

\superInitialam{IV}
\includescore{cantus/nrom02/adv_matresp3}

\vspace{1cm}

\rubrica{Dicto \textnormal{Te Deum} aut tertio Responsorio, 
statim incipiuntur Laudes a Versu \textnormal{Deus in adjutórium}.
Si autem Laudes a Matutino separantur, dicitur hic Oratio et finis
horæ ut in Vesperis, pg. \pageref{vesperaefinisadvent}.
(\textnormal{Kyrie eléison} non dicitur.)}
