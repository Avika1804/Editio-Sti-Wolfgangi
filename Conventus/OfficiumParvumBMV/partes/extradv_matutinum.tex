% Extra Adventum.

\hora{Ad Matutinum.}

\domineLabiaMea

\deusInAdiutorium

\pars{Invitatorium.}

\superInitialam{VII}
\includescore{cantus/manuscripti/matinvit.tex}

\scriptura{Psalmus 95. (Textus antiquus latinus.)}

\superInitialam{VII}
\includescore{cantus/manuscripti/venite7/venite7a.tex}

\invitatoriumIntegrum

\includescore{cantus/manuscripti/venite7/venite7b.tex}

\invitatoriumAltera

\rubricatum{In sequenti Psalmi versu, ad verba 
  \textnormal{veníte, adorémus et procidámus ante Deum,} 
  genuflectitur.}

\includescore{cantus/manuscripti/venite7/venite7c.tex}

\invitatoriumIntegrum

\includescore{cantus/manuscripti/venite7/venite7d.tex}

\invitatoriumAltera

\includescore{cantus/manuscripti/venite7/venite7e.tex}

\invitatoriumIntegrum

\includescore{cantus/manuscripti/venite7/venite7f.tex}

\rubrica{Repetitur altera pars Invitatorii. 
Denique repetitur integrum Invitatorium.}

\pars{Hymnus.}

\superInitialam{II}
\includescore{temporalia/hymnus-QuemTerra.tex}

\vfill \pagebreak



\subhora{Dominica, Feria II et V.}

\pars{psalmus 1.}

\antiphona{IV A*}{cantus/arom12/matant1.tex}{}

\scriptura{Psalmus 8.}

\includescore{temporalia/ps8-initium-iv-A-auto.tex}

\psalmusEtTranslatio{temporalia/ps8-iv-a.tex}{temporalia/ps8-boh.tex}

\vfill \pagebreak

\pars{psalmus 2.}

\antiphona{IV A*}{cantus/manuscripti/matant2b.tex}{}

\scriptura{Psalmus 18.}

\includescore{temporalia/ps18-initium-iv-A-auto.tex}

\psalmusEtTranslatio{temporalia/ps18-iv-a.tex}{empty.tex}

\vfill \pagebreak

\pars{psalmus 3.}

\antiphona{IV A*}{cantus/manuscripti/matant3.tex}{}

\scriptura{Psalmus 23.}

\includescore{temporalia/ps23-initium-iv-A-auto.tex}

\psalmusEtTranslatio{temporalia/ps23-iv-a.tex}{empty.tex}

\vfill \pagebreak



\subhora{Feria III et VI.}

\pars{psalmus 1.}

\antiphona{VII c}{cantus/manuscripti/matant4.tex}{}

\scriptura{Psalmus 44.}

\includescore{temporalia/ps44-initium-vii-c-auto.tex}

\psalmusEtTranslatio{temporalia/ps44-vii-c.tex}{empty.tex}

\vfill \pagebreak

\pars{psalmus 2.}

\antiphona{VII c}{cantus/manuscripti/matant5.tex}{}

\scriptura{Psalmus 45.}

\includescore{temporalia/ps45-initium-vii-c-auto.tex}

\psalmusEtTranslatio{temporalia/ps45-vii-c.tex}{empty.tex}

\vfill \pagebreak

\pars{psalmus 3.}

\antiphona{VII c}{cantus/manuscripti/matant6.tex}{}

\scriptura{Psalmus 86.}

\includescore{temporalia/ps86-initium-vii-c-auto.tex}

\psalmusEtTranslatio{temporalia/ps86-vii-c.tex}{empty.tex}

\vfill \pagebreak



\subhora{Feria IV et Sabbato.}

\pars{psalmus 1.}

\antiphona{IV A*}{cantus/manuscripti/matant7.tex}{}

\scriptura{Psalmus 95.}

\includescore{temporalia/ps95-initium-iv-A-auto.tex}

\psalmusEtTranslatio{temporalia/ps95-iv-a.tex}{empty.tex}

\vfill \pagebreak

\pars{psalmus 2.}

\antiphona{IV A*}{cantus/manuscripti/matant8.tex}{}

\scriptura{Psalmus 96.}

\includescore{temporalia/ps96-initium-iv-A-auto.tex}

\psalmusEtTranslatio{temporalia/ps96-iv-a.tex}{empty.tex}

\vfill \pagebreak

\pars{psalmus 3.}

\rubrica{Extra Adventum:}

\antiphona{IV A*}{cantus/manuscripti/matant9.tex}{}

\scriptura{Psalmus 97.}

\includescore{temporalia/ps97-initium-iv-A-auto.tex}

\psalmusEtTranslatio{temporalia/ps97-iv-a.tex}{empty.tex}

\vfill \pagebreak

\rubrica{Tempore Adventus:}

\antiphona{I f}{cantus/nrom02/adv_matant9.tex}{}

\scriptura{Psalmus 97.}

\includescore{temporalia/ps97-initium-i-f-auto.tex}

\psalmusEtTranslatio{temporalia/ps97-i-f.tex}{empty.tex}

\vfill \pagebreak


\versiculusDiffusaEst


\rubrica{\textnormal{Pater noster} secreto usque ad}
℣. Et ne nos indúcas in tentatiónem. ℟. Sed líbera nos a malo.

\pars{Absolutio}

\includescore{cantus/arom12/absolutio-PrecibusEtMeritis.tex}

\pars{Benedictio}

\rubrica{Lector petit benedictionem. 
(Hoc modo dicitur \textnormal{Jube Domne} etiam pro lectione 2. et 3.)
Hebdomadarius benedicit.
Chorus respondet \textnormal{Amen}.}

\includescore{cantus/arom12/benedictio-NosCumProle.tex}

\pars{Lectio i}

\scriptura{Sir 24,11-13}

In ómnibus réquiem quæsívi, et in hereditáte Dómini morábor.
Tunc præcépit et dixit mihi Creátor ómnium, 
et qui creávit me, requiévit in tabernáculo meo,
et dixit mihi: 
In Jacob inhábita, et in Israel hereditáre,
et in eléctis meis mitte radíces.

\tuAutem

\superInitialam{II}
\includescore{cantus/nrom02/matresp1.tex}

\vfill \pagebreak

\pars{Benedictio}

\rubrica{Lector dicit \textnormal{Jube Domne benedicere} ut supra.}

\includescore{cantus/arom12/benedictio-IpsaVirgo.tex}

\pars{Lectio ii}

\scriptura{Sir 24,15-16}

Et sic in Sion firmáta sum, et in civitáte sanctificáta simíliter requiévi,
et in Jerúsalem potéstas mea.
Et radicávi in pópulo honorificáto, et in parte Dei mei heréditas illíus,
et in plenitúdine sanctórum deténtio mea.

\tuAutem

\rubrica{Quando dicitur \textnormal{Te Deum,} id est in Festis B. Mariæ
Virg. et sancti Joseph et toto tempore ante Septuagesimam ac post
Sabbatum sanctum, sequens responsorium dicitur cum \textnormal{Glória Patri.}
Cum autem \textnormal{Te Deum} non dicitur, \textnormal{Glória Patri}
et ultima repetitio \textnormal{Genuísti} omittitur.}

\superInitialam{I}
\includescore{cantus/nrom02/matresp2.tex}

\vfill \pagebreak

\pars{Benedictio}

\rubrica{Lector dicit \textnormal{Jube Domne benedicere} ut supra.}

\includescore{cantus/arom12/benedictio-PerVirginem.tex}

\pars{Lectio iii}

\scriptura{Sir 24,17-20}

Quasi cedrus exaltáta sum in Líbano, et quasi cypréssus in monte Sion:
quasi palma exaltáta sum in Cades, et quasi plantátio rosæ in Jéricho:
quasi olíva speciósa in campis, et quasi plátanus exaltáta sum juxta aquam
in platéis.
Sicut cinnamónum et bálsamum aromatízans odórem dedi; 
quasi myrrha elécta dedi suavitátem odóris.

\tuAutem

\rubrica{A Dominica Septuagesimæ usque ad Sabbatum sanctum inclusive,
extra Festa B. Mariæ Virg. et sancti Joseph, Hymnus Ambrosianus
omittitur ejusque loco dicitur responsorium:}

\superInitialam{I}
\includescore{cantus/nrom02/matresp3.tex}

\rubrica{In Festis B. Mariæ Virg. et sancti Joseph et toto tempore
ante Septuagesimam ac post Sabbatum sanctum:}

\pars{Hymnus Ambrosianus}

\superInitialam{III}
\includescore{../../cantuscommunes/arom12/tedeum-simplex.tex}

\rubrica{Et incipiunt Laudes, dicto \textnormal{Deus in adjutórium}.
Si autem Laudes a Matutino separantur, dicitur hic Oratio et finis
horæ ut in Vesperis, pg. \pageref{vesperaefinis}.
(\textnormal{Kyrie eléison} non dicitur.)}
