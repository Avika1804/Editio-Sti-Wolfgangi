\hora{Mutationes}

Hic liber secundum editiones divini officii Romani 
a Pio PP X. reformati paratus est.
Sed in textus necnon rubricas mutationes inductæ sunt ad mentem
reformarum
Pii PP XII.\footnote{Sacra Congregatio Rituum:
  Decretum generale
  De rubricis ad simpliciorem formam redigendis.
  AAS 47 [1955], 218 et seq.
  Infra \textit{Rubricae 1955.}} 
et Joannis PP XXIII.\footnote{Sacra Congregatio Rituum:
  Rubricae Breviarii et Missalis Romani.
  AAS 52 [1960], 597 et seq.
  Infra \textit{Codex rubricarum 1960.}\\
  Variationes in Breviario et Missali Romano ad normam novi Codicis rubricarum.
  AAS 52 [1960], 706 et seq. 
  Infra \textit{Variationes 1960.}}
%
Oratio \textit{Ave Maria} ante singulis horis omissa est.\footnote{Rubricae 1955, tit.~IV. §~1.}
%
Similimodo omissus est Versiculus \textit{Fidelium animæ} in fine horarum;
benedictio \textit{Benedicat et custodiat} in fine completorii.\footnote{ibidem, §~3.}
%
Antiphonae dicuntur ante et post psalmum integræ.\footnote{Codex rubricarum 1960, cap. V. E) §~191.}
%
In rubricis circa tonos communes mutatæ sunt classes festorum.\footnote{Variationes 1960, cap. I.}

% Zohlednena nebyla norma Codexu 1960 cap. II §~2:
% Celebratio diei liturgici decurrit per se a Matutino ad Completorium. 
% Sunt tamen dies solemniores, quorum Officium inchoatur a I Vesperis, 
% die praecedenti.


