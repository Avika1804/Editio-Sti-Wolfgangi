% Parts used several times in the booklet

%%%% COMMON

\newcommand{\scripturaMagnificat}{\scriptura{Lucæ 1, 46-55}}

\newcommand{\kyrieEleison}{\includescore{\ccommunesAR/kyrieeleison.tex}}

\newcommand{\dominusVobiscum}{
  
  \rubrica{Recto tono:}

  ℣. Dóminus vobíscum.
  ℟. Et cum spíritu tuo.

  \rubrica{In choro monialium loco \textnormal{Dominus vobiscum} dicitur:}

  ℣. Dómine exáudi oratiónem meam.
  ℟. Et clamor meus ad te véniat.
  
}

% when the Dominus vobiscum is repeated on the same page.
\newcommand{\dominusVobiscumRep}{
  \rubrica{Iterum dicitur \textnormal{Dóminus vobíscum} aut 
    \textnormal{Dómine exáudi}.}
}

\newcommand{\benedicamusDomino}{
  \includescore{\ccommunesAR/benedicamus-feria.tex}
}

\newcommand{\fideliumAnimae}{
  \rubrica{Voce recta et paululum depressa:}

  ℣. Fidélium ánimæ per misericórdiam Dei requiéscant in pace.
  ℟. Amen.
}

\newcommand{\perDominum}{
  Per Dóminum nostrum Jesum Christum Fílium tuum,
  qui tecum vivit et regnat in unitáte Spíritus Sancti Deus,
  per ómnia sǽcula sæculórum. ℟. Amen.
}

\newcommand{\invitatoriumIntegrum}{
  \rubrica{Repetitur integrum Invitatorium.}
}
\newcommand{\invitatoriumAltera}{
  \rubrica{Repetitur altera pars Invitatorii.}
}

\newcommand{\tuAutem}{
  Tu autem, Dómine, miserére nobis.
  ℟. Deo grátias. 
}

\newcommand{\initiumHoraeMinores}{
  \rubrica{Versus \textnormal{Deus, in adjutórium} et 
    Hymnus \textnormal{Meménto, rerum Cónditor} 
    ut ad Primam, pg. \pageref{prima}.}
}

\newcommand{\finisHoraeMinores}{
  \rubrica{Sequitur \textnormal{Kýrie eléison} et omnia usque ad finem horæ
    ut in Vesperis, p. \pageref{vesperaefinis}. Sed Oratio fit ut infra.}
}

%%%% EXTRA ADVENTUM

\newcommand{\domineLabiaMea}{
  \includescore{../../cantuscommunes/arom12/dominelabiamea.tex}
}

\newcommand{\deusInAdiutorium}{
  \includescore{\ccommunesAR/deusinadiutorium-ferialis.tex}

  \rubrica{A Completorio Sabbati ante Septuagesimam dicitur loco Alleluia:}

  \includescore{\ccommunesAR/laustibi-ferialis.tex}
}

% antiphonae ad laudes

\newcommand{\antiphonaLaudI}{
  \antiphona{VII a}{cantus/arom12/antlaud1.tex}{\translatioLaudAntI}}
\newcommand{\antiphonaLaudII}{
  \antiphona{VIII G}{cantus/arom12/antlaud2.tex}{\translatioLaudAntII}}
\newcommand{\antiphonaLaudIII}{
  \antiphona{IV A*}{cantus/arom12/antlaud3.tex}{\translatioLaudAntIII}}
\newcommand{\antiphonaLaudIV}{
  \antiphona{VII c2}{cantus/arom12/antlaud4.tex}{\translatioLaudAntIV}}
\newcommand{\antiphonaLaudV}{
  \antiphona{I g2}{cantus/arom12/antlaud5.tex}{\translatioLaudAntV}}

% antiphonae ad vesperas

\newcommand{\antiphonaI}{
  \antiphona{III a}{cantus/arom12/ant1.tex}{\translatioAntI}}
\newcommand{\antiphonaII}{
  \antiphona{IV A*}{cantus/arom12/ant2.tex}{\translatioAntII}}
\newcommand{\antiphonaIII}{
  \antiphona{III b}{cantus/arom12/ant3.tex}{\translatioAntIII}}
\newcommand{\antiphonaIV}{
  \antiphona{VIII G}{cantus/arom12/ant4.tex}{\translatioAntIV}}
\newcommand{\antiphonaV}{
  \antiphona{IV A*}{cantus/arom12/ant5.tex}{\translatioAntV}}

\newcommand{\capitulumAbInitio}{
  \scriptura{Sir 24,14}

  \includescore{cantus/arom12/capitulum-AbInitio.tex}

  % preklad Jeruz. bible
  \translatioCapituli
}

\newcommand{\versiculusDiffusaEst}{
  \includescore{cantus/arom12/versiculus-DiffusaEst.tex}
}

\newcommand{\inFineHorarumExtraAdventum}{
  \kyrieEleison

  \vspace{7mm}

  \pars{Oratio.}

  \rubrica{Tono feriali, i.e. tono recto:}

  Concéde nos fámulos tuos, quǽsumus, Dómine Deus,
  perpétua mentis et córporis sanitáte gaudére:
  et, gloriósa beátæ Maríæ semper Vírginis intercessióne,
  a præsénti liberáti tristítia et ætérna pérfrui lætítia.

  \perDominum

  %%% finis

  \dominusVobiscum

  \vspace{7mm}

  \includescore{\ccommunesAR/benedicamus-feria.tex}

  \vspace{7mm}

  \fideliumAnimae
}

\newcommand{\rubricaAnnuntiatioQuadragesimaAlleluja}{
  \rubrica{In Festo Annuntiationis, quoties ipsum Tempore Quadragesimæ
    celebretur, \textnormal{Allelúja} omittitur.}
}

%%%% IN ADVENTU

\newcommand{\adventDeusInAdiutorium}{
  \includescore{\ccommunesAR/deusinadiutorium-ferialis.tex}
}

\newcommand{\adventAntiphonaI}{
  \antiphona{VIII G}{cantus/arom12/adv_ant1.tex}{\translatioAntI}}
\newcommand{\adventAntiphonaII}{
  \antiphona{I g}{cantus/arom12/adv_ant2.tex}{\translatioAntII}}
\newcommand{\adventAntiphonaIII}{
  \antiphona{VIII G}{cantus/arom12/adv_ant3.tex}{\translatioAntIII}}
\newcommand{\adventAntiphonaIV}{
  \antiphona{I f}{cantus/arom12/adv_ant4.tex}{\translatioAntIV}}
\newcommand{\adventAntiphonaV}{
  \antiphona{VIII c}{cantus/arom12/adv_ant5.tex}{\translatioAntV}}

% ad Magnificat, Benedictus, Nunc Dimittis
\newcommand{\adventAntiphonaSpiritusSanctus}{
  \antiphona{VIII G}{cantus/arom12/adv_antmag.tex}{}}

\newcommand{\inFineHorarumInAdventu}{
  \kyrieEleison

  \vspace{7mm}

  \pars{Oratio.}

  Deus, qui de beátæ Maríæ Vírginis útero Verbum tuum,
  Angelo nuntiánte, carnem suscípere voluísti: †
  præsta supplícibus tuis;
  ut qui vere eam Genitrícem Dei crédimus, *
  ejus apud te intercessiónibus adjuvémur.

  Per eúmdem Christum Dóminum nostrum.

  %%% finis

  \dominusVobiscum

  \vspace{7mm}

  \includescore{\ccommunesAR/benedicamus-feria.tex}

  \vspace{7mm}

  \fideliumAnimae
}

\newcommand{\rubricaFinisHoraeAdvent}{
  \rubrica{Sequitur \textnormal{Kýrie eléison} etc. usque ad finem horæ
    ut in Vesperis, p. \pageref{vesperaefinisadvent}.}}

\newcommand{\initiumHoraeMinoresAdvent}{
  \rubrica{Versus \textnormal{Deus, in adjutórium} et 
    Hymnus \textnormal{Meménto, rerum Cónditor} 
    ut ad Primam extraAdventum, pg. \pageref{prima}.}
}

%%%% TEMPORE NATIVITATIS

%%%% TEMPORE PASCHALI

\newcommand{\paschaAntiphona}{
  \antiphona{I D2}{cantus/arom12/pasch_antevang.tex}{}}

