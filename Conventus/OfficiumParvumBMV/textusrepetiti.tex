% Parts used several times in the booklet

%%%% COMMON

\newcommand{\scripturaMagnificat}{\scriptura{Lucæ 1, 46-55}}

\newcommand{\kyrieEleison}{\includescore{\ccommunesAR/kyrieeleison.tex}}

\newcommand{\dominusVobiscum}{
  
  \rubrica{Recto tono:}

  ℣. Dóminus vobíscum.
  ℟. Et cum spíritu tuo.

  \rubrica{In choro monialium loco \textnormal{Dominus vobiscum} dicitur:}

  ℣. Dómine exáudi oratiónem meam.
  ℟. Et clamor meus ad te véniat.
  
}

% when the Dominus vobiscum is repeated on the same page.
\newcommand{\dominusVobiscumRep}{
  \rubrica{Iterum dicitur \textnormal{Dóminus vobíscum} aut 
    \textnormal{Dómine exáudi}.}
}

\newcommand{\benedicamusDomino}{
  \includescore{\ccommunesAR/benedicamus-feria.tex}
}

\newcommand{\fideliumAnimae}{
  \rubrica{Voce recta et paululum depressa:}

  ℣. Fidélium ánimæ per misericórdiam Dei requiéscant in pace.
  ℟. Amen.
}

\newcommand{\perDominum}{
  Per Dóminum nostrum Jesum Christum Fílium tuum,
  qui tecum vivit et regnat in unitáte Spíritus Sancti Deus,
  per ómnia sǽcula sæculórum. ℟. Amen.
}

%%%% EXTRA ADVENTUM

\newcommand{\deusInAdiutorium}{
  \includescore{\ccommunesAR/deusinadiutorium-ferialis.tex}

  \rubrica{A Completorio Sabbati ante Septuagesimam dicitur loco Alleluia:}

  \includescore{\ccommunesAR/laustibi-ferialis.tex}
}

\newcommand{\antiphonaI}{
  \antiphona{III a}{cantus/arom12/ant1.tex}{\translatioAntI}}
\newcommand{\antiphonaII}{
  \antiphona{IV A*}{cantus/arom12/ant2.tex}{\translatioAntII}}
\newcommand{\antiphonaIII}{
  \antiphona{III b}{cantus/arom12/ant3.tex}{\translatioAntIII}}
\newcommand{\antiphonaIV}{
  \antiphona{VIII G}{cantus/arom12/ant4.tex}{\translatioAntIV}}
\newcommand{\antiphonaV}{
  \antiphona{IV A*}{cantus/arom12/ant5.tex}{\translatioAntV}}

\newcommand{\capitulumAbInitio}{
  \scriptura{Sir 24,14}

  \includescore{cantus/arom12/capitulum-AbInitio.tex}

  % preklad Jeruz. bible
  \translatioCapituli
}

\newcommand{\versiculusDiffusaEst}{
  \includescore{cantus/arom12/versiculus-DiffusaEst.tex}
}

\newcommand{\oratio}{
  \rubrica{Tono feriali, i.e. tono recto:}

  Concéde nos fámulos tuos, quǽsumus, Dómine Deus,
  perpétua mentis et córporis sanitáte gaudére:
  et, gloriósa beátæ Maríæ semper Vírginis intercessióne,
  a præsénti liberáti tristítia et ætérna pérfrui lætítia.

  \perDominum
}

%%%% IN ADVENTU

\newcommand{\adventDeusInAdiutorium}{
  \includescore{\ccommunesAR/deusinadiutorium-ferialis.tex}
}

\newcommand{\adventAntiphonaI}{
  \antiphona{VIII G}{cantus/arom12/adv_ant1.tex}{\translatioAntI}}
\newcommand{\adventAntiphonaII}{
  \antiphona{I g}{cantus/arom12/adv_ant2.tex}{\translatioAntII}}
\newcommand{\adventAntiphonaIII}{
  \antiphona{VIII G}{cantus/arom12/adv_ant3.tex}{\translatioAntIII}}
\newcommand{\adventAntiphonaIV}{
  \antiphona{I f}{cantus/arom12/adv_ant4.tex}{\translatioAntIV}}
\newcommand{\adventAntiphonaV}{
  \antiphona{VIII c}{cantus/arom12/adv_ant5.tex}{\translatioAntV}}

%%%% TEMPORE NATIVITATIS



%%%%%%%%%%%%%%%%% OBSOLETE

\newcommand{\responsoriumAdVesperas}{
  \gresetfirstlineaboveinitial{\small \textsc{\textbf{VI.}}}{\small \textsc{\textbf{VI.}}}
  \includescore{cantus/amon33/resp-vesp.tex}

  \translatioRespVesp
}

\newcommand{\responsoriumAdLaudes}{
  \gresetfirstlineaboveinitial{\small \textsc{\textbf{VI.}}}{\small \textsc{\textbf{VI.}}}
  \includescore{cantus/amon33/resp-laud.tex}

  \translatioRespVesp
}

\newcommand{\hymnusAdVesperas}{
  \pars{Hymnus.}

  \superInitialam{IV}
  \includescore{temporalia/hymnus-ConditorAlme.tex}
}

\newcommand{\hymnusAdLaudes}{
  \pars{Hymnus.}

  % no mode info in AM1933
  \includescore{temporalia/hymnus-VoxClara.tex}
}

\newcommand{\versiculusDiei}{
  \pars{Versiculus.}

  % Versus. %%%
    \includescore{cantus/amon33/versus-voxclamantis.tex}
    % \vspace{-5mm}
    \noindent ℟.\hspace{0.5mm}Rectas fácite sémitas e\textit{jus.}
    
    \noindent \translatioVersus
}

\newcommand{\inFineHorarum}{
  \pars{In fine horarum.}

  \rubrica{Post \textnormal{Benedicamus Domino} dicitur voce recta
  et paululum depressa:}

  \includescore{\cantusCommunesAM/fideliumanimae.tex}

  Pater noster \rubrica{totum secreto. Deinde, si discedendum est a Choro:}

  \includescore{\cantusCommunesAM/dominusdet.tex}

  \pagebreak

  \includescore{\cantusCommunesAM/an_alma_redemptoris_mater.tex}

  \vspace{5mm}

  \rubrica{Versiculus tono simplici:}

  \noindent ℣. Angelus Dómini nuntiávit Ma\textbf{rí}æ.\\
  ℟. Et concépit de Spíritu \textbf{Sanc}to.

  \vspace{5mm}

  \includescore{\cantusCommunesAM/oratio-AlmaRedemptorisAdvent.tex}

  \includescore{\cantusCommunesAM/divinumauxilium.tex}
}

\newcommand{\rubricaOratio}{
  \rubrica{Sequitur Supplicatio Litani"ae et cetera usque ad finem hor"ae,
  pg. \pageref{oratioetc}.}
}
