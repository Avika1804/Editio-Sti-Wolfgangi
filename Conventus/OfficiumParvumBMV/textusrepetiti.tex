% Parts used several times in the booklet

%%%% COMMON

\newcommand{\scripturaMagnificat}{\scriptura{Lucæ 1, 46-55}}

\newcommand{\kyrieEleison}{\includescore{\ccommunesAR/kyrieeleison.tex}}

\newcommand{\dominusVobiscum}{
  
  ℣. Dóminus vobíscum.
  ℟. Et cum spíritu tuo.

  %\rubrica{In choro monialium loco \textnormal{Dóminus vobíscum} dicitur:}
  \rubrica{Absente sacerdote vel diacono, loco \textnormal{Dóminus vobíscum}
  dicitur:}

  ℣. Dómine exáudi oratiónem meam.
  ℟. Et clamor meus ad te véniat.
  
}

% when the Dominus vobiscum is repeated on the same page.
\newcommand{\dominusVobiscumRep}{
  \rubrica{Iterum dicitur \textnormal{Dóminus vobíscum} aut 
    \textnormal{Dómine exáudi}.}
}

\newcommand{\benedicamusDomino}{%
  Benedicámus Dómino. \rubricatum{Pg. \pageref{tc:benedicamus:minor}.}}

\newcommand{\perDominum}{%
  Per Dóminum nostrum Jesum Christum Fílium tuum,
  qui tecum vivit et regnat in unitáte Spíritus Sancti Deus:
  per ómnia sǽcula sæculórum. ℟. Amen.
}
\newcommand{\perChristum}{%
  Per Christum Dóminum nostrum. ℟. Amen.
}

\newcommand{\invitatoriumIntegrum}{
  \hfill \rubricatum{Repetitur integrum Invitatorium.}
}
\newcommand{\invitatoriumAltera}{
  \hfill \rubricatum{Repetitur altera pars Invitatorii.}
}

\newcommand{\tuAutem}{
  \hspace{0.5cm}
  Tu autem, Dómine, miserére nobis.
  ℟. Deo grátias. 
}

\newcommand{\initiumHoraeMinores}{
  \vspace{2mm}
  \deusInAdiutorium

  \rubrica{Hymnus \textnormal{Meménto, rerum Cónditor} 
    ut ad Primam, pg. \pageref{prima}.}
  \vspace{5mm}
}

\newcommand{\finisHoraeMinores}{
  \vspace{7mm}
  \rubrica{Sequitur \textnormal{Kýrie eléison} et omnia usque ad finem horæ
    ut in Vesperis, p. \pageref{vesperaefinis}. Sed Oratio fit ut infra.}
  \vspace{2mm}
}

\newcommand{\rubricaLaudesPrincipium}{
  \rubrica{Si Laudes separentur a Matutino:}

  \deusInAdiutorium
  \vspace{5mm}
}

%%%% EXTRA ADVENTUM

\newcommand{\domineLabiaMea}{%
  Dómine lábia mea apéries.
  \rubricatum{Pg. \pageref{tc:dominelabia}.}
}

\newcommand{\deusInAdiutorium}{%
  Deus in adjutórium meum inténde.
  \rubricatum{Pg. \pageref{tc:deusinadiutorium}.}}

% Deus in adiutorium with spaces around for places where it stands
% alone before the hour title and the beginning of it's proper content.
\newcommand{\deusInAdiutoriumS}{%
  \vspace{2mm}
  \deusInAdiutorium
  \vspace{5mm}}

% antiphonae ad laudes

\newcommand{\antiphonaLaudI}{
  \antiphona{VII a}{cantus/arom12/antlaud1.tex}{\translatioLaudAntI}}
\newcommand{\antiphonaLaudII}{
  \antiphona{VIII G}{cantus/arom12/antlaud2.tex}{\translatioLaudAntII}}
\newcommand{\antiphonaLaudIII}{
  \antiphona{IV A*}{cantus/arom12/antlaud3.tex}{\translatioLaudAntIII}}
\newcommand{\antiphonaLaudIV}{
  \antiphona{VII c2}{cantus/arom12/antlaud4.tex}{\translatioLaudAntIV}}
\newcommand{\antiphonaLaudV}{
  \antiphona{I g2}{cantus/arom12/antlaud5.tex}{\translatioLaudAntV}}

% antiphonae ad vesperas

\newcommand{\antiphonaI}{
  \antiphona{III a}{cantus/arom12/ant1.tex}{\translatioAntI}}
\newcommand{\antiphonaII}{
  \antiphona{IV A*}{cantus/arom12/ant2.tex}{\translatioAntII}}
\newcommand{\antiphonaIII}{
  \antiphona{III b}{cantus/arom12/ant3.tex}{\translatioAntIII}}
\newcommand{\antiphonaIV}{
  \setspaceafterinitial{4mm plus 0em minus 0em}
  \setspacebeforeinitial{4mm plus 0em minus 0em}
  \antiphona{VIII G}{cantus/arom12/ant4.tex}{\translatioAntIV}
  \spacearoundinitialNormal
}
\newcommand{\antiphonaV}{
  \antiphona{IV A*}{cantus/arom12/ant5.tex}{\translatioAntV}}

\newcommand{\capitulumAbInitio}{
  \scriptura{Sir 24,14}

  \includescore{cantus/arom12/capitulum-AbInitio.tex}

  % preklad Jeruz. bible
  \translatioCapituli
}

\newcommand{\versiculusDiffusaEst}{
  \includescore{cantus/arom12/versiculus-DiffusaEst.tex}
}


\newcommand{\finisAnteOrationem}{
  \kyrieEleison

  \vspace{2mm}

  \dominusVobiscum

  \vspace{3mm}

  \pars{Oratio.}
}
\newcommand{\finisPostOrationem}{
  \dominusVobiscumRep

  \vspace{2mm}
}
\newcommand{\finisPostBenedicamus}{
}

% end of the hours - extra Adventum ut in Vesperis
\newcommand{\inFineHorarumExtraAdventum}{
  \finisAnteOrationem

  Concéde nos fámulos tuos, quǽsumus Dómine Deus,
  perpétua mentis et córporis sanitáte gaudére:
  et gloriósa beátæ Maríæ semper Vírginis intercessióne,
  a præsénti liberári tristítia, et ætérna pérfrui lætítia.\hspace{0.5cm}
  \perChristum

  \finisPostOrationem

  \benedicamusDomino

  \finisPostBenedicamus
}

% end of the hours - extra Adventum ut in Laudibus
\newcommand{\inFineHorarumExtraAdventumLaudes}{
  \finisAnteOrationem

  Deus, qui de beátæ Maríæ Vírginis útero Verbum tuum,
  Angelo nuntiánte, carnem suscípere voluísti: 
  præsta supplícibus tuis; ut qui vere eam Genitrícem Dei crédimus, 
  ejus apud te intercessiónibus adjuvémur.
  Per eúmdem Christum Dóminum nostrum. 
  ℟. Amen.

  \finisPostOrationem

  \benedicamusDomino

  \finisPostBenedicamus
}


\newcommand{\rubricaAnnuntiatioQuadragesimaAlleluja}{
  \rubrica{In Festo Annuntiationis, quoties ipsum Tempore Quadragesimæ
    celebretur, \textnormal{Allelúja} omittitur.}}

\newcommand{\rubricaLectiones}{
  \rubrica{Sequuntur lectiones, pg. \pageref{lectiones}.}
}

%%%% IN ADVENTU

\newcommand{\adventDeusInAdiutorium}{
  \includescore{\ccommunesAR/deusinadiutorium-ferialis.tex}
}

\newcommand{\adventAntiphonaI}{
  \antiphona{VIII G}{cantus/arom12/adv_ant1.tex}{\translatioAntI}}
\newcommand{\adventAntiphonaII}{
  \rubricaAnnuntiatioQuadragesimaAlleluja

  \antiphona{I g}{cantus/arom12/adv_ant2.tex}{\translatioAntII}}
\newcommand{\adventAntiphonaIII}{
  \rubricaAnnuntiatioQuadragesimaAlleluja

  \antiphona{VIII G}{cantus/arom12/adv_ant3.tex}{\translatioAntIII}}
\newcommand{\adventAntiphonaIV}{
  \antiphona{I f}{cantus/arom12/adv_ant4.tex}{\translatioAntIV}}
\newcommand{\adventAntiphonaV}{
  \antiphona{VIII c}{cantus/arom12/adv_ant5.tex}{\translatioAntV}}

% ad Magnificat, Benedictus, Nunc Dimittis
\newcommand{\adventAntiphonaSpiritusSanctus}{
  \rubricaAnnuntiatioQuadragesimaAlleluja

  \antiphona{VIII G}{cantus/arom12/adv_antmag.tex}{}}

\newcommand{\inFineHorarumInAdventu}{
  \kyrieEleison

  \vspace{7mm}

  \pars{Oratio.}

  Deus, qui de beátæ Maríæ Vírginis útero Verbum tuum,
  Angelo nuntiánte, carnem suscípere voluísti: 
  præsta supplícibus tuis;
  ut qui vere eam Genitrícem Dei crédimus, 
  ejus apud te intercessiónibus adjuvémur.

  Per eúmdem Christum Dóminum nostrum.
  ℟. Amen.

  %%% finis

  \dominusVobiscum

  \vspace{7mm}

  \benedicamusDomino

}

\newcommand{\rubricaFinisHoraeAdvent}{
  \rubrica{Sequitur \textnormal{Kýrie eléison} etc. usque ad finem horæ
    ut in Vesperis, p. \pageref{vesperaefinisadvent}.}}

\newcommand{\initiumHoraeMinoresAdvent}{
  \vspace{2mm}
  \deusInAdiutorium

  \rubrica{Hymnus \textnormal{Meménto, rerum Cónditor} 
    ut ad Primam Extra Adventum, pg. \pageref{prima}.}
  \vspace{5mm}
}

%%%% TEMPORE NATIVITATIS

\newcommand{\nativitasAntiphonaI}{
  \antiphona{VI F}{cantus/arom12/nat_ant1.tex}{}}
\newcommand{\nativitasAntiphonaII}{
  \antiphona{III a2}{cantus/arom12/nat_ant2.tex}{}}
\newcommand{\nativitasAntiphonaIII}{
  \antiphona{IV E}{cantus/arom12/nat_ant3.tex}{}}
\newcommand{\nativitasAntiphonaIV}{
  \antiphona{I f}{cantus/arom12/nat_ant4.tex}{}}
\newcommand{\nativitasAntiphonaV}{
  \antiphona{II D}{cantus/arom12/nat_ant5.tex}{}}

% ad Magnificat, Nunc Dimittis
\newcommand{\nativitasAntiphonaMagnificat}{
  \antiphona{II A}{cantus/arom12/nat_antmag.tex}{}}

\newcommand{\nativitasAntiphonaBenedictus}{
  \antiphona{VIII G}{cantus/arom12/nat_antben.tex}{}}

\newcommand{\inFineHorarumPostNativitatem}{
  \kyrieEleison

  \vspace{7mm}

  \pars{Oratio.}

  Deus, qui salútis ætérnæ, beátæ Maríæ virginitáte fecúnda,
  humáno géneri prǽmia præstitísti:
  tríbue, quǽsumus;
  ut ipsam pro nobis intercédere sentiámus,
  per quam merúimus auctórem vitæ suscípere,
  Dóminum nostrum Jesum Christum, Fílium tuum:

  Qui tecum vivit et regnat in unitáte Spíritus Sancti Deus,
  per ómnia sǽcula sæculórum.
  ℟. Amen.
  %%% finis

  \dominusVobiscum

  \vspace{7mm}

  \benedicamusDomino
}

\newcommand{\rubricaFinisHoraePostNativitatem}{
  \rubrica{Sequitur \textnormal{Kýrie eléison} etc. usque ad finem horæ
    ut in Vesperis, p. \pageref{vesperaefinisnativitas}.}}


%%%% TEMPORE PASCHALI

\newcommand{\paschaAntiphona}{
  \antiphona{I D2}{cantus/arom12/pasch_antevang.tex}{}}

%%%% piae orationes

\newcommand{\anteOfficiumOratio}{
\lettrine{A}{peri,} Dómine, os meum ad benedicéndum nomen sanctum tuum:
munda quoque cor meum ab ómnibus vanis, pervérsis, et aliénis
cogitatiónibus:
intelléctum illúmina, afféctum inflámma,
ut digne, atténte ac devóte hoc Offícium recitáre váleam,
et exaudíri mérear ante conspéctum Divínæ Majestátis tuæ.
Per Christum, Dominum nostrum.
℟. Amen.

Dómine, in unióne illíus divínæ intentiónis,
qua ipse in terris laudes Deo persolvísti,
has tibi Horas \rubricatum{(vel \textnormal{hanc tibi Horam})} persólvo.
}

\newcommand{\postOfficiumOratio}{
\rubrica{
  Orationem sequentem devote post Officium recitantibus
  Leo Papa X. defectus, et culpas in eo persolvendo ex humana
  fragilitate contractas, indulsit, et dicitur flexis genibus.
}

\lettrine{S}{acrosánctæ} et indivíduæ Trinitáti,
crucifíxi Dómini nostri Jesu Christi humanitáti,
beatíssimæ et gloriosíssimæ sempérque Vírginis Maríæ
fecúndæ integritáti, 
et ómnium Sanctórum universitáti
sit sempitérna laus, honor, virtus et glória
ab omni creatúra,
nobísque remíssio ómnium peccatórum,
per infiníta sǽcula sæculórum.
℟. Amen.

\noindent ℣. Beáta víscera Maríæ Virginis, quæ portavérunt
ætérni Patris Fílium.\\
℟. Et beáta úbera, quæ lactavérunt Christum Dominum.

\rubrica{Et dicitur secreto \textnormal{Pater noster.} et \textnormal{Ave María.}}
}
