% Parts used several times in the booklet

\newcommand{\antiphonaI}{
  \antiphona{III a}{cantus/arom12/ant1.tex}{\translatioAntI}}
\newcommand{\antiphonaII}{
  \antiphona{IV A*}{cantus/arom12/ant2.tex}{\translatioAntII}}
\newcommand{\antiphonaIII}{
  \antiphona{III b}{cantus/arom12/ant3.tex}{\translatioAntIII}}
\newcommand{\antiphonaIV}{
  \antiphona{VIII G}{cantus/arom12/ant4.tex}{\translatioAntIV}}
\newcommand{\antiphonaV}{
  \antiphona{IV A*}{cantus/arom12/ant5.tex}{\translatioAntV}}

\newcommand{\adventAntiphonaI}{
  \antiphona{VIII G}{cantus/arom12/adv_ant1.tex}{\translatioAntI}}
\newcommand{\adventAntiphonaII}{
  \antiphona{I g}{cantus/arom12/adv_ant2.tex}{\translatioAntII}}
\newcommand{\adventAntiphonaIII}{
  \antiphona{VIII G}{cantus/arom12/adv_ant3.tex}{\translatioAntIII}}
\newcommand{\adventAntiphonaIV}{
  \antiphona{I f}{cantus/arom12/adv_ant4.tex}{\translatioAntIV}}
\newcommand{\adventAntiphonaV}{
  \antiphona{VIII c}{cantus/arom12/adv_ant5.tex}{\translatioAntV}}

\newcommand{\capitulumDiei}{
  \pars{Capitulum.} \scriptura{}

  \includescore{cantus/amon33/capitulum-FratresQuaecumque.tex}

  % preklad Jeruz. bible
  \translatioCapituli

  \vspace{1cm}
  \pars{Responsorium breve.}
}

\newcommand{\responsoriumAdVesperas}{
  \gresetfirstlineaboveinitial{\small \textsc{\textbf{VI.}}}{\small \textsc{\textbf{VI.}}}
  \includescore{cantus/amon33/resp-vesp.tex}

  \translatioRespVesp
}

\newcommand{\responsoriumAdLaudes}{
  \gresetfirstlineaboveinitial{\small \textsc{\textbf{VI.}}}{\small \textsc{\textbf{VI.}}}
  \includescore{cantus/amon33/resp-laud.tex}

  \translatioRespVesp
}

\newcommand{\hymnusAdVesperas}{
  \pars{Hymnus.}

  \superInitialam{IV}
  \includescore{temporalia/hymnus-ConditorAlme.tex}
}

\newcommand{\hymnusAdLaudes}{
  \pars{Hymnus.}

  % no mode info in AM1933
  \includescore{temporalia/hymnus-VoxClara.tex}
}

\newcommand{\versiculusDiei}{
  \pars{Versiculus.}

  % Versus. %%%
    \includescore{cantus/amon33/versus-voxclamantis.tex}
    % \vspace{-5mm}
    \noindent ℟.\hspace{0.5mm}Rectas fácite sémitas e\textit{jus.}
    
    \noindent \translatioVersus
}

\newcommand{\anteOrationem}{
  \rubrica{Ante Orationem, cantatur a Superiore:}

  \pars{Supplicatio Litaniæ.}

  \includescore{\cantusCommunesAM/supplicatiolitaniae.tex}

  \pars{Oratio Dominica.}

  \includescore{\cantusCommunesAM/oratiodominica.tex}

  \rubrica{Deinde dicitur ab Hebdomadario:}

  \includescore{\cantusCommunesAM/dominusvobiscum-solemnis.tex}

  \rubrica{In choro monialium loco Dominus vobiscum dicitur:}

  \includescore{\cantusCommunesAM/domineexaudi.tex}
}

\newcommand{\oratioDiei}{
  \pars{Oratio.}

  \includescore{cantus/amon33/oratio.tex}
}

\newcommand{\benedicamusDomino}{
  \includescore{\cantusCommunesAM/benedicamus-dominica-advequad.tex}
}

\newcommand{\inFineHorarum}{
  \pars{In fine horarum.}

  \rubrica{Post \textnormal{Benedicamus Domino} dicitur voce recta
  et paululum depressa:}

  \includescore{\cantusCommunesAM/fideliumanimae.tex}

  Pater noster \rubrica{totum secreto. Deinde, si discedendum est a Choro:}

  \includescore{\cantusCommunesAM/dominusdet.tex}

  \pagebreak

  \includescore{\cantusCommunesAM/an_alma_redemptoris_mater.tex}

  \vspace{5mm}

  \rubrica{Versiculus tono simplici:}

  \noindent ℣. Angelus Dómini nuntiávit Ma\textbf{rí}æ.\\
  ℟. Et concépit de Spíritu \textbf{Sanc}to.

  \vspace{5mm}

  \includescore{\cantusCommunesAM/oratio-AlmaRedemptorisAdvent.tex}

  \includescore{\cantusCommunesAM/divinumauxilium.tex}
}

\newcommand{\rubricaOratio}{
  \rubrica{Sequitur Supplicatio Litani"ae et cetera usque ad finem hor"ae,
  pg. \pageref{oratioetc}.}
}
