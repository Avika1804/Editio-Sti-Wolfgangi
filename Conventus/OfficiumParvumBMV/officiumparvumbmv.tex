% LuaLaTeX

\documentclass[a4paper, twoside, 12pt]{article}

\usepackage[latin]{babel} 
\usepackage{ecclesiastic}

\usepackage[landscape, left=3cm, right=1.5cm, top=2cm, bottom=1cm]{geometry}

\usepackage{fontspec}
\setmainfont[Ligatures={Common, TeX, Rare}]{Junicode}

% shortcut for Junicode without ligatures (for the Czech texts)
\newfontfamily\nlfont[Ligatures={Common, TeX}]{Junicode}

\usepackage{multicol}
\usepackage{color}
\usepackage{lettrine}
\usepackage{fancyhdr}

% usual packages loading:
\usepackage{luatextra}
\usepackage{graphicx} % support the \includegraphics command and options
\usepackage{gregoriotex} % for gregorio score inclusion
\usepackage{gregoriosyms}
\usepackage{parcolumns}
\usepackage{multicol}

% Commands used to produce a typical "Conventus" booklet

\newenvironment{titulusOfficii}{\begin{center}}{\end{center}}
\newcommand{\dies}[1]{#1

}
\newcommand{\nomenFesti}[1]{\textbf{\Large #1}

}
\newcommand{\celebratio}[1]{#1

}

\newcommand{\hora}[1]{%
\vspace{0.5cm}{\large \textbf{#1}}

\fancyhead[LE]{\thepage\ / #1}
\fancyhead[RO]{#1 / \thepage}
\addcontentsline{toc}{subsection}{#1}
}

% larger unit than a hora
\newcommand{\divisio}[1]{%
\begin{center}
{\Large \textsc{#1}}
\end{center}
\fancyhead[CO,CE]{#1}
\addcontentsline{toc}{section}{#1}
}

% a part of a hora, larger than pars
\newcommand{\subhora}[1]{
\begin{center}
{\large \textit{#1}}
\end{center}
%\fancyhead[CO,CE]{#1}
\addcontentsline{toc}{subsubsection}{#1}
}

% rubricated inline text
\newcommand{\rubricatum}[1]{\textit{#1}}

% standalone rubric
\newcommand{\rubrica}[1]{\vspace{3mm}\rubricatum{#1}}

\newcommand{\notitia}[1]{\textcolor{red}{#1}}

\newcommand{\scriptura}[1]{\hfill \small\textit{#1}}

\newcommand{\translatioCantus}[1]{\vspace{1mm}%
{\noindent\footnotesize \nlfont{#1}}}

% pruznejsi varianta nasledujiciho - umoznuje nastavit sirku sloupce
% s prekladem
\newcommand{\psalmusEtTranslatioB}[3]{
  \vspace{0.5cm}
  \begin{parcolumns}[colwidths={2=#3}, nofirstindent=true]{2}
    \colchunk{
      \input{#1}
    }

    \colchunk{
      \vspace{-0.5cm}
      {\footnotesize \nlfont
        \input{#2}
      }
    }
  \end{parcolumns}
}

\newcommand{\psalmusEtTranslatio}[2]{
  \psalmusEtTranslatioB{#1}{#2}{8.5cm}
}


\newcommand{\canticumMagnificatEtTranslatio}[1]{
  \psalmusEtTranslatioB{#1}{temporalia/extra-adventum-vespers/magnificat-boh.tex}{12cm}
}
\newcommand{\canticumBenedictusEtTranslatio}[1]{
  \psalmusEtTranslatioB{#1}{temporalia/extra-adventum-laudes/benedictus-boh.tex}{10.5cm}
}

% volne misto nad antifonami, kam si zpevaci dokresli neumy
\newcommand{\hicSuntNeumae}{\vspace{0.5cm}}

% prepinani mista mezi notovymi osnovami: pro neumovane a neneumovane zpevy
\newcommand{\cantusCumNeumis}{
  \setgrefactor{17}
  \global\advance\grespaceabovelines by 5mm%
}
\newcommand{\cantusSineNeumas}{
  \setgrefactor{17}
  \global\advance\grespaceabovelines by -5mm%
}

% znaky k umisteni nad inicialu zpevu
\newcommand{\superInitialam}[1]{\gresetfirstlineaboveinitial{\small {\textbf{#1}}}{\small {\textbf{#1}}}}

% pars officii, i.e. "oratio", ...
\newcommand{\pars}[1]{\textbf{#1}}

\newenvironment{psalmus}{
  \setlength{\parindent}{0pt}
  \setlength{\parskip}{5pt}
}{
  \setlength{\parindent}{10pt}
  \setlength{\parskip}{10pt}
}

%%%% Prejmenovat na latinske:
\newcommand{\nadpisZalmu}[1]{
  \hspace{2cm}\textbf{#1}\vspace{2mm}%
  \nopagebreak%

}

% mode, score, translation
\newcommand{\antiphona}[3]{%
\hicSuntNeumae
\superInitialam{#1}
\includescore{#2}

#3
}

%%%% Preklady jednotlivych zpevu (nektere se opakuji, a je dobre mit je
% vsechny na jedne hromade)

\newcommand{\trOratioAnteOfficium}{\translatioCantus{Otevři, Pane, má ústa, abych chválil tvé svaté jméno.
Očisti mé srdce od všech marnivých, zvrácených a~jiných myšlenek, osvěť rozum, rozněť cit,
abych mohl důstojně, soustředěně a~zbožně recitovat a~vysloužil si být
vyslyšen před tváří tvé velebnosti. Skrze Krista…}}

\newcommand{\trOratioPostOfficium}{\translatioCantus{\textit{Následující modlitbu
opatřil pro ty, kdo ji zbožně vyřknou po hodinkách, zesnulý papež Lev X.
odpustky za hříchy vzniklé při konání hodinek z~lidské křehkosti. Říká se
vkleče.}
Svatosvaté a~nerozdílné Trojici, ukřižovanému lidství našeho Pána Ježíše
Krista, přeblažené a~přeslavné plodné neporušenosti vždy Panny Marie
i~souhrnu všech svatých buď ode všeho stvoření věčná chvála, čest a~sláva, nám
pak buď dáno odpuštění všech hříchů, po nekonečné věky věků. Amen.}}

% HOURS ---

\newcommand{\trAntI}{\translatioCantus{Jasné narození slavné Panny Marie,
z pokolení (dosl. ze semene) Abrahámova, vzešlé z kmene Judova, z rodu Davidova.}}
\newcommand{\trAntII}{\translatioCantus{Dnes je Narození svaté Panny 
Marie, jejíž předrahý život osvěcuje všechny církve.}}

\newcommand{\trAntIII}{\translatioCantus{Maria, jež vzešla 
z královského rodu, září; myslí i duchem ji zbožně prosíme, aby 
nám pomáhala svými přímluvami.}}

\newcommand{\trAntIV}{\translatioCantus{Srdcem i duchem pějme Kristu 
k slávě o této svaté slavnosti vznešené Rodičky Boží Marie.}}

\newcommand{\trAntV}{\translatioCantus{Příjemně \notitia{?} 
oslavujme Narození blahoslavené Marie,
aby se ona za nás přimlouvala u Pána Ježíše Krista.}}

\newcommand{\trCapituli}{\translatioCantus{Před věky, na počátku mě stvořil, potrvám věčně. Ve svatém Stanu jsem před ním konala službu.}}

\newcommand{\trRespVesp}{\translatioCantus{Buď zdráva, Maria,
plná milosti: \grestar{} Pán s tebou. \Vbardot{} Požehnaná jsi mezi ženami,
a požehnaný plod života (ve smyslu lůna, břicha) tvého.}}

\newcommand{\trVersus}{\translatioCantus{\Vbardot{} Dnes je Narození svaté Panny Marie. \Rbardot{} Jejíž předrahý život osvěcuje všechny církve.}}

\newcommand{\trAntMagnificatI}{\translatioCantus{Konejme památku
veledůstojného narození slavné Panny Marie,
jíž se dostalo mateřské důstojnosti bez ztráty panenské cudnosti.}}

% Tento preklad je vice nez nejisty a ani alternativy, ktere jsem
% videl, me nepresvedcily...
\newcommand{\trAntBenedictus}{\translatioCantus{Slavnostně slavme 
dnešní narození Marie, vždy Panny a Rodičky Boží: v něm se objevuje
vysokost trůnu (totiž Marie, trůnu Božího Syna), aleluja.}}

\newcommand{\trAntMagnificatII}{\translatioCantus{Tvé narození,
Bohorodičko Panno, vyhlásilo radost celému světu:
z tebe totiž vzešlo Slunce spravedlnosti, Kristus, náš Bůh:
jenž zrušil kletbu a dal nám požehnání: přemohl smrt a dal nám život věčný.}}

\newcommand{\trOrationis}{\translatioCantus{Prosíme tě, Bože, 
uděl svým služebníkům dar nebeské milosti,
aby těm, jimž slehnutím blahoslavené Panny vyvstal počátek spásy, 
slavnost k poctě jejího narození přinesla
rozhojnění pokoje.
Skrze tvého Syna, našeho Pána Ježíše Krista, který s tebou žije a kraluje,
Bůh, v jednotě Ducha svatého po všechny věky věků.}}

\newcommand{\trFideliumAnimae}{\translatioCantus{\Vbardot{} Duše věrných ať pro
milosrdenství Boží odpočívají v~pokoji. \Rbardot{} Amen.}}

% Completorium

\newcommand{\trJubeDomne}{\translatioCantus{Rač, pane, požehnat.}}

\newcommand{\trComplBenedictio}{\translatioCantus{Pokojnou noc a~svatou smrt
nechť nám dopřeje všemohoucí Pán. \Rbardot{} Amen.}}

\newcommand{\trComplLectioBr}{\translatioCantus{Buďte střízliví, bděte.
Váš protivník Ďábel obchází jako lev řvoucí a~hledá, koho by pohltil.
Postavte se proti němu pevní ve víře.  Ale ty, Pane, smiluj se nad námi.
\Rbardot{} Bohu díky.}}

\newcommand{\trComplAntI}{\translatioCantus{Rač se smilovati nade mnou,
Hospodine, a vyslyš mou modlitbu.}}

\newcommand{\trComplCapituli}{\translatioCantus{Jsi přece, Hospodine,
uprostřed nás a~jmenujeme se po tobě.  Neopouštěj nás, Pane, náš Bože.}}

\newcommand{\trRespCompl}{\translatioCantus{Do tvých rukou, Pane, \grestar{}
poroučím svého ducha. \Vbardot{} Ty mne zachráníš, Pane, Bože věrný.}}

\newcommand{\trComplVersus}{\translatioCantus{\Vbardot{} Střez mne jako zřítelnici oka,
aleluja. \Rbardot{} Ve stínu svých křídel uschovej mne, aleluja.}}

\newcommand{\trAntSalvaNos}{\translatioCantus{Ochraňuj nás, Pane, když
bdíme, a~buď s~námi, když spíme, ať bdíme s~Kristem a~odpočíváme v~pokoji.}}

\newcommand{\trComplOrationis}{\translatioCantus{Zavítej, prosíme, Pane, sem
do našeho příbytku a~daleko od něj zažeň všechny úklady nepřítele. Ať tu
bydlí tví svatí andělé a~tvoje požehnání buď nad ním stále. Skrze…}}

\newcommand{\trSalveRegina}{\translatioCantus{Zdrávas Královno, matko
milosrdenství, živote, sladkosti a naděje naše, buď zdráva!
K tobě voláme, vyhnaní synové Evy,
k tobě vzdycháme, lkajíce a plačíce
v tomto slzavém údolí.
A proto, orodovnice naše,
obrať k nám své milosrdné oči
a Ježíše, požehnaný plod života svého,
nám po tomto putování ukaž,
ó milostivá, ó přívětivá,
ó přesladká, Panno Maria!}}

\newcommand{\trOraProNobis}{\translatioCantus{\Vbardot{} 
Oroduj za nás, svatá Boží Rodičko,
\Rbardot{} aby nám Kristus dal účast na svých zaslíbeních.}}

% Matutinum

\newcommand{\trMatInvitatorium}{\translatioCantus{}}

\newcommand{\trMatVeniteA}{\translatioCantus{Pojďte, chvalme s~radostí Pána,
s~jásotem slavme Boha, svou spásu; předstupme před tvář jeho s~díky, písně plesu pějme jemu.}}

\newcommand{\trMatVeniteB}{\translatioCantus{Neboť Bůh veliký jest Hospodin, a~král nade všecky bohy.
Jsouť v~jeho ruce všecky hlubiny země, temena hor jsou majetek jeho.}}

\newcommand{\trMatVeniteC}{\translatioCantus{Jehoť jest moře, neb on je učinil; i~souš
je dílo jeho rukou. Pojďme, klanějme se, padněme, klekněme před Pánem, svým
tvůrcem. Jeť on Pán, náš Bůh, a~my jsme lid, jejž on vodí a~ovce, jež pase.}}

\newcommand{\trMatVeniteD}{\translatioCantus{Kéž byste poslechli dnes hlasu jeho:
,,Nezatvrzujte svých srdcí jak v~Hádce, jak v~Pokušení na poušti, kde vaši otcové pokoušeli mne,
zkoušeli mne, ač vídali skutky mé.``}}

\newcommand{\trMatVeniteE}{\translatioCantus{Čtyřicet roků mrzel jsem se na to pokolení
a~řekl jsem: ,,Lid je to myslí stále bloudící``! Oni však nechtěli znáti mé cesty, takže jsem
přisáhl ve svém hněvu: ,,Nedojdou odpočinku mého!\mbox{}``}}

\newcommand{\trMatAntI}{\translatioCantus{}}

\newcommand{\trMatAntII}{\translatioCantus{}}

\newcommand{\trMatAntIII}{\translatioCantus{}}

\newcommand{\trMatVersusI}{\translatioCantus{}}

\newcommand{\trMatAbsolutioI}{\translatioCantus{Vyslyš Pane Ježíši Kriste
prosby svých služebníků \gredagger{} a~smiluj se nad námi, \grestar{} jenž
s~Otcem a~Duchem…}}

\newcommand{\trMatBenedictioI}{\translatioCantus{Rač, pane, požehnat.
Věčný Otec nám stále žehnej. \Rbardot{} Amen.}}

\newcommand{\trMatLecI}{\translatioCantus{Kéž by mě zulíbal polibky svých úst. 
Tvé milování je nad víno lahodnější;
vybraně voní tvé voňavky;
rozlévající se olej je tvé jméno,
proto tě dívky milují.
Strhni mě za sebou, poběžme!
Král mě uvedl do svých komnat;
budeš nám radostí a jásotem.
Víc než víno oslavíme tvé milování;
věru po právu jsi milován!
Snědá jsem, a přece krásná, jeruzalémské dcery,
jako stany kedarské,
jako šalmské závěsy.
}}

\newcommand{\trMatRespI}{\translatioCantus{}}

\newcommand{\trMatBenedictioII}{\translatioCantus{Rač, pane, požehnat.
Jednorozený Boží Syn nám žehnej \grestar{} a nám pomáhej. \Rbardot{} Amen.}}

\newcommand{\trMatLecII}{\translatioCantus{Nehleďte na mou osmahlou pleť:
to mě slunce ožehlo.
Synové mé matky se na mne rozzlobili,
poslali mě hlídat vinice.
A svou vinici, tu jsem nehlídala!
Pověz mi tedy, ty, jehož miluje mé srdce:
kam zavedeš své stádo pást,
kde ho necháš za poledne odpočívat?
Abych už nebloudila jako tulačka
poblíž stád druhů tvých.
Nevíš-li to, nejrásnější z žen,
jdi po stopách stáda
a kůzlata svá zaveď, ať se pasou
poblíž obydlí pastýřů.
Ke své klisně zapřažené do vozu faraonova
tebe, mé milá, přirovnávám.
Stále krásné jsou tvé líce s náušnicemi
i tvé hrdlo v náhrdelnících.}}

\newcommand{\trMatRespII}{\translatioCantus{}}

\newcommand{\trMatBenedictioIII}{\translatioCantus{Rač, pane, požehnat.
Milost Ducha Svatého ať osvítí nám smysly \grestar{} i srdce. \Rbardot{} Amen.}}

\newcommand{\trMatLecIII}{\translatioCantus{Zhotovíme ti zlaté náušnice
a kuličky ze stříbra.
Když král stoluje,
vydechuje můj nard svou vůni.
Můj milý je polštářek s myrhou,
jenž mi odpočívá mezi ňadry.
Můj milý je hrozen šáchoru
ve vinicích v Engadi.
Jak jsi krásná, milá moje,
jak jsi krásná!
Tvé oči jsou holubice.
Jak jsi krásný, můj milý,
jak líbezný!
Naše lože je samá zeleň.
Trámoví našeho domu je z cedru,
naše ostění z cypřiše.}}

\newcommand{\trMatRespIII}{\translatioCantus{}}

\newcommand{\trMatAntIV}{\translatioCantus{}}

\newcommand{\trMatAntV}{\translatioCantus{}}

\newcommand{\trMatAntVI}{\translatioCantus{}}

\newcommand{\trMatVersusII}{\translatioCantus{}}

\newcommand{\trMatAbsolutioII}{\translatioCantus{
Tvá milost a laskavost nechť nám pomáhá, jenž žiješ a vládneš s Otcem a Svatým Duchem na věky věků.}}

\newcommand{\trMatBenedictioIV}{\translatioCantus{Rač, pane, požehnat.
Bůh Otec všemohoucí, \grestar{} buď k nám milostivý a odpouštějící. \Rbardot{} Amen.}}

\newcommand{\trMatLecIV}{\translatioCantus{}}

\newcommand{\trMatRespIV}{\translatioCantus{}}

\newcommand{\trMatBenedictioV}{\translatioCantus{}}

\newcommand{\trMatLecV}{\translatioCantus{}}

\newcommand{\trMatRespV}{\translatioCantus{}}

\newcommand{\trMatBenedictioVI}{\translatioCantus{Rač, pane, požehnat.
Bůh rozněť v nás oheň své lásky. \Rbardot{} Amen.}}

\newcommand{\trMatLecVI}{\translatioCantus{}}

\newcommand{\trMatRespVI}{\translatioCantus{}}

\newcommand{\trMatAntVII}{\translatioCantus{}}

\newcommand{\trMatAntVIII}{\translatioCantus{}}

\newcommand{\trMatAntIX}{\translatioCantus{}}

\newcommand{\trMatVersusIII}{\translatioCantus{}}

\newcommand{\trMatAbsolutioIII}{\translatioCantus{Z okovů našich hříchů,
\grestar{} vysvoboď nás všemohoucí a milosrdný Pán. \Rbardot{} Amen.}}

\newcommand{\trMatBenedictioVII}{\translatioCantus{Rač, pane, požehnat.
Čtení evangelia nechť je nám \grestar{} spásou a ochranou. \Rbardot{} Amen.}}

\newcommand{\trMatLecVIIa}{\translatioCantus{
  Rodokmen Ježíše Krista, syna Davidova, syna Abrahámova:
  Abrahám zplodil Izáka,
  Izák zplodil Jakuba.}}

\newcommand{\trMatLecVIIb}{\translatioCantus{}}

\newcommand{\trMatRespVII}{\translatioCantus{}}

\newcommand{\trMatBenedictioVIII}{\translatioCantus{Rač, pane, požehnat.
\Rbardot{} Amen.}}

\newcommand{\trMatLecVIII}{\translatioCantus{}}

\newcommand{\trMatRespVIII}{\translatioCantus{}}

\newcommand{\trMatBenedictioIX}{\translatioCantus{Rač, pane, požehnat.
Do společnosti občanů nebes \grestar{} ať nás dovede král andělů.
\Rbardot{} Amen.}}

\newcommand{\trMatLecIX}{\translatioCantus{}}

% from the Czech Liturgia horarum
\newcommand{\trTeDeum}{\begin{translatioMulticol}{3}

Bože, tebe chválíme, 
tebe, Pane, velebíme.

Tebe, věčný Otče, 
oslavuje celá země.

Všichni andělé, 
cherubové i~serafové,

všechny mocné nebeské zástupy 
bez ustání volají:

Svatý, Svatý, Svatý, 
Pán, Bůh zástupů.

Plná jsou nebesa i~země 
tvé vznešené slávy.

Oslavuje tě 
sbor tvých apoštolů,

chválí tě 
velký počet proroků,

vydává o~tobě svědectví 
zástup mučedníků;

a~po celém světě 
vyznává tě tvá církev:

neskonale velebný, 
všemohoucí Otče,

úctyhodný Synu Boží, 
pravý a~jediný,

božský Utěšiteli, 
Duchu svatý.

Kriste, Králi slávy, 
tys od věků Syn Boha Otce;

abys člověka vykoupil, 
stal ses člověkem a~narodil ses z~Panny;

zlomil jsi osten smrti 
a~otevřel věřícím nebe;

sedíš po Otcově pravici 
a~máš účast na jeho slávě.

Věříme, že přijdeš soudit, 

a~proto tě prosíme:
přispěj na pomoc svým služebníkům, 
vždyť jsi je vykoupil svou předrahou krví;

dej, ať se radují s~tvými svatými 
ve věčné slávě.

Zachraň, Pane, svůj lid, žehnej svému dědictví, 
veď ho a~stále pozvedej.

Každý den tě budeme velebit 
a~chválit tvé jméno po všechny věky.

Pomáhej nám i~dnes, 
ať se nedostaneme do područí hříchu.

Smiluj se nad námi, Pane, 
smiluj se nad námi.

Ať spočine na nás tvé milosrdenství, 
jak doufáme v~tebe.

Pane, k~tobě se utíkáme, 
ať nejsme zahanbeni na věky. 
\end{translatioMulticol}}

\newcommand{\trMatEvangelium}{\translatioCantus{
  Rodokmen Ježíše Krista, syna Davidova, syna Abrahámova:
  Abrahám zplodil Izáka,
  Izák zplodil Jakuba,
  Jakub zplodil Judu a jeho bratry,
  Juda zplodil Farese a Zaru z Tamary,
  Fares zplodil Esroma,
  Esrom zplodil Arama,
  Aram zplodil Aminadaba,
  Aminadab zplodil Naasona,
  Naason zplodil Salmona,
  Salmon zplodil Boaze z Rahaby,
  Boaz zplodil Jobeda z Rut,
  Jobed zplodil Jessea,
  Jesse zplodil krále Davida.
  David zplodil Šalomouna z Uriášovy ženy,
  Šalomoun zplodil Roboama,
  Roboam zplodil Abiu,
  Abia zplodil Asu,
  Asa zplodil Josafata,
  Josafat zplodil Jorama,
  Joram zplodil Oziáše,
  Oziáš zplodil Joatama,
  Joatam zplodil Achaze,
  Achaz zplodil Ezechiáše,
  Ezechiáš zplodil Manasesa,
  Manases zplodil Amona,
  Amon zplodil Josiáše,
  Josiáš zplodil Jechoniáše a jeho bratry;
  tehdy došlo k odvlečení do Babylonu.
  Po odvlečení do Babylonu:
  Jechoniáš zplodil Salatiela,
  Salatiel zplodil Zorobabela,
  Zorobabel zplodil Abiuda,
  Abiud zplodil Eljakima,
  Eljakim zplodil Azora,
  Ator zplodil Sadoka,
  Sadok zplodil Achima,
  Achim zplodil Eliuda,
  Eliud zplodil Eleazara,
  Eleatar zplodil Matana,
  Matan zplodil Jakuba,
  Jakub zplodil Josefa, manžela Marie,
  z níž se narodil Ježíš, který se nazývá Kristus.}}

\newcommand{\trTeDecetLaus}{\translatioCantus{Tobě chvála, Tobě zpěvy, Tobě
sláva, Bohu Otci i~Synu i~Svatému Duchu, na věky věků. \Rbardot{} Amen.}}

% MASS ---

\newcommand{\trIntroitus}{\translatioCantus{Radujme se všichni
v Pánu, slavíce svátek ke cti Panny Marie: z něj se radují andělé
a spoluchválí Božího Syna. \textit{\color{red}Žl.} Má ústa vydala dobré slovo,
přednáším svá díla králi.}}

\newcommand{\trGraduale}{\translatioCantus{Požehnaná a ctihodná jsi,
Panno Maria: nedotčená (co do panenství) jsi byla shledána matkou
Spasitele. \Vbardot{} Panno Boží Rodičko, ten, jehož nepojme ani celý svět,
se uzavřel do tvých útrob, když se stal člověkem.}}

\newcommand{\trAlleluia}{\translatioCantus{Aleluja. \Vbardot{} Skvělá slavnost
slavné Panny Marie, z pokolení (dosl. ze semene) Abrahámova, vzešlé z kmene 
Judova, z rodu Davidova.}}

\newcommand{\trOffertorium}{\translatioCantus{Blažená jsi, Panno Maria,
tys nosila Stvořitele všeho; porodila jsi toho, který tě utvořil,
a na věky zůstáváš Pannou.}}

\newcommand{\trCommunio}{\translatioCantus{Budou mě blahoslavit
všechna pokolení, protože mi učinil veliké věci ten, který je mocný.}}

% LITTLE HOURS ---

\newcommand{\trVersusTertia}{\translatioCantus{\Vbardot{} \Rbardot{}}}

\newcommand{\trCapituliEtSic}{\translatioCantus{
Tak jsem se usadila na Sionu a v milovaném městě jsem nalezla odpočinek,
v Jeruzalémě vykonávám svou moc.
Zakořenila jsem u lidu plného slývy, na panství Páně, v jeho dědictví.}}

\newcommand{\trVersusSexta}{\translatioCantus{\Vbardot{} \Rbardot{}}}

\newcommand{\trCapituliInPlateis}{\translatioCantus{
Na planině jako skořicovník a akant jsem vydávala vůni, jako vybraná myrha
jsem voněla.}}

\newcommand{\trVersusNona}{\translatioCantus{\Vbardot{} \Rbardot{}}}

% Parts used several times in the booklet

%%%% COMMON

\newcommand{\scripturaMagnificat}{\scriptura{Lucæ 1, 46-55}}

\newcommand{\kyrieEleison}{\includescore{\ccommunesAR/kyrieeleison.tex}}

\newcommand{\dominusVobiscum}{
  
  ℣. Dóminus vobíscum.
  ℟. Et cum spíritu tuo.

  %\rubrica{In choro monialium loco \textnormal{Dóminus vobíscum} dicitur:}
  \rubrica{Absente sacerdote vel diacono, loco \textnormal{Dóminus vobíscum}
  dicitur:}

  ℣. Dómine exáudi oratiónem meam.
  ℟. Et clamor meus ad te véniat.
  
}

% when the Dominus vobiscum is repeated on the same page.
\newcommand{\dominusVobiscumRep}{
  \rubrica{Iterum dicitur \textnormal{Dóminus vobíscum} aut 
    \textnormal{Dómine exáudi}.}
}

\newcommand{\benedicamusDomino}{%
  Benedicámus Dómino. \rubricatum{Pg. \pageref{tc:benedicamus:minor}.}}

\newcommand{\perDominum}{%
  Per Dóminum nostrum Jesum Christum Fílium tuum,
  qui tecum vivit et regnat in unitáte Spíritus Sancti Deus:
  per ómnia sǽcula sæculórum. ℟. Amen.
}
\newcommand{\perChristum}{%
  Per Christum Dóminum nostrum. ℟. Amen.
}

\newcommand{\invitatoriumIntegrum}{
  \hfill \rubricatum{Repetitur integrum Invitatorium.}
}
\newcommand{\invitatoriumAltera}{
  \hfill \rubricatum{Repetitur altera pars Invitatorii.}
}

\newcommand{\tuAutem}{
  \hspace{0.5cm}
  Tu autem, Dómine, miserére nobis.
  ℟. Deo grátias. 
}

\newcommand{\initiumHoraeMinores}{
  \vspace{2mm}
  \deusInAdiutorium

  \rubrica{Hymnus \textnormal{Meménto, rerum Cónditor} 
    ut ad Primam, pg. \pageref{prima}.}
  \vspace{5mm}
}

\newcommand{\finisHoraeMinores}{
  \vspace{7mm}
  \rubrica{Sequitur \textnormal{Kýrie eléison} et omnia usque ad finem horæ
    ut in Vesperis, p. \pageref{vesperaefinis}. Sed Oratio fit ut infra.}
  \vspace{2mm}
}

\newcommand{\rubricaLaudesPrincipium}{
  \rubrica{Si Laudes separentur a Matutino:}

  \deusInAdiutorium
  \vspace{5mm}
}

\newcommand{\laudesAntiphonaBMV}{
  \rubrica{Si terminatur Officium, alioquin, si alia subsequatur Hora,
    in fine ultimæ Horæ dicitur:}

  \rubrica{\textnormal{Pater noster,} totum secreto,
    eoque recitato subjungitur}

  ℣. Dóminus det nobis suam pacem.
  ℟. Et vitam ætérnam. Amen.

  \rubrica{Et immediate dicitur, cum suis Versu et Oratione,
    una ex finalibus beatæ Mariæ Virginis Antiphonis, pro diversitate Temporis,
    pg. \pageref{antiphonaefinales}, flexis quidem genibus;
    diebus autem Dominicis, a Vesperis Sabbati inclusive et Tempore Paschali,
    stando.}

  \rubrica{Postea concluditur:}

  ℣. Divínum auxílium maneat semper nobíscum.
  ℟. Amen.
}

%%%% EXTRA ADVENTUM

\newcommand{\domineLabiaMea}{%
  Dómine lábia mea apéries.
  \rubricatum{Pg. \pageref{tc:dominelabia}.}
}

\newcommand{\deusInAdiutorium}{%
  Deus in adjutórium meum inténde.
  \rubricatum{Pg. \pageref{tc:deusinadiutorium}.}}

% Deus in adiutorium with spaces around for places where it stands
% alone before the hour title and the beginning of it's proper content.
\newcommand{\deusInAdiutoriumS}{%
  \vspace{2mm}
  \deusInAdiutorium
  \vspace{5mm}}

% antiphonae ad laudes

\newcommand{\antiphonaLaudI}{
  \antiphona{VII a}{cantus/arom12/antlaud1.tex}{\trLaudAntI}}
\newcommand{\antiphonaLaudII}{
  \antiphona{VIII G}{cantus/arom12/antlaud2.tex}{\trLaudAntII}}
\newcommand{\antiphonaLaudIII}{
  \antiphona{IV A*}{cantus/arom12/antlaud3.tex}{\trLaudAntIII}}
\newcommand{\antiphonaLaudIV}{
  \antiphona{VII c2}{cantus/arom12/antlaud4.tex}{\trLaudAntIV}}
\newcommand{\antiphonaLaudV}{
  \antiphona{I g2}{cantus/arom12/antlaud5.tex}{\trLaudAntV}}

% antiphonae ad vesperas

\newcommand{\antiphonaI}{
  \antiphona{III a}{cantus/arom12/ant1.tex}{\translatioAntI}}
\newcommand{\antiphonaII}{
  \antiphona{IV A*}{cantus/arom12/ant2.tex}{\translatioAntII}}
\newcommand{\antiphonaIII}{
  \antiphona{III b}{cantus/arom12/ant3.tex}{\translatioAntIII}}
\newcommand{\antiphonaIV}{
  \setspaceafterinitial{4mm plus 0em minus 0em}
  \setspacebeforeinitial{4mm plus 0em minus 0em}
  \antiphona{VIII G}{cantus/arom12/ant4.tex}{\translatioAntIV}
  \spacearoundinitialNormal
}
\newcommand{\antiphonaV}{
  \antiphona{IV A*}{cantus/arom12/ant5.tex}{\translatioAntV}}

\newcommand{\capitulumAbInitio}{
  \scriptura{Sir 24,14}

  \includescore{cantus/arom12/capitulum-AbInitio.tex}

  % preklad Jeruz. bible
  \translatioCapituli
}

\newcommand{\versiculusDiffusaEst}{
  \includescore{cantus/arom12/versiculus-DiffusaEst.tex}

  \trMatVers
}

\newcommand{\hymnusMementoRerumConditor}{
  \noindent
  \begin{minipage}{20cm}
    \superInitialam{II}
    \includescore{temporalia/hymnus-MementoRerumConditor.tex}
  \end{minipage}
  \hspace{1cm}
  \begin{minipage}{5cm}
    \footnotesize
    \trHymMementoRerumConditor
  \end{minipage}
}

\newcommand{\finisAnteOrationem}{
  \kyrieEleison

  \vspace{2mm}

  \dominusVobiscum

  \vspace{3mm}

  \pars{Oratio.}
}
\newcommand{\finisPostOrationem}{
  \dominusVobiscumRep

  \vspace{2mm}
}
\newcommand{\finisPostBenedicamus}{
}

\newcommand{\commemoratioExtraAdventum}{
  \superInitialam{VII}
  \includescore{cantus/arom12/comm.tex}

  \vspace{3mm}

  \noindent ℣. Lætámini in Dómino et exsultáte, justi.\\
  ℟. Et gloriámini, omnes recti corde.

  \vspace{3mm}

  Prótege, Dómine, pópulum tuum, et, Apostolórum tuórum
  Petri et Pauli et aliórum Apostolórum patrocínio confidéntem,
  perpétua defensióne consérva.

  Omnes Sancti tui, quǽsumus, Dómine, nos ubíque ádjuvent:
  ut, dum eórum mérita recólimus, patrocínia sentiámus:
  et pacem tuam nostris concéde tempóribus, et ab Ecclésia tua cunctam
  repélle nequítiam:
  iter, actus et voluntátes nostras, et ómnium famulórum tuórum,
  in salútis tuæ prosperitáte dispóne, benefactóribus nostris sempitérna bona
  retríbue, et ómnibus fidélibus defúnctis réquiem ætérnam concéde.
  
  \perDominum
}

% end of the hours - extra Adventum ut in Vesperis
\newcommand{\inFineHorarumExtraAdventum}{
  \finisAnteOrationem

  \textusEtTranslatio{
    Concéde nos fámulos tuos, quǽsumus Dómine Deus,
    perpétua mentis et córporis sanitáte gaudére:
    et gloriósa beátæ Maríæ semper Vírginis intercessióne,
    a præsénti liberári tristítia, et ætérna pérfrui lætítia.\hspace{0.5cm}
    \perChristum
  }{\trOratioVesperae}{8cm}

  \vfill

  \pars{Commemoratio de Sanctis} \rubricatum{(pro Vesperis et Laudibus tantum).}

  \commemoratioExtraAdventum

  \vfill

  \finisPostOrationem

  \benedicamusDomino

  \finisPostBenedicamus
}

% end of the hours - extra Adventum ut in Laudibus
\newcommand{\inFineHorarumExtraAdventumLaudes}{
  \finisAnteOrationem

  \textusEtTranslatio{
    Deus, qui de beátæ Maríæ Vírginis útero Verbum tuum,
    Angelo nuntiánte, carnem suscípere voluísti: 
    præsta supplícibus tuis; ut qui vere eam Genitrícem Dei crédimus, 
    ejus apud te intercessiónibus adjuvémur.
    Per eúmdem Christum Dóminum nostrum. 
    ℟. Amen.
  }{\trLaudOratio}{10cm}

  \vfill

  \pars{Commemoratio de Sanctis}

  \commemoratioExtraAdventum

  \vfill

  \finisPostOrationem

  \benedicamusDomino

  \finisPostBenedicamus

  \laudesAntiphonaBMV
}


\newcommand{\rubricaAnnuntiatioQuadragesimaAlleluja}{
  \rubrica{In Festo Annuntiationis, quoties ipsum Tempore Quadragesimæ
    celebretur, \textnormal{Allelúja} omittitur.}}

\newcommand{\rubricaLectiones}{
  \rubrica{Sequuntur lectiones, pg. \pageref{lectiones}.}
}

%%%% IN ADVENTU

\newcommand{\adventAntiphonaI}{
  \antiphona{VIII G}{cantus/arom12/adv_ant1.tex}{\trAdvAntI}}
\newcommand{\adventAntiphonaII}{
  \rubricaAnnuntiatioQuadragesimaAlleluja

  \antiphona{I g}{cantus/arom12/adv_ant2.tex}{\trAdvAntII}}
\newcommand{\adventAntiphonaIII}{
  \rubricaAnnuntiatioQuadragesimaAlleluja

  \antiphona{VIII G}{cantus/arom12/adv_ant3.tex}{\trAdvAntIII}}
\newcommand{\adventAntiphonaIV}{
  \antiphona{I f}{cantus/arom12/adv_ant4.tex}{\trAdvAntIV}}
\newcommand{\adventAntiphonaV}{
  \antiphona{VIII c}{cantus/arom12/adv_ant5.tex}{\trAdvAntV}}

% ad Magnificat, Benedictus, Nunc Dimittis
\newcommand{\adventAntiphonaSpiritusSanctus}{
  \rubricaAnnuntiatioQuadragesimaAlleluja

  \antiphona{VIII G}{cantus/arom12/adv_antmag.tex}{\trAntSpiritusSanctus}}

\newcommand{\commemoratioInAdventu}{
  \superInitialam{V}
  \includescore{cantus/arom12/adv_comm.tex}

  \vspace{3mm}

  \noindent ℣. Ecce apparébit Dóminus super nubem cándidam.\\
  ℟. Et cum eo sanctórum míllia.

  \vspace{3mm}

  Consciéntias nostras, quǽsumus, Dómine, visitándo purífica:
  ut, véniens Jesus Christus, Fílius tuus, Dóminus noster,
  cum ómnibus Sanctis, parátam sibi in nobis invéniat mansiónem:
  Qui tecum vivit et regnat in unitáte Spíritus Sancti Deus,
  per ómnia sǽcula sæculórum.
  ℟. Amen.
}

\newcommand{\inFineHorarumInAdventu}{
  \finisAnteOrationem

  \textusEtTranslatio{
    Deus, qui de beátæ Maríæ Vírginis útero Verbum tuum,
    Angelo nuntiánte, carnem suscípere voluísti: 
    præsta supplícibus tuis;
    ut qui vere eam Genitrícem Dei crédimus, 
    ejus apud te intercessiónibus adjuvémur.

    Per eúmdem Christum Dóminum nostrum.
    ℟. Amen.
  }{\trAdvOratio}{10cm}

  \vfill

  \pars{Commemoratio de Sanctis} \rubricatum{(pro Vesperis et Laudibus tantum).}

  \commemoratioInAdventu

  \vfill

  \finisPostOrationem

  \benedicamusDomino

  \finisPostBenedicamus
}

\newcommand{\rubricaFinisHoraeAdvent}{
  \rubrica{Sequitur \textnormal{Kýrie eléison} etc. usque ad finem horæ
    ut in Vesperis, p. \pageref{vesperaefinisadvent}.}}

\newcommand{\initiumHoraeMinoresAdvent}{
  \vspace{2mm}
  \deusInAdiutorium

  \rubrica{Hymnus \textnormal{Meménto, rerum Cónditor} 
    ut ad Primam Extra Adventum, pg. \pageref{prima}.}
  \vspace{5mm}
}

%%%% TEMPORE NATIVITATIS

\newcommand{\nativitasAntiphonaI}{
  \antiphona{VI F}{cantus/arom12/nat_ant1.tex}{\trNatAntI}}
\newcommand{\nativitasAntiphonaII}{
  \antiphona{III a2}{cantus/arom12/nat_ant2.tex}{\trNatAntII}}
\newcommand{\nativitasAntiphonaIII}{
  \antiphona{IV E}{cantus/arom12/nat_ant3.tex}{\trNatAntIII}}
\newcommand{\nativitasAntiphonaIV}{
  \antiphona{I f}{cantus/arom12/nat_ant4.tex}{\trNatAntIV}}
\newcommand{\nativitasAntiphonaV}{
  \antiphona{II D}{cantus/arom12/nat_ant5.tex}{\trNatAntV}}

% ad Magnificat, Nunc Dimittis
\newcommand{\nativitasAntiphonaMagnificat}{
  \antiphona{II A}{cantus/arom12/nat_antmag.tex}{\trAntMagnumHaereditatisMysterium}}

\newcommand{\nativitasAntiphonaBenedictus}{
  \antiphona{VIII G}{cantus/arom12/nat_antben.tex}{\trNatAntBenedictus}}

\newcommand{\inFineHorarumPostNativitatem}{
  \finisAnteOrationem

  \textusEtTranslatio{
    Deus, qui salútis ætérnæ, beátæ Maríæ virginitáte fecúnda,
    humáno géneri prǽmia præstitísti:
    tríbue, quǽsumus;
    ut ipsam pro nobis intercédere sentiámus,
    per quam merúimus auctórem vitæ suscípere,
    Dóminum nostrum Jesum Christum, Fílium tuum:

    Qui tecum vivit et regnat in unitáte Spíritus Sancti Deus,
    per ómnia sǽcula sæculórum.
    ℟. Amen.
  }{\trNatOratio}{10cm}
    %%% finis

  \vfill

  \pars{Commemoratio de Sanctis} \rubricatum{(pro Vesperis et Laudibus tantum).}

  \commemoratioExtraAdventum

  \vfill

  \finisPostOrationem

  \benedicamusDomino

  \finisPostBenedicamus
}

\newcommand{\rubricaFinisHoraePostNativitatem}{
  \rubrica{Sequitur \textnormal{Kýrie eléison} etc. usque ad finem horæ
    ut in Vesperis, p. \pageref{vesperaefinisnativitas}.}}


%%%% TEMPORE PASCHALI

\newcommand{\paschaAntiphona}{
  \antiphona{I D2}{cantus/arom12/pasch_antevang.tex}{\trAntReginaCaeli}}

%%%% piae orationes

\newcommand{\anteOfficiumOratio}{
\lettrine{A}{peri,} Dómine, os meum ad benedicéndum nomen sanctum tuum:
munda quoque cor meum ab ómnibus vanis, pervérsis, et aliénis
cogitatiónibus:
intelléctum illúmina, afféctum inflámma,
ut digne, atténte ac devóte hoc Offícium recitáre váleam,
et exaudíri mérear ante conspéctum Divínæ Majestátis tuæ.
Per Christum, Dominum nostrum.
℟. Amen.

Dómine, in unióne illíus divínæ intentiónis,
qua ipse in terris laudes Deo persolvísti,
has tibi Horas \rubricatum{(vel \textnormal{hanc tibi Horam})} persólvo.
}

\newcommand{\postOfficiumOratio}{
\rubrica{
  Orationem sequentem devote post Officium recitantibus
  Leo Papa X. defectus, et culpas in eo persolvendo ex humana
  fragilitate contractas, indulsit, et dicitur flexis genibus.
}

\lettrine{S}{acrosánctæ} et indivíduæ Trinitáti,
crucifíxi Dómini nostri Jesu Christi humanitáti,
beatíssimæ et gloriosíssimæ sempérque Vírginis Maríæ
fecúndæ integritáti, 
et ómnium Sanctórum universitáti
sit sempitérna laus, honor, virtus et glória
ab omni creatúra,
nobísque remíssio ómnium peccatórum,
per infiníta sǽcula sæculórum.
℟. Amen.

\noindent ℣. Beáta víscera Maríæ Virginis, quæ portavérunt
ætérni Patris Fílium.\\
℟. Et beáta úbera, quæ lactavérunt Christum Dominum.

\rubrica{Et dicitur secreto \textnormal{Pater noster.} et \textnormal{Ave María.}}
}


\newcommand{\annusEditionis}{2013}
% directories
\newcommand{\ccommunesAM}{../../cantuscommunes/amon33}
\newcommand{\ccommunesAR}{../../cantuscommunes/arom12}

\begin{document}

% GREGORIO general settings:
% staff size
\setgrefactor{15}
% space around the initial.
\setspaceafterinitial{2.2mm plus 0em minus 0em}
\setspacebeforeinitial{2.2mm plus 0em minus 0em}
% initial font. 
\def\greinitialformat#1{%
{\fontsize{43}{43}\selectfont #1}%
}

\pagestyle{empty}

% multicols setting
\setlength{\columnseprule}{1pt} % cara oddelujici sloupce
\setlength{\columnsep}{20pt} % prostor mezi sloupci

%%%% Titulni stranka
\begin{titulusOfficii}
\nomenFesti{Officium Parvum Beatæ Mariæ Virginis.}
\end{titulusOfficii}

\vfill

\begin{center}
Ad usum et secundum consuetudines chori \guillemotright Conventus Choralis\guillemotleft.

Editio Sancti Wolfgangi \annusEditionis
\end{center}

\pagebreak

\quasiHora{Regulæ generales}

\begin{multicols}{2}

1. Officium parvum B. Mariæ Virg. octo complectitur partes, quas Horas vocant;
Matutinum nempe, Laudes, Primam, Tertiam, Sextam, Nonam, Vesperas, denique
Completorium. Quamlibet autem Horam separatim ab alia recitare licet.

2. Horas hoc fere tempore recitare convenit, nempe Matutinum cum Laudibus 
pridie ab hora secunda pomeridiana, Primam, Tertiam, Sextam et Nonam mane,
Vesperas et Completorium post meridiem.
Tempore autem Quadragesimæ, id est a Sabbato~I in Quadragesima usque ad
Sabbatum sanctum inclusive, satius est Vesperas ante comestionem recitare.

3. Ad lucrandas indulgentias Officio parvo B.~Mariæ Virg. annexas, in publica
recitatione omnes lingua latina uti debent, in privata autem recitatione
qualibet versione fideli et probata uti valent.
Porro recitatio retinenda est adhuc privata, quamvis locum habeat in communi
intra sæpta domus religiosæ, immo et in ipsa Ecclesia vel publico Oratorio
prædictæ domui annexis, sed januis clausis (S.~C. Indulg. 13~Sept. 1888,
28~Aug. 1903, 18~Dec. 1906).

4. Quamvis Officium parvum B.~Mariæ Virg. infra annum sit unum idemque,
exceptis partibus propriis Tempore Adventus et post Nativitatem Domini,
communiter tamen in tria Officia distinguitur:
Primum ergo Officium est dicendum a Matutino diei 3 Februarii usque ad Nonam
Sabbati ante Dominicam I Adventus inclusive, præterquam in Festo Annuntiationis
B.~M.~V., in quo dicitur Officium ut in Adventu.
Secundum recitandum est a Vesperis Sabbati ante Dominicam~I Adventus usque ad
Nonam Vigiliæ Nativitatis Domini inclusive et in Festo Annuntiationis
B.~Mariæ Virg.
Tertium persolvendum est a Vesperis Diei 24 Decembris usque ad Completorium
diei 2 Februarii inclusive.

5. Hoc Officium quovis tempore persolvendum est plane, prout in Breviario
præscribitur. Quamobrem Tempore Passionis, incluso etiam ultimo Triduo 
sacro Majoris Hebdomadæ, non omittitur \rubricatum{Glória Patri} in Invitatorio
ac tertio Responsorio; nec Tempore Paschali numerus Antiphonarum imminuitur,
neque Invitatorio, Antiphonis, Versibus et Responsoriis additur in fine
\rubricatum{Allelúja} (Rubr. gen. Brev. tit.~37, num.~2. S.~R.~C. num. 1334 ad~6).

6. Tempore Paschali, idest a Vesperis Sabbati sancti usque ad Nonam Sabbati
infra Octavam Pentecostes inclusive, Officium parvum B.~Mariæ Virg.
dicitur sicuti per Annum; sed ad \rubricatum{Benedíctus}, ad \rubricatum{Magníficat}
et ad \rubricatum{Nunc dimíttis} dicitur Antiphona \rubricatum{Regína cæli}.

7. In Festo Annuntiationis (a Matutino usque ad Completorium inclusive)
idem Officium recitatur, quod præscribitur Tempore Adventus, et in fine
dicitur Antiphona \rubricatum{Ave, Regína cælórum} vel \rubricatum{Regina cæli}
juxta temporis diversitatem. Si hoc Festum celebretur Tempore Quadragesimæ,
loco \rubricatum{Allelúja} in principio omnium Horarum dicitur
\rubricatum{Laus tibi, Dómine, Rex ætérnæ glóriæ}.

8. Quando Officium parvum B.~Mariæ Virg. recitatur separatim ab Officio divino,
Hymnus \rubricatum{Te Deum} dicitur 
a Nativitate Domini usque ad Sabbatum ante Dominicam Septuagesimæ inclusive
et a Dominica Resurrectionis usque ad Sabbatum ante Dominicam I Adventus
pariter inclusive;
in Adventu autem et a Septuagesima usque ad Pascha nonnisi in Festis B.~Mariæ
Virg. (S.~R.~C. n.~3572 ad~1 et 3659), quæ in Ecclesia universali
celebrantur vel in Kalendario approbato respectivæ Dioecesis vel Instituti
religiosi vel Ecclesiæ sive Oratorii assignantur,
atque in Festo sancti Joseph.
Quando dicitur \rubricatum{Te Deum}, in fine secundi Responsorii adjungitur
℣. \rubricatum{Glória Patri, et Fílio, et Spirítui Sancto}, ac repetitur
altera pars Responsorii.

9. Ante Orationem, etiam quando quis solus recitat Officium,
semper dicitur Versus \rubricatum{Dominus vobiscum} et respondetur
\rubricatum{Et cum spiritu tuo}. Qui Versus non dicitur ab eo, qui non est saltem
in ordine Diaconatus. Si quis autem ad Diaconatus ordinem non pervenerit,
ejus loco dicat Versum \rubricatum{Dómine, exáudi oratiónem meam},
et ei respondetur \rubricatum{Et clamor meus ad te véniat}.
Deinde dicitur \rubricatum{Orémus}, postea Oratio.
Et post ultimam Orationem repetitur ℣.~\rubricatum{Dóminus vobíscum}
vel \rubricatum{Dómine exáudi}. (Rubr. gen. tit.~30, n.~3.)

10. Omnes manu extensa se signent signo crucis a fronte ad pectus et a sinistro
humero ad dexterum
ad \rubricatum{Benedíctus} et ad \rubricatum{Magníficat}. Ceterum servetur
consuetudo quoad sugnum crucis communiter faciendum ad alias Officii partes,
scilicet \rubricatum{Dómine, lábia mea; Convérte nos, Deus; Deus, in adjutórium;
Nunc dimíttis;} et ad benedictionem in fine Completorii.

\end{multicols}


\vfill

\pagebreak

\renewcommand{\headrulewidth}{0pt} % no horiz. rule at the header
\fancyhf{}
\pagestyle{fancy}

\divisio{Extra Adventum.}

\rubrica{Quod dicitur a Matutino diei 3 Februarii usque ad Nonam
Sabbati post Cineres inclusive.}

% Extra Adventum.

\hora{Ad Vesperas.} %%%%%%%%%%%%%%%%%%%%%%%%%%%%%%%%%%%%%%%%

\vspace{1cm}
\deusInAdiutorium

\vfill

\pagebreak

%%% Psalms

\pars{psalmus 1.}

\antiphonaI

\scriptura{Psalmus 109.}

\includescore{temporalia/ps109-initium-iii-a-auto.tex}

\psalmusEtTranslatio{temporalia/ps109-iii-a.tex}{temporalia/ps109-boh.tex}

\vfill

\pagebreak

\pars{psalmus 2.}

\antiphonaII

\scriptura{Psalmus 112.}

\includescore{temporalia/ps112-initium-iv-A-auto.tex}

\psalmusEtTranslatio{temporalia/ps112-iv-a.tex}{temporalia/ps112-boh.tex}

\vfill

\pagebreak

\pars{psalmus 3.}

\antiphonaIII

\scriptura{Psalmus 121.}

\includescore{temporalia/ps121-initium-iii-b-auto.tex}

\psalmusEtTranslatio{temporalia/ps121-iii-b.tex}{temporalia/ps121-boh.tex}

\vfill

\pagebreak

\pars{psalmus 4.}

\antiphonaIV

\scriptura{Psalmus 126.}

\includescore{temporalia/ps126-initium-viii-G-auto.tex}

\psalmusEtTranslatio{temporalia/ps126-viii-g.tex}{temporalia/ps126-boh.tex}

\vfill

\pagebreak

\pars{psalmus 5.}

\antiphonaV

\scriptura{Psalmus 147.}

\includescore{temporalia/ps147-initium-iv-A-auto.tex}

\psalmusEtTranslatio{temporalia/ps147-iv-a.tex}{temporalia/ps147-boh.tex}

\vfill

\pagebreak

%%% capitulum

\pars{Capitulum.}

\label{vesperaecapitulum}

\capitulumAbInitio

%%% hymnus

\pars{Hymnus}

\superInitialam{VII}
\includescore{temporalia/hymnus-AveMarisStella.tex}

\versiculusDiffusaEst

\pagebreak

%%% Magnificat

\pars{Canticum Beatae Mariae Virginis}

\antiphona{II D}{cantus/arom12/antmag}{}

\scripturaMagnificat

\includescore{../../tonipsalmorum/arom12/magnificat-initium-ii-D.tex}

\psalmusEtTranslatioB{temporalia/magnificat-ii-d.tex}{temporalia/magnificat-boh.tex}{10cm}

\vfill

\pagebreak

%%% oratio

\label{vesperaefinis}

\inFineHorarumExtraAdventum

\vfill



\pagebreak

\label{pars:completorium}

% Extra Adventum.

\hora{Ad Completorium.} %%%%%%%%%%%%%%%%%%%%%%%%%%%%%%%%%%%%%%%%

\vspace{1cm}

\rubrica{Hebdomadarius dicit clara voce, pollice signans sibi pectus
  signo crucis, quod et alii faciunt, Versum:}

\includescore{\ccommunesAR/convertenosdeus.tex}

\vspace{5mm}

\deusInAdiutorium

\vfill

\pagebreak

\pars{Psalmus 1.}

\scriptura{Psalmus 128.}

\includescore{temporalia/ps128-initium-dir-auto.tex}

\psalmusEtTranslatio{temporalia/ps128-dir.tex}{temporalia/ps128-boh.tex}

\vfill


\pars{Psalmus 2.}

\scriptura{Psalmus 129.}

\includescore{temporalia/ps129-initium-dir-auto.tex}

\psalmusEtTranslatio{temporalia/ps129-dir.tex}{temporalia/ps129-boh.tex}


\pagebreak

\pars{Psalmus 3.}

\scriptura{Psalmus 130.}

\includescore{temporalia/ps130-initium-dir-auto.tex}

\psalmusEtTranslatio{temporalia/ps130-dir.tex}{temporalia/ps130-boh.tex}

\vfill

\pagebreak

\pars{Hymnus}

\superInitialam{II}
\includescore{temporalia/hymnus-MementoRerumConditor.tex}

\pars{Capitulum}

\rubrica{Extra Adventum.}

\scriptura{Sir 24,24}

\includescore{cantus/arom12/capitulum-EgoMater.tex}

℣. Ora pro nobis sancta Dei \textbf{Gé}nitrix.

℟. Ut digni efficiámur promissiónibus \textbf{Chris}ti.

\rubrica{Tempore Adventus.}

\scriptura{Is 7,14-15}

\includescore{cantus/arom12/capitulum-EcceVirgo.tex}

℣. Angelus Dómini nuntiávit Ma\textbf{rí}æ.

℟. Et concépit de Spíritu \textbf{Sanc}to.



\vfill

\pagebreak

\pars{Canticum Simeonis}

%%%
\rubrica{Extra Adventum.}

\superInitialam{VII a}

\includescore{cantus/arom12/antnuncdim.tex}

\scriptura{Luc 2,29-32}

\includescore{temporalia/nuncdimittis-initium-vii-a-auto.tex}

\psalmusEtTranslatio{temporalia/nuncdimittis-vii-a.tex}{temporalia/nuncdimittis-boh.tex}

%%%
\rubrica{Tempore Adventus.}

\adventAntiphonaSpiritusSanctus

\scriptura{Luc 2,29-32}

\includescore{temporalia/nuncdimittis-initium-viii-G-auto.tex}

\psalmusEtTranslatio{temporalia/nuncdimittis-viii-g.tex}{temporalia/nuncdimittis-boh.tex}

%%%
\rubrica{Tempore Paschali.}

\paschaAntiphona

\scriptura{Luc 2,29-32}

\includescore{temporalia/nuncdimittis-initium-i-D2-auto.tex}

\psalmusEtTranslatio{temporalia/nuncdimittis-i-d2.tex}{temporalia/nuncdimittis-boh.tex}

\vfill


\kyrieEleison


\pars{Oratio}

\dominusVobiscum

\rubrica{Extra Adventum.}

Beátæ et gloriósæ semper Vírginis Maríæ, quǽsumus Dómine,
intercéssio gloriósa nos prótegat, *
et ad vitam perdúcat ætérnam.

% +
\perDominum


\rubrica{Tempore Adventus.}

Deus, qui de beátæ Maríæ Vírginis útero Verbum tuum,
Angelo nuntiánte, carnem suscípere voluísti: †
præsta supplícibus tuis; ut qui vere eam Genitrícem Dei crédimus, *
ejus apud te intercessiónibus adjuvémur.
Per eúmdem Christum Dóminum nostrum. 
℟. Amen.

\dominusVobiscumRep

\benedicamusDomino

\pars{Benedictio}

Benedícat et custódiat nos
omnípotens et miséricors Dóminus,
Pater, et Fílius, et Spíritus Sanctus.
℟. Amen.

\pars{Antiphona finalis}

\rubrica{
  Et immediate subjungitur flexis genibus,
  diebus Dominicis autem a Vesperis Sabbati inclusive et Tempore Paschali stando,
  cum suis Versu et Oratione, una e sequentibus Antiphonis finalibus
  beatæ Mariæ Virginis, pro diversitate Temporis.
}

%%%
\rubrica{A Vesperis Sabbati ante Dominicam I. Adventus
usque ad secundas Vesperas Purificationis inclusive.}

\superInitialam{V}
\includescore{\ccommunesAR/an_alma_redemptoris_mater.tex}

\rubrica{In Adventu:}

℣. Angelus Dómini nuntiávit Maríæ.
℟. Et concépit de Spíritu Sancto.

Orémus.
Grátiam tuam, quǽsumus Dómine, méntibus nostris infúnde: †
ut qui, Angelo nuntiánte, Christi Fílii tui incarnatiónem cognóvimus, *
per passiónem eius et crucem ad resurrectiónis glóriam perducámur.
Per eúmdem Christum Dóminum nostrum. 
℟. Amen.

%%%
\rubrica{A primis Vesperis Nativitatis Domini et deinceps:}

℣. Post partum Virgo invioláta permansísti.
℟. Dei Génitrix intercéde pro nobis.

Orémus.
Deus, qui salútis ætérnæ, 
beátæ Maríæ virginitáte fœcúnda,
humáno géneri prǽmia præstitísti: †
tríbue, quǽsumus; ut ipsam pro nobis intercédere sentiámus, *
per quam merúimus auctórem vitæ suscípere,
Dóminum nostrum Jesum Christum Fílium tuum.
℟. Amen.

%%%
\rubrica{Post Purificationem, id est, a Completorio diei 2. Februarii,
etiam quando transferatur Festum Purificationis B.M.V., usque ad Nonam
Sabbati Sancti inclusive:}

\superInitialam{VI}
\includescore{\ccommunesAR/an_ave_regina_caelorum.tex}

℣. Dignáre me laudáre te Virgo sacráta.
℟. Da mihi virtútem contra hostes tuos.

Orémus.
Concéde, miséricors Deus, fragilitáti nostræ præsídium: †
ut qui sanctæ Dei Genitrícis memóriam ágimus, *
intercessiónis ejus auxílio a nostris iniquitátibus resurgámus.
Per eúmdem Christum Dóminum nostrum.
℟. Amen. 

%%%
\vspace{0.5cm}
\rubrica{A Completorio Sabbati sancti usque ad Nonam Sabbati 
  infra Octavam Pentecostes inclusive:}

\superInitialam{VI}
\includescore{\ccommunesAR/an_regina_caeli.tex}

℣. Gaude et lætáre Virgo María, allelúja.
℟. Quia surréxit Dóminus vere, allelúja.

Orémus.
Deus, qui per resurrectiónem Fílii tui Dómini nostri Jesu Christi
mundum lætificáre dignátus es: †
præsta, quǽsumus; ut per ejus Genitrícem Vírginem Maríam, *
perpétuæ capiámus gáudia vitæ.
Per eúmdem Christum Dóminum nostrum.
℟. Amen. 

%%%
\vspace{0.5cm}
\rubrica{A Vesperis Sabbati infra Octavam Pentecostes usque ad Nonam Sabbati
ante Adventum inclusive.}

\superInitialam{I}
\includescore{\ccommunesAR/an_salve_regina.tex}

℣. Ora pro nobis sancta Dei Génitrix.
℟. Ut digni efficiámur promissiónibus Christi.

Orémus.
Omnípotens sempitérne Deus,
qui gloriósæ Vírginis Matris Maríæ corpus et ánimam,
ut dignum Fílii tui habitáculum éffici mererétur,
Spíritu Sancto cooperánte præparásti: †
da, ut cujus commemoratióne lætámur, *
ejus pia intercessióne ab instántibus malis et a morte perpétua liberémur.
Per eúmdem Christum Dóminum nostrum. 
℟. Amen.

\vspace{2cm}

\pars{Benedictio}

Divínum auxílium máneat semper nobíscum.
℟. Amen.

\vfill


\pagebreak

\label{pars:matutinum}

% Extra Adventum.

\hora{Ad Matutinum.}

\domineLabiaMea

\deusInAdiutorium

\pars{Invitatorium.}

\superInitialam{VII}
\includescore{cantus/manuscripti/matinvit.tex}

\scriptura{Psalmus 95. (Textus antiquus latinus.)}

\superInitialam{VII}
\includescore{cantus/manuscripti/venite7/venite7a.tex}

\invitatoriumIntegrum

\includescore{cantus/manuscripti/venite7/venite7b.tex}

\invitatoriumAltera

\rubricatum{In sequenti Psalmi versu, ad verba 
  \textnormal{veníte, adorémus et procidámus ante Deum,} 
  genuflectitur.}

\includescore{cantus/manuscripti/venite7/venite7c.tex}

\invitatoriumIntegrum

\includescore{cantus/manuscripti/venite7/venite7d.tex}

\invitatoriumAltera

\includescore{cantus/manuscripti/venite7/venite7e.tex}

\invitatoriumIntegrum

\includescore{cantus/manuscripti/venite7/venite7f.tex}

\rubrica{Repetitur altera pars Invitatorii. 
Denique repetitur integrum Invitatorium.}

\pars{Hymnus.}

\superInitialam{II}
\includescore{temporalia/hymnus-QuemTerra.tex}

\vfill \pagebreak

Dominica, Feria II et V.

\pars{psalmus 1.}

\antiphona{IV A*}{cantus/arom12/matant1.tex}{}

\scriptura{Psalmus 8.}

\includescore{temporalia/ps8-initium-iv-A-auto.tex}

\psalmusEtTranslatio{temporalia/ps8-iv-a.tex}{temporalia/ps8-boh.tex}

\vfill \pagebreak

\pars{psalmus 2.}

\antiphona{IV A*}{cantus/manuscripti/matant2b.tex}{}

\scriptura{Psalmus 18.}

\includescore{temporalia/ps18-initium-iv-A-auto.tex}

\psalmusEtTranslatio{temporalia/ps18-iv-a.tex}{empty.tex}

\vfill \pagebreak

\pars{psalmus 3.}

\antiphona{IV A*}{cantus/manuscripti/matant3.tex}{}

\scriptura{Psalmus 23.}

\includescore{temporalia/ps23-initium-iv-A-auto.tex}

\psalmusEtTranslatio{temporalia/ps23-iv-a.tex}{empty.tex}

\vfill \pagebreak



Feria III et VI.

\pars{psalmus 1.}

\antiphona{VII c}{cantus/manuscripti/matant4.tex}{}

\scriptura{Psalmus 44.}

\includescore{temporalia/ps44-initium-vii-c-auto.tex}

\psalmusEtTranslatio{temporalia/ps44-vii-c.tex}{empty.tex}

\vfill \pagebreak

\pars{psalmus 2.}

\antiphona{VII c}{cantus/manuscripti/matant5.tex}{}

\scriptura{Psalmus 45.}

\includescore{temporalia/ps45-initium-vii-c-auto.tex}

\psalmusEtTranslatio{temporalia/ps45-vii-c.tex}{empty.tex}

\vfill \pagebreak

\pars{psalmus 3.}

\antiphona{VII c}{cantus/manuscripti/matant6.tex}{}

\scriptura{Psalmus 86.}

\includescore{temporalia/ps86-initium-vii-c-auto.tex}

\psalmusEtTranslatio{temporalia/ps86-vii-c.tex}{empty.tex}

\vfill \pagebreak



Feria IV et Sabbato.

\pars{psalmus 1.}

\antiphona{IV A*}{cantus/manuscripti/matant7.tex}{}

\scriptura{Psalmus 95.}

\includescore{temporalia/ps95-initium-iv-A-auto.tex}

\psalmusEtTranslatio{temporalia/ps95-iv-a.tex}{empty.tex}

\vfill \pagebreak

\pars{psalmus 2.}

\antiphona{IV A*}{cantus/manuscripti/matant8.tex}{}

\scriptura{Psalmus 96.}

\includescore{temporalia/ps96-initium-iv-A-auto.tex}

\psalmusEtTranslatio{temporalia/ps96-iv-a.tex}{empty.tex}

\vfill \pagebreak

\pars{psalmus 3.}

\rubrica{Extra Adventum:}

\antiphona{IV A*}{cantus/manuscripti/matant9.tex}{}

\scriptura{Psalmus 97.}

\includescore{temporalia/ps97-initium-iv-A-auto.tex}

\psalmusEtTranslatio{temporalia/ps97-iv-a.tex}{empty.tex}

\vfill \pagebreak

\rubrica{Tempore Adventus:}

\antiphona{I f}{cantus/nrom02/adv_matant9.tex}{}

\scriptura{Psalmus 97.}

\includescore{temporalia/ps97-initium-i-f-auto.tex}

\psalmusEtTranslatio{temporalia/ps97-i-f.tex}{empty.tex}

\vfill \pagebreak


\versiculusDiffusaEst


\rubrica{\textnormal{Pater noster} secreto usque ad}
℣. Et ne nos indúcas in tentatiónem. ℟. Sed líbera nos a malo.

\pars{Absolutio}

\includescore{cantus/arom12/absolutio-PrecibusEtMeritis.tex}

\pars{Benedictio}

\rubrica{Lector petit benedictionem. 
(Hoc modo dicitur \textnormal{Jube Domne} etiam pro lectione 2. et 3.)
Hebdomadarius benedicit.
Chorus respondet \textnormal{Amen}.}

\includescore{cantus/arom12/benedictio-NosCumProle.tex}

\pars{Lectio i}

\scriptura{Sir 24,11-13}

In ómnibus réquiem quæsívi, et in hereditáte Dómini morábor.
Tunc præcépit et dixit mihi Creátor ómnium, 
et qui creávit me, requiévit in tabernáculo meo,
et dixit mihi: 
In Jacob inhábita, et in Israel hereditáre,
et in eléctis meis mitte radíces.

\tuAutem

\superInitialam{II}
\includescore{cantus/nrom02/matresp1.tex}

\vfill \pagebreak

\pars{Benedictio}

\rubrica{Lector dicit \textnormal{Jube Domne benedicere} ut supra.}

\includescore{cantus/arom12/benedictio-IpsaVirgo.tex}

\pars{Lectio ii}

\scriptura{Sir 24,15-16}

Et sic in Sion firmáta sum, et in civitáte sanctificáta simíliter requiévi,
et in Jerúsalem potéstas mea.
Et radicávi in pópulo honorificáto, et in parte Dei mei heréditas illíus,
et in plenitúdine sanctórum deténtio mea.

\tuAutem

\rubrica{Quando dicitur \textnormal{Te Deum,} id est in Festis B. Mariæ
Virg. et sancti Joseph et toto tempore ante Septuagesimam ac post
Sabbatum sanctum, sequens responsorium dicitur cum \textnormal{Glória Patri.}
Cum autem \textnormal{Te Deum} non dicitur, \textnormal{Glória Patri}
et ultima repetitio \textnormal{Genuísti} omittitur.}

\superInitialam{I}
\includescore{cantus/nrom02/matresp2.tex}

\vfill \pagebreak

\pars{Benedictio}

\rubrica{Lector dicit \textnormal{Jube Domne benedicere} ut supra.}

\includescore{cantus/arom12/benedictio-PerVirginem.tex}

\pars{Lectio iii}

\scriptura{Sir 24,17-20}

Quasi cedrus exaltáta sum in Líbano, et quasi cypréssus in monte Sion:
quasi palma exaltáta sum in Cades, et quasi plantátio rosæ in Jéricho:
quasi olíva speciósa in campis, et quasi plátanus exaltáta sum juxta aquam
in platéis.
Sicut cinnamónum et bálsamum aromatízans odórem dedi; 
quasi myrrha elécta dedi suavitátem odóris.

\tuAutem

\rubrica{A Dominica Septuagesimæ usque ad Sabbatum sanctum inclusive,
extra Festa B. Mariæ Virg. et sancti Joseph, Hymnus Ambrosianus
omittitur ejusque loco dicitur responsorium:}

\superInitialam{I}
\includescore{cantus/nrom02/matresp3.tex}

\rubrica{In Festis B. Mariæ Virg. et sancti Joseph et toto tempore
ante Septuagesimam ac post Sabbatum sanctum:}

\pars{Hymnus Ambrosianus}

\superInitialam{III}
\includescore{../../cantuscommunes/arom12/tedeum-simplex.tex}

\rubrica{Et incipiunt Laudes, dicto \textnormal{Deus in adjutórium}.
Si autem Laudes a Matutino separantur, dicitur hic Oratio et finis
horæ ut in Vesperis, pg. \pageref{vesperaefinis}.
(\textnormal{Kyrie eléison} non dicitur.)}


\pagebreak

% Extra Adventum.

\hora{Ad Laudes.} %%%%%%%%%%%%%%%%%%%%%%%%%%%%%%%%%%%%%%%%

\rubricaLaudesPrincipium

\vspace{1cm}

\pars{Psalmus 1.}

\antiphonaLaudI

\scriptura{Psalmus 92.}

\includescore{temporalia/extra-adventum-laudes/ps92-initium-vii-a-auto.tex}

\psalmusEtTranslatioB{temporalia/extra-adventum-laudes/ps92-vii-a.tex}{temporalia/extra-adventum-laudes/ps92-boh.tex}{11cm}

\vfill
\pagebreak

\pars{Psalmus 2.}

\antiphonaLaudII

\scriptura{Psalmus 99.}

\includescore{temporalia/extra-adventum-laudes/ps99-initium-viii-G-auto.tex}

\psalmusEtTranslatioB{temporalia/extra-adventum-laudes/ps99-viii-g.tex}{temporalia/extra-adventum-laudes/ps99-boh.tex}{7cm}

\vfill

\pagebreak

\pars{Psalmus 3.}

\antiphonaLaudIII

\scriptura{Psalmus 62.}

\includescore{temporalia/extra-adventum-laudes/ps62-initium-iv-A-auto.tex}

\psalmusEtTranslatio{temporalia/extra-adventum-laudes/ps62-iv-a.tex}{temporalia/extra-adventum-laudes/ps62-boh.tex}

\vfill

\pagebreak

\pars{Canticum trium Puerorum}

\antiphonaLaudIV

\scriptura{Dan. 3, 57-88.56}

\includescore{temporalia/extra-adventum-laudes/dan3-initium-vii-c2-auto.tex}

\psalmusEtTranslatioB{temporalia/extra-adventum-laudes/dan3-vii-c2.tex}{temporalia/extra-adventum-laudes/dan3-boh.tex}{9cm}

\vfill

\pagebreak

\pars{Psalmus 4.}

\antiphonaLaudV

\scriptura{Psalmus 148.}

\includescore{temporalia/extra-adventum-laudes/ps148-initium-i-g2-auto.tex}

\psalmusEtTranslatio{temporalia/extra-adventum-laudes/ps148-i-g2.tex}{temporalia/extra-adventum-laudes/ps148-boh.tex}

\vfill

\pagebreak

\pars{Capitulum}

\label{laudescapitulum}

\scriptura{Cant. 6, 8}

\includescore{cantus/arom12/capitulum-VideruntEam.tex}

\trCapitVideruntEam

\vfill

\pars{Hymnus}

\superInitialam{II}
\includescore{temporalia/hymnus-OGloriosaVirginum.tex}

\trHymOGloriosa

\vfill

\includescore{cantus/arom12/versiculus-BenedictaTu.tex}

\trVersBenedictaTu

\pagebreak

\pars{Canticum Zachariæ}

\rubrica{Extra Tempus Paschale:}

\antiphona{VIII G}{cantus/arom12/antlaudben}{\trLaudAntBenedictus}

\scripturaBenedictus

\includescore{temporalia/extra-adventum-laudes/benedictus-initium-viiisoll-G-auto.tex}

\canticumBenedictusEtTranslatio{temporalia/extra-adventum-laudes/benedictus-viiisoll-g.tex}

\pagebreak

\rubrica{Tempore Paschali:}

\paschaAntiphona

\scripturaBenedictus

\includescore{temporalia/tempore-paschali/benedictus-initium-isoll-D2-auto.tex}

\canticumBenedictusEtTranslatio{temporalia/tempore-paschali/benedictus-isoll-d2.tex}

\vfill

\pagebreak

\label{laudesfinis}
\inFineHorarumExtraAdventumLaudes



\pagebreak

% Extra Adventum.

\hora{Ad Primam.} %%%%%%%%%%%%%%%%%%%%%%%%%%%%%%%%%%%%%%%%

\label{prima}

\vspace{1cm}

\deusInAdiutorium

\vfill

\pagebreak

\pars{Hymnus}

\includescore{temporalia/hymnus-MementoRerumConditor.tex}

\pars{Psalmi}

\antiphonaLaudI

\scriptura{Psalmus 53.}

\includescore{temporalia/ps53-initium-vii-a-auto.tex}

\psalmusEtTranslatio{temporalia/ps53-vii-a.tex}{temporalia/ps53-boh.tex}

\scriptura{Psalmus 84.}

\psalmusEtTranslatio{temporalia/ps84-vii-a.tex}{temporalia/ps84-boh.tex}

\scriptura{Psalmus 116.}

\psalmusEtTranslatio{temporalia/ps116-vii-a.tex}{temporalia/ps116-boh.tex}

\pars{Capitulum}

\label{primacapitulum}

\scriptura{Cant. 6, 9}

\includescore{cantus/arom12/capitulum-QuaeEst}

\noindent ℣. Dignáre me laudáre te, Virgo sa\textbf{crá}ta.\\
℟. Da mihi virtútem contra hostes \textbf{tu}os.

\finisHoraeMinores

\pars{Oratio}

Deus, qui virginálem aulam beátæ Maríæ, in qua habitáres,
elígere dignátus es: †
da, quǽsumus; ut sua nos defensióne munítos, *
jucúndos fácias suæ interésse commemoratióni.

Qui vivis et regnas cum Deo Patre in unitáte Spíritus Sancti Deus,
per ómnia sǽcula sæculórum.
℟. Amen.

\vfill


\pagebreak

% Extra Adventum.

\hora{Ad Tertiam.} %%%%%%%%%%%%%%%%%%%%%%%%%%%%%%%%%%%%%%%%

\initiumHoraeMinores

\pars{Psalmi}

\antiphonaLaudII

\scriptura{Psalmus 119.}

\includescore{temporalia/ps119-initium-viii-G-auto.tex}

\psalmusEtTranslatio{temporalia/ps119-viii-g.tex}{empty.tex}

\scriptura{Psalmus 120.}

\includescore{temporalia/ps120-initium-viii-G-auto.tex}

\psalmusEtTranslatio{temporalia/ps120-viii-g.tex}{empty.tex}

\scriptura{Psalmus 121.}

\includescore{temporalia/ps121-initium-viii-G-auto.tex}

\psalmusEtTranslatio{temporalia/ps121-viii-g.tex}{empty.tex}

\pars{Capitulum}

\scriptura{Eccli. 24, 15}

\includescore{cantus/arom12/capitulum-EtSicInSion}

\noindent ℣. Diffúsa est grátia in lábiis \textbf{tu}is.\\
℟. Proptérea benedíxit te Deus in æ\textbf{tér}num.

\finisHoraeMinores

\pars{Oratio}

Deus, qui salútis ætérnæ, beátæ Maríæ virginitáte fœcúnda,
humáno géneri prǽmia præstitísti: †
tríbue, quǽsumus; ut ipsam pro nobis intercédere sentiámus, *
per quam merúimus auctórem vitæ suscípere, 
Dóminum nostrum Jesum Christum Fílium tuum.

Qui tecum vivit et regnat in unitáte Spíritus Sancti Deus,
per ómnia sǽcula sæculórum.


\pagebreak

% Extra Adventum.

\hora{Ad Sextam.} %%%%%%%%%%%%%%%%%%%%%%%%%%%%%%%%%%%%%%%%

\initiumHoraeMinores

\pars{Psalmi}

\antiphonaLaudIII

\scriptura{Psalmus 122.}

\includescore{temporalia/extra-adventum-sexta/ps122-initium-iv-A-auto.tex}

\psalmusEtTranslatio{temporalia/extra-adventum-sexta/ps122-iv-a.tex}{temporalia/extra-adventum-sexta/ps122-boh.tex}


\scriptura{Psalmus 123.}

\psalmusEtTranslatio{temporalia/extra-adventum-sexta/ps123-iv-a.tex}{temporalia/extra-adventum-sexta/ps123-boh.tex}


\scriptura{Psalmus 124.}

\psalmusEtTranslatio{temporalia/extra-adventum-sexta/ps124-iv-a.tex}{temporalia/extra-adventum-sexta/ps124-boh.tex}

% stejna antifona, ale bez udani modu etc - tim se signalizuje konec
\antiphona{}{cantus/arom12/antlaud3.tex}{\translatioLaudAntIII}}


\pars{Capitulum}

\label{sextacapitulum}

\scriptura{Eccli. 24, 16}

\includescore{cantus/arom12/capitulum-EtRadicavi.tex}

\noindent ℣. Benedícta tu in muli\textbf{é}ribus.\\
℟. Et benedíctus fructus ventris \textbf{tu}i.

\finisHoraeMinores

\pars{Oratio}

Concéde, miséricors Deus, fragilitáti nostræ præsídium:
ut, qui sanctæ Dei Genetrícis memóriam ágimus;
intercessiónis ejus auxílio, a nostris iniquitátibus resurgámus.

Per eúndem Dóminum nostrum Jesum Christum, Fílium tuum:
Qui tecum vivit et regnat in unitáte Spíritus Sancti Deus,
per ómnia sǽcula sæculórum.

\vfill


\pagebreak

% Extra Adventum.

\hora{Ad Nonam.} %%%%%%%%%%%%%%%%%%%%%%%%%%%%%%%%%%%%%%%%

\vspace{1cm}

\deusInAdiutorium

\vfill

\pagebreak

\pars{Hymnus}

\pars{Psalmi}

\antiphonaLaudV

\scriptura{Psalmus 125.}

\includescore{temporalia/ps125-initium-i-g2-auto.tex}

\psalmusEtTranslatio{temporalia/ps125-i-g2.tex}{empty.tex}


\scriptura{Psalmus 126.}

\includescore{temporalia/ps126-initium-i-g2-auto.tex}

\psalmusEtTranslatio{temporalia/ps126-i-g2.tex}{empty.tex}


\scriptura{Psalmus 127.}

\includescore{temporalia/ps127-initium-i-g2-auto.tex}

\psalmusEtTranslatio{temporalia/ps127-i-g2.tex}{empty.tex}


\pagebreak

%%%%%%%%%%%%%%%%%%%%%%%%%%%%%%%%%%%%%%%%%%%%%%%%%%%%%%%%%%%%%%%%%%%%%%

\divisio{In Adventu.}

\rubrica{Quod dicitur a Vesperis Sabbati ante Dominicam I Adventus 
usque ad Nonam Vigiliae Nativitatis Domini inclusive.}

% In Adventu.

\hora{Ad Vesperas.} %%%%%%%%%%%%%%%%%%%%%%%%%%%%%%%%%%%%%%%%

\vspace{1cm}
\includescore{\ccommunesAR/deusinadiutorium-ferialis.tex}
\vspace{1cm}

\vfill

\pagebreak

\pars{psalmus 1.}

\adventAntiphonaI

\scriptura{Psalmus 109.}

\includescore{temporalia/ps109-initium-viii-G-auto.tex}

\psalmusEtTranslatio{temporalia/ps109-viii-g.tex}{temporalia/ps109-boh.tex}

\vfill

\pagebreak

\pars{psalmus 2.}

\rubricaAnnuntiatioQuadragesimaAlleluja

\adventAntiphonaII

\scriptura{Psalmus 112.}

\includescore{temporalia/ps112-initium-i-g-auto.tex}

\psalmusEtTranslatio{temporalia/ps112-i-g.tex}{temporalia/ps112-boh.tex}

\vfill

\pagebreak

\pars{psalmus 3.}

\rubricaAnnuntiatioQuadragesimaAlleluja

\adventAntiphonaIII

\scriptura{Psalmus 121.}

\includescore{temporalia/ps121-initium-viii-G-auto.tex}

\psalmusEtTranslatio{temporalia/ps121-viii-g.tex}{temporalia/ps121-boh.tex}

\vfill

\pagebreak

\pars{psalmus 4.}

\adventAntiphonaIV

\scriptura{Psalmus 126.}

\includescore{temporalia/ps126-initium-i-f-auto.tex}

\psalmusEtTranslatio{temporalia/ps126-i-f.tex}{temporalia/ps126-boh.tex}

\vfill

\pagebreak

\pars{psalmus 5.}

\adventAntiphonaV

\scriptura{Psalmus 147.}

\includescore{temporalia/ps147-initium-viii-c-auto.tex}

\psalmusEtTranslatio{temporalia/ps147-viii-c.tex}{temporalia/ps147-boh.tex}

\vfill

\pagebreak

\pars{Capitulum}

\scriptura{Is. 11,1-2}

\includescore{cantus/arom12/capitulum-EgredieturVirga.tex}

\pars{Hymnus}

\rubrica{Prima stropha sequentis Hymni dicitur flexis genibus.}

\superInitialam{VII}
\includescore{temporalia/hymnus-AveMarisStella.tex}

\includescore{cantus/arom12/versiculus-DiffusaEst.tex}

\vfill

\pagebreak

\pars{Canticum B. Mariæ Virginis}

\rubricaAnnuntiatioQuadragesimaAlleluja

\antiphona{VIII G}{cantus/arom12/adv_antmag.tex}{}

\scripturaMagnificat

\includescore{../../tonipsalmorum/arom12/magnificat-initium-viii-G.tex}

\psalmusEtTranslatioB{temporalia/magnificat-viii-g.tex}{temporalia/magnificat-boh.tex}{10cm}

\vfill

\pagebreak

\label{vesperaefinisadvent}

\inFineHorarumInAdventu

\vfill


\pagebreak

\hora{Completorium}

\rubrica{Completorium fit ut Extra Adventum, pg. \pageref{pars:completorium}, 
tantum Capitulum
cum Versiculo suo, Antiphona pro Nunc dimittis et Oratio sunt propria.}

\hora{Ad Matutinum.}

\rubrica{Matutinum fit ut Extra Adventum, pg. \pageref{pars:matutinum}.
Feria IV et Sabbato dicitur tertius psalmus cum antiphona speciali, 
\textnormal{Angelus Dómini}. Lectiones et Responsoria sunt propria,
Absolutiones atque Benedictiones ut Extra Adventum.}

\pars{Lectio i}

\scriptura{Luc. 1, 26-28}

Missus est Angelus Gábriel a Deo in civitátem Galilǽæ,
cui nomen Názareth, ad Vírginem desponsátam viro, cui nomen erat Joseph,
de domo David, et nomen Vírginis María.
Et, ingréssus Angelus ad eam, dixit: Ave, grátia plena;
Dóminus tecum: benedícta tu in muliéribus.

\tuAutem

\pars{Lectio ii}

\scriptura{Luc 1, 29-33}

Quæ cum audísset, turbáta est in sermóne ejus, et cogitábat
quális esset ista salutátio. Et ait Angelus ei:
Ne tímeas, María; invenísti grátiam apud Deum: ecce, concípies
in útero et páries fílium, et vocábis nomen ejus Jesum.
Hic erit magnus, et Fílius Altíssimi vocábitur;
et dabit illi Dóminus Deus sedem David, patris ejus;
et regnábit in domo Jacob in ætérnum, et regni ejus non erit finis.

\tuAutem

\pars{Lectio iii}

\scriptura{Luc 1, 34-38}

Dixit autem María ad Angelum: Quómodo fiet istud, quóniam virum non
cognósco? Et respóndens Angelus dixit ei: Spíritus Sanctus supervéniet
in te, et virtus Altíssimi obumbrábit tibi. Ideóque et quod nascétur
ex te Sanctum, vocábitur Fílius Dei. Et ecce, Elísabeth, cognáta tua,
et ipsa concépit fílium en senectúte sua, et hic mensis sextus est illi,
quæ vocátur stérilis; quia non erit impossíbile apud Deum omne verbum.
Dixit autem María: Ecce ancílla Dómini: fiat mihi secúndum verbum tuum.

\tuAutem


\pagebreak

% In Adventu.

\hora{Ad Laudes.} %%%%%%%%%%%%%%%%%%%%%%%%%%%%%%%%%%%%%%%%

\vspace{1cm}
\includescore{\ccommunesAR/deusinadiutorium-ferialis.tex}
\vspace{1cm}

\vfill

\pagebreak
\pars{psalmus 1.}

\adventAntiphonaI

\scriptura{Psalmus 92.}

\includescore{temporalia/ps92-initium-viii-G-auto.tex}

\psalmusEtTranslatio{temporalia/ps92-viii-g.tex}{temporalia/ps92-boh.tex}

\vfill

\pagebreak

\pars{psalmus 2.}

\rubricaAnnuntiatioQuadragesimaAlleluja

\adventAntiphonaII

\scriptura{Psalmus 99.}

\includescore{temporalia/ps99-initium-i-g-auto.tex}

\psalmusEtTranslatio{temporalia/ps99-i-g.tex}{temporalia/ps99-boh.tex}

\vfill

\pagebreak

\pars{psalmus 3.}

\rubricaAnnuntiatioQuadragesimaAlleluja

\adventAntiphonaIII

\scriptura{Psalmus 62.}

\includescore{temporalia/ps62-initium-viii-G-auto.tex}

\psalmusEtTranslatio{temporalia/ps62-viii-g.tex}{temporalia/ps62-boh.tex}

\vfill

\pagebreak

\pars{Canticum trium puerorum.}

\adventAntiphonaIV

\scriptura{Dan. 57-88 et 56.}

\includescore{temporalia/dan3-initium-i-f-auto.tex}

\psalmusEtTranslatio{temporalia/dan3-i-f.tex}{temporalia/dan3-boh.tex}

\vfill

\pagebreak

\pars{psalmus 4.}

\adventAntiphonaV

\scriptura{Psalmus 148.}

\includescore{temporalia/ps148-initium-viii-c-auto.tex}

\psalmusEtTranslatio{temporalia/ps148-viii-c.tex}{temporalia/ps148-boh.tex}

\vfill

\pagebreak

\pars{Capitulum}

\scriptura{Is. 11,1-2}

\includescore{cantus/arom12/capitulum-EgredieturVirga.tex}

\pars{Hymnus}

\superInitialam{II}
\includescore{temporalia/hymnus-OGloriosaVirginum}

\includescore{cantus/arom12/versiculus-BenedictaTu}

\pagebreak

\pars{Canticum Zachariæ}

\adventAntiphonaSpiritusSanctus

\scripturaBenedictus

\includescore{temporalia/benedictus-initium-viii-G-auto.tex}

\psalmusEtTranslatioB{temporalia/benedictus-viii-g.tex}{temporalia/benedictus-boh.tex}{10cm}

\rubricaFinisHoraeAdvent

% Commemorations would follow, but these were cancelled 
% by the reforms of 1955. 


\pagebreak

% Advent.

\hora{Ad Primam.} %%%%%%%%%%%%%%%%%%%%%%%%%%%%%%%%%%%%%%%%

\initiumHoraeMinoresAdvent

\pars{Psalmi}

\adventAntiphonaI

\scriptura{Psalmus 53.}

\includescore{temporalia/ps53-initium-viii-G-auto.tex}

\psalmusEtTranslatio{temporalia/ps53-viii-g.tex}{empty.tex}

\scriptura{Psalmus 84.}

\psalmusEtTranslatio{temporalia/ps84-viii-g.tex}{empty.tex}

\scriptura{Psalmus 116.}

\psalmusEtTranslatio{temporalia/ps116-viii-g.tex}{empty.tex}

\pars{Capitulum}

\scriptura{Is 7, 14-15}

\includescore{cantus/arom12/capitulum-EcceVirgo}

\noindent ℣. Dignáre me laudáre te, Virgo sa\textbf{crá}ta.\\
℟. Da mihi virtútem contra hostes \textbf{tu}os.

\rubricaFinisHoraeAdvent


\pagebreak

% Advent.

\hora{Ad Tertiam.} %%%%%%%%%%%%%%%%%%%%%%%%%%%%%%%%%%%%%%%%

\initiumHoraeMinoresAdvent

\pars{Psalmi}

\adventAntiphonaII

\scriptura{Psalmus 119.}

\includescore{temporalia/advent-tertia/ps119-initium-i-g-auto.tex}

\psalmusEtTranslatioB{temporalia/advent-tertia/ps119-i-g.tex}{temporalia/advent-tertia/ps119-boh.tex}{10cm}

\scriptura{Psalmus 120.}

\psalmusEtTranslatioB{temporalia/advent-tertia/ps120-i-g.tex}{temporalia/advent-tertia/ps120-boh.tex}{10cm}

\scriptura{Psalmus 121.}

\psalmusEtTranslatioB{temporalia/advent-tertia/ps121-i-g.tex}{temporalia/advent-tertia/ps121-boh.tex}{10cm}

\vfill

\pars{Capitulum}

\scriptura{Is. 11, 1-2}

\includescore{cantus/arom12/capitulum-EgredieturVirga}

\vfill

\noindent ℣. Diffúsa est grátia in lábiis \textbf{tu}is.\\
℟. Proptérea benedíxit te Deus in æ\textbf{tér}num.

\vfill

\rubricaFinisHoraeAdvent



\pagebreak

% Advent.

\hora{Ad Sextam.} %%%%%%%%%%%%%%%%%%%%%%%%%%%%%%%%%%%%%%%%

\initiumHoraeMinoresAdvent

\pars{Psalmi}

\rubricaAnnuntiatioQuadragesimaAlleluja

\adventAntiphonaIII

\scriptura{Psalmus 122.}

\includescore{temporalia/ps122-initium-viii-G-auto.tex}

\psalmusEtTranslatio{temporalia/ps122-viii-g.tex}{temporalia/ps122-boh.tex}


\scriptura{Psalmus 123.}

\psalmusEtTranslatio{temporalia/ps123-viii-g.tex}{temporalia/ps123-boh.tex}


\scriptura{Psalmus 124.}

\psalmusEtTranslatio{temporalia/ps124-viii-g.tex}{temporalia/ps124-boh.tex}



\pars{Capitulum}

\scriptura{Luc. 1, 32-33}
% O_O is it normal to read a Gospel as a Capitulum? 

\includescore{cantus/arom12/capitulum-DabitIlli}

\noindent ℣. Benedícta tu in muli\textbf{é}ribus.\\
℟. Et benedíctus fructus ventris \textbf{tu}i.

\rubricaFinisHoraeAdvent


\pagebreak

% Advent.

\hora{Ad Nonam.} %%%%%%%%%%%%%%%%%%%%%%%%%%%%%%%%%%%%%%%%

\initiumHoraeMinoresAdvent

\pars{Psalmi}

\adventAntiphonaV

\scriptura{Psalmus 125.}

\includescore{temporalia/advent-nona/ps125-initium-viii-c-auto.tex}

\psalmusEtTranslatioB{temporalia/advent-nona/ps125-viii-c.tex}{temporalia/advent-nona/ps125-boh.tex}{10cm}


\scriptura{Psalmus 126.}

\psalmusEtTranslatioB{temporalia/advent-nona/ps126-viii-c.tex}{temporalia/advent-nona/ps126-boh.tex}{10cm}


\scriptura{Psalmus 127.}

\psalmusEtTranslatioB{temporalia/advent-nona/ps127-viii-c.tex}{temporalia/advent-nona/ps127-boh.tex}{10cm}

\vfill

\pars{Capitulum}

\scriptura{Is 7, 14-15}

\includescore{cantus/arom12/capitulum-EcceVirgo}

\vfill

\noindent ℣. Angelus Dómini nuntiávit Ma\textbf{rí}æ.\\
℟. Et concépit de Spíritu \textbf{Sanc}to.

\vfill

\rubricaFinisHoraeAdvent

\vfill


\pagebreak

%%%%%%%%%%%%%%%%%%%%%%%%%%%%%%%%%%%%%%%%%%%%%%%%%%%%%%%%%%%%%%%%%%%%%%

\divisio{Post Nativitatem.}

\rubrica{Quod dicitur a Vesperis diei 24 Decembris usque ad Completorium
diei 2 Februarii inclusive.}

% Post Nativitatem.

\hora{Ad Vesperas.} %%%%%%%%%%%%%%%%%%%%%%%%%%%%%%%%%%%%%%%%

\deusInAdiutorium
\vspace{1cm}

\pars{Psalmus 1.}

\nativitasAntiphonaI

\scriptura{Psalmus 109.}

\includescore{temporalia/post-nativitatem-vespers/ps109-initium-vi-F-auto.tex}

\psalmusEtTranslatio{temporalia/post-nativitatem-vespers/ps109-vi-f.tex}{temporalia/post-nativitatem-vespers/ps109-boh.tex}

\vfill

\pagebreak

\pars{Psalmus 2.}

\nativitasAntiphonaII

\scriptura{Psalmus 112.}

\includescore{temporalia/post-nativitatem-vespers/ps112-initium-iii-a3-auto.tex}

\psalmusEtTranslatio{temporalia/post-nativitatem-vespers/ps112-iii-a3.tex}{temporalia/post-nativitatem-vespers/ps112-boh.tex}

\vfill

\pagebreak

\pars{Psalmus 3.}

\nativitasAntiphonaIII

\scriptura{Psalmus 121.}

\includescore{temporalia/post-nativitatem-vespers/ps121-initium-iv-E-auto.tex}

\psalmusEtTranslatio{temporalia/post-nativitatem-vespers/ps121-iv-e.tex}{temporalia/post-nativitatem-vespers/ps121-boh.tex}

\vfill

\pagebreak

\pars{Psalmus 4.}

\nativitasAntiphonaIV

\scriptura{Psalmus 126.}

\includescore{temporalia/post-nativitatem-vespers/ps126-initium-i-f-auto.tex}

\psalmusEtTranslatio{temporalia/post-nativitatem-vespers/ps126-i-f.tex}{temporalia/post-nativitatem-vespers/ps126-boh.tex}

\vfill

\pagebreak

\pars{Psalmus 5.}

\nativitasAntiphonaV

\scriptura{Psalmus 147.}

\includescore{temporalia/post-nativitatem-vespers/ps147-initium-ii-D-auto.tex}

\psalmusEtTranslatio{temporalia/post-nativitatem-vespers/ps147-ii-d.tex}{temporalia/post-nativitatem-vespers/ps147-boh.tex}

\vfill

\pagebreak

\pars{Capitulum et Hymnus}

\rubrica{Capitulum, Hymnus et Versiculus post Hymnum
  ut extra Adventum, pg. \pageref{vesperaecapitulum}.}


\pars{Canticum B. Mariæ Virginis}

\nativitasAntiphonaMagnificat

\scripturaMagnificat

\includescore{../../tonipsalmorum/arom12/magnificat-initium-ii-A.tex}

\psalmusEtTranslatioB{temporalia/post-nativitatem-vespers/magnificat-iisoll-a.tex}{temporalia/post-nativitatem-vespers/magnificat-boh.tex}{10cm}

\vfill

\pagebreak

\label{vesperaefinisnativitas}

\inFineHorarumPostNativitatem

\vfill


\pagebreak

\hora{Ad Matutinum.}

\rubrica{Totum ut Extra Adventum, pg. \pageref{matutinum}.}

% Post Nativitatem.

\hora{Ad Laudes.} %%%%%%%%%%%%%%%%%%%%%%%%%%%%%%%%%%%%%%%%

\vspace{1cm}

\deusInAdiutorium

\vfill

\pagebreak

\pars{psalmus 1.}

\nativitasAntiphonaI

\scriptura{Psalmus 92.}

\includescore{temporalia/ps92-initium-vi-F-auto.tex}

\psalmusEtTranslatio{temporalia/ps92-vi-f.tex}{temporalia/ps92-boh.tex}

\vfill

\pagebreak

\pars{psalmus 2.}

\nativitasAntiphonaII

\scriptura{Psalmus 99.}

\includescore{temporalia/ps99-initium-iii-a3-auto.tex}

\psalmusEtTranslatio{temporalia/ps99-iii-a3.tex}{temporalia/ps99-boh.tex}

\vfill

\pagebreak

\pars{psalmus 3.}

\nativitasAntiphonaIII

\scriptura{Psalmus 62.}

\includescore{temporalia/ps62-initium-iv-E-auto.tex}

\psalmusEtTranslatio{temporalia/ps62-iv-e.tex}{temporalia/ps62-boh.tex}

\vfill

\pagebreak

\pars{Canticum trium Puerorum}

\nativitasAntiphonaIV

\scriptura{Dan. 3, 57-88.56}

\includescore{temporalia/dan3-initium-i-f-auto.tex}

\psalmusEtTranslatio{temporalia/dan3-i-f.tex}{temporalia/dan3-boh.tex}

\vfill

\pagebreak

\pars{psalmus 4.}

\nativitasAntiphonaV

\scriptura{Psalmus 148.}

\includescore{temporalia/ps148-initium-ii-D-auto.tex}

\psalmusEtTranslatio{temporalia/ps148-ii-d.tex}{temporalia/ps148-boh.tex}

\vfill


\pars{Capitulum et Hymnus}

\rubrica{Capitulum, Hymnus et Versiculus post Hymnum
  ut extra Adventum, pg. \pageref{laudescapitulum}.}

\pagebreak

\pars{Canticum Zachariæ}

\nativitasAntiphonaBenedictus

\scripturaBenedictus

\includescore{temporalia/benedictus-initium-viii-G-auto}

\psalmusEtTranslatioB{temporalia/benedictus-viii-g.tex}{temporalia/benedictus-boh.tex}{10cm}

\vfill

\rubrica{Sequitur \textnormal{Kýrie eléison} etc. usque ad finem horæ
ut in Vesperis, p. \pageref{nativitasvesperaefinis}.}


\pagebreak

% Advent.

\hora{Ad Primam.} %%%%%%%%%%%%%%%%%%%%%%%%%%%%%%%%%%%%%%%%

\initiumHoraeMinoresAdvent % this is ok, really

\pars{Psalmi}

\nativitasAntiphonaI

\scriptura{Psalmus 53.}

\includescore{temporalia/post-nativitatem-prima/ps53-initium-vi-F-auto.tex}

\psalmusEtTranslatioB{temporalia/post-nativitatem-prima/ps53-vi-f.tex}{temporalia/post-nativitatem-prima/ps53-boh.tex}{10cm}

\scriptura{Psalmus 84.}

\psalmusEtTranslatioB{temporalia/post-nativitatem-prima/ps84-vi-f.tex}{temporalia/post-nativitatem-prima/ps84-boh.tex}{10cm}

\scriptura{Psalmus 116.}

\psalmusEtTranslatioB{temporalia/post-nativitatem-prima/ps116-vi-f.tex}{temporalia/post-nativitatem-prima/ps116-boh.tex}{10cm}

\vfill

\pars{Capitulum}

\rubrica{Capitulum et Versiculus ut Extra Adventum, 
  pg. \pageref{primacapitulum}.}

\rubricaFinisHoraePostNativitatem

\vfill


\pagebreak

% Post Nativitatem.

\hora{Ad Tertiam.} %%%%%%%%%%%%%%%%%%%%%%%%%%%%%%%%%%%%%%%%

\initiumHoraeMinoresAdvent % this IS ok

\pars{Psalmi}

\nativitasAntiphonaII

\scriptura{Psalmus 119.}

\includescore{temporalia/ps119-initium-i-g-auto.tex}

\psalmusEtTranslatio{temporalia/ps119-i-g.tex}{empty.tex}

\scriptura{Psalmus 120.}

\psalmusEtTranslatio{temporalia/ps120-i-g.tex}{empty.tex}

\scriptura{Psalmus 121.}

\psalmusEtTranslatio{temporalia/ps121-i-g.tex}{empty.tex}

\pars{Capitulum}

\rubrica{Capitulum et Versiculus ut Extra Adventum, 
  pg. \pageref{tertiacapitulum}.}

\rubricaFinisHoraePostNativitatem

\vfill


\pagebreak

% Post Nativitatem.

\hora{Ad Sextam.} %%%%%%%%%%%%%%%%%%%%%%%%%%%%%%%%%%%%%%%%

\initiumHoraeMinoresAdvent % this IS ok

\pars{Psalmi}

\nativitasAntiphonaIII

\scriptura{Psalmus 122.}

\includescore{temporalia/post-nativitatem-sexta/ps122-initium-iv-E-auto.tex}

\psalmusEtTranslatio{temporalia/post-nativitatem-sexta/ps122-iv-e.tex}{temporalia/post-nativitatem-sexta/ps122-boh.tex}


\scriptura{Psalmus 123.}

\psalmusEtTranslatio{temporalia/post-nativitatem-sexta/ps123-iv-e.tex}{temporalia/post-nativitatem-sexta/ps123-boh.tex}


\scriptura{Psalmus 124.}

\psalmusEtTranslatio{temporalia/post-nativitatem-sexta/ps124-iv-e.tex}{temporalia/post-nativitatem-sexta/ps124-boh.tex}



\pars{Capitulum}

\rubrica{Capitulum et Versiculus ut Extra Adventum, 
  pg. \pageref{sextacapitulum}.}

\rubricaFinisHoraePostNativitatem

\vfill


\pagebreak

% Post Nativitatem.

\hora{Ad Nonam.} %%%%%%%%%%%%%%%%%%%%%%%%%%%%%%%%%%%%%%%%

\initiumHoraeMinoresAdvent % this IS ok

\pars{Psalmi}

\nativitasAntiphonaV

\scriptura{Psalmus 125.}

\includescore{temporalia/ps125-initium-ii-D-auto.tex}

\psalmusEtTranslatio{temporalia/ps125-ii-d.tex}{empty.tex}


\scriptura{Psalmus 126.}

\psalmusEtTranslatio{temporalia/ps126-ii-d.tex}{empty.tex}


\scriptura{Psalmus 127.}

\psalmusEtTranslatio{temporalia/ps127-ii-d.tex}{empty.tex}


\pars{Capitulum}

\rubrica{Capitulum et Versiculus ut Extra Adventum, 
  pg. \pageref{nonacapitulum}.}

\rubricaFinisHoraePostNativitatem

\vfill


% \pagebreak

%%%%%%%%%%%%%%%%%%%%%%%%%%%%%%%%%%%%%%%%%%%%%%%%%%%%%%%%%%%%%%%%%%%%%%

\divisio{Tempore Paschali.}

% no hour in the headers of pages
\fancyhead[LE]{\thepage\ / }
\fancyhead[RO]{ / \thepage}

\rubrica{Officium B. Mariæ Virginis dicitur sicut extra Adventum,
sed ad Benedictus, ad Magnificat et ad Nunc dimittis, dicitur
Ant. Regina cæli.}

% those special characters are undefined in the italic version of Junicode
\rubrica{Antiphonis autem, \textnormal{℣℣.} et \textnormal{℟℟.} 
non additur in fine Allelúia.}

\vfill



\pagebreak

% colophon
\pagestyle{empty}

\tableofcontents

\vfill

\pagebreak

Fontes. 
Textus secundum 
\textit{Officium parvum beatæ Mariæ Virginis et Officium
defunctorum cum Psalmis gradualibus et pænitentialibus ac Litaniis sanctorum
e Breviario romano a Pio papa X reformato excerpta.}
Editio I juxta typicam III Vaticanam Breviarii romani. 
Ratisbonæ, sumptibus et typis Friderici Pustet 1925. /
Omnes cantus Horarum cursus diurni atque Completorii 
necnon prima Antiphona Matutini secundum
\textit{Antiphonale sacrosanctæ Romanæ Ecclesiæ pro diurnis horis
SS. D. N. Pii X. Pontificis maximi jussu restitutum et editum.}
Romæ typis Polyglottis Vaticanis 1912. /
Responsoria pro Matutino secundum 
\textit{Nocturnale Romanum. Antiphonale sacrosanctæ Romanæ Ecclesiæ
pro nocturnis horis.}
2002. /
Antiphonæ reliquæ pro Matutino sumptæ sunt de manuscriptis:
\textit{Antiphonarium pro Ecclesia Einsidlensi.}
Einsiedeln, Stiftsbibliothek, Codex 611(89), CANTUS CH-E 611.
(Antiphonæ 2, 3, 7, 8, 9) /
\textit{Antiphonarium Fratrum Minorum Secundum Consuetudinem Romanae Curiae.}
Fribourg/Freiburg, Couvent des Cordeliers/Franziskanerkloster, Ms. 2,
CANTUS CH-Fco 2.
(Antiphonæ 4, 5, 6)

Translationes psalmorum ex
Hejčl Jan: Žaltář čili Kniha žalmů, Praha~1922.

Collaborantes.
Textus latinos cantusque transcripsit et omnem laborem typographicum peregit
Jakub Pavlík. /
Psalmos in lingua bohemica de libro supra dicto transcripsit
Barbora Maturová et idem Jakub. /
%Filip Srovnal librum istum pr"aeparare mandavit et laborem exprobrationibus
%utilissimis comitabatur. /
%Imaginem, qu"ae paginam tituli ornat, Klára Jirsová pinxit.

Instrumenta adhibita.
LuaTeX, %http://www.luatex.org / 
Gregorio, %http://home.gna.org/gregorio /
typi Junicode. %http://junicode.sourceforge.net

\begin{center}
Liber hic imprimis ad usum chori 
\guillemotright Conventus Choralis\guillemotleft\ 
paratus est
et secundum eius consuetudines.
http://www.introitus.cz

\vspace{1cm}

{\large Editio Sancti Wolfgangi \annusEditionis .}

\vspace{2mm}

Series \guillemotright Conventus\guillemotleft, vol. III.

\vspace{1cm}

http://stiwolfgangi.xf.cz

\vfill

\today

\end{center}

\end{document}
