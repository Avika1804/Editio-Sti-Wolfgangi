% LuaLaTeX

\documentclass[a4paper, twoside, 12pt]{article}
\usepackage[latin]{babel}
%\usepackage[landscape, left=3cm, right=1.5cm, top=2cm, bottom=1cm]{geometry} % okraje stranky
%\usepackage[landscape, a4paper, mag=1166, truedimen, left=2cm, right=1.5cm, top=1.6cm, bottom=0.95cm]{geometry} % okraje stranky
\usepackage[landscape, a4paper, mag=1400, truedimen, left=0.5cm, right=0.5cm, top=0.5cm, bottom=0.5cm]{geometry} % okraje stranky

\usepackage{fontspec}
\setmainfont[FeatureFile={junicode.fea}, Ligatures={Common, TeX}, RawFeature=+fixi]{Junicode}
%\setmainfont{Junicode}

% shortcut for Junicode without ligatures (for the Czech texts)
\newfontfamily\nlfont[FeatureFile={junicode.fea}, Ligatures={Common, TeX}, RawFeature=+fixi]{Junicode}

\usepackage{multicol}
\usepackage{color}
\usepackage{lettrine}
\usepackage{fancyhdr}

% usual packages loading:
\usepackage{luatextra}
\usepackage{graphicx} % support the \includegraphics command and options
\usepackage{gregoriotex} % for gregorio score inclusion
\usepackage{gregoriosyms}
\usepackage{wrapfig} % figures wrapped by the text
\usepackage{parcolumns}
\usepackage[contents={},opacity=1,scale=1,color=black]{background}
\usepackage{tikzpagenodes}
\usepackage{calc}
\usepackage{longtable}
\usetikzlibrary{calc}

\setlength{\headheight}{14.5pt}

% Commands used to produce a typical "Conventus" booklet

\newenvironment{titulusOfficii}{\begin{center}}{\end{center}}
\newcommand{\dies}[1]{#1

}
\newcommand{\nomenFesti}[1]{\textbf{\Large #1}

}
\newcommand{\celebratio}[1]{#1

}

\newcommand{\hora}[1]{%
\vspace{0.5cm}{\large \textbf{#1}}

\fancyhead[LE]{\thepage\ / #1}
\fancyhead[RO]{#1 / \thepage}
\addcontentsline{toc}{subsection}{#1}
}

% larger unit than a hora
\newcommand{\divisio}[1]{%
\begin{center}
{\Large \textsc{#1}}
\end{center}
\fancyhead[CO,CE]{#1}
\addcontentsline{toc}{section}{#1}
}

% a part of a hora, larger than pars
\newcommand{\subhora}[1]{
\begin{center}
{\large \textit{#1}}
\end{center}
%\fancyhead[CO,CE]{#1}
\addcontentsline{toc}{subsubsection}{#1}
}

% rubricated inline text
\newcommand{\rubricatum}[1]{\textit{#1}}

% standalone rubric
\newcommand{\rubrica}[1]{\vspace{3mm}\rubricatum{#1}}

\newcommand{\notitia}[1]{\textcolor{red}{#1}}

\newcommand{\scriptura}[1]{\hfill \small\textit{#1}}

\newcommand{\translatioCantus}[1]{\vspace{1mm}%
{\noindent\footnotesize \nlfont{#1}}}

% pruznejsi varianta nasledujiciho - umoznuje nastavit sirku sloupce
% s prekladem
\newcommand{\psalmusEtTranslatioB}[3]{
  \vspace{0.5cm}
  \begin{parcolumns}[colwidths={2=#3}, nofirstindent=true]{2}
    \colchunk{
      \input{#1}
    }

    \colchunk{
      \vspace{-0.5cm}
      {\footnotesize \nlfont
        \input{#2}
      }
    }
  \end{parcolumns}
}

\newcommand{\psalmusEtTranslatio}[2]{
  \psalmusEtTranslatioB{#1}{#2}{8.5cm}
}


\newcommand{\canticumMagnificatEtTranslatio}[1]{
  \psalmusEtTranslatioB{#1}{temporalia/extra-adventum-vespers/magnificat-boh.tex}{12cm}
}
\newcommand{\canticumBenedictusEtTranslatio}[1]{
  \psalmusEtTranslatioB{#1}{temporalia/extra-adventum-laudes/benedictus-boh.tex}{10.5cm}
}

% volne misto nad antifonami, kam si zpevaci dokresli neumy
\newcommand{\hicSuntNeumae}{\vspace{0.5cm}}

% prepinani mista mezi notovymi osnovami: pro neumovane a neneumovane zpevy
\newcommand{\cantusCumNeumis}{
  \setgrefactor{17}
  \global\advance\grespaceabovelines by 5mm%
}
\newcommand{\cantusSineNeumas}{
  \setgrefactor{17}
  \global\advance\grespaceabovelines by -5mm%
}

% znaky k umisteni nad inicialu zpevu
\newcommand{\superInitialam}[1]{\gresetfirstlineaboveinitial{\small {\textbf{#1}}}{\small {\textbf{#1}}}}

% pars officii, i.e. "oratio", ...
\newcommand{\pars}[1]{\textbf{#1}}

\newenvironment{psalmus}{
  \setlength{\parindent}{0pt}
  \setlength{\parskip}{5pt}
}{
  \setlength{\parindent}{10pt}
  \setlength{\parskip}{10pt}
}

%%%% Prejmenovat na latinske:
\newcommand{\nadpisZalmu}[1]{
  \hspace{2cm}\textbf{#1}\vspace{2mm}%
  \nopagebreak%

}

% mode, score, translation
\newcommand{\antiphona}[3]{%
\hicSuntNeumae
\superInitialam{#1}
\includescore{#2}

#3
}
 % Often used macros

\newcommand{\annusEditionis}{2021}

%%%% Vicekrat opakovane kousky

\newcommand{\anteOrationem}{
  \rubrica{Ante Orationem, cantatur a Superiore:}

  \pars{Supplicatio Litaniæ.}

  \cuminitiali{}{temporalia/supplicatiolitaniae.gtex}

  \pars{Oratio Dominica.}

  \cuminitiali{}{temporalia/oratiodominica.gtex}

  \rubrica{Deinde dicitur ab Hebdomadario:}

  \cuminitiali{}{temporalia/dominusvobiscum-solemnis.gtex}

  \rubrica{In choro monialium loco Dominus vobiscum dicitur:}

  \sineinitiali{temporalia/domineexaudi.gtex}
}

\setlength{\columnsep}{30pt} % prostor mezi sloupci

%%%%%%%%%%%%%%%%%%%%%%%%%%%%%%%%%%%%%%%%%%%%%%%%%%%%%%%%%%%%%%%%%%%%%%%%%%%%%%%%%%%%%%%%%%%%%%%%%%%%%%%%%%%%%
\begin{document}

% Here we set the space around the initial.
% Please report to http://home.gna.org/gregorio/gregoriotex/details for more details and options
\grechangedim{afterinitialshift}{2.2mm}{scalable}
\grechangedim{beforeinitialshift}{2.2mm}{scalable}
\grechangedim{interwordspacetext}{0.22 cm plus 0.15 cm minus 0.05 cm}{scalable}%
\grechangedim{annotationraise}{-0.2cm}{scalable}

% Here we set the initial font. Change 38 if you want a bigger initial.
% Emit the initials in red.
\grechangestyle{initial}{\color{red}\fontsize{38}{38}\selectfont}

\pagestyle{empty}

%%%% Titulni stranka
\begin{titulusOfficii}
\dies{21. Ianuarii.}
\nomenFesti{S. Agnetis, Virginis et Martyris.}
\end{titulusOfficii}

\vfill

\begin{center}
%Ad usum et secundum consuetudines chori \guillemotright{}Conventus Choralis\guillemotleft.

%Editio Sancti Wolfgangi \annusEditionis
\end{center}

\scriptura{}

\pars{}

\pagebreak

\renewcommand{\headrulewidth}{0pt} % no horiz. rule at the header
\fancyhf{}
\pagestyle{fancy}

\cantusSineNeumas

\newcommand{\oratio}{\pars{Oratio.}

\noindent Omnípotens sempitérne Deus, qui infírma mundi éligis ut fórtia quæque confúndas, concéde propítius, ut, qui beátæ Agnétis, mártyris, tuæ natalícia celebrámus, eius in fide constántiam subsequámur.

\noindent Per Dóminum nostrum Iesum Christum, Fílium tuum, qui tecum vivit et regnat in unitáte Spíritus Sancti, Deus, per ómnia sǽcula sæculórum.

\noindent \Rbardot{} Amen.}

\newcommand{\lectioi}{\pars{Lectio I.} \scriptura{Dt. 9, 7-21.25-29}

\noindent De libro Deuteronómii.

\noindent In diébus illis: Locútus est Móyses pópulo dicens: Cum transíssent quadragínta dies et tótidem noctes, dedit mihi Dóminus duas tábulas lapídeas, tábulas fœ́deris, dixítque mihi: “Surge et descénde hinc cito, quia peccávit pópulus tuus, quem eduxísti de Ægýpto: deseruérunt velóciter viam, quam præcépi eis, fecerúntque sibi conflátile”. Rursúmque ait Dóminus ad me: “Cerno quod pópulus iste duræ cervícis sit; dimítte me, ut cónteram eos et déleam nomen eórum sub cælo et fáciam te in gentem, quæ hac fórtior et maior sit”. Cumque revérsus de monte ardénte descénderem et duas tábulas fœ́deris utráque tenérem manu vidissémque vos peccásse Dómino Deo vestro et fecísse vobis vítulum conflátilem ac deseruísse velóciter viam eius, quam Dóminus vobis præcéperat, arrípui duas tábulas et proiéci eas de mánibus meis confregíque eas in conspéctu vestro; Et iácui coram Dómino quadragínta diébus ac nóctibus, quibus eum supplíciter deprecábar, ne deléret vos, ut fúerat comminátus. Et orans dixi: Dómine Deus, ne dispérdas pópulum tuum et hereditátem tuam, quam redemísti in magnitúdine tua, quod eduxísti de Ægýpto in manu forti. Recordáre servórum tuórum Abraham, Isaac et Iacob; ne aspícias durítiam pópuli huius et impietátem atque peccátum, ne forte dicant habitatóres terræ, de qua eduxísti nos: “Non póterat Dóminus introdúcere eos in terram, quam pollícitus est eis, et óderat illos; idcírco edúxit, ut interfíceret eos in solitúdine”. Attamen ipsi sunt pópulus tuus et heréditas tua, quos eduxísti in fortitúdine tua magna et in bráchio tuo exténto».}
\newcommand{\responsoriumi}{\pars{Responsorium 1.} \scriptura{\Rbardot{} Ps. 85, 12-13 \Vbardot{} ibid., 13; \textbf{H88}}

\vspace{-5mm}

\responsorium{IV}{temporalia/resp-confitebortibidomine-CROCHU.gtex}{}}
\newcommand{\lectioii}{\pars{Lectio II.} \scriptura{Lib. 1, cap. 2. 5. 7-9: PL 16 [edit. 1845], 189-191}

\noindent Ex Tractátu sancti Ambrósii epíscopi De virgínibus.

\noindent Natális est vírginis, integritátem sequámur; natális est mártyris, hóstias immolémus. Natális est sanctæ Agnétis. Hæc duódecim annórum martýrium fecísse tráditur. Quo detestabílior crudélitas, quæ nec minúsculæ pepércit ætáti, immo magna vis fídei, quæ étiam ab illa testimónium invénit ætáte.

\noindent Fuítne in illo corpúsculo vúlneri locus? Et quæ non hábuit quo ferrum recíperet, hábuit quo ferrum vínceret. At istíus ætátis puéllæ torvos étiam vultus paréntum ferre non possunt, et acu distrícta solent puncta flere quasi vúlnera.

\noindent Hæc inter cruéntas carníficum impávida manus, hæc stridéntium grávibus immóbilis tráctibus catenárum, nunc furéntis mucróni mílitis totum offérre corpus, mori adhuc néscia, sed paráta; vel si ad aras invíta raperétur, téndere Christo inter ignes manus atque in ipsis sacrílegis focis trophǽum Dómini signáre victóris; nunc ferrátis colla manúsque ambas insérere néxibus, sed nullus tam ténuia membra póterat nexus inclúdere.

\noindent Novum martýrii genus? Nondum idónea pœnæ et iam matúra victóriæ; certáre diffícilis, fácilis coronári; magistérium virtútis implévit, quæ præiudícium vehébat ætátis. Non sic ad thálamum nupta properáret, ut ad supplícii locum læta succéssu, gradu festína virgo procéssit, non intórto crine caput compta sed Christo; non flósculis redimíta, sed móribus.}
\newcommand{\responsoriumii}{\pars{Responsorium 2.} \scriptura{\textbf{H111}}

\vspace{-5mm}

\responsorium{VII}{temporalia/resp-amochristum-CROCHU.gtex}{}}
\newcommand{\lectioiii}{\pars{Lectio III.}

\noindent Flere omnes, ipsa sine fletu. Mirári pleríque quod tam fácile vitæ suæ pródiga, quam nondum háuserat, iam quasi perfúncta donáret. Stupére univérsi quod iam Divinitátis testis exsísteret, quæ adhuc árbitra sui per ætátem esse non posset. Effécit dénique ut ei de Deo crederétur cui de hómine adhuc non crederétur, quia quod ultra natúram est, de Auctóre natúræ est.

\noindent Quanto terróre egit cárnifex ut timerétur, quantis blandítiis ut suadéret, quantórum vota ut sibi ad núptias perveníret. At illa: «Et hæc Sponsi iniúria est exspectáre placitúrum; qui me sibi prior elégit, accípiet. Quid, percússor, moráris? Péreat corpus, quod amári potest óculis, quibus nolo». Stetit, orávit, cervícem infléxit.

\noindent Cérneres trepidáre carníficem, quasi ipse addíctus fuísset; trémere percussóris déxteram, pallére ora aliéno timéntis perículo, cum puélla non timéret suo. Habétis ígitur in una hóstia duplex martýrium, pudóris et religiónis. Et virgo permánsit et martýrium obtínuit.}
\newcommand{\responsoriumiii}{\pars{Responsorium 3.} \scriptura{\textbf{H113}}

\vspace{-5mm}

\responsorium{VIII}{temporalia/resp-ipsisumdesponsata-CROCHU-cumdox.gtex}{}}

\hora{Ad Matutinum.} %%%%%%%%%%%%%%%%%%%%%%%%%%%%%%%%%%%%%%%%%%%%%%%%%%%%%
%\sideThumbs{Matutinum}

\vspace{2mm}

\cuminitiali{}{temporalia/dominelabiamea.gtex}

\vfill
%\pagebreak

\vspace{2mm}

\pars{Invitatorium.}

%\vspace{-6mm}

\antiphona{IV*}{temporalia/inv-regemmartyrum.gtex}

\vfill
\pagebreak

\pars{Hymnus.}

{
\grechangedim{interwordspacetext}{0.10 cm plus 0.15 cm minus 0.05 cm}{scalable}%
\antiphona{IV}{temporalia/hym-IgneDivini.gtex}
\grechangedim{interwordspacetext}{0.22 cm plus 0.15 cm minus 0.05 cm}{scalable}%
}

\vspace{-3mm}

\vfill
\pagebreak

\pars{Psalmus 1.} \scriptura{\textbf{H111}}

\vspace{-4mm}

\antiphona{I g}{temporalia/ant-ipsisoli.gtex}

%\vspace{-2mm}

\scriptura{Ps. 43, 2-9}

%\vspace{-2mm}

\initiumpsalmi{temporalia/ps43i-initium-i-g-auto.gtex}

\input{temporalia/ps43i-i-g.tex} \Abardot{}

\vfill
\pagebreak

\pars{Psalmus 2.} \scriptura{\textbf{H110}}

\vspace{-4mm}

\antiphona{I g}{temporalia/ant-christivirgo.gtex}

%\vspace{-2mm}

\scriptura{Ps. 43, 10-17}

%\vspace{-2mm}

\initiumpsalmi{temporalia/ps43ii-initium-i-g-auto.gtex}

\input{temporalia/ps43ii-i-g.tex} \Abardot{}

\vfill
\pagebreak

\pars{Psalmus 3.} \scriptura{\textbf{H112}}

\vspace{-4mm}

\antiphona{I f}{temporalia/ant-statadextris.gtex}

%\vspace{-2mm}

\scriptura{Ps. 43, 18-26}

%\vspace{-2mm}

\initiumpsalmi{temporalia/ps43iii-initium-i-f-auto.gtex}

\input{temporalia/ps43iii-i-f.tex} \Abardot{}

\vfill
\pagebreak

\pars{Versus.}

\noindent \Vbardot{} Tribulátio et angústia invenérunt me.

\noindent \Rbardot{} Mandáta tua meditátio mea est.

\vspace{5mm}

\sineinitiali{temporalia/oratiodominica-mat.gtex}

\vspace{5mm}

\pars{Absolutio.}

\cuminitiali{}{temporalia/absolutio-exaudi.gtex}

\vfill
\pagebreak

\cuminitiali{}{temporalia/benedictio-solemn-benedictione.gtex}

\vspace{7mm}

\lectioi

\noindent \Vbardot{} Tu autem, Dómine, miserére nobis.
\noindent \Rbardot{} Deo grátias.

\vfill
\pagebreak

\responsoriumi

\vfill
\pagebreak

\cuminitiali{}{temporalia/benedictio-solemn-unigenitus.gtex}

\vspace{7mm}

\lectioii

\noindent \Vbardot{} Tu autem, Dómine, miserére nobis.
\noindent \Rbardot{} Deo grátias.

\vfill
\pagebreak

\responsoriumii

\vfill
\pagebreak

\cuminitiali{}{temporalia/benedictio-solemn-spiritus.gtex}

\vspace{7mm}

\lectioiii

\noindent \Vbardot{} Tu autem, Dómine, miserére nobis.
\noindent \Rbardot{} Deo grátias.

\vfill
\pagebreak

\responsoriumiii

\vfill
\pagebreak

\rubrica{Reliqua omittuntur, nisi Laudes separandæ sint.}

\sineinitiali{temporalia/domineexaudi.gtex}

\vfill

\oratio

\vfill

\noindent \Vbardot{} Dómine, exáudi oratiónem meam.
\Rbardot{} Et clamor meus ad te véniat.

\vfill

\noindent \Vbardot{} Benedicámus Dómino.
\noindent \Rbardot{} Deo grátias.

\vfill

\noindent \Vbardot{} Fidélium ánimæ per misericórdiam Dei requiéscant in pace.
\Rbardot{} Amen.

\vfill
\pagebreak

\hora{Ad Laudes.} %%%%%%%%%%%%%%%%%%%%%%%%%%%%%%%%%%%%%%%%%%%%%%%%%%%%%
%\sideThumbs{Laudes}

\cantusSineNeumas

\vspace{0.5cm}
\grechangedim{interwordspacetext}{0.18 cm plus 0.15 cm minus 0.05 cm}{scalable}%
\cuminitiali{}{temporalia/deusinadiutorium-communis.gtex}
\grechangedim{interwordspacetext}{0.22 cm plus 0.15 cm minus 0.05 cm}{scalable}%

\vfill
\pagebreak

\pars{Hymnus}

%\vspace{-4mm}

\grechangedim{interwordspacetext}{0.14 cm plus 0.15 cm minus 0.05 cm}{scalable}%
\cuminitiali{VIII}{temporalia/hym-AgnesBeatae.gtex}
\grechangedim{interwordspacetext}{0.22 cm plus 0.15 cm minus 0.05 cm}{scalable}%
\vspace{-3mm}

\vfill
\pagebreak

\pars{Psalmus 1.} \scriptura{Cf. Is. 61, 10; \textbf{H114}}

\vspace{-4mm}

\antiphona{VII c}{temporalia/ant-annulosuo.gtex}

%\vspace{-2mm}

\scriptura{Psalmus 62}

%\vspace{-2mm}

\initiumpsalmi{temporalia/ps62-initium-vii-c-auto.gtex}

%\vspace{-1.5mm}

\input{temporalia/ps62-vii-c.tex} \Abardot{}

\vfill
\pagebreak

\pars{Psalmus 2.} \scriptura{\textbf{H112}}

\vspace{-4mm}

\antiphona{VII c}{temporalia/ant-ipsisumdesponsata.gtex}

%\vspace{-2mm}

\scriptura{Canticum trium puerorum, Dan. 3, 57-88 et 56}

%\vspace{-3mm}

\initiumpsalmi{temporalia/dan3-initium-vii-c-auto.gtex}

\input{temporalia/dan3-vii-c.tex}

\rubrica{Hic non dicitur Gloria Patri, neque Amen.}

\vfill

\antiphona{}{temporalia/ant-ipsisumdesponsata.gtex}

\vfill
\pagebreak

\pars{Psalmus 3.} \scriptura{\textbf{H114}}

\vspace{-4mm}

\antiphona{VIII G}{temporalia/ant-congaudetemecum.gtex}

\scriptura{Psalmus 149}

\initiumpsalmi{temporalia/ps149-initium-viii-G-auto.gtex}

\input{temporalia/ps149-viii-G.tex} \Abardot{}

\vfill
\pagebreak

\pars{Lectio Brevis.} \scriptura{2 Cor. 1, 3-5}

\noindent Benedíctus Deus et Pater Dómini nostri Iesu Christi, Pater misericordiárum et Deus totíus consolatiónis, qui consolátur nos in omni tribulatióne nostra, ut possímus et ipsi consolári eos, qui in omni pressúra sunt, per exhortatiónem, qua exhortámur et ipsi a Deo; quóniam, sicut abúndant passiónes Christi in nobis, ita per Christum abúndat et consolátio nostra.

\vfill

\pars{Responsorium breve.}

\cuminitiali{VI}{temporalia/resp-adiuvabit.gtex}

\vfill
\pagebreak

\pars{Canticum Zachariæ.} \scriptura{\textbf{H114}}

\vspace{-4mm}

{
\grechangedim{interwordspacetext}{0.18 cm plus 0.15 cm minus 0.05 cm}{scalable}%
\antiphona{I g}{temporalia/ant-eccequodcupivi.gtex}
\grechangedim{interwordspacetext}{0.22 cm plus 0.15 cm minus 0.05 cm}{scalable}%
}

\vspace{-2mm}

\scriptura{Lc. 1, 68-79}

\vspace{-2mm}

\cantusSineNeumas
\initiumpsalmi{temporalia/benedictus-initium-i-g-auto.gtex}

%\vspace{-1.5mm}

\input{temporalia/benedictus-i-g.tex} \Abardot{}

\vspace{-1cm}

\vfill
\pagebreak

%\sideThumbs{{\scriptsize{}Fine horarum}}

\pars{Preces.}

\sineinitiali{}{temporalia/tonusprecum.gtex}

\noindent Fratres, Salvatórem nostrum, testem fidélem, per mártyres interféctos propter verbum Dei, \gredagger{} celebrémus, clamántes:

\Rbardot{} Redemísti nos Deo in sánguine tuo.

\noindent Per mártyres tuos, qui líbere mortem in testimónium fídei sunt ampléxi, \gredagger{} da nobis, Dómine, veram spíritus libertátem.

\Rbardot{} Redemísti nos Deo in sánguine tuo.

\noindent Per mártyres tuos, qui fidem usque ad sánguinem sunt conféssi, \gredagger{} da nobis, Dómine, puritátem fideíque constántiam.

\Rbardot{} Redemísti nos Deo in sánguine tuo.

\noindent Per mártyres tuos, qui, sustinéntes crucem, tua vestígia sunt secúti, \gredagger{} da nobis, Dómine, ærúmnas vitæ fórtiter sustinére.

\Rbardot{} Redemísti nos Deo in sánguine tuo.

\noindent Per mártyres tuos, qui stolas suas lavérunt in sánguine Agni, \gredagger{} da nobis, Dómine, omnes insídias carnis mundíque devíncere.

\Rbardot{} Redemísti nos Deo in sánguine tuo.

\vfill

\pars{Oratio Dominica.}

\cuminitiali{}{temporalia/oratiodominicaalt.gtex}

\vfill
\pagebreak

\rubrica{vel:}

\pars{Supplicatio Litaniæ.}

\cuminitiali{}{temporalia/supplicatiolitaniae.gtex}

\vfill

\pars{Oratio Dominica.}

\cuminitiali{}{temporalia/oratiodominica.gtex}

\vfill
\pagebreak

% Oratio. %%%
\oratio

\vspace{-1mm}

\vfill

\rubrica{Hebdomadarius dicit Dominus vobiscum, vel, absente sacerdote vel diacono, sic concluditur:}

\vspace{2mm}

\antiphona{C}{temporalia/dominusnosbenedicat.gtex}

\rubrica{Postea cantatur a cantore:}

\vspace{2mm}

\cuminitiali{IV}{temporalia/benedicamus-feria-laudes.gtex}

\vspace{1mm}

\vfill
\pagebreak

\end{document}
