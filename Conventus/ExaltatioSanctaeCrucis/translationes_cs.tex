% Where not otherwise stated, author of the translation is 
% Václav Ondráček.

\newcommand{\trAntiphonaI}{\translatioCantus{
  Jak veliký lásky div! Smrt zemřela, když na dřevě zemřel život.}}
\newcommand{\trAntiphonaII}{\translatioCantus{
  Zachraň nás, Kriste, Zachránce, silou kříže. 
  Petra jsi zachránil na moři - smiluj se nad námi.}}
\newcommand{\trAntiphonaIII}{\translatioCantus{
  Hle, před křížem Páně, prchněte síly odpůrčí. 
  Vítěz je lev z kmene Juda, kořen Davidův, aleluja.}}
\newcommand{\trAntiphonaIV}{\translatioCantus{
  Sluší se nám chlubit se křížem našeho Pána Ježíše Krista.}}
\newcommand{\trAntiphonaV}{\translatioCantus{
  Znamením kříže vysvoboď nás od nepřátel, Pane náš.}}

\newcommand{\trResponsoriumVesperae}{\translatioCantus{
  Klaníme se ti Kriste * a dobrořečíme ti. 
  ℣. Neboť svým křížem vykoupil jsi svět.}}
\newcommand{\trResponsoriumLaudes}{\translatioCantus{
  Toto kříže znamení *  bude na nebi. 
  ℣. Až Pán k soudu přijde.}}

% Jeruzalémská bible, KNA + Krystal 2009.
\newcommand{\trCapitulum}{\translatioCantus{
  Ať je mezi vámi takové smýšlení jako v Kristu Ježíši:
  On, ač jeho údělem bylo božství,
  nelpěl žárlivě na stavu, jenž ho činil rovným Bohu.
  Ale sám sebe zmařil tak, že na sebe vzal úděl otroka
  a stal se podobným lidem.}}

% Markéta Koronthályová: Vexilla Regis, BB art Praha 2004, 102.
% With the author's permission.
\newcommand{\trHymnusVexillaRegis}{\translatioCantus{
Korouhve královské jdou vpřed,\\
již září kříže tajemství,\\
na němž vtělený Stvořitel\\
visel na místě popravčím.\\

\noindent Byl také zraněn do srdce\\
bodnutím kopí hrozného;\\
aby nás obmyl od viny,\\
krev s vodou vytekla z něho. \columnbreak

\noindent Splnilo se, co pověděl\\
David v svém věrném proroctví:\\
nad národy teď kraluje\\
Bůh, který visel na dřevě.\\

\noindent Strome krásný a zářící,\\
zdobený krví Královou,\\
vybraný, by ses kmenem svým\\
svatého těla dotýkal. \columnbreak

\noindent Blažený, na tvých ramenou\\
cena za svět se platila;\\
stal jsi se váhou pro tělo,\\
jež podsvětí oloupila.\\

\noindent Buď zdráv, oltáři obětní,\\
kde ve slávě utrpení\\
život nad smrtí zvítězil,\\
svou smrtí život navrátil. \columnbreak

\noindent Buď vítán, Kříži, naděje,\\
v tomto čase Utrpení\\
dej zbožným větší milosti,\\
provinilým zahlaď viny.\\

\noindent Prameni spásy, Trojice,\\
tobě ať každý vzdá chválu:\\
komu dáš vítězství Kříže,\\
přidej i hojnou odměnu.
}}
\newcommand{\trVersiculusVesperae}{\translatioCantus{
℣. Toto znamení kříže bude na nebi.
℟. Až Bůh se v soudu vrátí den.
}}

\newcommand{\trHymnusLustraSex}{\scriptsize
\noindent Tělesnosti když šest věků\\
vyměřených minulo,\\
Beránek se na zem zrodil\\
za jediným účelem:\\
Aby v svůj čas na kříž vzepjat\\
na něm by obětován. \columnbreak

\noindent Hle žluč, ocet, třtina, posměch,\\
hřeby, kopí naposled;\\
jemně tělo pronikají\\
a krev s vodou prýští ven.\\
A proudem tím mocným, silným,\\
obmyta je celá zem. \columnbreak

\noindent Víře naší mezi všemi\\
kříži, strome vznešený\\
tobě rovný v žádném lese\\
plodem, květem neroste. \columnbreak

\noindent Svěš své paže strome mocný\\
a nitro své připrav spíš,\\
ten jenž stvořil pevnost tvoji\\
ti poshovět povolí,\\
abys měkce napjal na se\\
velkou údů krále tíž. \columnbreak

\noindent Byls jediný vyvolený\\
nést časnosti výkupné,\\
přístav chystat pro loď světa\\
když jej s mořem a zemí,\\
i hvězdami  zalévala\\
krev Beránkem prolitá. \columnbreak

\noindent Sláva a čas Bohu věků\\
jenž je nejvýš vznešený,\\
Otci stejná, Synu, Duchu,\\
nejvyšší nám útěše.\\
Chvála a moc buď mu navždy\\
po věky trvající.
}
\newcommand{\trVersiculusLaudes}{\translatioCantus{
  ℣. Klaníme se ti Kriste  a dobrořečíme ti. 
  ℟. Neboť svým křížem vykoupil jsi svět.}}

\newcommand{\trAntiphonaMagnificat}{\translatioCantus{
  Ó kříži nad hvězdy jasnější, na světě proslulý, 
  lidmi velmi milovaný, nedevše světější: 
  jediný jsi hoden byl tíži sěta nést: 
  sladké dříví, sladké hřeby, jež nesou sladké ovoce: 
  Tento  veškerý zachraň lid, jež k tvým chválám se zde shromáždil.}}
\newcommand{\trAntiphonaBenedictus}{\translatioCantus{
  Nad stromy vše cedrové jsi vyvýšen, 
  ty, na němž život světa visel, 
  na němž Kristus zvítězil, 
  a smrt na věky přemohla smrt.}}

\newcommand{\trOratio}{\translatioCantus{
  Bože, jenž nám dnes působíš radost výroční slavností Povýšení svatého kříže, 
  uděl nám, prosíme, poznání jeho tajemství na zemi, 
  abychom si zasloužili odměnu vykoupených v nebi. Skrze našeho Pána.}}

\newcommand{\trIntroitus}{\translatioCantus{
  Sluší se nám chlubit se křížem našeho Pána Ježíše Krista, 
  v němž je naše spása, život i  zmrtvýchvstání, 
  a skrze nějž jsme byli zachráněni a osvobozeni.
}}

\newcommand{\trGraduale}{\translatioCantus{
  Kristus stal se pro nás poslušný až k smrti, a to k smrti kříže! 
  Proto jej Bůh vyvýšil a dal mu jméno, jež je nade všechno jméno.}}

\newcommand{\trAlleluia}{\translatioCantus{
  Aleluja: Sladké dřevo, sladké hřeby, sladké neseš břemeno; 
  jedinýs byl hoden nésti pána nebes a krále.}}

\newcommand{\trOffertorium}{\translatioCantus{
  Zaštiť Pane svůj lid znamením svatého kříže 
  před všemi úklady veškerých nepřátel, 
  abychom ti mohli vděčně sloužit a naše oběť byla hodna přijetí. Aleluja.}}

\newcommand{\trCommunio}{\translatioCantus{
  Znamením kříže nás vysboboď, Bože náš, od našich nepřátel.}}
