\newcommand{\antiphonaI}{
  \antiphona{VII c}{cantus/amon33/crux_laud_ant1.tex}{}}
\newcommand{\antiphonaII}{
  \antiphona{III a}{cantus/amon33/crux_laud_ant2.tex}{}}
\newcommand{\antiphonaIII}{
  \antiphona{I f}{cantus/amon33/crux_laud_ant3.tex}{}}
\newcommand{\antiphonaIV}{
  \antiphona{VII c}{cantus/amon33/crux_laud_ant4.tex}{}}
\newcommand{\antiphonaV}{
  \antiphona{II D}{cantus/amon33/crux_laud_ant5.tex}{}}

\newcommand{\capitulumHocEnim}{
  \scriptura{Phil 2, 5-7.}
  
  \includescore{cantus/amon33/capitulum-HocEnimSentite}
}

\newcommand{\anteOrationem}{
  \rubrica{Ante Orationem, cantatur a Superiore:}

  \pars{Supplicatio Litaniæ.}

  \includescore{\ccommunesAM/supplicatiolitaniae.tex}

  \pars{Oratio Dominica.}

  \includescore{\ccommunesAM/oratiodominica.tex}

  \rubrica{Deinde dicitur ab Hebdomadario:}

  \includescore{\ccommunesAM/dominusvobiscum-solemnis.tex}

  % original rubric from the Antiphonale Monasticum:
  %\rubrica{In choro monialium loco Dominus vobiscum dicitur:}

  \rubrica{Absente sacerdote vel diacono, loco \textnormal{Dóminus vobíscum}
  dicitur:}

  \includescore{\ccommunesAM/domineexaudi.tex}
}

\newcommand{\oratio}{
  \pars{Oratio.}

  \includescore{cantus/amon33/crux_oratio}
}

\newcommand{\rubricaBenedicamus}{
  \rubrica{Repetito \textnormal{Dóminus vobíscum} in eodem tono ac antea,
  cantatur a cantore:}
}

\newcommand{\postBenedicamus}{
  \rubrica{Postea dicitur voce recta et paululum depressa:}

  \noindent ℣. Fidélium ánimæ per misericórdiam Dei requiéscant in pace.\\
  ℟. Amen.

  \noindent Pater noster. \rubricatum{totum secreto.}

  \rubrica{Deinde, si discedendum est a Choro:}

  \noindent ℣. Dóminus det nobis suam pacem.\\
  ℟. Et vitam ætérnam. Amen.

  \rubrica{Tunc dicitur Antiphona B.~M.~V. pro tempore, 
    cum \textnormal{℣.} et Oratione
    in Tono simplici, pg. \pageref{antiphonafinalis}. Deinde:}

  \noindent ℣. Divínum auxílium máneat semper nobíscum.\\
  ℟. Et cum frátribus nostris abséntibus. Amen.
}

% the two following prayers are from
% Breviarium monasticum, Romae sumptibus Josephi Salviucci 1831, p. lxxxiv.
% http://books.google.cz/books?id=GnFG6Z4Xuc8C&dq=breviarium%20monasticum&hl=cs&pg=PR1#v=onepage&q=breviarium%20monasticum&f=false
% but - this needs to be verified - it seems that exactly the same prayers
% were in the secular as well as monastic breviary, possibly since the
% post-Tridentine reform
\newcommand{\anteOfficiumOratio}{
\rubrica{Oratio dicenda ante inchoationem Divini Officii.}

\lettrine{A}{peri} Dómine os meum ad benedicéndum nomen sanctum tuum:
munda quoque cor meum ab ómnibus vanis, pervérsis, et aliénis
cogitatiónibus;
intelléctum illúmina, afféctum inflámma,
ud digne, atténte ac devóte hoc Offícium recitáre váleam,
et exaudíri mérear ante conspéctum Divínæ Majestátis tuæ.
Per Christum, Dominum nostrum.
℟. Amen.

Dómine, in unióne illíus Divínæ intentiónis,
qua ipse in terris laudes Deo persolvísti,
has tibi \rubricatum{(vel \textnormal{hanc tibi Horam})} Horas persólvo.
}

\newcommand{\postOfficiumOratio}{
\rubrica{
  Orationem sequentem devote post Officium recitantibus
  Leo Papa X. defectus, et culpas in eo persolvendo ex humana
  fragilitate contractas, indulsit, et dicitur flexis genibus.
}

\lettrine{S}{acrosánctæ} et indivíduæ Trinitáti,
crucifixi Domini nostri Jesu Christi humanitáti,
beatissimæ et gloriosíssimæ sempérque Virginis Maríæ
fecúndæ integritáti, 
et omnium Sanctórum universitáti,
sit sempitérna laus, honor, virtus et gloria
ab omni creatúra,
nobísque remíssio omnium peccatórum,
per infiníta sæcula sæculórum.
℟. Amen.

\noindent ℣. Beáta viscera Maríæ Virginis, quæ portavérunt
ætérni Patris Fílium.\\
℟. Et beáta ubera, quæ lactavérunt Christum Dominum.

Pater noster. Ave María.
}
