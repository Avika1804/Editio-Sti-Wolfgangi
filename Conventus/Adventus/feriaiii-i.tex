\newcommand{\titulus}{\nomenFesti{S. Andreæ, Apostoli.}
\dies{Die 30. Novembris.}}
\newcommand{\oratio}{\pars{Oratio.}

\noindent Maiestátem tuam, Dómine, supplíciter exorámus, ut, sicut Ecclésiæ tuæ beátus Andréas apóstolus éxstitit prædicátor et rector, ita apud te sit pro nobis perpétuus intercéssor.

\noindent Per Dóminum nostrum Iesum Christum, Fílium tuum, qui tecum vivit et regnat in unitáte Spíritus Sancti, Deus, per ómnia sǽcula sæculórum.

\noindent \Rbardot{} Amen.}
\newcommand{\invitatorium}{\pars{Invitatorium.}

\antiphona{IV*}{temporalia/inv-regemapostolorum-iv.gtex}}
\newcommand{\hymnusmatutinum}{\pars{Hymnus.}

\antiphona{III}{temporalia/hym-OSempiternaeCuriae.gtex}}
\newcommand{\matutinum}{\pars{Psalmus 1.} \scriptura{Mt. 4, 18; \textbf{H356}}

\vspace{-4mm}

\antiphona{VII c}{temporalia/ant-dignumsibi.gtex}

%\vspace{-2mm}

\scriptura{Ps. 18}

%\vspace{-2mm}

\initiumpsalmi{temporalia/ps18-initium-vii-c-auto.gtex}

\input{temporalia/ps18-vii-c.tex}

\antiphona{}{temporalia/ant-dignumsibi.gtex}

\vfill
\pagebreak

\pars{Psalmus 2.} \scriptura{\textbf{H355}}

\vspace{-4mm}

\antiphona{VII c}{temporalia/ant-egosipatibulum.gtex}

%\vspace{-2mm}

\scriptura{Ps. 63}

\initiumpsalmi{temporalia/ps63-initium-vii-c-auto.gtex}

\input{temporalia/ps63-vii-c.tex} \Abardot{}

\vfill
\pagebreak

\pars{Psalmus 3.} \scriptura{\textbf{H359}}

\vspace{-4mm}

\antiphona{VIII G}{temporalia/ant-beatusandreas.gtex}

%\vspace{-2mm}

\scriptura{Ps. 96}

%\vspace{-2mm}

\initiumpsalmi{temporalia/ps96-initium-viii-g-auto.gtex}

\input{temporalia/ps96-viii-g.tex} \Abardot{}

\vfill
\pagebreak}
\newcommand{\matversus}{\noindent \Vbardot{} Narravérunt laudes Dómini et virtútes eius.

\noindent \Rbardot{} Et mirabília eius quæ fecit.}

\newcommand{\lectioi}{\pars{Lectio I.} \scriptura{1 Cor. 1, 18-31; 2, 1-5}

\noindent De Epístola prima beáti Pauli apóstoli ad Corínthios.

\noindent Fratres: Verbum crucis pereúntibus quidem stultítia est, his autem, qui salvi fiunt, id est nobis, virtus Dei est. Scriptum est enim: \emph{«Perdam sapiéntiam sapiéntium et prudéntiam prudéntium reprobábo».} Ubi sápiens? Ubi scriba? Ubi conquisítor huius sǽculi? Nonne stultam fecit Deus sapiéntiam huius mundi? Nam quia in Dei sapiéntia non cognóvit mundus per sapiéntiam Deum, plácuit Deo per stultítiam prædicatiónis salvos fácere credéntes. Quóniam et Iudǽi signa petunt et Græci sapiéntiam quærunt, nos autem prædicámus Christum crucifíxum, Iudǽis quidem scándalum, géntibus autem stultítiam; ipsis autem vocátis, Iudǽis atque Græcis, Christum Dei virtútem et Dei sapiéntiam, quia quod stultum est Dei, sapiéntius est homínibus et, quod infírmum est Dei, fórtius est homínibus.

\noindent Vidéte enim vocatiónem vestram, fratres, quia non multi sapiéntes secúndum carnem, non multi poténtes, non multi nóbiles; sed, quæ stulta sunt mundi, elégit Deus, ut confúndat sapiéntes, et infírma mundi elégit Deus, ut confúndat fórtia, et ignobília mundi et contemptibília elégit Deus, quæ non sunt, ut ea, quæ sunt, destrúeret, ut non gloriétur omnis caro in conspéctu Dei. Ex ipso autem vos estis in Christo Iesu, qui factus est sapiéntia nobis a Deo et iustítia et sanctificátio et redémptio, ut quemádmodum scriptum est: \emph{«Qui gloriátur, in Dómino gloriétur».}

\noindent Et, ego, cum veníssem ad vos, fratres, veni non per sublimitátem sermónis aut sapiéntiæ annúntians vobis mystérium Dei. Non enim iudicávi scire me áliquid inter vos nisi Iesum Christum et hunc crucifíxum. Et ego in infirmitáte et timóre et tremóre multo fui apud vos, et sermo meus et prædicátio mea non in persuasibílibus sapiéntiæ verbis, sed in ostensióne Spíritus et virtútis, ut fides vestra non sit in sapiéntia hóminum sed in virtúte Dei.}
\newcommand{\responsoriumi}{\pars{Responsorium 1.} \scriptura{\textbf{H357}}

\vspace{-5mm}

\responsorium{III}{temporalia/resp-doctorbonusetamicusdei-CROCHU.gtex}{}}
\newcommand{\lectioii}{\pars{Lectio II.} \scriptura{Sermo 1 de S. Andréa, 1.5.10 : EC 5,427.430.433}

\noindent Ex Sermónibus sancti Bernárdi abbátis.

\noindent Celebrántes hódie gloriósum beáti Andréæ triúmphum, in verbis grátiæ quæ procedébant de ore eius, exsultávimus et delectáti sumus. Neque enim locus póterat esse tristítiæ, ubi tam veheménter lætabátur et ipse. Nemo ex nobis compássus est sic patiénti, nemo ausus est plángere exsultántem. Dénique cum ducerétur ipse beátus Andréas ad crucem, pópulus, qui sanctum et iustum dolébat iniúste damnári, prohibére vóluit ne punirétur; sed magis ipse instantíssima prece prohíbuit eos, ne non coronarétur, immo ne non paterétur. Desiderábat síquidem dissólvi et cum Christo esse, sed in cruce quam semper amáverat.

\noindent Desiderábat regnum intráre, sed per patíbulum. Quid enim dicit illi amátæ suæ? «Per te, inquit, me recípiat, qui per te me redémit.» Ergo si dilígimus eum, congaudémus ei, non solum quia coronátus, sed et quia crucifíxus, quia desidérium ánimæ eius tríbuit ei Dóminus et \emph{pósuit in cápite eius corónam de lápide pretióso.} Verúmtamen dum congratulámur ei, quod diu desiderátæ crucis frui mereátur ampléxu, mirum valde est, si non ipsum eius mirámur gáudium, cui congratulámur.}
\newcommand{\responsoriumii}{\pars{Responsorium 2.} \scriptura{\textbf{H358}}

\vspace{-5mm}

\responsorium{VIII}{temporalia/resp-obonacrux-CROCHU.gtex}}
\newcommand{\lectioiii}{\pars{Lectio III.}

\noindent Tríplicem licet consideráre gradum: incipiéntium, proficiéntium, perfectórum. \emph{Inítium} enim \emph{sapiéntiæ, timor Dómini;}  médium, spes; cáritas, plenitúdo. Dénique Apóstolum audi, quia \emph{plenitúdo legis est cáritas.} Qui initiátur a timóre, crucem Christi sústinet patiénter; qui próficit in spe, portat libénter; qui consummátur in caritáte, ampléctitur iam ardénter. Solus iste est qui dícere possit, quia «amátor tuus semper fui et desiderávi amplécti te».

\noindent Felix ánima, quæ ad hunc caritátis pervénerit statum! Nec sane desperándum nobis, quandóquidem eius qui pervénit, ob hoc máxime memória celebrátur, ut ipsíus et invocémus auxílium et provocémur exémplo. Si dicis beátum Andréam apóstolum esse, non posse te, qui pusíllus es, eius sequi vestígia, púdeat certe vel eos qui tecum sunt non imitári. Nemo repénte fit summus: ascendéndo, non volándo apprehénditur súmmitas scalæ.}
\newcommand{\responsoriumiii}{\pars{Responsorium 3.} \scriptura{\textbf{H357}}

\vspace{-5mm}

\responsorium{VII}{temporalia/resp-salvecrux-CROCHU-cumdox.gtex}{}

\vfill
\pagebreak

\pars{Hymnus Ambrosianus} \scriptura{Alio modo, iuxta morem Romanum}

\vspace{-2mm}

{
\grechangedim{interwordspacetext}{0.26 cm plus 0.15 cm minus 0.05 cm}{scalable}%
\cuminitiali{III}{temporalia/tedeum-romanum-gn.gtex}
\grechangedim{interwordspacetext}{0.22 cm plus 0.15 cm minus 0.05 cm}{scalable}%
}}
\newcommand{\deusinadiutorium}{\grechangedim{interwordspacetext}{0.18 cm plus 0.15 cm minus 0.05 cm}{scalable}%
\cuminitiali{}{temporalia/deusinadiutorium-alter.gtex}}
\newcommand{\hymnuslaudes}{\pars{Hymnus.} \scriptura{Petrus Damianus (\olddag{} 1072)}

\cuminitiali{IV}{temporalia/hym-CaptatorOlim.gtex}}
\newcommand{\laudes}{\pars{Psalmus 1.} \scriptura{Io. 1, 40; \textbf{H355}}

\vspace{-4mm}

\antiphona{I f}{temporalia/ant-unusexduobus.gtex}

\vspace{-2mm}

\scriptura{Psalmus 62.}

\vspace{-1mm}

\initiumpsalmi{temporalia/ps62-initium-i-f-auto.gtex}

%\vspace{-1.5mm}

\input{temporalia/ps62-i-f.tex} \Abardot{}

\vfill
\pagebreak

\pars{Psalmus 2.} \scriptura{\textbf{H357}}

\vspace{-4mm}

\antiphona{VIII G}{temporalia/ant-dilexitandream.gtex}

%\vspace{-2mm}

\scriptura{Canticum trium puerorum, Dan. 3, 57-88 et 56}

\initiumpsalmi{temporalia/dan3-initium-viii-g-auto.gtex}

\input{temporalia/dan3-viii-g-sinedox.tex}

\rubrica{Hic non dicitur Gloria Patri, neque Amen.}

\vfill

\antiphona{}{temporalia/ant-dilexitandream.gtex}

\vfill
\pagebreak

\pars{Psalmus 3.} \scriptura{Io. 1, 41}

\vspace{-4mm}

\antiphona{VIII G}{temporalia/ant-dixitandreas.gtex}

%\vspace{-2mm}

\scriptura{Psalmus 149}

%\vspace{-2mm}

\initiumpsalmi{temporalia/ps149-initium-viii-g-auto.gtex}

\input{temporalia/ps149-viii-g.tex} \Abardot{}

\vfill
\pagebreak}
\newcommand{\lectiobrevis}{\pars{Lectio Brevis.} \scriptura{Eph. 2, 19-22}

\noindent Iam non estis extránei et ádvenæ, sed estis concíves sanctórum et doméstici Dei, superædificáti super fundaméntum apostolórum et prophetárum, ipso summo angulári lápide Christo Iesu, in quo omnis ædificátio compácta crescit in templum sanctum in Dómino, in quo et vos coædificámini in habitáculum Dei in Spíritu.}
\newcommand{\responsoriumbreve}{\pars{Responsorium breve.} \scriptura{Ps. 44, 17-18}

\cuminitiali{VI}{temporalia/resp-constitueseos.gtex}}
\newcommand{\preces}{\noindent Fratres caríssimi, hereditátem cæléstem ab Apóstolis habéntes, \gredagger{} grátias agámus Patri nostro pro ómnibus donis eius, clamántes:

\Rbardot{} Te laudat Apostolórum chorus, Dómine.

\noindent Laus tibi, Dómine, pro mensa Córporis et Sánguinis Christi, nobis ab Apóstolis trádita, \gredagger{} qua refícimur et vívimus:

\Rbardot{} Te laudat Apostolórum chorus, Dómine.

\noindent Pro mensa verbi tui, nobis ab Apóstolis paráta, \gredagger{} qua lumen et gáudium nobis dantur:

\Rbardot{} Te laudat Apostolórum chorus, Dómine.

\noindent Pro Ecclésia tua sancta, super Apóstolos ædificáta, \gredagger{} qua in unum concorporámur:

\Rbardot{} Te laudat Apostolórum chorus, Dómine.

\noindent Pro lavácro baptísmi et pæniténtiæ, Apóstolis concrédito, \gredagger{} quo ab ómnibus peccátis ablúimur:

\Rbardot{} Te laudat Apostolórum chorus, Dómine.}
\newcommand{\benedictus}{\pars{Canticum Zachariæ.} \scriptura{\textbf{H359}}

\vspace{-4mm}

\antiphona{VII c}{temporalia/ant-salvecruxpretiosa.gtex}

%\vspace{-3mm}

\scriptura{Lc. 1, 68-79}

%\vspace{-1mm}

\cantusSineNeumas
\initiumpsalmi{temporalia/benedictus-initium-vii-c-auto.gtex}

\input{temporalia/benedictus-vii-c.tex} \Abardot{}}
\newcommand{\magnificat}{\pars{Canticum B. Mariæ V.} \scriptura{Is. 55, 6; \textbf{H20}}

\vspace{-4mm}

{
\grechangedim{interwordspacetext}{0.18 cm plus 0.15 cm minus 0.05 cm}{scalable}%
\antiphona{\textit{IV A}}{temporalia/ant-quaeritedominum.gtex}
\grechangedim{interwordspacetext}{0.22 cm plus 0.15 cm minus 0.05 cm}{scalable}%
}

%\vspace{-3mm}

\scriptura{Lc. 1, 46-55}

%\vspace{-2mm}

\cantusSineNeumas

\initiumpsalmi{temporalia/magnificat-initium-iv-A.gtex}

%\vspace{-2mm}

\input{temporalia/magnificat-iv-A.tex} \Abardot{}

\vspace{-1cm}}
\newcommand{\benedicamuslaudes}{\cuminitiali{II}{temporalia/benedicamus-solemnism-laud.gtex}}
\newcommand{\hebdomada}{infra Hebdom. I per Annum.}
\newcommand{\matua}{Matutinum Hebdomadae A}
\newcommand{\matuac}{Matutinum Hebdomadae A vel C}
\newcommand{\lauda}{Laudes Hebdomadae A}
\newcommand{\laudac}{Laudes Hebdomadae A vel C}

% LuaLaTeX

\documentclass[a4paper, twoside, 12pt]{article}
\usepackage[latin]{babel}
%\usepackage[landscape, left=3cm, right=1.5cm, top=2cm, bottom=1cm]{geometry} % okraje stranky
%\usepackage[landscape, a4paper, mag=1166, truedimen, left=2cm, right=1.5cm, top=1.6cm, bottom=0.95cm]{geometry} % okraje stranky
\usepackage[landscape, a4paper, mag=1400, truedimen, left=0.5cm, right=0.5cm, top=0.5cm, bottom=0.5cm]{geometry} % okraje stranky

\usepackage{fontspec}
\setmainfont[FeatureFile={junicode.fea}, Ligatures={Common, TeX}, RawFeature=+fixi]{Junicode}
%\setmainfont{Junicode}

% shortcut for Junicode without ligatures (for the Czech texts)
\newfontfamily\nlfont[FeatureFile={junicode.fea}, Ligatures={Common, TeX}, RawFeature=+fixi]{Junicode}

\usepackage{multicol}
\usepackage{color}
\usepackage{lettrine}
\usepackage{fancyhdr}

% usual packages loading:
\usepackage{luatextra}
\usepackage{graphicx} % support the \includegraphics command and options
\usepackage{gregoriotex} % for gregorio score inclusion
\usepackage{gregoriosyms}
\usepackage{wrapfig} % figures wrapped by the text
\usepackage{parcolumns}
\usepackage[contents={},opacity=1,scale=1,color=black]{background}
\usepackage{tikzpagenodes}
\usepackage{calc}
\usepackage{longtable}
\usetikzlibrary{calc}

\setlength{\headheight}{14.5pt}

% Commands used to produce a typical "Conventus" booklet

\newenvironment{titulusOfficii}{\begin{center}}{\end{center}}
\newcommand{\dies}[1]{#1

}
\newcommand{\nomenFesti}[1]{\textbf{\Large #1}

}
\newcommand{\celebratio}[1]{#1

}

\newcommand{\hora}[1]{%
\vspace{0.5cm}{\large \textbf{#1}}

\fancyhead[LE]{\thepage\ / #1}
\fancyhead[RO]{#1 / \thepage}
\addcontentsline{toc}{subsection}{#1}
}

% larger unit than a hora
\newcommand{\divisio}[1]{%
\begin{center}
{\Large \textsc{#1}}
\end{center}
\fancyhead[CO,CE]{#1}
\addcontentsline{toc}{section}{#1}
}

% a part of a hora, larger than pars
\newcommand{\subhora}[1]{
\begin{center}
{\large \textit{#1}}
\end{center}
%\fancyhead[CO,CE]{#1}
\addcontentsline{toc}{subsubsection}{#1}
}

% rubricated inline text
\newcommand{\rubricatum}[1]{\textit{#1}}

% standalone rubric
\newcommand{\rubrica}[1]{\vspace{3mm}\rubricatum{#1}}

\newcommand{\notitia}[1]{\textcolor{red}{#1}}

\newcommand{\scriptura}[1]{\hfill \small\textit{#1}}

\newcommand{\translatioCantus}[1]{\vspace{1mm}%
{\noindent\footnotesize \nlfont{#1}}}

% pruznejsi varianta nasledujiciho - umoznuje nastavit sirku sloupce
% s prekladem
\newcommand{\psalmusEtTranslatioB}[3]{
  \vspace{0.5cm}
  \begin{parcolumns}[colwidths={2=#3}, nofirstindent=true]{2}
    \colchunk{
      \input{#1}
    }

    \colchunk{
      \vspace{-0.5cm}
      {\footnotesize \nlfont
        \input{#2}
      }
    }
  \end{parcolumns}
}

\newcommand{\psalmusEtTranslatio}[2]{
  \psalmusEtTranslatioB{#1}{#2}{8.5cm}
}


\newcommand{\canticumMagnificatEtTranslatio}[1]{
  \psalmusEtTranslatioB{#1}{temporalia/extra-adventum-vespers/magnificat-boh.tex}{12cm}
}
\newcommand{\canticumBenedictusEtTranslatio}[1]{
  \psalmusEtTranslatioB{#1}{temporalia/extra-adventum-laudes/benedictus-boh.tex}{10.5cm}
}

% volne misto nad antifonami, kam si zpevaci dokresli neumy
\newcommand{\hicSuntNeumae}{\vspace{0.5cm}}

% prepinani mista mezi notovymi osnovami: pro neumovane a neneumovane zpevy
\newcommand{\cantusCumNeumis}{
  \setgrefactor{17}
  \global\advance\grespaceabovelines by 5mm%
}
\newcommand{\cantusSineNeumas}{
  \setgrefactor{17}
  \global\advance\grespaceabovelines by -5mm%
}

% znaky k umisteni nad inicialu zpevu
\newcommand{\superInitialam}[1]{\gresetfirstlineaboveinitial{\small {\textbf{#1}}}{\small {\textbf{#1}}}}

% pars officii, i.e. "oratio", ...
\newcommand{\pars}[1]{\textbf{#1}}

\newenvironment{psalmus}{
  \setlength{\parindent}{0pt}
  \setlength{\parskip}{5pt}
}{
  \setlength{\parindent}{10pt}
  \setlength{\parskip}{10pt}
}

%%%% Prejmenovat na latinske:
\newcommand{\nadpisZalmu}[1]{
  \hspace{2cm}\textbf{#1}\vspace{2mm}%
  \nopagebreak%

}

% mode, score, translation
\newcommand{\antiphona}[3]{%
\hicSuntNeumae
\superInitialam{#1}
\includescore{#2}

#3
}
 % Often used macros

\newcommand{\annusEditionis}{2021}

%%%% Vicekrat opakovane kousky

\newcommand{\anteOrationem}{
  \rubrica{Ante Orationem, cantatur a Superiore:}

  \pars{Supplicatio Litaniæ.}

  \cuminitiali{}{temporalia/supplicatiolitaniae.gtex}

  \pars{Oratio Dominica.}

  \cuminitiali{}{temporalia/oratiodominica.gtex}

  \rubrica{Deinde dicitur ab Hebdomadario:}

  \cuminitiali{}{temporalia/dominusvobiscum-solemnis.gtex}

  \rubrica{In choro monialium loco Dominus vobiscum dicitur:}

  \sineinitiali{temporalia/domineexaudi.gtex}
}

\setlength{\columnsep}{30pt} % prostor mezi sloupci

%%%%%%%%%%%%%%%%%%%%%%%%%%%%%%%%%%%%%%%%%%%%%%%%%%%%%%%%%%%%%%%%%%%%%%%%%%%%%%%%%%%%%%%%%%%%%%%%%%%%%%%%%%%%%
\begin{document}

% Here we set the space around the initial.
% Please report to http://home.gna.org/gregorio/gregoriotex/details for more details and options
\grechangedim{afterinitialshift}{2.2mm}{scalable}
\grechangedim{beforeinitialshift}{2.2mm}{scalable}
\grechangedim{interwordspacetext}{0.22 cm plus 0.15 cm minus 0.05 cm}{scalable}%
\grechangedim{annotationraise}{-0.2cm}{scalable}

% Here we set the initial font. Change 38 if you want a bigger initial.
% Emit the initials in red.
\grechangestyle{initial}{\color{red}\fontsize{38}{38}\selectfont}

\pagestyle{empty}

%%%% Titulni stranka
\begin{titulusOfficii}
\ifx\titulus\undefined
\nomenFesti{Feria III \hebdomada{}}
\else
\titulus
\fi
\end{titulusOfficii}

\vfill

\begin{center}
%Ad usum et secundum consuetudines chori \guillemotright{}Conventus Choralis\guillemotleft.

%Editio Sancti Wolfgangi \annusEditionis
\end{center}

\scriptura{}

\pars{}

\pagebreak

\renewcommand{\headrulewidth}{0pt} % no horiz. rule at the header
\fancyhf{}
\pagestyle{fancy}

\cantusSineNeumas

\hora{Ad Matutinum.} %%%%%%%%%%%%%%%%%%%%%%%%%%%%%%%%%%%%%%%%%%%%%%%%%%%%%

\vspace{2mm}

\cuminitiali{}{temporalia/dominelabiamea.gtex}

\vfill
%\pagebreak

\vspace{2mm}

\ifx\invitatorium\undefined
\pars{Invitatorium.} \scriptura{Lc. 24, 34; Psalmus 94; \textbf{H232}}

\vspace{-6mm}

\antiphona{VI}{temporalia/inv-surrexitdominusvere.gtex}
\else
\invitatorium
\fi

\vfill
\pagebreak

\ifx\hymnusmatutinum\undefined
\pars{Hymnus}

\cuminitiali{VIII}{temporalia/hym-LaetareCaelum.gtex}
\else
\hymnusmatutinum
\fi

\vspace{-3mm}

\vfill
\pagebreak

\ifx\matutinum\undefined
\ifx\matua\undefined
\else
% MAT A
\pars{Psalmus 1.}

\vspace{-4mm}

\antiphona{II D}{temporalia/ant-alleluia-turco7.gtex}

%\vspace{-2mm}

\scriptura{Ps. 9, 22-32}

%\vspace{-2mm}

\initiumpsalmi{temporalia/ps9xxii_xxxii-initium-ii-D-auto.gtex}

\input{temporalia/ps9xxii_xxxii-ii-D.tex}

\vfill
\pagebreak

\pars{Psalmus 2.} \scriptura{Ps. 9, 33-39}

%\vspace{-2mm}

\initiumpsalmi{temporalia/ps9xxxiii_xxxix-initium-ii-D-auto.gtex}

\input{temporalia/ps9xxxiii_xxxix-ii-D.tex}

\vfill
\pagebreak

\pars{Psalmus 3.} \scriptura{Ps. 11}

%\vspace{-2mm}

\initiumpsalmi{temporalia/ps11-initium-ii-D-auto.gtex}

\input{temporalia/ps11-ii-D.tex}

\vfill

\antiphona{}{temporalia/ant-alleluia-turco7.gtex}

\vfill
\pagebreak
\fi
\ifx\matub\undefined
\else
% MAT B
\pars{Psalmus 1.}

\vspace{-4mm}

\antiphona{VI F}{temporalia/ant-alleluia-turco6.gtex}

%\vspace{-2mm}

\scriptura{Ps. 36, 1-11}

%\vspace{-2mm}

\initiumpsalmi{temporalia/ps36i_xi-initium-vi-F-auto.gtex}

\input{temporalia/ps36i_xi-vi-F.tex}

\vfill
\pagebreak

\pars{Psalmus 2.}

\scriptura{Ps. 36, 12-29}

\vspace{-2mm}

\initiumpsalmi{temporalia/ps36xii_xxix-initium-vi-F-auto.gtex}

\input{temporalia/ps36xii_xxix-vi-F.tex}

\vfill
\pagebreak

\pars{Psalmus 3.}

\scriptura{Ps. 36, 30-40}

%\vspace{-2mm}

\initiumpsalmi{temporalia/ps36iii-initium-vi-F-auto.gtex}

\input{temporalia/ps36iii-vi-F.tex}

\antiphona{}{temporalia/ant-alleluia-turco6.gtex}

\vfill
\pagebreak
\fi
\ifx\matuc\undefined
\else
% MAT C
\pars{Psalmus 1.}

\vspace{-4mm}

\antiphona{I g\textsuperscript{5}}{temporalia/ant-alleluia-auglx2.gtex}

%\vspace{-2mm}

\scriptura{Ps. 67, 2-11}

\initiumpsalmi{temporalia/ps67i-initium-i-g5.gtex}

\input{temporalia/ps67i-i-g.tex}

\vfill
\pagebreak

\pars{Psalmus 2.}

\scriptura{Ps. 67, 12-24}

%\vspace{-2mm}

\initiumpsalmi{temporalia/ps67ii-initium-i-g5.gtex}

\input{temporalia/ps67ii-i-g.tex}

\vfill
\pagebreak

\pars{Psalmus 3.}

\scriptura{Ps. 67, 25-36}

\initiumpsalmi{temporalia/ps67iii-initium-i-g5.gtex}

\input{temporalia/ps67iii-i-g.tex}

\vfill

\antiphona{}{temporalia/ant-alleluia-auglx2.gtex}

\vfill
\pagebreak
\fi
\ifx\matud\undefined
\else
% MAT D
\pars{Psalmus 1.}

\vspace{-4mm}

\antiphona{I d\textsuperscript{3}}{temporalia/ant-alleluia-auglx6.gtex}

%\vspace{-2mm}

\scriptura{Ps. 101, 2-12}

%\vspace{-2mm}

\initiumpsalmi{temporalia/ps101ii_xii-initium-i-d3-auto.gtex}

\input{temporalia/ps101ii_xii-i-d3.tex}

\vfill
\pagebreak

\pars{Psalmus 2.} \scriptura{Ps. 101, 13-23}

\vspace{-2mm}

\initiumpsalmi{temporalia/ps101xiii_xxiii-initium-i-d3-auto.gtex}

\input{temporalia/ps101xiii_xxiii-i-d3.tex}

\vfill
\pagebreak

\pars{Psalmus 3.} \scriptura{Ps. 101, 24-29}

%\vspace{-2mm}

\initiumpsalmi{temporalia/ps101iii-initium-i-d3-auto.gtex}

\input{temporalia/ps101iii-i-d3.tex}

\vfill

\antiphona{}{temporalia/ant-alleluia-auglx6.gtex}

\vfill
\pagebreak
\fi
\else
\matutinum
\fi

\pars{Versus.}

\ifx\matversus\undefined
\noindent \Vbardot{} Christus resúrgens ex mórtuis iam non móritur, allelúia.

\noindent \Rbardot{} Mors illi ultra non dominábitur, allelúia.
\else
\matversus
\fi

\vspace{5mm}

\sineinitiali{temporalia/oratiodominica-mat.gtex}

\vspace{5mm}

\pars{Absolutio.}

\cuminitiali{}{temporalia/absolutio-ipsius.gtex}

\vfill
\pagebreak

\cuminitiali{}{temporalia/benedictio-solemn-deus.gtex}

\vspace{7mm}

\lectioi

\noindent \Vbardot{} Tu autem, Dómine, miserére nobis.
\noindent \Rbardot{} Deo grátias.

\vfill
\pagebreak

\responsoriumi

\vfill
\pagebreak

\cuminitiali{}{temporalia/benedictio-solemn-christus.gtex}

\vspace{7mm}

\lectioii

\noindent \Vbardot{} Tu autem, Dómine, miserére nobis.
\noindent \Rbardot{} Deo grátias.

\vfill
\pagebreak

\responsoriumii

\vfill
\pagebreak

\cuminitiali{}{temporalia/benedictio-solemn-ignem.gtex}

\vspace{7mm}

\lectioiii

\noindent \Vbardot{} Tu autem, Dómine, miserére nobis.
\noindent \Rbardot{} Deo grátias.

\vfill
\pagebreak

\responsoriumiii

\vfill
\pagebreak

\rubrica{Reliqua omittuntur, nisi Laudes separandæ sint.}

\sineinitiali{temporalia/domineexaudi.gtex}

\vfill

\oratio

\vfill

\noindent \Vbardot{} Dómine, exáudi oratiónem meam.
\Rbardot{} Et clamor meus ad te véniat.

\vfill

\noindent \Vbardot{} Benedicámus Dómino.
\noindent \Rbardot{} Deo grátias.

\vfill

\noindent \Vbardot{} Fidélium ánimæ per misericórdiam Dei requiéscant in pace.
\Rbardot{} Amen.

\vfill
\pagebreak

\hora{Ad Laudes.} %%%%%%%%%%%%%%%%%%%%%%%%%%%%%%%%%%%%%%%%%%%%%%%%%%%%%

\cantusSineNeumas

\vspace{0.5cm}
\grechangedim{interwordspacetext}{0.18 cm plus 0.15 cm minus 0.05 cm}{scalable}%
\cuminitiali{}{temporalia/deusinadiutorium-communis.gtex}
\grechangedim{interwordspacetext}{0.22 cm plus 0.15 cm minus 0.05 cm}{scalable}%

\vfill
\pagebreak

\ifx\hymnuslaudes\undefined
\ifx\laudac\undefined
\else
\pars{Hymnus}

\cuminitiali{I}{temporalia/hym-ChorusNovae-praglia.gtex}
\fi
\ifx\laudbd\undefined
\else
\pars{Hymnus}

\cuminitiali{I}{temporalia/hym-ChorusNovae.gtex}
\fi
\else
\hymnuslaudes
\fi

\vfill
\pagebreak

\ifx\laudes\undefined
\ifx\lauda\undefined
\else
\pars{Psalmus 1.}

\vspace{-4mm}

\antiphona{IV* e}{temporalia/ant-alleluia-turco9.gtex}

\scriptura{Psalmus 23.}

\initiumpsalmi{temporalia/ps23-initium-iv_-e-auto.gtex}

\input{temporalia/ps23-iv_-e.tex} \Abardot{}

\vfill
\pagebreak

\pars{Psalmus 2.} \scriptura{Tob. 13, 10}

\vspace{-4mm}

\antiphona{VIII G}{temporalia/ant-benedicitedominumomneselecti.gtex}

\scriptura{Canticum Tobiæ, Tob. 13, 2-8}

\initiumpsalmi{temporalia/tobiae-initium-viii-g-auto.gtex}

\input{temporalia/tobiae-viii-g.tex} \Abardot{}

\vfill
\pagebreak

\pars{Psalmus 3.}

\vspace{-4mm}

\antiphona{E}{temporalia/ant-alleluia-praglia-e2.gtex}

%\vspace{-4mm}

\scriptura{Psalmus 32.}

%\vspace{-2mm}

\initiumpsalmi{temporalia/ps32-initium-e-auto.gtex}

\input{temporalia/ps32-e.tex}

\vfill

\antiphona{}{temporalia/ant-alleluia-praglia-e2.gtex}

\vfill
\pagebreak
\fi
\ifx\laudb\undefined
\else
\pars{Psalmus 1.}

\vspace{-4mm}

\antiphona{E}{temporalia/ant-alleluia-praglia-e.gtex}

\scriptura{Psalmus 42.}

\initiumpsalmi{temporalia/ps42-initium-e-e-auto.gtex}

\input{temporalia/ps42-e-e.tex} \Abardot{}

\vfill
\pagebreak

\pars{Psalmus 2.} \scriptura{Is. 38, 17}

\vspace{-4mm}

\antiphona{I g}{temporalia/ant-eruistidomine-tp.gtex}

%\vspace{-2mm}

\scriptura{Canticum Ezechiæ, Is. 38, 10-20}

%\vspace{-2mm}

\initiumpsalmi{temporalia/ezechiae-initium-i-g-auto.gtex}

%\vspace{-1.5mm}

\input{temporalia/ezechiae-i-g.tex}

\vfill

\antiphona{}{temporalia/ant-eruistidomine-tp.gtex}

\vfill
\pagebreak

\pars{Psalmus 3.}

\vspace{-4mm}

\antiphona{VIII c}{temporalia/ant-alleluia-turco16.gtex}

\vspace{-2mm}

\scriptura{Psalmus 64.}

\vspace{-2mm}

\initiumpsalmi{temporalia/ps64-initium-viii-C-auto.gtex}

\input{temporalia/ps64-viii-C.tex} \Abardot{}

\vfill
\pagebreak
\fi
\ifx\laudc\undefined
\else
\pars{Psalmus 1.}

\vspace{-4mm}

\antiphona{VI F}{temporalia/ant-alleluia-turco5.gtex}

\vspace{-2mm}

\scriptura{Psalmus 84.}

\vspace{-2mm}

\initiumpsalmi{temporalia/ps84-initium-vi-F-auto.gtex}

\input{temporalia/ps84-vi-F.tex} \Abardot{}

\vfill
\pagebreak

\pars{Psalmus 2.}

\vspace{-4mm}

\antiphona{VII d}{temporalia/ant-denoctespiritusmeus-tp.gtex}

\vspace{-2mm}

\scriptura{Canticum Isaiæ, Is. 26, 1-12}

\vspace{-2mm}

\initiumpsalmi{temporalia/isaiae3-initium-vii-d.gtex}

\input{temporalia/isaiae3-vii-d.tex} \Abardot{}

\vfill
\pagebreak

\pars{Psalmus 3.}

\vspace{-4mm}

\antiphona{E}{temporalia/ant-alleluia-praglia-e2.gtex}

%\vspace{-2mm}

\scriptura{Psalmus 66.}

%\vspace{-2mm}

\initiumpsalmi{temporalia/ps66-initium-e-auto.gtex}

\input{temporalia/ps66-e.tex} \Abardot{}

\vfill
\pagebreak
\fi
\ifx\laudd\undefined
\else
\pars{Psalmus 1.}

\vspace{-4mm}

\antiphona{VIII G}{temporalia/ant-alleluia-turco12.gtex}

\vspace{-2mm}

\scriptura{Psalmus 100.}

\vspace{-2mm}

\initiumpsalmi{temporalia/ps100-initium-viii-G-auto.gtex}

\input{temporalia/ps100-viii-G.tex} \Abardot{}

\vfill
\pagebreak

\pars{Psalmus 2.} \scriptura{Ps. 50, 19}

\vspace{-4mm}

\antiphona{I f}{temporalia/ant-sacrificiumdeo-tp.gtex}

%\vspace{-2mm}

\scriptura{Canticum Danielis, Dan. 3, 26.27.29.34-41}

%\vspace{-2mm}

\initiumpsalmi{temporalia/dan32-initium-i-f-auto.gtex}

\input{temporalia/dan32-i-f.tex} \Abardot{}

\vfill
\pagebreak

\pars{Psalmus 3.}

\vspace{-4mm}

\antiphona{VI F}{temporalia/ant-alleluia-turco5.gtex}

%\vspace{-2mm}

\scriptura{Psalmus 143, 1-10.}

%\vspace{-2mm}

\initiumpsalmi{temporalia/ps143i_x-initium-vi-F-auto.gtex}

\input{temporalia/ps143i_x-vi-F.tex} \Abardot{}

\vfill
\pagebreak
\fi
\else
\laudes
\fi

\ifx\lectiobrevis\undefined
\pars{Lectio Brevis.} \scriptura{Ac. 13, 30-33}

\noindent Deus suscitávit Iesum a mórtuis; qui visus est per dies multos his, qui simul ascénderant cum eo de Galilǽa in Ierúsalem, qui nunc sunt testes eius ad plebem. Et nos vobis evangelizámus eam, quæ ad patres promíssio facta est, quóniam hanc Deus adimplévit fíliis eórum, nobis resúscitans Iesum, sicut et in Psalmo secúndo scriptum est: Fílius meus es tu; ego hódie génui te.
\else
\lectiobrevis
\fi

\vfill

\ifx\responsoriumbreve\undefined
\pars{Responsorium breve.} \scriptura{Cf. Mt. 28, 6; Cf. Gal. 3, 13}

\cuminitiali{VI}{temporalia/resp-surrexitdominusdesepulcro.gtex}
\else
\responsoriumbreve
\fi

\vfill
\pagebreak

\benedictus

\vspace{-1cm}

\vfill
\pagebreak

\ifx\precestotum\undefined
\pars{Preces.}

\sineinitiali{}{temporalia/tonusprecum.gtex}

\ifx\preces\undefined
\ifx\lauda\undefined
\else
\noindent Exsultémus Christo, qui perémptum sui córporis templum sua virtúte restítuit,~\gredagger{} eíque supplicémus:

\Rbardot{} Fructus resurrectiónis tuæ, Dómine, nobis concéde.

\noindent Christe salvátor, qui in resurrectióne tua muliéribus et Apóstolis gáudium nuntiásti, totum orbem salvíficans,~\gredagger{} testes tuos nos éffice.

\Rbardot{} Fructus resurrectiónis tuæ, Dómine, nobis concéde.

\noindent Qui resurrectiónem ómnibus promisísti, qua ad vitam novam resurgerémus,~\gredagger{} Evangélii tui nos redde præcónes.

\Rbardot{} Fructus resurrectiónis tuæ, Dómine, nobis concéde.

\noindent Tu, qui Apóstolis sǽpius apparuísti et Sanctum eis Spíritum insufflásti,~\gredagger{} creatórem Spíritum rénova in nobis.

\Rbardot{} Fructus resurrectiónis tuæ, Dómine, nobis concéde.

\noindent Tu, qui discípulis tuis promisísti te cum eis mansúrum usque ad consummatiónem sǽculi,~\gredagger{} mane nobíscum hódie sempérque nobis adésto.

\Rbardot{} Fructus resurrectiónis tuæ, Dómine, nobis concéde.
\fi
\ifx\laudb\undefined
\else
\noindent Deum Patrem, cuius Agnus immaculátus tollit peccáta mundi nosque vivíficat,~\gredagger{} grati rogémus:

\Rbardot{} Auctor vitæ, vivífica nos.

\noindent Deus, auctor vitæ, meménto passiónis et resurrectiónis Agni, in cruce occísi,~\gredagger{} eúmque audi, semper interpellántem pro nobis.

\Rbardot{} Auctor vitæ, vivífica nos.

\noindent Expurgáto vétere ferménto malítiæ et nequítiæ,~\gredagger{} fac nos vívere in ázymis sinceritátis et veritátis Christi.

\Rbardot{} Auctor vitæ, vivífica nos.

\noindent Da, ut hódie reiciámus peccátum discórdiæ atque invídiæ,~\gredagger{} nosque redde fratrum necessitátibus magis inténtos.

\Rbardot{} Auctor vitæ, vivífica nos.

\noindent Spíritum evangélicum pone in médio nostri,~\gredagger{} ut hódie et semper in præcéptis tuis ambulémus.

\Rbardot{} Auctor vitæ, vivífica nos.
\fi
\ifx\laudc\undefined
\else
\noindent Exsultémus Christo, qui perémptum sui córporis templum sua virtúte restítuit,~\gredagger{} eíque supplicémus:

\Rbardot{} Fructus resurrectiónis tuæ, Dómine, nobis concéde.

\noindent Christe salvátor, qui in resurrectióne tua muliéribus et Apóstolis gáudium nuntiásti, totum orbem salvíficans,~\gredagger{} testes tuos nos éffice.

\Rbardot{} Fructus resurrectiónis tuæ, Dómine, nobis concéde.

\noindent Qui resurrectiónem ómnibus promisísti, qua ad vitam novam resurgerémus,~\gredagger{} Evangélii tui nos redde præcónes.

\Rbardot{} Fructus resurrectiónis tuæ, Dómine, nobis concéde.

\noindent Tu, qui Apóstolis sǽpius apparuísti et Sanctum eis Spíritum insufflásti,~\gredagger{} creatórem Spíritum rénova in nobis.

\Rbardot{} Fructus resurrectiónis tuæ, Dómine, nobis concéde.

\noindent Tu, qui discípulis tuis promisísti te cum eis mansúrum usque ad consummatiónem sǽculi,~\gredagger{} mane nobíscum hódie sempérque nobis adésto.

\Rbardot{} Fructus resurrectiónis tuæ, Dómine, nobis concéde.
\fi
\ifx\laudd\undefined
\else
\noindent Deum Patrem, cuius Agnus immaculátus tollit peccáta mundi nosque vivíficat,~\gredagger{} grati rogémus:

\Rbardot{} Auctor vitæ, vivífica nos.

\noindent Deus, auctor vitæ, meménto passiónis et resurrectiónis Agni, in cruce occísi,~\gredagger{} eúmque audi, semper interpellántem pro nobis.

\Rbardot{} Auctor vitæ, vivífica nos.

\noindent Expurgáto vétere ferménto malítiæ et nequítiæ,~\gredagger{} fac nos vívere in ázymis sinceritátis et veritátis Christi.

\Rbardot{} Auctor vitæ, vivífica nos.

\noindent Da, ut hódie reiciámus peccátum discórdiæ atque invídiæ,~\gredagger{} nosque redde fratrum necessitátibus magis inténtos.

\Rbardot{} Auctor vitæ, vivífica nos.

\noindent Spíritum evangélicum pone in médio nostri,~\gredagger{} ut hódie et semper in præcéptis tuis ambulémus.

\Rbardot{} Auctor vitæ, vivífica nos.
\fi
\else
\preces
\fi

\vfill

\pars{Oratio Dominica.}

\cuminitiali{}{temporalia/oratiodominicaalt.gtex}

\vfill
\pagebreak

\rubrica{vel:}

\pars{Supplicatio Litaniæ.}

\cuminitiali{}{temporalia/supplicatiolitaniae.gtex}

\vfill

\pars{Oratio Dominica.}

\cuminitiali{}{temporalia/oratiodominica.gtex}
\else
\precestotum
\fi

\vfill
\pagebreak

% Oratio. %%%
\oratio

\vspace{-1mm}

\vfill

\rubrica{Hebdomadarius dicit Dominus vobiscum, vel, absente sacerdote vel diacono, sic concluditur:}

\vspace{2mm}

\ifx\dominusnosbenedicat\undefined
\antiphona{C}{temporalia/dominusnosbenedicat.gtex}
\else
\dominusnosbenedicat
\fi

\rubrica{Postea cantatur a cantore:}

\vspace{2mm}

\ifx\benedicamuslaudes\undefined
\cuminitiali{VII}{temporalia/benedicamus-tempore-paschali.gtex}
\else
\benedicamuslaudes
\fi

\vspace{1mm}

\vfill
\pagebreak

\end{document}

