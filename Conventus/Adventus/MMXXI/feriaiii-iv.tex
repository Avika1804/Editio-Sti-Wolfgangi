\newcommand{\oratio}{\pars{Oratio.}

\noindent Preces pópuli tui, quǽsumus, Dómine, cleménter exáudi, ut qui de Unigéniti tui in nostra carne advéntu lætántur, cum vénerit in sua maiestáte, ætérnæ vitæ prǽmium consequántur.

\pars{Pro commemoratione Sancti Petri Canisii, Presbyteri \& Ecclesiæ Doctoris.} \scriptura{Dn. 12, 3}

\vspace{-4mm}

\antiphona{VII a trans.}{temporalia/ant-quidoctifuerint.gtex}

\vfill

\noindent Deus, qui ad tuéndam cathólicam fidem virtúte et doctrína beátum Petrum, presbýterum, roborásti, eius intercessióne concéde, ut, qui veritátem quærunt, te Deum gaudénter invéniant, et in tua confessióne pópulus credéntium persevéret.

\noindent Per Dóminum nostrum Iesum Christum, Fílium tuum, qui tecum vivit et regnat in unitáte Spíritus Sancti, Deus, per ómnia sǽcula sæculórum.

\noindent \Rbardot{} Amen.}
\newcommand{\matversus}{\noindent \Vbardot{} Dómine Deus noster, convérte nos.

\noindent \Rbardot{} Et osténde fáciem tuam et salvi érimus.}
\newcommand{\lectioi}{\pars{Lectio I.} \scriptura{Is. 48, 12-21; 49, 9b-13}

\noindent De libro Isaíæ prophétæ.

\noindent Hæc dicit Dóminus: «Audi me, Iacob, et, Israel, quem ego vocávi; ego, ego primus et ego novíssimus. Manus mea fundávit terram, et déxtera mea expándit cælos; ego voco eos, et stant simul. Congregámini, omnes vos et audíte: Quis de eis annuntiávit hæc? Dóminus diléxit eum; fáciet voluntátem suam in Babylóne et bráchium suum in Chaldǽis.Ego, ego locútus sum et vocávi eum; addúxi eum, et próspera fuit via eius. Accédite ad me et audíte hoc: Non a princípio in abscóndito locútus sum; ex témpore, ántequam fíeret, ibi eram; et nunc Dóminus Deus misit me cum spíritu suo».

\noindent Hæc dicit Dóminus, redémptor tuus, Sanctus Israel: «Ego Dóminus Deus tuus docens te utília, gubérnans te in via, qua ámbulas. Utinam attendísses mandáta mea! Facta fuísset sicut flumen pax tua, et iustítia tua sicut gúrgites maris; et fuísset quasi aréna semen tuum, et stirps úteri tui ut lapílli eius; non interísset et non fuísset attrítum nomen eius a fácie mea. Egredímini de Babylóne, fúgite a Chaldǽis, in voce exsultatiónis annuntiáte; audítum fácite hoc, efférte illud usque ad extréma terræ, dícite: “Redémit Dóminus servum suum Iacob”. Non sitiérunt, cum per desértum dúceret eos; aquam de petra prodúxit eis et scidit petram, et fluxérunt aquæ.

\noindent Super vias pascéntur, et in ómnibus cóllibus decalvátis páscua eórum; non esúrient neque sítient, et non percútiet eos æstus vel sol, quia miserátor eórum reget eos et ad fontes aquárum addúcet eos. Et ponam omnes montes meos in viam, et sémitæ meæ exaltabúntur. Ecce isti de longe vénient, et ecce illi ab aquilóne et mari, et isti de terra Sinim». Laudáte, cæli, et exsúlta, terra; iubiláte, montes, laudem, quia consolátur Dóminus pópulum suum et páuperum suórum miserétur.}
\newcommand{\responsoriumi}{\pars{Responsorium 1.} \scriptura{\Rbar{} Rom. 15, 12; Cf. Ps. 71, 17 \Vbar{} Ps. 49, 2-3; \textbf{H36}}

\vspace{-5mm}

\responsorium{VIII}{temporalia/resp-radixjesse-CROCHU.gtex}{}}
\newcommand{\lectioii}{\pars{Lectio II.} \scriptura{Lib. 2, 19. 22-23. 26-27: CCL 14, 39-42}

\noindent Ex Expositióne sancti Ambrósii epíscopi in Lucam.

\noindent Angelus, cum abscóndita nuntiáret, ut fides astruerétur exémplo, senióris féminæ sterilísque concéptum Vírgini Maríæ nuntiávit, ut possíbile Deo omne quod ei placúerit asséreret. Ubi audívit hoc María, non quasi incrédula de oráculo nec quasi incérta de núntio nec quasi dúbitans de exémplo, sed quasi læta pro voto, religiósa pro offício, festína pro gáudio in montána perréxit. Quo enim iam Deo plena nisi ad superióra cum festinatióne conténderet? Nescit tarda molímina Sancti Spíritus grátia. Cito quoque advéntus Maríæ et præséntiæ domínicæ benefícia declarántur; \emph{simul enim ut audívit salutatiónem Maríæ Elísabeth, exsultávit infans in útero eius et repléta est Spíritu Sancto.}

\noindent Vide distinctiónem singulorúmque verbórum proprietátes. Vocem prior Elísabeth audívit, sed Ioánnes prior grátiam sensit: illa natúræ órdine audívit, iste exsultávit ratióne mystérii; illa Maríæ, iste Dómini sensit advéntum, fémina mulíeris et pignus pígnoris; istæ grátiam loquúntur, illi intus operántur pietatísque mystérium matérnis adoriúntur proféctibus duplicíque miráculo prophétant matres spíritu parvulórum.}
\newcommand{\responsoriumii}{\pars{Responsorium 2.} \scriptura{\Rbar{} Is. 16, 1 \Vbar{} Ps. 49, 2; \textbf{H35}}

\vspace{-5mm}

\responsorium{II}{temporalia/resp-emitteagnumdomine-CROCHU.gtex}{}}
\newcommand{\lectioiii}{\pars{Lectio III.}

\noindent Exsultávit infans, repléta mater est. Non prius mater repléta quam fílius, sed, cum fílius esset replétus Spíritu Sancto, replévit et matrem. Exsultávit Ioánnes, \emph{exsultávit} et Maríæ \emph{spíritus.} Exsultánte Ioánne replétur Elísabeth, Maríam tamen non repléri Spíritu, sed spíritum eius exsultáre cognóvimus —incomprehensíbilis enim incomprehensibíliter operabátur in matre— et illa post concéptum replétur, ista ante concéptum. \emph{Beáta,} inquit, \emph{quæ credidísti.} Sed et vos beáti, qui audístis et credidístis; quæcúmque enim credíderit ánima, et cóncipit et génerat Dei Verbum et ópera eius agnóscit. Sit in síngulis Maríæ ánima, ut magníficet Dóminum; sit in síngulis spíritus Maríæ, ut exsúltet in Deo; si secúndum carnem una mater est Christi, secúndum fidem tamen ómnium fructus est Christus; omnis enim ánima áccipit Dei Verbum, si tamen immaculáta et immúnis a vítiis intemeráto castimóniam pudóre custódiat. Quæcúmque ígitur talis esse potúerit \emph{ánima magníficat Dóminum,} sicut ánima Maríæ magnificávit Dóminum \emph{et exsultávit spíritus eius in Deo salutári.}

\noindent Magnificátur enim Dóminus, sicut et álibi legístis: \emph{Magnificáte Dóminum mecum,} non quo Dómino áliquid humána voce possit adiúngi, sed quia magnificátur in nobis: imágo enim Dei Christus est, et ídeo, si quid iustum religiosúmque fécerit ánima, illam imáginem Dei, ad cuius est similitúdinem creáta, magníficat, et ídeo, dum magníficat eam, magnitúdinis eius quadam participatióne sublímior fit.}
\newcommand{\responsoriumiii}{\pars{Responsorium 3.} \scriptura{\Rbar{} Cf. Is. 35, 2 \Vbar{} Is. 40, 10; \textbf{H35}}

\vspace{-5mm}

\responsorium{I}{temporalia/resp-germinaveruntcampi-CROCHU-cumdox.gtex}{}}
\newcommand{\laudes}{\pars{Psalmus 1.} \scriptura{\textbf{H37}}

\vspace{-4mm}

\antiphona{II* a}{temporalia/ant-egredieturdominus.gtex}

\vspace{-2mm}

\scriptura{Psalmus 100.}

\vspace{-2mm}

\initiumpsalmi{temporalia/ps100-initium-ii_-a-auto.gtex}

\input{temporalia/ps100-ii_-a.tex} \Abardot{}

\vfill
\pagebreak

\pars{Psalmus 2.} \scriptura{Is. 26, 1.2; \textbf{H24}}

\vspace{-4mm}

\antiphona{VII d}{temporalia/ant-urbsfortitudinisnostrae.gtex}

%\vspace{-2mm}

\scriptura{Canticum Danielis, Dan. 3, 26.27.29.34-41}

%\vspace{-2mm}

\initiumpsalmi{temporalia/dan32-initium-vii-d3-auto.gtex}

\input{temporalia/dan32-vii-d3.tex}

\vfill

\antiphona{}{temporalia/ant-urbsfortitudinisnostrae.gtex}

\vfill
\pagebreak

\pars{Psalmus 3.} \scriptura{Ps. 66, 3; \textbf{H37}}

\vspace{-4mm}

\antiphona{II* a}{temporalia/ant-utcognoscamus.gtex}

%\vspace{-4mm}

\scriptura{Psalmus 143, 1-10.}

%\vspace{-2mm}

\initiumpsalmi{temporalia/ps143i_x-initium-ii_-a-auto.gtex}

\input{temporalia/ps143i_x-ii_-a.tex} \Abardot{}


\vfill
\pagebreak}
\newcommand{\lectiobrevis}{\pars{Lectio Brevis.} \scriptura{Is. 7, 14-15}

\noindent Ecce, Virgo concípiet et páriet fílium et vocábit nomen eius Emmánuel; butýrum et mel cómedet, ut ipse sciat reprobáre malum et elígere bonum.}
\newcommand{\benedictus}{\pars{Canticum Zachariæ.} \scriptura{\textbf{H37}}

\vspace{-4mm}

{
\grechangedim{interwordspacetext}{0.18 cm plus 0.15 cm minus 0.05 cm}{scalable}%
\antiphona{VIII g}{temporalia/ant-nolitetimere.gtex}
\grechangedim{interwordspacetext}{0.22 cm plus 0.15 cm minus 0.05 cm}{scalable}%
}

%\vspace{-2mm}

\scriptura{Lc. 1, 68-79}

%\vspace{-2mm}

\cantusSineNeumas
\initiumpsalmi{temporalia/benedictus-initium-viii-g-auto.gtex}

%\vspace{-1.5mm}

\input{temporalia/benedictus-viii-g.tex} \Abardot{}}
\newcommand{\preces}{\noindent Dóminum nostrum Iesum Christum, fratres caríssimi, exorémus, qui in sua misericórdia nos vísitat, \gredagger{} lætámque vocem iterémus:

\Rbardot{} Veni, Dómine Iesu.

\noindent Qui de sinu Patris egréssus, venísti ut carnis nostræ vestiméntum indúeres, \gredagger{} líbera quod períerat natúræ vitiátæ contágio.

\Rbardot{} Veni, Dómine Iesu.

\noindent Qui ventúrus, in eléctis agnoscéris gloriósus, \gredagger{} nunc véniens, in peccatóribus clemens semper et pius inveniáris.

\Rbardot{} Veni, Dómine Iesu.

\noindent Gloriántes in laude tua, Christe Dómine, \gredagger{} vísita nos in salutári tuo.

\Rbardot{} Veni, Dómine Iesu.

\noindent Qui nos iam eduxísti in lucem per fidem, \gredagger{} fac nos iustítia tua pro dignis opéribus tibi plácitos.

\Rbardot{} Veni, Dómine Iesu.}
\newcommand{\vesperas}{\vspace{4mm}

\pars{Psalmus 1.} \scriptura{Is. 16, 1; \textbf{H37}}

\vspace{-4mm}

\antiphona{II* a}{temporalia/ant-emitteagnum.gtex}

\vspace{-4mm}

\scriptura{Psalmus 129.}

\initiumpsalmi{temporalia/ps129-initium-ii_-a-auto.gtex}

\input{temporalia/ps129-ii_-a.tex} \Abardot{}

\vspace{-1cm}

\vfill
\pagebreak

\pars{Psalmus 2.} \scriptura{\textbf{H38}}

\vspace{-4mm}

\antiphona{VIII G}{temporalia/ant-convertere.gtex}

\vspace{-4mm}

\scriptura{Psalmus 130.}

\initiumpsalmi{temporalia/ps130-initium-viii-g-auto.gtex}

\input{temporalia/ps130-viii-g.tex}

\vfill
\pagebreak

\pars{Psalmus 3.} \scriptura{Ps. 66, 3; \textbf{H37}}

\vspace{-4mm}

\antiphona{II* a}{temporalia/ant-utcognoscamus.gtex}

\vspace{-4mm}

\scriptura{Psalmus 131.}

\initiumpsalmi{temporalia/ps131-initium-ii_-a-auto.gtex}

\input{temporalia/ps131-ii_-a.tex}

\vfill

\antiphona{}{temporalia/ant-utcognoscamus.gtex}

\vfill
\pagebreak

\pars{Psalmus 4.} \scriptura{Is. 30, 18}

\vspace{-4mm}

\antiphona{I g}{temporalia/ant-deusiudicii.gtex}

\vspace{-4mm}

\scriptura{Psalmus 132.}

\initiumpsalmi{temporalia/ps132-initium-i-g-auto.gtex}

\input{temporalia/ps132-i-g.tex} \Abardot{}

\vfill
\pagebreak}
\newcommand{\magnificat}{\pars{Canticum B. Mariæ V.} \scriptura{Sap. 7, 26; Mal. 4, 2; Lc. 1, 78-79; \textbf{H40}}

\vspace{-5mm}

{
\grechangedim{interwordspacetext}{0.18 cm plus 0.15 cm minus 0.05 cm}{scalable}%
\antiphona{II D}{temporalia/ant-ooriens.gtex}
\grechangedim{interwordspacetext}{0.22 cm plus 0.15 cm minus 0.05 cm}{scalable}%
}

\vspace{-2mm}

\scriptura{Lc. 1, 46-55}

\vspace{-2mm}

\cantusSineNeumas

\initiumpsalmi{temporalia/magnificat-initium-iisoll-D.gtex}

\vspace{-1.5mm}

\input{temporalia/magnificat-iisoll-D.tex} \Abardot{}

\vspace{-1cm}}
\newcommand{\hebdomada}{infra Hebdom. IV per Annum.}
%\newcommand{\hiemalis}{Hiemalis}
\newcommand{\matud}{Matutinum Hebdomadae D}
\newcommand{\matubd}{Matutinum Hebdomadae B vel D}
\newcommand{\laudd}{Laudes Hebdomadae D}
\newcommand{\laudbd}{Laudes Hebdomadae B vel D}

\renewcommand{\hebdomada}{infra Hebdom. Ultima Adventus.}
\ifx\invitatorium\undefined
\newcommand{\invitatorium}{\pars{Invitatorium.} \scriptura{Phil. 4, 4.5}

\vspace{-6mm}

\antiphona{VI}{temporalia/inv-propeestiamsimplex.gtex}}
\fi
\ifx\hymnusmatutinum\undefined
\newcommand{\hymnusmatutinum}{\pars{Hymnus.}

\vspace{-5mm}

\antiphona{II}{temporalia/hym-VeniRedemptor.gtex}}
\fi
\ifx\hymnuslaudes\undefined
\newcommand{\hymnuslaudes}{\pars{Hymnus}

\cuminitiali{D}{temporalia/hym-MagnisProphetae.gtex}}
\fi
\ifx\hymnusvesperas\undefined
\newcommand{\hymnusvesperas}{\pars{Hymnus}

\cuminitiali{IV}{temporalia/hym-VerbumSalutis.gtex}}
\fi

% LuaLaTeX

\documentclass[a4paper, twoside, 12pt]{article}
\usepackage[latin]{babel}
%\usepackage[landscape, left=3cm, right=1.5cm, top=2cm, bottom=1cm]{geometry} % okraje stranky
%\usepackage[landscape, a4paper, mag=1166, truedimen, left=2cm, right=1.5cm, top=1.6cm, bottom=0.95cm]{geometry} % okraje stranky
\usepackage[landscape, a4paper, mag=1400, truedimen, left=0.5cm, right=0.5cm, top=0.5cm, bottom=0.5cm]{geometry} % okraje stranky

\usepackage{fontspec}
\setmainfont[FeatureFile={junicode.fea}, Ligatures={Common, TeX}, RawFeature=+fixi]{Junicode}
%\setmainfont{Junicode}

% shortcut for Junicode without ligatures (for the Czech texts)
\newfontfamily\nlfont[FeatureFile={junicode.fea}, Ligatures={Common, TeX}, RawFeature=+fixi]{Junicode}

\usepackage{multicol}
\usepackage{color}
\usepackage{lettrine}
\usepackage{fancyhdr}

% usual packages loading:
\usepackage{luatextra}
\usepackage{graphicx} % support the \includegraphics command and options
\usepackage{gregoriotex} % for gregorio score inclusion
\usepackage{gregoriosyms}
\usepackage{wrapfig} % figures wrapped by the text
\usepackage{parcolumns}
\usepackage[contents={},opacity=1,scale=1,color=black]{background}
\usepackage{tikzpagenodes}
\usepackage{calc}
\usepackage{longtable}
\usetikzlibrary{calc}

\setlength{\headheight}{14.5pt}

% Commands used to produce a typical "Conventus" booklet

\newenvironment{titulusOfficii}{\begin{center}}{\end{center}}
\newcommand{\dies}[1]{#1

}
\newcommand{\nomenFesti}[1]{\textbf{\Large #1}

}
\newcommand{\celebratio}[1]{#1

}

\newcommand{\hora}[1]{%
\vspace{0.5cm}{\large \textbf{#1}}

\fancyhead[LE]{\thepage\ / #1}
\fancyhead[RO]{#1 / \thepage}
\addcontentsline{toc}{subsection}{#1}
}

% larger unit than a hora
\newcommand{\divisio}[1]{%
\begin{center}
{\Large \textsc{#1}}
\end{center}
\fancyhead[CO,CE]{#1}
\addcontentsline{toc}{section}{#1}
}

% a part of a hora, larger than pars
\newcommand{\subhora}[1]{
\begin{center}
{\large \textit{#1}}
\end{center}
%\fancyhead[CO,CE]{#1}
\addcontentsline{toc}{subsubsection}{#1}
}

% rubricated inline text
\newcommand{\rubricatum}[1]{\textit{#1}}

% standalone rubric
\newcommand{\rubrica}[1]{\vspace{3mm}\rubricatum{#1}}

\newcommand{\notitia}[1]{\textcolor{red}{#1}}

\newcommand{\scriptura}[1]{\hfill \small\textit{#1}}

\newcommand{\translatioCantus}[1]{\vspace{1mm}%
{\noindent\footnotesize \nlfont{#1}}}

% pruznejsi varianta nasledujiciho - umoznuje nastavit sirku sloupce
% s prekladem
\newcommand{\psalmusEtTranslatioB}[3]{
  \vspace{0.5cm}
  \begin{parcolumns}[colwidths={2=#3}, nofirstindent=true]{2}
    \colchunk{
      \input{#1}
    }

    \colchunk{
      \vspace{-0.5cm}
      {\footnotesize \nlfont
        \input{#2}
      }
    }
  \end{parcolumns}
}

\newcommand{\psalmusEtTranslatio}[2]{
  \psalmusEtTranslatioB{#1}{#2}{8.5cm}
}


\newcommand{\canticumMagnificatEtTranslatio}[1]{
  \psalmusEtTranslatioB{#1}{temporalia/extra-adventum-vespers/magnificat-boh.tex}{12cm}
}
\newcommand{\canticumBenedictusEtTranslatio}[1]{
  \psalmusEtTranslatioB{#1}{temporalia/extra-adventum-laudes/benedictus-boh.tex}{10.5cm}
}

% volne misto nad antifonami, kam si zpevaci dokresli neumy
\newcommand{\hicSuntNeumae}{\vspace{0.5cm}}

% prepinani mista mezi notovymi osnovami: pro neumovane a neneumovane zpevy
\newcommand{\cantusCumNeumis}{
  \setgrefactor{17}
  \global\advance\grespaceabovelines by 5mm%
}
\newcommand{\cantusSineNeumas}{
  \setgrefactor{17}
  \global\advance\grespaceabovelines by -5mm%
}

% znaky k umisteni nad inicialu zpevu
\newcommand{\superInitialam}[1]{\gresetfirstlineaboveinitial{\small {\textbf{#1}}}{\small {\textbf{#1}}}}

% pars officii, i.e. "oratio", ...
\newcommand{\pars}[1]{\textbf{#1}}

\newenvironment{psalmus}{
  \setlength{\parindent}{0pt}
  \setlength{\parskip}{5pt}
}{
  \setlength{\parindent}{10pt}
  \setlength{\parskip}{10pt}
}

%%%% Prejmenovat na latinske:
\newcommand{\nadpisZalmu}[1]{
  \hspace{2cm}\textbf{#1}\vspace{2mm}%
  \nopagebreak%

}

% mode, score, translation
\newcommand{\antiphona}[3]{%
\hicSuntNeumae
\superInitialam{#1}
\includescore{#2}

#3
}
 % Often used macros

\newcommand{\annusEditionis}{2021}

%%%% Vicekrat opakovane kousky

\newcommand{\anteOrationem}{
  \rubrica{Ante Orationem, cantatur a Superiore:}

  \pars{Supplicatio Litaniæ.}

  \cuminitiali{}{temporalia/supplicatiolitaniae.gtex}

  \pars{Oratio Dominica.}

  \cuminitiali{}{temporalia/oratiodominica.gtex}

  \rubrica{Deinde dicitur ab Hebdomadario:}

  \cuminitiali{}{temporalia/dominusvobiscum-solemnis.gtex}

  \rubrica{In choro monialium loco Dominus vobiscum dicitur:}

  \sineinitiali{temporalia/domineexaudi.gtex}
}

\setlength{\columnsep}{30pt} % prostor mezi sloupci

%%%%%%%%%%%%%%%%%%%%%%%%%%%%%%%%%%%%%%%%%%%%%%%%%%%%%%%%%%%%%%%%%%%%%%%%%%%%%%%%%%%%%%%%%%%%%%%%%%%%%%%%%%%%%
\begin{document}

% Here we set the space around the initial.
% Please report to http://home.gna.org/gregorio/gregoriotex/details for more details and options
\grechangedim{afterinitialshift}{2.2mm}{scalable}
\grechangedim{beforeinitialshift}{2.2mm}{scalable}
\grechangedim{interwordspacetext}{0.22 cm plus 0.15 cm minus 0.05 cm}{scalable}%
\grechangedim{annotationraise}{-0.2cm}{scalable}

% Here we set the initial font. Change 38 if you want a bigger initial.
% Emit the initials in red.
\grechangestyle{initial}{\color{red}\fontsize{38}{38}\selectfont}

\pagestyle{empty}

%%%% Titulni stranka
\begin{titulusOfficii}
\ifx\titulus\undefined
\nomenFesti{Feria III \hebdomada{}}
\else
\titulus
\fi
\end{titulusOfficii}

\vfill

\begin{center}
%Ad usum et secundum consuetudines chori \guillemotright{}Conventus Choralis\guillemotleft.

%Editio Sancti Wolfgangi \annusEditionis
\end{center}

\scriptura{}

\pars{}

\pagebreak

\renewcommand{\headrulewidth}{0pt} % no horiz. rule at the header
\fancyhf{}
\pagestyle{fancy}

\cantusSineNeumas

\hora{Ad Matutinum.} %%%%%%%%%%%%%%%%%%%%%%%%%%%%%%%%%%%%%%%%%%%%%%%%%%%%%

\vspace{2mm}

\cuminitiali{}{temporalia/dominelabiamea.gtex}

\vfill
%\pagebreak

\vspace{2mm}

\ifx\invitatorium\undefined
\pars{Invitatorium.} \scriptura{Lc. 24, 34; Psalmus 94; \textbf{H232}}

\vspace{-6mm}

\antiphona{VI}{temporalia/inv-surrexitdominusvere.gtex}
\else
\invitatorium
\fi

\vfill
\pagebreak

\ifx\hymnusmatutinum\undefined
\pars{Hymnus}

\cuminitiali{VIII}{temporalia/hym-LaetareCaelum.gtex}
\else
\hymnusmatutinum
\fi

\vspace{-3mm}

\vfill
\pagebreak

\ifx\matutinum\undefined
\ifx\matua\undefined
\else
% MAT A
\pars{Psalmus 1.}

\vspace{-4mm}

\antiphona{II D}{temporalia/ant-alleluia-turco7.gtex}

%\vspace{-2mm}

\scriptura{Ps. 9, 22-32}

%\vspace{-2mm}

\initiumpsalmi{temporalia/ps9xxii_xxxii-initium-ii-D-auto.gtex}

\input{temporalia/ps9xxii_xxxii-ii-D.tex}

\vfill
\pagebreak

\pars{Psalmus 2.} \scriptura{Ps. 9, 33-39}

%\vspace{-2mm}

\initiumpsalmi{temporalia/ps9xxxiii_xxxix-initium-ii-D-auto.gtex}

\input{temporalia/ps9xxxiii_xxxix-ii-D.tex}

\vfill
\pagebreak

\pars{Psalmus 3.} \scriptura{Ps. 11}

%\vspace{-2mm}

\initiumpsalmi{temporalia/ps11-initium-ii-D-auto.gtex}

\input{temporalia/ps11-ii-D.tex}

\vfill

\antiphona{}{temporalia/ant-alleluia-turco7.gtex}

\vfill
\pagebreak
\fi
\ifx\matub\undefined
\else
% MAT B
\pars{Psalmus 1.}

\vspace{-4mm}

\antiphona{VI F}{temporalia/ant-alleluia-turco6.gtex}

%\vspace{-2mm}

\scriptura{Ps. 36, 1-11}

%\vspace{-2mm}

\initiumpsalmi{temporalia/ps36i_xi-initium-vi-F-auto.gtex}

\input{temporalia/ps36i_xi-vi-F.tex}

\vfill
\pagebreak

\pars{Psalmus 2.}

\scriptura{Ps. 36, 12-29}

\vspace{-2mm}

\initiumpsalmi{temporalia/ps36xii_xxix-initium-vi-F-auto.gtex}

\input{temporalia/ps36xii_xxix-vi-F.tex}

\vfill
\pagebreak

\pars{Psalmus 3.}

\scriptura{Ps. 36, 30-40}

%\vspace{-2mm}

\initiumpsalmi{temporalia/ps36iii-initium-vi-F-auto.gtex}

\input{temporalia/ps36iii-vi-F.tex}

\antiphona{}{temporalia/ant-alleluia-turco6.gtex}

\vfill
\pagebreak
\fi
\ifx\matuc\undefined
\else
% MAT C
\pars{Psalmus 1.}

\vspace{-4mm}

\antiphona{I g\textsuperscript{5}}{temporalia/ant-alleluia-auglx2.gtex}

%\vspace{-2mm}

\scriptura{Ps. 67, 2-11}

\initiumpsalmi{temporalia/ps67i-initium-i-g5.gtex}

\input{temporalia/ps67i-i-g.tex}

\vfill
\pagebreak

\pars{Psalmus 2.}

\scriptura{Ps. 67, 12-24}

%\vspace{-2mm}

\initiumpsalmi{temporalia/ps67ii-initium-i-g5.gtex}

\input{temporalia/ps67ii-i-g.tex}

\vfill
\pagebreak

\pars{Psalmus 3.}

\scriptura{Ps. 67, 25-36}

\initiumpsalmi{temporalia/ps67iii-initium-i-g5.gtex}

\input{temporalia/ps67iii-i-g.tex}

\vfill

\antiphona{}{temporalia/ant-alleluia-auglx2.gtex}

\vfill
\pagebreak
\fi
\ifx\matud\undefined
\else
% MAT D
\pars{Psalmus 1.}

\vspace{-4mm}

\antiphona{I d\textsuperscript{3}}{temporalia/ant-alleluia-auglx6.gtex}

%\vspace{-2mm}

\scriptura{Ps. 101, 2-12}

%\vspace{-2mm}

\initiumpsalmi{temporalia/ps101ii_xii-initium-i-d3-auto.gtex}

\input{temporalia/ps101ii_xii-i-d3.tex}

\vfill
\pagebreak

\pars{Psalmus 2.} \scriptura{Ps. 101, 13-23}

\vspace{-2mm}

\initiumpsalmi{temporalia/ps101xiii_xxiii-initium-i-d3-auto.gtex}

\input{temporalia/ps101xiii_xxiii-i-d3.tex}

\vfill
\pagebreak

\pars{Psalmus 3.} \scriptura{Ps. 101, 24-29}

%\vspace{-2mm}

\initiumpsalmi{temporalia/ps101iii-initium-i-d3-auto.gtex}

\input{temporalia/ps101iii-i-d3.tex}

\vfill

\antiphona{}{temporalia/ant-alleluia-auglx6.gtex}

\vfill
\pagebreak
\fi
\else
\matutinum
\fi

\pars{Versus.}

\ifx\matversus\undefined
\noindent \Vbardot{} Christus resúrgens ex mórtuis iam non móritur, allelúia.

\noindent \Rbardot{} Mors illi ultra non dominábitur, allelúia.
\else
\matversus
\fi

\vspace{5mm}

\sineinitiali{temporalia/oratiodominica-mat.gtex}

\vspace{5mm}

\pars{Absolutio.}

\cuminitiali{}{temporalia/absolutio-ipsius.gtex}

\vfill
\pagebreak

\cuminitiali{}{temporalia/benedictio-solemn-deus.gtex}

\vspace{7mm}

\lectioi

\noindent \Vbardot{} Tu autem, Dómine, miserére nobis.
\noindent \Rbardot{} Deo grátias.

\vfill
\pagebreak

\responsoriumi

\vfill
\pagebreak

\cuminitiali{}{temporalia/benedictio-solemn-christus.gtex}

\vspace{7mm}

\lectioii

\noindent \Vbardot{} Tu autem, Dómine, miserére nobis.
\noindent \Rbardot{} Deo grátias.

\vfill
\pagebreak

\responsoriumii

\vfill
\pagebreak

\cuminitiali{}{temporalia/benedictio-solemn-ignem.gtex}

\vspace{7mm}

\lectioiii

\noindent \Vbardot{} Tu autem, Dómine, miserére nobis.
\noindent \Rbardot{} Deo grátias.

\vfill
\pagebreak

\responsoriumiii

\vfill
\pagebreak

\rubrica{Reliqua omittuntur, nisi Laudes separandæ sint.}

\sineinitiali{temporalia/domineexaudi.gtex}

\vfill

\oratio

\vfill

\noindent \Vbardot{} Dómine, exáudi oratiónem meam.
\Rbardot{} Et clamor meus ad te véniat.

\vfill

\noindent \Vbardot{} Benedicámus Dómino.
\noindent \Rbardot{} Deo grátias.

\vfill

\noindent \Vbardot{} Fidélium ánimæ per misericórdiam Dei requiéscant in pace.
\Rbardot{} Amen.

\vfill
\pagebreak

\hora{Ad Laudes.} %%%%%%%%%%%%%%%%%%%%%%%%%%%%%%%%%%%%%%%%%%%%%%%%%%%%%

\cantusSineNeumas

\vspace{0.5cm}
\grechangedim{interwordspacetext}{0.18 cm plus 0.15 cm minus 0.05 cm}{scalable}%
\cuminitiali{}{temporalia/deusinadiutorium-communis.gtex}
\grechangedim{interwordspacetext}{0.22 cm plus 0.15 cm minus 0.05 cm}{scalable}%

\vfill
\pagebreak

\ifx\hymnuslaudes\undefined
\ifx\laudac\undefined
\else
\pars{Hymnus}

\cuminitiali{I}{temporalia/hym-ChorusNovae-praglia.gtex}
\fi
\ifx\laudbd\undefined
\else
\pars{Hymnus}

\cuminitiali{I}{temporalia/hym-ChorusNovae.gtex}
\fi
\else
\hymnuslaudes
\fi

\vfill
\pagebreak

\ifx\laudes\undefined
\ifx\lauda\undefined
\else
\pars{Psalmus 1.}

\vspace{-4mm}

\antiphona{IV* e}{temporalia/ant-alleluia-turco9.gtex}

\scriptura{Psalmus 23.}

\initiumpsalmi{temporalia/ps23-initium-iv_-e-auto.gtex}

\input{temporalia/ps23-iv_-e.tex} \Abardot{}

\vfill
\pagebreak

\pars{Psalmus 2.} \scriptura{Tob. 13, 10}

\vspace{-4mm}

\antiphona{VIII G}{temporalia/ant-benedicitedominumomneselecti.gtex}

\scriptura{Canticum Tobiæ, Tob. 13, 2-8}

\initiumpsalmi{temporalia/tobiae-initium-viii-g-auto.gtex}

\input{temporalia/tobiae-viii-g.tex} \Abardot{}

\vfill
\pagebreak

\pars{Psalmus 3.}

\vspace{-4mm}

\antiphona{E}{temporalia/ant-alleluia-praglia-e2.gtex}

%\vspace{-4mm}

\scriptura{Psalmus 32.}

%\vspace{-2mm}

\initiumpsalmi{temporalia/ps32-initium-e-auto.gtex}

\input{temporalia/ps32-e.tex}

\vfill

\antiphona{}{temporalia/ant-alleluia-praglia-e2.gtex}

\vfill
\pagebreak
\fi
\ifx\laudb\undefined
\else
\pars{Psalmus 1.}

\vspace{-4mm}

\antiphona{E}{temporalia/ant-alleluia-praglia-e.gtex}

\scriptura{Psalmus 42.}

\initiumpsalmi{temporalia/ps42-initium-e-e-auto.gtex}

\input{temporalia/ps42-e-e.tex} \Abardot{}

\vfill
\pagebreak

\pars{Psalmus 2.} \scriptura{Is. 38, 17}

\vspace{-4mm}

\antiphona{I g}{temporalia/ant-eruistidomine-tp.gtex}

%\vspace{-2mm}

\scriptura{Canticum Ezechiæ, Is. 38, 10-20}

%\vspace{-2mm}

\initiumpsalmi{temporalia/ezechiae-initium-i-g-auto.gtex}

%\vspace{-1.5mm}

\input{temporalia/ezechiae-i-g.tex}

\vfill

\antiphona{}{temporalia/ant-eruistidomine-tp.gtex}

\vfill
\pagebreak

\pars{Psalmus 3.}

\vspace{-4mm}

\antiphona{VIII c}{temporalia/ant-alleluia-turco16.gtex}

\vspace{-2mm}

\scriptura{Psalmus 64.}

\vspace{-2mm}

\initiumpsalmi{temporalia/ps64-initium-viii-C-auto.gtex}

\input{temporalia/ps64-viii-C.tex} \Abardot{}

\vfill
\pagebreak
\fi
\ifx\laudc\undefined
\else
\pars{Psalmus 1.}

\vspace{-4mm}

\antiphona{VI F}{temporalia/ant-alleluia-turco5.gtex}

\vspace{-2mm}

\scriptura{Psalmus 84.}

\vspace{-2mm}

\initiumpsalmi{temporalia/ps84-initium-vi-F-auto.gtex}

\input{temporalia/ps84-vi-F.tex} \Abardot{}

\vfill
\pagebreak

\pars{Psalmus 2.}

\vspace{-4mm}

\antiphona{VII d}{temporalia/ant-denoctespiritusmeus-tp.gtex}

\vspace{-2mm}

\scriptura{Canticum Isaiæ, Is. 26, 1-12}

\vspace{-2mm}

\initiumpsalmi{temporalia/isaiae3-initium-vii-d.gtex}

\input{temporalia/isaiae3-vii-d.tex} \Abardot{}

\vfill
\pagebreak

\pars{Psalmus 3.}

\vspace{-4mm}

\antiphona{E}{temporalia/ant-alleluia-praglia-e2.gtex}

%\vspace{-2mm}

\scriptura{Psalmus 66.}

%\vspace{-2mm}

\initiumpsalmi{temporalia/ps66-initium-e-auto.gtex}

\input{temporalia/ps66-e.tex} \Abardot{}

\vfill
\pagebreak
\fi
\ifx\laudd\undefined
\else
\pars{Psalmus 1.}

\vspace{-4mm}

\antiphona{VIII G}{temporalia/ant-alleluia-turco12.gtex}

\vspace{-2mm}

\scriptura{Psalmus 100.}

\vspace{-2mm}

\initiumpsalmi{temporalia/ps100-initium-viii-G-auto.gtex}

\input{temporalia/ps100-viii-G.tex} \Abardot{}

\vfill
\pagebreak

\pars{Psalmus 2.} \scriptura{Ps. 50, 19}

\vspace{-4mm}

\antiphona{I f}{temporalia/ant-sacrificiumdeo-tp.gtex}

%\vspace{-2mm}

\scriptura{Canticum Danielis, Dan. 3, 26.27.29.34-41}

%\vspace{-2mm}

\initiumpsalmi{temporalia/dan32-initium-i-f-auto.gtex}

\input{temporalia/dan32-i-f.tex} \Abardot{}

\vfill
\pagebreak

\pars{Psalmus 3.}

\vspace{-4mm}

\antiphona{VI F}{temporalia/ant-alleluia-turco5.gtex}

%\vspace{-2mm}

\scriptura{Psalmus 143, 1-10.}

%\vspace{-2mm}

\initiumpsalmi{temporalia/ps143i_x-initium-vi-F-auto.gtex}

\input{temporalia/ps143i_x-vi-F.tex} \Abardot{}

\vfill
\pagebreak
\fi
\else
\laudes
\fi

\ifx\lectiobrevis\undefined
\pars{Lectio Brevis.} \scriptura{Ac. 13, 30-33}

\noindent Deus suscitávit Iesum a mórtuis; qui visus est per dies multos his, qui simul ascénderant cum eo de Galilǽa in Ierúsalem, qui nunc sunt testes eius ad plebem. Et nos vobis evangelizámus eam, quæ ad patres promíssio facta est, quóniam hanc Deus adimplévit fíliis eórum, nobis resúscitans Iesum, sicut et in Psalmo secúndo scriptum est: Fílius meus es tu; ego hódie génui te.
\else
\lectiobrevis
\fi

\vfill

\ifx\responsoriumbreve\undefined
\pars{Responsorium breve.} \scriptura{Cf. Mt. 28, 6; Cf. Gal. 3, 13}

\cuminitiali{VI}{temporalia/resp-surrexitdominusdesepulcro.gtex}
\else
\responsoriumbreve
\fi

\vfill
\pagebreak

\benedictus

\vspace{-1cm}

\vfill
\pagebreak

\ifx\precestotum\undefined
\pars{Preces.}

\sineinitiali{}{temporalia/tonusprecum.gtex}

\ifx\preces\undefined
\ifx\lauda\undefined
\else
\noindent Exsultémus Christo, qui perémptum sui córporis templum sua virtúte restítuit,~\gredagger{} eíque supplicémus:

\Rbardot{} Fructus resurrectiónis tuæ, Dómine, nobis concéde.

\noindent Christe salvátor, qui in resurrectióne tua muliéribus et Apóstolis gáudium nuntiásti, totum orbem salvíficans,~\gredagger{} testes tuos nos éffice.

\Rbardot{} Fructus resurrectiónis tuæ, Dómine, nobis concéde.

\noindent Qui resurrectiónem ómnibus promisísti, qua ad vitam novam resurgerémus,~\gredagger{} Evangélii tui nos redde præcónes.

\Rbardot{} Fructus resurrectiónis tuæ, Dómine, nobis concéde.

\noindent Tu, qui Apóstolis sǽpius apparuísti et Sanctum eis Spíritum insufflásti,~\gredagger{} creatórem Spíritum rénova in nobis.

\Rbardot{} Fructus resurrectiónis tuæ, Dómine, nobis concéde.

\noindent Tu, qui discípulis tuis promisísti te cum eis mansúrum usque ad consummatiónem sǽculi,~\gredagger{} mane nobíscum hódie sempérque nobis adésto.

\Rbardot{} Fructus resurrectiónis tuæ, Dómine, nobis concéde.
\fi
\ifx\laudb\undefined
\else
\noindent Deum Patrem, cuius Agnus immaculátus tollit peccáta mundi nosque vivíficat,~\gredagger{} grati rogémus:

\Rbardot{} Auctor vitæ, vivífica nos.

\noindent Deus, auctor vitæ, meménto passiónis et resurrectiónis Agni, in cruce occísi,~\gredagger{} eúmque audi, semper interpellántem pro nobis.

\Rbardot{} Auctor vitæ, vivífica nos.

\noindent Expurgáto vétere ferménto malítiæ et nequítiæ,~\gredagger{} fac nos vívere in ázymis sinceritátis et veritátis Christi.

\Rbardot{} Auctor vitæ, vivífica nos.

\noindent Da, ut hódie reiciámus peccátum discórdiæ atque invídiæ,~\gredagger{} nosque redde fratrum necessitátibus magis inténtos.

\Rbardot{} Auctor vitæ, vivífica nos.

\noindent Spíritum evangélicum pone in médio nostri,~\gredagger{} ut hódie et semper in præcéptis tuis ambulémus.

\Rbardot{} Auctor vitæ, vivífica nos.
\fi
\ifx\laudc\undefined
\else
\noindent Exsultémus Christo, qui perémptum sui córporis templum sua virtúte restítuit,~\gredagger{} eíque supplicémus:

\Rbardot{} Fructus resurrectiónis tuæ, Dómine, nobis concéde.

\noindent Christe salvátor, qui in resurrectióne tua muliéribus et Apóstolis gáudium nuntiásti, totum orbem salvíficans,~\gredagger{} testes tuos nos éffice.

\Rbardot{} Fructus resurrectiónis tuæ, Dómine, nobis concéde.

\noindent Qui resurrectiónem ómnibus promisísti, qua ad vitam novam resurgerémus,~\gredagger{} Evangélii tui nos redde præcónes.

\Rbardot{} Fructus resurrectiónis tuæ, Dómine, nobis concéde.

\noindent Tu, qui Apóstolis sǽpius apparuísti et Sanctum eis Spíritum insufflásti,~\gredagger{} creatórem Spíritum rénova in nobis.

\Rbardot{} Fructus resurrectiónis tuæ, Dómine, nobis concéde.

\noindent Tu, qui discípulis tuis promisísti te cum eis mansúrum usque ad consummatiónem sǽculi,~\gredagger{} mane nobíscum hódie sempérque nobis adésto.

\Rbardot{} Fructus resurrectiónis tuæ, Dómine, nobis concéde.
\fi
\ifx\laudd\undefined
\else
\noindent Deum Patrem, cuius Agnus immaculátus tollit peccáta mundi nosque vivíficat,~\gredagger{} grati rogémus:

\Rbardot{} Auctor vitæ, vivífica nos.

\noindent Deus, auctor vitæ, meménto passiónis et resurrectiónis Agni, in cruce occísi,~\gredagger{} eúmque audi, semper interpellántem pro nobis.

\Rbardot{} Auctor vitæ, vivífica nos.

\noindent Expurgáto vétere ferménto malítiæ et nequítiæ,~\gredagger{} fac nos vívere in ázymis sinceritátis et veritátis Christi.

\Rbardot{} Auctor vitæ, vivífica nos.

\noindent Da, ut hódie reiciámus peccátum discórdiæ atque invídiæ,~\gredagger{} nosque redde fratrum necessitátibus magis inténtos.

\Rbardot{} Auctor vitæ, vivífica nos.

\noindent Spíritum evangélicum pone in médio nostri,~\gredagger{} ut hódie et semper in præcéptis tuis ambulémus.

\Rbardot{} Auctor vitæ, vivífica nos.
\fi
\else
\preces
\fi

\vfill

\pars{Oratio Dominica.}

\cuminitiali{}{temporalia/oratiodominicaalt.gtex}

\vfill
\pagebreak

\rubrica{vel:}

\pars{Supplicatio Litaniæ.}

\cuminitiali{}{temporalia/supplicatiolitaniae.gtex}

\vfill

\pars{Oratio Dominica.}

\cuminitiali{}{temporalia/oratiodominica.gtex}
\else
\precestotum
\fi

\vfill
\pagebreak

% Oratio. %%%
\oratio

\vspace{-1mm}

\vfill

\rubrica{Hebdomadarius dicit Dominus vobiscum, vel, absente sacerdote vel diacono, sic concluditur:}

\vspace{2mm}

\ifx\dominusnosbenedicat\undefined
\antiphona{C}{temporalia/dominusnosbenedicat.gtex}
\else
\dominusnosbenedicat
\fi

\rubrica{Postea cantatur a cantore:}

\vspace{2mm}

\ifx\benedicamuslaudes\undefined
\cuminitiali{VII}{temporalia/benedicamus-tempore-paschali.gtex}
\else
\benedicamuslaudes
\fi

\vspace{1mm}

\vfill
\pagebreak

\end{document}

