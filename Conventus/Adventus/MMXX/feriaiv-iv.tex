\newcommand{\lectioi}{\pars{Lectio I.} \scriptura{Is. 51, 1-5.7-8.11}

\noindent De libro Isaíæ prophétæ.

\noindent Audíte me, qui sequímini iustítiam, qui quǽritis Dóminum; atténdite ad petram, unde excísi estis, et ad cavérnam laci, de qua præcísi estis. Atténdite ad Abraham patrem vestrum et ad Saram, quæ péperit vos; quia unum vocávi eum et benedíxi ei et multiplicávi eum. Consolátur enim Dóminus Sion, consolátur omnes ruínas eius; et ponit desértum eius quasi Eden et solitúdinem eius quasi hortum Dómini. Gáudium et lætítia inveniétur in ea, gratiárum áctio et vox laudis. Atténdite ad me, pópule meus, et, natiónes, me audíte, quia lex a me éxiet, et iudícium meum in lucem populórum státuam. Prope est iustítia mea, egréssa est salus mea, et bráchia mea pópulos iudicábunt; in me ínsulæ sperábunt et ad bráchium meum atténdent. Audíte me, qui scitis iustítiam, pópule, in cuius corde est lex mea: nolíte timére oppróbrium hóminum et blasphémias eórum ne metuátis. Sicut enim vestiméntum sic cómedet eos vermis, et sicut lanam sic devorábit eos tínea; iustítia autem mea in sempitérnum erit et salus mea in generatiónes generatiónum. Et redémpti a Dómino reverténtur et vénient in Sion laudántes; et lætítia sempitérna super cápita eórum, gáudium et lætítiam obtinébunt; fúgiet dolor et gémitus.}
\newcommand{\responsoriumi}{\pars{Responsorium 1.} \scriptura{\Rbar{} Is. 40, 9.10 \Vbar{} ibid. 40, 9; \textbf{H34}}

\vspace{-5mm}

\responsorium{VI}{temporalia/resp-clamainfortitudine-CROCHU.gtex}{}}
\newcommand{\lectioii}{\pars{Lectio II.} \scriptura{Cap. 9-12: PG 10, 815-819}

\noindent Ex Tractátu sancti Hippólyti presbýteri Contra hǽresim Noéti.

\noindent Deus, solus cum esset, nihílque sibi coǽvum habéret, vóluit mundum effícere. Et mundum cógitans ac volens et dicens effécit; continuóque éxstitit ei factus sicut vóluit, et sicut vóluit, perfécit. Satis ígitur nobis est scire solum, nihil esse Deo coǽvum. Nihil erat præter ipsum; ipse solus, multus erat. Nec enim erat sine ratióne, sine sapiéntia, sine poténtia, sine consílio. Omnia erant in eo: ipse erat ómnia. Quando vóluit, et quómodo vóluit, osténdit Verbum suum tempóribus apud eum definítis; per quod ómnia fecit.

\noindent Quod Verbum cum in se habéret, essétque mundo creáto inaspectábile, fecit aspectábile, emíttens priórem vocem; et lumen ex lúmine génerans, deprómpsit ipsi creatúræ Dóminum, sensum suum; et qui prius ipsi tantum erat visíbilis, mundo autem invisíbilis, hunc visíbilem facit, ut mundus, cum eum qui appáruit vidéret, salvus fíeri posset.}
\newcommand{\responsoriumii}{\pars{Responsorium 2.} \scriptura{\Rbar{} Is. 1, 11 \Vbar{} Is. 45, 8; \textbf{H36}}

\vspace{-5mm}

\responsorium{I}{temporalia/resp-egredieturvirga-CROCHU.gtex}{}}
\newcommand{\lectioiii}{\pars{Lectio III.}

\noindent Hoc vero mens est, quod pródiens in mundum, osténsum est puer Dei. Omnia ígitur per ipsum, ipse autem solus ex Patre.

\noindent Hic autem dedit legem et prophétas; et dando coégit hos per Spíritum Sanctum loqui, ut accipiéntes virtútis patérnæ inspiratiónem, consílium et voluntátem Patris nuntiárent.

\noindent Factum est ígitur maniféstum Verbum, sicut ait beátus Ioánnes. Répetit enim summátim quæ a prophétis dicta sunt, demónstrans hoc esse Verbum per quod ómnia facta sunt. Sic enim ait: In princípio erat Verbum et Verbum erat apud Deum et Deus erat Verbum. Omnia per ipsum facta sunt et sine ipso factum est nihil. Et infra ait: Mundus per ipsum factus est et mundus eum non cognóvit. In própria venit et sui eum non recepérunt.}
\newcommand{\responsoriumiii}{\pars{Responsorium 3.} \scriptura{\Rbar{} Cantor \Vbar{} Lc. 1, 28; \textbf{H36}}

\vspace{-5mm}

\responsorium{I}{temporalia/resp-annuntiatumest-CROCHU-cumdox.gtex}{}}
\newcommand{\lectiobrevis}{\pars{Lectio Brevis.} \scriptura{Ier. 30, 21,22}

\noindent Hæc dicit Dóminus: Erit dux eius ex Iacob, et princeps de médio eius procédet; et applicábo eum, et accédet ad me. Et éritis mihi in pópulum, et ego ero vobis in Deum.}
\newcommand{\benedictus}{\pars{Canticum Zachariæ.} \scriptura{\textbf{H81}}

\vspace{-4mm}

{
\grechangedim{interwordspacetext}{0.18 cm plus 0.15 cm minus 0.05 cm}{scalable}%
\antiphona{VIII C}{temporalia/ant-eccecompletasunt.gtex}
\grechangedim{interwordspacetext}{0.22 cm plus 0.15 cm minus 0.05 cm}{scalable}%
}

%\trAntIMagnificat

%\vspace{-2mm}

\scriptura{Lc. 1, 68-79}

%\vspace{-2mm}

\cantusSineNeumas
\initiumpsalmi{temporalia/benedictus-initium-viii-c-auto.gtex}

%\vspace{-1.5mm}

\input{temporalia/benedictus-viii-c.tex} \Abardot{}}
\newcommand{\preces}{\noindent Deum Patrem, fratres caríssimi, implorántes, qui misit Fílium suum ad salvándos hómines,~\gredagger{} súpplices acclamémus:

\Rbardot{} Osténde nobis, Dómine, misericórdiam tuam.

\noindent Christum tuum, Pater clementíssime, quem plena fide os nostrum annúntiat,~\gredagger{} conversátio nostra ópere ne despíciat.

\Rbardot{} Osténde nobis, Dómine, misericórdiam tuam.

\noindent Qui Fílium tuum misísti ad salútem,~\gredagger{} univérsum aufer a fácie terræ et a civitáte ista dolórem.

\Rbardot{} Osténde nobis, Dómine, misericórdiam tuam.

\noindent Terra nostra advéntu Fílii tui iucunditáte perfúsa,~\gredagger{} tuæ plenitúdinis gáudium ubérius experiátur.

\Rbardot{} Osténde nobis, Dómine, misericórdiam tuam.

\noindent Per misericórdiam tuam, fac ut nos pie et sóbrie in hoc sǽculo vivámus,~\gredagger{} exspectántes beátam spem et advéntum glóriæ Christi.

\Rbardot{} Osténde nobis, Dómine, misericórdiam tuam.}
\newcommand{\oratio}{\pars{Oratio.}

\noindent Omnípotens sempitérne Deus, nativitátem Fílii tui secúndum carnem propinquáre cernéntes, quǽsumus,~\gredagger{} ut nobis indígnis fámulis tuis misericórdiam præstet Verbum,~\grestar{} quod ex Vírgine María dignátum est caro fíeri et habitáre in nobis.

\noindent Qui tecum vivit et regnat in unitáte Spíritus Sancti, Deus, per ómnia sǽcula sæculórum.

\noindent \Rbardot{} Amen.}
\newcommand{\magnificat}{\pars{Canticum B. Mariæ V.} \scriptura{Is. 7, 14; 33, 22; Gn. 49, 10; \textbf{H41}}

\vspace{-6.5mm}

{
\grechangedim{interwordspacetext}{0.18 cm plus 0.15 cm minus 0.05 cm}{scalable}%
\antiphona{II D}{temporalia/ant-oemmanuel.gtex}
\grechangedim{interwordspacetext}{0.22 cm plus 0.15 cm minus 0.05 cm}{scalable}%
}

%\trAntIMagnificat

\vspace{-3mm}

\scriptura{Lc. 1, 46-55}

\vspace{-2mm}

\cantusSineNeumas

\initiumpsalmi{temporalia/magnificat-initium-iisoll-D.gtex}

\vspace{-1.5mm}

\input{temporalia/magnificat-iisoll-D.tex} \Abardot{}

\vspace{-1cm}}
\newcommand{\hebdomada}{infra Hebdom. IV per Annum.}
%\newcommand{\hiemalis}{Hiemalis}
\newcommand{\matud}{Matutinum Hebdomadae D}
\newcommand{\matubd}{Matutinum Hebdomadae B vel D}
\newcommand{\laudd}{Laudes Hebdomadae D}
\newcommand{\laudbd}{Laudes Hebdomadae B vel D}

% LuaLaTeX

\documentclass[a4paper, twoside, 12pt]{article}
\usepackage[latin]{babel} 
%\usepackage[landscape, left=3cm, right=1.5cm, top=2cm, bottom=1cm]{geometry} % okraje stranky
%\usepackage[landscape, a4paper, mag=1166, truedimen, left=2cm, right=1.5cm, top=1.6cm, bottom=0.95cm]{geometry} % okraje stranky
\usepackage[landscape, a4paper, mag=1400, truedimen, left=0.5cm, right=0.5cm, top=0.5cm, bottom=0.5cm]{geometry} % okraje stranky

\usepackage{fontspec}
\setmainfont[FeatureFile={junicode.fea}, Ligatures={Common, TeX}, RawFeature=+fixi]{Junicode}
%\setmainfont{Junicode}

% shortcut for Junicode without ligatures (for the Czech texts)
\newfontfamily\nlfont[FeatureFile={junicode.fea}, Ligatures={Common, TeX}, RawFeature=+fixi]{Junicode}

% Hebrew font:
% http://scripts.sil.org/cms/scripts/page.php?site_id=nrsi&id=SILHebrUnic2
\newfontfamily\hebfont[Scale=1]{Ezra SIL}

\usepackage{multicol}
\usepackage{color}
\usepackage{lettrine}
\usepackage{fancyhdr}

% usual packages loading:
\usepackage{luatextra}
\usepackage{graphicx} % support the \includegraphics command and options
\usepackage{gregoriotex} % for gregorio score inclusion
\usepackage{gregoriosyms}
\usepackage{wrapfig} % figures wrapped by the text
\usepackage{parcolumns}
\usepackage[contents={},opacity=1,scale=1,color=black]{background}
\usepackage{tikzpagenodes}
\usepackage{calc}
\usepackage{longtable}
\usetikzlibrary{calc}

\setlength{\headheight}{14.5pt}

% Commands used to produce a typical "Conventus" booklet

\newenvironment{titulusOfficii}{\begin{center}}{\end{center}}
\newcommand{\dies}[1]{#1

}
\newcommand{\nomenFesti}[1]{\textbf{\Large #1}

}
\newcommand{\celebratio}[1]{#1

}

\newcommand{\hora}[1]{%
\vspace{0.5cm}{\large \textbf{#1}}

\fancyhead[LE]{\thepage\ / #1}
\fancyhead[RO]{#1 / \thepage}
\addcontentsline{toc}{subsection}{#1}
}

% larger unit than a hora
\newcommand{\divisio}[1]{%
\begin{center}
{\Large \textsc{#1}}
\end{center}
\fancyhead[CO,CE]{#1}
\addcontentsline{toc}{section}{#1}
}

% a part of a hora, larger than pars
\newcommand{\subhora}[1]{
\begin{center}
{\large \textit{#1}}
\end{center}
%\fancyhead[CO,CE]{#1}
\addcontentsline{toc}{subsubsection}{#1}
}

% rubricated inline text
\newcommand{\rubricatum}[1]{\textit{#1}}

% standalone rubric
\newcommand{\rubrica}[1]{\vspace{3mm}\rubricatum{#1}}

\newcommand{\notitia}[1]{\textcolor{red}{#1}}

\newcommand{\scriptura}[1]{\hfill \small\textit{#1}}

\newcommand{\translatioCantus}[1]{\vspace{1mm}%
{\noindent\footnotesize \nlfont{#1}}}

% pruznejsi varianta nasledujiciho - umoznuje nastavit sirku sloupce
% s prekladem
\newcommand{\psalmusEtTranslatioB}[3]{
  \vspace{0.5cm}
  \begin{parcolumns}[colwidths={2=#3}, nofirstindent=true]{2}
    \colchunk{
      \input{#1}
    }

    \colchunk{
      \vspace{-0.5cm}
      {\footnotesize \nlfont
        \input{#2}
      }
    }
  \end{parcolumns}
}

\newcommand{\psalmusEtTranslatio}[2]{
  \psalmusEtTranslatioB{#1}{#2}{8.5cm}
}


\newcommand{\canticumMagnificatEtTranslatio}[1]{
  \psalmusEtTranslatioB{#1}{temporalia/extra-adventum-vespers/magnificat-boh.tex}{12cm}
}
\newcommand{\canticumBenedictusEtTranslatio}[1]{
  \psalmusEtTranslatioB{#1}{temporalia/extra-adventum-laudes/benedictus-boh.tex}{10.5cm}
}

% volne misto nad antifonami, kam si zpevaci dokresli neumy
\newcommand{\hicSuntNeumae}{\vspace{0.5cm}}

% prepinani mista mezi notovymi osnovami: pro neumovane a neneumovane zpevy
\newcommand{\cantusCumNeumis}{
  \setgrefactor{17}
  \global\advance\grespaceabovelines by 5mm%
}
\newcommand{\cantusSineNeumas}{
  \setgrefactor{17}
  \global\advance\grespaceabovelines by -5mm%
}

% znaky k umisteni nad inicialu zpevu
\newcommand{\superInitialam}[1]{\gresetfirstlineaboveinitial{\small {\textbf{#1}}}{\small {\textbf{#1}}}}

% pars officii, i.e. "oratio", ...
\newcommand{\pars}[1]{\textbf{#1}}

\newenvironment{psalmus}{
  \setlength{\parindent}{0pt}
  \setlength{\parskip}{5pt}
}{
  \setlength{\parindent}{10pt}
  \setlength{\parskip}{10pt}
}

%%%% Prejmenovat na latinske:
\newcommand{\nadpisZalmu}[1]{
  \hspace{2cm}\textbf{#1}\vspace{2mm}%
  \nopagebreak%

}

% mode, score, translation
\newcommand{\antiphona}[3]{%
\hicSuntNeumae
\superInitialam{#1}
\includescore{#2}

#3
}
 % Often used macros
%%%% Preklady jednotlivych zpevu (nektere se opakuji, a je dobre mit je
% vsechny na jedne hromade)

\newcommand{\trOratioAnteOfficium}{\translatioCantus{Otevři, Pane, má ústa, abych chválil tvé svaté jméno.
Očisti mé srdce od všech marnivých, zvrácených a~jiných myšlenek, osvěť rozum, rozněť cit,
abych mohl důstojně, soustředěně a~zbožně recitovat a~vysloužil si být
vyslyšen před tváří tvé velebnosti. Skrze Krista…}}

\newcommand{\trOratioPostOfficium}{\translatioCantus{\textit{Následující modlitbu
opatřil pro ty, kdo ji zbožně vyřknou po hodinkách, zesnulý papež Lev X.
odpustky za hříchy vzniklé při konání hodinek z~lidské křehkosti. Říká se
vkleče.}
Svatosvaté a~nerozdílné Trojici, ukřižovanému lidství našeho Pána Ježíše
Krista, přeblažené a~přeslavné plodné neporušenosti vždy Panny Marie
i~souhrnu všech svatých buď ode všeho stvoření věčná chvála, čest a~sláva, nám
pak buď dáno odpuštění všech hříchů, po nekonečné věky věků. Amen.}}

% HOURS ---

\newcommand{\trAntI}{\translatioCantus{Jasné narození slavné Panny Marie,
z pokolení (dosl. ze semene) Abrahámova, vzešlé z kmene Judova, z rodu Davidova.}}
\newcommand{\trAntII}{\translatioCantus{Dnes je Narození svaté Panny 
Marie, jejíž předrahý život osvěcuje všechny církve.}}

\newcommand{\trAntIII}{\translatioCantus{Maria, jež vzešla 
z královského rodu, září; myslí i duchem ji zbožně prosíme, aby 
nám pomáhala svými přímluvami.}}

\newcommand{\trAntIV}{\translatioCantus{Srdcem i duchem pějme Kristu 
k slávě o této svaté slavnosti vznešené Rodičky Boží Marie.}}

\newcommand{\trAntV}{\translatioCantus{Příjemně \notitia{?} 
oslavujme Narození blahoslavené Marie,
aby se ona za nás přimlouvala u Pána Ježíše Krista.}}

\newcommand{\trCapituli}{\translatioCantus{Před věky, na počátku mě stvořil, potrvám věčně. Ve svatém Stanu jsem před ním konala službu.}}

\newcommand{\trRespVesp}{\translatioCantus{Buď zdráva, Maria,
plná milosti: \grestar{} Pán s tebou. \Vbardot{} Požehnaná jsi mezi ženami,
a požehnaný plod života (ve smyslu lůna, břicha) tvého.}}

\newcommand{\trVersus}{\translatioCantus{\Vbardot{} Dnes je Narození svaté Panny Marie. \Rbardot{} Jejíž předrahý život osvěcuje všechny církve.}}

\newcommand{\trAntMagnificatI}{\translatioCantus{Konejme památku
veledůstojného narození slavné Panny Marie,
jíž se dostalo mateřské důstojnosti bez ztráty panenské cudnosti.}}

% Tento preklad je vice nez nejisty a ani alternativy, ktere jsem
% videl, me nepresvedcily...
\newcommand{\trAntBenedictus}{\translatioCantus{Slavnostně slavme 
dnešní narození Marie, vždy Panny a Rodičky Boží: v něm se objevuje
vysokost trůnu (totiž Marie, trůnu Božího Syna), aleluja.}}

\newcommand{\trAntMagnificatII}{\translatioCantus{Tvé narození,
Bohorodičko Panno, vyhlásilo radost celému světu:
z tebe totiž vzešlo Slunce spravedlnosti, Kristus, náš Bůh:
jenž zrušil kletbu a dal nám požehnání: přemohl smrt a dal nám život věčný.}}

\newcommand{\trOrationis}{\translatioCantus{Prosíme tě, Bože, 
uděl svým služebníkům dar nebeské milosti,
aby těm, jimž slehnutím blahoslavené Panny vyvstal počátek spásy, 
slavnost k poctě jejího narození přinesla
rozhojnění pokoje.
Skrze tvého Syna, našeho Pána Ježíše Krista, který s tebou žije a kraluje,
Bůh, v jednotě Ducha svatého po všechny věky věků.}}

\newcommand{\trFideliumAnimae}{\translatioCantus{\Vbardot{} Duše věrných ať pro
milosrdenství Boží odpočívají v~pokoji. \Rbardot{} Amen.}}

% Completorium

\newcommand{\trJubeDomne}{\translatioCantus{Rač, pane, požehnat.}}

\newcommand{\trComplBenedictio}{\translatioCantus{Pokojnou noc a~svatou smrt
nechť nám dopřeje všemohoucí Pán. \Rbardot{} Amen.}}

\newcommand{\trComplLectioBr}{\translatioCantus{Buďte střízliví, bděte.
Váš protivník Ďábel obchází jako lev řvoucí a~hledá, koho by pohltil.
Postavte se proti němu pevní ve víře.  Ale ty, Pane, smiluj se nad námi.
\Rbardot{} Bohu díky.}}

\newcommand{\trComplAntI}{\translatioCantus{Rač se smilovati nade mnou,
Hospodine, a vyslyš mou modlitbu.}}

\newcommand{\trComplCapituli}{\translatioCantus{Jsi přece, Hospodine,
uprostřed nás a~jmenujeme se po tobě.  Neopouštěj nás, Pane, náš Bože.}}

\newcommand{\trRespCompl}{\translatioCantus{Do tvých rukou, Pane, \grestar{}
poroučím svého ducha. \Vbardot{} Ty mne zachráníš, Pane, Bože věrný.}}

\newcommand{\trComplVersus}{\translatioCantus{\Vbardot{} Střez mne jako zřítelnici oka,
aleluja. \Rbardot{} Ve stínu svých křídel uschovej mne, aleluja.}}

\newcommand{\trAntSalvaNos}{\translatioCantus{Ochraňuj nás, Pane, když
bdíme, a~buď s~námi, když spíme, ať bdíme s~Kristem a~odpočíváme v~pokoji.}}

\newcommand{\trComplOrationis}{\translatioCantus{Zavítej, prosíme, Pane, sem
do našeho příbytku a~daleko od něj zažeň všechny úklady nepřítele. Ať tu
bydlí tví svatí andělé a~tvoje požehnání buď nad ním stále. Skrze…}}

\newcommand{\trSalveRegina}{\translatioCantus{Zdrávas Královno, matko
milosrdenství, živote, sladkosti a naděje naše, buď zdráva!
K tobě voláme, vyhnaní synové Evy,
k tobě vzdycháme, lkajíce a plačíce
v tomto slzavém údolí.
A proto, orodovnice naše,
obrať k nám své milosrdné oči
a Ježíše, požehnaný plod života svého,
nám po tomto putování ukaž,
ó milostivá, ó přívětivá,
ó přesladká, Panno Maria!}}

\newcommand{\trOraProNobis}{\translatioCantus{\Vbardot{} 
Oroduj za nás, svatá Boží Rodičko,
\Rbardot{} aby nám Kristus dal účast na svých zaslíbeních.}}

% Matutinum

\newcommand{\trMatInvitatorium}{\translatioCantus{}}

\newcommand{\trMatVeniteA}{\translatioCantus{Pojďte, chvalme s~radostí Pána,
s~jásotem slavme Boha, svou spásu; předstupme před tvář jeho s~díky, písně plesu pějme jemu.}}

\newcommand{\trMatVeniteB}{\translatioCantus{Neboť Bůh veliký jest Hospodin, a~král nade všecky bohy.
Jsouť v~jeho ruce všecky hlubiny země, temena hor jsou majetek jeho.}}

\newcommand{\trMatVeniteC}{\translatioCantus{Jehoť jest moře, neb on je učinil; i~souš
je dílo jeho rukou. Pojďme, klanějme se, padněme, klekněme před Pánem, svým
tvůrcem. Jeť on Pán, náš Bůh, a~my jsme lid, jejž on vodí a~ovce, jež pase.}}

\newcommand{\trMatVeniteD}{\translatioCantus{Kéž byste poslechli dnes hlasu jeho:
,,Nezatvrzujte svých srdcí jak v~Hádce, jak v~Pokušení na poušti, kde vaši otcové pokoušeli mne,
zkoušeli mne, ač vídali skutky mé.``}}

\newcommand{\trMatVeniteE}{\translatioCantus{Čtyřicet roků mrzel jsem se na to pokolení
a~řekl jsem: ,,Lid je to myslí stále bloudící``! Oni však nechtěli znáti mé cesty, takže jsem
přisáhl ve svém hněvu: ,,Nedojdou odpočinku mého!\mbox{}``}}

\newcommand{\trMatAntI}{\translatioCantus{}}

\newcommand{\trMatAntII}{\translatioCantus{}}

\newcommand{\trMatAntIII}{\translatioCantus{}}

\newcommand{\trMatVersusI}{\translatioCantus{}}

\newcommand{\trMatAbsolutioI}{\translatioCantus{Vyslyš Pane Ježíši Kriste
prosby svých služebníků \gredagger{} a~smiluj se nad námi, \grestar{} jenž
s~Otcem a~Duchem…}}

\newcommand{\trMatBenedictioI}{\translatioCantus{Rač, pane, požehnat.
Věčný Otec nám stále žehnej. \Rbardot{} Amen.}}

\newcommand{\trMatLecI}{\translatioCantus{Kéž by mě zulíbal polibky svých úst. 
Tvé milování je nad víno lahodnější;
vybraně voní tvé voňavky;
rozlévající se olej je tvé jméno,
proto tě dívky milují.
Strhni mě za sebou, poběžme!
Král mě uvedl do svých komnat;
budeš nám radostí a jásotem.
Víc než víno oslavíme tvé milování;
věru po právu jsi milován!
Snědá jsem, a přece krásná, jeruzalémské dcery,
jako stany kedarské,
jako šalmské závěsy.
}}

\newcommand{\trMatRespI}{\translatioCantus{}}

\newcommand{\trMatBenedictioII}{\translatioCantus{Rač, pane, požehnat.
Jednorozený Boží Syn nám žehnej \grestar{} a nám pomáhej. \Rbardot{} Amen.}}

\newcommand{\trMatLecII}{\translatioCantus{Nehleďte na mou osmahlou pleť:
to mě slunce ožehlo.
Synové mé matky se na mne rozzlobili,
poslali mě hlídat vinice.
A svou vinici, tu jsem nehlídala!
Pověz mi tedy, ty, jehož miluje mé srdce:
kam zavedeš své stádo pást,
kde ho necháš za poledne odpočívat?
Abych už nebloudila jako tulačka
poblíž stád druhů tvých.
Nevíš-li to, nejrásnější z žen,
jdi po stopách stáda
a kůzlata svá zaveď, ať se pasou
poblíž obydlí pastýřů.
Ke své klisně zapřažené do vozu faraonova
tebe, mé milá, přirovnávám.
Stále krásné jsou tvé líce s náušnicemi
i tvé hrdlo v náhrdelnících.}}

\newcommand{\trMatRespII}{\translatioCantus{}}

\newcommand{\trMatBenedictioIII}{\translatioCantus{Rač, pane, požehnat.
Milost Ducha Svatého ať osvítí nám smysly \grestar{} i srdce. \Rbardot{} Amen.}}

\newcommand{\trMatLecIII}{\translatioCantus{Zhotovíme ti zlaté náušnice
a kuličky ze stříbra.
Když král stoluje,
vydechuje můj nard svou vůni.
Můj milý je polštářek s myrhou,
jenž mi odpočívá mezi ňadry.
Můj milý je hrozen šáchoru
ve vinicích v Engadi.
Jak jsi krásná, milá moje,
jak jsi krásná!
Tvé oči jsou holubice.
Jak jsi krásný, můj milý,
jak líbezný!
Naše lože je samá zeleň.
Trámoví našeho domu je z cedru,
naše ostění z cypřiše.}}

\newcommand{\trMatRespIII}{\translatioCantus{}}

\newcommand{\trMatAntIV}{\translatioCantus{}}

\newcommand{\trMatAntV}{\translatioCantus{}}

\newcommand{\trMatAntVI}{\translatioCantus{}}

\newcommand{\trMatVersusII}{\translatioCantus{}}

\newcommand{\trMatAbsolutioII}{\translatioCantus{
Tvá milost a laskavost nechť nám pomáhá, jenž žiješ a vládneš s Otcem a Svatým Duchem na věky věků.}}

\newcommand{\trMatBenedictioIV}{\translatioCantus{Rač, pane, požehnat.
Bůh Otec všemohoucí, \grestar{} buď k nám milostivý a odpouštějící. \Rbardot{} Amen.}}

\newcommand{\trMatLecIV}{\translatioCantus{}}

\newcommand{\trMatRespIV}{\translatioCantus{}}

\newcommand{\trMatBenedictioV}{\translatioCantus{}}

\newcommand{\trMatLecV}{\translatioCantus{}}

\newcommand{\trMatRespV}{\translatioCantus{}}

\newcommand{\trMatBenedictioVI}{\translatioCantus{Rač, pane, požehnat.
Bůh rozněť v nás oheň své lásky. \Rbardot{} Amen.}}

\newcommand{\trMatLecVI}{\translatioCantus{}}

\newcommand{\trMatRespVI}{\translatioCantus{}}

\newcommand{\trMatAntVII}{\translatioCantus{}}

\newcommand{\trMatAntVIII}{\translatioCantus{}}

\newcommand{\trMatAntIX}{\translatioCantus{}}

\newcommand{\trMatVersusIII}{\translatioCantus{}}

\newcommand{\trMatAbsolutioIII}{\translatioCantus{Z okovů našich hříchů,
\grestar{} vysvoboď nás všemohoucí a milosrdný Pán. \Rbardot{} Amen.}}

\newcommand{\trMatBenedictioVII}{\translatioCantus{Rač, pane, požehnat.
Čtení evangelia nechť je nám \grestar{} spásou a ochranou. \Rbardot{} Amen.}}

\newcommand{\trMatLecVIIa}{\translatioCantus{
  Rodokmen Ježíše Krista, syna Davidova, syna Abrahámova:
  Abrahám zplodil Izáka,
  Izák zplodil Jakuba.}}

\newcommand{\trMatLecVIIb}{\translatioCantus{}}

\newcommand{\trMatRespVII}{\translatioCantus{}}

\newcommand{\trMatBenedictioVIII}{\translatioCantus{Rač, pane, požehnat.
\Rbardot{} Amen.}}

\newcommand{\trMatLecVIII}{\translatioCantus{}}

\newcommand{\trMatRespVIII}{\translatioCantus{}}

\newcommand{\trMatBenedictioIX}{\translatioCantus{Rač, pane, požehnat.
Do společnosti občanů nebes \grestar{} ať nás dovede král andělů.
\Rbardot{} Amen.}}

\newcommand{\trMatLecIX}{\translatioCantus{}}

% from the Czech Liturgia horarum
\newcommand{\trTeDeum}{\begin{translatioMulticol}{3}

Bože, tebe chválíme, 
tebe, Pane, velebíme.

Tebe, věčný Otče, 
oslavuje celá země.

Všichni andělé, 
cherubové i~serafové,

všechny mocné nebeské zástupy 
bez ustání volají:

Svatý, Svatý, Svatý, 
Pán, Bůh zástupů.

Plná jsou nebesa i~země 
tvé vznešené slávy.

Oslavuje tě 
sbor tvých apoštolů,

chválí tě 
velký počet proroků,

vydává o~tobě svědectví 
zástup mučedníků;

a~po celém světě 
vyznává tě tvá církev:

neskonale velebný, 
všemohoucí Otče,

úctyhodný Synu Boží, 
pravý a~jediný,

božský Utěšiteli, 
Duchu svatý.

Kriste, Králi slávy, 
tys od věků Syn Boha Otce;

abys člověka vykoupil, 
stal ses člověkem a~narodil ses z~Panny;

zlomil jsi osten smrti 
a~otevřel věřícím nebe;

sedíš po Otcově pravici 
a~máš účast na jeho slávě.

Věříme, že přijdeš soudit, 

a~proto tě prosíme:
přispěj na pomoc svým služebníkům, 
vždyť jsi je vykoupil svou předrahou krví;

dej, ať se radují s~tvými svatými 
ve věčné slávě.

Zachraň, Pane, svůj lid, žehnej svému dědictví, 
veď ho a~stále pozvedej.

Každý den tě budeme velebit 
a~chválit tvé jméno po všechny věky.

Pomáhej nám i~dnes, 
ať se nedostaneme do područí hříchu.

Smiluj se nad námi, Pane, 
smiluj se nad námi.

Ať spočine na nás tvé milosrdenství, 
jak doufáme v~tebe.

Pane, k~tobě se utíkáme, 
ať nejsme zahanbeni na věky. 
\end{translatioMulticol}}

\newcommand{\trMatEvangelium}{\translatioCantus{
  Rodokmen Ježíše Krista, syna Davidova, syna Abrahámova:
  Abrahám zplodil Izáka,
  Izák zplodil Jakuba,
  Jakub zplodil Judu a jeho bratry,
  Juda zplodil Farese a Zaru z Tamary,
  Fares zplodil Esroma,
  Esrom zplodil Arama,
  Aram zplodil Aminadaba,
  Aminadab zplodil Naasona,
  Naason zplodil Salmona,
  Salmon zplodil Boaze z Rahaby,
  Boaz zplodil Jobeda z Rut,
  Jobed zplodil Jessea,
  Jesse zplodil krále Davida.
  David zplodil Šalomouna z Uriášovy ženy,
  Šalomoun zplodil Roboama,
  Roboam zplodil Abiu,
  Abia zplodil Asu,
  Asa zplodil Josafata,
  Josafat zplodil Jorama,
  Joram zplodil Oziáše,
  Oziáš zplodil Joatama,
  Joatam zplodil Achaze,
  Achaz zplodil Ezechiáše,
  Ezechiáš zplodil Manasesa,
  Manases zplodil Amona,
  Amon zplodil Josiáše,
  Josiáš zplodil Jechoniáše a jeho bratry;
  tehdy došlo k odvlečení do Babylonu.
  Po odvlečení do Babylonu:
  Jechoniáš zplodil Salatiela,
  Salatiel zplodil Zorobabela,
  Zorobabel zplodil Abiuda,
  Abiud zplodil Eljakima,
  Eljakim zplodil Azora,
  Ator zplodil Sadoka,
  Sadok zplodil Achima,
  Achim zplodil Eliuda,
  Eliud zplodil Eleazara,
  Eleatar zplodil Matana,
  Matan zplodil Jakuba,
  Jakub zplodil Josefa, manžela Marie,
  z níž se narodil Ježíš, který se nazývá Kristus.}}

\newcommand{\trTeDecetLaus}{\translatioCantus{Tobě chvála, Tobě zpěvy, Tobě
sláva, Bohu Otci i~Synu i~Svatému Duchu, na věky věků. \Rbardot{} Amen.}}

% MASS ---

\newcommand{\trIntroitus}{\translatioCantus{Radujme se všichni
v Pánu, slavíce svátek ke cti Panny Marie: z něj se radují andělé
a spoluchválí Božího Syna. \textit{\color{red}Žl.} Má ústa vydala dobré slovo,
přednáším svá díla králi.}}

\newcommand{\trGraduale}{\translatioCantus{Požehnaná a ctihodná jsi,
Panno Maria: nedotčená (co do panenství) jsi byla shledána matkou
Spasitele. \Vbardot{} Panno Boží Rodičko, ten, jehož nepojme ani celý svět,
se uzavřel do tvých útrob, když se stal člověkem.}}

\newcommand{\trAlleluia}{\translatioCantus{Aleluja. \Vbardot{} Skvělá slavnost
slavné Panny Marie, z pokolení (dosl. ze semene) Abrahámova, vzešlé z kmene 
Judova, z rodu Davidova.}}

\newcommand{\trOffertorium}{\translatioCantus{Blažená jsi, Panno Maria,
tys nosila Stvořitele všeho; porodila jsi toho, který tě utvořil,
a na věky zůstáváš Pannou.}}

\newcommand{\trCommunio}{\translatioCantus{Budou mě blahoslavit
všechna pokolení, protože mi učinil veliké věci ten, který je mocný.}}

% LITTLE HOURS ---

\newcommand{\trVersusTertia}{\translatioCantus{\Vbardot{} \Rbardot{}}}

\newcommand{\trCapituliEtSic}{\translatioCantus{
Tak jsem se usadila na Sionu a v milovaném městě jsem nalezla odpočinek,
v Jeruzalémě vykonávám svou moc.
Zakořenila jsem u lidu plného slývy, na panství Páně, v jeho dědictví.}}

\newcommand{\trVersusSexta}{\translatioCantus{\Vbardot{} \Rbardot{}}}

\newcommand{\trCapituliInPlateis}{\translatioCantus{
Na planině jako skořicovník a akant jsem vydávala vůni, jako vybraná myrha
jsem voněla.}}

\newcommand{\trVersusNona}{\translatioCantus{\Vbardot{} \Rbardot{}}}
 % Czech translations of the proper texts

\newcommand{\annusEditionis}{2020}

\def\hebinitial#1{%
\leavevmode{\newbox\hebbox\setbox\hebbox\hbox{\hebfont{#1}\hskip 1mm}\kern -\wd\hebbox\hbox{\hebfont{#1}\hskip 1mm}}%
}

%%%% Vicekrat opakovane kousky

\newcommand{\anteOrationem}{
  \rubrica{Ante Orationem, cantatur a Superiore:}

  \pars{Supplicatio Litaniæ.}

  \cuminitiali{}{temporalia/supplicatiolitaniae.gtex}

  \pars{Oratio Dominica.}

  \cuminitiali{}{temporalia/oratiodominica.gtex}

  \rubrica{Deinde dicitur ab Hebdomadario:}

  \cuminitiali{}{temporalia/dominusvobiscum-solemnis.gtex}

  \rubrica{In choro monialium loco Dominus vobiscum dicitur:}

  \sineinitiali{temporalia/domineexaudi.gtex}
}

\setlength{\columnsep}{30pt} % prostor mezi sloupci

%%%%%%%%%%%%%%%%%%%%%%%%%%%%%%%%%%%%%%%%%%%%%%%%%%%%%%%%%%%%%%%%%%%%%%%%%%%%%%%%%%%%%%%%%%%%%%%%%%%%%%%%%%%%%
\begin{document}

% Here we set the space around the initial.
% Please report to http://home.gna.org/gregorio/gregoriotex/details for more details and options
\grechangedim{afterinitialshift}{2.2mm}{scalable}
\grechangedim{beforeinitialshift}{2.2mm}{scalable}

\grechangedim{interwordspacetext}{0.22 cm plus 0.15 cm minus 0.05 cm}{scalable}%
\grechangedim{annotationraise}{-0.2cm}{scalable}

% Here we set the initial font. Change 38 if you want a bigger initial.
% Emit the initials in red.
\grechangestyle{initial}{\color{red}\fontsize{38}{38}\selectfont}

\pagestyle{empty}

%%%% Titulni stranka
\begin{titulusOfficii}
\nomenFesti{Feria IV infra Hebdom. Ultima Adventus.}
\end{titulusOfficii}

\pars{}

\scriptura{}

\pagebreak

% graphic
\renewcommand{\headrulewidth}{0pt} % no horiz. rule at the header
\fancyhf{}
\pagestyle{fancy}

\cantusSineNeumas

\hora{Ad Matutinum.}

\vspace{2mm}

\cuminitiali{}{temporalia/dominelabiamea.gtex}

\vspace{2mm}

\pars{Invitatorium.} \scriptura{Phil. 4, 4.5}

\vspace{-6mm}

\antiphona{VI}{temporalia/inv-propeestiamsimplex.gtex}

\vfill
\pagebreak

\pars{Hymnus.}

\vspace{-5mm}

\antiphona{II}{temporalia/hym-VeniRedemptor.gtex}
%{
%\vspace{-5mm}
%\setlength{\columnsep}{0pt} % prostor mezi sloupci
%\input{hym-VeniRedemptor-bohtext.tex}
%\setlength{\columnsep}{30pt} % prostor mezi sloupci
%}

\vfill
\pagebreak

% MB

\pars{Psalmus 1.} \scriptura{Ps. 38, 2; \textbf{H93}}

\vspace{-4mm}

\antiphona{IV E}{temporalia/ant-utnondelinquam.gtex}

%\vspace{-5mm}

\scriptura{Ps. 38, 2-7}

%\vspace{-2mm}

\initiumpsalmi{temporalia/ps38i-initium-iv-E-auto.gtex}

%\psalmusEtTranslatioT{temporalia/ps38i-IV-comb.tex}{10cm}
\input{temporalia/ps38i-IV.tex} \Abardot{}

\vfill
\pagebreak

\pars{Psalmus 2.} \scriptura{Ier. 17, 17; \textbf{H177}}

\vspace{-4mm}

\antiphona{VII c}{temporalia/ant-nonsismihi.gtex}

%\vspace{-5mm}

\scriptura{Ps. 38, 8-14}

%\vspace{-2mm}

\initiumpsalmi{temporalia/ps38ii-initium-vii-c-auto.gtex}

%\psalmusEtTranslatioT{temporalia/ps38ii-IV-comb.tex}{10cm}
\input{temporalia/ps38ii-IV.tex} \Abardot{}

\vfill
\pagebreak

\pars{Psalmus 3.}

\vspace{-4mm}

\antiphona{IV e}{temporalia/ant-exspectabonomentuum.gtex}

%\vspace{-5mm}

\scriptura{Ps. 51}

%\vspace{-2mm}

\initiumpsalmi{temporalia/ps51-initium-iv-e-auto.gtex}

%\psalmusEtTranslatioT{temporalia/ps51-IV-comb.tex}{10cm}
\input{temporalia/ps51-IV.tex} \Abardot{}

\vfill
\pagebreak

\pars{Psalmus 4.} \scriptura{Ps. 102, 1; \textbf{H99}}

\vspace{-4mm}

\antiphona{VIII c}{temporalia/ant-benedicanimamea.gtex}

%\vspace{-5mm}

\scriptura{Ps. 102, 1-7}

%\vspace{-2mm}

\initiumpsalmi{temporalia/ps102i-initium-viii-C-auto.gtex}

%\psalmusEtTranslatioT{temporalia/ps102i-IV-comb.tex}{10cm}
\input{temporalia/ps102i-IV.tex} \Abardot{}

\vfill
\pagebreak

\pars{Psalmus 5.} \scriptura{Ps. 102, 11}

\vspace{-4mm}

\antiphona{I D}{temporalia/ant-supertimentesdominum.gtex}

%\vspace{-5mm}

\scriptura{Ps. 102, 8-16}

%\vspace{-2mm}

\initiumpsalmi{temporalia/ps102ii-initium-i-D-auto.gtex}

%\psalmusEtTranslatioT{temporalia/ps102ii-IV-comb.tex}{10cm}
\input{temporalia/ps102ii-IV.tex} \Abardot{}

\vfill
\pagebreak

\pars{Psalmus 6.} \scriptura{Ps. 102, 20; \textbf{H332}}

\vspace{-4mm}

\antiphona{III g}{temporalia/ant-benedicitedomino.gtex}

%\vspace{-5mm}

\scriptura{Ps. 102, 17-22}

%\vspace{-2mm}

\initiumpsalmi{temporalia/ps102iii-initium-iii-g.gtex}

%\psalmusEtTranslatioT{temporalia/ps102iii-IV-comb.tex}{10cm}
\input{temporalia/ps102iii-IV.tex} \Abardot{}

\vfill
\pagebreak

\pars{Versus.} \scriptura{Mc. 1, 3; Is. 40, 3}

% Versus. %%%
\sineinitiali{temporalia/versus-voxclamantis-simplex.gtex}

\vspace{5mm}

\sineinitiali{temporalia/oratiodominica-mat.gtex}

\vspace{5mm}

\pars{Absolutio.}

\cuminitiali{}{temporalia/absolutio-avinculis.gtex}

\vfill
\pagebreak

\cuminitiali{}{temporalia/benedictio-solemn-ille.gtex}

\vspace{7mm}

\lectioi

\noindent \Vbardot{} Tu autem, Dómine, miserére nobis.
\noindent \Rbardot{} Deo grátias.

\vfill
\pagebreak

\responsoriumi

\vfill
\pagebreak

\cuminitiali{}{temporalia/benedictio-solemn-divinum.gtex}

\vspace{7mm}

\lectioii

\noindent \Vbardot{} Tu autem, Dómine, miserére nobis.
\noindent \Rbardot{} Deo grátias.

\vfill
\pagebreak

\responsoriumii

\vfill
\pagebreak

\cuminitiali{}{temporalia/benedictio-solemn-ignem.gtex}

\vspace{7mm}

\lectioiii

\noindent \Vbardot{} Tu autem, Dómine, miserére nobis.
\noindent \Rbardot{} Deo grátias.

\vfill
\pagebreak

\responsoriumiii

\vfill
\pagebreak

\rubrica{Reliqua omittuntur, nisi Laudes separandæ sint.}

\pars{Oratio}

\noindent \Vbardot{} Dómine, exáudi oratiónem meam.

\noindent \Rbardot{} Et clamor meus ad te véniat.

\oratio

\vspace{7mm}

\pars{Conclusio}

\noindent \Vbardot{} Dómine, exáudi oratiónem meam.

\noindent \Rbardot{} Et clamor meus ad te véniat.

\noindent \Vbardot{} Benedicámus Dómino.

\noindent \Rbardot{} Deo grátias.

\noindent \Vbardot{} Fidélium ánimæ per misericórdiam Dei requiéscant in pace.

\noindent \Rbardot{} Amen.

\vfill
\pagebreak

\hora{Ad Laudes.} %%%%%%%%%%%%%%%%%%%%%%%%%%%%%%%%%%%%%%%%%%%%%%%%%%%%%
%\sideThumbs{Laudes}

\cantusSineNeumas

\vspace{0.5cm}
\grechangedim{interwordspacetext}{0.18 cm plus 0.15 cm minus 0.05 cm}{scalable}%
\cuminitiali{}{temporalia/deusinadiutorium-communis.gtex}
\grechangedim{interwordspacetext}{0.22 cm plus 0.15 cm minus 0.05 cm}{scalable}%

\vfill
%\pagebreak

\pars{Psalmus 1.} \scriptura{Ps. 50, 4; \textbf{H95}}

\vspace{-4mm}

\antiphona{VII a}{temporalia/ant-ampliuslavame.gtex}

\scriptura{Psalmus 50.}

\initiumpsalmi{temporalia/ps50-initium-vii-a-auto.gtex}

%\psalmusEtTranslatioT{temporalia/ps50-IV-comb.tex}{10cm}
\input{temporalia/ps50-IV.tex}

\vfill

\antiphona{}{temporalia/ant-ampliuslavame.gtex}

\vfill
\pagebreak

\pars{Psalmus 2.} \scriptura{Ps. 63, 2; \textbf{H96}}

\vspace{-5mm}

\antiphona{II D}{temporalia/ant-atimore.gtex}

\scriptura{Psalmus 63.}

\initiumpsalmi{temporalia/ps63-initium-ii-D-auto.gtex}

%\psalmusEtTranslatioT{temporalia/ps63-IV-comb.tex}{10cm}
\input{temporalia/ps63-IV.tex} \Abardot{}

\vfill
\pagebreak

\pars{Psalmus 3.} \scriptura{Ps. 64, 2; \textbf{H96}}

\vspace{-4mm}

\antiphona{VIII a}{temporalia/ant-tedecet.gtex}

\vspace{-2mm}

\scriptura{Psalmus 64.}

\vspace{-2mm}

\initiumpsalmi{temporalia/ps64-initium-viii-A-auto.gtex}

%\psalmusEtTranslatioT{temporalia/ps64-IV-comb.tex}{10cm}
\input{temporalia/ps64-IV.tex} \Abardot{}

\vfill
\pagebreak

\pars{Psalmus 4.} \scriptura{1 Sam. 2, 10; \textbf{H96}}

\vspace{-7mm}

\antiphona{I g\textsuperscript{2}}{temporalia/ant-dominusjudicabit.gtex}

%\vspace{-4mm}

\scriptura{Canticum Annæ, 1 Reg. 2, 1-10}

%\vspace{-3mm}

\initiumpsalmi{temporalia/anna-initium-i-g2-auto.gtex}

%\psalmusEtTranslatioT{temporalia/anna-comb.tex}{10cm}
\input{temporalia/anna.tex}

%\vfill

\antiphona{}{temporalia/ant-dominusjudicabit.gtex}

\vfill
\pagebreak

\pars{Psalmus 5.} \scriptura{Ps. 148, 4; \textbf{H96}}

\vspace{-4mm}

\antiphona{II D}{temporalia/ant-caelicaelorum.gtex}

\scriptura{Psalmus 148.}

\initiumpsalmi{temporalia/ps148-initium-ii-D-auto.gtex}

%\psalmusEtTranslatioT{temporalia/ps148-IV-comb.tex}{10cm}
\input{temporalia/ps148-IV.tex}

\rubrica{Hic non dicitur Gloria Patri.}

\vfill
\pagebreak

%
\scriptura{Psalmus 149.}

\initiumpsalmi{temporalia/ps149-initium-ii-D-auto.gtex}

%\psalmusEtTranslatioT{temporalia/ps149-IV-comb.tex}{10cm}
\input{temporalia/ps149-IV.tex}

\rubrica{Hic non dicitur Gloria Patri.}

\vfill
\pagebreak

%
\scriptura{Psalmus 150.}

\initiumpsalmi{temporalia/ps150-initium-ii-D-auto.gtex}

%\psalmusEtTranslatioT{temporalia/ps150-IV-comb.tex}{10cm}
\input{temporalia/ps150-IV.tex}

\vfill

\vspace{-6mm}

\antiphona{}{temporalia/ant-caelicaelorum.gtex} % repeat the antiphon - new page

\vfill
\pagebreak

\lectiobrevis

% preklad Jeruz. bible
%\trCapituliI

\vfill

\responsoriumbreve

%\trResp

\vfill
\pagebreak

\pars{Hymnus} \scriptura{Ambrosius (\olddag{} 397)}

\cuminitiali{I}{temporalia/hym-VoxClara.gtex}
\vspace{-3mm}
%\begin{translatioMulticol}{3}
Jasný hlas k~nebi zní\\
temnoty všechny zahání\\
a~trestá sen, ten prchá, jak\\
z~výšin se Kristův zaskví zrak\\
\\
Probuď se mysli strnulá,\\
hluchotou povstaň zraněná,\\
nová se hvězda zažíhá,\\
škody od tebe odnímá.\columnbreak

Shůry přichází Beránek\\
polehčit vinným jejich stesk,\\
v~slzách o~milost volejme,\\
smilování si žádejme.\\
\\
Až pak podruhé zabuší\\
a~svět jeho hrůza zkruší\\
za hříchy nás nepotrestal\\
a~ochranu svou by nám dal.\columnbreak

Chválu, sílu, čest a~slávu,\\
Bohu Otci, jeho Synu,\\
Duchu také rady svaté\\
vzdej na věky věků světe.\\
Amen.
\end{translatioMulticol}


\vfill
%\pagebreak

\pars{Versus.} \scriptura{Mc. 1, 3; Is. 40, 3}

% Versus. %%%
\sineinitiali{temporalia/versus-voxclamantis.gtex}

%\noindent \trVersus

\vfill
\pagebreak

\benedictus

\vfill
\pagebreak

%\sideThumbs{{\scriptsize{}Fine horarum}}

\ifx\preces\undefined
\rubrica{Ante Orationem, cantatur a Superiore:}

\pars{Supplicatio Litaniæ.}

\cuminitiali{}{temporalia/supplicatiolitaniae.gtex}

\pars{Oratio Dominica.}

\cuminitiali{}{temporalia/oratiodominica.gtex}
\else
\pars{Preces.}

\sineinitiali{}{temporalia/tonusprecum.gtex}

\preces

\vfill

\pars{Oratio Dominica.}

\cuminitiali{}{temporalia/oratiodominicaalt.gtex}

\vfill
\pagebreak

\rubrica{vel:}

\pars{Supplicatio Litaniæ.}

\cuminitiali{}{temporalia/supplicatiolitaniae.gtex}

\vfill

\pars{Oratio Dominica.}

\cuminitiali{}{temporalia/oratiodominica.gtex}
\fi

\vfill
\pagebreak

% Oratio. %%%
\oratio

\vspace{-1mm}
%\trOrationisI

\vfill

\rubrica{Hebdomadarius dicit Dominus vobiscum, vel, absente sacerdote vel diacono, sic concluditur:}

\vspace{2mm}

\antiphona{C}{temporalia/dominusnosbenedicat.gtex}

\rubrica{Postea cantatur a cantore:}

\vspace{2mm}

\cuminitiali{IV}{temporalia/benedicamus-feria-advequad.gtex}

\vspace{1mm}

\vfill
\pagebreak

\hora{Ad Vesperas.} %%%%%%%%%%%%%%%%%%%%%%%%%%%%%%%%%%%%%%%%%%%%%%%%%%%%%
%\sideThumbs{Vesperæ}

\cantusSineNeumas

%\vspace{0.5cm}
\grechangedim{interwordspacetext}{0.18 cm plus 0.15 cm minus 0.05 cm}{scalable}%
\cuminitiali{}{temporalia/deusinadiutorium-communis.gtex}
\grechangedim{interwordspacetext}{0.22 cm plus 0.15 cm minus 0.05 cm}{scalable}%

\vfill
%\pagebreak

\vspace{4mm}

\pars{Psalmus 1.} \scriptura{Ps. 134, 6; \textbf{H97}}

\vspace{-4mm}

\antiphona{III g}{temporalia/ant-omniaquaecumque.gtex}

\vspace{-4mm}

\scriptura{Psalmus 134.}

\initiumpsalmi{temporalia/ps134-initium-iii-g-auto.gtex}

%\psalmusEtTranslatioT{temporalia/ps134-IV-comb.tex}{10cm}
\input{temporalia/ps134-IV.tex}

\vfill

\antiphona{}{temporalia/ant-omniaquaecumque.gtex}

\vfill
\pagebreak

\pars{Psalmus 2.} \scriptura{Ps. 135, 1; \textbf{H97}}

\vspace{-4mm}

\antiphona{III g}{temporalia/ant-quoniaminaeternum.gtex}

\scriptura{Psalmus 135.}

\initiumpsalmi{temporalia/ps135-initium-iii-g-auto.gtex}

%\psalmusEtTranslatioT{temporalia/ps135-IV-comb.tex}{10cm}
\input{temporalia/ps135-IV.tex}

\vfill

\antiphona{}{temporalia/ant-quoniaminaeternum.gtex}

\vfill
\pagebreak

\pars{Psalmus 3.} \scriptura{Ps. 136, 3; \textbf{H97}}

\vspace{-4mm}

\antiphona{VIII G}{temporalia/ant-hymnumcantate.gtex}

\scriptura{Psalmus 136.}

\initiumpsalmi{temporalia/ps136-initium-viii-G-auto.gtex}

%\psalmusEtTranslatioT{temporalia/ps136-IV-comb.tex}{10cm}
\input{temporalia/ps136-IV.tex} \Abardot{}

\vfill
\pagebreak

\pars{Psalmus 4.} \scriptura{Ps. 137, 1; \textbf{H97}}

\vspace{-4mm}

\antiphona{V a}{temporalia/ant-inconspectuangelorum.gtex}

\scriptura{Psalmus 137.}

\initiumpsalmi{temporalia/ps137-initium-v-a-auto.gtex}

%\psalmusEtTranslatioT{temporalia/ps137-IV-comb.tex}{10cm}
\input{temporalia/ps137-IV.tex} \Abardot{}

\vfill
\pagebreak

\pars{Capitulum.} \scriptura{Gen. 49, 10}

\grechangedim{interwordspacetext}{0.12 cm plus 0.15 cm minus 0.05 cm}{scalable}%
\cuminitiali{}{temporalia/capitulum-NosAuferetur.gtex}
\grechangedim{interwordspacetext}{0.22 cm plus 0.15 cm minus 0.05 cm}{scalable}%

% preklad Jeruz. bible
%\trCapituliI

\vfill

\pars{Responsorium breve.} \scriptura{Ps. 84, 8; \textbf{H20}}

\cuminitiali{IV}{temporalia/resp-ostendenobis.gtex}

%\trResp

\vfill
\pagebreak

\pars{Hymnus} \scriptura{Ambrosius (\olddag{} 397)}

\cuminitiali{IV}{temporalia/hym-ConditorAlme.gtex}
\vspace{-3mm}
%\begin{translatioMulticol}{3}
Štědrý nebes stvořiteli\\
Věčný osvítiteli\\
Kriste vykupiteli\\
prosby vyslyš věrných milý.\\
\\
Nad zánikem věku v~smrti\\
smiloval ses, milý choti\\
a~světu spásy lačnému\\
lék jsi přinesl vinnému.\columnbreak

Když se večer světa chýlí,\\
vyšels jako ženich milý\\
z~domu matky, panny čestné,\\
z~její komnaty milostné.\\
\\
Mocí tvojí velmi silnou\\
kolena se k~zemi ohnou\\
a~nebe, země tvorové\\
se poddávají vůli tvé.\columnbreak

Jenž přicházíš věku soudce,\\
prosíme tě světovládce,\\
uchovej nás v~našem čase\\
před úkladnou ranou zhoubce.\\
\\
Chválu, sílu, čest a~slávu,\\
Bohu Otci, jeho Synu,\\
Duchu také rady svaté\\
vzdej na věky věků světe.\\
Amen.
\end{translatioMulticol}


\vfill
%\pagebreak

\pars{Versus.} \scriptura{Is. 45, 8}

% Versus. %%%
\sineinitiali{temporalia/versus-rorate.gtex}

%\noindent \trVersus

\vfill
\pagebreak

\magnificat

\vfill
\pagebreak

%\sideThumbs{{\scriptsize{}Fine horarum}}

\anteOrationem

\pagebreak

% Oratio. %%%
\oratioLaudes

\vspace{-1mm}
%\trOrationisI

\vfill

\rubrica{Hebdomadarius dicit iterum Dominus vobiscum, vel cantor dicit:}

\vspace{2mm}

\sineinitiali{temporalia/domineexaudi.gtex}

\rubrica{Postea cantatur a cantore:}

\vspace{2mm}

\cuminitiali{IV}{temporalia/benedicamus-feria-advequad.gtex}

\vspace{1mm}

\end{document}

