\newcommand{\oratio}{\pars{Oratio.}

\noindent Adésto, Dómine, supplícibus tuis et spem suam in tua misericórdia collocántes tuére propítius, ut, a peccatórum labe mundáti, in sancta conversatióne permáneant et promissiónis tuæ perficiántur herédes.

\pars{Pro pace in Ucraina.} \scriptura{Sir. 50, 25; 2 Esdr. 4, 20; \textbf{H416}}

\vspace{-4mm}

\antiphona{II D}{temporalia/ant-dapacemdomine.gtex}

\vfill

\noindent Deus, a quo sancta desidéria, recta consília et iusta sunt ópera: da servis tuis illam, quam mundus dare non potest, pacem; ut et corda nostra mandátis tuis dédita, et hóstium subláta formídine, témpora sint tua protectióne tranquílla.

\noindent Per Dóminum nostrum Iesum Christum, Fílium tuum, qui tecum vivit et regnat in unitáte Spíritus Sancti, Deus, per ómnia sǽcula sæculórum.

\noindent \Rbardot{} Amen.}
\newcommand{\invitatorium}{\pars{Invitatorium.} \scriptura{Ps. 94, 8; Psalmus 94; \textbf{H143}}

\vspace{-4mm}

\antiphona{E}{temporalia/inv-hodiesivocem.gtex}}
\newcommand{\hymnusmatutinum}{\pars{Hymnus} \scriptura{Venantius Fortunatus (sæc. VI)}

\cuminitiali{I}{temporalia/hym-PangeLingua.gtex}}
\newcommand{\matversus}{\noindent \Vbardot{} Qui meditátur in lege Dómini. 

\noindent \Rbardot{} Dabit fructum suum in témpore suo.}
\newcommand{\lectioi}{\vspace{-4mm}

\pars{Lectio I.} \scriptura{Ier. 18, 1-12}

\noindent De libro Ieremíæ Prophétæ.

\noindent Verbum, quod factum est ad Ieremíam a Dómino dicens: «Surge et descénde in domum fíguli et ibi áudies verba mea».  Et descéndi in domum fíguli, et ecce ipse faciébat opus super rotam; et dissipátum est vas, quod ipse faciébat e luto mánibus suis, et rursus fecit illud vas álterum, sicut placúerat in óculis eius, ut fáceret.

\noindent Et factum est verbum Dómini ad me dicens: «Numquid sicut fígulus iste non pótero vobis fácere, domus Israel?, ait Dóminus. Ecce, sicut lutum in manu fíguli, sic vos in manu mea, domus Israel. Repénte loquar advérsum gentem et advérsum regnum, ut eradícem et déstruam et dispérdam illud; si pæniténtiam égerit gens illa a malo suo, propter quod locútus sum advérsus eam, agam et ego pæniténtiam super malo, quod cogitávi ut fácerem ei. Et subíto loquar de gente et de regno, ut ædíficem et plantem illud; si fécerit malum in óculis meis, ut non áudiat vocem meam, pæniténtiam agam super bono, quod locútus sum ut fácerem ei. Nunc ergo, dic viro Iudæ et habitatóribus Ierúsalem dicens: Hæc dicit Dóminus: Ecce ego fingo contra vos malum et cógito contra vos cogitatiónem; revertátur unusquísque a via sua mala, et dirígite vias vestras et ópera vestra». Qui dixérunt: «Vanum est; post cogitatiónes enim nostras íbimus et unusquísque pravitátem cordis sui mali faciémus».}
\newcommand{\responsoriumi}{\pars{Responsorium 1.} \scriptura{\Rbardot{} Ps. 21, 11 \Vbardot{} ibid., 21; \textbf{H166}}

\vspace{-5mm}

\responsorium{II}{temporalia/resp-inteiactatussum-CROCHU.gtex}{}}
\newcommand{\lectioii}{\pars{Lectio II.} \scriptura{Hom. 18,1-2: PL 76,1150-1151}

\noindent Ex Homíliis sancti Gregórii Magni papæ in Evangélia.

\noindent Pensáte, fratres caríssimi, mansuetúdinem Dei. Relaxáre peccáta vénerat et dicébat: \emph{Quis ex vobis árguet me de peccáto?} Non dedignátur ex ratióne osténdere se peccatórem non esse, qui ex virtúte divinitátis póterat peccatóres iustificáre. Sed terríbile est valde quod súbditur \emph{Qui ex Deo est verba Dei audit; proptérea vos non audítis, quia ex Deo non estis.} Si enim ipse verba Dei audit qui ex Deo est, et audíre verba eius non potest quisquis ex illo non est, intérroget se unusquísque si verba Dei in aure cordis pércipit, et intélleget unde sit. Cæléstem pátriam desideráre Véritas iubet, carnis desidéria cónteri, mundi glóriam declináre, aliéna non appétere, própria largíri.

\noindent Penset ergo apud se unusquísque vestrum si hæc vox Dei in cordis eius aure conváluit, et quia iam ex Deo sit agnóscit. Nam sunt nonnúlli qui præcépta Dei nec aure córporis percípere dignántur. Et sunt nonnúlli qui hæc quidem aure percípiunt, sed nullo ea mentis desidério complectúntur. Et sunt nonnúlli qui libénter verba Dei suscípiunt, ita ut étiam in flétibus compungántur, sed post lacrymárum tempus ad iniquitátem rédeunt. Hi profécto verba Dei non áudiunt, qui hæc exercére ópere contémnunt.}
\newcommand{\responsoriumii}{\pars{Responsorium 2.} \scriptura{\Rbardot{} Ps. 118, 114 \Vbardot{} Ps. 58, 2; \textbf{H165}}

\vspace{-5mm}

\responsorium{IV}{temporalia/resp-adiutoretsusceptormeus-CROCHU.gtex}{}}
\newcommand{\lectioiii}{\pars{Lectio III.}

\noindent Vitam ergo vestram, fratres caríssimi, ante mentis óculos revocáte, et alta consideratióne pertiméscite hoc quod ex ore Veritátis sonat: \emph{Proptérea vos non audítis, quia ex Deo non estis.} Sed hoc quod de réprobis Véritas lóquitur, ipsi hoc de semetípsis réprobi suis opéribus osténdunt. Nam séquitur: \emph{Respondérunt ígitur Iudǽi et dixérunt ei: «Nonne bene dícimus nos quia Samaritánus es tu, et dæmónium habes?»}

\noindent Accépta autem tanta contumélia, quid Dóminus respóndeat audiámus: \emph{Ego dæmónium non hábeo, sed honorífico Patrem meum et vos inhonorástis me.} Quia enim Samaritánus interpretátur custos, et ipse veráciter custos est de quo Psalmísta ait: \emph{Nisi Dóminus custodíerit civitátem, in vanum vígilant qui custódiunt eam;} et cui per Isaíam dícitur: \emph{Custos, quid de nocte? Custos, quid de nocte?} respondére nóluit Dóminus «Samaritánus non sum»; sed: \emph{Ego dæmónium non hábeo.} Duc quippe ei illáta fuérunt: unum negávit, áliud tacéndo consénsit.}
\newcommand{\responsoriumiii}{\pars{Responsorium 3.} \scriptura{\Rbardot{} Ps. 21, 11 \Vbardot{} ibid., 2; \textbf{H165}}

\vspace{-5mm}

\responsorium{VIII}{temporalia/resp-deusmeusestune-CROCHU.gtex}{}

\rubrica{Omittitur Versus \textnormal{Gloria Patri}, repetitur integrum Responsorium usque ad Versum.}}
\newcommand{\lectiobrevis}{\pars{Lectio Brevis.} \scriptura{Hebr. 2, 9-10}

\noindent Vidémus Iesum propter passiónem mortis glória et honóre coronátum, ut grátia Dei pro ómnibus gustáverit mortem. Decébat enim eum, propter quem ómnia et per quem ómnia, qui multos fílios in glóriam addúxit, auctórem salútis eórum per passiónes consummáre.}
\newcommand{\responsoriumbreve}{\pars{Responsorium breve.} \scriptura{Ps. 21, 21}

\cuminitiali{IV}{temporalia/resp-erueaframea-feriale.gtex}}
\newcommand{\hymnuslaudes}{\pars{Hymnus} \scriptura{Venantius Fortunatus (sæc. VI)}

\cuminitiali{I}{temporalia/hym-HicAccetum.gtex}}
\newcommand{\preces}{\noindent Benedicátur Auctor salútis nostræ, qui vult hómines fíeri in se novam creatúram, ut vétera tránseant et ómnia renovéntur. \gredagger{} Quaprópter viva spe fulti eum rogémus:

\Rbardot{} Rénova nos, Dómine, in Spíritu tuo.

\noindent Dómine, qui cælum novum terrámque novam promisísti, semper nos rénova per Spíritum tuum, \gredagger{} ut in cælésti Ierúsalem te iúgiter perfruámur.

\Rbardot{} Rénova nos, Dómine, in Spíritu tuo.

\noindent Da nos tecum operári ut hunc mundum Spíritu tuo imbuámus \gredagger{} atque in iustítia, caritáte et pace cívitas terréna finem suum efficácius assequátur.

\Rbardot{} Rénova nos, Dómine, in Spíritu tuo.

\noindent Tríbue nos omnes desídias et neglegéntias castigáre \gredagger{} et supérnis delectári munéribus.

\Rbardot{} Rénova nos, Dómine, in Spíritu tuo.

\noindent Líbera nos a malo \gredagger{} nosque a fascinatióne nugacitátis, quæ bona obscúrat, defénde.

\Rbardot{} Rénova nos, Dómine, in Spíritu tuo.}
\newcommand{\benedictus}{\pars{Canticum Zachariæ.} \scriptura{Io. 8, 56; \textbf{H167}}

\vspace{-4mm}

{
\grechangedim{interwordspacetext}{0.18 cm plus 0.15 cm minus 0.05 cm}{scalable}%
\antiphona{II D}{temporalia/ant-abrahampatervester.gtex}
\grechangedim{interwordspacetext}{0.22 cm plus 0.15 cm minus 0.05 cm}{scalable}%
}

\vspace{-2mm}

\scriptura{Lc. 1, 68-79}

\vspace{-2mm}

\initiumpsalmi{temporalia/benedictus-initium-ii-D-auto.gtex}

%\vspace{-1mm}

\input{temporalia/benedictus-ii-D.tex} \Abardot{}}
\newcommand{\magnificat}{\pars{Canticum B. Mariæ V.} \scriptura{Lc. 22, 15; \textbf{H169}}

\vspace{-4mm}

{
\grechangedim{interwordspacetext}{0.18 cm plus 0.15 cm minus 0.05 cm}{scalable}%
\antiphona{\textit{IV A*}}{temporalia/ant-desideriodesideravi.gtex}
\grechangedim{interwordspacetext}{0.22 cm plus 0.15 cm minus 0.05 cm}{scalable}%
}

%\vspace{-2mm}

\scriptura{Lc. 1, 46-55}

%\vspace{-2mm}

\cantusSineNeumas
\initiumpsalmi{temporalia/magnificat-initium-iv-A_.gtex}

%\vspace{-1.5mm}

\input{temporalia/magnificat-iv-A_.tex} \Abardot{}}
\newcommand{\oratiovesperas}{\pars{Oratio.}

\noindent Esto, quǽsumus Dómine, propítius plebi tuæ: \grestar{} ut quæ tibi non placent respuéntes, tuórum pótius repleántur delectatiónibus mandatórum.

\noindent Per Dóminum nostrum Iesum Christum, Fílium tuum, qui tecum vivit et regnat in unitáte Spíritus Sancti, Deus, per ómnia sǽcula sæculórum.

\noindent \Rbardot{} Amen.}
\newcommand{\precestotum}{\pars{Deprecatio Gelasii}

\vspace{-5mm}

\grechangedim{interwordspacetext}{0.16 cm plus 0.15 cm minus 0.05 cm}{scalable}%
\antiphona{D\textsuperscript{1}}{temporalia/deprecatio4-propace.gtex}
\grechangedim{interwordspacetext}{0.22 cm plus 0.15 cm minus 0.05 cm}{scalable}%

\vfill

\pars{Oratio Dominica.}

\cuminitiali{D}{temporalia/oratiodominica-d.gtex}}
\newcommand{\dominusnosbenedicat}{\antiphona{D}{temporalia/dominusnosbenedicat-d.gtex}}
\newcommand{\hebdomada}{in Rogationibus.}
\newcommand{\oratioMatutinum}{\noindent Deus, a quo bona cuncta procédunt, largíre supplícibus tuis:~\gredagger{} ut cogitémus, te inspiránte, quæ recta sunt;~\grestar{} et, te gubernánte, éadem faciámus. Per Dóminum.}
\newcommand{\oratioLaudes}{\cuminitiali{}{temporalia/oratio5.gtex}}

% LuaLaTeX

\documentclass[a4paper, twoside, 12pt]{article}
\usepackage[latin]{babel}
%\usepackage[landscape, left=3cm, right=1.5cm, top=2cm, bottom=1cm]{geometry} % okraje stranky
%\usepackage[landscape, a4paper, mag=1166, truedimen, left=2cm, right=1.5cm, top=1.6cm, bottom=0.95cm]{geometry} % okraje stranky
\usepackage[landscape, a4paper, mag=1400, truedimen, left=0.5cm, right=0.5cm, top=0.5cm, bottom=0.5cm]{geometry} % okraje stranky

\usepackage{fontspec}
\setmainfont[FeatureFile={junicode.fea}, Ligatures={Common, TeX}, RawFeature=+fixi]{Junicode}
%\setmainfont{Junicode}

% shortcut for Junicode without ligatures (for the Czech texts)
\newfontfamily\nlfont[FeatureFile={junicode.fea}, Ligatures={Common, TeX}, RawFeature=+fixi]{Junicode}

\usepackage{multicol}
\usepackage{color}
\usepackage{lettrine}
\usepackage{fancyhdr}

% usual packages loading:
\usepackage{luatextra}
\usepackage{graphicx} % support the \includegraphics command and options
\usepackage{gregoriotex} % for gregorio score inclusion
\usepackage{gregoriosyms}
\usepackage{wrapfig} % figures wrapped by the text
\usepackage{parcolumns}
\usepackage[contents={},opacity=1,scale=1,color=black]{background}
\usepackage{tikzpagenodes}
\usepackage{calc}
\usepackage{longtable}
\usetikzlibrary{calc}

\setlength{\headheight}{14.5pt}

% Commands used to produce a typical "Conventus" booklet

\newenvironment{titulusOfficii}{\begin{center}}{\end{center}}
\newcommand{\dies}[1]{#1

}
\newcommand{\nomenFesti}[1]{\textbf{\Large #1}

}
\newcommand{\celebratio}[1]{#1

}

\newcommand{\hora}[1]{%
\vspace{0.5cm}{\large \textbf{#1}}

\fancyhead[LE]{\thepage\ / #1}
\fancyhead[RO]{#1 / \thepage}
\addcontentsline{toc}{subsection}{#1}
}

% larger unit than a hora
\newcommand{\divisio}[1]{%
\begin{center}
{\Large \textsc{#1}}
\end{center}
\fancyhead[CO,CE]{#1}
\addcontentsline{toc}{section}{#1}
}

% a part of a hora, larger than pars
\newcommand{\subhora}[1]{
\begin{center}
{\large \textit{#1}}
\end{center}
%\fancyhead[CO,CE]{#1}
\addcontentsline{toc}{subsubsection}{#1}
}

% rubricated inline text
\newcommand{\rubricatum}[1]{\textit{#1}}

% standalone rubric
\newcommand{\rubrica}[1]{\vspace{3mm}\rubricatum{#1}}

\newcommand{\notitia}[1]{\textcolor{red}{#1}}

\newcommand{\scriptura}[1]{\hfill \small\textit{#1}}

\newcommand{\translatioCantus}[1]{\vspace{1mm}%
{\noindent\footnotesize \nlfont{#1}}}

% pruznejsi varianta nasledujiciho - umoznuje nastavit sirku sloupce
% s prekladem
\newcommand{\psalmusEtTranslatioB}[3]{
  \vspace{0.5cm}
  \begin{parcolumns}[colwidths={2=#3}, nofirstindent=true]{2}
    \colchunk{
      \input{#1}
    }

    \colchunk{
      \vspace{-0.5cm}
      {\footnotesize \nlfont
        \input{#2}
      }
    }
  \end{parcolumns}
}

\newcommand{\psalmusEtTranslatio}[2]{
  \psalmusEtTranslatioB{#1}{#2}{8.5cm}
}


\newcommand{\canticumMagnificatEtTranslatio}[1]{
  \psalmusEtTranslatioB{#1}{temporalia/extra-adventum-vespers/magnificat-boh.tex}{12cm}
}
\newcommand{\canticumBenedictusEtTranslatio}[1]{
  \psalmusEtTranslatioB{#1}{temporalia/extra-adventum-laudes/benedictus-boh.tex}{10.5cm}
}

% volne misto nad antifonami, kam si zpevaci dokresli neumy
\newcommand{\hicSuntNeumae}{\vspace{0.5cm}}

% prepinani mista mezi notovymi osnovami: pro neumovane a neneumovane zpevy
\newcommand{\cantusCumNeumis}{
  \setgrefactor{17}
  \global\advance\grespaceabovelines by 5mm%
}
\newcommand{\cantusSineNeumas}{
  \setgrefactor{17}
  \global\advance\grespaceabovelines by -5mm%
}

% znaky k umisteni nad inicialu zpevu
\newcommand{\superInitialam}[1]{\gresetfirstlineaboveinitial{\small {\textbf{#1}}}{\small {\textbf{#1}}}}

% pars officii, i.e. "oratio", ...
\newcommand{\pars}[1]{\textbf{#1}}

\newenvironment{psalmus}{
  \setlength{\parindent}{0pt}
  \setlength{\parskip}{5pt}
}{
  \setlength{\parindent}{10pt}
  \setlength{\parskip}{10pt}
}

%%%% Prejmenovat na latinske:
\newcommand{\nadpisZalmu}[1]{
  \hspace{2cm}\textbf{#1}\vspace{2mm}%
  \nopagebreak%

}

% mode, score, translation
\newcommand{\antiphona}[3]{%
\hicSuntNeumae
\superInitialam{#1}
\includescore{#2}

#3
}
 % Often used macros

\newcommand{\annusEditionis}{2021}

%%%% Vicekrat opakovane kousky

\newcommand{\anteOrationem}{
  \rubrica{Ante Orationem, cantatur a Superiore:}

  \pars{Supplicatio Litaniæ.}

  \cuminitiali{}{temporalia/supplicatiolitaniae.gtex}

  \pars{Oratio Dominica.}

  \cuminitiali{}{temporalia/oratiodominica.gtex}

  \rubrica{Deinde dicitur ab Hebdomadario:}

  \cuminitiali{}{temporalia/dominusvobiscum-solemnis.gtex}

  \rubrica{In choro monialium loco Dominus vobiscum dicitur:}

  \sineinitiali{temporalia/domineexaudi.gtex}
}

\setlength{\columnsep}{30pt} % prostor mezi sloupci

%%%%%%%%%%%%%%%%%%%%%%%%%%%%%%%%%%%%%%%%%%%%%%%%%%%%%%%%%%%%%%%%%%%%%%%%%%%%%%%%%%%%%%%%%%%%%%%%%%%%%%%%%%%%%
\begin{document}

% Here we set the space around the initial.
% Please report to http://home.gna.org/gregorio/gregoriotex/details for more details and options
\grechangedim{afterinitialshift}{2.2mm}{scalable}
\grechangedim{beforeinitialshift}{2.2mm}{scalable}
\grechangedim{interwordspacetext}{0.22 cm plus 0.15 cm minus 0.05 cm}{scalable}%
\grechangedim{annotationraise}{-0.2cm}{scalable}

% Here we set the initial font. Change 38 if you want a bigger initial.
% Emit the initials in red.
\grechangestyle{initial}{\color{red}\fontsize{38}{38}\selectfont}

\pagestyle{empty}

%%%% Titulni stranka
\begin{titulusOfficii}
\ifx\titulus\undefined
\nomenFesti{Feria V \hebdomada{}}
\else
\titulus
\fi
\end{titulusOfficii}

\vfill

\begin{center}
%Ad usum et secundum consuetudines chori \guillemotright{}Conventus Choralis\guillemotleft.

%Editio Sancti Wolfgangi \annusEditionis
\end{center}

\scriptura{}

\pars{}

\pagebreak

\renewcommand{\headrulewidth}{0pt} % no horiz. rule at the header
\fancyhf{}
\pagestyle{fancy}

\cantusSineNeumas

\ifx\oratio\undefined
\ifx\lauda\undefined
\else
\newcommand{\oratio}{\pars{Oratio.}

\noindent Omnípotens sempitérne Deus, véspere, mane et merídie maiestátem tuam supplíciter deprecámur, ut, expúlsis de córdibus nostris peccatórum ténebris, ad veram lucem, quæ Christus est, nos fácias perveníre.

\noindent Qui tecum vivit et regnat in unitáte Spíritus Sancti, Deus, per ómnia sǽcula sæculórum.

\noindent \Rbardot{} Amen.}
\fi
\ifx\laudb\undefined
\else
\newcommand{\oratio}{\pars{Oratio.}

\noindent Te lucem veram et lucis auctórem, Dómine, deprecámur, ut, quæ sancta sunt fidéliter meditántes, in tua iúgiter claritáte vivámus.

\noindent Per Dóminum nostrum Iesum Christum, Fílium tuum, qui tecum vivit et regnat in unitáte Spíritus Sancti, Deus, per ómnia sǽcula sæculórum.

\noindent \Rbardot{} Amen.}
\fi
\ifx\laudc\undefined
\else
\newcommand{\oratio}{\pars{Oratio.}

\noindent Omnípotens ætérne Deus, pópulos, qui in umbra mortis sedent, lúmine tuæ claritátis illústra, qua visitávit nos Oriens ex alto, Iesus Christus Dóminus noster.

\noindent Qui tecum vivit et regnat in unitáte Spíritus Sancti, Deus, per ómnia sǽcula sæculórum.

\noindent \Rbardot{} Amen.}
\fi
\ifx\laudd\undefined
\else
\newcommand{\oratio}{\pars{Oratio.}

\noindent Sciéntiam salútis, Dómine, nobis concéde sincéram, ut sine timóre, de manu inimicórum nostrórum liberáti, ómnibus diébus nostris tibi fidéliter serviámus.

\noindent Per Dóminum nostrum Iesum Christum, Fílium tuum, qui tecum vivit et regnat in unitáte Spíritus Sancti, Deus, per ómnia sǽcula sæculórum.

\noindent \Rbardot{} Amen.}
\fi
\fi

\hora{Ad Matutinum.} %%%%%%%%%%%%%%%%%%%%%%%%%%%%%%%%%%%%%%%%%%%%%%%%%%%%%
%\sideThumbs{Matutinum}

\vspace{2mm}

\cuminitiali{}{temporalia/dominelabiamea.gtex}

\vfill
%\pagebreak

\vspace{2mm}

\ifx\invitatorium\undefined
\pars{Invitatorium.} \scriptura{Ps. 94, 6; Psalmus 94; \textbf{H136}}

\vspace{-6mm}

\antiphona{E}{temporalia/inv-adoremusdominum.gtex}
\else
\invitatorium
\fi

\vfill
\pagebreak

\ifx\hymnusmatutinum\undefined
\ifx\hiemalis\undefined
\ifx\matua\undefined
\else
\pars{Hymnus.}

\antiphona{II}{temporalia/hym-ChristePrecamur-MMMA.gtex}
\fi
\ifx\matub\undefined
\else
\pars{Hymnus.}

\antiphona{IV}{temporalia/hym-AmorisSensusErige-kn.gtex}
\fi
\ifx\matuc\undefined
\else
\pars{Hymnus.}

\antiphona{IV}{temporalia/hym-ChristePrecamur-kempten.gtex}
\fi
\ifx\matud\undefined
\else
\pars{Hymnus.}

\antiphona{II}{temporalia/hym-AmorisSensusErige.gtex}
\fi
\else
\ifx\matuac\undefined
\else
\pars{Hymnus.} \scriptura{Gregorius Magnus (\olddag{} 604)}

{
\grechangedim{interwordspacetext}{0.10 cm plus 0.15 cm minus 0.05 cm}{scalable}%
\antiphona{IV}{temporalia/hym-NoxAtra.gtex}
\grechangedim{interwordspacetext}{0.22 cm plus 0.15 cm minus 0.05 cm}{scalable}%
}
\fi
\ifx\matubd\undefined
\else
\pars{Hymnus.} \scriptura{Prudentius (\olddag{} 405)}

\antiphona{II}{temporalia/hym-AlesDiei.gtex}
\fi
\fi
\else
\hymnusmatutinum
\fi

\vspace{-3mm}

\vfill
\pagebreak

\ifx\matutinum\undefined
\ifx\matua\undefined
\else
% MAT A
\pars{Psalmus 1.} \scriptura{Ps. 17, 3; \textbf{H99}}

\vspace{-4mm}

\antiphona{VIII G}{temporalia/ant-dominusfirmamentum.gtex}

%\vspace{-2mm}

\scriptura{Ps. 17, 31-35}

%\vspace{-2mm}

\initiumpsalmi{temporalia/ps17xxxi_xxxv-initium-viii-G-auto.gtex}

\input{temporalia/ps17xxxi_xxxv-viii-G.tex} \Abardot{}

\vfill
\pagebreak

\pars{Psalmus 2.} \scriptura{Ps. 62, 9; \textbf{H393}}

\vspace{-4mm}

\antiphona{VII c trans.}{temporalia/ant-mesuscepit.gtex}

%\vspace{-2mm}

\scriptura{Ps. 17, 36-46}

%\vspace{-2mm}

\initiumpsalmi{temporalia/ps17xxxvi_xlvi-initium-vii-c-trans.gtex}

\input{temporalia/ps17xxxvi_xlvi-vii-c.tex} \Abardot{}

\vfill
\pagebreak

\pars{Psalmus 3.} \scriptura{Ps. 17, 47; \textbf{H100}}

\vspace{-4mm}

\antiphona{VII c\textsuperscript{2}}{temporalia/ant-vivitdominus.gtex}

%\vspace{-2mm}

\scriptura{Ps. 17, 47-51}

%\vspace{-2mm}

\initiumpsalmi{temporalia/ps17xlvii_li-initium-vii-c2-auto.gtex}

\input{temporalia/ps17xlvii_li-vii-c2.tex} \Abardot{}

\vfill
\pagebreak
\fi
\ifx\matub\undefined
\else
% MAT B
\pars{Psalmus 1.} \scriptura{\textbf{H416}}

\vspace{-4mm}

\antiphona{VIII G}{temporalia/ant-extendedomine.gtex}

\vspace{-1mm}

\scriptura{Ps. 43, 2-9}

\vspace{-2mm}

\initiumpsalmi{temporalia/ps43i-initium-viii-G-auto.gtex}

\vspace{-1.5mm}

\input{temporalia/ps43i-viii-G.tex} \Abardot{}

\vfill
\pagebreak

\pars{Psalmus 2.} \scriptura{Ie. 17, 18; \textbf{H174}}

\vspace{-4mm}

\antiphona{II* a}{temporalia/ant-confundanturqui.gtex}

%\vspace{-2mm}

\scriptura{Ps. 43, 10-17}

\initiumpsalmi{temporalia/ps43ii-initium-ii_-a-auto.gtex}

\input{temporalia/ps43ii-ii_-a.tex} \Abardot{}

\vfill
\pagebreak

\pars{Psalmus 3.} \scriptura{2 Esr. 6, 14; Tb. 3, 13}

\vspace{-4mm}

\antiphona{II D}{temporalia/ant-mementodomine.gtex}

%\vspace{-2mm}

\scriptura{Ps. 43, 18-26}

%\vspace{-2mm}

\initiumpsalmi{temporalia/ps43iii-initium-ii-D-auto.gtex}

\input{temporalia/ps43iii-ii-D.tex} \Abardot{}

\vfill
\pagebreak

\fi
\ifx\matuc\undefined
\else
% MAT C
\pars{Psalmus 1.} \scriptura{Lam. 1, 21; \textbf{H177}}

\vspace{-4mm}

\antiphona{VII a}{temporalia/ant-omnesinimici.gtex}

%\vspace{-2mm}

\scriptura{Ps. 88, 39-46}

%\vspace{-2mm}

\initiumpsalmi{temporalia/ps88xxxix_xlvi-initium-vii-a-auto.gtex}

\input{temporalia/ps88xxxix_xlvi-vii-a.tex} \Abardot{}

\vfill
\pagebreak

\pars{Psalmus 2.} \scriptura{Ps. 88, 53; \textbf{H98}}

\vspace{-4mm}

\antiphona{VI F}{temporalia/ant-benedictusdominusinaeternum.gtex}

%\vspace{-2mm}

\scriptura{Ps. 88, 47-53}

%\vspace{-2mm}

\initiumpsalmi{temporalia/ps88xlvii_liii-initium-vi-F-auto.gtex}

\input{temporalia/ps88xlvii_liii-vi-F.tex} \Abardot{}

\vfill
\pagebreak

\pars{Psalmus 3.} \scriptura{Ps. 89, 13}

\vspace{-4mm}

\antiphona{I g}{temporalia/ant-converteredomine.gtex}

%\vspace{-2mm}

\scriptura{Ps. 89}

%\vspace{-2mm}

\initiumpsalmi{temporalia/ps89-initium-i-g-auto.gtex}

\input{temporalia/ps89-i-g.tex}

\vfill

\antiphona{}{temporalia/ant-converteredomine.gtex}

\vfill
\pagebreak
\fi
\ifx\matud\undefined
\else
% MAT D
\pars{Psalmus 1.}

\vspace{-4mm}

\antiphona{VIII G}{temporalia/ant-quantaaudivimus.gtex}

%\vspace{-2mm}

\scriptura{Ps. 43, 2-9}

%\vspace{-2mm}

\initiumpsalmi{temporalia/ps43i-initium-viii-G-auto.gtex}

\input{temporalia/ps43i-viii-G.tex} \Abardot{}

\vfill
\pagebreak

\pars{Psalmus 2.} \scriptura{Ier. 15, 15; \textbf{H176}}

\vspace{-4mm}

\antiphona{VIII c}{temporalia/ant-recordaremei.gtex}

%\vspace{-2mm}

\scriptura{Ps. 43, 10-17}

%\vspace{-2mm}

\initiumpsalmi{temporalia/ps43ii-initium-viii-C-auto.gtex}

\input{temporalia/ps43ii-viii-C.tex} \Abardot{}

\vfill
\pagebreak

\pars{Psalmus 3.} \scriptura{Ps. 9, 20}

\vspace{-4mm}

\antiphona{I g\textsuperscript{3}}{temporalia/ant-exsurgedominenon.gtex}

%\vspace{-2mm}

\scriptura{Ps. 43, 18-27}

%\vspace{-2mm}

\initiumpsalmi{temporalia/ps43iii-initium-i-g3-auto.gtex}

\input{temporalia/ps43iii-i-g3.tex} \Abardot{}

\vfill
\pagebreak
\fi
\else
\matutinum
\fi

\pars{Versus.}

\ifx\matversus\undefined
\ifx\matua\undefined
\else
\noindent \Vbardot{} Révela, Dómine, óculos meos.

\noindent \Rbardot{} Et considerábo mirabília de lege tua.
\fi
\ifx\matub\undefined
\else
\noindent \Vbardot{} Dómine, ad quem íbimus?

\noindent \Rbardot{} Verba vitæ ætérnæ habes.
\fi
\ifx\matuc\undefined
\else
\noindent \Vbardot{} Audies de ore meo verbum.

\noindent \Rbardot{} Et annuntiábis eis ex me.
\fi
\ifx\matud\undefined
\else
\noindent \Vbardot{} Fáciem tuam illúmina super servum tuum, Dómine.

\noindent \Rbardot{} Et doce me iustificatiónes tuas.
\fi
\else
\matversus
\fi

\vspace{5mm}

\sineinitiali{temporalia/oratiodominica-mat.gtex}

\vspace{5mm}

\pars{Absolutio.}

\ifx\absolutio\undefined
\cuminitiali{}{temporalia/absolutio-exaudi.gtex}
\else
\absolutio
\fi

\vfill
\pagebreak

\ifx\benedictioi\undefined
\cuminitiali{}{temporalia/benedictio-solemn-benedictione.gtex}
\else
\benedictioi
\fi

\vspace{7mm}

\lectioi

\noindent \Vbardot{} Tu autem, Dómine, miserére nobis.
\noindent \Rbardot{} Deo grátias.

\vfill
\pagebreak

\responsoriumi

\vfill
\pagebreak

\ifx\benedictioii\undefined
\cuminitiali{}{temporalia/benedictio-solemn-unigenitus.gtex}
\else
\benedictioii
\fi

\vspace{7mm}

\lectioii

\noindent \Vbardot{} Tu autem, Dómine, miserére nobis.
\noindent \Rbardot{} Deo grátias.

\vfill
\pagebreak

\responsoriumii

\vfill
\pagebreak

\ifx\benedictioiii\undefined
\cuminitiali{}{temporalia/benedictio-solemn-spiritus.gtex}
\else
\benedictioiii
\fi

\vspace{7mm}

\lectioiii

\noindent \Vbardot{} Tu autem, Dómine, miserére nobis.
\noindent \Rbardot{} Deo grátias.

\vfill
\pagebreak

\responsoriumiii

\vfill
\pagebreak

\rubrica{Reliqua omittuntur, nisi Laudes separandæ sint.}

\sineinitiali{temporalia/domineexaudi.gtex}

\vfill

\oratio

\vfill

\noindent \Vbardot{} Dómine, exáudi oratiónem meam.
\Rbardot{} Et clamor meus ad te véniat.

\vfill

\noindent \Vbardot{} Benedicámus Dómino.
\noindent \Rbardot{} Deo grátias.

\vfill

\noindent \Vbardot{} Fidélium ánimæ per misericórdiam Dei requiéscant in pace.
\Rbardot{} Amen.

\vfill
\pagebreak

\hora{Ad Laudes.} %%%%%%%%%%%%%%%%%%%%%%%%%%%%%%%%%%%%%%%%%%%%%%%%%%%%%
%\sideThumbs{Laudes}

\cantusSineNeumas

\vspace{0.5cm}
\ifx\deusinadiutorium\undefined
\grechangedim{interwordspacetext}{0.18 cm plus 0.15 cm minus 0.05 cm}{scalable}%
\cuminitiali{}{temporalia/deusinadiutorium-communis.gtex}
\grechangedim{interwordspacetext}{0.22 cm plus 0.15 cm minus 0.05 cm}{scalable}%
\else
\deusinadiutorium
\fi

\vfill
\pagebreak

\ifx\hymnuslaudes\undefined
\ifx\hiemalislaudes\undefined
\ifx\lauda\undefined
\else
\pars{Hymnus}

\cuminitiali{I}{temporalia/hym-SolEcce.gtex}
\fi
\ifx\laudb\undefined
\else
\pars{Hymnus}

\cuminitiali{I}{temporalia/hym-IamLucis-hk.gtex}
\fi
\ifx\laudc\undefined
\else
\pars{Hymnus}

\cuminitiali{VIII}{temporalia/hym-SolEcce-einsiedeln.gtex}
\fi
\ifx\laudd\undefined
\else
\pars{Hymnus}

\cuminitiali{IV}{temporalia/hym-IamLucis.gtex}
\fi
\else
\ifx\laudac\undefined
\else
\pars{Hymnus}

\grechangedim{interwordspacetext}{0.16 cm plus 0.15 cm minus 0.05 cm}{scalable}%
\cuminitiali{I}{temporalia/hym-SolEcce.gtex}
\grechangedim{interwordspacetext}{0.22 cm plus 0.15 cm minus 0.05 cm}{scalable}%
\vspace{-3mm}
\fi
\ifx\laudbd\undefined
\else
\pars{Hymnus}

\grechangedim{interwordspacetext}{0.16 cm plus 0.15 cm minus 0.05 cm}{scalable}%
\cuminitiali{IV}{temporalia/hym-IamLucis.gtex}
\grechangedim{interwordspacetext}{0.22 cm plus 0.15 cm minus 0.05 cm}{scalable}%
\vspace{-3mm}
\fi
\fi
\else
\hymnuslaudes
\fi

\vfill
\pagebreak

\ifx\laudes\undefined
\ifx\lauda\undefined
\else
\pars{Psalmus 1.}

\vspace{-4mm}

\antiphona{VIII G}{temporalia/ant-exsurgamdiluculo.gtex}

%\vspace{-2mm}

\scriptura{Psalmus 56}

%\vspace{-2mm}

\initiumpsalmi{temporalia/ps56-initium-viii-g-auto.gtex}

%\vspace{-1.5mm}

\input{temporalia/ps56-viii-g.tex} \Abardot{}

\vfill
\pagebreak

\pars{Psalmus 2.} \scriptura{Ier. 31, 14}

\vspace{-4mm}

\antiphona{IV* e}{temporalia/ant-populusmeusait.gtex}

%\vspace{-2mm}

\scriptura{Canticum Ieremiæ, 1 Ier. 31, 10-14}

%\vspace{-3mm}

\initiumpsalmi{temporalia/jeremiae3-initium-iv_-e-auto.gtex}

\input{temporalia/jeremiae3-iv_-e.tex} \Abardot{}

\vfill
\pagebreak

\pars{Psalmus 3.} \scriptura{Ps. 95, 4; \textbf{H94}}

\vspace{-4mm}

\antiphona{IV a}{temporalia/ant-magnusdominus.gtex}

\scriptura{Psalmus 47}

\initiumpsalmi{temporalia/ps47-initium-iv-a.gtex}

\input{temporalia/ps47-iv-a.tex} \Abardot{}

\vfill
\pagebreak
\fi
\ifx\laudb\undefined
\else
\pars{Psalmus 1.} \scriptura{Ps. 79, 3; \textbf{H19}}

\vspace{-4mm}

\antiphona{II* b}{temporalia/ant-tuamdomineexcita.gtex}

\vspace{-2mm}

\scriptura{Psalmus 79.}

\vspace{-1mm}

\initiumpsalmi{temporalia/ps79-initium-ii_-B-auto.gtex}

\input{temporalia/ps79-ii_-B.tex}

\vfill

\antiphona{}{temporalia/ant-tuamdomineexcita.gtex}

\vfill
\pagebreak

\pars{Psalmus 2.} \scriptura{Is. 12, 1; \textbf{H93}}

\vspace{-4mm}

\antiphona{VIII G}{temporalia/ant-conversusestfuror.gtex}

\scriptura{Canticum Isaiæ Prophetæ, Is. 12, 1-7}

\initiumpsalmi{temporalia/isaiae-initium-viii-G-auto.gtex}

\input{temporalia/isaiae-viii-G.tex} \Abardot{}

\vfill
\pagebreak

\pars{Psalmus 3.} \scriptura{Ps. 80, 2}

\vspace{-4.5mm}

\antiphona{I g\textsuperscript{5}}{temporalia/ant-exsultatedeo.gtex}

\vspace{-2.5mm}

\scriptura{Psalmus 80.}

\vspace{-2mm}

\initiumpsalmi{temporalia/ps80-initium-i-g5-auto.gtex}

\vspace{-1.5mm}

\input{temporalia/ps80-i-g5.tex} \Abardot{}

\vfill
\pagebreak
\fi
\ifx\laudc\undefined
\else
\pars{Psalmus 1.} \scriptura{Ps. 86, 1; \textbf{H98}}

\vspace{-4mm}

\antiphona{I g}{temporalia/ant-fundamentaeius.gtex}

%\vspace{-2mm}

\scriptura{Psalmus 86}

%\vspace{-2mm}

\initiumpsalmi{temporalia/ps86-initium-i-g-auto.gtex}

%\vspace{-1.5mm}

\input{temporalia/ps86-i-g.tex} \Abardot{}

\vfill
\pagebreak

\pars{Psalmus 2.}

\vspace{-4mm}

\antiphona{II D}{temporalia/ant-eccedominusnosterbrachio.gtex}

%\vspace{-2mm}

\scriptura{Canticum Isaiæ, Is. 40, 10-17}

%\vspace{-3mm}

\initiumpsalmi{temporalia/isaiae9-initium-ii-D-auto.gtex}

\input{temporalia/isaiae9-ii-D.tex} \Abardot{}

\vfill
\pagebreak

\pars{Psalmus 3.} \scriptura{Ps. 144, 17}

\vspace{-4mm}

\antiphona{E}{temporalia/ant-iustusetsanctus.gtex}

\scriptura{Psalmus 98}

\initiumpsalmi{temporalia/ps98-initium-e.gtex}

\input{temporalia/ps98-e.tex} \Abardot{}

\vfill
\pagebreak
\fi
\ifx\laudd\undefined
\else
\pars{Psalmus 1.} \scriptura{Ps. 142, 1; \textbf{H100}}

\vspace{-4mm}

\antiphona{VIII G}{temporalia/ant-inveritatetua.gtex}

%\vspace{-2mm}

\scriptura{Psalmus 142}

%\vspace{-2mm}

\initiumpsalmi{temporalia/ps142-initium-viii-G-auto.gtex}

%\vspace{-1.5mm}

\input{temporalia/ps142-viii-G.tex}

\vfill

\antiphona{}{temporalia/ant-inveritatetua.gtex}

\vfill
\pagebreak

\pars{Psalmus 2.}

\vspace{-4mm}

\antiphona{IV* e}{temporalia/ant-declinabitdominus.gtex}

%\vspace{-2mm}

\scriptura{Canticum Isaiæ, Is. 66, 10-14}

%\vspace{-3mm}

\initiumpsalmi{temporalia/isaiae5-initium-iv_-e-auto.gtex}

\input{temporalia/isaiae5-iv_-e.tex} \Abardot{}

\vfill
\pagebreak

\pars{Psalmus 3.} \scriptura{Ps. 146, 1; \textbf{H101}}

\vspace{-4mm}

\antiphona{VIII G}{temporalia/ant-deonostroiucunda.gtex}

\scriptura{Psalmus 146}

\initiumpsalmi{temporalia/ps146-initium-viii-g-auto.gtex}

\input{temporalia/ps146-viii-g.tex} \Abardot{}

\vfill
\pagebreak
\fi
\else
\laudes
\fi

\ifx\lectiobrevis\undefined
\ifx\lauda\undefined
\else
\pars{Lectio Brevis.} \scriptura{Is. 66, 1-2}

\noindent Hæc dicit Dóminus: Cælum thronus meus, terra autem scabéllum pedum meórum. Quæ ista domus, quam ædificábitis mihi, et quis iste locus quiétis meæ? Omnia hæc manus mea fecit et mea sunt univérsa ista, dicit Dóminus. Ad hunc autem respíciam, ad paupérculum et contrítum spíritu et treméntem sermónes meos.
\fi
\ifx\laudb\undefined
\else
\pars{Lectio Brevis.} \scriptura{Rom. 14, 17-19}

\noindent Non est regnum Dei esca et potus, sed iustítia et pax et gáudium in Spíritu Sancto; qui enim in hoc servit Christo, placet Deo et probátus est homínibus. Itaque, quæ pacis sunt, sectémur et quæ ædificatiónis sunt in ínvicem.
\fi
\ifx\laudc\undefined
\else
\pars{Lectio Brevis.} \scriptura{1 Petr. 4, 10-11}

\noindent Unusquísque, sicut accépit donatiónem, in altérutrum illam administrántes sicut boni dispensatóres multifórmis grátiæ Dei. Si quis lóquitur, quasi sermónes Dei; si quis minístrat, tamquam ex virtúte, quam largítur Deus, ut in ómnibus glorificétur Deus per Iesum Christum.
\fi
\ifx\laudd\undefined
\else
\pars{Lectio Brevis.} \scriptura{Rom. 8, 18-21}

\noindent Non sunt condígnæ passiónes huius témporis ad futúram glóriam, quæ revelánda est in nobis. Nam exspectátio creatúræ revelatiónem filiórum Dei exspéctat; vanitáti enim creatúra subiécta est, non volens sed propter eum, qui subiécit, in spem, quia et ipsa creatúra liberábitur a servitúte corruptiónis in libertátem glóriæ filiórum Dei.
\fi
\else
\lectiobrevis
\fi

\vfill

\ifx\responsoriumbreve\undefined
\ifx\laudac\undefined
\else
\pars{Responsorium breve.} \scriptura{Ps. 118, 145}

\cuminitiali{VI}{temporalia/resp-clamaviintotocorde.gtex}
\fi
\ifx\laudbd\undefined
\else
\pars{Responsorium breve.} \scriptura{Ps. 62, 7-8}

\cuminitiali{VI}{temporalia/resp-inmatutinis.gtex}
\fi
\else
\responsoriumbreve
\fi

\vfill
\pagebreak

\ifx\benedictus\undefined
\ifx\laudac\undefined
\else
\pars{Canticum Zachariæ.} \scriptura{Lc. 1, 74.75; \textbf{H423}}

%\vspace{-4mm}

{
\grechangedim{interwordspacetext}{0.18 cm plus 0.15 cm minus 0.05 cm}{scalable}%
\antiphona{VII a}{temporalia/ant-insanctitate.gtex}
\grechangedim{interwordspacetext}{0.22 cm plus 0.15 cm minus 0.05 cm}{scalable}%
}

%\vspace{-3mm}

\scriptura{Lc. 1, 68-79}

%\vspace{-2mm}

\cantusSineNeumas
\initiumpsalmi{temporalia/benedictus-initium-vii-a-auto.gtex}

%\vspace{-1.5mm}

\input{temporalia/benedictus-vii-a.tex} \Abardot{}
\fi
\ifx\laudbd\undefined
\else
\pars{Canticum Zachariæ.} \scriptura{Lc. 1, 77; \textbf{H423}}

%\vspace{-4mm}

{
\grechangedim{interwordspacetext}{0.18 cm plus 0.15 cm minus 0.05 cm}{scalable}%
\antiphona{VII c\textsuperscript{2}}{temporalia/ant-dascientiamplebituae.gtex}
\grechangedim{interwordspacetext}{0.22 cm plus 0.15 cm minus 0.05 cm}{scalable}%
}

%\vspace{-3mm}

\scriptura{Lc. 1, 68-79}

%\vspace{-2mm}

\cantusSineNeumas
\initiumpsalmi{temporalia/benedictus-initium-vii-c2-auto.gtex}

%\vspace{-1.5mm}

\input{temporalia/benedictus-vii-c2.tex} \Abardot{}
\fi
\else
\benedictus
\fi

\vspace{-1cm}

\vfill
\pagebreak

%\sideThumbs{{\scriptsize{}Fine horarum}}

\pars{Preces.}

\sineinitiali{}{temporalia/tonusprecum.gtex}

\ifx\preces\undefined
\ifx\lauda\undefined
\else
\noindent Grátias agámus Christo, qui lumen huius diéi nobis concédit,~\gredagger{} et ad eum clamémus:

\Rbardot{} Bénedic et sanctífica nos, Dómine.

\noindent Qui te pro peccátis nostris hóstiam obtulísti,~\gredagger{} incépta et propósita suscípias hodiérna.

\Rbardot{} Bénedic et sanctífica nos, Dómine.

\noindent Qui óculos nostros lucis dono lætíficas novæ,~\gredagger{} lúcifer oriáris in córdibus nostris.

\Rbardot{} Bénedic et sanctífica nos, Dómine.

\noindent Tríbue hódie nos esse ómnibus longánimes,~\gredagger{} ut imitatóres tui fíeri possímus.

\Rbardot{} Bénedic et sanctífica nos, Dómine.

\noindent Audítam, Dómine, fac nobis mane misericórdiam tuam.~\gredagger{} Sit hódie gáudium tuum fortitúdo nostra.

\Rbardot{} Bénedic et sanctífica nos, Dómine.
\fi
\ifx\laudb\undefined
\else
\noindent Benedíctus Deus, Pater noster, qui fílios suos prótegit neque preces spernit eórum.~\gredagger{} Omnes humíliter eum implorémus orántes:

\Rbardot{} Illúmina óculos nostros, Dómine.

\noindent Grátias tibi, Dómine, quia per Fílium tuum nos illuminásti,~\gredagger{} eius luce per longitúdinem diéi nos satiári concéde.

\Rbardot{} Illúmina óculos nostros, Dómine.

\noindent Sapiéntia tua, Dómine, dedúcat nos hódie,~\gredagger{} ut in novitáte vitæ ambulémus.

\Rbardot{} Illúmina óculos nostros, Dómine.

\noindent Præsta nobis advérsa pro te fórtiter sustinére,~\gredagger{} ut corde magno tibi iúgiter serviámus.

\Rbardot{} Illúmina óculos nostros, Dómine.

\noindent Dírige in nobis hódie cogitatiónes, sensus et ópera,~\gredagger{} ut tibi providénti dóciles obsequámur.

\Rbardot{} Illúmina óculos nostros, Dómine.
\fi
\ifx\laudc\undefined
\else
\noindent Grátias agámus Deo Patri, qui amóre suo dedúcit et nutrit pópulum suum,~\gredagger{} lætíque clamémus:

\Rbardot{} Glória tibi, Dómine, in sǽcula.

\noindent Pater clementíssime, de tuo nos te laudámus amóre,~\gredagger{} quia nos mirabíliter condidísti et mirabílius reformásti.

\Rbardot{} Glória tibi, Dómine, in sǽcula.

\noindent In huius diéi princípio serviéndi tibi stúdium córdibus nostris infúnde,~\gredagger{} ut cogitatiónes et actiónes nostræ te semper gloríficent.

\Rbardot{} Glória tibi, Dómine, in sǽcula.

\noindent Ab omni desidério malo corda nostra purífica,~\gredagger{} ut tuæ voluntáti simus semper inténti.

\Rbardot{} Glória tibi, Dómine, in sǽcula.

\noindent Fratrum omniúmque necessitátibus corda résera nostra,~\gredagger{} ne fratérna nostra dilectióne privéntur.

\Rbardot{} Glória tibi, Dómine, in sǽcula.
\fi
\ifx\laudd\undefined
\else
\noindent Deum, a quo óbvenit salus pópulo suo,~\gredagger{} celebrémus ita dicéntes:

\Rbardot{} Tu es vita nostra, Dómine.

\noindent Benedíctus es, Pater Dómini nostri Iesu Christi, qui secúndum misericórdiam tuam regenerásti nos in spem vivam,~\gredagger{} per resurrectiónem Iesu Christi ex mórtuis.

\Rbardot{} Tu es vita nostra, Dómine.

\noindent Qui hóminem, ad imáginem tuam creátum, in Christo renovásti,~\gredagger{} fac nos confórmes imágini Fílii tui.

\Rbardot{} Tu es vita nostra, Dómine.

\noindent In córdibus nostris invídia et ódio vulnerátis,~\gredagger{} caritátem per Spíritum Sanctum datam effúnde.

\Rbardot{} Tu es vita nostra, Dómine.

\noindent Da hódie operáriis labórem, esuriéntibus panem, mæréntibus gáudium,~\gredagger{} ómnibus homínibus grátiam atque salútem.

\Rbardot{} Tu es vita nostra, Dómine.
\fi
\else
\preces
\fi

\vfill

\pars{Oratio Dominica.}

\cuminitiali{}{temporalia/oratiodominicaalt.gtex}

\vfill
\pagebreak

\rubrica{vel:}

\pars{Supplicatio Litaniæ.}

\cuminitiali{}{temporalia/supplicatiolitaniae.gtex}

\vfill

\pars{Oratio Dominica.}

\cuminitiali{}{temporalia/oratiodominica.gtex}

\vfill
\pagebreak

% Oratio. %%%
\oratio

\vspace{-1mm}

\vfill

\rubrica{Hebdomadarius dicit Dominus vobiscum, vel, absente sacerdote vel diacono, sic concluditur:}

\vspace{2mm}

\antiphona{C}{temporalia/dominusnosbenedicat.gtex}

\rubrica{Postea cantatur a cantore:}

\vspace{2mm}

\ifx\benedicamuslaudes\undefined
\cuminitiali{IV}{temporalia/benedicamus-feria-laudes.gtex}
\else
\benedicamuslaudes
\fi

\vspace{1mm}

\vfill
\pagebreak

\end{document}

