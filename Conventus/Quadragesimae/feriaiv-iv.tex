\newcommand{\oratio}{\pars{Oratio.}

\noindent Deus, qui et iustis prǽmia meritórum et peccatóribus véniam per pæniténtiam præbes, tuis supplícibus miserére, ut reátus nostri conféssio indulgéntiam váleat percípere delictórum.

\vfill

\pars{Pro commemoratione Sancti Patricii episcopi.} \scriptura{Mt. 28, 19}

\vspace{-4mm}

\antiphona{VIII G}{temporalia/ant-euntesdoceteomnes.gtex}

\vfill

\noindent Deus, qui ad prædicándam Hibérniæ pópulis glóriam tuam beátum Patrícium, epíscopum, providísti, eius méritis et intercessióne concéde, ut, qui christiáno nómine gloriántur, tua mirabília homínibus iúgiter annúntient.

\noindent Per Dóminum nostrum Iesum Christum, Fílium tuum, qui tecum vivit et regnat in unitáte Spíritus Sancti, Deus, per ómnia sǽcula sæculórum.

\noindent \Rbardot{} Amen.}
\newcommand{\invitatorium}{\pars{Invitatorium.} \scriptura{Ps. 94, 8; Psalmus 94; \textbf{H143}}

\vspace{-4mm}

\antiphona{E}{temporalia/inv-hodiesivocem.gtex}}
\newcommand{\hymnusmatutinum}{\pars{Hymnus}

\cuminitiali{I}{temporalia/hym-NuncTempus.gtex}}
\newcommand{\matversus}{\noindent \Vbardot{} Convertímini et ágite pæniténtiam.

\noindent \Rbardot{} Fácite vobis cor novum et spíritum novum.}
\newcommand{\lectioi}{\vspace{-4mm}

\pars{Lectio I.} \scriptura{Ex. 13, 17-22; 14, 1-9}

\noindent De libro Exodi.

\noindent Cum emisísset phárao pópulum, non eos duxit Deus per viam terræ Philísthim, quæ vicína est, réputans ne forte pænitéret pópulum, si vidísset advérsum se bella consúrgere, et reverterétur in Ægýptum, sed circumdúxit per viam desérti, quæ est iuxta mare Rubrum. Et armáti ascendérunt fílii Israel de terra Ægýpti. Tulit quoque Móyses ossa Ioseph secum, eo quod adiurásset fílios Israel dicens: «Visitábit vos Deus; efférte ossa mea hinc vobíscum».

\noindent Profectíque de Succoth castrametáti sunt in Etham, in extrémis fínibus solitúdinis. Dóminus autem præcedébat eos ad ostendéndam viam per diem in colúmna nubis et per noctem in colúmna ignis, ut dux esset itíneris utróque témpore. Nunquam défuit colúmna nubis per diem nec colúmna ignis per noctem coram pópulo.

\noindent Locútus est autem Dóminus ad Móysen dicens: «Lóquere fíliis Israel: Revérsi castrameténtur e regióne Phihahíroth, quæ est inter Mógdolum et mare contra Beélsephon; in conspéctu eius castra ponétis super mare. Dicturúsque est phárao super fíliis Israel: “Errant in terra, conclúsit eos desértum”. Et indurábo cor eius, ac persequétur eos, et glorificábor in pharaóne et in omni exércitu eius; sciéntque Ægýptii quia ego sum Dóminus». Fecerúntque ita.

\noindent Et nuntiátum est regi Ægyptiórum quod fugísset pópulus; immutatúmque est cor pharaónis et servórum eius super pópulo, et dixérunt: «Quid hoc fécimus, ut dimitterémus Israel, ne servírent nobis?». Iunxit ergo currum et omnem pópulum suum assúmpsit secum; tulítque sescéntos currus eléctos et quidquid in Ægýpto cúrruum fuit et bellatóres in síngulis cúrribus. Induravítque Dóminus cor pharaónis regis Ægýpti, et persecútus est fílios Israel; at illi egréssi erant in manu excélsa. Cumque persequeréntur Ægýptii vestígia præcedéntium, repperérunt eos in castris super mare; omnes equi et currus pharaónis, équites et exércitus eius erant in Phihahíroth contra Beélsephon.}
\newcommand{\responsoriumi}{\pars{Responsorium 1.} \scriptura{\Rbardot{} Ps. 76, 20 \Vbardot{} ibid., 19; \textbf{H159}}

\vspace{-5mm}

\responsorium{II}{temporalia/resp-inmariviatua-CROCHU.gtex}{}}
\newcommand{\lectioii}{\pars{Lectio II.} \scriptura{Ex. 14, 10-18}

\noindent In diébus illis: Cum appropinquásset phárao, levántes fílii Israel óculos vidérunt Ægýptios post se et timuérunt valde clamaverúntque ad Dóminum et dixérunt ad Móysen: «Fórsitan non erant sepúlcra in Ægýpto? Ideo tulísti nos, ut morerémur in solitúdine. Quid hoc fecísti, ut edúceres nos ex Ægýpto? Nonne iste est sermo, quem loquebámur ad te in Ægýpto, dicéntes: Recéde a nobis, ut serviámus Ægýptiis? Multo enim mélius erat servíre eis quam mori in solitúdine». Et ait Móyses ad pópulum: «Nolíte timére; state et vidéte salútem Dómini, quam factúrus est vobis hódie; Ægýptios enim, quos nunc vidétis, nequáquam ultra vidébitis usque in sempitérnum. Dóminus pugnábit pro vobis, et vos silébitis».

\noindent Dixítque Dóminus ad Móysen: «Quid clamas ad me? Lóquere fíliis Israel, ut proficiscántur. Tu autem éleva virgam tuam et exténde manum tuam super mare et dívide illud, ut gradiántur fílii Israel in médio mari per siccum. Ego autem indurábo cor Ægyptiórum, ut persequántur eos; et glorificábor in pharaóne et in omni exércitu eius, in cúrribus et in equítibus illíus. Et scient Ægýptii quia ego sum Dóminus, cum glorificátus fúero in pharaóne, in cúrribus atque in equítibus eius».}
\newcommand{\responsoriumii}{\pars{Responsorium 2.} \scriptura{\Rbardot{} Cf. Sap. 10, 19; \textbf{H160}}

\vspace{-5mm}

\responsorium{VII}{temporalia/resp-quipersequebanturpopulum-CROCHU.gtex}{}}
\newcommand{\lectioiii}{\pars{Lectio III.} \scriptura{Ex. 14, 19-31; 15, 1}

\noindent Tollénsque se ángelus Dei, qui præcedébat castra Israel, ábiit post eos; et cum eo páriter colúmna nubis, prióra dimíttens, post tergum. Stetit inter castra Ægyptiórum et castra Israel; et erat nubes tenebrósa et illúminans noctem, ita ut ad se ínvicem toto noctis témpore accédere non valérent.

\noindent Cumque extendísset Móyses manum super mare, réppulit illud Dóminus, flante vento veheménti et urénte tota nocte, et vertit in siccum; divisáque est aqua. Et ingréssi sunt fílii Israel per médium maris sicci; erat enim aqua quasi murus a dextra eórum et læva. Persequentésque Ægýptii ingréssi sunt post eos, omnis equitátus pharaónis, currus eius et équites per médium maris. Iamque advénerat vigília matutína, et ecce respíciens Dóminus super castra Ægyptiórum per colúmnam ignis et nubis perturbávit exércitum eórum; et impedívit rotas cúrruum, ita ut diffícile moveréntur. Dixérunt ergo Ægýptii: «Fugiámus Israélem! Dóminus enim pugnat pro eis contra nos».

\noindent Et ait Dóminus ad Móysen: «Exténde manum tuam super mare, ut revertántur aquæ ad Ægýptios super currus et équites eórum». Cumque extendísset Móyses manum contra mare, revérsum est primo dilúculo ad priórem locum; fugientibúsque Ægýptiis occurrérunt aquæ, et invólvit eos Dóminus in médiis flúctibus. Reversǽque sunt aquæ et operuérunt currus et équites cuncti exércitus pharaónis, qui sequéntes ingréssi fúerant mare; ne unus quidem supérfuit ex eis. Fílii autem Israel perrexérunt per médium sicci maris, et aquæ eis erant quasi pro muro a dextris et a sinístris.

\noindent Liberavítque Dóminus in die illo Israel de manu Ægyptiórum. Et vidérunt Ægýptios mórtuos super litus maris et manum magnam, quam exercúerat Dóminus contra eos; timuítque pópulus Dóminum et credidérunt Dómino et Móysi servo eius.

\noindent Tunc cécinit Móyses et fílii Israel carmen hoc Dómino, et dixérunt:

\rubrica{Hic non dicitur Tu Autem.}}
\newcommand{\lectioiiisinetuautem}{Sine tu autem}
\newcommand{\responsoriumiii}{\pars{Responsorium 3.} \scriptura{\Rbardot{} Ex. 15, 1-2 \Vbardot{} ibid., 4; \textbf{H159}}

\vspace{-5mm}

\responsorium{VIII}{temporalia/resp-cantemusdominogloriose-CROCHU-cumdox.gtex}{}}
\newcommand{\lectiobrevis}{\pars{Lectio Brevis.} \scriptura{Dt. 7, 6.8-9}

\noindent Te elégit Dóminus Deus tuus, ut sis ei pópulus peculiáris de cunctis pópulis, qui sunt super terram, quia diléxit vos Dóminus et custodívit iuraméntum, quod iurávit pátribus vestris, edúxit vos in manu forti et redémit te de domo servitútis, de manu pharaónis regis Ægýpti. Et scies quia Dóminus Deus tuus ipse est Deus, Deus fidélis, custódiens pactum et misericórdiam diligéntibus se et his, qui custódiunt mandáta eius, in mille generatiónes.}
\newcommand{\responsoriumbreve}{\pars{Responsorium breve.} \scriptura{Ps. 90, 3}

\cuminitiali{IV}{temporalia/resp-ipseliberavitme.gtex}}
\newcommand{\hymnuslaudes}{\pars{Hymnus}

\cuminitiali{D}{temporalia/hym-IamChriste.gtex}}
\newcommand{\preces}{\noindent Grátias agámus Deo Patri, qui per infusiónem et operatiónem Spíritus Sancti corda nostra puríficat et in caritáte confírmat. \gredagger{} Ei súpplici prece dicámus:

\Rbardot{} Da nobis, Dómine, Spíritum Sanctum tuum.

\noindent Præsta nobis, ut bona de manu tua semper grati suscipiámus \gredagger{} et mala quoque cum patiéntia accipiámus.

\Rbardot{} Da nobis, Dómine, Spíritum Sanctum tuum.

\noindent Concéde nobis non in magnis tantum rebus caritátem sectári, \gredagger{} sed étiam in sólitis vitæ adiúnctis prótenus exercére.

\Rbardot{} Da nobis, Dómine, Spíritum Sanctum tuum.

\noindent Tríbue nobis a supérfluis abstinére, \gredagger{} ut frátribus indigéntibus opem ferre valeámus.

\Rbardot{} Da nobis, Dómine, Spíritum Sanctum tuum.

\noindent Da nobis mortificatiónem Fílii tui in córpore nostro circumférre, \gredagger{} qui nos vivificásti in córpore eius.

\Rbardot{} Da nobis, Dómine, Spíritum Sanctum tuum.}
\newcommand{\benedictus}{\pars{Canticum Zachariæ.} \scriptura{Io. 5, 24}

\vspace{-4mm}

{
\grechangedim{interwordspacetext}{0.18 cm plus 0.15 cm minus 0.05 cm}{scalable}%
\antiphona{I f}{temporalia/ant-quiverbummeumaudit.gtex}
\grechangedim{interwordspacetext}{0.22 cm plus 0.15 cm minus 0.05 cm}{scalable}%
}

%\vspace{-2mm}

\scriptura{Lc. 1, 68-79}

%\vspace{-2mm}

\initiumpsalmi{temporalia/benedictus-initium-i-f-auto.gtex}

%\vspace{-1mm}

\input{temporalia/benedictus-i-f.tex} \Abardot{}}
\newcommand{\magnificat}{\pars{Canticum B. Mariæ V.} \scriptura{Io. 9, 11.6.25; \textbf{H163}}

\vspace{-4mm}

{
\grechangedim{interwordspacetext}{0.18 cm plus 0.15 cm minus 0.05 cm}{scalable}%
\antiphona{I a}{temporalia/ant-illehomoquidicituriesus.gtex}
\grechangedim{interwordspacetext}{0.22 cm plus 0.15 cm minus 0.05 cm}{scalable}%
}

%\vspace{-2mm}

\scriptura{Lc. 1, 46-55}

%\vspace{-2mm}

\cantusSineNeumas
\initiumpsalmi{temporalia/magnificat-initium-i-a.gtex}

%\vspace{-1.5mm}

\input{temporalia/magnificat-i-a.tex} \Abardot{}}
\newcommand{\oratiovesperas}{\pars{Oratio.}

\noindent Páteant aures misericórdiæ tuæ Dómine précibus supplicántium: \grestar{} et ut peténtibus desideráta concédas, fac eos quæ tibi sunt plácita postuláre.

\noindent Per Dóminum nostrum Iesum Christum, Fílium tuum, qui tecum vivit et regnat in unitáte Spíritus Sancti, Deus, per ómnia sǽcula sæculórum.

\noindent \Rbardot{} Amen.}
\newcommand{\hebdomada}{infra Hebdom. IV per Annum.}
%\newcommand{\hiemalis}{Hiemalis}
\newcommand{\matud}{Matutinum Hebdomadae D}
\newcommand{\matubd}{Matutinum Hebdomadae B vel D}
\newcommand{\laudd}{Laudes Hebdomadae D}
\newcommand{\laudbd}{Laudes Hebdomadae B vel D}

% LuaLaTeX

\documentclass[a4paper, twoside, 12pt]{article}
\usepackage[latin]{babel}
%\usepackage[landscape, left=3cm, right=1.5cm, top=2cm, bottom=1cm]{geometry} % okraje stranky
%\usepackage[landscape, a4paper, mag=1166, truedimen, left=2cm, right=1.5cm, top=1.6cm, bottom=0.95cm]{geometry} % okraje stranky
\usepackage[landscape, a4paper, mag=1400, truedimen, left=0.5cm, right=0.5cm, top=0.5cm, bottom=0.5cm]{geometry} % okraje stranky

\usepackage{fontspec}
\setmainfont[FeatureFile={junicode.fea}, Ligatures={Common, TeX}, RawFeature=+fixi]{Junicode}
%\setmainfont{Junicode}

% shortcut for Junicode without ligatures (for the Czech texts)
\newfontfamily\nlfont[FeatureFile={junicode.fea}, Ligatures={Common, TeX}, RawFeature=+fixi]{Junicode}

\usepackage{multicol}
\usepackage{color}
\usepackage{lettrine}
\usepackage{fancyhdr}

% usual packages loading:
\usepackage{luatextra}
\usepackage{graphicx} % support the \includegraphics command and options
\usepackage{gregoriotex} % for gregorio score inclusion
\usepackage{gregoriosyms}
\usepackage{wrapfig} % figures wrapped by the text
\usepackage{parcolumns}
\usepackage[contents={},opacity=1,scale=1,color=black]{background}
\usepackage{tikzpagenodes}
\usepackage{calc}
\usepackage{longtable}
\usetikzlibrary{calc}

\setlength{\headheight}{14.5pt}

% Commands used to produce a typical "Conventus" booklet

\newenvironment{titulusOfficii}{\begin{center}}{\end{center}}
\newcommand{\dies}[1]{#1

}
\newcommand{\nomenFesti}[1]{\textbf{\Large #1}

}
\newcommand{\celebratio}[1]{#1

}

\newcommand{\hora}[1]{%
\vspace{0.5cm}{\large \textbf{#1}}

\fancyhead[LE]{\thepage\ / #1}
\fancyhead[RO]{#1 / \thepage}
\addcontentsline{toc}{subsection}{#1}
}

% larger unit than a hora
\newcommand{\divisio}[1]{%
\begin{center}
{\Large \textsc{#1}}
\end{center}
\fancyhead[CO,CE]{#1}
\addcontentsline{toc}{section}{#1}
}

% a part of a hora, larger than pars
\newcommand{\subhora}[1]{
\begin{center}
{\large \textit{#1}}
\end{center}
%\fancyhead[CO,CE]{#1}
\addcontentsline{toc}{subsubsection}{#1}
}

% rubricated inline text
\newcommand{\rubricatum}[1]{\textit{#1}}

% standalone rubric
\newcommand{\rubrica}[1]{\vspace{3mm}\rubricatum{#1}}

\newcommand{\notitia}[1]{\textcolor{red}{#1}}

\newcommand{\scriptura}[1]{\hfill \small\textit{#1}}

\newcommand{\translatioCantus}[1]{\vspace{1mm}%
{\noindent\footnotesize \nlfont{#1}}}

% pruznejsi varianta nasledujiciho - umoznuje nastavit sirku sloupce
% s prekladem
\newcommand{\psalmusEtTranslatioB}[3]{
  \vspace{0.5cm}
  \begin{parcolumns}[colwidths={2=#3}, nofirstindent=true]{2}
    \colchunk{
      \input{#1}
    }

    \colchunk{
      \vspace{-0.5cm}
      {\footnotesize \nlfont
        \input{#2}
      }
    }
  \end{parcolumns}
}

\newcommand{\psalmusEtTranslatio}[2]{
  \psalmusEtTranslatioB{#1}{#2}{8.5cm}
}


\newcommand{\canticumMagnificatEtTranslatio}[1]{
  \psalmusEtTranslatioB{#1}{temporalia/extra-adventum-vespers/magnificat-boh.tex}{12cm}
}
\newcommand{\canticumBenedictusEtTranslatio}[1]{
  \psalmusEtTranslatioB{#1}{temporalia/extra-adventum-laudes/benedictus-boh.tex}{10.5cm}
}

% volne misto nad antifonami, kam si zpevaci dokresli neumy
\newcommand{\hicSuntNeumae}{\vspace{0.5cm}}

% prepinani mista mezi notovymi osnovami: pro neumovane a neneumovane zpevy
\newcommand{\cantusCumNeumis}{
  \setgrefactor{17}
  \global\advance\grespaceabovelines by 5mm%
}
\newcommand{\cantusSineNeumas}{
  \setgrefactor{17}
  \global\advance\grespaceabovelines by -5mm%
}

% znaky k umisteni nad inicialu zpevu
\newcommand{\superInitialam}[1]{\gresetfirstlineaboveinitial{\small {\textbf{#1}}}{\small {\textbf{#1}}}}

% pars officii, i.e. "oratio", ...
\newcommand{\pars}[1]{\textbf{#1}}

\newenvironment{psalmus}{
  \setlength{\parindent}{0pt}
  \setlength{\parskip}{5pt}
}{
  \setlength{\parindent}{10pt}
  \setlength{\parskip}{10pt}
}

%%%% Prejmenovat na latinske:
\newcommand{\nadpisZalmu}[1]{
  \hspace{2cm}\textbf{#1}\vspace{2mm}%
  \nopagebreak%

}

% mode, score, translation
\newcommand{\antiphona}[3]{%
\hicSuntNeumae
\superInitialam{#1}
\includescore{#2}

#3
}
 % Often used macros

\newcommand{\annusEditionis}{2021}

%%%% Vicekrat opakovane kousky

\newcommand{\anteOrationem}{
  \rubrica{Ante Orationem, cantatur a Superiore:}

  \pars{Supplicatio Litaniæ.}

  \cuminitiali{}{temporalia/supplicatiolitaniae.gtex}

  \pars{Oratio Dominica.}

  \cuminitiali{}{temporalia/oratiodominica.gtex}

  \rubrica{Deinde dicitur ab Hebdomadario:}

  \cuminitiali{}{temporalia/dominusvobiscum-solemnis.gtex}

  \rubrica{In choro monialium loco Dominus vobiscum dicitur:}

  \sineinitiali{temporalia/domineexaudi.gtex}
}

\setlength{\columnsep}{30pt} % prostor mezi sloupci

%%%%%%%%%%%%%%%%%%%%%%%%%%%%%%%%%%%%%%%%%%%%%%%%%%%%%%%%%%%%%%%%%%%%%%%%%%%%%%%%%%%%%%%%%%%%%%%%%%%%%%%%%%%%%
\begin{document}

% Here we set the space around the initial.
% Please report to http://home.gna.org/gregorio/gregoriotex/details for more details and options
\grechangedim{afterinitialshift}{2.2mm}{scalable}
\grechangedim{beforeinitialshift}{2.2mm}{scalable}
\grechangedim{interwordspacetext}{0.22 cm plus 0.15 cm minus 0.05 cm}{scalable}%
\grechangedim{annotationraise}{-0.2cm}{scalable}

% Here we set the initial font. Change 38 if you want a bigger initial.
% Emit the initials in red.
\grechangestyle{initial}{\color{red}\fontsize{38}{38}\selectfont}

\pagestyle{empty}

%%%% Titulni stranka
\begin{titulusOfficii}
\ifx\titulus\undefined
\nomenFesti{Feria IV \hebdomada{}}
\else
\titulus
\fi
\end{titulusOfficii}

\vfill

\begin{center}
%Ad usum et secundum consuetudines chori \guillemotright{}Conventus Choralis\guillemotleft.

%Editio Sancti Wolfgangi \annusEditionis
\end{center}

\scriptura{}

\pars{}

\pagebreak

\renewcommand{\headrulewidth}{0pt} % no horiz. rule at the header
\fancyhf{}
\pagestyle{fancy}

\cantusSineNeumas

\ifx\oratio\undefined
\ifx\lauda\undefined
\else
\newcommand{\oratio}{\pars{Oratio.}

\noindent Exáudi nos, Deus, salutáris noster et nos pérfice sectatóres lucis et operários veritátis, ut, qui ex te nati sumus lucis fílii, tui testes coram homínibus esse valeámus.

\noindent Per Dóminum nostrum Iesum Christum, Fílium tuum, qui tecum vivit et regnat in unitáte Spíritus Sancti, Deus, per ómnia sǽcula sæculórum.

\noindent \Rbardot{} Amen.}
\fi
\ifx\laudb\undefined
\else
\newcommand{\oratio}{\pars{Oratio.}

\noindent Emítte, quǽsumus, Dómine, in corda nostra tui lúminis claritátem, ut, in via mandatórum tuórum iúgiter ambulántes, nihil umquam patiámur erróris.

\noindent Per Dóminum nostrum Iesum Christum, Fílium tuum, qui tecum vivit et regnat in unitáte Spíritus Sancti, Deus, per ómnia sǽcula sæculórum.

\noindent \Rbardot{} Amen.}
\fi
\ifx\laudc\undefined
\else
\newcommand{\oratio}{\pars{Oratio.}

\noindent Sénsibus nostris, quǽsumus, Dómine, lumen sanctum tuum benígnus infúnde, ut tibi semper simus conversatióne devóti, cuius sapiéntia creáti sumus et providéntia gubernámur.

\noindent Per Dóminum nostrum Iesum Christum, Fílium tuum, qui tecum vivit et regnat in unitáte Spíritus Sancti, Deus, per ómnia sǽcula sæculórum.

\noindent \Rbardot{} Amen.}
\fi
\ifx\laudd\undefined
\else
\newcommand{\oratio}{\pars{Oratio.}

\noindent Memoráre, Dómine, testaménti tui sancti, quod sanguis Agni novo fœ́dere consecrávit, ut pópulus tuus et remissiónem obtíneat peccatórum et ad redemptiónis indesinénter profíciat increméntum.

\noindent Per Dóminum nostrum Iesum Christum, Fílium tuum, qui tecum vivit et regnat in unitáte Spíritus Sancti, Deus, per ómnia sǽcula sæculórum.

\noindent \Rbardot{} Amen.}
\fi
\fi

\hora{Ad Matutinum.} %%%%%%%%%%%%%%%%%%%%%%%%%%%%%%%%%%%%%%%%%%%%%%%%%%%%%

\vspace{2mm}

\cuminitiali{}{temporalia/dominelabiamea.gtex}

\vfill
%\pagebreak

\vspace{2mm}

\ifx\invitatorium\undefined
\pars{Invitatorium.} \scriptura{Ps. 94, 6; Psalmus 94; \textbf{H451}}

\vspace{-4mm}

\antiphona{VI}{temporalia/inv-dominumquifecit.gtex}
\else
\invitatorium
\fi

\vfill
\pagebreak

\ifx\hymnusmatutinum\undefined
\ifx\hiemalis\undefined
\ifx\matua\undefined
\else
\pars{Hymnus}

\cuminitiali{II}{temporalia/hym-ScientiarumDomino-MMMA.gtex}
\fi
\ifx\matub\undefined
\else
\pars{Hymnus.}

\antiphona{VIII}{temporalia/hym-ChristeLuxVera-kempten.gtex}
\fi
\ifx\matuc\undefined
\else
\pars{Hymnus}

\cuminitiali{IV}{temporalia/hym-ScientiarumDomino-kempten.gtex}
\fi
\ifx\matud\undefined
\else
\pars{Hymnus.}

\antiphona{I}{temporalia/hym-ChristeLuxVera.gtex}
\fi
\else
\ifx\matuac\undefined
\else
\pars{Hymnus}

\cuminitiali{IV}{temporalia/hym-RerumCreator.gtex}
\fi
\ifx\matubd\undefined
\else
\pars{Hymnus.}

{
\grechangedim{interwordspacetext}{0.10 cm plus 0.15 cm minus 0.05 cm}{scalable}%
\antiphona{I}{temporalia/hym-OSatorRerum.gtex}
\grechangedim{interwordspacetext}{0.22 cm plus 0.15 cm minus 0.05 cm}{scalable}%
}
\fi
\fi
\else
\hymnusmatutinum
\fi

\vspace{-3mm}

\vfill
\pagebreak

\ifx\matutinum\undefined
\ifx\matua\undefined
\else
% MAT A
\pars{Psalmus 1.} \scriptura{Ps. 17, 2}

\vspace{-4mm}

\antiphona{VI F}{temporalia/ant-diligamtedomine.gtex}

%\vspace{-2mm}

\scriptura{Ps. 17, 2-7}

%\vspace{-2mm}

\initiumpsalmi{temporalia/ps17ii_vii-initium-vi-F-auto.gtex}

\input{temporalia/ps17ii_vii-vi-F.tex} \Abardot{}

\vfill
\pagebreak

\pars{Psalmus 2.} \scriptura{Ps. 29, 11; \textbf{H151}}

\vspace{-4mm}

\antiphona{I g}{temporalia/ant-factusestadiutormeus.gtex}

%\vspace{-2mm}

\scriptura{Ps. 17, 8-20}

%\vspace{-2mm}

\initiumpsalmi{temporalia/ps17viii_xx-initium-i-g-auto.gtex}

\input{temporalia/ps17viii_xx-i-g.tex}

\vfill

\antiphona{}{temporalia/ant-factusestadiutormeus.gtex}

\vfill
\pagebreak

\pars{Psalmus 3.} \scriptura{Ps. 17, 21}

\vspace{-4mm}

\antiphona{IV* e}{temporalia/ant-retribuetmihi.gtex}

%\vspace{-2mm}

\scriptura{Ps. 17, 21-30}

%\vspace{-2mm}

\initiumpsalmi{temporalia/ps17xxi_xxx-initium-iv_-e-auto.gtex}

\input{temporalia/ps17xxi_xxx-iv_-e.tex} \Abardot{}

\vfill
\pagebreak
\fi
\ifx\matub\undefined
\else
% MAT B
\pars{Psalmus 1.} \scriptura{Ps. 38, 2; \textbf{H93}}

\vspace{-4mm}

\antiphona{IV E}{temporalia/ant-utnondelinquam.gtex}

%\vspace{-2mm}

\scriptura{Ps. 38, 2-7}

%\vspace{-2mm}

\initiumpsalmi{temporalia/ps38i-initium-iv-E-auto.gtex}

\input{temporalia/ps38i-iv-E.tex} \Abardot{}

\vfill
\pagebreak

\pars{Psalmus 2.} \scriptura{Ier. 17, 17; \textbf{H177}}

\vspace{-4mm}

\antiphona{VII c}{temporalia/ant-nonsismihi.gtex}

%\vspace{-2mm}

\scriptura{Ps. 38, 8-14}

%\vspace{-2mm}

\initiumpsalmi{temporalia/ps38ii-initium-vii-c-auto.gtex}

\input{temporalia/ps38ii-vii-c.tex} \Abardot{}

\vfill
\pagebreak

\pars{Psalmus 3.}

\vspace{-4mm}

\antiphona{IV* e}{temporalia/ant-exspectabonomentuum.gtex}

%\vspace{-2mm}

\scriptura{Ps. 51}

%\vspace{-2mm}

\initiumpsalmi{temporalia/ps51-initium-iv_-e-auto.gtex}

\input{temporalia/ps51-iv_-e.tex} \Abardot{}

\vfill
\pagebreak
\fi
\ifx\matuc\undefined
\else
% MAT C
\pars{Psalmus 1.} \scriptura{Ps. 75, 12}

\vspace{-4mm}

\antiphona{VIII G}{temporalia/ant-deusquiglorificatur.gtex}

%\vspace{-2mm}

\scriptura{Ps. 88, 2-19}

%\vspace{-2mm}

\initiumpsalmi{temporalia/ps88ii_xix-initium-viii-G-auto.gtex}

\input{temporalia/ps88ii_xix-viii-G.tex}

\vfill

\antiphona{}{temporalia/ant-deusquiglorificatur.gtex}

\vfill
\pagebreak

\pars{Psalmus 2.} \scriptura{Ps. 131, 11; \textbf{H52}}

\vspace{-5mm}

\antiphona{VIII G}{temporalia/ant-defructuventris.gtex}

\vspace{-2mm}

\scriptura{Ps. 88, 20-30}

\vspace{-2mm}

\initiumpsalmi{temporalia/ps88xx_xxx-initium-viii-G-auto.gtex}

\input{temporalia/ps88xx_xxx-viii-G.tex} \Abardot{}

\vfill
\pagebreak

\pars{Psalmus 3.} \scriptura{Ps. 111, 2; \textbf{H365}}

\vspace{-4mm}

\antiphona{VIII G*}{temporalia/ant-potensinterra.gtex}

%\vspace{-2mm}

\scriptura{Ps. 88, 31-38}

%\vspace{-2mm}

\initiumpsalmi{temporalia/ps88xxxi_xxxviii-initium-viii-G_.gtex}

\input{temporalia/ps88xxxi_xxxviii-viii-G.tex} \Abardot{}

\vfill
\pagebreak
\fi
\ifx\matud\undefined
\else
% MAT D
\pars{Psalmus 1.} \scriptura{Ps. 102, 1; \textbf{H99}}

\vspace{-4mm}

\antiphona{VIII c}{temporalia/ant-benedicanimamea.gtex}

%\vspace{-2mm}

\scriptura{Ps. 102, 1-7}

%\vspace{-2mm}

\initiumpsalmi{temporalia/ps102i-initium-viii-C-auto.gtex}

\input{temporalia/ps102i-viii-C.tex} \Abardot{}

\vfill
\pagebreak

\pars{Psalmus 2.} \scriptura{Ps. 102, 11}

\vspace{-4mm}

\antiphona{I d}{temporalia/ant-supertimentesdominum.gtex}

%\vspace{-2mm}

\scriptura{Ps. 102, 8-16}

%\vspace{-2mm}

\initiumpsalmi{temporalia/ps102ii-initium-i-d-auto.gtex}

\input{temporalia/ps102ii-i-d.tex} \Abardot{}

\vfill
\pagebreak

\pars{Psalmus 3.} \scriptura{Ps. 102, 20; \textbf{H332}}

\vspace{-4mm}

\antiphona{III g}{temporalia/ant-benedicitedomino.gtex}

%\vspace{-5mm}

\scriptura{Ps. 102, 17-22}

%\vspace{-2mm}

\initiumpsalmi{temporalia/ps102iii-initium-iii-g.gtex}

\input{temporalia/ps102iii-iii-g.tex} \Abardot{}

\vfill
\pagebreak
\fi
\else
\matutinum
\fi

\pars{Versus.}

\ifx\matversus\undefined
\ifx\matub\undefined
\else
\noindent \Vbardot{} Sustínuit ánima mea in verbo eius.

\noindent \Rbardot{} Sperávit ánima mea in Dómino.
\fi
\ifx\matuc\undefined
\else
\noindent \Vbardot{} Declarátio sermónum tuórum illúminat.

\noindent \Rbardot{} Et intelléctum dat párvulis.
\fi
\ifx\matud\undefined
\else
\noindent \Vbardot{} Viam mandatórum tuórum, Dómine, fac me intellégere.

\noindent \Rbardot{} Et exercébor in mirabílibus tuis.
\fi
\else
\matversus
\fi

\vspace{5mm}

\sineinitiali{temporalia/oratiodominica-mat.gtex}

\vspace{5mm}

\pars{Absolutio.}

\cuminitiali{}{temporalia/absolutio-avinculis.gtex}

\vfill
\pagebreak

\cuminitiali{}{temporalia/benedictio-solemn-ille.gtex}

\vspace{7mm}

\lectioi

\noindent \Vbardot{} Tu autem, Dómine, miserére nobis.
\noindent \Rbardot{} Deo grátias.

\vfill
\pagebreak

\responsoriumi

\vfill
\pagebreak

\cuminitiali{}{temporalia/benedictio-solemn-divinum.gtex}

\vspace{7mm}

\lectioii

\noindent \Vbardot{} Tu autem, Dómine, miserére nobis.
\noindent \Rbardot{} Deo grátias.

\vfill
\pagebreak

\responsoriumii

\vfill
\pagebreak

\cuminitiali{}{temporalia/benedictio-solemn-adsocietatem.gtex}

\vspace{7mm}

\lectioiii

\noindent \Vbardot{} Tu autem, Dómine, miserére nobis.
\noindent \Rbardot{} Deo grátias.

\vfill
\pagebreak

\responsoriumiii

\vfill
\pagebreak

\rubrica{Reliqua omittuntur, nisi Laudes separandæ sint.}

\sineinitiali{temporalia/domineexaudi.gtex}

\vfill

\oratio

\vfill

\noindent \Vbardot{} Dómine, exáudi oratiónem meam.
\Rbardot{} Et clamor meus ad te véniat.

\vfill

\noindent \Vbardot{} Benedicámus Dómino.
\noindent \Rbardot{} Deo grátias.

\vfill

\noindent \Vbardot{} Fidélium ánimæ per misericórdiam Dei requiéscant in pace.
\Rbardot{} Amen.

\vfill
\pagebreak

\hora{Ad Laudes.} %%%%%%%%%%%%%%%%%%%%%%%%%%%%%%%%%%%%%%%%%%%%%%%%%%%%%

\cantusSineNeumas

\vspace{0.5cm}
\ifx\deusinadiutorium\undefined
\grechangedim{interwordspacetext}{0.18 cm plus 0.15 cm minus 0.05 cm}{scalable}%
\cuminitiali{}{temporalia/deusinadiutorium-communis.gtex}
\grechangedim{interwordspacetext}{0.22 cm plus 0.15 cm minus 0.05 cm}{scalable}%
\else
\deusinadiutorium
\fi

\vfill
\pagebreak

\ifx\hymnuslaudes\undefined
\ifx\lauda\undefined
\else
\pars{Hymnus} \scriptura{Prudentius (\olddag{} 413)}

\cuminitiali{I}{temporalia/hym-NoxEtTenebrae.gtex}
\fi
\ifx\laudb\undefined
\else
\pars{Hymnus}

\cuminitiali{I}{temporalia/hym-FulgentisAuctor-hk.gtex}
\fi
\ifx\laudc\undefined
\else
\pars{Hymnus} \scriptura{Prudentius (\olddag{} 413)}

\cuminitiali{VIII}{temporalia/hym-NoxEtTenebrae-einsiedeln.gtex}
\fi
\ifx\laudd\undefined
\else
\pars{Hymnus}

\grechangedim{interwordspacetext}{0.16 cm plus 0.15 cm minus 0.05 cm}{scalable}%
\cuminitiali{IV}{temporalia/hym-FulgentisAuctor.gtex}
\grechangedim{interwordspacetext}{0.22 cm plus 0.15 cm minus 0.05 cm}{scalable}%
\vspace{-3mm}
\fi
\else
\hymnuslaudes
\fi

\vfill
\pagebreak

\ifx\laudes\undefined
\ifx\lauda\undefined
\else
\pars{Psalmus 1.} \scriptura{Ps. 35, 6}

\vspace{-4mm}

\antiphona{II D}{temporalia/ant-domineincaelo.gtex}

\scriptura{Psalmus 35.}

\initiumpsalmi{temporalia/ps35-initium-ii-D-auto.gtex}

\input{temporalia/ps35-ii-D.tex} \Abardot{}

\vfill
\pagebreak

\pars{Psalmus 2.} \scriptura{Idt. 16, 16}

\vspace{-4mm}

\antiphona{II* d}{temporalia/ant-dominemagnuses.gtex}

\scriptura{Canticum Iudith, Idt. 16, 2.15-19}

\initiumpsalmi{temporalia/iudith2-initium-ii_-d-auto.gtex}

\input{temporalia/iudith2-ii_-d.tex} \Abardot{}

\vfill
\pagebreak

\pars{Psalmus 3.} \scriptura{Ps. 46, 7-8; \textbf{H44}}

\vspace{-4mm}

\antiphona{VIII c}{temporalia/ant-rexomnisterrae.gtex}

%\vspace{-2mm}

\scriptura{Psalmus 46.}

%\vspace{-2mm}

\initiumpsalmi{temporalia/ps46-initium-viii-c-auto.gtex}

\input{temporalia/ps46-viii-c.tex} \Abardot{}

\vfill
\pagebreak
\fi
\ifx\laudb\undefined
\else
\pars{Psalmus 1.} \scriptura{Ps. 76, 14}

\vspace{-4mm}

\antiphona{VIII G}{temporalia/ant-deusinsancto.gtex}

\scriptura{Psalmus 76.}

\initiumpsalmi{temporalia/ps76-initium-viii-G-auto.gtex}

\input{temporalia/ps76-viii-G.tex}

\vfill

\antiphona{}{temporalia/ant-deusinsancto.gtex}

\vfill
\pagebreak

\pars{Psalmus 2.} \scriptura{1 Reg. 2, 1; \textbf{H96}}

\vspace{-4mm}

\antiphona{II D}{temporalia/ant-exsultavitcormeum.gtex}

\vspace{-4mm}

\scriptura{Canticum Annæ, 1 Reg. 2, 1-10}

\vspace{-3mm}

\initiumpsalmi{temporalia/anna-initium-ii-D-auto.gtex}

\input{temporalia/anna-ii-D.tex}

\vfill

\antiphona{}{temporalia/ant-exsultavitcormeum.gtex}

\vfill
\pagebreak

\pars{Psalmus 3.} \scriptura{Ps. 96, 1}

\vspace{-4mm}

\antiphona{III a\textsuperscript{2}}{temporalia/ant-dominusregnavit.gtex}

\vspace{-2mm}

\scriptura{Psalmus 96.}

\vspace{-2mm}

\initiumpsalmi{temporalia/ps96-initium-iii-a2.gtex}

\input{temporalia/ps96-iii-a2.tex} \Abardot{}

\vfill
\pagebreak
\fi
\ifx\laudc\undefined
\else
\pars{Psalmus 1.} \scriptura{Ps. 85, 1}

\vspace{-4mm}

\antiphona{II D}{temporalia/ant-inclinadomineaurem.gtex}

\scriptura{Psalmus 85.}

\initiumpsalmi{temporalia/ps85-initium-ii-D-auto.gtex}

\input{temporalia/ps85-ii-D.tex}

\vfill

\antiphona{}{temporalia/ant-inclinadomineaurem.gtex}

\vfill
\pagebreak

\pars{Psalmus 2.} \scriptura{Ps. 118, 1; \textbf{H91}}

\vspace{-4mm}

\antiphona{VIII G}{temporalia/ant-beatiquiambulant.gtex}

%\vspace{-2mm}

\scriptura{Canticum Isaiaæ, Is. 33, 13-16}

%\vspace{-2mm}

\initiumpsalmi{temporalia/isaiae8-initium-viii-g-auto.gtex}

\input{temporalia/isaiae8-viii-g.tex} \Abardot{}

\vfill
\pagebreak

\pars{Psalmus 3.} \scriptura{Ps. 97, 1; \textbf{H98}}

\vspace{-4mm}

\antiphona{E}{temporalia/ant-quiamirabiliafecit.gtex}

%\vspace{-2mm}

\scriptura{Psalmus 97.}

%\vspace{-2mm}

\initiumpsalmi{temporalia/ps97-initium-e.gtex}

\input{temporalia/ps97-e.tex} \Abardot{}

\vfill
\pagebreak
\fi
\ifx\laudd\undefined
\else
\pars{Psalmus 1.} \scriptura{Ps. 107, 2}

\vspace{-4mm}

\antiphona{VIII G}{temporalia/ant-paratumcormeum.gtex}

\vspace{-2mm}

\scriptura{Psalmus 107.}

\vspace{-2mm}

\initiumpsalmi{temporalia/ps107-initium-viii-G-auto.gtex}

\input{temporalia/ps107-viii-G.tex} \Abardot{}

\vfill
\pagebreak

\pars{Psalmus 2.} \scriptura{Is. 61, 10}

\vspace{-4mm}

\antiphona{VII c}{temporalia/ant-induitmedominus.gtex}

\vspace{-1mm}

\scriptura{Canticum Isaiaæ, Is. 61, 10-11; 62, 1-7}

\vspace{-3mm}

\initiumpsalmi{temporalia/isaiae4-initium-vii-c-auto.gtex}

\vspace{-1.5mm}

\input{temporalia/isaiae4-vii-c.tex} \Abardot{}

\vfill
\pagebreak

\pars{Psalmus 3.} \scriptura{Ps. 145, 2; \textbf{H100}}

\vspace{-4mm}

\antiphona{IV* e}{temporalia/ant-laudabodeum.gtex}

\scriptura{Psalmus 145.}

\initiumpsalmi{temporalia/ps145-initium-iv_-e-auto.gtex}

\input{temporalia/ps145-iv_-e.tex} \Abardot{}

\vfill
\pagebreak
\fi
\else
\laudes
\fi

\ifx\lectiobrevis\undefined
\ifx\lauda\undefined
\else
\pars{Lectio Brevis.} \scriptura{Tob. 4, 14-15.16.19}

\noindent Atténde tibi, fili, in ómnibus opéribus tuis et esto sápiens in ómnibus sermónibus tuis et, quod óderis, némini féceris. De pane tuo commúnica esuriénti et de vestiméntis tuis nudis; ex ómnibus, quæcúmque tibi abundáverint, fac eleemósynam. Omni témpore bénedic Dóminum et póstula ab illo, ut dirigántur viæ tuæ et omnes sémitæ tuæ et consília bene disponántur.
\fi
\ifx\laudb\undefined
\else
\pars{Lectio Brevis.} \scriptura{Rom. 8, 35.37}

\noindent Quis nos separábit a caritáte Christi? Tribulátio an angústia an persecútio an fames an núditas an perículum an gládius? Sed in his ómnibus supervíncimus per eum, qui diléxit nos.
\fi
\ifx\laudc\undefined
\else
\pars{Lectio Brevis.} \scriptura{Iob 1, 21; 2, 10}

\noindent Nudus egréssus sum de útero matris meæ et nudus revértar illuc. Dóminus dedit, Dóminus ábstulit; sicut Dómino plácuit, ita factum est: sit nomen Dómini benedíctum. Si bona suscépimus de manu Dei, mala quare non suscipiámus?
\fi
\ifx\laudd\undefined
\else
\pars{Lectio Brevis.} \scriptura{Dt. 4, 39-40}

\noindent Scito hódie et cogitáto in corde tuo quod Dóminus ipse sit Deus in cælo sursum et in terra deórsum, et non sit álius. Custódi præcépta eius atque mandáta, quæ ego præcípio tibi hódie.
\fi
\else
\lectiobrevis
\fi

\vfill

\ifx\responsoriumbreve\undefined
\ifx\laudac\undefined
\else
\pars{Responsorium breve.} \scriptura{Ps. 118, 36-37}

\cuminitiali{IV}{temporalia/resp-inclinacormeum.gtex}
\fi
\ifx\laudbd\undefined
\else
\pars{Responsorium breve.} \scriptura{Ps. 33, 2}

\cuminitiali{VI}{temporalia/resp-benedicamdominum.gtex}
\fi
\else
\responsoriumbreve
\fi
\vfill
\pagebreak

\ifx\benedictus\undefined
\ifx\laudac\undefined
\else
\pars{Canticum Zachariæ.} \scriptura{Lc. 1, 72.68}

\vspace{-4mm}

{
\grechangedim{interwordspacetext}{0.18 cm plus 0.15 cm minus 0.05 cm}{scalable}%
\antiphona{VIII G}{temporalia/ant-memoraredominetestamenti.gtex}
\grechangedim{interwordspacetext}{0.22 cm plus 0.15 cm minus 0.05 cm}{scalable}%
}

%\vspace{-3mm}

\scriptura{Lc. 1, 68-79}

%\vspace{-1mm}

\initiumpsalmi{temporalia/benedictus-initium-viii-G-auto.gtex}

\input{temporalia/benedictus-viii-G.tex} \Abardot{}
\fi
\ifx\laudbd\undefined
\else
\pars{Canticum Zachariæ.} \scriptura{Lc. 1, 74-75}

\vspace{-4mm}

{
\grechangedim{interwordspacetext}{0.18 cm plus 0.15 cm minus 0.05 cm}{scalable}%
\antiphona{VIII G\textsuperscript{2}}{temporalia/ant-liberatiserviamusdeo.gtex}
\grechangedim{interwordspacetext}{0.22 cm plus 0.15 cm minus 0.05 cm}{scalable}%
}

%\vspace{-3mm}

\scriptura{Lc. 1, 68-79}

%\vspace{-1mm}

\initiumpsalmi{temporalia/benedictus-initium-viii-G2-auto.gtex}

\input{temporalia/benedictus-viii-G2.tex} \Abardot{}
\fi
\else
\benedictus
\fi

\vspace{-1cm}

\vfill
\pagebreak

\pars{Preces.}

\sineinitiali{}{temporalia/tonusprecum.gtex}

\ifx\preces\undefined
\ifx\lauda\undefined
\else
\noindent Grátias agámus Christo eúmque semper laudémus, quia non dedignátur fratres vocáre quos sanctíficat. \gredagger{} Ideo ei supplicémus:

\Rbardot{} Sanctífica fratres tuos, Dómine.

\noindent Fac ut huius diéi inítia in honórem resurrectiónis tuæ puris tibi córdibus consecrémus, \gredagger{} et diem totum tibi gratum sanctificatiónis opéribus faciámus.

\Rbardot{} Sanctífica fratres tuos, Dómine.

\noindent Qui diem, amóris tui signum, ad salútem et lætítiam nobis renovásti, \gredagger{} rénova nos cotídie ad glóriam tuam.

\Rbardot{} Sanctífica fratres tuos, Dómine.

\noindent Doce nos hódie te in ómnibus præséntem agnóscere, \gredagger{} teque in mæréntibus præsértim et paupéribus inveníre.

\Rbardot{} Sanctífica fratres tuos, Dómine.

\noindent Da nos hódie cum ómnibus pacem habére, \gredagger{} némini vero malum réddere pro malo.

\Rbardot{} Sanctífica fratres tuos, Dómine.
\fi
\ifx\laudb\undefined
\else
\noindent Benedíctus Deus salvátor noster, qui usque ad consummatiónem sǽculi se ómnibus diébus cum Ecclésia sua mansúrum promísit. \gredagger{} Ideo ei grátias agéntes clamémus:

\Rbardot{} Mane nobíscum, Dómine.
Mane nobíscum, Dómine, toto die, \gredagger{} numquam declínet a nobis sol grátiæ tuæ.

\Rbardot{} Mane nobíscum, Dómine.
Hunc diem tibi tamquam oblatiónem consecrámus, \gredagger{} dum nos nihil pravi factúros aut probatúros pollicémur.

\Rbardot{} Mane nobíscum, Dómine.
Fac, Dómine, ut donum lucis hic totus dies evádat, \gredagger{} ut simus sal terræ et lux mundi.

\Rbardot{} Mane nobíscum, Dómine.
Spíritus Sancti tui cáritas dírigat corda et lábia nostra, \gredagger{} ut in tua iustítia semper et laude maneámus.

\Rbardot{} Mane nobíscum, Dómine.
\fi
\ifx\laudc\undefined
\else
\noindent Christum, qui nutrit et fovet Ecclésiam pro qua seípsum trádidit, \gredagger{} hac deprecatióne rogémus:

\Rbardot{} Réspice Ecclésiam tuam, Dómine.

\noindent Pastor Ecclésiæ tuæ, benedíctus es, qui nobis hódie tríbuis lucem et vitam, \gredagger{} nos gratos redde pro múnere tam iucúndo.

\Rbardot{} Réspice Ecclésiam tuam, Dómine.

\noindent Gregem tuum misericórditer réspice, quem in tuo nómine congregásti, \gredagger{} ne quis péreat eórum, quos Pater dedit tibi.

\Rbardot{} Réspice Ecclésiam tuam, Dómine.

\noindent Ecclésiam tuam duc in via mandatórum tuórum, \gredagger{} fidélem tibi Spíritus Sanctus eam effíciat.

\Rbardot{} Réspice Ecclésiam tuam, Dómine.

\noindent Ecclésiam ad mensam verbi tui panísque vivífica, \gredagger{} ut in fortitúdine huius cibi te læta sequátur.

\Rbardot{} Réspice Ecclésiam tuam, Dómine.
\fi
\ifx\laudd\undefined
\else
\noindent Christus, splendor patérnæ glóriæ, verbo suo nos illúminat. \gredagger{} Eum amánter invocémus dicéntes:

\Rbardot{} Exáudi nos, Rex ætérnæ glóriæ.

\noindent Benedíctus es, auctor fídei nostræ et consummátor, \gredagger{} qui de ténebris vocásti nos in admirábile lumen tuum.

\Rbardot{} Exáudi nos, Rex ætérnæ glóriæ.

\noindent Qui cæcórum óculos aperuísti, et surdos fecísti audíre, \gredagger{} ádiuva incredulitátem nostram.

\Rbardot{} Exáudi nos, Rex ætérnæ glóriæ.

\noindent Dómine, in dilectióne tua iúgiter maneámus, \gredagger{} ne ab ínvicem separémur.

\Rbardot{} Exáudi nos, Rex ætérnæ glóriæ.

\noindent Da nobis in tentatióne resístere, in tribulatióne sustinére, \gredagger{} et in prósperis grátias ágere.

\Rbardot{} Exáudi nos, Rex ætérnæ glóriæ.
\fi
\else
\preces
\fi

\vfill

\pars{Oratio Dominica.}

\cuminitiali{}{temporalia/oratiodominicaalt.gtex}

\vfill
\pagebreak

\rubrica{vel:}

\pars{Supplicatio Litaniæ.}

\cuminitiali{}{temporalia/supplicatiolitaniae.gtex}

\vfill

\pars{Oratio Dominica.}

\cuminitiali{}{temporalia/oratiodominica.gtex}

\vfill
\pagebreak

% Oratio. %%%
\oratio

\vspace{-1mm}

\vfill

\rubrica{Hebdomadarius dicit Dominus vobiscum, vel, absente sacerdote vel diacono, sic concluditur:}

\vspace{2mm}

\antiphona{C}{temporalia/dominusnosbenedicat.gtex}

\rubrica{Postea cantatur a cantore:}

\vspace{2mm}

\cuminitiali{IV}{temporalia/benedicamus-feria-laudes.gtex}

\vspace{1mm}

\vfill
\pagebreak

\end{document}

