\newcommand{\sineobmv}{Sine Officium B.M.V. in Sabbato.}
\newcommand{\oratio}{\pars{Oratio.}

\noindent Ad te corda nostra, Pater ætérne, convérte, ut nos, unum necessárium semper quæréntes et ópera caritátis exercéntes, tuo cúltui præstes esse dicátos.

\pars{Pro commemoratione Sancti Casimiri.} \scriptura{Mt. 7, 24}

\vspace{-4mm}

\antiphona{I d\textsuperscript{3}}{temporalia/ant-similaboeum.gtex}

\vfill

\noindent Deus omnípotens, cui servíre regnáre est, concéde nobis, beáti Casimíri intercedénte suffrágio, tibi in sanctitáte et iustítia perpétuo famulári.

\pars{Pro pace in universo mundo.} \scriptura{Sir. 50, 25; 2 Esdr. 4, 20; \textbf{H416}}

\vspace{-4mm}

\antiphona{II D}{temporalia/ant-dapacemdomine.gtex}

\vfill

\noindent Deus, a quo sancta desidéria, recta consília et iusta sunt ópera: da servis tuis illam, quam mundus dare non potest, pacem; ut et corda nostra mandátis tuis dédita, et hóstium subláta formídine, témpora sint tua protectióne tranquílla.

\noindent Per Dóminum nostrum Iesum Christum, Fílium tuum, qui tecum vivit et regnat in unitáte Spíritus Sancti, Deus, per ómnia sǽcula sæculórum.

\noindent \Rbardot{} Amen.}
\newcommand{\invitatorium}{\pars{Invitatorium.} \scriptura{Ps. 94, 8; Psalmus 94; \textbf{H143}}

\vspace{-4mm}

\antiphona{E}{temporalia/inv-hodiesivocem.gtex}}
\newcommand{\hymnusmatutinum}{\pars{Hymnus}

\cuminitiali{I}{temporalia/hym-NuncTempus.gtex}}
\newcommand{\matutinum}{\pars{Psalmus 1.} \scriptura{Ps. 104, 3; \textbf{H99}}

\vspace{-6mm}

\antiphona{D}{temporalia/ant-laeteturcor.gtex}

\vspace{-4mm}

\scriptura{Ps. 104, 1-15}

\vspace{-2mm}

\initiumpsalmi{temporalia/ps104i-initium-d-g-auto.gtex}

\vspace{-1.5mm}

\input{temporalia/ps104i-d-g.tex} \Abardot{}

\vfill
\pagebreak

\pars{Psalmus 2.} \scriptura{Ps. 113, 1; \textbf{H94}}

\vspace{-4mm}

\antiphona{VIII a}{temporalia/ant-domusiacob.gtex}

%\vspace{-2mm}

\scriptura{Ps. 104, 16-27}

%\vspace{-2mm}

\initiumpsalmi{temporalia/ps104ii-initium-viii-a-auto.gtex}

\input{temporalia/ps104ii-viii-a.tex} \Abardot{}

\vfill
\pagebreak

\pars{Psalmus 3.} \scriptura{Ps. 104, 43}

\vspace{-4mm}

\antiphona{IV E}{temporalia/ant-eduxitdeus.gtex}

%\vspace{-2mm}

\scriptura{Ps. 104, 28-45}

%\vspace{-2mm}

\initiumpsalmi{temporalia/ps104iii-initium-iv-E-auto.gtex}

\input{temporalia/ps104iii-iv-E.tex}

\vfill

\antiphona{}{temporalia/ant-eduxitdeus.gtex}

\vfill
\pagebreak}
\newcommand{\matversus}{\noindent \Vbardot{} Qui facit veritátem, venit ad lucem.

\noindent \Rbardot{} Ut manifesténtur ópera eius.}
\newcommand{\lectioi}{\vspace{-4mm}

\pars{Lectio I.} \scriptura{Gn. 25, 19-34}

\noindent De libro Génesis.

\noindent Hæ quoque sunt generatiónes Isaac fílii Abraham: Abraham génuit Isaac: qui cum quadragínta esset annórum, duxit uxórem Rebéccam fíliam Bathuélis Syri de Mesopotámia, sorórem Laban. Deprecatúsque est Isaac Dóminum pro uxóre sua, eo quod esset stérilis: qui exaudívit eum, et dedit concéptum Rebéccæ. Sed collidebántur in útero eius párvuli; quæ ait: Si sic mihi futúrum erat, quid necésse fuit concípere? perrexítque ut consúleret Dóminum. Qui respóndens ait: Duæ gentes sunt in útero tuo, et duo pópuli ex ventre tuo dividéntur, populúsque pópulum superábit, et maior sérviet minóri.Iam tempus pariéndi advénerat, et ecce gémini in útero eius repérti sunt. Qui prior egréssus est, rufus erat, et totus in morem pellis híspidus: vocatúmque est nomen eius Esáu. Prótinus alter egrédiens, plantam fratris tenébat manu: et idcírco appellávit eum Iacob. Sexagenárius erat Isaac quando nati sunt ei párvuli.

\noindent Quibus adúltis, factus est Esáu vir gnarus venándi, et homo agrícola: Iacob autem vir simplex habitábat in tabernáculis. Isaac amábat Esáu, eo quod de venatiónibus illíus vescerétur: et Rebécca diligébat Iacob. Coxit autem Iacob pulméntum: ad quem cum venísset Esáu de agro lassus, ait: Da mihi de coctióne hac rufa, quia óppido lassus sum. Quam ob causam vocátum est nomen eius Edom. Cui dixit Iacob: Vende mihi primogénita tua. Ille respóndit: En mórior, quid mihi próderunt primogénita? Ait Iacob: iura ergo mihi. iurávit ei Esáu et véndidit primogénita. Et sic, accépto pane et lentis edúlio, cómedit et bibit, et ábiit, parvipéndens quod primogénita vendidísset.}
\newcommand{\responsoriumi}{\pars{Responsorium 1.} \scriptura{\Vbardot{} 2 Cor. 6, 2.3; \textbf{H143}}

\vspace{-5mm}

\responsorium{VIII}{temporalia/resp-paradisiportas-CROCHU.gtex}{}}
\newcommand{\lectioii}{\pars{Lectio II.} \scriptura{Sermo 5,5-6: CCL 91A, 921-923}

\noindent Ex Sermónibus sancti Fulgéntii Ruspénsis epíscopi.

\noindent Recordémur, fratres, verbórum Dómini dicéntis: \emph{Dilígite inimícos vestros, benefácite his qui vos odérunt, et oráte pro persequéntibus et calumniántibus vos.} Ecce Dóminus usque ad inimícos caritátem iubet exténdi et usque ad persecutóres christiáni cordis benevoléntiam dilatári. Et quæ merces óperum tantórum dábitur? Vel quod munus præcépto huic obœdiéntibus conferétur? Ipse demónstret a se præparátam caritáti mercédem, qui per Spíritum Sanctum gratis ipsam dignátur infúndere caritátem; ipse nobis dicat quid pro caritáte redditúrus sit dignis, qui eámdem caritátem donáre dignátur indígnis. Dicat ígitur Dóminus, dicat, et próprio sermóne magnitúdinem nobis suæ promissiónis osténdat: \emph{Ut sitis fílii Patris vestri qui in cælis est.}

\noindent Cum enim diligéndos præcíperet inimícos, amárum fórsitan erat quod audiénti iubebátur; sit, quæso, dulce quod obœdiénti promíttitur. Teneátur ergo dulcédinis huius in corde suávitas, et amaritúdinis illíus superábitur difficúltas. Qui enim diléxerint inimícos suos et benefécerint eis qui eos odérunt, fílii Dei erunt.}
\newcommand{\responsoriumii}{\pars{Responsorium 2.} \scriptura{\Rbardot{} Cf. Bar. 3, 2 \Vbardot{} Idt. 7, 19; \textbf{H144}}

\vspace{-5mm}

\responsorium{II}{temporalia/resp-emendemusinmelius-CROCHU.gtex}{}}

\newcommand{\lectioiii}{\pars{Lectio III.} \scriptura{Sermo 5,5-6: CCL 91A, 921-923}

\noindent Quid vero acceptúri sint isti fílii Dei, beátus Apóstolus enúntiat dicens: \emph{Ipse Spíritus testimónium reddit spirítui nostro, quia sumus fílii Dei. Si autem fílii, et herédes; herédes quidem Dei, coherédes autem Christi.} Audíte ígitur, christiáni; audíte, fílii Dei; audíte, herédes Dei et coherédes Christi. Ut patérnam possideátis hereditátem, non solum amícis sed étiam inimícis impéndite caritátem.

\noindent Nulli cáritas negétur, quæ ab homínibus bonis commúniter possidétur. Omnes eam simul habéte, et ut eam magis habeátis, et bonis eam et malis impéndite. Posséssio enim est bonórum ista commúnis, non terréna útique sed cæléstis; et ídeo nullum in ea facit angustiári persóna consórtis. Tantum vero augétur cáritas, quantum fúerit imminúta cupíditas; et illum facit cáritas semper líberum, quem non tenúerit cupíditas mundána captívum. Cáritas donum Dei est, dicénte Apóstolo: \emph{Cáritas Dei diffúsa est in córdibus nostris per Spíritum Sanctum qui datus est nobis.} Cupíditas láqueus est diáboli et non solum láqueus sed étiam gládius; per ipsam captos interfécit. Cáritas est radix ómnium bonórum ; cáritas semper lætátur quóniam quantum multiplicátur, tantum lárgius erogátur.}
\newcommand{\responsoriumiii}{\pars{Responsorium 3.} \scriptura{\Rbardot{} Ps. 90, 11-12 \Vbardot{} ibid., 13; \textbf{H145}}

\vspace{-5mm}

\responsorium{I}{temporalia/resp-angelissuismandavit-CROCHU-cumdox.gtex}{}}
\newcommand{\lectiobrevis}{\pars{Lectio Brevis.} \scriptura{Is. 1, 16-18}

\noindent Lavámini, mundi estóte, auférte malum cogitatiónum vestrárum ab óculis meis; quiéscite ágere pervérse, díscite benefácere: quǽrite iudícium, subveníte opprésso, iudicáte pupíllo, deféndite víduam. Et veníte et iudício contendámus, dicit Dóminus. Si fúerint peccáta vestra ut cóccinum, quasi nix dealbabúntur; et si fúerint rubra quasi vermículus, velut lana erunt.}
\newcommand{\responsoriumbreve}{\pars{Responsorium breve.} \scriptura{Ps. 90, 3}

\cuminitiali{IV}{temporalia/resp-ipseliberavitme.gtex}}
\newcommand{\hymnuslaudes}{\pars{Hymnus}

\cuminitiali{D}{temporalia/hym-IamChriste.gtex}}
\newcommand{\preces}{\noindent Christum Dóminum glorificémus,~\gredagger{} qui, ut nova creatúra hómines fíerent, lavácrum regeneratiónis instítuit eísque córporis et verbi sui mensam appósuit.~\grestar{} Eum deprecémur, dicéntes:

\Rbardot{} Rénova nos, Dómine, grátia tua.

\noindent Iesu, mitis et húmilis corde, índue nos víscera misericórdiæ,~\gredagger{} benignitátem et humilitátem concéde,~\grestar{} ac patiéntiam cum ómnibus fac nos sectári.

\Rbardot{} Rénova nos, Dómine, grátia tua.

\noindent Doce nos vere próximos esse míseris atque afflíctis,~\grestar{} ut te bonum Samaritánum imitémur.

\Rbardot{} Rénova nos, Dómine, grátia tua.

\noindent Beáta Virgo, mater tua, intercédat pro sacris virgínibus,~\grestar{} ut consecratiónem, qua tibi sunt devótæ, in Ecclésia impénsius colant.

\Rbardot{} Rénova nos, Dómine, grátia tua.

\noindent Donum tuæ misericórdiæ nobis largíre~\grestar{} ac peccáta et pœnas nobis dimítte.

\Rbardot{} Rénova nos, Dómine, grátia tua.}
\newcommand{\benedictus}{\pars{Canticum Zachariæ.} \scriptura{Mt. 5, 44-45}

%\vspace{-4mm}

{
\grechangedim{interwordspacetext}{0.18 cm plus 0.15 cm minus 0.05 cm}{scalable}%
\antiphona{VII d\textsuperscript{2}}{temporalia/ant-oratepropersequentibus.gtex}
\grechangedim{interwordspacetext}{0.22 cm plus 0.15 cm minus 0.05 cm}{scalable}%
}

\vspace{-3mm}

\scriptura{Lc. 1, 68-79}

\vspace{-2mm}

\initiumpsalmi{temporalia/benedictus-initium-vii-d2-auto.gtex}

\vspace{-1.5mm}

\input{temporalia/benedictus-vii-d2.tex} \Abardot{}}
\newcommand{\hebdomada}{infra Hebdom. I per Annum.}
\newcommand{\matua}{Matutinum Hebdomadae A}
\newcommand{\matuac}{Matutinum Hebdomadae A vel C}
\newcommand{\lauda}{Laudes Hebdomadae A}
\newcommand{\laudac}{Laudes Hebdomadae A vel C}

% LuaLaTeX

\documentclass[a4paper, twoside, 12pt]{article}
\usepackage[latin]{babel}
%\usepackage[landscape, left=3cm, right=1.5cm, top=2cm, bottom=1cm]{geometry} % okraje stranky
%\usepackage[landscape, a4paper, mag=1166, truedimen, left=2cm, right=1.5cm, top=1.6cm, bottom=0.95cm]{geometry} % okraje stranky
\usepackage[landscape, a4paper, mag=1400, truedimen, left=0.5cm, right=0.5cm, top=0.5cm, bottom=0.5cm]{geometry} % okraje stranky

\usepackage{fontspec}
\setmainfont[FeatureFile={junicode.fea}, Ligatures={Common, TeX}, RawFeature=+fixi]{Junicode}
%\setmainfont{Junicode}

% shortcut for Junicode without ligatures (for the Czech texts)
\newfontfamily\nlfont[FeatureFile={junicode.fea}, Ligatures={Common, TeX}, RawFeature=+fixi]{Junicode}

% Hebrew font:
% http://scripts.sil.org/cms/scripts/page.php?site_id=nrsi&id=SILHebrUnic2
\newfontfamily\hebfont[Scale=1]{Ezra SIL}

\usepackage{multicol}
\usepackage{color}
\usepackage{lettrine}
\usepackage{fancyhdr}

% usual packages loading:
\usepackage{luatextra}
\usepackage{graphicx} % support the \includegraphics command and options
\usepackage{gregoriotex} % for gregorio score inclusion
\usepackage{gregoriosyms}
\usepackage{wrapfig} % figures wrapped by the text
\usepackage{parcolumns}
\usepackage[contents={},opacity=1,scale=1,color=black]{background}
\usepackage{tikzpagenodes}
\usepackage{calc}
\usepackage{longtable}
\usetikzlibrary{calc}

\setlength{\headheight}{14.5pt}

% Commands used to produce a typical "Conventus" booklet

\newenvironment{titulusOfficii}{\begin{center}}{\end{center}}
\newcommand{\dies}[1]{#1

}
\newcommand{\nomenFesti}[1]{\textbf{\Large #1}

}
\newcommand{\celebratio}[1]{#1

}

\newcommand{\hora}[1]{%
\vspace{0.5cm}{\large \textbf{#1}}

\fancyhead[LE]{\thepage\ / #1}
\fancyhead[RO]{#1 / \thepage}
\addcontentsline{toc}{subsection}{#1}
}

% larger unit than a hora
\newcommand{\divisio}[1]{%
\begin{center}
{\Large \textsc{#1}}
\end{center}
\fancyhead[CO,CE]{#1}
\addcontentsline{toc}{section}{#1}
}

% a part of a hora, larger than pars
\newcommand{\subhora}[1]{
\begin{center}
{\large \textit{#1}}
\end{center}
%\fancyhead[CO,CE]{#1}
\addcontentsline{toc}{subsubsection}{#1}
}

% rubricated inline text
\newcommand{\rubricatum}[1]{\textit{#1}}

% standalone rubric
\newcommand{\rubrica}[1]{\vspace{3mm}\rubricatum{#1}}

\newcommand{\notitia}[1]{\textcolor{red}{#1}}

\newcommand{\scriptura}[1]{\hfill \small\textit{#1}}

\newcommand{\translatioCantus}[1]{\vspace{1mm}%
{\noindent\footnotesize \nlfont{#1}}}

% pruznejsi varianta nasledujiciho - umoznuje nastavit sirku sloupce
% s prekladem
\newcommand{\psalmusEtTranslatioB}[3]{
  \vspace{0.5cm}
  \begin{parcolumns}[colwidths={2=#3}, nofirstindent=true]{2}
    \colchunk{
      \input{#1}
    }

    \colchunk{
      \vspace{-0.5cm}
      {\footnotesize \nlfont
        \input{#2}
      }
    }
  \end{parcolumns}
}

\newcommand{\psalmusEtTranslatio}[2]{
  \psalmusEtTranslatioB{#1}{#2}{8.5cm}
}


\newcommand{\canticumMagnificatEtTranslatio}[1]{
  \psalmusEtTranslatioB{#1}{temporalia/extra-adventum-vespers/magnificat-boh.tex}{12cm}
}
\newcommand{\canticumBenedictusEtTranslatio}[1]{
  \psalmusEtTranslatioB{#1}{temporalia/extra-adventum-laudes/benedictus-boh.tex}{10.5cm}
}

% volne misto nad antifonami, kam si zpevaci dokresli neumy
\newcommand{\hicSuntNeumae}{\vspace{0.5cm}}

% prepinani mista mezi notovymi osnovami: pro neumovane a neneumovane zpevy
\newcommand{\cantusCumNeumis}{
  \setgrefactor{17}
  \global\advance\grespaceabovelines by 5mm%
}
\newcommand{\cantusSineNeumas}{
  \setgrefactor{17}
  \global\advance\grespaceabovelines by -5mm%
}

% znaky k umisteni nad inicialu zpevu
\newcommand{\superInitialam}[1]{\gresetfirstlineaboveinitial{\small {\textbf{#1}}}{\small {\textbf{#1}}}}

% pars officii, i.e. "oratio", ...
\newcommand{\pars}[1]{\textbf{#1}}

\newenvironment{psalmus}{
  \setlength{\parindent}{0pt}
  \setlength{\parskip}{5pt}
}{
  \setlength{\parindent}{10pt}
  \setlength{\parskip}{10pt}
}

%%%% Prejmenovat na latinske:
\newcommand{\nadpisZalmu}[1]{
  \hspace{2cm}\textbf{#1}\vspace{2mm}%
  \nopagebreak%

}

% mode, score, translation
\newcommand{\antiphona}[3]{%
\hicSuntNeumae
\superInitialam{#1}
\includescore{#2}

#3
}
 % Often used macros

\newcommand{\annusEditionis}{2021}

\def\hebinitial#1{%
\leavevmode{\newbox\hebbox\setbox\hebbox\hbox{\hebfont{#1}\hskip 1mm}\kern -\wd\hebbox\hbox{\hebfont{#1}\hskip 1mm}}%
}

%%%% Vicekrat opakovane kousky

\newcommand{\anteOrationem}{
  \rubrica{Ante Orationem, cantatur a Superiore:}

  \pars{Supplicatio Litaniæ.}

  \cuminitiali{}{temporalia/supplicatiolitaniae.gtex}

  \pars{Oratio Dominica.}

  \cuminitiali{}{temporalia/oratiodominica.gtex}
}

\setlength{\columnsep}{30pt} % prostor mezi sloupci

%%%%%%%%%%%%%%%%%%%%%%%%%%%%%%%%%%%%%%%%%%%%%%%%%%%%%%%%%%%%%%%%%%%%%%%%%%%%%%%%%%%%%%%%%%%%%%%%%%%%%%%%%%%%%
\begin{document}

% Here we set the space around the initial.
% Please report to http://home.gna.org/gregorio/gregoriotex/details for more details and options
\grechangedim{afterinitialshift}{2.2mm}{scalable}
\grechangedim{beforeinitialshift}{2.2mm}{scalable}

\grechangedim{interwordspacetext}{0.22 cm plus 0.15 cm minus 0.05 cm}{scalable}%
\grechangedim{annotationraise}{-0.2cm}{scalable}

% Here we set the initial font. Change 38 if you want a bigger initial.
% Emit the initials in red.
\grechangestyle{initial}{\color{red}\fontsize{38}{38}\selectfont}

\pagestyle{empty}

%%%% Titulni stranka
\begin{titulusOfficii}
\ifx\titulus\undefined
\nomenFesti{Sabbato \hebdomada{}}
\else
\titulus
\fi
\end{titulusOfficii}

\vfill

\pars{}

\scriptura{}

\pagebreak

% graphic
\renewcommand{\headrulewidth}{0pt} % no horiz. rule at the header
\fancyhf{}
\pagestyle{fancy}

\cantusSineNeumas

\hora{Ad Matutinum.}

\vspace{2mm}

\cuminitiali{}{temporalia/dominelabiamea.gtex}

\vspace{2mm}

\ifx\invitatorium\undefined
\pars{Invitatorium.} \scriptura{\textbf{H14}}

\vspace{-6mm}

\antiphona{VI}{temporalia/inv-regemventurumsimplex.gtex}
\else
\invitatorium
\fi

\vfill
\pagebreak

\ifx\hymnusmatutinum\undefined
\pars{Hymnus.}

\vspace{-5mm}

\antiphona{II}{temporalia/hym-VerbumSupernum.gtex}
\else
\hymnusmatutinum
\fi

\vfill
\pagebreak

\ifx\matutinum\undefined
\ifx\matua\undefined
\else
% MAT A
\pars{Psalmus 1.} \scriptura{Ps. 104, 3; \textbf{H99}}

\vspace{-6mm}

\antiphona{D}{temporalia/ant-laeteturcor.gtex}

\vspace{-4mm}

\scriptura{Ps. 104, 1-15}

\vspace{-2mm}

\initiumpsalmi{temporalia/ps104i-initium-d-g-auto.gtex}

\vspace{-1.5mm}

\input{temporalia/ps104i-d-g.tex} \Abardot{}

\vfill
\pagebreak

\pars{Psalmus 2.} \scriptura{Ps. 113, 1; \textbf{H94}}

\vspace{-4mm}

\antiphona{VIII a}{temporalia/ant-domusiacob.gtex}

%\vspace{-2mm}

\scriptura{Ps. 104, 16-27}

%\vspace{-2mm}

\initiumpsalmi{temporalia/ps104ii-initium-viii-a-auto.gtex}

\input{temporalia/ps104ii-viii-a.tex} \Abardot{}

\vfill
\pagebreak

\pars{Psalmus 3.} \scriptura{Ps. 104, 43}

\vspace{-4mm}

\antiphona{IV E}{temporalia/ant-eduxitdeus.gtex}

%\vspace{-2mm}

\scriptura{Ps. 104, 28-45}

%\vspace{-2mm}

\initiumpsalmi{temporalia/ps104iii-initium-iv-E-auto.gtex}

\input{temporalia/ps104iii-iv-E.tex}

\vfill

\antiphona{}{temporalia/ant-eduxitdeus.gtex}

\vfill
\pagebreak\fi
\ifx\matub\undefined
\else
% MAT B
\pars{Psalmus 1.} \scriptura{Ps. 105, 4; \textbf{H100}}

\vspace{-4mm}

\antiphona{E}{temporalia/ant-visitanos.gtex}

%\vspace{-2mm}

\scriptura{Ps. 105, 1-15}

%\vspace{-2mm}

\initiumpsalmi{temporalia/ps105i-initium-e.gtex}

\input{temporalia/ps105i-e.tex}

\vfill

\antiphona{}{temporalia/ant-visitanos.gtex}

\vfill
\pagebreak

\pars{Psalmus 2.} \scriptura{Ps. 117, 6; \textbf{H156}}

\vspace{-8mm}

\antiphona{VIII G}{temporalia/ant-dominusmihi.gtex}

\vspace{-3mm}

\scriptura{Ps. 105, 16-31}

\vspace{-2.5mm}

\initiumpsalmi{temporalia/ps105ii-initium-viii-G-auto.gtex}

\vspace{-1.5mm}

\input{temporalia/ps105ii-viii-G.tex} \Abardot{}

\vspace{-5mm}

\vfill
\pagebreak

\pars{Psalmus 3.} \scriptura{Ps. 105, 44}

\vspace{-4mm}

\antiphona{VII a}{temporalia/ant-cumtribularentur.gtex}

%\vspace{-2mm}

\scriptura{Ps. 105, 32-48}

%\vspace{-2mm}

\initiumpsalmi{temporalia/ps105iii-initium-vii-a-auto.gtex}

\input{temporalia/ps105iii-vii-a.tex}

\vfill

\antiphona{}{temporalia/ant-cumtribularentur.gtex}

\vfill
\pagebreak
\fi
\ifx\matuc\undefined
\else
% MAT C
\pars{Psalmus 1.} \scriptura{Ps. 106, 8}

\vspace{-4mm}

\antiphona{IV* e}{temporalia/ant-confiteanturdomino.gtex}

%\vspace{-2mm}

\scriptura{Ps. 106, 1-14}

%\vspace{-2mm}

\initiumpsalmi{temporalia/ps106i-initium-iv_-e-auto.gtex}

\input{temporalia/ps106i-iv_-e.tex} \Abardot{}

\vfill
\pagebreak

\pars{Psalmus 2.} \scriptura{Ps. 24, 17; \textbf{H100}}

\vspace{-4mm}

\antiphona{C}{temporalia/ant-denecessitatibus.gtex}

%\vspace{-2mm}

\scriptura{Ps. 106, 15-30}

%\vspace{-2mm}

\initiumpsalmi{temporalia/ps106ii-initium-c-c2-auto.gtex}

\input{temporalia/ps106ii-c-c2.tex}

\vfill

\antiphona{}{temporalia/ant-denecessitatibus.gtex}

\vfill
\pagebreak

\pars{Psalmus 3.} \scriptura{Ps. 106, 24}

\vspace{-4mm}

\antiphona{III a\textsuperscript{2}}{temporalia/ant-ipsividerunt.gtex}

%\vspace{-2mm}

\scriptura{Ps. 106, 31-43}

%\vspace{-2mm}

\initiumpsalmi{temporalia/ps106iii-initium-iii-a2-auto.gtex}

\input{temporalia/ps106iii-iii-a2.tex} \Abardot{}

\vfill
\pagebreak
\fi
\ifx\matud\undefined
\else
% MAT D
\pars{Psalmus 1.} \scriptura{1 Sam. 2, 10; \textbf{H96}}

\vspace{-4mm}

\antiphona{I g\textsuperscript{2}}{temporalia/ant-dominusjudicabit.gtex}

%\vspace{-2mm}

\scriptura{Ps. 49, 1-6}

%\vspace{-2mm}

\initiumpsalmi{temporalia/ps49i_vi-initium-i-g2-auto.gtex}

\input{temporalia/ps49i_vi-i-g2.tex} \Abardot{}

\vfill
\pagebreak

\pars{Psalmus 2.}

\vspace{-4mm}

\antiphona{VIII G}{temporalia/ant-attenditepopulemeus.gtex}

%\vspace{-2mm}

\scriptura{Ps. 49, 7-15}

%\vspace{-2mm}

\initiumpsalmi{temporalia/ps49vii_xv-initium-viii-G-auto.gtex}

\input{temporalia/ps49vii_xv-viii-G.tex} \Abardot{}

\vfill
\pagebreak

\pars{Psalmus 3.} \scriptura{Ps. 49, 14; \textbf{H94}}

\vspace{-4mm}

\antiphona{E}{temporalia/ant-immoladeo.gtex}

%\vspace{-2mm}

\scriptura{Ps. 49, 16-23}

%\vspace{-2mm}

\initiumpsalmi{temporalia/ps49xvi_xxiii-initium-e-auto.gtex}

\input{temporalia/ps49xvi_xxiii-e.tex} \Abardot{}

\vfill
\pagebreak
\fi
\else
\matutinum
\fi

\ifx\matversus\undefined
\pars{Versus} \scriptura{Mc. 1, 3; Is. 40, 3}

% Versus. %%%
\sineinitiali{temporalia/versus-voxclamantis-simplex.gtex}
\else
\matversus
\fi

\vspace{5mm}

\sineinitiali{temporalia/oratiodominica-mat.gtex}

\vspace{5mm}

\pars{Absolutio.}

\cuminitiali{}{temporalia/absolutio-avinculis.gtex}

\vfill
\pagebreak

\cuminitiali{}{temporalia/benedictio-solemn-ille.gtex}

\vspace{7mm}

\lectioi

\noindent \Vbardot{} Tu autem, Dómine, miserére nobis.
\noindent \Rbardot{} Deo grátias.

\vfill
\pagebreak

\responsoriumi

\vfill
\pagebreak

\cuminitiali{}{temporalia/benedictio-solemn-divinum.gtex}

\vspace{7mm}

\lectioii

\noindent \Vbardot{} Tu autem, Dómine, miserére nobis.
\noindent \Rbardot{} Deo grátias.

\vfill
\pagebreak

\responsoriumii

\vfill
\pagebreak

\cuminitiali{}{temporalia/benedictio-solemn-adsocietatem.gtex}

\vspace{7mm}

\lectioiii

\noindent \Vbardot{} Tu autem, Dómine, miserére nobis.
\noindent \Rbardot{} Deo grátias.

\vfill
\pagebreak

\responsoriumiii

\vfill
\pagebreak

\rubrica{Reliqua omittuntur, nisi Laudes separandæ sint.}

\sineinitiali{temporalia/domineexaudi.gtex}

\vfill

\oratio

\vfill

\noindent \Vbardot{} Dómine, exáudi oratiónem meam.

\noindent \Rbardot{} Et clamor meus ad te véniat.

\noindent \Vbardot{} Benedicámus Dómino, allelúia, allelúia.

\noindent \Rbardot{} Deo grátias, allelúia, allelúia.

\noindent \Vbardot{} Fidélium ánimæ per misericórdiam Dei requiéscant in pace.

\noindent \Rbardot{} Amen.

\vfill
\pagebreak

\hora{Ad Laudes.} %%%%%%%%%%%%%%%%%%%%%%%%%%%%%%%%%%%%%%%%%%%%%%%%%%%%%

\cantusSineNeumas

\vspace{0.5cm}
\ifx\deusinadiutorium\undefined
\grechangedim{interwordspacetext}{0.18 cm plus 0.15 cm minus 0.05 cm}{scalable}%
\cuminitiali{}{temporalia/deusinadiutorium-communis.gtex}
\grechangedim{interwordspacetext}{0.22 cm plus 0.15 cm minus 0.05 cm}{scalable}%
\else
\deusinadiutorium
\fi

\vfill
\pagebreak

\ifx\hymnuslaudes\undefined
\pars{Hymnus} \scriptura{Ambrosius (\olddag{} 397)}

\cuminitiali{I}{temporalia/hym-VoxClara-aromi.gtex}
\vspace{-3mm}
\else
\hymnuslaudes
\fi

\vfill
\pagebreak

\ifx\laudes\undefined
\ifx\lauda\undefined
\else
\pars{Psalmus 1.} \scriptura{Ps. 62, 2.3; \textbf{H142}}

\vspace{-4mm}

\antiphona{VII a}{temporalia/ant-adtedeluce.gtex}

\scriptura{Psalmus 118, 145-152; \hspace{5mm} \hebinitial{ק}}

\initiumpsalmi{temporalia/ps118xix-initium-vii-a-auto.gtex}

\input{temporalia/ps118xix-vii-a.tex} \Abardot{}

\vfill
\pagebreak

\pars{Psalmus 2.} \scriptura{Ex. 15, 1; \textbf{H98}}

\vspace{-4mm}

\antiphona{E}{temporalia/ant-cantemusdomino.gtex}

\scriptura{Canticum Moysis, Ex. 15, 1-19}

\initiumpsalmi{temporalia/moysis-initium-e-auto.gtex}

\input{temporalia/moysis-e.tex}

\antiphona{}{temporalia/ant-cantemusdomino.gtex}

\vfill
\pagebreak

\pars{Psalmus 3.} \scriptura{Ps. 116, 1; \textbf{H94}}

\vspace{-4mm}

\antiphona{E}{temporalia/ant-laudatedominumomnes.gtex}

\scriptura{Psalmus 116.}

\initiumpsalmi{temporalia/ps116-initium-e.gtex}

\input{temporalia/ps116-e.tex} \Abardot{}

\vfill
\pagebreak
\fi
\ifx\laudb\undefined
\else
\pars{Psalmus 1.} \scriptura{Ps. 91, 6}

\vspace{-4.5mm}

\antiphona{E}{temporalia/ant-quammagnificatasunt.gtex}

\vspace{-3mm}

\scriptura{Psalmus 91.}

\vspace{-2mm}

\initiumpsalmi{temporalia/ps91-initium-e.gtex}

\vspace{-1.5mm}

\input{temporalia/ps91-e.tex} \Abardot{}

\vfill
\pagebreak

\pars{Psalmus 2.} \scriptura{Dt. 32, 3}

%\vspace{-4mm}

\antiphona{VI F}{temporalia/ant-datemagnitudinem.gtex}

\vspace{-4mm}

\scriptura{Canticum Moysi, Dt. 32, 1-32}

\initiumpsalmi{temporalia/moysis2i_xii-initium-vi-F-auto.gtex}

\input{temporalia/moysis2i_xii-vi-F.tex}

\vfill

\antiphona{}{temporalia/ant-datemagnitudinem.gtex}

\vfill
\pagebreak

\pars{Psalmus 3.} \scriptura{Ps. 8, 2}

\vspace{-4mm}

\antiphona{I g}{temporalia/ant-quamadmirabileest.gtex}

%\vspace{-2mm}

\scriptura{Ps. 8}

%\vspace{-2mm}

\initiumpsalmi{temporalia/ps8-initium-i-g-auto.gtex}

\input{temporalia/ps8-i-g.tex} \Abardot{}

\vfill
\pagebreak
\fi
\ifx\laudc\undefined
\else
\pars{Psalmus 1.} \scriptura{Ps. 62, 7}

\vspace{-4mm}

\antiphona{E}{temporalia/ant-inmatutinis.gtex}

%\vspace{-2mm}

\scriptura{Psalmus 118, 145-152.}

%\vspace{-2mm}

\initiumpsalmi{temporalia/ps118xix-initium-e-auto.gtex}

%\vspace{-1.5mm}

\input{temporalia/ps118xix-e.tex} \Abardot{}

\vfill
\pagebreak

\pars{Psalmus 2.}

\vspace{-4mm}

\antiphona{V a}{temporalia/ant-mecumsitdomine.gtex}

%\vspace{-2mm}

\scriptura{Canticum Sapientiæ, Sap. 9, 1-6.9-11}

\initiumpsalmi{temporalia/sapientia-initium-v-a-auto.gtex}

\input{temporalia/sapientia-v-a.tex} \Abardot{}

\vfill
\pagebreak

\pars{Psalmus 3.}

\vspace{-4mm}

\antiphona{II* b}{temporalia/ant-veritasdomini.gtex}

%\vspace{-2mm}

\scriptura{Ps. 116}

%\vspace{-2mm}

\initiumpsalmi{temporalia/ps116-initium-ii_-B-auto.gtex}

\input{temporalia/ps116-ii_-B.tex} \Abardot{}

\vfill
\pagebreak
\fi
\ifx\laudd\undefined
\else
\pars{Psalmus 1.} \scriptura{Ps. 91, 2; \textbf{H99}}

\vspace{-4mm}

\antiphona{VIII G}{temporalia/ant-bonumestconfiteri.gtex}

%\vspace{-2mm}

\scriptura{Psalmus 91.}

%\vspace{-2mm}

\initiumpsalmi{temporalia/ps91-initium-viii-g-auto.gtex}

%\vspace{-1.5mm}

\input{temporalia/ps91-viii-g.tex}

\vfill

\antiphona{}{temporalia/ant-bonumestconfiteri.gtex}

\vfill
\pagebreak

\pars{Psalmus 2.}

\vspace{-4mm}

\antiphona{IV* e}{temporalia/ant-dabovobiscor.gtex}

%\vspace{-2mm}

\scriptura{Canticum Habacuc, Hab. 3, 2-19}

\initiumpsalmi{temporalia/habacuc-initium-iv_-e.gtex}

\input{temporalia/habacuc-iv_-e.tex}

\vfill

\antiphona{}{temporalia/ant-dabovobiscor.gtex}

\vfill
\pagebreak

\pars{Psalmus 3.}

\vspace{-4mm}

\antiphona{I f}{temporalia/ant-exoreinfantium.gtex}

%\vspace{-2mm}

\scriptura{Ps. 8}

%\vspace{-2mm}

\initiumpsalmi{temporalia/ps8-initium-i-f-auto.gtex}

\input{temporalia/ps8-i-f.tex} \Abardot{}

\vfill
\pagebreak
\fi
\else
\laudes
\fi

\ifx\lectiobrevis\undefined
\pars{Lectio Brevis.} \scriptura{Is. 11, 1-3}

\noindent Egrediétur virga de stirpe Iesse, et flos de radíce eius ascéndet; et requiéscet super eum spíritus Dómini: spíritus sapiéntiæ et intelléctus, spíritus consílii et fortitúdinis, spíritus sciéntiæ et timóris Dómini; et delíciæ eius in timóre Dómini.
\else
\lectiobrevis
\fi

\vfill

\ifx\responsoriumbreve\undefined
\pars{Responsorium breve.} \scriptura{Is. 60, 2; \textbf{H20}}

\cuminitiali{IV}{temporalia/resp-superte.gtex}
\else
\responsoriumbreve
\fi

\vfill
\pagebreak

\benedictus

\vfill
\pagebreak

\pars{Preces.}

\sineinitiali{}{temporalia/tonusprecum.gtex}

\ifx\preces\undefined
\noindent Deum Patrem, qui antíqua dispositióne pópulum suum salváre státuit, \gredagger{} orémus dicéntes:

\Rbardot{} Custódi plebem tuam, Dómine.

\noindent Deus, qui pópulo tuo germen iustítiæ promisísti, \gredagger{} custódi sanctitátem Ecclésiæ tuæ.

\Rbardot{} Custódi plebem tuam, Dómine.

\noindent Inclína cor hóminum, Deus, in verbum tuum \gredagger{} et confírma fidéles tuos sine queréla in sanctitáte.

\Rbardot{} Custódi plebem tuam, Dómine.

\noindent Consérva nos in dilectióne Spíritus tui, \gredagger{} ut Fílii tui, qui ventúrus est, misericórdiam suscipiámus.

\Rbardot{} Custódi plebem tuam, Dómine.

\noindent Confírma nos, Deus clementíssime, usque in finem, \gredagger{} in diem advéntus Dómini Iesu Christi.

\Rbardot{} Custódi plebem tuam, Dómine.
\else
\preces
\fi

\vfill

\pars{Oratio Dominica.}

\cuminitiali{}{temporalia/oratiodominicaalt.gtex}

\vfill
\pagebreak

\rubrica{vel:}

\pars{Supplicatio Litaniæ.}

\cuminitiali{}{temporalia/supplicatiolitaniae.gtex}

\vfill

\pars{Oratio Dominica.}

\cuminitiali{}{temporalia/oratiodominica.gtex}

\vfill
\pagebreak

% Oratio. %%%
\oratio

\vspace{-1mm}

\vfill

\rubrica{Hebdomadarius dicit Dominus vobiscum, vel, absente sacerdote vel diacono, sic concluditur:}

\vspace{2mm}

\antiphona{C}{temporalia/dominusnosbenedicat.gtex}

\rubrica{Postea cantatur a cantore:}

\vspace{2mm}

\ifx\benedicamuslaudes\undefined
\cuminitiali{IV}{temporalia/benedicamus-feria-advequad.gtex}
\else
\benedicamuslaudes
\fi

\vfill

\vspace{1mm}

\end{document}

