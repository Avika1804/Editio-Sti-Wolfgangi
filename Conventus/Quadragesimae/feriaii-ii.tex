\newcommand{\oratio}{\pars{Oratio.}

\noindent Deus, qui ob animárum medélam castigáre córpora præcepísti, concéde, ut ab ómnibus possímus abstinére peccátis et corda nostra pietátis tuæ váleant exercére mandáta.

\pars{Pro pace in universo mundo.} \scriptura{Sir. 50, 25; 2 Esdr. 4, 20; \textbf{H416}}

\vspace{-4mm}

\antiphona{II D}{temporalia/ant-dapacemdomine.gtex}

\vfill

\noindent Deus, a quo sancta desidéria, recta consília et iusta sunt ópera: da servis tuis illam, quam mundus dare non potest, pacem; ut et corda nostra mandátis tuis dédita, et hóstium subláta formídine, témpora sint tua protectióne tranquílla.

\noindent Per Dóminum nostrum Iesum Christum, Fílium tuum, qui tecum vivit et regnat in unitáte Spíritus Sancti, Deus, per ómnia sǽcula sæculórum.

\noindent \Rbardot{} Amen.}
\newcommand{\invitatorium}{\pars{Invitatorium.}

\vspace{-4mm}

\antiphona{IV*}{temporalia/inv-christumdominum-cumdox.gtex}}
\newcommand{\hymnusmatutinum}{\pars{Hymnus}

\cuminitiali{I}{temporalia/hym-NuncTempus.gtex}}
\newcommand{\matversus}{\noindent \Vbardot{} Pænitémini et crédite Evangélio.

\noindent \Rbardot{} Appropinquávit enim regnum Dei.}
\newcommand{\lectioi}{\vspace{-4mm}

\pars{Lectio I.} \scriptura{Gn. 27, 30-40}

\noindent De libro Génesis.

\noindent Vix Isaac sermónem impléverat, et egrésso Iacob foras, venit Esau, coctósque de venatióne cibos íntulit patri, dicens: "Surge, pater mi, et cómede de venatióne fílii tui, ut benedícat mihi ánima tua."

\noindent Dixítque illi Isaac: "Quis enim es tu?" 

\noindent Qui respóndit: "Ego sum fílius tuus primogénitus Esau." 

\noindent Expávit Isaac stupóre veheménti; et ultra quam credi potest admírans, ait: "Quis ígitur ille est qui dudum captam venatiónem áttulit mihi, et comédi ex ómnibus priúsquam tu veníres; benedixíque ei, et erit benedíctus?" 

\noindent Audítis Esau sermónibus patris, irrúgiit clamóre magno; et consternátus, ait: "Bénedic étiam et mihi, pater mi." 

\noindent Qui ait: "Venit germánus tuus fraudulénter, et accépit benedictiónem tuam." 

\noindent At ille subiúnxit: "Iuste vocátum est nomen eius Iacob: supplantávit enim me en áltera vice; primogénita mea ante tulit, et nunc secúndo surrípuit benedictiónem meam."

\noindent Rursúmque ad patrem: "Numquid non reservásti, ait, et mihi benedictiónem?" 

\noindent Respóndit Isaac: "Dóminum tuum illum constítui, et omnes fratres eius servitúti illíus subiugávi; fruménto et vino stabilívi eum: et tibi post hæc, fili mi, ultra quid fáciam?"

\noindent Cui Esau: "Num unam, inquit, tantum benedictiónem habes, pater? mihi quoque óbsecro ut benedícas." 

\noindent Cumque eiulátu magno fleret, motus Isaac, dixit ad eum: "In pinguédine terræ, et in rore cæli désuper, erit benedíctio tua. Vives in gládio, et fratri tuo sérvies: tempúsque véniet, cum excútias et solvas iugum eius de cervícibus tuis."}
\newcommand{\responsoriumi}{\pars{Responsorium 1.} \scriptura{\Rbardot{} Gn. 27, 3-4 \Vbardot{} ibid.; \textbf{H149}}

\vspace{-5mm}

\responsorium{VII}{temporalia/resp-tollearmatua-CROCHU.gtex}{}}
\newcommand{\lectioii}{\pars{Lectio II.} \scriptura{Gn. 27, 40-46}

\noindent Oderat ergo semper Esau Iacob pro benedictióne qua benedíxerat ei pater; dixítque in corde suo: "Vénient dies luctus patris mei, et occídam Iacob fratrem meum."

\noindent Nuntiáta sunt hæc Rebéccæ; quæ mittens et vocans Iacob fílium suum, dixit ad eum: "Ecce Esau frater tuus minátur ut occídat te. Nunc ergo, fili mi, audi vocem meam, et consúrgens fuge ad Laban fratrem meum in Haran. Habitabísque cum eo dies paucos, donec requiéscat furor fratris tui, et cesset indignátio eius, obliviscatúrque eórum quæ fecísti in eum. Póstea mittam, et addúcam te inde huc: cur utróque orbábor fílio in uno die?" 

\noindent Dixítque Rebécca ad Isaac: "Tædet me vitæ meæ propter fílias Heth: si accéperit Iacob uxórem de stirpe huius terræ, nolo vívere."}
\newcommand{\responsoriumii}{\pars{Responsorium 2.} \scriptura{\Rbardot{} Gn. 27, 27-28 \Vbardot{} ibid., 29; \textbf{H149}}

\vspace{-5mm}

\responsorium{VII}{temporalia/resp-ecceodorfiliimei-CROCHU.gtex}{}}
\newcommand{\lectioiii}{\pars{Lectio III.} \scriptura{Sermo 145: PL 39, 2027-2028}

\noindent Ex Sermónibus sancto Augustíno epíscopo attribútis.

\noindent Hos Quadragésimæ dies, fratres caríssimi, debémus omni veneratióne suscípere, nec longiórem númerum huius témporis fastidíre: quia quanto plures dies sunt ieiúnii, tanto maior est causa remédii; quanto prolíxior abstinéntiæ cursus, tanto redémptio copiósior est salútis; quanto austérior cura vúlnerum, tanto medicína est salúbrior peccatórum. Deus enim qui est nostrárum médicus animárum, cóngruum tempus instítuit, quod et iustis satis sit ad orándum, et peccatóribus suffíciat ad rogándum; illis réquiem postulántibus, his véniam deprecántibus. Illi longum et fastidiósum forte sit tempus, qui nec orat de culpa, nec sperat de vénia. Desperátio enim nec confitéri de scélere, nec indulgéntiam novit speráre de iúdice. Sanctus ígitur et salutáris Quadragésimæ cursus est, quo iudex addúcitur ad misericórdiam, peccátor ad pœniténtiam, iustus ad réquiem.

\noindent His enim diébus sólito ámplius et divínitas miserétur, et delinquéntia deprecátur, et iustítia promerétur. Patent enim ómnia, et cæli ad indulgéndum, et peccátor ad confiténdum, et lingua ad postulándum.

\noindent {\color{gray} Salutáris, inquam, et mýsticus est quadragenárius númerus. Nam primum cum mundi fáciem iníquitas hóminum possidéret, tot diérum currículo Deus effúsis de cælo ímbribus univérsam terram dilúvio superfúdit. Vides ergo iam illo témpore mystérium in figúra dispósitum. Nam sicut tunc quadragínta diébus pluit ad purgándum mundum; ita et nunc quadragínta diébus miserétur ad hóminem purificándum. Quamquam et illíus témporis dilúvium misericórdia dicénda est, quo iníquitas oppréssa est, et iustítia conserváta.

\noindent Pro misericórdia plane vidémus illud fuísse dilúvium, quo, véluti baptísmo quodam, totíus mundi fácies est innováta. Dilúvium illud huius nostri fuit similitúdo baptísmatis. Hoc enim tunc gestum est, quod nunc ágitur: hoc est, ut exuberántibus aquárum fóntibus periclitaréntur vítia, iustítia sola regnáret, mergeréntur in profúndum peccáta, sánctitas vicína cælo portarétur. Tunc enim, sicut dixi, hoc agebátur quod nunc ágitur in Ecclésia Christi.}}
\newcommand{\responsoriumiii}{\pars{Responsorium 3.} \scriptura{\Rbardot{} Gn. 28, 28-29 \Vbardot{} ibid.; \textbf{H150}}

\vspace{-5mm}

\responsorium{VII}{temporalia/resp-dettibideus-CROCHU-cumdox.gtex}{}}
\newcommand{\lectiobrevis}{\pars{Lectio Brevis.} \scriptura{Ex. 19, 4-6}

\noindent Vos ipsi vidístis quómodo portáverim vos super alas aquilárum et addúxerim ad me. Si ergo audiéritis vocem meam et custodiéritis pactum meum, éritis mihi in pecúlium de cunctis pópulis, mea est enim omnis terra. Et vos éritis mihi regnum sacerdótum et gens sancta.}
\newcommand{\responsoriumbreve}{\pars{Responsorium breve.} \scriptura{Ps. 90, 3}

\cuminitiali{IV}{temporalia/resp-ipseliberavitme.gtex}}
\newcommand{\hymnuslaudes}{\pars{Hymnus}

\cuminitiali{D}{temporalia/hym-IamChriste.gtex}}
\newcommand{\preces}{\noindent Benedicámus Deo Patri,\gredagger{} qui nobis largítur ut hoc quadragesimáli die sacrifícium laudis ei offerámus.\grestar{} Eum deprecémur, invocántes:

\Rbardot{} Cæléstibus, Dómine, nos ínstrue disciplínis.

\noindent Omnípotens et miséricors Deus,\gredagger{} concéde nobis spíritum oratiónis et pæniténtiæ,\grestar{} ut caritáte tui et hóminum ardeámus.

\Rbardot{} Cæléstibus, Dómine, nos ínstrue disciplínis.

\noindent Da nos tibi cooperári,\gredagger{} ut ómnia instauréntur in Christo,\grestar{} atque iustítia et pax in terris abúndent.

\Rbardot{} Cæléstibus, Dómine, nos ínstrue disciplínis.

\noindent Intimam totíus creatúræ natúram et prétium áperi nobis,\grestar{} ut, te celebrántes, eam in cármine laudis nobis consociémus.

\Rbardot{} Cæléstibus, Dómine, nos ínstrue disciplínis.

\noindent Ignósce nobis, qui Christi tui præséntiam in paupéribus, míseris et moléstis ignorávimus,~\grestar{} nec vériti sumus Fílium tuum in his frátribus nostris.

\Rbardot{} Cæléstibus, Dómine, nos ínstrue disciplínis.}
\newcommand{\benedictus}{\pars{Canticum Zachariæ.} \scriptura{Lc. 6, 36; \textbf{H428}}

\vspace{-4mm}

{
\grechangedim{interwordspacetext}{0.18 cm plus 0.15 cm minus 0.05 cm}{scalable}%
\antiphona{I f}{temporalia/ant-estoteergomisericordes.gtex}
\grechangedim{interwordspacetext}{0.22 cm plus 0.15 cm minus 0.05 cm}{scalable}%
}

%\vspace{-2mm}

\scriptura{Lc. 1, 68-79}

%\vspace{-2mm}

\initiumpsalmi{temporalia/benedictus-initium-i-f-auto.gtex}

%\vspace{-1mm}

\input{temporalia/benedictus-i-f.tex} \Abardot{}}
\newcommand{\magnificat}{\pars{Canticum B. Mariæ V.} \scriptura{Io. 8, 29; \textbf{H152}}

\vspace{-4mm}

{
\grechangedim{interwordspacetext}{0.18 cm plus 0.15 cm minus 0.05 cm}{scalable}%
\antiphona{I g}{temporalia/ant-quimemisitmecumest.gtex}
\grechangedim{interwordspacetext}{0.22 cm plus 0.15 cm minus 0.05 cm}{scalable}%
}

%\vspace{-2mm}

\scriptura{Lc. 1, 46-55}

%\vspace{-2mm}

\cantusSineNeumas
\initiumpsalmi{temporalia/magnificat-initium-i-g.gtex}

%\vspace{-1.5mm}

\input{temporalia/magnificat-i-g.tex} \Abardot{}}
\newcommand{\oratiovesperas}{\pars{Oratio.}

\noindent Adésto supplicatiónibus nostris omnípotens Deus:~\grestar{} et quibus fidúciam sperándæ pietátis indúlges, consuétæ misericórdiæ tríbue benígnus efféctum.

\noindent Per Dóminum nostrum Iesum Christum, Fílium tuum, qui tecum vivit et regnat in unitáte Spíritus Sancti, Deus, per ómnia sǽcula sæculórum.

\noindent \Rbardot{} Amen.}
\newcommand{\hebdomada}{infra Hebdom. II Adventus.}
\newcommand{\oratioLaudes}{\cuminitiali{}{temporalia/oratio2vo.gtex}}
\newcommand{\responsoriumbreve}{\pars{Responsorium breve.}

\cuminitiali{IV}{temporalia/resp-christe.gtex}}

% LuaLaTeX

\documentclass[a4paper, twoside, 12pt]{article}
\usepackage[latin]{babel}
%\usepackage[landscape, left=3cm, right=1.5cm, top=2cm, bottom=1cm]{geometry} % okraje stranky
%\usepackage[landscape, a4paper, mag=1166, truedimen, left=2cm, right=1.5cm, top=1.6cm, bottom=0.95cm]{geometry} % okraje stranky
\usepackage[landscape, a4paper, mag=1400, truedimen, left=0.5cm, right=0.5cm, top=0.5cm, bottom=0.5cm]{geometry} % okraje stranky

\usepackage{fontspec}
\setmainfont[FeatureFile={junicode.fea}, Ligatures={Common, TeX}, RawFeature=+fixi]{Junicode}
%\setmainfont{Junicode}

% shortcut for Junicode without ligatures (for the Czech texts)
\newfontfamily\nlfont[FeatureFile={junicode.fea}, Ligatures={Common, TeX}, RawFeature=+fixi]{Junicode}

\usepackage{multicol}
\usepackage{color}
\usepackage{lettrine}
\usepackage{fancyhdr}

% usual packages loading:
\usepackage{luatextra}
\usepackage{graphicx} % support the \includegraphics command and options
\usepackage{gregoriotex} % for gregorio score inclusion
\usepackage{gregoriosyms}
\usepackage{wrapfig} % figures wrapped by the text
\usepackage{parcolumns}
\usepackage[contents={},opacity=1,scale=1,color=black]{background}
\usepackage{tikzpagenodes}
\usepackage{calc}
\usepackage{longtable}
\usetikzlibrary{calc}

\setlength{\headheight}{14.5pt}

% Commands used to produce a typical "Conventus" booklet

\newenvironment{titulusOfficii}{\begin{center}}{\end{center}}
\newcommand{\dies}[1]{#1

}
\newcommand{\nomenFesti}[1]{\textbf{\Large #1}

}
\newcommand{\celebratio}[1]{#1

}

\newcommand{\hora}[1]{%
\vspace{0.5cm}{\large \textbf{#1}}

\fancyhead[LE]{\thepage\ / #1}
\fancyhead[RO]{#1 / \thepage}
\addcontentsline{toc}{subsection}{#1}
}

% larger unit than a hora
\newcommand{\divisio}[1]{%
\begin{center}
{\Large \textsc{#1}}
\end{center}
\fancyhead[CO,CE]{#1}
\addcontentsline{toc}{section}{#1}
}

% a part of a hora, larger than pars
\newcommand{\subhora}[1]{
\begin{center}
{\large \textit{#1}}
\end{center}
%\fancyhead[CO,CE]{#1}
\addcontentsline{toc}{subsubsection}{#1}
}

% rubricated inline text
\newcommand{\rubricatum}[1]{\textit{#1}}

% standalone rubric
\newcommand{\rubrica}[1]{\vspace{3mm}\rubricatum{#1}}

\newcommand{\notitia}[1]{\textcolor{red}{#1}}

\newcommand{\scriptura}[1]{\hfill \small\textit{#1}}

\newcommand{\translatioCantus}[1]{\vspace{1mm}%
{\noindent\footnotesize \nlfont{#1}}}

% pruznejsi varianta nasledujiciho - umoznuje nastavit sirku sloupce
% s prekladem
\newcommand{\psalmusEtTranslatioB}[3]{
  \vspace{0.5cm}
  \begin{parcolumns}[colwidths={2=#3}, nofirstindent=true]{2}
    \colchunk{
      \input{#1}
    }

    \colchunk{
      \vspace{-0.5cm}
      {\footnotesize \nlfont
        \input{#2}
      }
    }
  \end{parcolumns}
}

\newcommand{\psalmusEtTranslatio}[2]{
  \psalmusEtTranslatioB{#1}{#2}{8.5cm}
}


\newcommand{\canticumMagnificatEtTranslatio}[1]{
  \psalmusEtTranslatioB{#1}{temporalia/extra-adventum-vespers/magnificat-boh.tex}{12cm}
}
\newcommand{\canticumBenedictusEtTranslatio}[1]{
  \psalmusEtTranslatioB{#1}{temporalia/extra-adventum-laudes/benedictus-boh.tex}{10.5cm}
}

% volne misto nad antifonami, kam si zpevaci dokresli neumy
\newcommand{\hicSuntNeumae}{\vspace{0.5cm}}

% prepinani mista mezi notovymi osnovami: pro neumovane a neneumovane zpevy
\newcommand{\cantusCumNeumis}{
  \setgrefactor{17}
  \global\advance\grespaceabovelines by 5mm%
}
\newcommand{\cantusSineNeumas}{
  \setgrefactor{17}
  \global\advance\grespaceabovelines by -5mm%
}

% znaky k umisteni nad inicialu zpevu
\newcommand{\superInitialam}[1]{\gresetfirstlineaboveinitial{\small {\textbf{#1}}}{\small {\textbf{#1}}}}

% pars officii, i.e. "oratio", ...
\newcommand{\pars}[1]{\textbf{#1}}

\newenvironment{psalmus}{
  \setlength{\parindent}{0pt}
  \setlength{\parskip}{5pt}
}{
  \setlength{\parindent}{10pt}
  \setlength{\parskip}{10pt}
}

%%%% Prejmenovat na latinske:
\newcommand{\nadpisZalmu}[1]{
  \hspace{2cm}\textbf{#1}\vspace{2mm}%
  \nopagebreak%

}

% mode, score, translation
\newcommand{\antiphona}[3]{%
\hicSuntNeumae
\superInitialam{#1}
\includescore{#2}

#3
}
 % Often used macros

\newcommand{\annusEditionis}{2021}

%%%% Vicekrat opakovane kousky

\newcommand{\anteOrationem}{
  \rubrica{Ante Orationem, cantatur a Superiore:}

  \pars{Supplicatio Litaniæ.}

  \cuminitiali{}{temporalia/supplicatiolitaniae.gtex}

  \pars{Oratio Dominica.}

  \cuminitiali{}{temporalia/oratiodominica.gtex}

  \rubrica{Deinde dicitur ab Hebdomadario:}

  \cuminitiali{}{temporalia/dominusvobiscum-solemnis.gtex}

  \rubrica{In choro monialium loco Dominus vobiscum dicitur:}

  \sineinitiali{temporalia/domineexaudi.gtex}
}

\setlength{\columnsep}{30pt} % prostor mezi sloupci

%%%%%%%%%%%%%%%%%%%%%%%%%%%%%%%%%%%%%%%%%%%%%%%%%%%%%%%%%%%%%%%%%%%%%%%%%%%%%%%%%%%%%%%%%%%%%%%%%%%%%%%%%%%%%
\begin{document}

% Here we set the space around the initial.
% Please report to http://home.gna.org/gregorio/gregoriotex/details for more details and options
\grechangedim{afterinitialshift}{2.2mm}{scalable}
\grechangedim{beforeinitialshift}{2.2mm}{scalable}
\grechangedim{interwordspacetext}{0.22 cm plus 0.15 cm minus 0.05 cm}{scalable}%
\grechangedim{annotationraise}{-0.2cm}{scalable}

% Here we set the initial font. Change 38 if you want a bigger initial.
% Emit the initials in red.
\grechangestyle{initial}{\color{red}\fontsize{38}{38}\selectfont}

\pagestyle{empty}

%%%% Titulni stranka
\begin{titulusOfficii}
\ifx\titulus\undefined
\nomenFesti{Feria II \hebdomada{}}
\else
\titulus
\fi
\end{titulusOfficii}

\vfill

\begin{center}
%Ad usum et secundum consuetudines chori \guillemotright{}Conventus Choralis\guillemotleft.

%Editio Sancti Wolfgangi \annusEditionis
\end{center}

\scriptura{}

\pars{}

\pagebreak

\renewcommand{\headrulewidth}{0pt} % no horiz. rule at the header
\fancyhf{}
\pagestyle{fancy}

\cantusSineNeumas

\hora{Ad Matutinum.} %%%%%%%%%%%%%%%%%%%%%%%%%%%%%%%%%%%%%%%%%%%%%%%%%%%%%

\vspace{2mm}

\cuminitiali{}{temporalia/dominelabiamea.gtex}

\vfill
%\pagebreak

\vspace{2mm}

\ifx\invitatorium\undefined
\pars{Invitatorium.} \scriptura{Lc. 24, 34; Psalmus 94; \textbf{H232}}

\vspace{-6mm}

\antiphona{VI}{temporalia/inv-surrexitdominusvere.gtex}
\else
\invitatorium
\fi

\vfill
\pagebreak

\ifx\hymnusmatutinum\undefined
\pars{Hymnus}

\cuminitiali{VIII}{temporalia/hym-LaetareCaelum.gtex}
\else
\hymnusmatutinum
\fi

\vspace{-3mm}

\vfill
\pagebreak

\ifx\matua\undefined
\else
% MAT A
\pars{Psalmus 1.} \scriptura{Ps. 6, 3}

\vspace{-4mm}

\antiphona{IV E}{temporalia/ant-misereremihi.gtex}

%\vspace{-2mm}

\scriptura{Ps. 6}

%\vspace{-2mm}

\initiumpsalmi{temporalia/ps6-initium-iv-E-auto.gtex}

\input{temporalia/ps6-iv-E.tex} \Abardot{}

\vfill
\pagebreak

\pars{Psalmus 2.} \scriptura{Ps. 110, 1; \textbf{H91}}

\vspace{-4mm}

\antiphona{VIII G}{temporalia/ant-confitebortibi.gtex}

%\vspace{-2mm}

\scriptura{Ps. 9, 2-11}

%\vspace{-2mm}

\initiumpsalmi{temporalia/ps9ii_xi-initium-viii-G-auto.gtex}

\input{temporalia/ps9ii_xi-viii-G.tex} \Abardot{}

\vfill
\pagebreak

\pars{Psalmus 3.} \scriptura{Ps. 9, 20}

\vspace{-4mm}

\antiphona{I g\textsuperscript{3}}{temporalia/ant-exsurgedominenon.gtex}

%\vspace{-2mm}

\scriptura{Ps. 9, 12-21}

%\vspace{-2mm}

\initiumpsalmi{temporalia/ps9xii_xxi-initium-i-g3-auto.gtex}

\input{temporalia/ps9xii_xxi-i-g3.tex} \Abardot{}

\vfill
\pagebreak
\fi
\ifx\matub\undefined
\else
% MAT B
\pars{Psalmus 1.} \scriptura{Ps. 30, 2; \textbf{H90}}

\vspace{-4mm}

\antiphona{VIII G}{temporalia/ant-intuaiustitia.gtex}

%\vspace{-2mm}

\scriptura{Ps. 30, 2-9}

%\vspace{-2mm}

\initiumpsalmi{temporalia/ps30i-initium-viii-G-auto.gtex}

\vspace{-1.5mm}

\input{temporalia/ps30i-viii-G.tex} \Abardot{}

\vfill
\pagebreak

\pars{Psalmus 2.} \scriptura{Ps. 66, 2}

\vspace{-4mm}

\antiphona{E}{temporalia/ant-illuminadomine.gtex}

%\vspace{-2mm}

\scriptura{Ps. 30, 10-17}

%\vspace{-2mm}

\initiumpsalmi{temporalia/ps30ii-initium-e-a-auto.gtex}

\input{temporalia/ps30ii-e-a.tex} \Abardot{}

\vfill
\pagebreak

\pars{Psalmus 3.} \scriptura{Ps. 30, 24}

\vspace{-4mm}

\antiphona{II D}{temporalia/ant-diligitedominum.gtex}

%\vspace{-5mm}

\scriptura{Ps. 30, 20-25}

%\vspace{-2mm}

\initiumpsalmi{temporalia/ps30iii-initium-ii-D-auto.gtex}

\input{temporalia/ps30iii-ii-D.tex} \Abardot{}

\vfill
\pagebreak
\fi
\ifx\matuc\undefined
\else
% MAT C
\pars{Psalmus 1.}

\vspace{-4mm}

\antiphona{VIII G\textsuperscript{3}}{temporalia/ant-alleluia-bv21-n4.gtex}

%\vspace{-2mm}

\scriptura{Ps. 49, 1-6}

%\vspace{-2mm}

\initiumpsalmi{temporalia/ps49i_vi-initium-viii-G3.gtex}

\input{temporalia/ps49i_vi-viii-G2.tex}

\vfill
\pagebreak

\pars{Psalmus 2.}

\scriptura{Ps. 49, 7-15}

%\vspace{-2mm}

\initiumpsalmi{temporalia/ps49vii_xv-initium-viii-G3.gtex}

\input{temporalia/ps49vii_xv-viii-G2.tex}

\vfill
\pagebreak

\pars{Psalmus 3.}

\scriptura{Ps. 49, 16-23}

%\vspace{-2mm}

\initiumpsalmi{temporalia/ps49xvi_xxiii-initium-viii-G3.gtex}

\input{temporalia/ps49xvi_xxiii-viii-G2.tex} \Abardot{}

\vfill
\pagebreak
\fi
\ifx\matud\undefined
\else
% MAT D
\pars{Psalmus 1.} \scriptura{Ps. 72, 1}

\vspace{-4mm}

\antiphona{VIII c}{temporalia/ant-quambonusdeus.gtex}

%\vspace{-2mm}

\scriptura{Ps. 72, 1-12}

%\vspace{-2mm}

\initiumpsalmi{temporalia/ps72i-initium-viii-c-auto.gtex}

%\vspace{-1.5mm}

\input{temporalia/ps72i-viii-c.tex} \Abardot{}

\vfill
\pagebreak

\pars{Psalmus 2.} \scriptura{Ps. 15, 7; \textbf{H99}}

\vspace{-4mm}

\antiphona{II D}{temporalia/ant-benedicamdomino.gtex}

%\vspace{-2mm}

\scriptura{Ps. 72, 13-20}

%\vspace{-2mm}

\initiumpsalmi{temporalia/ps72ii-initium-ii-D-auto.gtex}

\input{temporalia/ps72ii-ii-D.tex} \Abardot{}

\vfill
\pagebreak

\pars{Psalmus 3.} \scriptura{Ps. 72, 28}

\vspace{-4mm}

\antiphona{III b}{temporalia/ant-adhaereredeobonumest.gtex}

%\vspace{-2mm}

\scriptura{Ps. 72, 21-28}

%\vspace{-2mm}

\initiumpsalmi{temporalia/ps72iii-initium-iii-b.gtex}

\input{temporalia/ps72iii-iii-b.tex} \Abardot{}

\vfill
\pagebreak
\fi

\pars{Versus.}

\ifx\matversus\undefined
\ifx\matua\undefined
\else
\noindent \Vbardot{} Da mihi intelléctum et servábo legem tuam. 

\noindent \Rbardot{} Et custódiam illam in toto corde meo.
\fi
\ifx\matub\undefined
\else
\noindent \Vbardot{} Dírige me, Dómine, in veritáte tua, et doce me.

\noindent \Rbardot{} Quia tu es Deus salútis meæ.
\fi
\ifx\matuc\undefined
\else
\noindent \Vbardot{} Cor meum et caro mea, allelúia.

\noindent \Rbardot{} Exsultavérunt in Deum vivum, allelúia.
\fi
\ifx\matud\undefined
\else
\noindent \Vbardot{} Quam dúlcia fáucibus meis elóquia tua, Dómine.

\noindent \Rbardot{} Super mel ori meo.
\fi
\else
\matversus
\fi

\vspace{5mm}

\sineinitiali{temporalia/oratiodominica-mat.gtex}

\vspace{5mm}

\pars{Absolutio.}

\cuminitiali{}{temporalia/absolutio-exaudi.gtex}

\vfill
\pagebreak

\cuminitiali{}{temporalia/benedictio-solemn-benedictione.gtex}

\vspace{7mm}

\lectioi

\noindent \Vbardot{} Tu autem, Dómine, miserére nobis.
\noindent \Rbardot{} Deo grátias.

\vfill
\pagebreak

\responsoriumi

\vfill
\pagebreak

\cuminitiali{}{temporalia/benedictio-solemn-unigenitus.gtex}

\vspace{7mm}

\lectioii

\noindent \Vbardot{} Tu autem, Dómine, miserére nobis.
\noindent \Rbardot{} Deo grátias.

\vfill
\pagebreak

\responsoriumii

\vfill
\pagebreak

\cuminitiali{}{temporalia/benedictio-solemn-spiritus.gtex}

\vspace{7mm}

\lectioiii

\noindent \Vbardot{} Tu autem, Dómine, miserére nobis.
\noindent \Rbardot{} Deo grátias.

\vfill
\pagebreak

\responsoriumiii

\vfill
\pagebreak

\rubrica{Reliqua omittuntur, nisi Laudes separandæ sint.}

\sineinitiali{temporalia/domineexaudi.gtex}

\vfill

\oratio

\vfill

\noindent \Vbardot{} Dómine, exáudi oratiónem meam.
\Rbardot{} Et clamor meus ad te véniat.

\vfill

\noindent \Vbardot{} Benedicámus Dómino.
\noindent \Rbardot{} Deo grátias.

\vfill

\noindent \Vbardot{} Fidélium ánimæ per misericórdiam Dei requiéscant in pace.
\Rbardot{} Amen.

\vfill
\pagebreak

\hora{Ad Laudes.} %%%%%%%%%%%%%%%%%%%%%%%%%%%%%%%%%%%%%%%%%%%%%%%%%%%%%

\cantusSineNeumas

\vspace{0.5cm}
\grechangedim{interwordspacetext}{0.18 cm plus 0.15 cm minus 0.05 cm}{scalable}%
\cuminitiali{}{temporalia/deusinadiutorium-communis.gtex}
\grechangedim{interwordspacetext}{0.22 cm plus 0.15 cm minus 0.05 cm}{scalable}%

\vfill
\pagebreak

\ifx\hymnuslaudes\undefined
\ifx\laudac\undefined
\else
\pars{Hymnus}

\cuminitiali{I}{temporalia/hym-ChorusNovae-praglia.gtex}
\fi
\ifx\laudbd\undefined
\else
\pars{Hymnus}

\cuminitiali{I}{temporalia/hym-ChorusNovae.gtex}
\vspace{-3mm}
\fi
\else
\hymnuslaudes
\fi

\vfill
\pagebreak

\ifx\lauda\undefined
\else
\pars{Psalmus 1.} \scriptura{Ps. 5, 2; \textbf{H93}}

\vspace{-6mm}

\antiphona{VIII a}{temporalia/ant-intellegeclamorem.gtex}

\vspace{-4mm}

\scriptura{Psalmus 5.}

\vspace{-2mm}

\initiumpsalmi{temporalia/ps5-initium-viii-A-auto.gtex}

\vspace{-1.5mm}

\input{temporalia/ps5-viii-A.tex} \Abardot{}

\vfill
\pagebreak

\pars{Psalmus 2.} \scriptura{1 Par. 29, 13}

\vspace{-4mm}

\antiphona{I f}{temporalia/ant-laudamusnomentuum.gtex}

%\vspace{-2mm}

\scriptura{Canticum David, 1 Par. 29, 10-13}

%\vspace{-2mm}

\initiumpsalmi{temporalia/david-initium-i-f-auto.gtex}

\input{temporalia/david-i-f.tex} \Abardot{}

\vfill
\pagebreak

\pars{Psalmus 3.} \scriptura{Ps. 28, 1.2; \textbf{H72}}

\vspace{-4mm}

\antiphona{VII a}{temporalia/ant-affertedomino.gtex}

\scriptura{Psalmus. 28}

\initiumpsalmi{temporalia/ps28-initium-vii-a-auto.gtex}

\input{temporalia/ps28-vii-a.tex} \Abardot{}

\vfill
\pagebreak
\fi
\ifx\laudb\undefined
\else
\pars{Psalmus 1.} \scriptura{Ps. 41, 3; \textbf{H391}}

\vspace{-4mm}

\antiphona{II D}{temporalia/ant-sitivitanima.gtex}

%\vspace{-2mm}

\scriptura{Psalmus 41}

%\vspace{-2mm}

\initiumpsalmi{temporalia/ps41-initium-ii-D-auto.gtex}

%\vspace{-1.5mm}

\input{temporalia/ps41-ii-D.tex}

\vfill

\antiphona{}{temporalia/ant-sitivitanima.gtex}

\vfill
\pagebreak

\pars{Psalmus 2.}

\vspace{-4mm}

\antiphona{III a}{temporalia/ant-ostendenobisdomine.gtex}

%\vspace{-2mm}

\scriptura{Canticum Ecclesiastici, Sir. 36, 1-7.13-16}

%\vspace{-3mm}

\initiumpsalmi{temporalia/ecclesiastici-initium-iii-a-auto.gtex}

\input{temporalia/ecclesiastici-iii-a.tex} \Abardot{}

\vfill
\pagebreak

\pars{Psalmus 3.}

\vspace{-4mm}

\antiphona{II D}{temporalia/ant-operamanuumeius.gtex}

\scriptura{Psalmus 18, 1-7}

\initiumpsalmi{temporalia/ps18i-initium-ii-D-auto.gtex}

\input{temporalia/ps18i-ii-D.tex} \Abardot{}

\vfill
\pagebreak
\fi
\ifx\laudc\undefined
\else
\pars{Psalmus 1.}

\vspace{-4mm}

\antiphona{VIII G}{temporalia/ant-alleluia-turco12.gtex}

%\vspace{-2mm}

\scriptura{Psalmus 83}

%\vspace{-2mm}

\initiumpsalmi{temporalia/ps83-initium-viii-G-auto.gtex}

%\vspace{-1.5mm}

\input{temporalia/ps83-viii-G.tex} \Abardot{}

\vfill
\pagebreak

\pars{Psalmus 2.} \scriptura{Mi. 4, 2}

\vspace{-4mm}

\antiphona{VIII G\textsuperscript{2}}{temporalia/ant-veniteascendamus-tp.gtex}

%\vspace{-2mm}

\scriptura{Canticum Isaiæ, Is. 2, 2-5}

%\vspace{-2mm}

\initiumpsalmi{temporalia/isaiae11-initium-viii-g5-auto.gtex}

\input{temporalia/isaiae11-viii-g5.tex} \Abardot{}

\vfill
\pagebreak

\pars{Psalmus 3.}

\vspace{-4mm}

\antiphona{II D}{temporalia/ant-alleluia-turco8.gtex}

\scriptura{Psalmus 95}

\initiumpsalmi{temporalia/ps95-initium-ii-D-auto.gtex}

\input{temporalia/ps95-ii-D.tex} \Abardot{}

\vfill
\pagebreak
\fi
\ifx\laudd\undefined
\else
\pars{Psalmus 1.} \scriptura{Ps. 89, 1; \textbf{H98}}

\vspace{-4mm}

\antiphona{VI F}{temporalia/ant-dominerefugium.gtex}

%\vspace{-2mm}

\scriptura{Psalmus 89}

%\vspace{-2mm}

\initiumpsalmi{temporalia/ps89-initium-vi-F-auto.gtex}

%\vspace{-1.5mm}

\input{temporalia/ps89-vi-F.tex}

\vfill

\antiphona{}{temporalia/ant-dominerefugium.gtex}

\vfill
\pagebreak

\pars{Psalmus 2.} \scriptura{Is. 42, 10; \textbf{H98}}

\vspace{-4mm}

\antiphona{VI F}{temporalia/ant-cantatedominocanticum.gtex}

%\vspace{-2mm}

\scriptura{Canticum Isaiæ, Is. 42, 10-16}

%\vspace{-3mm}

\initiumpsalmi{temporalia/isaiae10-initium-vi-F-auto.gtex}

\input{temporalia/isaiae10-vi-F.tex} \Abardot{}

\vfill
\pagebreak

\pars{Psalmus 3.} \scriptura{Ps. 134, 1-2}

\vspace{-4mm}

\antiphona{I a}{temporalia/ant-laudatenomendomini.gtex}

\scriptura{Psalmus 134, 1-12}

\initiumpsalmi{temporalia/ps134i-initium-i-a-auto.gtex}

\input{temporalia/ps134i-i-a.tex} \Abardot{}

\vfill
\pagebreak
\fi

\ifx\lectiobrevis\undefined
\ifx\lauda\undefined
\else
\pars{Lectio Brevis.} \scriptura{2 Th. 3, 10-13}

\noindent Si quis non vult operári, nec mandúcet. Audímus enim inter vos quosdam ambuláre inordináte, nihil operántes sed curióse agéntes; his autem, qui eiúsmodi sunt, præcípimus et obsecrámus in Dómino Iesu Christo, ut cum quiéte operántes suum panem mandúcent. Vos autem, fratres, nolíte defícere benefaciéntes.
\fi
\ifx\laudb\undefined
\else
\pars{Lectio Brevis.} \scriptura{Ier. 15, 16}

\noindent Invénti sunt sermónes tui, et comédi eos, et factum est mihi verbum tuum in gáudium et in lætítiam cordis mei, quóniam invocátum est nomen tuum super me, Dómine Deus exercítuum.
\fi
\ifx\laudc\undefined
\else
\pars{Lectio Brevis.} \scriptura{Rom. 10, 8-10}

\noindent Prope te est verbum, in ore tuo et in corde tuo; hoc est verbum fídei, quod prædicámus. Quia si confiteáris in ore tuo: «Dóminum Iesum!», et in corde tuo credíderis quod Deus illum excitávit ex mórtuis, salvus eris. Corde enim créditur ad iustítiam, ore autem conféssio fit in salútem.
\fi
\ifx\laudd\undefined
\else
\pars{Lectio Brevis.} \scriptura{Idt. 8, 25-27}

\noindent Grátias agámus Dómino Deo nostro, qui temptat nos sicut et patres nostros. Mémores estóte quanta fécerit cum Abraham et Isaac, et quanta facta sint Iacob. Quia non sicut illos combússit in inquisitiónem cordis illórum et in nos non ultus est, sed in monitiónem flagéllat Dóminus appropinquántes sibi.
\fi
\else
\lectiobrevis
\fi

\vfill

\ifx\responsoriumbreve\undefined
\pars{Responsorium breve.} \scriptura{Cf. Mt. 28, 6; Cf. Gal. 3, 13}

\cuminitiali{VI}{temporalia/resp-surrexitdominusdesepulcro.gtex}
\else
\responsoriumbreve
\fi

\vfill
\pagebreak

\benedictus

\vspace{-1cm}

\vfill
\pagebreak

\pars{Preces.}

\sineinitiali{}{temporalia/tonusprecum.gtex}

\ifx\preces\undefined
\ifx\lauda\undefined
\else
\noindent Christum magnificémus, plenum grátia et Spíritu Sancto, \gredagger{} et fidénter eum implorémus:

\Rbardot{} Spíritum tuum da nobis, Dómine.

\noindent Concéde nobis diem istum iucúndum, pacíficum et sine mácula, \gredagger{} ut, ad vésperam perdúcti, cum gáudio et mundo corde te collaudáre valeámus.

\Rbardot{} Spíritum tuum da nobis, Dómine.

\noindent Sit hódie splendor tuus super nos, \gredagger{} et opus mánuum nostrárum dírige.

\Rbardot{} Spíritum tuum da nobis, Dómine.

\noindent Osténde fáciem tuam super nos ad bonum in pace, \gredagger{} ut hódie manu tua válida contegámur.

\Rbardot{} Spíritum tuum da nobis, Dómine.

\noindent Réspice propítius omnes, qui oratiónibus nostris confídunt, \gredagger{} eos adímple bonis ánimæ et córporis univérsis.

\Rbardot{} Spíritum tuum da nobis, Dómine.
\fi
\ifx\laudb\undefined
\else
\noindent Salvátor noster fecit nos regnum et sacerdótium, ut hóstias Deo acceptábiles offerámus. \gredagger{} Grati ígitur eum invocémus:

\Rbardot{} Serva nos in tuo ministério, Dómine.

\noindent Christe, sacérdos ætérne, qui sanctum pópulo tuo sacerdótium concessísti, \gredagger{} concéde, ut spiritáles hóstias Deo acceptábiles iúgiter offerámus.

\Rbardot{} Serva nos in tuo ministério, Dómine.

\noindent Spíritus tui fructus nobis largíre propítius, \gredagger{} patiéntiam, benignitátem et mansuetúdinem.

\Rbardot{} Serva nos in tuo ministério, Dómine.

\noindent Da nobis te amáre, ut te, qui es cáritas, possideámus, \gredagger{} et bene ágere, ut per vitam étiam nostram te laudémus.

\Rbardot{} Serva nos in tuo ministério, Dómine.

\noindent Quæ frátribus nostris sunt utília, nos quǽrere concéde, \gredagger{} ut salútem facílius consequántur.

\Rbardot{} Serva nos in tuo ministério, Dómine.
\fi
\ifx\laudc\undefined
\else
\noindent Iesum, quem Pater glorificávit et herédem ómnium géntium constítuit, \gredagger{} exaltémus, orántes:

\Rbardot{} Per victóriam tuam salva nos, Dómine.

\noindent Christe, qui victória tua portas contrivísti infernáles, peccátum delens et mortem, \gredagger{} fac nos hódie peccáti victóres.

\Rbardot{} Per victóriam tuam salva nos, Dómine.

\noindent Tu, qui mortem evacuásti, vitam nobis impértiens novam, \gredagger{} da ut hódie in hac vitæ novitáte ambulémus.

\Rbardot{} Per victóriam tuam salva nos, Dómine.

\noindent Qui vitam mórtuis tribuísti, totum genus humánum de morte ad vitam redúcens, \gredagger{} ómnibus, qui nobis occúrrent, ætérnam vitam concéde.

\Rbardot{} Per victóriam tuam salva nos, Dómine.

\noindent Qui, sepúlcri tui custódes confúndens, discípulos tuos lætificásti, \gredagger{} plenam tibi serviéntibus largíre lætítiam.

\Rbardot{} Per victóriam tuam salva nos, Dómine.
\fi
\ifx\laudd\undefined
\else
\noindent Christum, qui exáudit et salvos facit sperántes in se, \gredagger{} precémur acclamántes:

\Rbardot{} Te laudámus, in te sperámus, Dómine.

\noindent Grátias ágimus tibi, qui dives es in misericórdia, \gredagger{} propter nímiam caritátem, qua dilexísti nos.

\Rbardot{} Te laudámus, in te sperámus, Dómine.

\noindent Qui omni témpore in mundo cum Patre operáris, \gredagger{} nova fac ómnia per Spíritus Sancti virtútem.

\Rbardot{} Te laudámus, in te sperámus, Dómine.

\noindent Aperi óculos nostros et fratrum nostrórum, \gredagger{} ut videámus hódie mirabília tua.

\Rbardot{} Te laudámus, in te sperámus, Dómine.

\noindent Qui nos hódie ad tuum servítium vocas, \gredagger{} nos erga fratres multifórmis grátiæ tuæ fac minístros.

\Rbardot{} Te laudámus, in te sperámus, Dómine.
\fi
\else
\preces
\fi

\vfill

\pars{Oratio Dominica.}

\cuminitiali{}{temporalia/oratiodominicaalt.gtex}

\vfill
\pagebreak

\rubrica{vel:}

\pars{Supplicatio Litaniæ.}

\cuminitiali{}{temporalia/supplicatiolitaniae.gtex}

\vfill

\pars{Oratio Dominica.}

\cuminitiali{}{temporalia/oratiodominica.gtex}

\vfill
\pagebreak

% Oratio. %%%
\oratio

\vspace{-1mm}

\vfill

\rubrica{Hebdomadarius dicit Dominus vobiscum, vel, absente sacerdote vel diacono, sic concluditur:}

\vspace{2mm}

\antiphona{C}{temporalia/dominusnosbenedicat.gtex}

\rubrica{Postea cantatur a cantore:}

\vspace{2mm}

\cuminitiali{VII}{temporalia/benedicamus-tempore-paschali.gtex}

\vspace{1mm}

\vfill
\pagebreak

\end{document}

