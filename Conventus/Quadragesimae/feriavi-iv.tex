\newcommand{\oratio}{\pars{Oratio.}

\noindent Deus, qui fragilitáti nostræ cóngrua subsídia præparásti, concéde, quǽsumus, ut suæ reparatiónis efféctum et cum exsultatióne suscípiat et pia conversatióne recénseat.

\pars{Pro commemoratione Sancti Isidori Episcopi et Ecclesiæ Doctoris.} \scriptura{Dn. 12, 3}

\vspace{-4mm}

\antiphona{VII a trans.}{temporalia/ant-quidoctifuerint.gtex}

\vfill

\noindent Exáudi, quǽsumus, Dómine, preces nostras, quas in beáti Isidóri commemoratióne deférimus, ut Ecclésia tua eius intercessiónibus adiuvétur, cuius cæléstibus instrúitur disciplínis.

\pars{Pro pace in universo mundo.} \scriptura{Sir. 50, 25; 2 Esdr. 4, 20; \textbf{H416}}

\vspace{-4mm}

\antiphona{II D}{temporalia/ant-dapacemdomine.gtex}

\vfill

\noindent Deus, a quo sancta desidéria, recta consília et iusta sunt ópera: da servis tuis illam, quam mundus dare non potest, pacem; ut et corda nostra mandátis tuis dédita, et hóstium subláta formídine, témpora sint tua protectióne tranquílla.

\noindent Per Dóminum nostrum Iesum Christum, Fílium tuum, qui tecum vivit et regnat in unitáte Spíritus Sancti, Deus, per ómnia sǽcula sæculórum.

\noindent \Rbardot{} Amen.}
\newcommand{\invitatorium}{\pars{Invitatorium.}

\vspace{-4mm}

\antiphona{IV*}{temporalia/inv-christumdominum-cumdox.gtex}}
\newcommand{\hymnusmatutinum}{\pars{Hymnus}

\cuminitiali{I}{temporalia/hym-NuncTempus.gtex}}
\newcommand{\matutinum}{\pars{Psalmus 1.} \scriptura{Ps. 77, 1}

\vspace{-4mm}

\antiphona{I g\textsuperscript{3}}{temporalia/ant-inclinateaurem.gtex}

%\vspace{-2mm}

\scriptura{Ps. 77, 1-16}

%\vspace{-2mm}

\initiumpsalmi{temporalia/ps77i_xvi-initium-i-g3-auto.gtex}

\input{temporalia/ps77i_xvi-i-g3.tex}

\vfill

\antiphona{}{temporalia/ant-inclinateaurem.gtex}

\vfill
\pagebreak

\pars{Psalmus 2.} \scriptura{Sap. 16, 20}

\vspace{-4mm}

\antiphona{II D}{temporalia/ant-angelorumesca.gtex}

%\vspace{-2mm}

\scriptura{Ps. 77, 17-31}

%\vspace{-2mm}

\initiumpsalmi{temporalia/ps77iii-initium-ii-D-auto.gtex}

\input{temporalia/ps77iii-ii-D.tex}

\vfill

\antiphona{}{temporalia/ant-angelorumesca.gtex}

\vfill
\pagebreak

\pars{Psalmus 3.} \scriptura{Ps. 53, 6; \textbf{H223}}

\vspace{-4mm}

\antiphona{VIII G}{temporalia/ant-deusadiuvatme.gtex}

%\vspace{-2mm}

\scriptura{Ps. 77, 32-39}

%\vspace{-2mm}

\initiumpsalmi{temporalia/ps77xxxii_xxxix-initium-viii-G-auto.gtex}

\input{temporalia/ps77xxxii_xxxix-viii-G.tex} \Abardot{}

\vfill
\pagebreak}
\newcommand{\matversus}{\noindent \Vbardot{} Convertímini ad Dóminum Deum vestrum.

\noindent \Rbardot{} Quia benígnus et miséricors est.}
\newcommand{\lectioi}{\vspace{-4mm}

\pars{Lectio I.} \scriptura{Ex. 33, 7-11.18-23}

\noindent De libro Exodi.

\noindent In diébus illis: Móyses tollens tabernáculum teténdit ei extra castra procul vocavítque nomen eius Tabernáculum convéntus. Et omnis, qui quærébat Dóminum, egrediebátur ad tabernáculum convéntus extra castra. Cumque egrederétur Móyses ad tabernáculum, surgébat univérsa plebs et stabat unusquísque in óstio papiliónis sui; aspiciebántque tergum Móysi, donec ingrederétur tabernáculum. Ingrésso autem illo tabernáculum, descendébat colúmna nubis et stabat ad óstium loquebatúrque cum Móyse, cernéntibus univérsis quod colúmna nubis staret ad óstium tabernáculi. Stabántque ipsi et adorábant per fores tabernaculórum suórum.

\noindent Loquebátur autem Dóminus ad Móysen fácie ad fáciem, sicut solet loqui homo ad amícum suum. Cumque ille reverterétur in castra, miníster eius Iósue fílius Nun puer non recedébat de médio tabernáculi.

\noindent Dixit Móyses ad Dóminum: «Osténde mihi glóriam tuam».

\noindent Respóndit: «Ego osténdam omne bonum tibi et vocábo in nómine Dómini coram te; et miserébor, cui volúero, et clemens ero, in quem mihi placúerit». Rursúmque ait: «Non póteris vidére fáciem meam; non enim vidébit me homo et vivet». Et íterum: «Ecce, inquit, est locus apud me, stabis super petram; cumque transíbit glória mea, ponam te in forámine petræ et prótegam déxtera mea, donec tránseam; tollámque manum meam, et vidébis posterióra mea; fáciem autem meam vidére non póteris».}
\newcommand{\responsoriumi}{\pars{Responsorium 1.} \scriptura{\Rbardot{} Ex. 24, 18; \textbf{H160}}

\vspace{-5mm}

\responsorium{VIII}{temporalia/resp-moysesfamulusdomini-CROCHU.gtex}{}}
\newcommand{\lectioii}{\pars{Lectio II.} \scriptura{Ex. 34, 5-9.29-35}

\noindent Cumque descendísset Dóminus per nubem, stetit Móyses cum eo vocans in nómine Dómini. Et tránsiens coram eo clamávit: «Dóminus, Dóminus Deus, miséricors et clemens, pátiens et multæ miseratiónis ac verax, qui custódit misericórdiam in mília, qui aufert iniquitátem et scélera atque peccáta nihil autem impunítum sinit, qui reddit iniquitátem patrum in fíliis ac nepótibus in tértiam et quartam progéniem».

\noindent Festinúsque Móyses curvátus est pronus in terram et adórans ait: «Si invéni grátiam in conspéctu tuo, Dómine, óbsecro, ut gradiáris nobíscum; pópulus quidem duræ cervícis est, sed tu áuferes iniquitátes nostras atque peccáta nosque possidébis».

\noindent Cumque descénderet Móyses de monte Sínai, tenébat duas tábulas testimónii et ignorábat quod resplendéret cutis faciéi suæ ex consórtio sermónis Dómini. Vidéntes autem Aaron et fílii Israel resplendére cutem faciéi Móysi, timuérunt prope accédere; vocatíque ab eo revérsi sunt tam Aaron quam príncipes synagógæ. Et postquam locútus est ad eos, venérunt ad eum étiam omnes fílii Israel; quibus præcépit cuncta, quæ audíerat a Dómino in monte Sínai.

\noindent Impletísque sermónibus, pósuit velámen super fáciem suam, quod ingréssus ad Dóminum et loquens cum eo auferébat, donec exíret; et tunc loquebátur ad fílios Israel ómnia, quæ sibi fúerant imperáta. Qui vidébant cutem faciéi Móysi resplendére, sed operiébat ille rursus fáciem suam, donec ingréssus loquerétur cum eo.}
\newcommand{\responsoriumii}{\pars{Responsorium 2.} \scriptura{\Vbardot{} Ex. 34, 29; \textbf{H160}}

\vspace{-5mm}

\responsorium{VIII}{temporalia/resp-splendidafactaestfacies-CROCHU.gtex}{}}
\newcommand{\lectioiii}{\pars{Lectio III.} \scriptura{Ep. 5, 1-2: PG 26, 1379-1380}

\noindent Ex Epístolis paschálibus sancti Athanásii epíscopi.

\noindent Præclárum est, fratres mei, ab uno ad áliud festum perveníre, ab una ad áliam transíre oratiónem, ab una dénique ad áliam sollemnitátem. Adest vidélicet illud tempus, quod nobis novum inítium affert, beáti nempe notítiam Páschatis, in quo Dóminus fuit immolátus. Nos útique véscimur, tamquam vitæ cibo, animámque nostram pretióso illíus sánguine semper oblectámus, ceu fonte quodam: et nihilóminus semper sitímus, sempérque ardémus. Ipse vero sitiéntibus adest, et ob suam benignitátem diéi festáli ádmovet illos qui sitiéntia víscera habent, iuxta eiúsdem Salvatóris nostri effátum: \emph{Si quis sitit, véniat ad me et bibat.}

\noindent Non tunc tantúmmodo cum quisque accédit, sitim suam exstínguit; sed quotiescúmque étiam áliquis petit, libénter ei concéditur, ut ad Salvatórem accédat. Grátia festi haud uno témpore coarctátur, neque spléndidus eius rádius occásum pátitur, sed semper in promptu est illórum menti illuminándæ qui cúpiunt. Pollet autem contínua virtúte erga illos qui illuminátam mentem gerunt, et divínis libris atténdunt diu noctúque; velut homo ille qui beátus appellátur, prout in sancto psalmo scriptum est: \emph{Beátus vir, qui non ábiit in consílium impiórum et in via peccatórum non stetit, et in cáthedra pestiléntium non sedit; sed in lege Dómini volúntas eius, et in lege eius meditátur die ac nocte.}

\noindent Porro ille Deus, dilécti mei, qui inítio festum hoc nobis instítuit, ut id quotánnis peragátur concédit. Ipse qui Fílium suum morti trádidit, propter nostram salútem, eádem de causa sanctum festum largítur, quod in anni cursu notátur. Festum hoc regit nos per occurréntes nobis in hoc mundo ærúmnas: et nunc lætítiam nobis salútis confert Deus ex hoc festo emicántem, dum nos ad unum conséssum addúcit, cunctos spiritáliter ubíque cópulans, nobísque commúniter oráre concédens, communésque gratiárum actiónes persólvere, prout in festo factitáre opus est. Hoc est benignitátis eius miráculum: ipse scílicet ad hoc festum longínquos cóngregat, et eos qui forte córpore remóti sunt, fídei unitáte próximos facit.}
\newcommand{\responsoriumiii}{\pars{Responsorium 3.} \scriptura{\Rbardot{} Ps. 77, 1 \Vbardot{} ibid., 2; \textbf{H161}}

\vspace{-5mm}

\responsorium{VIII}{temporalia/resp-attenditepopulemeus-CROCHU-cumdox.gtex}{}}
\newcommand{\lectiobrevis}{\pars{Lectio Brevis.} \scriptura{Is. 53, 11-12}

\noindent Iustificábit iustus servus meus multos et iniquitátes eórum ipse portábit. Ideo dispértiam ei multos, et cum fórtibus dívidet spólia, pro eo quod trádidit in mortem ánimam suam et cum scelerátis reputátus est; et ipse peccátum multórum tulit et pro transgressóribus rogat.}
\newcommand{\responsoriumbreve}{\pars{Responsorium breve.} \scriptura{Ps. 90, 3}

\cuminitiali{IV}{temporalia/resp-ipseliberavitme.gtex}}
\newcommand{\hymnuslaudes}{\pars{Hymnus}

\cuminitiali{D}{temporalia/hym-IamChriste.gtex}}
\newcommand{\laudes}{\pars{Psalmus 1.} \scriptura{Ps. 50, 21; \textbf{H162}}

\vspace{-4mm}

\antiphona{VII a}{temporalia/ant-tuncacceptabis.gtex}

\scriptura{Psalmus 50.}

\initiumpsalmi{temporalia/ps50-initium-vii-a-auto.gtex}

\input{temporalia/ps50-vii-a.tex}

\vfill

\antiphona{}{temporalia/ant-tuncacceptabis.gtex}

\vfill
\pagebreak

\pars{Psalmus 2.}

\vspace{-4mm}

\antiphona{II D}{temporalia/ant-aedificansierusalem.gtex}

%\vspace{-2mm}

\scriptura{Canticum Tobiæ, Tob. 13, 10-18}

%\vspace{-2mm}

\initiumpsalmi{temporalia/tobiae2-initium-ii-D-auto.gtex}

\input{temporalia/tobiae2-ii-D.tex} \Abardot{}

\vfill
\pagebreak

\pars{Psalmus 3.} \scriptura{Ps. 147, 13; \textbf{H101}}

\vspace{-4mm}

\antiphona{VI F}{temporalia/ant-benedixitfiliistuis.gtex}

\vspace{-2mm}

\scriptura{Psalmus 147.}

%\vspace{-2mm}

\initiumpsalmi{temporalia/ps147-initium-vi-F-auto.gtex}

\input{temporalia/ps147-vi-F.tex} \Abardot{}

\vfill
\pagebreak}
\newcommand{\preces}{\noindent Christum salvatórem,~\gredagger{} qui per mortem et resurrectiónem suam nos redémit,~\grestar{} implorémus:

\Rbardot{} Dómine, miserére nostri.

\noindent Qui Ierúsalem ascendísti ad passiónem subeúndam,~\gredagger{} ut intráres in glóriam,~\grestar{} perduc Ecclésiam tuam in Pascha æternitátis.

\Rbardot{} Dómine, miserére nostri.

\noindent Qui, in cruce exaltátus,~\gredagger{} láncea mílitis transfígi voluísti,~\grestar{} sana vúlnera nostra.

\Rbardot{} Dómine, miserére nostri.

\noindent Qui crucem tuam árborem vitæ constituísti,~\grestar{} fructus eiúsdem baptísmate renátis largíre.

\Rbardot{} Dómine, miserére nostri.

\noindent Qui, in ligno pendens,~\gredagger{} latróni pæniténti pepercísti,~\grestar{} nobis peccatóribus ignósce.

\Rbardot{} Dómine, miserére nostri.}
\newcommand{\benedictus}{\pars{Canticum Zachariæ.} \scriptura{Io. 7, 28; \textbf{H163}}

\vspace{-4mm}

{
\grechangedim{interwordspacetext}{0.18 cm plus 0.15 cm minus 0.05 cm}{scalable}%
\antiphona{VII c}{temporalia/ant-etmescitis.gtex}
\grechangedim{interwordspacetext}{0.22 cm plus 0.15 cm minus 0.05 cm}{scalable}%
}

\vspace{-2mm}

\scriptura{Lc. 1, 68-79}

%\vspace{-2mm}

\initiumpsalmi{temporalia/benedictus-initium-vii-c-auto.gtex}

%\vspace{-1.5mm}

\input{temporalia/benedictus-vii-c.tex} \Abardot{}}
\newcommand{\hebdomada}{infra Hebdom. IV per Annum.}
%\newcommand{\hiemalis}{Hiemalis}
\newcommand{\matud}{Matutinum Hebdomadae D}
\newcommand{\matubd}{Matutinum Hebdomadae B vel D}
\newcommand{\laudd}{Laudes Hebdomadae D}
\newcommand{\laudbd}{Laudes Hebdomadae B vel D}

% LuaLaTeX

\documentclass[a4paper, twoside, 12pt]{article}
\usepackage[latin]{babel}
%\usepackage[landscape, left=3cm, right=1.5cm, top=2cm, bottom=1cm]{geometry} % okraje stranky
%\usepackage[landscape, a4paper, mag=1166, truedimen, left=2cm, right=1.5cm, top=1.6cm, bottom=0.95cm]{geometry} % okraje stranky
\usepackage[landscape, a4paper, mag=1400, truedimen, left=0.5cm, right=0.5cm, top=0.5cm, bottom=0.5cm]{geometry} % okraje stranky

\usepackage{fontspec}
\setmainfont[FeatureFile={junicode.fea}, Ligatures={Common, TeX}, RawFeature=+fixi]{Junicode}
%\setmainfont{Junicode}

% shortcut for Junicode without ligatures (for the Czech texts)
\newfontfamily\nlfont[FeatureFile={junicode.fea}, Ligatures={Common, TeX}, RawFeature=+fixi]{Junicode}

\usepackage{multicol}
\usepackage{color}
\usepackage{lettrine}
\usepackage{fancyhdr}

% usual packages loading:
\usepackage{luatextra}
\usepackage{graphicx} % support the \includegraphics command and options
\usepackage{gregoriotex} % for gregorio score inclusion
\usepackage{gregoriosyms}
\usepackage{wrapfig} % figures wrapped by the text
\usepackage{parcolumns}
\usepackage[contents={},opacity=1,scale=1,color=black]{background}
\usepackage{tikzpagenodes}
\usepackage{calc}
\usepackage{longtable}
\usetikzlibrary{calc}

\setlength{\headheight}{14.5pt}

% Commands used to produce a typical "Conventus" booklet

\newenvironment{titulusOfficii}{\begin{center}}{\end{center}}
\newcommand{\dies}[1]{#1

}
\newcommand{\nomenFesti}[1]{\textbf{\Large #1}

}
\newcommand{\celebratio}[1]{#1

}

\newcommand{\hora}[1]{%
\vspace{0.5cm}{\large \textbf{#1}}

\fancyhead[LE]{\thepage\ / #1}
\fancyhead[RO]{#1 / \thepage}
\addcontentsline{toc}{subsection}{#1}
}

% larger unit than a hora
\newcommand{\divisio}[1]{%
\begin{center}
{\Large \textsc{#1}}
\end{center}
\fancyhead[CO,CE]{#1}
\addcontentsline{toc}{section}{#1}
}

% a part of a hora, larger than pars
\newcommand{\subhora}[1]{
\begin{center}
{\large \textit{#1}}
\end{center}
%\fancyhead[CO,CE]{#1}
\addcontentsline{toc}{subsubsection}{#1}
}

% rubricated inline text
\newcommand{\rubricatum}[1]{\textit{#1}}

% standalone rubric
\newcommand{\rubrica}[1]{\vspace{3mm}\rubricatum{#1}}

\newcommand{\notitia}[1]{\textcolor{red}{#1}}

\newcommand{\scriptura}[1]{\hfill \small\textit{#1}}

\newcommand{\translatioCantus}[1]{\vspace{1mm}%
{\noindent\footnotesize \nlfont{#1}}}

% pruznejsi varianta nasledujiciho - umoznuje nastavit sirku sloupce
% s prekladem
\newcommand{\psalmusEtTranslatioB}[3]{
  \vspace{0.5cm}
  \begin{parcolumns}[colwidths={2=#3}, nofirstindent=true]{2}
    \colchunk{
      \input{#1}
    }

    \colchunk{
      \vspace{-0.5cm}
      {\footnotesize \nlfont
        \input{#2}
      }
    }
  \end{parcolumns}
}

\newcommand{\psalmusEtTranslatio}[2]{
  \psalmusEtTranslatioB{#1}{#2}{8.5cm}
}


\newcommand{\canticumMagnificatEtTranslatio}[1]{
  \psalmusEtTranslatioB{#1}{temporalia/extra-adventum-vespers/magnificat-boh.tex}{12cm}
}
\newcommand{\canticumBenedictusEtTranslatio}[1]{
  \psalmusEtTranslatioB{#1}{temporalia/extra-adventum-laudes/benedictus-boh.tex}{10.5cm}
}

% volne misto nad antifonami, kam si zpevaci dokresli neumy
\newcommand{\hicSuntNeumae}{\vspace{0.5cm}}

% prepinani mista mezi notovymi osnovami: pro neumovane a neneumovane zpevy
\newcommand{\cantusCumNeumis}{
  \setgrefactor{17}
  \global\advance\grespaceabovelines by 5mm%
}
\newcommand{\cantusSineNeumas}{
  \setgrefactor{17}
  \global\advance\grespaceabovelines by -5mm%
}

% znaky k umisteni nad inicialu zpevu
\newcommand{\superInitialam}[1]{\gresetfirstlineaboveinitial{\small {\textbf{#1}}}{\small {\textbf{#1}}}}

% pars officii, i.e. "oratio", ...
\newcommand{\pars}[1]{\textbf{#1}}

\newenvironment{psalmus}{
  \setlength{\parindent}{0pt}
  \setlength{\parskip}{5pt}
}{
  \setlength{\parindent}{10pt}
  \setlength{\parskip}{10pt}
}

%%%% Prejmenovat na latinske:
\newcommand{\nadpisZalmu}[1]{
  \hspace{2cm}\textbf{#1}\vspace{2mm}%
  \nopagebreak%

}

% mode, score, translation
\newcommand{\antiphona}[3]{%
\hicSuntNeumae
\superInitialam{#1}
\includescore{#2}

#3
}
 % Often used macros

\newcommand{\annusEditionis}{2021}

%%%% Vicekrat opakovane kousky

\newcommand{\anteOrationem}{
  \rubrica{Ante Orationem, cantatur a Superiore:}

  \pars{Supplicatio Litaniæ.}

  \cuminitiali{}{temporalia/supplicatiolitaniae.gtex}

  \pars{Oratio Dominica.}

  \cuminitiali{}{temporalia/oratiodominica.gtex}

  \rubrica{Deinde dicitur ab Hebdomadario:}

  \cuminitiali{}{temporalia/dominusvobiscum-solemnis.gtex}

  \rubrica{In choro monialium loco Dominus vobiscum dicitur:}

  \sineinitiali{temporalia/domineexaudi.gtex}
}

\setlength{\columnsep}{30pt} % prostor mezi sloupci

%%%%%%%%%%%%%%%%%%%%%%%%%%%%%%%%%%%%%%%%%%%%%%%%%%%%%%%%%%%%%%%%%%%%%%%%%%%%%%%%%%%%%%%%%%%%%%%%%%%%%%%%%%%%%
\begin{document}

% Here we set the space around the initial.
% Please report to http://home.gna.org/gregorio/gregoriotex/details for more details and options
\grechangedim{afterinitialshift}{2.2mm}{scalable}
\grechangedim{beforeinitialshift}{2.2mm}{scalable}
\grechangedim{interwordspacetext}{0.22 cm plus 0.15 cm minus 0.05 cm}{scalable}%
\grechangedim{annotationraise}{-0.2cm}{scalable}

% Here we set the initial font. Change 38 if you want a bigger initial.
% Emit the initials in red.
\grechangestyle{initial}{\color{red}\fontsize{38}{38}\selectfont}

\pagestyle{empty}

%%%% Titulni stranka
\begin{titulusOfficii}
\ifx\titulus\undefined
\nomenFesti{Feria VI \hebdomada{}}
\else
\titulus
\fi
\end{titulusOfficii}

\vfill

\begin{center}
%Ad usum et secundum consuetudines chori \guillemotright{}Conventus Choralis\guillemotleft.

%Editio Sancti Wolfgangi \annusEditionis
\end{center}

\scriptura{}

\pars{}

\pagebreak

\renewcommand{\headrulewidth}{0pt} % no horiz. rule at the header
\fancyhf{}
\pagestyle{fancy}

\cantusSineNeumas

\ifx\oratio\undefined
\ifx\lauda\undefined
\else
\newcommand{\oratio}{\pars{Oratio.}

\noindent Deus, qui ténebras ignorántiæ Verbi tui luce depéllis, auge in córdibus nostris virtútem fídei quam dedísti, ut ignis, quem grátia tua fecit accéndi, nullis tentatiónibus exstinguátur.

\noindent Per Dóminum nostrum Iesum Christum, Fílium tuum, qui tecum vivit et regnat in unitáte Spíritus Sancti, Deus, per ómnia sǽcula sæculórum.

\noindent \Rbardot{} Amen.}
\fi
\ifx\laudb\undefined
\else
\newcommand{\oratio}{\pars{Oratio.}

\noindent Præsta, quǽsumus, omnípotens Deus, ut laudes, quas nunc tibi persólvimus, in ætérnum cum sanctis tuis ubérius decantáre valeámus.

\noindent Per Dóminum nostrum Iesum Christum, Fílium tuum, qui tecum vivit et regnat in unitáte Spíritus Sancti, Deus, per ómnia sǽcula sæculórum.

\noindent \Rbardot{} Amen.}
\fi
\ifx\laudc\undefined
\else
\newcommand{\oratio}{\pars{Oratio.}

\noindent Illábere sénsibus nostris, omnípotens Pater, ut, in præceptórum tuórum lúmine gradiéntes, te ducem semper sequámur et príncipem.

\noindent Per Dóminum nostrum Iesum Christum, Fílium tuum, qui tecum vivit et regnat in unitáte Spíritus Sancti, Deus, per ómnia sǽcula sæculórum.

\noindent \Rbardot{} Amen.}
\fi
\fi

\hora{Ad Matutinum.} %%%%%%%%%%%%%%%%%%%%%%%%%%%%%%%%%%%%%%%%%%%%%%%%%%%%%

\vspace{2mm}

\cuminitiali{}{temporalia/dominelabiamea.gtex}

\vfill
%\pagebreak

\vspace{2mm}

\ifx\invitatorium\undefined
\pars{Invitatorium.} \scriptura{Ps. 94, 6.7; Psalmus 94}

\antiphona{E}{temporalia/inv-dominumdeum.gtex}
\else
\invitatorium
\fi

\vfill
\pagebreak

\ifx\hymnusmatutinum\undefined
\ifx\matuac\undefined
\else
\pars{Hymnus.} \scriptura{Gregorius Magnus (+604)}

{
\grechangedim{interwordspacetext}{0.10 cm plus 0.15 cm minus 0.05 cm}{scalable}%
\antiphona{IV}{temporalia/hym-TuTrinitatis.gtex}
\grechangedim{interwordspacetext}{0.22 cm plus 0.15 cm minus 0.05 cm}{scalable}%
}
\fi
\ifx\matubd\undefined
\else
\pars{Hymnus.}

{
\grechangedim{interwordspacetext}{0.10 cm plus 0.15 cm minus 0.05 cm}{scalable}%
\antiphona{II}{temporalia/hym-GalliCantu.gtex}
\grechangedim{interwordspacetext}{0.22 cm plus 0.15 cm minus 0.05 cm}{scalable}%
}
\fi
\else
\hymnusmatutinum
\fi

\vspace{-3mm}

\vfill
\pagebreak

\ifx\matua\undefined
\else
% MAT A
\pars{Psalmus 1.} \scriptura{Ps. 34, 1; \textbf{H93}}

\vspace{-4mm}

\antiphona{IV g}{temporalia/ant-expugnaimpugnantes.gtex}

%\vspace{-2mm}

\scriptura{Ps. 34, 1-10}

%\vspace{-2mm}

\initiumpsalmi{temporalia/ps34i-initium-iv-g-auto.gtex}

\input{temporalia/ps34i-iv-g.tex} \Abardot{}

\vfill
\pagebreak

\pars{Psalmus 2.} \scriptura{Ps. 118, 154; \textbf{H174}}

\vspace{-4mm}

\antiphona{VIII G}{temporalia/ant-iudicacausam.gtex}

%\vspace{-2mm}

\scriptura{Ps. 34, 11-17}

%\vspace{-2mm}

\initiumpsalmi{temporalia/ps34ii-initium-viii-G-auto.gtex}

\input{temporalia/ps34ii-viii-G.tex} \Abardot{}

\vfill
\pagebreak

\pars{Psalmus 3.} \scriptura{Ps. 50, 16; \textbf{H177}}

\vspace{-4mm}

\antiphona{VIII G}{temporalia/ant-liberame.gtex}

%\vspace{-2mm}

\scriptura{Ps. 34, 18-28}

\vspace{-2mm}

\initiumpsalmi{temporalia/ps34iii-initium-viii-G-auto.gtex}

\input{temporalia/ps34iii-viii-G.tex} \Abardot{}

\vfill
\pagebreak
\fi
\ifx\matub\undefined
\else
% MAT B
\pars{Psalmus 1.} \scriptura{Ps. 37, 2}

\vspace{-4mm}

\antiphona{VIII c}{temporalia/ant-neiniratua.gtex}

%\vspace{-2mm}

\scriptura{Ps. 37, 2-5}

%\vspace{-2mm}

\initiumpsalmi{temporalia/ps37ii_v-initium-viii-C-auto.gtex}

\input{temporalia/ps37ii_v-viii-C.tex} \Abardot{}

\vfill
\pagebreak

\pars{Psalmus 2.} \scriptura{Ps. 34, 4; \textbf{H218}}

\vspace{-4mm}

\antiphona{II* a}{temporalia/ant-confundantur.gtex}

%\vspace{-2mm}

\scriptura{Ps. 37, 6-13}

%\vspace{-2mm}

\initiumpsalmi{temporalia/ps37vi_xiii-initium-ii_-a-auto.gtex}

\input{temporalia/ps37vi_xiii-ii_-a.tex} \Abardot{}

\vfill
\pagebreak

\pars{Psalmus 3.} \scriptura{Ps. 139, 9.8}

\vspace{-4mm}

\antiphona{II A}{temporalia/ant-nederelinquasme.gtex}

%\vspace{-2mm}

\scriptura{Ps. 37, 14-23}

%\vspace{-2mm}

\initiumpsalmi{temporalia/ps37xiv_xxiii-initium-ii-A-auto.gtex}

\input{temporalia/ps37xiv_xxiii-ii-A.tex} \Abardot{}

\vfill
\pagebreak
\fi
\ifx\matuc\undefined
\else
% MAT C
\pars{Psalmus 1.} \scriptura{Ps. 68, 10; \textbf{H178}}

\vspace{-4mm}

\antiphona{VIII c}{temporalia/ant-zelusdomus.gtex}

%\vspace{-3mm}

\scriptura{Ps. 68, 2-13}

%\vspace{-2mm}

\initiumpsalmi{temporalia/ps68ii_xiii-initium-viii-c-auto.gtex}

%\vspace{-1.5mm}

\input{temporalia/ps68ii_xiii-viii-c.tex}

\vfill

\antiphona{}{temporalia/ant-zelusdomus.gtex}

\vfill
\pagebreak

\pars{Psalmus 2.}

\vspace{-4mm}

\antiphona{VIII c}{temporalia/ant-consolantemme.gtex}

%\vspace{-2mm}

\scriptura{Ps. 68, 14-22}

%\vspace{-2mm}

\initiumpsalmi{temporalia/ps68xiv_xxii-initium-viii-c-auto.gtex}

\input{temporalia/ps68xiv_xxii-viii-c.tex} \Abardot{}

\vfill
\pagebreak

\pars{Psalmus 3.} \scriptura{Ps. 68, 33; \textbf{H96}}

\vspace{-4mm}

\antiphona{VIII G}{temporalia/ant-quaeritedominumet.gtex}

%\vspace{-2mm}

\scriptura{Ps. 68, 30-37}

%\vspace{-2mm}

\initiumpsalmi{temporalia/ps68iii-initium-viii-G-auto.gtex}

\input{temporalia/ps68iii-viii-G.tex} \Abardot{}

\vfill
\pagebreak
\fi
\ifx\matud\undefined
\else
% MAT D
\pars{Psalmus 1.} \scriptura{Ps. 72, 8; \textbf{H179}}

\vspace{-4mm}

\antiphona{VIII c}{temporalia/ant-cogitaveruntimpii.gtex}

%\vspace{-3mm}

\scriptura{Ps. 54, 2-8}

%\vspace{-2mm}

\initiumpsalmi{temporalia/ps54i-initium-viii-c-auto.gtex}

%\vspace{-1.5mm}

\input{temporalia/ps54i-viii-c.tex} \Abardot{}

\vfill
\pagebreak

\pars{Psalmus 2.} \scriptura{Ps. 34, 4; \textbf{H178}}

\vspace{-4mm}

\antiphona{VII c\textsuperscript{2}}{temporalia/ant-avertanturretrorsum.gtex}

%\vspace{-2mm}

\scriptura{Ps. 54, 9-16}

%\vspace{-2mm}

\initiumpsalmi{temporalia/ps54ii-initium-vii-c2-auto.gtex}

\input{temporalia/ps54ii-vii-c2.tex} \Abardot{}

\vfill
\pagebreak

\pars{Psalmus 3.}

\vspace{-4mm}

\antiphona{VIII G}{temporalia/ant-iustusnonconturbabitur.gtex}

%\vspace{-2mm}

\scriptura{Ps. 54, 17-24}

%\vspace{-2mm}

\initiumpsalmi{temporalia/ps54iii-initium-viii-G-auto.gtex}

\input{temporalia/ps54iii-viii-G.tex} \Abardot{}

\vfill
\pagebreak
\fi

\pars{Versus.}

\ifx\matversus\undefined
\ifx\matua\undefined
\else
\noindent \Vbardot{} Fili mi, custódi sermónes meos.

\noindent \Rbardot{} Serva mandáta mea et vives.
\fi
\ifx\matub\undefined
\else
\noindent \Vbardot{} Oculi mei defecérunt in desidério salutáris tui.

\noindent \Rbardot{} Et elóquii iustítiæ tuæ.
\fi
\ifx\matuc\undefined
\else
\noindent \Vbardot{} Dóminus vias suas docébit nos.

\noindent \Rbardot{} Et ambulábimus in sémitis eius.
\fi
\ifx\matud\undefined
\else
\noindent \Vbardot{} Tribulátio et angústia invenérunt me.

\noindent \Rbardot{} Mandáta tua meditátio mea est.
\fi
\else
\matversus
\fi

\vspace{5mm}

\sineinitiali{temporalia/oratiodominica-mat.gtex}

\vspace{5mm}

\pars{Absolutio.}

\cuminitiali{}{temporalia/absolutio-ipsius.gtex}

\vfill
\pagebreak

\cuminitiali{}{temporalia/benedictio-solemn-deus.gtex}

\vspace{7mm}

\lectioi

\noindent \Vbardot{} Tu autem, Dómine, miserére nobis.
\noindent \Rbardot{} Deo grátias.

\vfill
\pagebreak

\responsoriumi

\vfill
\pagebreak

\cuminitiali{}{temporalia/benedictio-solemn-christus.gtex}

\vspace{7mm}

\lectioii

\noindent \Vbardot{} Tu autem, Dómine, miserére nobis.
\noindent \Rbardot{} Deo grátias.

\vfill
\pagebreak

\responsoriumii

\vfill
\pagebreak

\cuminitiali{}{temporalia/benedictio-solemn-ignem.gtex}

\vspace{7mm}

\lectioiii

\noindent \Vbardot{} Tu autem, Dómine, miserére nobis.
\noindent \Rbardot{} Deo grátias.

\vfill
\pagebreak

\responsoriumiii

\vfill
\pagebreak

\rubrica{Reliqua omittuntur, nisi Laudes separandæ sint.}

\sineinitiali{temporalia/domineexaudi.gtex}

\vfill

\oratio

\vfill

\noindent \Vbardot{} Dómine, exáudi oratiónem meam.
\Rbardot{} Et clamor meus ad te véniat.

\vfill

\noindent \Vbardot{} Benedicámus Dómino.
\noindent \Rbardot{} Deo grátias.

\vfill

\noindent \Vbardot{} Fidélium ánimæ per misericórdiam Dei requiéscant in pace.
\Rbardot{} Amen.

\vfill
\pagebreak

\hora{Ad Laudes.} %%%%%%%%%%%%%%%%%%%%%%%%%%%%%%%%%%%%%%%%%%%%%%%%%%%%%

\cantusSineNeumas

\vspace{0.5cm}
\grechangedim{interwordspacetext}{0.18 cm plus 0.15 cm minus 0.05 cm}{scalable}%
\cuminitiali{}{temporalia/deusinadiutorium-communis-quad.gtex}
\grechangedim{interwordspacetext}{0.22 cm plus 0.15 cm minus 0.05 cm}{scalable}%

\vfill
\pagebreak

\ifx\hymnuslaudes\undefined
\ifx\laudac\undefined
\else
\pars{Hymnus}

\grechangedim{interwordspacetext}{0.16 cm plus 0.15 cm minus 0.05 cm}{scalable}%
\cuminitiali{I}{temporalia/hym-AEternaCaeli.gtex}
\grechangedim{interwordspacetext}{0.22 cm plus 0.15 cm minus 0.05 cm}{scalable}%
\vspace{-3mm}
\fi
\ifx\laudbd\undefined
\else
\pars{Hymnus}

\grechangedim{interwordspacetext}{0.16 cm plus 0.15 cm minus 0.05 cm}{scalable}%
\cuminitiali{IV}{temporalia/hym-DeusQui.gtex}
\grechangedim{interwordspacetext}{0.22 cm plus 0.15 cm minus 0.05 cm}{scalable}%
\vspace{-3mm}
\fi
\else
\hymnuslaudes
\fi

\vfill
\pagebreak

\ifx\lauda\undefined
\else
\pars{Psalmus 1.} \scriptura{Ps. 50, 3; \textbf{H93}}

\vspace{-4mm}

\antiphona{VI F}{temporalia/ant-misereremeideus.gtex}

\scriptura{Psalmus 50.}

\initiumpsalmi{temporalia/ps50-initium-vi-F-auto.gtex}

\input{temporalia/ps50-vi-F.tex}

\vfill

\antiphona{}{temporalia/ant-misereremeideus.gtex}

\vfill
\pagebreak

\pars{Psalmus 2.} \scriptura{Is. 45, 25}

\vspace{-4mm}

\antiphona{V a}{temporalia/ant-indominoiustificabitur.gtex}

\scriptura{Canticum Isaiæ, Is. 45, 15-30}

%\vspace{-2mm}

\initiumpsalmi{temporalia/isaiae2-initium-v-a-auto.gtex}

\input{temporalia/isaiae2-v-a.tex}

\vfill

\antiphona{}{temporalia/ant-indominoiustificabitur.gtex}

\vfill
\pagebreak

\pars{Psalmus 3.} \scriptura{Ps. 99, 1; \textbf{H98}}

\vspace{-4mm}

\antiphona{IV* e}{temporalia/ant-iubilatedeo.gtex}

\scriptura{Psalmus 99.}

\initiumpsalmi{temporalia/ps99-initium-iv_-e-auto.gtex}

\input{temporalia/ps99-iv_-e.tex} \Abardot{}

\vfill
\pagebreak
\fi
\ifx\laudb\undefined
\else
\pars{Psalmus 1.} \scriptura{Ps. 50, 4; \textbf{H95}}

\vspace{-4mm}

\antiphona{VII a}{temporalia/ant-ampliuslavame.gtex}

\scriptura{Psalmus 50.}

\initiumpsalmi{temporalia/ps50-initium-vii-a-auto.gtex}

\input{temporalia/ps50-vii-a.tex}

\vfill

\antiphona{}{temporalia/ant-ampliuslavame.gtex}

\vfill
\pagebreak

\pars{Psalmus 2.} \scriptura{Hab. 3, 2; \textbf{H99}}

\vspace{-6mm}

\antiphona{IV* e}{temporalia/ant-domineaudivi.gtex}

\vspace{-2mm}

\scriptura{Canticum Habacuc, Hab. 3, 2-19}

%\vspace{-2mm}

%\initiumpsalmi{temporalia/habacuc-initium-iv_-e-auto.gtex}
\initiumpsalmi{temporalia/habacuc-initium-iv_-e.gtex}

\input{temporalia/habacuc-iv_-e.tex}

\vfill

\antiphona{}{temporalia/ant-domineaudivi.gtex}

\vfill
\pagebreak

\pars{Psalmus 3.} \scriptura{Ps. 147, 12}

\vspace{-4mm}

\antiphona{E}{temporalia/ant-laudaierusalem.gtex}

\vspace{-2mm}

\scriptura{Psalmus 147.}

%\vspace{-3mm}

%\initiumpsalmi{temporalia/ps147-initium-e-auto.gtex}
\initiumpsalmi{temporalia/ps147-initium-e.gtex}

\input{temporalia/ps147-e.tex} \Abardot{}

\vfill
\pagebreak
\fi
\ifx\laudc\undefined
\else
\pars{Psalmus 1.} \scriptura{Ps. 50, 6.3; \textbf{H96}}

\vspace{-4mm}

\antiphona{VIII G\textsuperscript{2}}{temporalia/ant-tibisoli.gtex}

\scriptura{Psalmus 50.}

\initiumpsalmi{temporalia/ps50-initium-viii-G2-auto.gtex}

\input{temporalia/ps50-viii-G2.tex}

\vfill

\antiphona{}{temporalia/ant-tibisoli.gtex}

\vfill
\pagebreak

\pars{Psalmus 2.}

\vspace{-4mm}

\antiphona{VIII G}{temporalia/ant-nosnosderelinquas.gtex}

%\vspace{-2mm}

\scriptura{Canticum Ieremiæ, Ier. 14, 17-31}

%\vspace{-2mm}

\initiumpsalmi{temporalia/jeremiae2-initium-viii-G.gtex}

\input{temporalia/jeremiae2-viii-G.tex} \Abardot{}

\vfill
\pagebreak

\pars{Psalmus 3.}

\vspace{-4mm}

\antiphona{E}{temporalia/ant-servitedominoinlaetitia.gtex}

\vspace{-2mm}

\scriptura{Psalmus 99.}

%\vspace{-2mm}

\initiumpsalmi{temporalia/ps99-initium-e.gtex}

\input{temporalia/ps99-e.tex} \Abardot{}

\vfill
\pagebreak
\fi
\ifx\laudd\undefined
\else
\pars{Psalmus 1.} \scriptura{Ps. 50, 12}

\vspace{-4mm}

\antiphona{I a\textsuperscript{2}}{temporalia/ant-cormundumcrea.gtex}

\scriptura{Psalmus 50.}

\initiumpsalmi{temporalia/ps50-initium-i-a2-auto.gtex}

\input{temporalia/ps50-i-a2.tex}

\vfill

\antiphona{}{temporalia/ant-cormundumcrea.gtex}

\vfill
\pagebreak

\pars{Psalmus 2.}

\vspace{-4mm}

\antiphona{II D}{temporalia/ant-aedificansierusalem.gtex}

%\vspace{-2mm}

\scriptura{Canticum Tobiæ, Tob. 13, 10-18}

%\vspace{-2mm}

\initiumpsalmi{temporalia/tobiae2-initium-ii-D-auto.gtex}

\input{temporalia/tobiae2-ii-D.tex} \Abardot{}

\vfill
\pagebreak

\pars{Psalmus 3.} \scriptura{Ps. 147, 13; \textbf{H101}}

\vspace{-4mm}

\antiphona{VI F}{temporalia/ant-benedixitfiliistuis.gtex}

\vspace{-2mm}

\scriptura{Psalmus 147.}

%\vspace{-2mm}

\initiumpsalmi{temporalia/ps147-initium-vi-F-auto.gtex}

\input{temporalia/ps147-vi-F.tex} \Abardot{}

\vfill
\pagebreak
\fi

\ifx\lectiobrevis\undefined
\ifx\lauda\undefined
\else
\pars{Lectio Brevis.} \scriptura{Eph. 4, 29-32}

\noindent Omnis sermo malus ex ore vestro non procédat, sed si quis bonus ad ædificatiónem opportunitátis, ut det grátiam audiéntibus. Et nolíte contristáre Spíritum Sanctum Dei, in quo signáti estis in diem redemptiónis. Omnis amaritúdo et ira et indignátio et clamor et blasphémia tollátur a vobis cum omni malítia. Estóte autem ínvicem benígni, misericórdes, donántes ínvicem, sicut et Deus in Christo donávit vobis.
\fi
\ifx\laudb\undefined
\else
\pars{Lectio Brevis.} \scriptura{Eph. 2, 13-16}

\noindent Nunc in Christo Iesu vos, qui aliquándo erátis longe, facti estis prope in sánguine Christi. Ipse est enim pax nostra, qui fecit utráque unum et médium paríetem macériæ solvit, inimicítiam, in carne sua, legem mandatórum in decrétis evácuans, ut duos condat in semetípso in unum novum hóminem, fáciens pacem, et reconcíliet ambos in uno córpore Deo per crucem interfíciens inimicítiam in semetípso.
\fi
\ifx\laudc\undefined
\else
\pars{Lectio Brevis.} \scriptura{2 Cor. 12, 9-10}

\noindent Libentíssime gloriábor in infirmitátibus meis, ut inhábitet in me virtus Christi. Propter quod pláceo mihi in infirmitátibus, in contuméliis, in necessitátibus, in persecutiónibus et in angústiis, pro Christo: cum enim infírmor, tunc potens sum.
\fi
\ifx\laudd\undefined
\else
\pars{Lectio Brevis.} \scriptura{2 Cor. 1, 3-5}

\noindent Benedíctus Deus et Pater Dómini nostri Iesu Christi, Pater misericordiárum et Deus totíus consolatiónis, qui consolátur nos in omni tribulatióne nostra, ut possímus et ipsi consolári eos, qui in omni pressúra sunt, per exhortatiónem, qua exhortámur et ipsi a Deo; quóniam, sicut abúndant passiónes Christi in nobis, ita per Christum abúndat et consolátio nostra.
\fi
\else
\lectiobrevis
\fi

\vfill

\ifx\responsoriumbreve\undefined
\ifx\laudac\undefined
\else
\pars{Responsorium breve.} \scriptura{Ps. 142, 8}

\cuminitiali{VI}{temporalia/resp-auditamfacmihi.gtex}
\fi
\ifx\laudbd\undefined
\else
\pars{Responsorium breve.} \scriptura{Ps. 56, 3-4}

\cuminitiali{VI}{temporalia/resp-clamaboaddeum.gtex}
\fi
\else
\responsoriumbreve
\fi

\vfill
\pagebreak

\ifx\benedictus\undefined
\ifx\laudac\undefined
\else
\pars{Canticum Zachariæ.} \scriptura{Lc. 1, 68; \textbf{H422}}

%\vspace{-4mm}

{
\grechangedim{interwordspacetext}{0.18 cm plus 0.15 cm minus 0.05 cm}{scalable}%
\antiphona{V a}{temporalia/ant-visitavitetfecit.gtex}
\grechangedim{interwordspacetext}{0.22 cm plus 0.15 cm minus 0.05 cm}{scalable}%
}

%\vspace{-3mm}

\scriptura{Lc. 1, 68-79}

%\vspace{-2mm}

\cantusSineNeumas
\initiumpsalmi{temporalia/benedictus-initium-v-a-auto.gtex}

%\vspace{-1.5mm}

\input{temporalia/benedictus-v-a.tex} \Abardot{}
\fi
\ifx\laudbd\undefined
\else
\pars{Canticum Zachariæ.} \scriptura{Lc. 1, 78; \textbf{H423}}

%\vspace{-4mm}

{
\grechangedim{interwordspacetext}{0.18 cm plus 0.15 cm minus 0.05 cm}{scalable}%
\antiphona{VIII G}{temporalia/ant-pervisceramisericordiae.gtex}
\grechangedim{interwordspacetext}{0.22 cm plus 0.15 cm minus 0.05 cm}{scalable}%
}

%\vspace{-3mm}

\scriptura{Lc. 1, 68-79}

%\vspace{-1mm}

\initiumpsalmi{temporalia/benedictus-initium-viii-G-auto.gtex}

\input{temporalia/benedictus-viii-G.tex} \Abardot{}
\fi
\else
\benedictus
\fi

\vspace{-1cm}

\vfill
\pagebreak

\pars{Preces.}

\sineinitiali{}{temporalia/tonusprecum.gtex}

\ifx\preces\undefined
\ifx\lauda\undefined
\else
\noindent Christum, qui per crucem suam salútem géneri cóntulit humáno, adorémus, \gredagger{} et pie clamémus:

\Rbardot{} Misericórdiam tuam nobis largíre, Dómine.

\noindent Christe, sol et dies noster, illúmina nos rádiis tuis, \gredagger{} et omnes sensus malos iam mane compésce.

\Rbardot{} Misericórdiam tuam nobis largíre, Dómine.

\noindent Custódi cogitatiónes, sermónes et ópera nostra, \gredagger{} ut hódie in conspéctu tuo placére possímus.

\Rbardot{} Misericórdiam tuam nobis largíre, Dómine.

\noindent Avérte fáciem tuam a peccátis nostris, \gredagger{} et omnes iniquitátes nostras dele.

\Rbardot{} Misericórdiam tuam nobis largíre, Dómine.

\noindent Per crucem et resurrectiónem tuam, \gredagger{} reple nos consolatióne Spíritus Sancti.

\Rbardot{} Misericórdiam tuam nobis largíre, Dómine.
\fi
\ifx\laudb\undefined
\else
\noindent Christum, qui sánguine suo per Spíritum Sanctum semetípsum óbtulit Patri ad emundándam consciéntiam nostram ab opéribus mórtuis, \gredagger{} adorémus et sincéro corde profiteámur:

\Rbardot{} In tua voluntáte pax nostra, Dómine.

\noindent Diéi exórdium a tua benignitáte suscépimus, \gredagger{} nobis páriter vitæ novæ concéde inítium.

\Rbardot{} In tua voluntáte pax nostra, Dómine.

\noindent Qui ómnia creásti providúsque consérvas, \gredagger{} fac ut inspiciámus perénne tui vestígium in creátis.

\Rbardot{} In tua voluntáte pax nostra, Dómine.

\noindent Qui sánguine tuo novum et ætérnum testaméntum sanxísti, \gredagger{} da ut, quæ prǽcipis faciéntes, tuo fidéles fœ́deri maneámus.

\Rbardot{} In tua voluntáte pax nostra, Dómine.

\noindent Qui, in cruce pendens, una cum sánguine aquam de látere effudísti, \gredagger{} hoc salutári flúmine áblue peccáta nostra et civitátem Dei lætífica.

\Rbardot{} In tua voluntáte pax nostra, Dómine.
\fi
\ifx\laudc\undefined
\else
\noindent Ad Christum óculos levémus, qui pro pópulo suo natus et mórtuus est ac resurréxit. \gredagger{} Itaque eum fidénter deprecémur:

\Rbardot{} Salva, Dómine, quos tuo sánguine redemísti.

\noindent Benedíctus es, Iesu hóminum salvátor, qui passiónem et crucem pro nobis subíre non dubitásti, \gredagger{} et sánguine tuo pretióso nos redemísti.

\Rbardot{} Salva, Dómine, quos tuo sánguine redemísti.

\noindent Qui promisísti te aquam esse datúrum saliéntem in vitam ætérnam, \gredagger{} Spíritum tuum effúnde super omnes hómines.

\Rbardot{} Salva, Dómine, quos tuo sánguine redemísti.

\noindent Qui discípulos misísti ad Evangélium géntibus prædicándum, \gredagger{} eos ádiuva, ut victóriam tuæ crucis exténdant.

\Rbardot{} Salva, Dómine, quos tuo sánguine redemísti.

\noindent Infírmis et míseris quos cruci tuæ sociásti, \gredagger{} virtútem et patiéntiam concéde.

\Rbardot{} Salva, Dómine, quos tuo sánguine redemísti.
\fi
\ifx\laudd\undefined
\else
\noindent Fratres, Salvatórem nostrum, testem fidélem, per mártyres interféctos propter verbum Dei, \gredagger{} celebrémus, clamántes:

\Rbardot{} Redemísti nos Deo in sánguine tuo.

\noindent Per mártyres tuos, qui líbere mortem in testimónium fídei sunt ampléxi, \gredagger{} da nobis, Dómine, veram spíritus libertátem.

\Rbardot{} Redemísti nos Deo in sánguine tuo.

\noindent Per mártyres tuos, qui fidem usque ad sánguinem sunt conféssi, \gredagger{} da nobis, Dómine, puritátem fideíque constántiam.

\Rbardot{} Redemísti nos Deo in sánguine tuo.

\noindent Per mártyres tuos, qui, sustinéntes crucem, tua vestígia sunt secúti, \gredagger{} da nobis, Dómine, ærúmnas vitæ fórtiter sustinére.

\Rbardot{} Redemísti nos Deo in sánguine tuo.

\noindent Per mártyres tuos, qui stolas suas lavérunt in sánguine Agni, \gredagger{} da nobis, Dómine, omnes insídias carnis mundíque devíncere.

\Rbardot{} Redemísti nos Deo in sánguine tuo.
\fi 
\else
\preces
\fi

\vfill

\pars{Oratio Dominica.}

\cuminitiali{}{temporalia/oratiodominicaalt.gtex}

\vfill
\pagebreak

\rubrica{vel:}

\pars{Supplicatio Litaniæ.}

\cuminitiali{}{temporalia/supplicatiolitaniae.gtex}

\vfill

\pars{Oratio Dominica.}

\cuminitiali{}{temporalia/oratiodominica.gtex}

\vfill
\pagebreak

% Oratio. %%%
\oratio

\vspace{-1mm}

\vfill

\rubrica{Hebdomadarius dicit Dominus vobiscum, vel, absente sacerdote vel diacono, sic concluditur:}

\vspace{2mm}

\antiphona{C}{temporalia/dominusnosbenedicat.gtex}

\rubrica{Postea cantatur a cantore:}

\vspace{2mm}

\cuminitiali{IV}{temporalia/benedicamus-feria-advequad.gtex}

\vspace{1mm}

\vfill
\pagebreak

\hora{Ad Vesperas.} %%%%%%%%%%%%%%%%%%%%%%%%%%%%%%%%%%%%%%%%%%%%%%%%%%%%%

\cantusSineNeumas

%\vspace{0.5cm}
\grechangedim{interwordspacetext}{0.18 cm plus 0.15 cm minus 0.05 cm}{scalable}%
\cuminitiali{}{temporalia/deusinadiutorium-communis-tq.gtex}
\grechangedim{interwordspacetext}{0.22 cm plus 0.15 cm minus 0.05 cm}{scalable}%

\vfill
%\pagebreak

\vspace{4mm}

\pars{Psalmus 1.} \scriptura{Ps. 141, 6; \textbf{H99}}

\vspace{-4mm}

\antiphona{VIII a}{temporalia/ant-portiomeadomine.gtex}

%\vspace{-4mm}

\scriptura{Psalmus 141.}

\initiumpsalmi{temporalia/ps141-initium-viii-A-auto.gtex}

\input{temporalia/ps141-viii-A.tex}

\vfill

\antiphona{}{temporalia/ant-portiomeadomine.gtex}

\vfill
\pagebreak

\pars{Psalmus 2.} \scriptura{Ps. 143, 1; \textbf{H99}}

\vspace{-4mm}

\antiphona{VI F}{temporalia/ant-benedictusdominus.gtex}

\scriptura{Psalmus 143, 1-8}

\initiumpsalmi{temporalia/ps143i-initium-vi-F-auto.gtex}

\input{temporalia/ps143i-vi-F.tex}

\vspace{4mm}

\rubrica{Hic non dicitur antiphonam.}

\vfill
\pagebreak

\pars{Psalmus 3.} \scriptura{Psalmus 143, 9-15}

\initiumpsalmi{temporalia/ps143ii-initium-vi-F-auto.gtex}

\input{temporalia/ps143ii-vi-F.tex}

\vfill

\antiphona{}{temporalia/ant-benedictusdominus.gtex}

\vfill
\pagebreak

\pars{Psalmus 4.} \scriptura{Ps. 144, 2; \textbf{H99}}

\vspace{-4mm}

\antiphona{VIII a}{temporalia/ant-persingulosdies.gtex}

\scriptura{Psalmus 144, 1-9}

\initiumpsalmi{temporalia/ps144i-initium-viii-A-auto.gtex}

\input{temporalia/ps144i-viii-A.tex} \Abardot{}

\vfill
\pagebreak

\pars{Capitulum.} \scriptura{Sir. 24, 14}

\grechangedim{interwordspacetext}{0.12 cm plus 0.15 cm minus 0.05 cm}{scalable}%
\cuminitiali{}{temporalia/capitulum-AbInitio.gtex}
\grechangedim{interwordspacetext}{0.22 cm plus 0.15 cm minus 0.05 cm}{scalable}%

\vfill

\pars{Responsorium breve.} \scriptura{Lc. 1, 28}

\cuminitiali{VI}{temporalia/resp-avemaria.gtex}

\vfill
\pagebreak

\pars{Hymnus}

\cuminitiali{I}{temporalia/hym-AveMarisStella.gtex}
\vspace{-3mm}

\vfill
%\pagebreak

\pars{Versus.} \scriptura{Ps. 43, 3}

\sineinitiali{temporalia/versus-diffusa-tq.gtex}

\vfill
\pagebreak

\ifx\magnificat\undefined
\pars{Canticum B. Mariæ V.} \scriptura{\textbf{H300}}

\vspace{-4mm}

{
\grechangedim{interwordspacetext}{0.18 cm plus 0.15 cm minus 0.05 cm}{scalable}%
\antiphona{II D}{temporalia/ant-beatamater.gtex}
\grechangedim{interwordspacetext}{0.22 cm plus 0.15 cm minus 0.05 cm}{scalable}%
}

\vspace{-3mm}

\scriptura{Lc. 1, 46-55}

\cantusSineNeumas
\initiumpsalmi{temporalia/magnificat-initium-ii-D.gtex}

%\vspace{-3mm}

\input{temporalia/magnificat-ii-D.tex} \Abardot{}
\else
\magnificat
\fi

\vspace{-1cm}

\vfill
\pagebreak

\anteOrationem

\pagebreak

% Oratio. %%%
\cuminitiali{}{temporalia/oratiobmv.gtex}

\vspace{-1mm}

\vfill

\rubrica{Hebdomadarius dicit iterum Dominus vobiscum, vel cantor dicit:}

\vspace{2mm}

\sineinitiali{temporalia/domineexaudi.gtex}

\rubrica{Postea cantatur a cantore:}

\vspace{2mm}

\cuminitiali{VIII}{temporalia/benedicamus-officium-bmv.gtex}

\vspace{1mm}

\vfill

\end{document}

