%%%% Preklady jednotlivych zpevu (nektere se opakuji, a je dobre mit je
% vsechny na jedne hromade)

\newcommand{\trOratioAnteOfficium}{\translatioCantus{Otevři, Pane, má ústa, abych chválil tvé svaté jméno.
Očisti mé srdce od všech marnivých, zvrácených a~jiných myšlenek, osvěť rozum, rozněť cit,
abych mohl důstojně, soustředěně a~zbožně recitovat a~vysloužil si být
vyslyšen před tváří tvé velebnosti. Skrze Krista…}}

\newcommand{\trOratioPostOfficium}{\translatioCantus{\textit{Následující modlitbu
opatřil pro ty, kdo ji zbožně vyřknou po hodinkách, papež Lev X.
odpustky za nedostatky a provinění vzniklé při konání hodinek z~lidské křehkosti. Říká se
vkleče.}
Svatosvaté a~nerozdílné Trojici, ukřižovanému lidství našeho Pána Ježíše
Krista, přeblažené a~přeslavné plodné neporušenosti vždy Panny Marie
i~souhrnu všech svatých buď ode všeho stvoření věčná chvála, čest a~sláva, nám
pak buď dáno odpuštění všech hříchů, po nekonečné věky věků. Amen.}}

% HOURS ---

\newcommand{\trAntI}{\translatioCantus{Muž boží měl kožený toulec, pečlivě
zavázaný, jenž mu visel na šíji a~často se ho dotýkal.}}

\newcommand{\trAntII}{\translatioCantus{Klíč od~něho tak dobře střežil, že
dokud žil v~těle, nikdo z~jeho žáků nezvěděl, co je uvnitř.}}

\newcommand{\trAntIII}{\translatioCantus{Ale když se odebral z~tohoto
života, schránku otevřeli a~objevili v~ní žíněné roucho a~měděný řetěz
potřísněný krví.}}

\newcommand{\trAntIV}{\translatioCantus{A když prohlédli mistrovo tělo,
nalezli jeho tělo na čtyřech místech hluboce zbrázděno ranami od řetězu.}}

\newcommand{\trAntV}{\translatioCantus{Krev vytékající z~těch ran, místy
prostoupila i~žíněným rouchem.}}

\newcommand{\trCapituli}{\translatioCantus{
Miláčkovi Boha a~lidí,
Mojžíšovi požehnané paměti,~\gredagger{}
dopřál slávu rovnou slávě svatých~\grestar{}
učinil ho mocným na postrach nepřátelům
a~jeho slovy zastavil divy.}}

\newcommand{\trRespVesp}{\translatioCantus{Ústa spravedlivého~\grestar{}
šeptají moudrost. \Vbardot{} A~jeho jazyk ohlašuje právo.}}

\newcommand{\trRespLaud}{\translatioCantus{Spravedlivého vodil Hospodin~\grestar{}
po přímých stezkách. \Vbardot{} A~ukázal mu Boží království.}}

\newcommand{\trVersus}{\translatioCantus{\Vbardot{} Ústa spravedlivého šeptají moudrost, aleluja.
\Rbardot{} A~jeho jazyk ohlašuje právo, aleluja.}}

\newcommand{\trAntMagnificatI}{\translatioCantus{Ctihodný Havel přistoupil
k~diákonovi, který poznal všechny cesty pouště, a~vyzvídal na něm, zda kdy
nalezl v~pustině místo vhodné k~přebývání: ,,Hořím``, pravil, ,,touhou vzníceného ducha
a~bažím po tom strávit dny vyměřené tomuto životu v~samotě.``}}

\newcommand{\trAntBenedictus}{\translatioCantus{Když na bujné oře vložili
nosítka a~sňali jim uzdu, vydali se přímo k~cele božího muže.}}

\newcommand{\trAntMagnificatII}{\translatioCantus{Právem se vrací na mysl
lidí tento světec, jenž přešel do radosti andělské, neboť za tohoto putování
zde byl pouze tělem, ale svou myslí již nedočkavě obcoval s~věčnou vlastí.}}

\newcommand{\trOrationis}{\translatioCantus{Bože, jenž nám dopřáváš radovat
se z~výroční slavnosti svatého tvého vyznavače Havla, uděl dobrotivě,
abychom když slavíme jeho narození, též se řídili podobou jeho skutků.
Skrze…}}

\newcommand{\trFideliumAnimae}{\translatioCantus{\Vbardot{} Duše věrných ať pro
milosrdenství Boží odpočívají v~pokoji. \Rbardot{} Amen.}}

% Completorium

\newcommand{\trJubeDomne}{\translatioCantus{Rač, pane, požehnat.}}

\newcommand{\trComplBenedictio}{\translatioCantus{Pokojnou noc a~svatou smrt
nechť nám dopřeje všemohoucí Pán. \Rbardot{} Amen.}}

\newcommand{\trComplLectioBr}{\translatioCantus{Buďte střízliví, bděte.
Váš protivník Ďábel obchází jako lev řvoucí a~hledá, koho by pohltil.
Postavte se proti němu pevní ve víře.  Ale ty, Pane, smiluj se nad námi.
\Rbardot{} Bohu díky.}}

\newcommand{\trComplAntI}{\translatioCantus{Rač se smilovati nade mnou,
Hospodine, a~vyslyš mou modlitbu.}}

\newcommand{\trComplCapituli}{\translatioCantus{Jsi přece, Hospodine,
uprostřed nás a~jmenujeme se po tobě.  Neopouštěj nás, Pane, náš Bože.}}

\newcommand{\trRespCompl}{\translatioCantus{Do tvých rukou, Pane,~\grestar{}
poroučím svého ducha. \Vbardot{} Ty mne zachráníš, Pane, Bože věrný.}}

\newcommand{\trComplVersus}{\translatioCantus{\Vbardot{} Střez mne jako zřítelnici oka,
aleluja. \Rbardot{} Ve stínu svých křídel uschovej mne, aleluja.}}

\newcommand{\trAntSalvaNos}{\translatioCantus{Ochraňuj nás, Pane, když
bdíme, a~buď s~námi, když spíme, ať bdíme s~Kristem a~odpočíváme v~pokoji.}}

\newcommand{\trComplOrationis}{\translatioCantus{Zavítej, prosíme, Pane, sem
do našeho příbytku a~daleko od něj zažeň všechny úklady nepřítele. Ať tu
bydlí tví svatí andělé a~tvoje požehnání buď nad ním stále. Skrze…}}

\newcommand{\trSalveRegina}{\translatioCantus{Zdrávas Královno,~\grestar{} matko
milosrdenství, živote, sladkosti a~naděje naše, buď zdráva!
K~tobě voláme, vyhnaní synové Evy,
k~tobě vzdycháme, lkajíce a~plačíce
v~tomto slzavém údolí.
A~proto, orodovnice naše,
obrať k~nám své milosrdné oči
a~Ježíše, požehnaný plod života svého,
nám po tomto putování ukaž,
ó milostivá, ó přívětivá,
ó přesladká,~\grestar{} Panno Maria!}}

\newcommand{\trOraProNobis}{\translatioCantus{\Vbardot{} 
Oroduj za nás, svatá Boží Rodičko,
\Rbardot{} aby nám Kristus dal účast na svých zaslíbeních.}}

% Matutinum

\newcommand{\trMatInvitatorium}{\translatioCantus{Pojďte, klaňme se Pánu a~králi vyznavačů.}}

\newcommand{\trMatVeniteA}{\translatioCantus{Pojďte, chvalme s~radostí Pána,
s~jásotem slavme Boha, svou spásu; předstupme před tvář jeho s~díky, písně plesu pějme jemu.}}

\newcommand{\trMatVeniteB}{\translatioCantus{Neboť Bůh veliký jest Hospodin, a~král nade všecky bohy.
Jsouť v~jeho ruce všecky hlubiny země, temena hor jsou majetek jeho.}}

\newcommand{\trMatVeniteC}{\translatioCantus{Jehoť jest moře, neb on je učinil; i~souš
je dílo jeho rukou. Pojďme, klanějme se, padněme, klekněme před Pánem, svým
tvůrcem. Jeť on Pán, náš Bůh, a~my jsme lid, jejž on vodí a~ovce, jež pase.}}

\newcommand{\trMatVeniteD}{\translatioCantus{Kéž byste poslechli dnes hlasu jeho:
,,Nezatvrzujte svých srdcí jak v~Hádce, jak v~Pokušení na poušti, kde vaši otcové pokoušeli mne,
zkoušeli mne, ač vídali skutky mé.``}}

\newcommand{\trMatVeniteE}{\translatioCantus{Čtyřicet roků mrzel jsem se na to pokolení
a~řekl jsem: ,,Lid je to myslí stále bloudící``! Oni však nechtěli znáti mé cesty, takže jsem
přisáhl ve svém hněvu: ,,Nedojdou odpočinku mého!\mbox{}``}}

\newcommand{\trMatAntI}{\translatioCantus{Rodiče~\grestar{} blaženého Havla zasvětili
svého syna v~ranném rozpuku dětství Pánu a~předali magistru Kolumbánovi.}}

\newcommand{\trMatAntII}{\translatioCantus{A když se příkladně živil něžným
citem, velice prospíval v~ctnosti. A~pilně do sebe ssál bohatství Písma.}}

\newcommand{\trMatAntIII}{\translatioCantus{Když nadešel čas odchodu,
schvátila blaženého Havla horečka.}}

\newcommand{\trMatAntIV}{\translatioCantus{Přihnal se ke svému opatu a~naznačil,
jak mocně nemocí trpí, a~proto nemůže vykonat uloženou cestu.}}

\newcommand{\trMatAntV}{\translatioCantus{Ó horečko, tak blahoslavená!
Ó nemoci, jež se rovná zdraví a~radosti!}}

\newcommand{\trMatAntVI}{\translatioCantus{Za nás~\grestar{} Havel trpěl
a~jít s~mistrem nemohl, neboť mu Bůh uložil, aby uzdravil naše nemoci.}}

\newcommand{\trMatVersusI}{\translatioCantus{\Vbardot{} Spravedlivého vodil
Hospodin po přímých stezkách. \Rbardot{} A~ukázal mu Boží království.}}

\newcommand{\trMatAbsolutioI}{\translatioCantus{Vyslyš Pane Ježíši Kriste
prosby svých služebníků~\gredagger{} a~smiluj se nad námi,~\grestar{} jenž
s~Otcem a~Duchem…}}

\newcommand{\trMatBenedictioI}{\translatioCantus{Rač, pane, požehnat.
Věčný Otec nám stále žehnej. \Rbardot{} Amen.}}

\newcommand{\trMatLecI}{\translatioCantus{Když se po celém Irsku slávou
skvěly mravy nejsvětějšího Kolumbána, tu rodiče svatého Havla,
ve světě vznešení a~u~Boha pobožní, v~útlém rozpuku života jej jako žáka obětovali Bohu,
aby prospíval ve cvičeních klášterního života.}}

\newcommand{\trMatRespI}{\translatioCantus{Rodiče~\grestar{} blaženého Havla svého
syna v~útlém rozpuku života obětovali Bohu a~\grestar{} svěřili Kolumbánovi do učení.
\Vbardot{} Byli totiž zbožní a~vznešení. Proto také svého syna~\grestar{}
svěřili…}}

\newcommand{\trMatBenedictioII}{\translatioCantus{Rač, pane, požehnat.
Jednorozený Boží Syn nám žehnej~\grestar{} a~nám pomáhej. \Rbardot{} Amen.}}

\newcommand{\trMatLecII}{\translatioCantus{A když se příkladně živil něžným
citem, velice prospíval v~ctnosti. Ne\-bes\-kou milostí posílen,
s~veškerou pílí do sebe ssál Svatá Písma, aby z~jejich pokladu dokázal přinést nové i~staré.}}

\newcommand{\trMatRespII}{\translatioCantus{Svatý Havel~\grestar{} vyzbrojen zbožnou
horlivostí ohněm pálil božiště démonů~\grestar{} a~všechny obětiny, jež nalezl, házel
do jezera. \Vbardot{} Vzav přede všemi modly, kameny je rozdrtil na kousky.
\grestar{} A…}}

\newcommand{\trMatBenedictioIII}{\translatioCantus{Rač, pane, požehnat.
Milost Ducha Svatého ať osvítí nám smysly~\grestar{} i~srdce. \Rbardot{} Amen.}}

\newcommand{\trMatLecIII}{\translatioCantus{Bystrým duchem sledoval pravidla
gramatiky i~jemnosti metra. A~těm, kdo chtěli, vykládal tajemství Písma tak
moudře, že všichni, kdo slyšeli hlas jeho moudrosti, ho měli za obdivuhodného a~chvály hodného.}}

\newcommand{\trMatRespIII}{\translatioCantus{Kolumbán proto~\grestar{} blaženému Havlu
uložil za úkol, aby odvedl lid od pohanského bludu,~\grestar{} neboť měl od Pána tu
milost, že měl nemalou znalost mluvy latinské i~barbarské.
\Vbardot{} Když přišel Havel v~čas modlitby, počal lidu ukazovat cestu pravdy.
\grestar{} Neboť…}}

\newcommand{\trMatBenedictioIV}{\translatioCantus{Rač, pane, požehnat.
Ať nám požehná Otec i~Syn v~sjednocení Svatého Ducha. \Rbardot{} Amen.}}

\newcommand{\trMatLecIV}{\translatioCantus{Jsa zralý touto moudrostí, přijal
se souhlasem všech a~na příkaz opata Kolumbána, postupuje po jednotlivých
jejích stupních, hodnost kněž\-skou.}}

\newcommand{\trMatRespIV}{\translatioCantus{Boží zápasník Havel~\grestar{}
se postil po tři dny, aby předzvěděné místo pro duchovní boj~\grestar{}
posvětil střídmým počátkem. \Vbardot{} Probděl noc na modlitbách, aby co
z~lásky k~Bohu počal~\grestar{} posvětil…}}

\newcommand{\trMatAntVII}{\translatioCantus{Když večeřeli,~\grestar{}
řekl diákon, že kdyby tam byl onen medvěd, Havel by mu udělil požehnání.}}

\newcommand{\trMatAntVIII}{\translatioCantus{A všichni přítomní viděli, že
na příkaz muže božího z~úst dívky vyšel démon jako černý a~přehrozný pták.}}

\newcommand{\trMatAntIX}{\translatioCantus{Všichni klerikové mluvili mezi
sebou: Ten Havel má dobrou pověst u~všech. Tento učedník ctnosti má být
pastýřem lidu.}}

\newcommand{\trMatAntX}{\translatioCantus{Svatý otec~\grestar{}
odvětil: Dobře praví, kéž by bylo pravda, co říkají.}}

\newcommand{\trMatAntXI}{\translatioCantus{Když pastýři církve~\grestar{}
slyšeli Havlovo učení, řekli: Vpravdě dnes ústy tohoto muže promluvil Duch Svatý.}}

\newcommand{\trMatAntXII}{\translatioCantus{Když měli řemeslníci starost, že
je jisté prkno příliš malé a~poodstoupili na chvilku od práce, hned se jim
pro zásluhy svatého Havla ukázalo větší.}}

\newcommand{\trMatVersusII}{\translatioCantus{\Vbardot{} Bůh si ho zamiloval a~ozdobil jej.
\Rbardot{} Oblékl mu roucho slávy.}}

\newcommand{\trMatAbsolutioII}{\translatioCantus{
Tvá milost a~laskavost nechť nám pomáhá, jenž žiješ a~vládneš s~Otcem a~Svatým Duchem na věky věků.}}

\newcommand{\trMatBenedictioV}{\translatioCantus{Rač, pane, požehnat.
Bůh Otec všemohoucí,~\grestar{} buď k~nám milostivý a~odpouštějící. \Rbardot{} Amen.}}

\newcommand{\trMatLecV}{\translatioCantus{Když tedy byl ustanoven do svatého úřadu,
dnem i~nocí Boha konejšil prosbami a~chtěje se zalíbit zraku Nejvyššího,
jenž vše vidí, se svými ctnost\-mi a~životem zalíbil všem a~byl jimi milován.
Když se toto dálo, blažený Kolumbán, chtěje tak následovat evangelickou dokonalost,
opustil vše, co ještě měl, vzal svůj kříž a~nahý následoval Pána.}}

\newcommand{\trMatRespV}{\translatioCantus{Když blahoslavený Havel~\grestar{}
kvůli modlitbě zabloudil v~houštinách a~křoví, sesul se na zem a~pravil:~\grestar{}
Zde bude mé spočinutí na věky. \Vbardot{} Když to diákon viděl, přiběhl
a~padlého pozdvihl. Ale muž boží, jenž věděl, co se stane, řekl: ,,Beze mne``~\grestar{}
zde…}}

\newcommand{\trMatBenedictioVI}{\translatioCantus{Rač, pane, požehnat.
Nechť nám Kristus dá radost věčného života. \Rbardot{} Amen.}}

\newcommand{\trMatLecVI}{\translatioCantus{A opět vsednuvše na loď,
připutovali do Británie, a~odtud se přeplavili do Galie, kde si vybudovali
příbytky a~tamější lidi, oddané dosud modlám, učili uctívat Otce a~Syna a~Svatého Ducha.}}

\newcommand{\trMatRespVI}{\translatioCantus{Pane Jezu Kriste,~\grestar{}
neodvrhuj mé tužby, ale ku cti tvé svaté Matky, svatých mučedníků i~vyznavačů~\grestar{}
si na tomto místě připrav příbytek, v~němž bychom ti mohli sloužit. \Vbardot{}
Ty, jenž ses ráčil narodit z~Panny a~vytrpět smrt pro spásu lidského rodu,~\grestar{} si…}}

\newcommand{\trMatBenedictioVII}{\translatioCantus{Rač, pane, požehnat.
Bůh rozněť v~nás oheň své lásky. \Rbardot{} Amen.}}

\newcommand{\trMatLecVII}{\translatioCantus{Blahoslavený Havel, vyzbrojen
zbožným zápalem, pálil ohněm svatyně, v~nichž se obětovalo démonům,
a~kdekoli nalezl obětiny, házel je do je\-ze\-ra. Pročež rozpálení hněvem pojali úmysl
Havla usmrtit. A~Kolumbána zbičovaného a~potupeného začali odhánět od svých hranic.}}

\newcommand{\trMatRespVII}{\translatioCantus{Boží vyvolenec Havel~\grestar{}
padl spolu s~diákonem na zem, a~tak se modlil k~všemohoucímu Bohu: Poruč, ať
démoni opustí toto místo,~\grestar{} aby bylo zasvěceno ke cti Tvého jména. \Vbardot{}
A~připrav si na tomto místě příbytek, v~němž bychom ti mohli sloužit.
\grestar{} Aby…}}

\newcommand{\trMatBenedictioVIII}{\translatioCantus{Rač, pane, požehnat.
Od neřestí a~hříchů osvoboď nás moc Svaté Trojice. \Rbardot{} Amen.}}

\newcommand{\trMatLecVIII}{\translatioCantus{Rozdrážděni takovým bezprávím,
rozhodli se zamířit do Itálie. Ale když nastal čas odchodu, blahoslaveného
Havla schvátila náhle horečka. Zašel tedy za svým opatem a~zjevil mu,
jak velice je soužen nemocí a~nemůže tudíž vykonat zamýšlenou cestu.}}

\newcommand{\trMatRespVIII}{\translatioCantus{Muž oddaný Bohu~\grestar{}
stejně jako prvního dne si k~sobě ráno zavolal diákona Magnoalda a~řekl: shledal jsem po
této probdělé noci, že se můj pán a~otec Kolumbán~\grestar{} odebral do nebes. \Vbardot{}
Za jeho pokoj musím obětovat spásnou oběť, když se z~úzkostí tohoto života
\grestar{} odebral…}}

\newcommand{\trMatAntXIII}{\translatioCantus{Služebník Boží Havel~\grestar{}
v~požehnaném stáří nechával duši naplněnou zásluhami přilnout k~blaženému
nepomíjejícímu dobru.}}

\newcommand{\trMatVersusIII}{\translatioCantus{\Vbardot{} Velkou slávu má.
\Rbardot{} Žes ho zachránil.}}

\newcommand{\trMatAbsolutioIII}{\translatioCantus{Z okovů našich hříchů,
\grestar{} vysvoboď nás všemohoucí a~milosrdný Pán. \Rbardot{} Amen.}}

\newcommand{\trMatBenedictioIX}{\translatioCantus{Rač, pane, požehnat.
Čtení evangelia nechť je nám~\grestar{} spásou a~ochranou. \Rbardot{} Amen.}}

\newcommand{\trMatLecIXa}{\translatioCantus{
Mějte opásaná bedra a~vaše lampy ať hoří.
Buďte jako lidé, kteří očekávají svého pána, když se vrací ze svatby.}}

\newcommand{\trMatLecIXb}{\translatioCantus{
Otevřeli jste, nejdražší bratři, evangelium a~přečetli z~něj čtení. Ale aby
se některým nejevila jen jeho prostota, krátce si je probereme, aby se
i~nevědomým stalo zřejmé, co i~učeným není zbytečné. Že mužská zpupnost
sídlí v~bedrech a~ženská v~pupku, dosvědčuje Pán, jenž blaženému Jobovi
praví o~ďáblu: Jeho moc je v~jeho bedrech a~síla v~pupku jeho břicha. A~po\-dle
tohoto prvotnějšího pohlaví se bedry míní zpupnost, když Pán mluví: Mějte
bedra přepásaná. Bedra si přepásáváme, když zdrženlivostí krotíme tělesnou
rozmařilost. Ale protože nedopouštět se zlého je méně nežli, když se
ně\-kdo snaží zapotit se u~dobrého díla, hned dodává: a~ve vašich rukou ať hoří
lampy. A~hořící lampy držíme v~rukou, když našim bližním ukazujeme svými
dobrými skutky ke světlu. A~o~těch zase Pán říká: Ať tak září vaše světlo
před lidmi, aby viděli vaše dobré skutky a~oslavovali vašeho Otce, který
je v~nebesích. Přikazuje se tedy dvojí: opásat si bedra a~uchopit lampu,
aby se tak zaskvěla čistota těla a~světlo pravdy ve skutcích. Našemu
Vykupiteli se jedno bez druhého nemůže zalíbit, to jest, kdyby ně\-kdo, kdo
koná dobro, neodložil poskvrny rozmařilosti, nebo kdyby ně\-kdo, kdo vyniká
čistotou s~necvičil v~dobrých skutcích. Čistota bez dobrých skutků není ničím
velkým, a~dobré skutky jsou bez čistoty ničím. Ale i~když obojí činíme,
je ještě třeba, aby každý směřoval k~nebeské vlasti a~neodříkal se neřestí
jenom pro čest a~dobrou pověst v~tomto světě. A~i~když ně\-kdo třeba začne
konat dobro jen pro světskou čest. nemá u~tohoto úmyslu vytrvávat a~pro
své dobré skutky nehledat slávu v~tomto světě, ale veš\-ke\-rou svou naději
vložit do příchodu svého Vykupitele. Proto se také hned dodává: A~buďte jako
lidé, kteří očekávají svého pána, až se vrátí ze svatby. Náš Pán se totiž
odebral na svatbu, neboť vstal z~mrtvých, vstoupil do nebe a~jako nový
člověk se spojil s~nadnebeskými zástupy andělů. A~přijde tehdy, až se nám
zjeví ve svém soudu.}}

\newcommand{\trMatRespIX}{\translatioCantus{Když svatý otec~\grestar{}
již po čtyřicet dní trpěl těžkou nemocí \ddag{}
v~požehnaném stáří, osvobozen od roboty tohoto světa~\grestar{}
nechal duši naplněnou zásluhami přilnout k~blaženému nepomíjejícímu dobru. \Vbardot{}
Když dovršil devadesát pět let svého svatého věku~\grestar{} nechal…}}

\newcommand{\trMatBenedictioX}{\translatioCantus{Rač, pane, požehnat.
Nechť nás požehná,~\grestar{} jenž bez konce žíti a~vládnout zná. \Rbardot{} Amen.}}

\newcommand{\trMatLecX}{\translatioCantus{
A~správně se připojuje o~služebnících očekávajících \textit{(svého pána)}:
,,Aby mu otevřeli hned jak přijde a~zabuší.`` Pán vskutku totiž přichází, když spěchá
k~soudu a~tluče, když obtížná nemoc oznamuje, že smrt je blízko. A~hned mu
otevřeme, když ho přijmeme s~láskou. A~tlukoucímu soudci otevřít nechce ten,
kdo se děsí odchodu z~těla a~děsí se spatřit toho, v~kom si pamatuje svého soudce.
Kdo si však je jist svou nadějí a~svými skutky, hned tlukoucímu otevře,
aby radostně soudce přijal; a~když pozná čas blížící se smrti, rozveselí se ze slavné odměny.
Proto se dále dodává: Blažení jsou ti služebníci, jež nalezne při svém příchodu pán bdící.
Bdí ten, kdo má oči otevřené ke zření pravého světla mysli, bdí ten, kdo svými činy uchovává
to, čemu věří, bdí ten, kdo od sebe odhání temnoty malátnosti a~nedbalosti.
Proto Pavel říká: Procitněte spravedliví a~nehřešte. A~proto také říká:
Již je hodina povstat ze spánku. A~poslyšme co prokáže pán, až přijde,
bdělým služebníkům: Přepáše se, tj. připraví k~odměně; rozesadí je, to znamená
dá jim procitnout ve věčném míru. Toto naše usednutí znamená spočinutí
v~království. Proto Pán opět říká: ,,Přijdou a~usednou ke stolu s~Abrahámem,
Izákem a~Jakobem.``
A~přecházeje pak Pán slouží, neboť nás sytí světlem své jasnosti.
,,Přecházeje`` se také praví proto, že přechází od soudu k~vládě. Jistě nás
Pán po soudu obejde, neboť nás pozdvihne z~lidské podoby ke zření jeho božství.
A~toto jeho obcházení znamená, že nás povede ke zření svého jasu, abychom tak toho,
jehož jsme uzřeli na soudu v~lidské podobě, také viděli po soudu
v~podobě božské. A~ovšemže když přijde k~soudu, zjeví se všem v~lidské
podobě, neboť je psáno: ,Uzří toho, kterého probodli.`` A~když zavržení
propukají v~nářek, spravedliví jsou přenášeni k~jasu slávy, jak je psáno:
,,Nechť je vymýcen bezbožný, aby neviděl slávu Boží``.}}

\newcommand{\trMatRespX}{\translatioCantus{Bože, pro lásku k~Tobě tento
světec opustil svoji vlast~\grestar{} učiň, ať tito nezkrotní koně přenesou jeho tělo
na místo, jež tvá vůle předurčila jeho zásluhám \Vbardot{}
Vždyť on pro velebnost Tvé moci je všude celý.~\grestar{} Učiň…}}

\newcommand{\trMatBenedictioXI}{\translatioCantus{Rač, pane, požehnat.
Když slavíme jeho svátek, nechť se za nás přimluví u~Boha. \Rbardot{} Amen.}}

\newcommand{\trMatLecXI}{\translatioCantus{
Ale co když se služebníci projeví jako nedbalí hned za první hlídky?
První hlídkou je ostražitost v~mladém věku. Ale i~tak si nemáme zoufat a~u\-pus\-tit
od dobrých skutků, protože Pán v~narážce na svou trpělivost dodává: Kdyby
přišel o~druhé hlídce nebo o~třetí, a~nalezl je tak, blažení jsou ti služebníci.
První hlídkou jsou první léta, dětství. Druhou je mládí čili jinošství,
což podle vlastního učení Písma je totéž, neboť Šalamoun praví:
Raduj se mladíku ve svém jinošství. A~za třetí hlídku se považuje stáří. Kdo
tedy nechtěl bdít o~první hlídce, ať ostříhá buď druhou - jako když ten,
kdo se opomněl odvrátit od nepravosti v~dětství, procitne na cestu života
až v~čase mládí. A~kdo pak nechtěl procitnout za druhé hlídky, nepohrdej
nápravou za hlídky třetí jakožto ten, kdo neprocitne na cestu života v~mládí,
ale ve stáří se upamatuje. Uvažte, bratři nejmilejší, jak odsuzuje
Boží spravedlnost naši zatvrzelost. Žádný člověk již nemá vý\-mlu\-vu. Bůh
pohlíží dolů a~čeká; vidí, jak naň nedbají a~volá všechny zpět; přijímá
nespravedlivé pohrdání a~přesto těm, kdo se navrátí slibuje odměnu. Avšak ať
nikdo nepromarní tuto jeho shovívavost, neboť tak vyvolá jen tím přísnější
spravedlivý soud, čím déle se dovolával před soudem trpělivosti. Proto
říká Pavel: Nevíš, že tě milosrdenství Boží vede k~pokání? Ty však svou
zatvrzelostí a~nekajícím srdcem střádáš hněv pro den hněvu a~zjevení
spravedlivého Božího soudu. A~proto také říká žalmista: Bůh je soudce
spravedlivý, silný a~shovívavý. Když mluví o~shovívavosti, předsunuje před ni
spravedlivost, abys věděl, když ho vidíš trpělivě snášet hříchy provinilců,
že také někdy přísně soudí. Proto také kdosi moudrý říká: Nejvyšší odplácí
s~trpělivostí. Říká se, že odplácí s~trpělivostí, poněvadž lidské hříchy
na jedné straně trpí, na druhé odplácí. Neboť když se neobrátí, kterým dlouho
shovívá pro jejich obrácení, tím tvrději je trestá. A~aby od nás zapudil zahálku,
přivádí na nás vnější pohromy, aby byl skrze ně duch probuzen
a~byl na stráži. Praví se totiž: Vězte, kdyby věděl hospodář v~kterou hodinu
přijde zloděj, jistě by bděl a~nenechal by ho prokopat se do domu. A~z~tohoto
mimochodem podobenství vyvozuje poučení, když říká: I~vy buďte připraveni,
neboť Syn člověka přijde v~hodinu, v~kterou nemyslíte. Zloděj se
vloupá do domu, když to hospodář neví, neboť když duch upustí od své bdělosti,
přijde nečekaná smrt a~vtrhne do příbytku našeho těla, a~toho, koho
pán zastihne v~domě spícího, zahubí, neboť když duch nejméně předpokládá,
že přijde zkáza, tehdy ho nevědomky smrt zachvátí, ani nevzdechne.
Zloději by se ubránil, kdyby bděl, protože by čekal dopředu příchod pána,
který tajně uchvacuje duši, a~předcházel by jej kajícností, aby nezahynul
v~nekajícnosti.}}

\newcommand{\trMatRespXI}{\translatioCantus{Jistý chudák,~\grestar{}
který se mezi jinými chorobami také nemohl postavit na nohy, jakmile si obul boty muže
božího a~zavázal si je řekl, že~\grestar{} byl uzdraven ve všech svých kloubech. \Vbardot{}
Když odcházel, velebil Boha a~svatého Havla, pro jehož zásluhy~\grestar{}
byl…}}

\newcommand{\trMatBenedictioXII}{\translatioCantus{Rač, pane, požehnat.
Do společnosti občanů nebes~\grestar{} ať nás dovede král andělů.
\Rbardot{} Amen.}}

\newcommand{\trMatLecXII}{\translatioCantus{Proto chtěl Pán, abychom neznali
svou poslední hodinu, abychom se ji vždy obávali, a~když ji nemůžeme znát
dopředu, abychom se na ni neustále připravovali. Proto popatřete, moji bratři,
na svou smrtelnost a~každodenně se s~pláčem připravujete na přicházejícího
soudce. A~ježto smrt je jistá pro všechny, nepomýšlejte na nejistou časnou
opatrnost. Nechť vás netíží starosti o~pozemské věci. I~kdybyste tělo oděli
do samého zlata a~stříbra a~do jemných látek, nezůstalo by přesto tělem?
Nedbejte proto na to, co máte, ale co jste. A~chcete slyšet, co jste?
Prorok vyhlašuje: Lid je tráva. Není-li lid tráva, kde jsou pak ti, kteří
před rokem s~námi slavili to, co slavíme dnes, totiž narození svatého Havla?
Jak a~jak mnoho přemítali a~zajištění svého života, ale hned, jakmile na ně
udeřila smrt, se octli tam, kde to neočekávali, a~opustili vše časné, co se
domnívali mít nastřádáno a~pevně pod kontrolou? Jestliže tedy mnohá lidská
pokolení se narozením rozvíjí v~tělo, smrtí pak padají v~prach, jsou trávou.
Nuže, když tak totiž unikají před svou hodinou, ať, milí bratři, obdrží odměnu za dobré
dílo. Slyšte, co říká Šalamoun: Neustále konej, cokoli je tvá ruka schopna
vykonat, poněvadž v~podsvětí, kam spěšně míříš, nebude ani díla, ani vědění,
ani rozumu ani moudrosti. Poněvadž tedy neznáme čas, v~němž umřeme, a~po
smrti nemůžeme nic udělat, zbývá, abychom využili čas vymezený nám před
smrtí. Tak bude přemožena smrt, až přijde, jestliže se jí budeme obávat,
než přijde.}}

\newcommand{\trMatRespXII}{\translatioCantus{Tento světec,~\grestar{}
jenž přešel do andělské radosti se právem vrací do mysli lidí~\grestar{}
neboť putuje zde jen podle těla, myslí dychtivě obcoval s~onou věčnou
vlastí. \Vbardot{} Když byl uvolněn z~tělesných pout přinesl svému Pánu
dvojnásobek svěřeného talentu.~\grestar{} Neboť…}}

% from the Czech Liturgia horarum
\newcommand{\trTeDeum}{\begin{translatioMulticol}{3}

Bože, tebe chválíme, 
tebe, Pane, velebíme.

Tebe, věčný Otče, 
oslavuje celá země.

Všichni andělé, 
cherubové i~serafové,

všechny mocné nebeské zástupy 
bez ustání volají:

Svatý, Svatý, Svatý, 
Pán, Bůh zástupů.

Plná jsou nebesa i~země 
tvé vznešené slávy.

Oslavuje tě 
sbor tvých apoštolů,

chválí tě 
velký počet proroků,

vydává o~tobě svědectví 
zástup mučedníků;

a~po celém světě 
vyznává tě tvá církev:

neskonale velebný, 
všemohoucí Otče,

úctyhodný Synu Boží, 
pravý a~jediný,

božský Utěšiteli, 
Duchu svatý.

Kriste, Králi slávy, 
tys od věků Syn Boha Otce;

abys člověka vykoupil, 
stal ses člověkem a~narodil ses z~Panny;

zlomil jsi osten smrti 
a~otevřel věřícím nebe;

sedíš po Otcově pravici 
a~máš účast na jeho slávě.

Věříme, že přijdeš soudit, 

a~proto tě prosíme:
přispěj na pomoc svým služebníkům, 
vždyť jsi je vykoupil svou předrahou krví;

dej, ať se radují s~tvými svatými 
ve věčné slávě.

Zachraň, Pane, svůj lid, žehnej svému dědictví, 
veď ho a~stále pozvedej.

Každý den tě budeme velebit 
a~chválit tvé jméno po všechny věky.

Pomáhej nám i~dnes, 
ať se nedostaneme do područí hříchu.

Smiluj se nad námi, Pane, 
smiluj se nad námi.

Ať spočine na nás tvé milosrdenství, 
jak doufáme v~tebe.

Pane, k~tobě se utíkáme, 
ať nejsme zahanbeni na věky. 
\end{translatioMulticol}}

\newcommand{\trMatEvangelium}{\translatioCantus{
Mějte opásaná bedra a~vaše lampy ať hoří.
Buďte jako lidé, kteří očekávají svého pána, když se vrací ze svatby,
aby mu otevřeli, jakmile přijde a~zaklepe. Blahoslavení ti služebníci, které
pán při svém příchodu najde, jak bdí! Vpravdě vám říkám, opáše se, usadí je
ke stolu, bude přecházet od jednoho ke druhému a~bude je obsluhovat. Když
přijde za druhé či třetí hlídky a~nalezne to tak, budou blahoslavení!
Uvědomte si tohle: kdyby hospodář věděl, v~kterou hodinu má přijít zloděj,
nenechal by ho prorazit zeď svého domu. I~vy buďte připraveni, neboť Syn
člověka přijde ve chvíli, kdy se toho nenadějete.}}

\newcommand{\trTeDecetLaus}{\translatioCantus{Tobě chvála, Tobě zpěvy, Tobě
sláva, Bohu Otci i~Synu i~Svatému Duchu, na věky věků. \Rbardot{} Amen.}}

% MASS ---

\newcommand{\trIntroitus}{\translatioCantus{Ústa spravedlivého~\grestar{}
šeptají moudrost, a~jeho jazyk ohlašuje právo; zákon svého Boha má v~srdci.
\textit{\color{red}Žl.} Nerozhořčuj se na ničemy, nežářli na strůjce klamu.}}

\newcommand{\trGraduale}{\translatioCantus{Ústa spravedlivého~\grestar{}
šeptají moudrost, a~jeho jazyk ohlašuje právo.
\Vbardot{} Zákon svého Boha má v~srdci, jeho kroky nekolísají.}}

\newcommand{\trAlleluia}{\translatioCantus{Aleluja. \Vbardot{} Blahoslavený
muž, který snáší zkoušku! Až se osvědčí, obdrží věnec života.}}

\newcommand{\trSequentia}{\translatioCantus{
Havle, miláčku Boží
i~lidí a andělských sborů,
jenž jsi poslechl neustálé horlivé nabádání Ježíše Krista
a~pohrdl jsi otcovskými statky, mateřským klínem,
manželskými starostmi i~dětskými hrami
a~dal kříži přednost před vrtkavou radostí.
Však Kristus to odměňuje odměnou stonásobnou
jak svědčí tento den, kdy ti nás všechny poddává ve sladké lásce jako syny.
On ti dal za vlast sladké Švábsko a usadil tě jako soudce v~nebesích spolu
se sborem apoštolů.
Tebe nyní Havle pokorně prosíme,
abys o~milost požádal Ježíše Krista
a~mírem naplnil celé jeho tělo po všech místech
a~podpořil častou přímluvou své prosebníky,
když ti neustále můžeme radostně prokazovat náležitou úctu,
ó~Havle Bohu milý.}}

\newcommand{\trOffertorium}{\translatioCantus{Dopřál jsi mu, po čem toužilo jeho srdce,~\grestar{}
neodmítl jsi přání jeho rtů.~\gredagger{}
Korunu z~ryzího zlata na hlavu jsi mu vsadil.
\Vbardot{} {\color{red}\textit{1.}} Dopřál jsi mu život, o~nějž žádal, Hospodine.
{\color{red}\textit{2.}} Těšíš ho štěstím blízko své tváře.
{\color{red}\textit{3.}} Tvá ruka najde všechny tvé protivníky, tvá pravice najde tvé nepřátele.}}

\newcommand{\trCommunio}{\translatioCantus{Proto vám říkám:~\grestar{}
věřte, že jste už dostali vše, oč v~modlitbě žádáte, a~bude vám to dáno.}}

% LITTLE HOURS ---

\newcommand{\trAntTertia}{\translatioCantus{Když Jan otevřel rakev, prolil
přehořké slzy a~pravil: Ach, běda, milovaný otče, běda učiteli zástupů, proč
jsi mě zde zanechal jako sirotka daleko od otcovského domu.}}

\newcommand{\trVersusTertia}{\translatioCantus{\Vbardot{} Bůh si ho
zamiloval. \Rbardot{} A~oblékl mu roucho slávy.}}

\newcommand{\trCapituliJustus}{\translatioCantus{Spravedlivý se od rána
celým svým srdcem obrací k~Pánu, svému stvořiteli;~\grestar{}
úpěnlivě prosí před Nejvyšším.}}

\newcommand{\trVersusSexta}{\translatioCantus{\Vbardot{} Ústa spravedlivého šeptají moudrost.
\Rbardot{} A~jeho jazyk ohlašuje právo.}}

\newcommand{\trCapituliJustum}{\translatioCantus{Spravedlivého vodil Hospodin po přímých stezkách,~\gredagger{}
a~ukázal mu Boží království a~dal mu poznání svatých věcí;~\grestar{}
dal mu úspěch v~jeho tvrdých pracích a~dopřál výnos jeho námaze.}}

\newcommand{\trVersusNona}{\translatioCantus{\Vbardot{} Zákon svého Boha má
v~srdci.
\Rbardot{} Jeho kroky nekolísají.}}
