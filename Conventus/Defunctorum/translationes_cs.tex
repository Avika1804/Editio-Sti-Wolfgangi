%%%% Preklady jednotlivych zpevu (nektere se opakuji, a je dobre mit je
% vsechny na jedne hromade)

\newcommand{\trOratioAnteOfficium}{\translatioCantus{Otevři, Pane, má ústa, abych chválil tvé svaté jméno.
Očisti mé srdce od všech marnivých, zvrácených a~jiných myšlenek, osvěť rozum, rozněť cit,
abych mohl důstojně, soustředěně a~zbožně recitovat a~vysloužil si být
vyslyšen před tváří tvé velebnosti. Skrze Krista…}}

\newcommand{\trOratioPostOfficium}{\translatioCantus{\textit{Následující modlitbu
opatřil pro ty, kdo ji zbožně vyřknou po hodinkách, papež Lev X.
odpustky za nedostatky a provinění vzniklé při konání hodinek z~lidské křehkosti. Říká se
vkleče.}
Svatosvaté a~nerozdílné Trojici, ukřižovanému lidství našeho Pána Ježíše
Krista, přeblažené a~přeslavné plodné neporušenosti vždy Panny Marie
i~souhrnu všech svatých buď ode všeho stvoření věčná chvála, čest a~sláva, nám
pak buď dáno odpuštění všech hříchů, po nekonečné věky věků. Amen.}}

% HOURS ---

\newcommand{\trVespAntI}{\translatioCantus{Před Hospodinem smím chodit~\grestar{}
na zemi mezi živými.}}

\newcommand{\trVespAntII}{\translatioCantus{Běda mně,~\grestar{} pobyt můj v cizině se
prodloužil!}}

\newcommand{\trVespAntIII}{\translatioCantus{Hospodin~\grestar{} střeží tě všeho
zlého, střeží tvůj život Hospodin.}}

\newcommand{\trVespAntIV}{\translatioCantus{Budeš-li, Pane, hříchy mít na
zřeteli, Hospodine, kdož obstojí?}}

\newcommand{\trVespAntV}{\translatioCantus{Díly~\grestar{} svých rukou, Pane, nepohrdneš!}}

\newcommand{\trRespVesp}{\translatioCantus{Ústa spravedlivého~\grestar{}
šeptají moudrost. \Vbardot{} A~jeho jazyk ohlašuje právo.}}

\newcommand{\trRespLaud}{\translatioCantus{Spravedlivého vodil Hospodin~\grestar{}
po přímých stezkách. \Vbardot{} A~ukázal mu Boží království.}}

\newcommand{\trVersusAudivi}{\translatioCantus{\Vbardot{} Potom jsem uslyšel, jak
mi jakýsi hlas z nebe říká. ~\Rbardot{}: Blahoslavení mrtví, kteří umírají v Pánu.}}

\newcommand{\trAntMagnificat}{\translatioCantus{Všechno,~\grestar{} co mi dává Otec,
přijde ke mně, a toho, kdo ke mně přichází, nevyvrhnu ven.}}

\newcommand{\trAntBenedictus}{\translatioCantus{Když na bujné oře vložili
nosítka a~sňali jim uzdu, vydali se přímo k~cele božího muže.}}

\newcommand{\trAntMagnificatII}{\translatioCantus{Právem se vrací na mysl
lidí tento světec, jenž přešel do radosti andělské, neboť za tohoto putování
zde byl pouze tělem, ale svou myslí již nedočkavě obcoval s~věčnou vlastí.}}

\newcommand{\trOrationis}{\translatioCantus{Bože, jenž nám dopřáváš radovat
se z~výroční slavnosti svatého tvého vyznavače Havla, uděl dobrotivě,
abychom když slavíme jeho narození, též se řídili podobou jeho skutků.
Skrze…}}

\newcommand{\trFideliumAnimae}{\translatioCantus{\Vbardot{} Duše věrných ať pro
milosrdenství Boží odpočívají v~pokoji. \Rbardot{} Amen.}}

% Completorium

\newcommand{\trJubeDomne}{\translatioCantus{Rač, pane, požehnat.}}

\newcommand{\trComplBenedictio}{\translatioCantus{Pokojnou noc a~svatou smrt
nechť nám dopřeje všemohoucí Pán. \Rbardot{} Amen.}}

\newcommand{\trComplLectioBr}{\translatioCantus{Buďte střízliví, bděte.
Váš protivník Ďábel obchází jako lev řvoucí a~hledá, koho by pohltil.
Postavte se proti němu pevní ve víře.  Ale ty, Pane, smiluj se nad námi.
\Rbardot{} Bohu díky.}}

\newcommand{\trComplAntI}{\translatioCantus{Rač se smilovati nade mnou,
Hospodine, a~vyslyš mou modlitbu.}}

\newcommand{\trComplCapituli}{\translatioCantus{Jsi přece, Hospodine,
uprostřed nás a~jmenujeme se po tobě.  Neopouštěj nás, Pane, náš Bože.}}

\newcommand{\trRespCompl}{\translatioCantus{Do tvých rukou, Pane, \grestar{}
poroučím svého ducha. \Vbardot{} Ty mne zachráníš, Pane, Bože věrný.}}

\newcommand{\trComplVersus}{\translatioCantus{\Vbardot{} Střez mne jako zřítelnici oka,
aleluja. \Rbardot{} Ve stínu svých křídel uschovej mne, aleluja.}}

\newcommand{\trAntSalvaNos}{\translatioCantus{Ochraňuj nás, Pane, když
bdíme, a~buď s~námi, když spíme, ať bdíme s~Kristem a~odpočíváme v~pokoji.}}

\newcommand{\trComplOrationis}{\translatioCantus{Zavítej, prosíme, Pane, sem
do našeho příbytku a~daleko od něj zažeň všechny úklady nepřítele. Ať tu
bydlí tví svatí andělé a~tvoje požehnání buď nad ním stále. Skrze…}}

\newcommand{\trSalveRegina}{\translatioCantus{Zdrávas Královno, \grestar{} matko
milosrdenství, živote, sladkosti a~naděje naše, buď zdráva!
K~tobě voláme, vyhnaní synové Evy,
k~tobě vzdycháme, lkajíce a~plačíce
v~tomto slzavém údolí.
A~proto, orodovnice naše,
obrať k~nám své milosrdné oči
a~Ježíše, požehnaný plod života svého,
nám po tomto putování ukaž,
ó milostivá, ó přívětivá,
ó přesladká, \grestar{} Panno Maria!}}

\newcommand{\trOraProNobis}{\translatioCantus{\Vbardot{} 
Oroduj za nás, svatá Boží Rodičko,
\Rbardot{} aby nám Kristus dal účast na svých zaslíbeních.}}

% Matutinum

\newcommand{\trMatInvitatorium}{\translatioCantus{}}

\newcommand{\trMatVeniteA}{\translatioCantus{Pojďte, chvalme s~radostí Pána,
s~jásotem slavme Boha, svou spásu; předstupme před tvář jeho s~díky, písně plesu pějme jemu.}}

\newcommand{\trMatVeniteB}{\translatioCantus{Neboť Bůh veliký jest Hospodin, a~král nade všecky bohy.
Jsouť v~jeho ruce všecky hlubiny země, temena hor jsou majetek jeho.}}

\newcommand{\trMatVeniteC}{\translatioCantus{Jehoť jest moře, neb on je učinil; i~souš
je dílo jeho rukou. Pojďme, klanějme se, padněme, klekněme před Pánem, svým
tvůrcem. Jeť on Pán, náš Bůh, a~my jsme lid, jejž on vodí a~ovce, jež pase.}}

\newcommand{\trMatVeniteD}{\translatioCantus{Kéž byste poslechli dnes hlasu jeho:
,,Nezatvrzujte svých srdcí jak v~Hádce, jak v~Pokušení na poušti, kde vaši otcové pokoušeli mne,
zkoušeli mne, ač vídali skutky mé.``}}

\newcommand{\trMatVeniteE}{\translatioCantus{Čtyřicet roků mrzel jsem se na to pokolení
a~řekl jsem: ,,Lid je to myslí stále bloudící``! Oni však nechtěli znáti mé cesty, takže jsem
přisáhl ve svém hněvu: ,,Nedojdou odpočinku mého!\mbox{}``}}

\newcommand{\trMatAntI}{\translatioCantus{Urovnej,~\grestar{} Pane, můj Bože, před sebou mou cestu.}}

\newcommand{\trMatAntII}{\translatioCantus{Obrať se,~\grestar{} Pane, a vytrhni život
můj, neboť, kdo ve smrti je tebe pamětliv.}}

\newcommand{\trMatAntIII}{\translatioCantus{Aby jak lvi mne neroztrhali,
když není, kdo by mne vytrhl, zachránil.}}

\newcommand{\trMatVersusI}{\translatioCantus{}}

\newcommand{\trMatLecI}{\translatioCantus{Stravuji se, nebudu žít pořád;
a~tak mě nech, můj život je pouhé dechnutí! Co tedy je člověk, že mu
přikládáš takovou váhu, že na něho upínáš svou pozornost,
že na něho každého rána dohlížíš, že ho každým okamžikem zkoumáš? 
Přestaneš se na mne konečně dívat, abych měl čas polknout slinu?
Pokud jsem zhřešil, co jsem tím udělal tobě, ty bedlivý pozorovateli člověka?
Proč sis mě vzal za terč, proč jsem ti na obtíž?
Nemůžeš ode mne strpět urážku, přejít mou vinu? Vždyť brzy
budu ležet v prachu, budeš mě hledat a já už nebudu.}}

\newcommand{\trMatRespI}{\translatioCantus{Věřím,~\grestar{}
že můj Vykupitel žije, že on jako poslední povstane nad prachem.~\gredagger{}
A ve svém těle uzřím Boha, Spasitele svého. \Vbardot{}
Ten, jehož uvidím, bude na mé straně; ten, na nějž budou hledět mé oči,
nebude cizinec.}}

\newcommand{\trMatLecII}{\translatioCantus{Protože se mi oškliví život, dám
volný průchod svému nářku, vyleji hořkost své duše. 
Řeknu Bohu: Neodsuzuj mě, prozraď mi, proč mi přičítáš vinu.
Dělá ti dobře, že mi činíš násilí, že pokořuješ dílo svých
rukou a že podporuješ záměry zlovolných?}}

\newcommand{\trMatLecIIa}{\translatioCantus{Ten člověk zrozený
z~ženy, jenž má život krátký, ale trápení do sytosti.
Podobá se květu, rozkvétá, pak uvadá, bez ustání prchá jako stín.
A tuto bytost nikdy nespouštíš z očí, přivádíš ji před sebe na soud! 
Kdo však vytěží čisté z nečistého? Nikdo!
Poněvadž jeho dny jsou sečteny a počet jeho měsíců závisí od tebe a ty mu
určuješ nepřekročitelnou mez, odvrať od něho oči a nechej ho, ať jako
nádeník skončí svůj den.}}

\newcommand{\trMatRespII}{\translatioCantus{}}

\newcommand{\trMatLecIII}{\translatioCantus{Tvé ruce mě ztvárnily, utvořily;
pak sis to rozmyslel a chtěl bys mě zničit! 
Vzpomeň si: udělal jsi mě, jako se hněte hlína, a pošleš mě nazpět do prachu. 
Což jsi mě neslil jako mléko a nenechal srazit jako sýr,
neoblékl do kůže a masa, neutkal z kostí a šlach?
Pak jsi mě obdařil životem a starostlivě jsi bděl nad mým dechem.}}

\newcommand{\trMatLecIIIa}{\translatioCantus{Mé maso pod kůží propadá
hnilobě a kosti se mi obnažují jako zuby.
Slitujte se, slitujte se nade mnou, přátelé moji! Neboť mě zasáhla Boží ruka.
Proč se na mne vrháte jako sám Bůh, aniž se nasytíte mým masem? 
Ach! Přál bych si, aby má slova byla sepsána, aby byla vyryta jako nápis, 
železným dlátem a bodcem navěky vytesána do skály! 
Já vím, že můj Obhájce žije, že on jako poslední povstane nad prachem.
Po mém probuzení mě postaví vedle sebe a ve svém těle uzřím Boha.
Ten, jehož uvidím, bude na mé straně; ten, na nějž budou hledět mé oči,
nebude cizinec. A mé ledví ve mně se stravuje.}}

\newcommand{\trMatRespIII}{\translatioCantus{}}

\newcommand{\trMatAntIV}{\translatioCantus{Kde pastvy hojnost,~\grestar{}
tam lehat smím.}}

\newcommand{\trMatAntV}{\translatioCantus{Čím jsem se v mládí provinil,
nevzpomínej, Hospodine.}}

\newcommand{\trMatAntVI}{\translatioCantus{Věřím a vidím~\grestar{}
dobrotu Hospodinovu v zemi živých!}}

\newcommand{\trMatVersusII}{\translatioCantus{\Vbardot{}Posadil je mezi knížata.
\Rbardot{} Mezi knížata svého lidu.}}

\newcommand{\trMatLecIV}{\translatioCantus{Pak začni rokovat a já odpovím;
nebo spíš budu mluvit já a odpověď mi dáš ty.
Kolik jsem spáchal provinění a hříchů? Řekni mi, jaký byl můj přestupek, můj hřích?
Proč ukrýváš svou tvář a pokládáš mě za svého nepřítele? 
Chceš děsit list zmítaný větrem, pronásledovat suché stéblo? 
Že proti mně vynášíš hořké rozsudky a přičítáš mi viny mládí, 
žes mi dal nohy do klády, pozoruješ všechny mé stezky a značíš si stopy mých kroků!
A on se rozpadá jako červotočivé dřevo nebo jako šat, jejž rozežírá mol.}}

\newcommand{\trMatLecIVa}{\translatioCantus{}}

\newcommand{\trMatRespIV}{\translatioCantus{Vzpomeň na mě,~\grestar{} Bože,
že můj život je pouhé dechnutí. \gredagger{} Unikám každému pohledu.
\Vbardot{} Z hlubin volám k tobě, Hospodine: Pane, slyš můj hlas.}}

\newcommand{\trMatLecV}{\trMatLecIIa}

\newcommand{\trMatLecVa}{\translatioCantus{}}

\newcommand{\trMatRespV}{\translatioCantus{}}

\newcommand{\trMatLecVI}{\translatioCantus{Ach! Kdybys mi poskytl přístřeší v šeolu,
kdybys mě tam ukryl, dokud potrvá tvůj hněv, kdybys mi určil lhůtu, a pak si na mne vzpomněl: 
- vždyť, když už člověk zemře, může zase ožít? - Po všechny
dny své služby bych čekal, až by mě přišli vystřídat. 
Ty bys zavolal a já bych ti odpověděl; zase bys chtěl uzřít dílo svých rukou.
A zatímco teď počítáš všechny mé kroky, už bys pak nečíhal na můj hřích.}}

\newcommand{\trMatLecVIa}{\translatioCantus{}}

\newcommand{\trMatRespVI}{\translatioCantus{}}

\newcommand{\trMatAntVII}{\translatioCantus{Rač mne, Pane, vysvobodit,
hleď, abys mi přispěl.}}

\newcommand{\trMatAntVIII}{\translatioCantus{Uzdrav mne, Hospodine, neb jsem hřešil
proti tobě.}}

\newcommand{\trMatAntIX}{\translatioCantus{Žízní~\grestar{} duše má po Bohu živém.
Kdy přijdu abych se ukázal před Boží tváří?}}

\newcommand{\trMatVersusIII}{\translatioCantus{\Vbardot{} Nevydávej zvěři duši své
hrdličky. \Rbardot{} Až do konce nezapomínej na život svých nešťastných.}}

\newcommand{\trMatLecVII}{\translatioCantus{Můj dech se ve mně vyčerpává a scházejí se mí hrobaři.
Mými druhy jsou jen posměváčci, jejichž tvrdost trýzní mé probděné noci.
Polož si tedy sám před sebe mou záruku, kdo by si totiž se mnou chtěl plácnout?
Mé dny utekly i s mými záměry a struny mého srdce se strhaly. 
Z noci se chce dělat den; prý už je blízko světlo zahánějící temnoty.
Mou nadějí je bydlet v šeolu, rozprostřít si lože v temnotách.
Volám na hrob: ,,Jsi můj otec!\mbox{}`` Na červa: ,,Ty jsi má matka a má
sestra!\mbox{}``
Kde tedy je moje naděje? A mé štěstí, kdo je spatří?}}

\newcommand{\trMatLecVIIa}{\translatioCantus{Hlásá-li se tedy, že Kristus
vstal z mrtvých, jak mohou někteří mezi vámi říkat, že vzkříšení z mrtvých není?
Není-li vzkříšení z mrtvých, nevstal z mrtvých ani Kristus. 
Ale jestliže Kristus nevstal z mrtvých, pak je prázdné naše poselství,
prázdná je též vaše víra. 
Ukazuje se pak dokonce, že o Bohu svědčíme falešně, protože jsme proti Bohu tvrdili,
že on vzkřísil Krista, zatímco ho nevzkřísil, je-li pravda, že mrtví nevstávají. 
Vždyť nevstávají-li mrtví, ani Kristus z mrtvých nevstal.
A jestliže Kristus nevstal z mrtvých, marná je vaše víra;
jste dosud ve svých hříších. 
Pak také ti, kdo usnuli v Kristu, zahynuli.
Jestliže jsme vložili svou naději v Krista jen pro tento život
jsme nejvíc politováníhodní ze všech lidí.
Ale ne: Kristus vstal z mrtvých jako prvotina těch, kdo usnuli. 
Protože totiž smrt přišla skrze člověka, přichází skrze člověka také zmrtvýchvstání.
Jako totiž všichni umírají v Adamovi, tak všichni budou opět oživeni v Kristu.}}

\newcommand{\trMatRespVII}{\translatioCantus{}}

\newcommand{\trMatLecVIII}{\trMatLecIIIa}

\newcommand{\trMatLecVIIIa}{\translatioCantus{Řekne se však, jak vstávají mrtví?
S jakým tělem zase přijdou?
Blázne! Co ty seješ, nenabývá opět života, pokud to nezemře.
A co seješ, není tělo, jež má přijít, ale pouze semeno, buď pšenice anebo nějaké jiné rostliny; 
a Bůh mu dává tělo, jak sám chtěl, každému semeni vlastní tělo. 
Všechna těla nejsou stejná, leč jiné je tělo lidí, jiné tělo zvířat, jiné tělo ptáků, jiné ryb.
Jsou také nebeská tělesa a tělesa pozemská, ale jinak září ta nebeská, jinak pozemská.
Jinak září slunce, jinak září měsíc, jinak září hvězdy. V záření se dokonce liší hvězda od hvězdy.
Tak tomu je se vzkříšením mrtvých: zasévá se v porušenosti, z mrtvých se vstává v neporušitelnosti;
zasévá se v hanbě, z mrtvých se vstává ve slávě; zasévá se ve slabosti, z mrtvých se vstává v síle;
zasévá se tělo obdařené duší, z mrtvých vstává tělo duchovní.}}

\newcommand{\trMatRespVIII}{\translatioCantus{}}

\newcommand{\trMatLecIX}{\translatioCantus{Ach! Proč jsi mi dal vyjít
z lůna? Tehdy bych byl zahynul: žádné oko by mě neuvidělo, 
byl bych, jako bych nikdy nebyl, z břicha by mě byli přenesli do hrobu.
A dny mého žití trvají tak krátce! Už mě tedy nepozoruj a popřej mi trochu radosti,
než nenávratně odejdu do země temnot a hustého stínu, 
kde vládne tma a nepořádek, kde se sám jas podobá tmavé noci.}}

\newcommand{\trMatLecIXa}{\translatioCantus{Hle, řeknu vám jedno tajemství:
nezemřeme všichni, ale všichni budeme proměněni. 
V jedné chvilce, v jednom okamžení, za zvuku poslední polnice, neboť ta polnice zazní,
a mrtví vstanou neporušitelní a my budeme proměněni. 
Je totiž nutné, aby ta porušitelná bytost oblékla neporušitelnost, aby ta smrtelná bytost oblékla nesmrtelnost. 
Až tedy ta porušitelná bytost obleče neporušitelnost
a až ta smrtelná bytost obleče nesmrtelnost, pak se naplní slovo, které je psáno:
\textit{Smrt byla pohlcena ve vítězství. 
Kde je, smrti, tvé vítězství? Kde je, smrti, tvůj bodec?}
Bodcem smrti je hřích a silou hříchu je Zákon. 
Ale buď dík Bohu, který nám dává vítězství skrze našeho Pána Ježíše Krista! 
Tak tedy, moji milovaní bratři, buďte pevní, nezviklatelní,
stále čiňte pokroky v díle Páně vědouce, že v Pánu vaše námaha nebude marná.}} 

\newcommand{\trMatRespIX}{\translatioCantus{}}

\newcommand{\trMatRespX}{\translatioCantus{}}

% MASS ---

\newcommand{\trIntroitus}{\translatioCantus{Ústa spravedlivého~\grestar{}
šeptají moudrost, a~jeho jazyk ohlašuje právo; zákon svého Boha má v~srdci.
\textit{\color{red}Žl.} Nerozhořčuj se na ničemy, nežářli na strůjce klamu.}}

\newcommand{\trGraduale}{\translatioCantus{Ústa spravedlivého~\grestar{}
šeptají moudrost, a~jeho jazyk ohlašuje právo.
\Vbardot{} Zákon svého Boha má v~srdci, jeho kroky nekolísají.}}

\newcommand{\trAlleluia}{\translatioCantus{Aleluja. \Vbardot{} Blahoslavený
muž, který snáší zkoušku! Až se osvědčí, obdrží věnec života.}}

\newcommand{\trSequentia}{\translatioCantus{
Havle, miláčku Boží
i~lidí a andělských sborů,
jenž jsi poslechl neustálé horlivé nabádání Ježíše Krista
a~pohrdl jsi otcovskými statky, mateřským klínem,
manželskými starostmi i~dětskými hrami
a~dal kříži přednost před vrtkavou radostí.
Však Kristus to odměňuje odměnou stonásobnou
jak svědčí tento den, kdy ti nás všechny poddává ve sladké lásce jako syny.
On ti dal za vlast sladké Švábsko a usadil tě jako soudce v~nebesích spolu
se sborem apoštolů.
Tebe nyní Havle pokorně prosíme,
abys o~milost požádal Ježíše Krista
a~mírem naplnil celé jeho tělo po všech místech
a~podpořil častou přímluvou své prosebníky,
když ti neustále můžeme radostně prokazovat náležitou úctu,
ó~Havle Bohu milý.}}

\newcommand{\trOffertorium}{\translatioCantus{Dopřál jsi mu, po čem toužilo jeho srdce,~\grestar{}
neodmítl jsi přání jeho rtů.~\gredagger{}
Korunu z~ryzího zlata na hlavu jsi mu vsadil.
\Vbardot{} {\color{red}\textit{1.}} Dopřál jsi mu život, o~nějž žádal, Hospodine.
{\color{red}\textit{2.}} Těšíš ho štěstím blízko své tváře.
{\color{red}\textit{3.}} Tvá ruka najde všechny tvé protivníky, tvá pravice najde tvé nepřátele.}}

\newcommand{\trCommunio}{\translatioCantus{Proto vám říkám:~\grestar{}
věřte, že jste už dostali vše, oč v~modlitbě žádáte, a~bude vám to dáno.}}

% LITTLE HOURS ---

\newcommand{\trAntTertia}{\translatioCantus{Když Jan otevřel rakev, prolil
přehořké slzy a~pravil: Ach, běda, milovaný otče, běda učiteli zástupů, proč
jsi mě zde zanechal jako sirotka daleko od otcovského domu.}}

\newcommand{\trVersusTertia}{\translatioCantus{\Vbardot{} Bůh si ho
zamiloval. \Rbardot{} A~oblékl mu roucho slávy.}}

\newcommand{\trCapituliJustus}{\translatioCantus{Spravedlivý se od rána
celým svým srdcem obrací k~Pánu, svému stvořiteli;~\grestar{}
úpěnlivě prosí před Nejvyšším.}}

\newcommand{\trVersusSexta}{\translatioCantus{\Vbardot{} Ústa spravedlivého šeptají moudrost.
\Rbardot{} A~jeho jazyk ohlašuje právo.}}

\newcommand{\trCapituliJustum}{\translatioCantus{Spravedlivého vodil Hospodin po přímých stezkách,~\gredagger{}
a~ukázal mu Boží království a~dal mu poznání svatých věcí;~\grestar{}
dal mu úspěch v~jeho tvrdých pracích a~dopřál výnos jeho námaze.}}

\newcommand{\trVersusNona}{\translatioCantus{\Vbardot{} Zákon svého Boha má
v~srdci.
\Rbardot{} Jeho kroky nekolísají.}}
