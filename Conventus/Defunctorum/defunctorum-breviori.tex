% LuaLaTeX

\documentclass[a4paper, twoside, 12pt]{article}
\usepackage[latin]{babel}
%\usepackage[landscape, left=3cm, right=1.5cm, top=2cm, bottom=1cm]{geometry} % okraje stranky
%\usepackage[landscape, a4paper, mag=1166, truedimen, left=2cm, right=1.5cm, top=1.6cm, bottom=0.95cm]{geometry} % okraje stranky
\usepackage[landscape, a4paper, mag=1400, truedimen, left=0.5cm, right=0.5cm, top=0.5cm, bottom=0.5cm]{geometry} % okraje stranky

\usepackage{fontspec}
\setmainfont[FeatureFile={junicode.fea}, Ligatures={Common, TeX}, RawFeature=+fixi]{Junicode}
%\setmainfont{Junicode}

% shortcut for Junicode without ligatures (for the Czech texts)
\newfontfamily\nlfont[FeatureFile={junicode.fea}, Ligatures={Common, TeX}, RawFeature=+fixi]{Junicode}

% Hebrew font:
% http://scripts.sil.org/cms/scripts/page.php?site_id=nrsi&id=SILHebrUnic2
\newfontfamily\hebfont[Scale=1]{Ezra SIL}

\usepackage{multicol}
\usepackage{color}
\usepackage{lettrine}
\usepackage{fancyhdr}

% usual packages loading:
\usepackage{luatextra}
\usepackage{graphicx} % support the \includegraphics command and options
\usepackage{gregoriotex} % for gregorio score inclusion
\usepackage{gregoriosyms}
\usepackage{wrapfig} % figures wrapped by the text
\usepackage{parcolumns}
\usepackage[contents={},opacity=1,scale=1,color=black]{background}
\usepackage{tikzpagenodes}
\usepackage{calc}
\usepackage{longtable}
\usetikzlibrary{calc}

\setlength{\headheight}{14.5pt}

% Commands used to produce a typical "Conventus" booklet

\newenvironment{titulusOfficii}{\begin{center}}{\end{center}}
\newcommand{\dies}[1]{#1

}
\newcommand{\nomenFesti}[1]{\textbf{\Large #1}

}
\newcommand{\celebratio}[1]{#1

}

\newcommand{\hora}[1]{%
\vspace{0.5cm}{\large \textbf{#1}}

\fancyhead[LE]{\thepage\ / #1}
\fancyhead[RO]{#1 / \thepage}
\addcontentsline{toc}{subsection}{#1}
}

% larger unit than a hora
\newcommand{\divisio}[1]{%
\begin{center}
{\Large \textsc{#1}}
\end{center}
\fancyhead[CO,CE]{#1}
\addcontentsline{toc}{section}{#1}
}

% a part of a hora, larger than pars
\newcommand{\subhora}[1]{
\begin{center}
{\large \textit{#1}}
\end{center}
%\fancyhead[CO,CE]{#1}
\addcontentsline{toc}{subsubsection}{#1}
}

% rubricated inline text
\newcommand{\rubricatum}[1]{\textit{#1}}

% standalone rubric
\newcommand{\rubrica}[1]{\vspace{3mm}\rubricatum{#1}}

\newcommand{\notitia}[1]{\textcolor{red}{#1}}

\newcommand{\scriptura}[1]{\hfill \small\textit{#1}}

\newcommand{\translatioCantus}[1]{\vspace{1mm}%
{\noindent\footnotesize \nlfont{#1}}}

% pruznejsi varianta nasledujiciho - umoznuje nastavit sirku sloupce
% s prekladem
\newcommand{\psalmusEtTranslatioB}[3]{
  \vspace{0.5cm}
  \begin{parcolumns}[colwidths={2=#3}, nofirstindent=true]{2}
    \colchunk{
      \input{#1}
    }

    \colchunk{
      \vspace{-0.5cm}
      {\footnotesize \nlfont
        \input{#2}
      }
    }
  \end{parcolumns}
}

\newcommand{\psalmusEtTranslatio}[2]{
  \psalmusEtTranslatioB{#1}{#2}{8.5cm}
}


\newcommand{\canticumMagnificatEtTranslatio}[1]{
  \psalmusEtTranslatioB{#1}{temporalia/extra-adventum-vespers/magnificat-boh.tex}{12cm}
}
\newcommand{\canticumBenedictusEtTranslatio}[1]{
  \psalmusEtTranslatioB{#1}{temporalia/extra-adventum-laudes/benedictus-boh.tex}{10.5cm}
}

% volne misto nad antifonami, kam si zpevaci dokresli neumy
\newcommand{\hicSuntNeumae}{\vspace{0.5cm}}

% prepinani mista mezi notovymi osnovami: pro neumovane a neneumovane zpevy
\newcommand{\cantusCumNeumis}{
  \setgrefactor{17}
  \global\advance\grespaceabovelines by 5mm%
}
\newcommand{\cantusSineNeumas}{
  \setgrefactor{17}
  \global\advance\grespaceabovelines by -5mm%
}

% znaky k umisteni nad inicialu zpevu
\newcommand{\superInitialam}[1]{\gresetfirstlineaboveinitial{\small {\textbf{#1}}}{\small {\textbf{#1}}}}

% pars officii, i.e. "oratio", ...
\newcommand{\pars}[1]{\textbf{#1}}

\newenvironment{psalmus}{
  \setlength{\parindent}{0pt}
  \setlength{\parskip}{5pt}
}{
  \setlength{\parindent}{10pt}
  \setlength{\parskip}{10pt}
}

%%%% Prejmenovat na latinske:
\newcommand{\nadpisZalmu}[1]{
  \hspace{2cm}\textbf{#1}\vspace{2mm}%
  \nopagebreak%

}

% mode, score, translation
\newcommand{\antiphona}[3]{%
\hicSuntNeumae
\superInitialam{#1}
\includescore{#2}

#3
}
 % Often used macros
%%%% Preklady jednotlivych zpevu (nektere se opakuji, a je dobre mit je
% vsechny na jedne hromade)

\newcommand{\trOratioAnteOfficium}{\translatioCantus{Otevři, Pane, má ústa, abych chválil tvé svaté jméno.
Očisti mé srdce od všech marnivých, zvrácených a~jiných myšlenek, osvěť rozum, rozněť cit,
abych mohl důstojně, soustředěně a~zbožně recitovat a~vysloužil si být
vyslyšen před tváří tvé velebnosti. Skrze Krista…}}

\newcommand{\trOratioPostOfficium}{\translatioCantus{\textit{Následující modlitbu
opatřil pro ty, kdo ji zbožně vyřknou po hodinkách, zesnulý papež Lev X.
odpustky za hříchy vzniklé při konání hodinek z~lidské křehkosti. Říká se
vkleče.}
Svatosvaté a~nerozdílné Trojici, ukřižovanému lidství našeho Pána Ježíše
Krista, přeblažené a~přeslavné plodné neporušenosti vždy Panny Marie
i~souhrnu všech svatých buď ode všeho stvoření věčná chvála, čest a~sláva, nám
pak buď dáno odpuštění všech hříchů, po nekonečné věky věků. Amen.}}

% HOURS ---

\newcommand{\trAntI}{\translatioCantus{Jasné narození slavné Panny Marie,
z pokolení (dosl. ze semene) Abrahámova, vzešlé z kmene Judova, z rodu Davidova.}}
\newcommand{\trAntII}{\translatioCantus{Dnes je Narození svaté Panny 
Marie, jejíž předrahý život osvěcuje všechny církve.}}

\newcommand{\trAntIII}{\translatioCantus{Maria, jež vzešla 
z královského rodu, září; myslí i duchem ji zbožně prosíme, aby 
nám pomáhala svými přímluvami.}}

\newcommand{\trAntIV}{\translatioCantus{Srdcem i duchem pějme Kristu 
k slávě o této svaté slavnosti vznešené Rodičky Boží Marie.}}

\newcommand{\trAntV}{\translatioCantus{Příjemně \notitia{?} 
oslavujme Narození blahoslavené Marie,
aby se ona za nás přimlouvala u Pána Ježíše Krista.}}

\newcommand{\trCapituli}{\translatioCantus{Před věky, na počátku mě stvořil, potrvám věčně. Ve svatém Stanu jsem před ním konala službu.}}

\newcommand{\trRespVesp}{\translatioCantus{Buď zdráva, Maria,
plná milosti: \grestar{} Pán s tebou. \Vbardot{} Požehnaná jsi mezi ženami,
a požehnaný plod života (ve smyslu lůna, břicha) tvého.}}

\newcommand{\trVersus}{\translatioCantus{\Vbardot{} Dnes je Narození svaté Panny Marie. \Rbardot{} Jejíž předrahý život osvěcuje všechny církve.}}

\newcommand{\trAntMagnificatI}{\translatioCantus{Konejme památku
veledůstojného narození slavné Panny Marie,
jíž se dostalo mateřské důstojnosti bez ztráty panenské cudnosti.}}

% Tento preklad je vice nez nejisty a ani alternativy, ktere jsem
% videl, me nepresvedcily...
\newcommand{\trAntBenedictus}{\translatioCantus{Slavnostně slavme 
dnešní narození Marie, vždy Panny a Rodičky Boží: v něm se objevuje
vysokost trůnu (totiž Marie, trůnu Božího Syna), aleluja.}}

\newcommand{\trAntMagnificatII}{\translatioCantus{Tvé narození,
Bohorodičko Panno, vyhlásilo radost celému světu:
z tebe totiž vzešlo Slunce spravedlnosti, Kristus, náš Bůh:
jenž zrušil kletbu a dal nám požehnání: přemohl smrt a dal nám život věčný.}}

\newcommand{\trOrationis}{\translatioCantus{Prosíme tě, Bože, 
uděl svým služebníkům dar nebeské milosti,
aby těm, jimž slehnutím blahoslavené Panny vyvstal počátek spásy, 
slavnost k poctě jejího narození přinesla
rozhojnění pokoje.
Skrze tvého Syna, našeho Pána Ježíše Krista, který s tebou žije a kraluje,
Bůh, v jednotě Ducha svatého po všechny věky věků.}}

\newcommand{\trFideliumAnimae}{\translatioCantus{\Vbardot{} Duše věrných ať pro
milosrdenství Boží odpočívají v~pokoji. \Rbardot{} Amen.}}

% Completorium

\newcommand{\trJubeDomne}{\translatioCantus{Rač, pane, požehnat.}}

\newcommand{\trComplBenedictio}{\translatioCantus{Pokojnou noc a~svatou smrt
nechť nám dopřeje všemohoucí Pán. \Rbardot{} Amen.}}

\newcommand{\trComplLectioBr}{\translatioCantus{Buďte střízliví, bděte.
Váš protivník Ďábel obchází jako lev řvoucí a~hledá, koho by pohltil.
Postavte se proti němu pevní ve víře.  Ale ty, Pane, smiluj se nad námi.
\Rbardot{} Bohu díky.}}

\newcommand{\trComplAntI}{\translatioCantus{Rač se smilovati nade mnou,
Hospodine, a vyslyš mou modlitbu.}}

\newcommand{\trComplCapituli}{\translatioCantus{Jsi přece, Hospodine,
uprostřed nás a~jmenujeme se po tobě.  Neopouštěj nás, Pane, náš Bože.}}

\newcommand{\trRespCompl}{\translatioCantus{Do tvých rukou, Pane, \grestar{}
poroučím svého ducha. \Vbardot{} Ty mne zachráníš, Pane, Bože věrný.}}

\newcommand{\trComplVersus}{\translatioCantus{\Vbardot{} Střez mne jako zřítelnici oka,
aleluja. \Rbardot{} Ve stínu svých křídel uschovej mne, aleluja.}}

\newcommand{\trAntSalvaNos}{\translatioCantus{Ochraňuj nás, Pane, když
bdíme, a~buď s~námi, když spíme, ať bdíme s~Kristem a~odpočíváme v~pokoji.}}

\newcommand{\trComplOrationis}{\translatioCantus{Zavítej, prosíme, Pane, sem
do našeho příbytku a~daleko od něj zažeň všechny úklady nepřítele. Ať tu
bydlí tví svatí andělé a~tvoje požehnání buď nad ním stále. Skrze…}}

\newcommand{\trSalveRegina}{\translatioCantus{Zdrávas Královno, matko
milosrdenství, živote, sladkosti a naděje naše, buď zdráva!
K tobě voláme, vyhnaní synové Evy,
k tobě vzdycháme, lkajíce a plačíce
v tomto slzavém údolí.
A proto, orodovnice naše,
obrať k nám své milosrdné oči
a Ježíše, požehnaný plod života svého,
nám po tomto putování ukaž,
ó milostivá, ó přívětivá,
ó přesladká, Panno Maria!}}

\newcommand{\trOraProNobis}{\translatioCantus{\Vbardot{} 
Oroduj za nás, svatá Boží Rodičko,
\Rbardot{} aby nám Kristus dal účast na svých zaslíbeních.}}

% Matutinum

\newcommand{\trMatInvitatorium}{\translatioCantus{}}

\newcommand{\trMatVeniteA}{\translatioCantus{Pojďte, chvalme s~radostí Pána,
s~jásotem slavme Boha, svou spásu; předstupme před tvář jeho s~díky, písně plesu pějme jemu.}}

\newcommand{\trMatVeniteB}{\translatioCantus{Neboť Bůh veliký jest Hospodin, a~král nade všecky bohy.
Jsouť v~jeho ruce všecky hlubiny země, temena hor jsou majetek jeho.}}

\newcommand{\trMatVeniteC}{\translatioCantus{Jehoť jest moře, neb on je učinil; i~souš
je dílo jeho rukou. Pojďme, klanějme se, padněme, klekněme před Pánem, svým
tvůrcem. Jeť on Pán, náš Bůh, a~my jsme lid, jejž on vodí a~ovce, jež pase.}}

\newcommand{\trMatVeniteD}{\translatioCantus{Kéž byste poslechli dnes hlasu jeho:
,,Nezatvrzujte svých srdcí jak v~Hádce, jak v~Pokušení na poušti, kde vaši otcové pokoušeli mne,
zkoušeli mne, ač vídali skutky mé.``}}

\newcommand{\trMatVeniteE}{\translatioCantus{Čtyřicet roků mrzel jsem se na to pokolení
a~řekl jsem: ,,Lid je to myslí stále bloudící``! Oni však nechtěli znáti mé cesty, takže jsem
přisáhl ve svém hněvu: ,,Nedojdou odpočinku mého!\mbox{}``}}

\newcommand{\trMatAntI}{\translatioCantus{}}

\newcommand{\trMatAntII}{\translatioCantus{}}

\newcommand{\trMatAntIII}{\translatioCantus{}}

\newcommand{\trMatVersusI}{\translatioCantus{}}

\newcommand{\trMatAbsolutioI}{\translatioCantus{Vyslyš Pane Ježíši Kriste
prosby svých služebníků \gredagger{} a~smiluj se nad námi, \grestar{} jenž
s~Otcem a~Duchem…}}

\newcommand{\trMatBenedictioI}{\translatioCantus{Rač, pane, požehnat.
Věčný Otec nám stále žehnej. \Rbardot{} Amen.}}

\newcommand{\trMatLecI}{\translatioCantus{Kéž by mě zulíbal polibky svých úst. 
Tvé milování je nad víno lahodnější;
vybraně voní tvé voňavky;
rozlévající se olej je tvé jméno,
proto tě dívky milují.
Strhni mě za sebou, poběžme!
Král mě uvedl do svých komnat;
budeš nám radostí a jásotem.
Víc než víno oslavíme tvé milování;
věru po právu jsi milován!
Snědá jsem, a přece krásná, jeruzalémské dcery,
jako stany kedarské,
jako šalmské závěsy.
}}

\newcommand{\trMatRespI}{\translatioCantus{}}

\newcommand{\trMatBenedictioII}{\translatioCantus{Rač, pane, požehnat.
Jednorozený Boží Syn nám žehnej \grestar{} a nám pomáhej. \Rbardot{} Amen.}}

\newcommand{\trMatLecII}{\translatioCantus{Nehleďte na mou osmahlou pleť:
to mě slunce ožehlo.
Synové mé matky se na mne rozzlobili,
poslali mě hlídat vinice.
A svou vinici, tu jsem nehlídala!
Pověz mi tedy, ty, jehož miluje mé srdce:
kam zavedeš své stádo pást,
kde ho necháš za poledne odpočívat?
Abych už nebloudila jako tulačka
poblíž stád druhů tvých.
Nevíš-li to, nejrásnější z žen,
jdi po stopách stáda
a kůzlata svá zaveď, ať se pasou
poblíž obydlí pastýřů.
Ke své klisně zapřažené do vozu faraonova
tebe, mé milá, přirovnávám.
Stále krásné jsou tvé líce s náušnicemi
i tvé hrdlo v náhrdelnících.}}

\newcommand{\trMatRespII}{\translatioCantus{}}

\newcommand{\trMatBenedictioIII}{\translatioCantus{Rač, pane, požehnat.
Milost Ducha Svatého ať osvítí nám smysly \grestar{} i srdce. \Rbardot{} Amen.}}

\newcommand{\trMatLecIII}{\translatioCantus{Zhotovíme ti zlaté náušnice
a kuličky ze stříbra.
Když král stoluje,
vydechuje můj nard svou vůni.
Můj milý je polštářek s myrhou,
jenž mi odpočívá mezi ňadry.
Můj milý je hrozen šáchoru
ve vinicích v Engadi.
Jak jsi krásná, milá moje,
jak jsi krásná!
Tvé oči jsou holubice.
Jak jsi krásný, můj milý,
jak líbezný!
Naše lože je samá zeleň.
Trámoví našeho domu je z cedru,
naše ostění z cypřiše.}}

\newcommand{\trMatRespIII}{\translatioCantus{}}

\newcommand{\trMatAntIV}{\translatioCantus{}}

\newcommand{\trMatAntV}{\translatioCantus{}}

\newcommand{\trMatAntVI}{\translatioCantus{}}

\newcommand{\trMatVersusII}{\translatioCantus{}}

\newcommand{\trMatAbsolutioII}{\translatioCantus{
Tvá milost a laskavost nechť nám pomáhá, jenž žiješ a vládneš s Otcem a Svatým Duchem na věky věků.}}

\newcommand{\trMatBenedictioIV}{\translatioCantus{Rač, pane, požehnat.
Bůh Otec všemohoucí, \grestar{} buď k nám milostivý a odpouštějící. \Rbardot{} Amen.}}

\newcommand{\trMatLecIV}{\translatioCantus{}}

\newcommand{\trMatRespIV}{\translatioCantus{}}

\newcommand{\trMatBenedictioV}{\translatioCantus{}}

\newcommand{\trMatLecV}{\translatioCantus{}}

\newcommand{\trMatRespV}{\translatioCantus{}}

\newcommand{\trMatBenedictioVI}{\translatioCantus{Rač, pane, požehnat.
Bůh rozněť v nás oheň své lásky. \Rbardot{} Amen.}}

\newcommand{\trMatLecVI}{\translatioCantus{}}

\newcommand{\trMatRespVI}{\translatioCantus{}}

\newcommand{\trMatAntVII}{\translatioCantus{}}

\newcommand{\trMatAntVIII}{\translatioCantus{}}

\newcommand{\trMatAntIX}{\translatioCantus{}}

\newcommand{\trMatVersusIII}{\translatioCantus{}}

\newcommand{\trMatAbsolutioIII}{\translatioCantus{Z okovů našich hříchů,
\grestar{} vysvoboď nás všemohoucí a milosrdný Pán. \Rbardot{} Amen.}}

\newcommand{\trMatBenedictioVII}{\translatioCantus{Rač, pane, požehnat.
Čtení evangelia nechť je nám \grestar{} spásou a ochranou. \Rbardot{} Amen.}}

\newcommand{\trMatLecVIIa}{\translatioCantus{
  Rodokmen Ježíše Krista, syna Davidova, syna Abrahámova:
  Abrahám zplodil Izáka,
  Izák zplodil Jakuba.}}

\newcommand{\trMatLecVIIb}{\translatioCantus{}}

\newcommand{\trMatRespVII}{\translatioCantus{}}

\newcommand{\trMatBenedictioVIII}{\translatioCantus{Rač, pane, požehnat.
\Rbardot{} Amen.}}

\newcommand{\trMatLecVIII}{\translatioCantus{}}

\newcommand{\trMatRespVIII}{\translatioCantus{}}

\newcommand{\trMatBenedictioIX}{\translatioCantus{Rač, pane, požehnat.
Do společnosti občanů nebes \grestar{} ať nás dovede král andělů.
\Rbardot{} Amen.}}

\newcommand{\trMatLecIX}{\translatioCantus{}}

% from the Czech Liturgia horarum
\newcommand{\trTeDeum}{\begin{translatioMulticol}{3}

Bože, tebe chválíme, 
tebe, Pane, velebíme.

Tebe, věčný Otče, 
oslavuje celá země.

Všichni andělé, 
cherubové i~serafové,

všechny mocné nebeské zástupy 
bez ustání volají:

Svatý, Svatý, Svatý, 
Pán, Bůh zástupů.

Plná jsou nebesa i~země 
tvé vznešené slávy.

Oslavuje tě 
sbor tvých apoštolů,

chválí tě 
velký počet proroků,

vydává o~tobě svědectví 
zástup mučedníků;

a~po celém světě 
vyznává tě tvá církev:

neskonale velebný, 
všemohoucí Otče,

úctyhodný Synu Boží, 
pravý a~jediný,

božský Utěšiteli, 
Duchu svatý.

Kriste, Králi slávy, 
tys od věků Syn Boha Otce;

abys člověka vykoupil, 
stal ses člověkem a~narodil ses z~Panny;

zlomil jsi osten smrti 
a~otevřel věřícím nebe;

sedíš po Otcově pravici 
a~máš účast na jeho slávě.

Věříme, že přijdeš soudit, 

a~proto tě prosíme:
přispěj na pomoc svým služebníkům, 
vždyť jsi je vykoupil svou předrahou krví;

dej, ať se radují s~tvými svatými 
ve věčné slávě.

Zachraň, Pane, svůj lid, žehnej svému dědictví, 
veď ho a~stále pozvedej.

Každý den tě budeme velebit 
a~chválit tvé jméno po všechny věky.

Pomáhej nám i~dnes, 
ať se nedostaneme do područí hříchu.

Smiluj se nad námi, Pane, 
smiluj se nad námi.

Ať spočine na nás tvé milosrdenství, 
jak doufáme v~tebe.

Pane, k~tobě se utíkáme, 
ať nejsme zahanbeni na věky. 
\end{translatioMulticol}}

\newcommand{\trMatEvangelium}{\translatioCantus{
  Rodokmen Ježíše Krista, syna Davidova, syna Abrahámova:
  Abrahám zplodil Izáka,
  Izák zplodil Jakuba,
  Jakub zplodil Judu a jeho bratry,
  Juda zplodil Farese a Zaru z Tamary,
  Fares zplodil Esroma,
  Esrom zplodil Arama,
  Aram zplodil Aminadaba,
  Aminadab zplodil Naasona,
  Naason zplodil Salmona,
  Salmon zplodil Boaze z Rahaby,
  Boaz zplodil Jobeda z Rut,
  Jobed zplodil Jessea,
  Jesse zplodil krále Davida.
  David zplodil Šalomouna z Uriášovy ženy,
  Šalomoun zplodil Roboama,
  Roboam zplodil Abiu,
  Abia zplodil Asu,
  Asa zplodil Josafata,
  Josafat zplodil Jorama,
  Joram zplodil Oziáše,
  Oziáš zplodil Joatama,
  Joatam zplodil Achaze,
  Achaz zplodil Ezechiáše,
  Ezechiáš zplodil Manasesa,
  Manases zplodil Amona,
  Amon zplodil Josiáše,
  Josiáš zplodil Jechoniáše a jeho bratry;
  tehdy došlo k odvlečení do Babylonu.
  Po odvlečení do Babylonu:
  Jechoniáš zplodil Salatiela,
  Salatiel zplodil Zorobabela,
  Zorobabel zplodil Abiuda,
  Abiud zplodil Eljakima,
  Eljakim zplodil Azora,
  Ator zplodil Sadoka,
  Sadok zplodil Achima,
  Achim zplodil Eliuda,
  Eliud zplodil Eleazara,
  Eleatar zplodil Matana,
  Matan zplodil Jakuba,
  Jakub zplodil Josefa, manžela Marie,
  z níž se narodil Ježíš, který se nazývá Kristus.}}

\newcommand{\trTeDecetLaus}{\translatioCantus{Tobě chvála, Tobě zpěvy, Tobě
sláva, Bohu Otci i~Synu i~Svatému Duchu, na věky věků. \Rbardot{} Amen.}}

% MASS ---

\newcommand{\trIntroitus}{\translatioCantus{Radujme se všichni
v Pánu, slavíce svátek ke cti Panny Marie: z něj se radují andělé
a spoluchválí Božího Syna. \textit{\color{red}Žl.} Má ústa vydala dobré slovo,
přednáším svá díla králi.}}

\newcommand{\trGraduale}{\translatioCantus{Požehnaná a ctihodná jsi,
Panno Maria: nedotčená (co do panenství) jsi byla shledána matkou
Spasitele. \Vbardot{} Panno Boží Rodičko, ten, jehož nepojme ani celý svět,
se uzavřel do tvých útrob, když se stal člověkem.}}

\newcommand{\trAlleluia}{\translatioCantus{Aleluja. \Vbardot{} Skvělá slavnost
slavné Panny Marie, z pokolení (dosl. ze semene) Abrahámova, vzešlé z kmene 
Judova, z rodu Davidova.}}

\newcommand{\trOffertorium}{\translatioCantus{Blažená jsi, Panno Maria,
tys nosila Stvořitele všeho; porodila jsi toho, který tě utvořil,
a na věky zůstáváš Pannou.}}

\newcommand{\trCommunio}{\translatioCantus{Budou mě blahoslavit
všechna pokolení, protože mi učinil veliké věci ten, který je mocný.}}

% LITTLE HOURS ---

\newcommand{\trVersusTertia}{\translatioCantus{\Vbardot{} \Rbardot{}}}

\newcommand{\trCapituliEtSic}{\translatioCantus{
Tak jsem se usadila na Sionu a v milovaném městě jsem nalezla odpočinek,
v Jeruzalémě vykonávám svou moc.
Zakořenila jsem u lidu plného slývy, na panství Páně, v jeho dědictví.}}

\newcommand{\trVersusSexta}{\translatioCantus{\Vbardot{} \Rbardot{}}}

\newcommand{\trCapituliInPlateis}{\translatioCantus{
Na planině jako skořicovník a akant jsem vydávala vůni, jako vybraná myrha
jsem voněla.}}

\newcommand{\trVersusNona}{\translatioCantus{\Vbardot{} \Rbardot{}}}
 % Czech translations of the proper texts

\newcommand{\annusEditionis}{2018}

\def\hebinitial#1{%
\leavevmode{\newbox\hebbox\setbox\hebbox\hbox{\hebfont{#1}\hskip 1mm}\kern -\wd\hebbox\hbox{\hebfont{#1}\hskip 1mm}}%
}

%%%% Vicekrat opakovane kousky

\newcommand{\anteOrationem}{
  \rubrica{Ante Orationem, cantatur a Superiore:}

  \pars{Supplicatio Litaniæ.}

  \cuminitiali{}{temporalia/supplicatiolitaniae.gtex}

  \pars{Oratio Dominica.}

  \cuminitiali{}{temporalia/oratiodominica.gtex}

  \rubrica{Deinde dicitur ab Hebdomadario:}

  \cuminitiali{}{temporalia/dominusvobiscum-solemnis.gtex}

  \rubrica{In choro monialium loco Dominus vobiscum dicitur:}

  \sineinitiali{temporalia/domineexaudi.gtex}
}

\setlength{\columnsep}{30pt} % prostor mezi sloupci

%%%%%%%%%%%%%%%%%%%%%%%%%%%%%%%%%%%%%%%%%%%%%%%%%%%%%%%%%%%%%%%%%%%%%%%%%%%%%%%%%%%%%%%%%%%%%%%%%%%%%%%%%%%%%
\begin{document}

% Here we set the space around the initial.
% Please report to http://home.gna.org/gregorio/gregoriotex/details for more details and options
\grechangedim{afterinitialshift}{2.2mm}{scalable}
\grechangedim{beforeinitialshift}{2.2mm}{scalable}

\grechangedim{interwordspacetext}{0.22 cm plus 0.15 cm minus 0.05 cm}{scalable}%
\grechangedim{annotationraise}{-2mm}{scalable}

% Here we set the initial font. Change 38 if you want a bigger initial.
% Emit the initials in red.
\grechangestyle{initial}{\color{red}\fontsize{38}{38}\selectfont}

\pagestyle{empty}

%%%% Titulni stranka
\begin{titulusOfficii}
\nomenFesti{Officium Defunctorum}
\end{titulusOfficii}

\pars{}

\scriptura{}

\vfill
\pagebreak

\renewcommand{\headrulewidth}{0pt} % no horiz. rule at the header
\fancyhf{}
\pagestyle{fancy}

\pars{Oratio ante divinum Officium.}

\lettrine{{\color{red}A}}{peri,} Dómine, os meum ad benedicéndum nomen sanctum tuum:
munda quoque cor meum ab ómnibus vanis, pervérsis, et aliénis
cogitatiónibus:
intelléctum illúmina, afféctum inflámma,
ut digne, atténte ac devóte hoc Offícium recitáre váleam,
et exaudíri mérear ante conspéctum Divínæ Maiestátis tuæ.
Per Christum, Dóminum nostrum.
\Rbardot{} Amen.

Dómine, in unióne illíus divínæ intentiónis,
qua ipse in terris laudes Deo persolvísti,
has tibi Horas \rubricatum{(vel \textnormal{hanc tibi Horam})} persólvo.

%\trOratioAnteOfficium

\vfill

\pars{Oratio post divinum Officium.}

\rubrica{
  Orationem sequentem devote post Officium recitantibus
  Leo Papa X. defectus, et culpas in eo persolvendo ex humana
  fragilitate contractas, indulsit, et dicitur flexis genibus.
}

\lettrine{{\color{red}S}}{acrosánctæ} et indivíduæ Trinitáti,
crucifíxi Dómini nostri Iesu Christi humanitáti,
beatíssimæ et gloriosíssimæ sempérque Vírginis Maríæ
fecúndæ integritáti,
et ómnium Sanctórum universitáti
sit sempitérna laus, honor, virtus et glória
ab omni creatúra,
nobísque remíssio ómnium peccatórum,
per infiníta sǽcula sæculórum.
\Rbardot{} Amen.

\noindent \Vbardot{} Beáta víscera Maríæ Vírginis, quæ portavérunt
ætérni Patris Fílium.\\
\Rbardot{} Et beáta úbera, quæ lactavérunt Christum Dóminum.

\rubrica{Et dicitur secreto \textnormal{Pater noster.} et \textnormal{Ave María.}}

%\trOratioPostOfficium

\vfill

\scriptura{} \pars{}

\cantusSineNeumas{}

\vfill
\pagebreak

\hora{Ad Matutinum.} %%%%%%%%%%%%%%%%%%%%%%%%%%%%%%%%%%%%%%%%%%%%%%%%%%%%%%%%%%
%\sideThumbs{Matutinum}

\rubrica{Absolute incipitur ab Invitatorio.}

\vspace{2mm}

\pars{Invitatorium.}

\vspace{-5mm}

\antiphona{VI}{temporalia/inv-regemcui.gtex}

\vspace{3mm}

\rubrica{Hymnus omittitur.}

\vfill
\pagebreak

\iffalse
\subhora{In I. Nocturno}
\fi

\pars{Psalmus 1.} \scriptura{Ps. 5, 9; \textbf{H390}}

\vspace{-5mm}

\antiphona{VII c}{temporalia/ant-dirigedomine.gtex}

%\trMatAntI

\scriptura{Psalmus 5.}

\initiumpsalmi{temporalia/ps5-initium-vii-c-auto.gtex}

%\vspace{-4mm}

%\psalmusEtTranslatioT{temporalia/ps5-comb.tex}{10cm}
\input{temporalia/ps5.tex}

\antiphona{}{temporalia/ant-dirigedomine.gtex} % repeat the antiphon - new page

\vfill
\pagebreak

\pars{Psalmus 2.} \scriptura{Ps. 6, 5.6; \textbf{H390}}

\vspace{-5mm}

\antiphona{VIII G}{temporalia/ant-converteredomine.gtex}

%\trMatAntII

%\vspace{-4mm}

\scriptura{Psalmus 6.}

\initiumpsalmi{temporalia/ps6-initium-viii-G-auto.gtex}

%\vspace{-5mm}

%\psalmusEtTranslatioT{temporalia/ps6-comb.tex}{10cm}
\input{temporalia/ps6.tex} \Abardot{}

\vfill
\pagebreak

\pars{Psalmus 3.} \scriptura{Ps. 7, 3; \textbf{H390}}

\vspace{-5mm}


\antiphona{VIII G}{temporalia/ant-nequandorapiat.gtex}

%\trMatAntIII

\scriptura{Psalmus 7.}

\initiumpsalmi{temporalia/ps7-initium-viii-G-auto.gtex}

%\psalmusEtTranslatioT{temporalia/ps7-comb.tex}{10cm}
\input{temporalia/ps7.tex}

\antiphona{}{temporalia/ant-nequandorapiat.gtex} % repeat the antiphon - new page

\vfill
\pagebreak

\pars{Versus.} \scriptura{Cf. Is. 38, 10.17}

\sineinitiali{temporalia/versus-aporta-alt.gtex}

\noindent Pater noster \rubricatum{quod dicitur totum secreto.}

\vfill

\rubrica{Lectiones leguntur sine Absolutione, Benedictione et Titulo in tono Prophetiæ.}

\cuminitiali{}{temporalia/tonus-lectionis-prophetiae.gtex}

\vfill

\pars{Lectio I.} \scriptura{Iob 7, 16-21}

\noindent Parce mihi, Dómine; nihil enim sunt dies mei. Quid est homo, quia magníficas eum? aut quid appónis erga eum cor tuum? Vísitas eum dilúculo, et súbito probas illum. Usquequo non parcis mihi, nec dimíttis me, ut glútiam salívam meam? Peccávi, quid fáciam tibi, o custos hóminum? quare posuísti me contrárium tibi, et factus sum mihimetípsi gravis? Cur non tollis peccátum meum, et quare non aufers iniquitátem meam? Ecce nunc in púlvere dórmiam: et, si mane me quæsíeris, non subsístam.

\rubrica{Lectiones terminantur sine \textnormal{Tu autem.} vel alia conclusione.}

\vfill
\pagebreak

\pars{Responsorium 1.} \scriptura{\Rbardot{} Iob 19, 25.26; \Vbardot{} ibid. 19, 27; \textbf{H390}}

\vspace{-5mm}

\responsorium{VIII}{temporalia/resp-credoquod-GN.gtex}{}

\vfill
\pagebreak

\pars{Lectio II.} \scriptura{Iob 14, 1-6}

\noindent Homo natus de mulíere, brevi vivens témpore, replétur multis misériis. Qui quasi flos egréditur et contéritur, et fugit velut umbra, et nunquam in eódem statu pérmanet. Et dignum ducis super huiuscémodi aperíre óculos tuos, et addúcere eum tecum in iudícium? Quis potest fácere mundum de immúndo concéptum sémine? Nonne tu qui solus es? Breves dies hóminis sunt: númerus ménsium eius apud te est: constituísti términos eius, qui præteríri non póterunt. Recéde páululum ab eo, ut quiéscat, donec optáta véniat, sicut mercenárii, dies eius.

\vfill
\pagebreak

\pars{Responsorium 2.} \scriptura{\Rbardot{} Cantor; \Vbardot{} ibidem; \textbf{H390}}

\vspace{-5mm}

\responsorium{IV}{temporalia/resp-quilazarum-GN.gtex}{}

\vfill
\pagebreak

\pars{Lectio III.} \scriptura{Iob 19, 20-27}

\noindent Pelli meæ, consúmptis cárnibus, adhǽsit os meum, et derelícta sunt tantúmmodo lábia circa dentes meos. Miserémini mei, miserémini mei saltem vos, amíci mei, quia manus Dómini tétigit me. Quare persequímini me sicut Deus, et cárnibus meis saturámini? Quis mihi tríbuat ut scribántur sermónes mei? quis mihi det ut exaréntur in libro stylo férreo et plumbi lámina, vel celte sculpántur in sílice? Scio enim quod redémptor meus vivit, et in novíssimo die de terra surrectúrus sum: et rursum circúmdabor pelle mea, et in carne mea vidébo Deum meum: quem visúrus sum ego ipse, et óculi mei conspectúri sunt, et non álius: repósita est hæc spes mea in sinu meo.

\vfill
\pagebreak

\pars{Responsorium 3.} \scriptura{\Rbardot{} Cantor; \Vbardot{} ibidem; \textbf{H390}}

\vspace{-5mm}

\responsorium{VIII}{temporalia/resp-dominequando-CROCHU.gtex}{}

\vfill
\pagebreak

\iffalse
\subhora{In II. Nocturno}

\pars{Psalmus 4.} \scriptura{Ps. 22, 2; \textbf{H391}}

\vspace{-5mm}

\antiphona{VIII G}{temporalia/matant4.gtex}

%\trMatAntIV

\scriptura{Psalmus 22.}

\initiumpsalmi{temporalia/ps22-initium-viii-G-auto.gtex}

%\psalmusEtTranslatioT{temporalia/ps22-comb.tex}{10cm}
\input{temporalia/ps22.tex} \Abardot{}

%\antiphona{}{temporalia/matant4.gtex} % repeat the antiphon - new page

\vfill
\pagebreak

\pars{Psalmus 5.} \scriptura{Ps. 24, 7; \textbf{H391}}

\vspace{-5mm}

\antiphona{VIII G}{temporalia/ant-delictaiuventutis.gtex}

%\trMatAntV

\scriptura{Psalmus 24.}

\initiumpsalmi{temporalia/ps24-initium-viii-G-auto.gtex}

%\psalmusEtTranslatioT{temporalia/ps24-comb.tex}{10cm}
\input{temporalia/ps24.tex}

\antiphona{}{temporalia/ant-delictaiuventutis.gtex} % repeat the antiphon - new page

\vfill
\pagebreak

\pars{Psalmus 6.} \scriptura{Ps. 26, 13; \textbf{H391}}

\vspace{-5mm}

\antiphona{IV* E}{temporalia/matant6.gtex}

%\trMatAntVI

\scriptura{Psalmus 26.}

%\initiumpsalmi{temporalia/ps26-initium-iv-E-auto.gtex}
\initiumpsalmi{temporalia/ps26-initium-iv_-e.gtex}

%\psalmusEtTranslatioT{temporalia/ps26-comb.tex}{10cm}
%\input{temporalia/ps26.tex}
\input{temporalia/ps26iv_e.tex}

\antiphona{}{temporalia/matant6.gtex} % repeat the antiphon - new page

\vfill
\pagebreak

\pars{Versus.} \scriptura{Ps. 112, 8}

\sineinitiali{temporalia/versus-collocet-alt.gtex}

\noindent Pater noster \rubricatum{quod dicitur totum secreto.}

\vfill

\pars{Lectio IV.} \scriptura{Cap. 2 et 3}

\noindent Ex Libro sancti Augustíni Epíscopi de cura pro mórtuis gerénda.

\noindent Curátio fúneris, condítio sepultúræ, pompa exsequiárum magis sunt vivórum solácia quam subsídia mortuórum. Nec ídeo tamen contemnénda et abiciénda sunt córpora defunctórum, maximéque iustórum ac fidélium, quibus tamquam órganis et vasis ad ómnia bona ópera sancte usus est spíritus. Si enim patérna vestis et ánulus, ac si quid huiúsmodi, tanto cárius est pósteris, quanto erga paréntes maior afféctus; nullo modo ipsa spernénda sunt córpora, quæ útique multo familiárius atque coniúnctius quam quǽlibet induménta gestámus. Hæc enim non ad ornaméntum vel adiutórium, quod adhibétur extrínsecus, sed ad ipsam natúram hóminis pértinent. Unde et antiquórum iustórum fúnera officiósa pietáte curáta sunt, et exséquiæ celebrátæ, et sepultúra provísa; ipsíque, cum víverent, de sepeliéndis vel étiam transferéndis suis corpóribus fíliis mandavérunt.

\vfill
\pagebreak

\pars{Responsorium 4.} \scriptura{\Rbardot{} Iob 7, 7.8; \Vbardot{} Ps. 129, 1.2; \textbf{H403}}

\vspace{-5mm}

\responsorium{II}{temporalia/resp-mementomei-CROCHU.gtex}{}

\vfill
\pagebreak

\pars{Lectio V.} \scriptura{Cap. 4}

\noindent Recordántis et precántis afféctus cum defúnctis a fidélibus caríssimis exhibétur, eum prodésse non dúbium est iis, qui cum in córpore víverent, tália sibi post hanc vitam prodésse meruérunt. Verum, etsi áliqua necéssitas vel humári córpora, vel in sacris locis humári nulla data facultáte permíttat, non sunt prætermitténdæ supplicatiónes pro spirítibus mortuórum: quas faciéndas pro ómnibus in christiána et cathólica societáte defúnctis, étiam tácitis eórum nomínibus, sub generáli commemoratióne suscépit Ecclésia; ut quibus ad ista desunt paréntes, aut fílii, aut quicúmque cognáti vel amíci, ab una eis exhibeántur pia matre commúni. Si autem deéssent istæ supplicatiónes, quæ fiunt recta fide ac pietáte pro mórtuis, puto quod nihil prodésset spirítibus eórum, quámlibet in locis sanctis exánima córpora poneréntur.

\vfill
\pagebreak

\pars{Responsorium 5.} \scriptura{\Rbardot{} Cantor; \Vbardot{} Ps. 6, 4.5; \textbf{H391}}

\vspace{-5mm}

\responsorium{II}{temporalia/resp-heumihi-CROCHU.gtex}{}

\vfill
\pagebreak

\pars{Lectio VI.} \scriptura{Cap. 18}

\noindent Quæ cum ita sint, non existimémus ad mórtuos, pro quibus curam gérimus, perveníre, nisi quod pro eis sive altáris, sive oratiónum, sive eleemosynárum sacrifíciis solémniter supplicámus: quamvis non pro quibus fiunt, ómnibus prosint; sed iis tantum pro quibus, dum vivunt, comparátur, ut prosint. Sed quia non discérnimus qui sunt, opórtet ea pro regenerátis ómnibus fácere, ut nullus eórum prætermittátur, ad quos hæc benefícia possint et débeant perveníre. Mélius enim supérerunt ista eis, quibus nec obsunt nec prosunt; quam eis déerunt, quibus prosunt. Diligéntius tamen facit hæc quisque pro necessáriis suis, quo pro illo fiat simíliter a suis. Córpori autem humándo quidquid impénditur, non est præsídium salútis, sed humanitátis offícium, secúndum afféctum quo nemo unquam carnem suam ódio habet. Unde opórtet ut quam potest pro carne próximi curam gerat, cum ille inde recésserit, qui gerébat. Et si hæc fáciunt, qui carnis resurrectiónem non credunt, quanto magis debent fácere qui credunt; ut córpori mórtuo, sed tamen resurrectúro et in æternitáte mansúro, impénsum eiúsmodi offícium sit étiam quodámmodo eiúsdem fídei testimónium!

\vfill
\pagebreak

\pars{Responsorium 6.} \scriptura{\Rbardot{} Cantor; \Vbardot{} Ps. 5, 9; \textbf{H391}}

\vspace{-5mm}

\responsorium{VI}{temporalia/resp-nerecorderis-CROCHU.gtex}{}

\vfill
\pagebreak

\subhora{In III. Nocturno}

\pars{Psalmus 7.} \scriptura{Ps. 39, 14; \textbf{H391}}

\vspace{-5mm}

\antiphona{II D}{temporalia/ant-complaceat.gtex}

%\trMatAntVII

\scriptura{Psalmus 39.}

\initiumpsalmi{temporalia/ps39-initium-ii-D-auto.gtex}

%\psalmusEtTranslatioT{temporalia/ps39-comb.tex}{10cm}
\input{temporalia/ps39.tex}

\antiphona{}{temporalia/ant-complaceat.gtex} % repeat the antiphon - new page

\vfill
\pagebreak

\pars{Psalmus 8.} \scriptura{Ps. 40, 5; \textbf{H93}}

\vspace{-5mm}

\antiphona{II D}{temporalia/matant8.gtex}

%\trMatAntVIII

\scriptura{Psalmus 40.}

\initiumpsalmi{temporalia/ps40-initium-ii-D-auto.gtex}

%\psalmusEtTranslatioT{temporalia/ps40-comb.tex}{10cm}
\input{temporalia/ps40.tex} \Abardot{}

\vfill
\pagebreak

\pars{Psalmus 9.} \scriptura{Ps. 41, 3; \textbf{H391}}

\vspace{-5mm}

\antiphona{II D}{temporalia/ant-sitivitanima.gtex}

%\trMatAntIX

\scriptura{Psalmus 41.}

\initiumpsalmi{temporalia/ps41-initium-ii-D-auto.gtex}

%\psalmusEtTranslatioT{temporalia/ps41-comb.tex}{10cm}
\input{temporalia/ps41.tex}

\antiphona{}{temporalia/ant-sitivitanima.gtex} % repeat the antiphon - new page

\vfill
\pagebreak

\pars{Versus.} \scriptura{Ps. 73, 19}

\sineinitiali{temporalia/versus-netradas-alt.gtex}

\noindent Pater noster \rubricatum{quod dicitur totum secreto.}

\vfill
%\pagebreak

% De Epístola prima beáti Pauli Apóstoli ad Corínthios.
\pars{Lectio VII.} \scriptura{1 Cor. 15, 12-22}

\noindent De Epístola prima beáti Pauli Apóstoli ad Corínthios.

\noindent Si Christus prædicátur quod resurréxit a mórtuis, quómodo quidam dicunt in vobis, quóniam resurréctio mortuórum non est? Si autem resurréctio mortuórum non est: neque Christus resurréxit. Si autem Christus non resurréxit, inánis est ergo prædicátio nostra, inánis est et fides vestra: invenímur autem et falsi testes Dei: quóniam testimónium díximus advérsus Deum quod suscitáverit Christum, quem non suscitávit, si mórtui non resúrgunt. Nam si mórtui non resúrgunt, neque Christus resurréxit. Quod si Christus non resurréxit, vana est fides vestra: adhuc enim estis in peccátis vestris. Ergo et qui dormiérunt in Christo, periérunt. Si in hac vita tantum in Christo sperántes sumus, miserabilióres sumus ómnibus homínibus. Nunc autem Christus resurréxit a mórtuis primítiæ dormiéntium, quóniam quidem per hóminem mors, et per hóminem resurréctio mortuórum. Et sicut in Adam omnes moriúntur, ita et in Christo omnes vivificabúntur.

\vfill
\pagebreak

\pars{Responsorium 7.} \scriptura{\Vbardot{} Cantor; \Rbardot{} Ps. 53, 3; \textbf{H391}}

\vspace{-5mm}

\responsorium{I}{temporalia/resp-peccantemme-CROCHU.gtex}{}

\vfill
\pagebreak

\pars{Lectio VIII.} \scriptura{1 Cor. 15, 35-44}

\noindent Sed dicet áliquis: Quómodo resúrgunt mórtui? qualíve córpore vénient? Insípiens, tu quod séminas non vivificátur, nisi prius moriátur: et quod séminas, non corpus, quod futúrum est, séminas, sed nudum granum, ut puta trítici, aut alicúius ceterórum. Deus autem dat illi corpus sicut vult, et unicuíque séminum próprium corpus. Non omnis caro, éadem caro: sed ália quidem hóminum, ália vero pécorum, ália vólucrum, ália autem píscium. Et córpora cæléstia, et córpora terréstria: sed ália quidem cæléstium glória, ália autem terréstrium. Alia cláritas solis, ália cláritas lunæ, et ália cláritas stellárum; stella enim a stella differt in claritáte: sic et resurréctio mortuórum. Seminátur in corruptióne, surget in incorruptióne. Seminátur in ignobilitáte, surget in glória: seminátur in infirmitáte, surget in virtúte: seminátur corpus animále, surget corpus spiritále.

\vfill
\pagebreak

\pars{Responsorium 8.} \scriptura{\Rbardot{} Cantor; \Vbardot{} Ps. 50, 4; \textbf{H392}}

\vspace{-5mm}

\responsorium{VIII}{temporalia/resp-dominesecundum-CROCHU.gtex}{}

\vfill
\pagebreak

\pars{Lectio IX.} \scriptura{1 Cor. 15, 51-58}

\noindent Ecce mystérium vobis dico: Omnes quidem resurgémus, sed non omnes immutábimur. In moménto, in ictu óculi, in novíssima tuba: canet enim tuba, et mórtui resúrgent incorrúpti: et nos immutábimur. Opórtet enim corruptíbile hoc indúere incorruptiónem: et mortále hoc indúere immortalitátem. Cum autem mortále hoc indúerit immortalitátem, tunc fiet sermo, qui scriptus est: Absórpta est mors in victória. Ubi est mors victória tua? ubi est mors stímulus tuus? Stímulus autem mortis peccátum est: virtus vero peccáti lex. Deo autem grátias, qui dedit nobis victóriam per Dóminum nostrum Iesum Christum. Itaque fratres mei dilécti, stábiles estóte, et immóbiles: abundántes in ópere Dómini semper, sciéntes quod labor vester non est inánis in Dómino.

\vfill
\pagebreak

\pars{Responsorium 9.} \scriptura{\Rbardot{} Ioel 3, 16; \Vbardot{} Cantor; \textbf{H392}}

\vspace{-5mm}

\responsorium{I}{temporalia/resp-liberamedominedemorte-CROCHU.gtex}{}
\fi

\vfill
\pagebreak

\hora{Ad Laudes.} %%%%%%%%%%%%%%%%%%%%%%%%%%%%%%%%%%%%%%%%%%%%%%%%%%%%%%%%%%
%\sideThumbs{Laudes}

\cantusSineNeumas

\rubrica{Absolute incipitur.}

\pars{Psalmus 1.} \scriptura{Ps. 50, 10; \textbf{H393}}

\vspace{-5mm}

\antiphona{I f}{temporalia/ant-exsultabunt.gtex}

%\trLaudAntI

\scriptura{Ps. 50}

\initiumpsalmi{temporalia/ps50-initium-i-f-auto.gtex}

%\psalmusEtTranslatioT{temporalia/ps50-comb.tex}{10cm}
\input{temporalia/ps50.tex}

\antiphona{}{temporalia/ant-exsultabunt.gtex} % repeat the antiphon - new page

\vfill
\pagebreak

\pars{Psalmus 2.} \scriptura{Ps. 64, 3; \textbf{H393}}

\vspace{-5mm}

\antiphona{VIII G}{temporalia/laud-ant2.gtex}

%\trLaudAntII

\vspace{-2mm}

\scriptura{Ps. 64}

\vspace{-2mm}

\initiumpsalmi{temporalia/ps64-initium-viii-G-auto.gtex}

%\psalmusEtTranslatioT{temporalia/ps64-comb.tex}{10cm}
\input{temporalia/ps64.tex} \Abardot{}

%\antiphona{}{temporalia/laud-ant2.gtex} % repeat the antiphon - new page

\vfill
\pagebreak

\pars{Psalmus 3.} \scriptura{Ps. 62, 9; \textbf{H393}}

\vspace{-5mm}

\antiphona{VII c transpos.}{temporalia/ant-mesuscepit.gtex}

%\trLaudAntIII

%\vspace{-3mm}

\scriptura{Ps. 62.}

\initiumpsalmi{temporalia/ps62-initium-vii-C-auto.gtex}

%\vspace{-7mm}

%\psalmusEtTranslatioT{temporalia/ps62-comb.tex}{10cm}
\input{temporalia/ps62.tex} \Abardot{}

%\antiphona{}{temporalia/ant-mesuscepit.gtex} % repeat the antiphon - new page

\vfill
\pagebreak

\pars{Psalmus 4.} \scriptura{Cf. Is. 38, 10.17; \textbf{H225}}

\vspace{-4mm}

\antiphona{II D}{temporalia/laud-ant4.gtex}

%\trLaudAntIV

%\vspace{-2mm}

\scriptura{Canticum Ezechiæ, Is. 38, 10-20}

%\vspace{-2mm}

\initiumpsalmi{temporalia/ezechiae-initium-ii-D-auto.gtex}

%\vspace{-1.5mm}

%\psalmusEtTranslatioT{temporalia/ezechiae-comb.tex}{10cm}
\input{temporalia/ezechiae.tex}

\antiphona{}{temporalia/laud-ant4.gtex} % repeat the antiphon - new page

\vspace{-1cm}

\vfill
\pagebreak

\pars{Psalmus 5.} \scriptura{Ps. 150, 6; \textbf{H393}}

\vspace{-5mm}

\antiphona{VII a}{temporalia/laud-ant5.gtex}

%\trLaudAntV

\scriptura{Ps. 150}

\vspace{-3mm}

\initiumpsalmi{temporalia/ps150-initium-vii-a-auto.gtex}

%\psalmusEtTranslatioT{temporalia/ps150-comb.tex}{10cm}
\input{temporalia/ps150.tex} \Abardot{}

%\antiphona{}{temporalia/laud-ant5.gtex} % repeat the antiphon - new page

\vfill

\pars{Versus (in loco Capituli).} \scriptura{Ap. 14, 13}

% Versus. %%%
\sineinitiali{temporalia/versus-audivi.gtex}

\vfill
\pagebreak

\cantusCumNeumis

\pars{Canticum Zachariæ.} \scriptura{Io. 11, 25.26; \textbf{H393}}

\vspace{-7mm}

\antiphona{II D}{temporalia/ant-egosum.gtex}

%\trAntBenedictus

\vspace{-2mm}

\scriptura{Lc. 1, 68-79}

\vspace{-2mm}

\initiumpsalmi{temporalia/benedictus-initium-ii-D-auto.gtex}

\vspace{-1.5mm}

%\psalmusEtTranslatioT{temporalia/benedictus-comb.tex}{10cm}
\input{temporalia/benedictusiiD.tex} \Abardot{}

%\antiphona{}{temporalia/ant-egosum.gtex} % repeat the antiphon - new page

\vspace{-1cm}

\vfill
\pagebreak

\rubrica{Preces infrascriptæ dicuntur flexis genibus.}

\vspace{2mm}

\pars{Supplicatio Litaniæ.}

\cuminitiali{}{temporalia/supplicatiolitaniae.gtex}

\vspace{2mm}

\pars{Oratio Dominica.}

\cuminitiali{}{temporalia/oratiodominica.gtex}

\vfill
\pagebreak

\pars{Psalmus 6.}

\scriptura{Ps. 129}

\initiumpsalmi{temporalia/ps129-initium-dir-auto.gtex}
%\psalmusEtTranslatioT{temporalia/ps129d-comb.tex}{10cm}
\input{temporalia/ps129d.tex}

\vfill
\pagebreak

\iffalse
\rubrica{Deinde:}

\noindent \Vbardot{} A porta ínferi. \Rbardot{} Erue ánimas eórum.

\noindent \Vbardot{} Requiéscant in pace. \Rbardot{} Amen.

\noindent \Vbardot{} Dómine exáudi oratiónem meam. \Rbardot{} Et clamor meus ad te véniat.

\noindent \Vbardot{} Dóminus vobíscum. \Rbardot{} Et cum spíritu tuo.

\vspace{2mm}

\pars{Oratio}

\grechangedim{spaceabovelines}{2mm}{scalable}
\cuminitiali{}{temporalia/oratio2.gtex}
\grechangedim{spaceabovelines}{0cm}{scalable}
\fi

\pars{Responsorium.} \scriptura{Cf. Lc. 16, 22; \textbf{H389}}

\vspace{-7mm}

\responsorium{IV}{temporalia/resp-subvenite-gn.gtex}{}

\vfill

\vspace{-2mm}

\begin{parcolumns}[nofirstindent=true]{2}
\colchunk{%
\noindent Orémus.

\noindent Absólve, quǽsumus, Dómine, ánimam fámuli tui {\color{red}N.}, ut defúnctus sǽculo tibi vivat: \gredagger{}

\noindent et quæ per fragilitátem carnis humána conversatióne commísit, \grestar{}

\noindent tu vénia misericordíssimæ pietátis abstérge.

\noindent Per Dóminum nostrum Iesum Christum fílium tuum: \gredagger{}

\noindent qui tecum vivit et regnat in unitáte Spíritus Sancti Deus, \grestar{}

\noindent per ómnia sǽcula sæculórum. \Rbardot{} Amen.
}
\colchunk{%
\noindent Orémus.

\noindent Deus, cui próprium est míseri semper et párcere: te súpplices exorámus pro ánima fámuli tui {\color{red}N.}, quam hódie de hoc sǽculo migráre jussísti, \gredagger{}

\noindent ut non tradas eam in manus inimíci, neque obliviscáris in finem, sed júbeas eam a sanctis Angelis súscipi, et ad pátriam paradísi perdúci; \grestar{}

\noindent ut quia in te sperávit et crédidit, non pœnas inférni sustíneat, sed gáudia ætérna possídeat.

\noindent Per Christum Dóminum nostrum. \Rbardot{} Amen.
}
\end{parcolumns}

\vfill

\noindent Fidélium ánimæ per misericórdiam Dei requiéscant in pace. \Rbardot{} Amen.

\vfill
\pagebreak

\rubrica{Post Orationem dicitur (semper plurali numero):}

\noindent \Vbardot{} Réquiem ætérnam dona eis Dómine. \Rbardot{} Et lux perpétua lúceat eis.

\vspace{1cm}
\rubrica{Deinde Cantores:}
\vspace{2mm}

\sineinitiali{temporalia/requiescant.gtex}

\vfill
\pagebreak

\hora{Ad Vesperas.} %%%%%%%%%%%%%%%%%%%%%%%%%%%%%%%%%%%%%%%%%%%%%%%%%%%%%
%\sideThumbs{Vesperæ}

\rubrica{Absolute incipitur ab Antiphona.}

\cantusCumNeumis

\pars{Psalmus 1.} \scriptura{Ps. 114, 9; \textbf{H394}}

\vspace{-5mm}

\antiphona{III b}{temporalia/ant-placebo.gtex}

%\trVespAntI

\scriptura{Ps. 114}

\initiumpsalmi{temporalia/ps114-initium-iii-b-auto.gtex}

%\psalmusEtTranslatioT{temporalia/ps114-comb.tex}{10cm}
\input{temporalia/ps114.tex} \Abardot{}

\vfill
\pagebreak

\pars{Psalmus 2.} \scriptura{Ps. 119, 5; \textbf{H394}}

\vspace{-5mm}

\antiphona{II D}{temporalia/ant-heume.gtex}

%\trVespAntII

%\vspace{-2mm}

\scriptura{Ps. 119}

\initiumpsalmi{temporalia/ps119-initium-ii-D-auto.gtex}

%\vspace{-2mm}

%\psalmusEtTranslatioT{temporalia/ps119-comb.tex}{9cm}
\input{temporalia/ps119.tex} \Abardot{}

\vfill
\pagebreak

\pars{Psalmus 3.} \scriptura{Ps. 120, 7; \textbf{H394}}

\vspace{-5mm}

\antiphona{VIII G\textsuperscript{2}}{temporalia/ant-dominuscustodit.gtex}

%\trVespAntIII

\scriptura{Ps. 120}

\initiumpsalmi{temporalia/ps120-initium-viii-G2-auto.gtex}
%\psalmusEtTranslatioT{temporalia/ps120-comb.tex}{10cm}
\input{temporalia/ps120.tex} \Abardot{}

\vfill
\pagebreak

\pars{Psalmus 4.} \scriptura{Ps. 129, 3; \textbf{H394}}

\vspace{-5mm}

\antiphona{VIII G}{temporalia/vesp-ant4.gtex}

%\trVespAntIV

\scriptura{Ps. 129}

\initiumpsalmi{temporalia/ps129-initium-viii-G-auto.gtex}
%\psalmusEtTranslatioT{temporalia/ps129-comb.tex}{10cm}
\input{temporalia/ps129.tex} \Abardot{}

%\antiphona{}{temporalia/vesp-ant4.gtex} % repeat the antiphon - new page

\vfill
\pagebreak

\pars{Psalmus 5.} \scriptura{Ps. 137, 8; \textbf{H394}}

\vspace{-5mm}

\antiphona{II D}{temporalia/ant-operamanuum.gtex}

%\trVespAntV

\scriptura{Ps. 137}

\initiumpsalmi{temporalia/ps137-initium-ii-D-auto.gtex}
%\psalmusEtTranslatioT{temporalia/ps137-comb.tex}{10cm}
\input{temporalia/ps137.tex} \Abardot{}

\vfill
%\pagebreak

% Capitulum. %%%
\cantusSineNeumas

\pars{Versus (in loco Capituli).} \scriptura{Ap. 14, 13}

% Versus. %%%
\sineinitiali{temporalia/versus-audivi.gtex}

\vfill
%\vspace{2mm}
\pagebreak

\cantusCumNeumis

\pars{Canticum B. Mariæ V.} \scriptura{Io. 6, 37; \textbf{H393}}

\vspace{-5mm}

\antiphona{VII c}{temporalia/ant-omnequoddat.gtex}

%\trAntMagnificat

%\vspace{-5mm}

\scriptura{Lc. 1, 46-55}

\cantusSineNeumas
\initiumpsalmi{temporalia/magnificat-initium-vii-c.gtex}

%\psalmusEtTranslatioT{temporalia/magnificat-comb.tex}{10.3cm}
\input{temporalia/magnificatviic.tex} \Abardot{}

\vfill
\pagebreak

\rubrica{Preces infrascriptæ dicuntur flexis genibus.}

\vspace{2mm}

\pars{Supplicatio Litaniæ.}

\cuminitiali{}{temporalia/supplicatiolitaniae.gtex}

\vspace{2mm}

\pars{Oratio Dominica.}

\cuminitiali{}{temporalia/oratiodominica.gtex}

\vfill
\pagebreak

\pars{Psalmus 6.}

\scriptura{Ps. 145}

\initiumpsalmi{temporalia/ps145-initium-dir-auto.gtex}
%\psalmusEtTranslatioT{temporalia/ps145-comb.tex}{10cm}
\input{temporalia/ps145.tex}

\vfill
\pagebreak

\rubrica{Deinde:}

\noindent \Vbardot{} A porta ínferi. \Rbardot{} Erue ánimas eórum.

\noindent \Vbardot{} Requiéscant in pace. \Rbardot{} Amen.

\noindent \Vbardot{} Dómine exáudi oratiónem meam. \Rbardot{} Et clamor meus ad te véniat.

\noindent \Vbardot{} Dóminus vobíscum. \Rbardot{} Et cum spíritu tuo.

\vspace{2mm}

\pars{Oratio}

\grechangedim{spaceabovelines}{2mm}{scalable}
\cuminitiali{}{temporalia/oratio2.gtex}
\grechangedim{spaceabovelines}{0cm}{scalable}

\rubrica{Post Orationem dicitur (semper plurali numero):}

\noindent \Vbardot{} Réquiem ætérnam dona eis Dómine. \Rbardot{} Et lux perpétua lúceat eis.

\vspace{1cm}
\rubrica{Deinde Cantores:}
\vspace{2mm}

\sineinitiali{temporalia/requiescant.gtex}

\vfill
\pagebreak

\newpage
\pagestyle{empty}

\end{document}
