%%%% Preklady jednotlivych zpevu (nektere se opakuji, a je dobre mit je
% vsechny na jedne hromade)

\newcommand{\trMatLecI}{\translatioCantus{
\hebinitial{ח} Hospodinovo milosrdenství, jež nepomíjí, jeho slitování, jež nekončí. Obnovuje se každého rána, tvá věrnost je neskonalá. ,,Můj podíl je Hospodin,`` praví má duše, proto na něj čekám.\\
\hebinitial{ט} Dobrý je Hospodin k~těm, kdo v~něho naději složí, k~duši, jež se na jeho vůli dotazuje. Je dobré, když člověk potichu čeká na spásu od Hospodina. Dobré je muži, jestliže nosil jho už ve svém mládí.\\
\hebinitial{י} Ať usedne osamocen a~ztichne, když je na něho vložil. Ať položí do prachu svá ústa, snad ještě naděje zbývá. Ať nastaví líce tomu, kdo ho bije, a~potupou se sytí.
Jeruzaléme, Jeruzaléme, obrať se k~Pánu, Bohu svému.}}

\newcommand{\trMatLecII}{\translatioCantus{
\hebinitial{א} Jak zčernalo zlato, změnil se jasný zlatý třpyt! Svaté kameny leží rozmetány po nárožích všech ulic.\\
\hebinitial{ב} Vzácní synové Siónu, cenění nad ryzí zlato, jak jsou pokládáni za hliněné džbány, vyrobené rukou hrnčíře!\\
\hebinitial{ג} I~šakalí matky podají prs, kojí svá mláďata, dcera mého lidu je však krutá jako pštrosové na poušti.\\
\hebinitial{ד} Kojenci se žízní lepí jazyk k~patru, pacholátka prosí o~chléb, a~nikdo jim nenaláme.\\
\hebinitial{ה} Ti, kdo jídali lahůdky, budí úděs na ulicích. Ti, kdo byli chováni v~purpuru, válejí se v~hnoji.\\
\hebinitial{ו} Větší byla nepravost dcery mého lidu nežli hřích Sodomy, která byla podvrácena v~okamžení, aniž ji zasáhla ruka.
Jeruzaléme, Jeruzaléme, obrať se k~Pánu, Bohu svému.}}

\newcommand{\trMatLecIII}{\translatioCantus{
Rozpomeň se, Hospodine, co se nám stalo, popatř a~pohleď na naši potupu!
Naše dědictví připadlo cizákům, naše domy cizincům.
Stali jsme se sirotky, nemáme otce, naše matky jsou jako vdovy.
Vlastní vodu pijeme za stříbro, své dřevo musíme platit.
Jsme pronásledováni, visí nám na šíji, vynakládáme námahu a~nedopřejí nám klidu.
Egyptu podáváme ruku, též Asýrii, abychom se nasytili chlebem.
Naši otcové zhřešili, už nejsou, a~my neseme jejich nepravosti.
Otroci se stali našimi vládci a~nikdo nás z~jejich rukou nevytrhne.
S~nasazením života přinášíme chléb, ohrožováni mečem z~pouště.
Kůže nám žhne jako pec od úporného hladu.
Na Siónu ponižují ženy, v~judských městech panny.
Jeruzaléme, Jeruzaléme, obrať se k~Pánu, Bohu svému.}}

\newcommand{\trMatLecIV}{\translatioCantus{
\textit{Prohlédne ho ten, jenž vidí na dno, do hlubiny člověkova srdce. Říkají si: ,,Kdo nás může vidět?\mbox{}``}
Selhaly jejich \textit{utajené záměry}.
Člověk prohlédl tyto záměry a~strpěl být zadržen jako člověk.
Nebyl by totiž zadržen, ne-li jako člověk, ani viděn, ne-li jako člověk, ani zbit, ne-li jako člověk, ani by nebyl ukřižován a~nezemřel by, ne-li jako člověk.
Člověk tedy přistoupil ke všem těm utrpením, která by proti němu neměla žádnou moc, kdyby nebyl člověkem.
Ale kdyby on nebyl člověkem, člověk by nebyl osvobozen.
Prohlédl utajené hlubiny člověkova srdce, navenek jako pouhý člověk, uvnitř sloužil Bohu.
Ukryl způsob bytí Boha, v~němž je roven Otci, a~vzal na sebe způsob bytí služebníka, v~němž je menší než Otec.}}

\newcommand{\trMatLecV}{\translatioCantus{
Utajené záměry vedly starší lidu až k~tomu, aby postavili stráže ke hrobu, i~když už Pán byl mrtev a~pohřben.
Řekli totiž Pilátovi:
,,Ten podvodník…``
Tímto jménem byl nazýván Pán Ježíš Kristus, k~útěše svých služebníků, až budou i~oni nazýváni podvodníky.
Řekli tedy Pilátovi:
\textit{,,Ten podvodník ještě zaživa prohlásil: ‚po třech dnech vstanu z~mrtvých.‘ 
Dej tedy rozkaz zajistit hrob až do třetího dne.
Jinak by mohli jeho učedníci přijít, ukradnout ho a~říci lidu: ‚Vstal z~mrtvých.‘
Pak by ten poslední  podvod byl ještě horší než první.``
Pilát jim odpověděl: ,,Tady máte stráž.
Jděte a~zajistěte hrob, jak uznáte za dobré.``
Oni šli a~zajistili hrob tím, že  zapečetili kámen a~postavili stráž.}}}

\newcommand{\trMatLecVI}{\translatioCantus{
Postavili stráž ke hrobu.
Otřásla se země, Pán vstal.
Kolem hrobu se udály takové zázraky, že i~vojáci, kteří přišli jako strážní, by se stali svědky, kdyby jen chtěli pravdivě zvěstovat.
Avšak ona hamižnost, která uchvátila Kristova učedníka, uchvátila i~strážce hrobu.
Dáme vám peníze, a~řekněte, že když jste spali, přišli jeho učedníci a~odnesli ho.
Vskutku selhaly utajené záměry.
Co jsi to řekla, ó nešťastná chytrosti?
Což jsi zcela neopustila světlo Božího úradku a~neponořila se do hlubin lstivosti, že říkáš:
,,Říkejte: ‚V noci přišli jeho učedníci a~ukradli ho,  zatímco my jsme spali.‘\mbox{}``
Přivoláváš spící svědky?
Vskutku ty sama jsi usnula, jestliže jsi selhala při zkoumání takové věci.}}

\newcommand{\trMatLecVII}{\translatioCantus{
Ale když přišel Kristus, velekněz, který nám přináší skutečné dobro, neprošel stánkem zhotoveným rukama, to jest patřícím k~tomuto světu, nýbrž stánkem větším a~dokonalejším.
A~nevešel do svatyně s~krví kozlů a~telat, ale jednou provždy dal svou vlastní krev, a~tak nám získal věčné vykoupení.
Jestliže již pokropení krví kozlů a~býků a~popel z~jalovice posvěcuje poskvrněné a~zevně je očišťuje, čím více krev Kristova očistí naše svědomí od mrtvých skutků k~službě živému Bohu!
Vždyť on přinesl sebe sama jako neposkvrněnou oběť Bohu mocí Ducha, který nepomíjí.}}

\newcommand{\trMatLecVIII}{\translatioCantus{
Proto je Kristus prostředníkem nové smlouvy, aby ti, kdo jsou od Boha povoláni, přijali věčné dědictví, které jim bylo zaslíbeno -- neboť jeho smrt přinesla vykoupení z~hříchů, spáchaných za první smlouvy.
Při závěti se musí prokázat smrt toho, kdo ji ustanovil.
Jen závěť zemřelých je totiž platná; nemá však platnost, dokud žije ten, kdo ji ustanovil.
Proto ani první smlouva nebyla uzavřena bez vylití krve.}}

\newcommand{\trMatLecIX}{\translatioCantus{
Když Mojžíš všemu lidu oznámil všecka přikázání podle zákona, vzal krev telat a~kozlů, vodu, červenou vlnu s~yzopem a~pokropil knihu Zákona i~všechen lid.
A~řekl jim:
,,Toto je krev smlouvy, kterou s~vámi uzavřel Bůh.``
Podobně pokropil krví i~stánek a~všecko bohoslužebné náčiní.
Podle zákona se skoro vše očišťuje krví, a~bez vylití krve není odpuštění.}}
