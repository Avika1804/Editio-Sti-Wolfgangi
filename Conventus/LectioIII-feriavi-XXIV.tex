\documentclass[options]{article}
\begin{document}


	\textbf{Ex Enarratiónibus sancti Augustíni epíscopi in psalmos}
	
	\textbf{(En. in ps. 85,4 : CCL 39,1179)}
	
	
	\textit{Custódi ánimam meam, quóniam sanctus sum.}
		Hoc vero :
		\textit{quóniam sanctus sum,}
		néscio utrum potúerit forte álius dicére, nisi ille qui sine peccáto erat in hoc mundo ; peccatórum ómnium non commíssor, sed dimíssor. Agnóscimus vocem dicéntis: 
		\textit{Quóniam sanctus sum, custódi ánimam meam,}
		útique in illa forma servi, quam assúmpserat. Ibi enim caro, ibi et ánima. Neque enim, ut nonnúlli dixérunt, caro sola erat et Verbum; sed et caro, et ánima, et Verbum; et totum hoc unus Fílius Dei, unus Christus, unus Salvátor; in forma Dei æquâlis Patri, in forma servi caput Ecclésiæ.
		
		Ergo 
		\textit{quóniam sanctus sum}
		cum áudio, vocem eius agnósco; et hic séparo meam? Certe inseparabíliter a córpore suo lóquitur, cum sic lóquitur. Et audébo ego dícere: 
		\textit{Quóniam sanctus sum ?}
		Si
		\textit{Sanctus}
		tamquam sanctíficans, et nullo sanctificánte índigens, supérbus et mendax; si autem 
		\textit{sanctus}
		sanctificátus, secúndum id quod dictum est: 
		\textit{Sancti estóte, quia et Ego Sanctus sum,}
		áudeat et corpus Christi, áudeat et unus ille homo clamans a fínibus terræ, cum cápite suo, et sub cápite suo dicére: 
		\textit{Quóniam sanctus sum.}
		Accépit enim grátiam sanctitátis, grátiam baptísmi et remissiónis peccatórum. 
		\textit{Et hæc quidem fuístis,}
		ait Apóstolus, enúmerans multa peccáta, et lévia et grávia, et usitáta et horribília:
		\textit{Et hæc quidem fuístis; sed ablúti estis, sed sanctificáti estis.}
		Si ergo sanctificátos dicit, dicat et unusquísque fidélium:
		\textit{Sanctus sum.}
		Non est ista supérbia eláti, sed conféssio non ingráti.
		
			
	\end{document}
	
