\documentclass[a4paper, twoside, 12pt]{article}
\usepackage[latin]{babel} 
\usepackage[left=3cm, right=1.5cm, top=2cm, bottom=2cm]{geometry} % okraje stranky

\usepackage{fontspec}

\usepackage{color}

\usepackage{luatextra}
\usepackage{graphicx} % support the \includegraphics command and options
\usepackage{gregoriotex} % for gregorio score inclusion

% Here we set the space around the initial.
% Please report to http://home.gna.org/gregorio/gregoriotex/details for more details and options
\setspaceafterinitial{2.2mm plus 0em minus 0em}
\setspacebeforeinitial{2.2mm plus 0em minus 0em}

% Here we set the initial font. Change 43 if you want a bigger initial.
\def\greinitialformat#1{%
{\fontsize{43}{43}\selectfont #1}%
}

% ===================================================================

% Definitions of common commands and environments
% used in the Proprium Antiphonalis Provinciae Pragensis

\newcommand{\modusinfo}[1]{\gresetfirstlineaboveinitial{\small #1}{\small #1}}

\newcommand{\rubrum}[1]{\textcolor{red}{#1}}

\newcommand{\rubrica}[1]{\vspace{3mm} \rubrum{\small #1}}

\newcommand{\notitiaEditorialis}[1]{
  \rubrum{\small \textsc{Notitia editorialis:} #1}
}

% pieces of the title of a feast
\newcommand{\diesFesti}[1]{{\small #1} \vspace{2mm}}
\newcommand{\nomenFesti}[1]{\textbf{\Large #1}\vspace{3mm}
}
\newcommand{\descriptioFesti}[1]{\rubrum{\textbf{#1}}
}
\newcommand{\dignitasFesti}[1]{\rubrum{\small #1}
}

\newenvironment{caputFesti}{\begin{center}}{\end{center}}

% headings of various parts
\newcommand{\hora}[1]{\vspace{1cm}\noindent\textbf{#1}\vspace{5mm}
}
\newcommand{\parsHorae}[1]{\rubrum{#1}\vspace{2mm}}

% often repeated content
\newcommand{\postUltimumResponsoriumNocturniRubrica}[1]{
  \rubrica{Sequitur Gloria Patri in modo ultimi responsorii. Postea repetitur
    altera pars responsi (\textit{#1}).}
}

\newcommand{\EditioSanctiWolfgangiInFine}{
  \vfill

  \begin{center}
    Editio Sancti Wolfgangi \annusPublicationis.
  \end{center}
}
