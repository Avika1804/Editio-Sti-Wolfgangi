Proto na památku požehnané smrti, slavného vzkříšení a~nanebevstoupení tvého Syna, našeho Pána Ježíše Krista, obětujeme, Bože, ke tvé slávě my, tvoji služebníci i~tvůj svatý lid, dar z tvých darů, oběť čistou, oběť svatou, oběť neposkvrněnou: svatý chléb věčného života a~kalich věčné spásy.

Shlédni na ně s~vlídnou a~jasnou tváří. Přijmi je se zalíbením jako oběť svého služebníka, spravedlivého Ábela, jako oběť našeho praotce Abraháma, jako oběť z rukou svého kněze Melchizedecha, jako oběť svatou a~neposkvrněnou.

V pokoře tě prosíme, všemohoucí Bože: Přikaž svému svatému andělu, ať ji přenese na tvůj nebeský oltář, před tvář tvé božské velebnosti. A~nás všechny, kdo máme účast na tomto oltáři a~přijmeme toto svaté tělo a~krev tvého Syna, naplň veškerým nebeským požehnáním a~milostí skrze Krista, našeho Pána. Amen.

Pamatuj také, Bože, na své služebníky a~služebnice, kteří nás předešli se znamením víry a~spí spánkem pokoje (zvláště na {\color{red}N.}).

Do místa občerstvení, světla a~míru uveď všechny, kdo odpočinuli v~Kristu našem Pánu. Amen.

I nám, hříšníkům, kteří ti sloužíme s~důvěrou ve tvé velké slitování, dej podíl a~společenství se svými svatými apoštoly a~mučedníky: Janem, Štěpánem, Matějem, Barnabášem, Ignácem, Alexandrem, Marcelinem, Petrem, Felicitou, Perpetuou, Agátou, Lucií, Anežkou, Cecílií, Anastázií a~se všemi svými svatými. Neposuzuj nás podle skutků, ale přijmi nás do jejich společnosti jako štědrý dárce milosti.

Skrze našeho Pána, Ježíše Krista.

Neboť skrze něho toto všechno stále tvoříš a~je to dobré, všechno posvěcuješ, životem naplňuješ, žehnáš a~nám rozděluješ.

Skrze něho a~s ním a~v něm je tvoje všechna čest a~sláva, Bože Otče všemohoucí, v~jednotě Ducha svatého po všechny věky věků.

\Rbardot{} Amen.
