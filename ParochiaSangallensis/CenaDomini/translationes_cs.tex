%%%% Preklady jednotlivych zpevu (nektere se opakuji, a je dobre mit je
% vsechny na jedne hromade)

\newcommand{\trIntroitus}{\translatioCantus{Naší jedinou chloubou je kříž
našeho Pána Ježíše Krista: v něm je naše naděje, život a vzkříšení, skrze
něj jsme spaseni a vysvobozeni.
{\color{red}\textit{Žl.}} Bože, buď milostiv a žehnej nám, \grestar{} ukaž nám svou
jasnou tvář,~\Abardot{}
kéž se pozná na zemi, jak jednáš, \grestar{} kéž poznají všechny národy, jak
zachraňuješ.~\Abardot{}
Ať tě, Bože, velebí národy, \grestar{} ať tě velebí kde který národ!~\Abardot{}}}

\newcommand{\trGloria}{\translatioCantus{Sláva na výsostech Bohu a na zemi
pokoj lidem dobré vůle. Chválíme tě. Velebíme tě. Klaníme se ti. Oslavujeme
tě. Vzdáváme ti díky pro tvou velikou slávu. Pane a Bože, nebeský Králi,
Bože, Otče všemohoucí. Pane, jednorozený Synu, Ježíši Kriste. Pane a Bože,
Beránku Boží, Synu Otce. Ty, který snímáš hříchy světa, smiluj se nad námi;
ty, který snímáš hříchy světa, přijmi naše prosby. Ty, který sedíš po
pravici Otce, smiluj se nad námi. Neboť ty jediný jsi Svatý, ty jediný jsi
Pán, ty jediný jsi Svrchovaný, Ježíši Kriste, se svatým Duchem ve slávě Boha
Otce. Amen.}}

\newcommand{\trAntI}{\translatioCantus{Když Ježíš vstal od večeře, nalil
vodu do umyvadla a začal učedníkům umývat nohy: a tak jim dal příklad.
{\color{red}\textit{Žl.}} Hospodin je veliký, velmi je hodný chvály
\grestar{} v městě našeho Boha, na jeho svaté hoře. \Abardot{}}}

\newcommand{\trAntII}{\translatioCantus{Po večeři se svými učedníky jim Pán
Ježíš umyl nohy a řekl jim: ,,Chápete, co jsem vám udělal, já, Pán a Mistr?
Dal jsem vám příklad: Jak jsem já udělal vám, tak máte dělat i vy.``
{\color{red}\textit{Žl.}} Byl jsi milostivý, Hospodine, své zemi, \grestar{}
změnil jsi k dobru Jakubův úděl. \Abardot{}}}

\newcommand{\trAntIII}{\translatioCantus{Pane, ty mi chceš mýt nohy? Ježíš
mu odpověděl: Jestliže tě neumyji, nebudeš mít se mnou podíl.
\Vbardot{} Přišel k Šimonu Petrovi a ten mu řekl:
\Abardot{} Pane…
\Vbardot{} Co já dělám, tomu ty nyní ještě nemůžeš rozumět, pochopíš to však později.
\Abardot{} Pane…}}

\newcommand{\trAntIV}{\translatioCantus{Jestliže jsem vám umyl nohy, já, váš
Pán a Mistr, máte také vy jeden druhému umývat nohy.
{\color{red}\textit{Žl.}} Slyšte to, všechny národy, \grestar{}
poslouchejte, všichni obyvatelé světa. \Abardot{}}}

\newcommand{\trAntV}{\translatioCantus{Podle toho všichni poznají, že jste
moji učedníci, budete-li mít lásku k sobě navzájem.
\Vbardot{} Ježíš řekl svým učedníkům:
\Abardot{} Podle…}}

\newcommand{\trAntVI}{\translatioCantus{Nové přikázání vám dávám: Milujte se
navzájem, jak jsem já miloval vás praví Pán.
{\color{red}\textit{Žl.}} Blaze těm, jejichž cesta je bezúhonná, \grestar{}
kteří kráčejí v zákoně Hospodinově. \Abardot{}}}

\newcommand{\trAntVII}{\translatioCantus{Ať ve vás trvá víra, naděje
a láska, tato trojice. Ale největší z nich je láska.
\Vbardot{} Nyní trvá víra, naděje a láska, tato trojice. \grestar{}
Ale největší z nich je láska. \Abardot{} Ať…}}

\newcommand{\trUbiCaritas}{\translatioCantus{Kde je opravdová láska, tam přebývá Bůh.
\Vbardot{} Spojila nás vjedno láska Krista Pána.
\Vbardot{} Jásejme a hledejme jen v něm svou radost!
\Vbardot{} Boha živého se bojme, milujme ho!
\Vbardot{} Upřímně se navzájem vždy mějme rádi!
\Abardot{} Kde…
\Vbardot{} Tvoříme-li tedy jedno společenství,
\Vbardot{} varujme se všeho, co nás vnitřně dělí,
\Vbardot{} nechme nerozumných hádek, nechme sporů,
\Vbardot{} ať je Kristus jako Bůh náš mezi námi!
\Abardot{} Kde…
\Vbardot{} Kéž pak jednou s blaženými patřit smíme
\Vbardot{} ve slávě na tvář tvou, Kriste, Boží Synu,
\Vbardot{} v dokonalé radosti nad pomyšlení,
\Vbardot{} bez mezí a bez konce na věčné věky. Amen.}}

\newcommand{\trDexteraDomini}{\translatioCantus{Hospodinova pravice mocně
zasáhla, Hospodinova pravice mě pozvedla. {\color{red}\textit{Žl.}} 
Děkuji ti, žes mě vyslyšel \grestar{} a stal se mou spásou. \Abardot{}
Kámen, který stavitelé zavrhli, \grestar{} stal se kvádrem nárožním. \Abardot{}
Hospodinovým řízením se tak stalo, \grestar{} je to podivuhodné v našich očích. \Abardot{}
Seřaďte průvod s hojnými ratolestmi \grestar{} až k rohům oltáře! \Abardot{}}}

\newcommand{\trSanctus}{\translatioCantus{Svatý, \grestar{} Svatý, Svatý, Pán Bůh
zástupů. Nebe i země jsou plny tvé slávy. Hosana na výsostech. Požehnaný,
jenž přichází ve jménu Páně. Hosana na výsostech.}}

\newcommand{\trMysteriumFidei}{\translatioCantus{Tajemství víry: \Rbardot{} Tvou smrt
zvěstujeme, tvé vzkříšení vyznáváme, na tvůj příchod čekáme, Pane Ježíši Kriste.}}

\newcommand{\trOratioDominica}{\translatioCantus{Otče náš, jenž jsi na
nebesích, posvěť se jméno tvé. Přijď království tvé. Buď vůle tvá jako v~nebi,
tak i~na zemi. Chléb náš vezdejší dej nám dnes. A~odpusť nám naše
viny, jako i~my odpouštíme našim viníkům. A~neuveď nás v pokušení, ale zbav
nás od zlého.}}

\newcommand{\trQuia}{\translatioCantus{Neboť tvé je království, i moc i sláva na věky.}}

\newcommand{\trPaxInCaelo}{\translatioCantus{Pokoj na nebi, \grestar{} pokoj na zemi, pokoj všemu lidu, pokoj kněžím Boží církve.
{\color{red}\textit{Žl.}} Vyprošujte Jeruzalému pokoj:~\grestar{}
ať jsou v~bezpečí, kdo tě milují,~\Abardot{}
ať vládne mír v~tvých hradbách,~\grestar{}
bezpečnost v~tvých palácích!~\Abardot{}}}

\newcommand{\trAgnus}{\translatioCantus{Beránku Boží, \grestar{} který snímáš hříchy světa, smiluj se nad námi.
Beránku Boží, \grestar{} který snímáš hříchy světa, smiluj se nad námi.
Beránku Boží, \grestar{} který snímáš hříchy světa, daruj nám pokoj.}}

\newcommand{\trHocCorpus}{\translatioCantus{Toto je moje tělo, které se za
vás vydává; tento kalich je nová smlouva, potvrzená mou krví, praví Pán,
kdykoli z něho budete pít, čiňte to na mou památku.}}

\newcommand{\trCalicem}{\translatioCantus{Vezmu kalich spásy a budu vzývat
jméno Hospodinovo.
{\color{red}\textit{Žl.}} Měl jsem důvěru, i když jsem si řekl:
,,Jsem tak sklíčen!\mbox{}`` Čím se odplatím Hospodinu za všechno, co mi prokázal?
Splním své sliby Hospodinu před veškerým jeho lidem. Drahocenná je
v Hospodinových očích smrt jeho zbožných. Ach, Hospodine, jsem tvůj
služebník, jsem tvůj služebník, syn tvé služebnice, rozvázal jsi moje
pouta. Přinesu ti oběť díků, Hospodine, a budu vzývat tvé jméno. Splním
své sliby Hospodinu před veškerým jeho lidem v nádvořích domu Hospodinova,
uprostřed tebe, Jeruzaléme!}}

\newcommand{\trAdoroTe}{\translatioCantus{Klaním se ti vroucně, skrytý Bože náš,
jenž tu ve svátosti sebe ukrýváš.
Tobě srdcem svým se zcela poddávám,
před Tebou svou slabost, Bože, vyznávám.
{\color{red}\textit{2.}} Zrakem, hmatem, chutí tebe nevnímám,
a jen sluchem svým Tě jistě poznávám.
Věřím vše, co hlásal světu Kristus Pán,
v něm je základ pravdy lidstvu všemu dán.
Na kříži jsi tajil jenom božství své.
{\color{red}\textit{3.}} Zde je také skryto člověčenství tvé.
Obojí však věřím celým srdcem svým,
o milost Tě prosím s lotrem kajícím.
{\color{red}\textit{4.}} Rány tvé jak Tomáš vidět nežádám;
že jsi Pán a Bůh můj vroucně vyznávám.
Rač mé chabé víře větší sílu dát,
více v Tebe doufat, víc tě milovat.
{\color{red}\textit{5.}} Plode smrti Páně, Chlebe života,
v němž se lidem dává Boží dobrota,
dej mé duši stále jenom z tebe žít,
v tobě Boží lásku rozjímat a ctít.
{\color{red}\textit{6.}} Dobrý Pelikáne, Jezu, Pane můj,
krví svou nás hříšné z hříchů očišťuj,
vždyť jediná krůpěj její stačila,
aby všeho světa viny obmyla.
{\color{red}\textit{7.}} Rač, ó Jezu Kriste, jenž jsi nám zde skryt,
srdci toužícímu touhu vyplnit,
abych, jak zde hledím s vírou na oltář,
v nebesích Tě jednou spatřil tváří v tvář. Amen.}}

\newcommand{\trPangeLingua}{\translatioCantus{Chvalte, ústa, vznešeného
těla Páně tajemství,
chvalte předrahou krev jeho,
kterou z milosrdenství,
zrozen z lůna panenského,
prolil Král všech království.
{\color{red}\textit{2.}} Nám byl dán a nám se zrodil
z Panny neporušené,
jako poutník světem chodil,
sil zde zrno pravdy své,
skvělým řádem vyvrcholil
svoje žití pozemské.
{\color{red}\textit{3.}} Na poslední večer k stolu
s bratřími když zasedal,
beránka by s nimi spolu
podle zvyku pojídal,
dvanácteru apoštolů
za pokrm sám sebe dal.
{\color{red}\textit{4.}} Slovo v těle slovu dává
moc chléb v tělo proměnit,
z vína Boží krev se stává.
I když Pán je smyslům skryt,
stačí sama víra pravá
srdce čisté přesvědčit.
{\color{red}\textit{5.}} Před svátostí touto slavnou
v úctě skloňme kolena,
bohoslužbu starodávnou
nahraď nová, vznešená,
doplň smysly, které slábnou,
víra s láskou spojená.
{\color{red}\textit{6.}} Otci, Synu bez ustání
chvála nadšená ať zní,
sláva, pocta, díkůvzdání
naše chvály provází.
Duchu též ať vše se klaní,
který z obou vychází.
Amen.}}
