% LuaLaTeX

\documentclass[a4paper, twoside, 12pt]{article}
\usepackage[latin]{babel}
%\usepackage[landscape, left=3cm, right=1.5cm, top=2cm, bottom=1cm]{geometry} % okraje stranky
\usepackage[portrait, a4paper, mag=1500, truedimen, left=0.8cm, right=0.8cm, top=0.8cm, bottom=0.8cm]{geometry} % okraje stranky

\usepackage{fontspec}
\setmainfont[FeatureFile={junicode.fea}, Ligatures={Common, TeX}, RawFeature=+fixi]{Junicode}
%\setmainfont{Junicode}

% shortcut for Junicode without ligatures (for the Czech texts)
\newfontfamily\nlfont[FeatureFile={junicode.fea}, Ligatures={Common, TeX}, RawFeature=+fixi]{Junicode}

\usepackage{multicol}
\usepackage{color}
\usepackage{lettrine}
\usepackage{fancyhdr}

% usual packages loading:
\usepackage{luatextra}
\usepackage{graphicx} % support the \includegraphics command and options
\usepackage{gregoriotex} % for gregorio score inclusion
\usepackage{gregoriosyms}
\usepackage{wrapfig} % figures wrapped by the text
\usepackage{parcolumns}
\usepackage[contents={},opacity=1,scale=1,color=black]{background}
\usepackage{tikzpagenodes}
\usepackage{calc}
\usepackage{longtable}

\setlength{\headheight}{12pt}

% Commands used to produce a typical "Conventus" booklet

\newenvironment{titulusOfficii}{\begin{center}}{\end{center}}
\newcommand{\dies}[1]{#1

}
\newcommand{\nomenFesti}[1]{\textbf{\Large #1}

}
\newcommand{\celebratio}[1]{#1

}

\newcommand{\hora}[1]{%
\vspace{0.5cm}{\large \textbf{#1}}

\fancyhead[LE]{\thepage\ / #1}
\fancyhead[RO]{#1 / \thepage}
\addcontentsline{toc}{subsection}{#1}
}

% larger unit than a hora
\newcommand{\divisio}[1]{%
\begin{center}
{\Large \textsc{#1}}
\end{center}
\fancyhead[CO,CE]{#1}
\addcontentsline{toc}{section}{#1}
}

% a part of a hora, larger than pars
\newcommand{\subhora}[1]{
\begin{center}
{\large \textit{#1}}
\end{center}
%\fancyhead[CO,CE]{#1}
\addcontentsline{toc}{subsubsection}{#1}
}

% rubricated inline text
\newcommand{\rubricatum}[1]{\textit{#1}}

% standalone rubric
\newcommand{\rubrica}[1]{\vspace{3mm}\rubricatum{#1}}

\newcommand{\notitia}[1]{\textcolor{red}{#1}}

\newcommand{\scriptura}[1]{\hfill \small\textit{#1}}

\newcommand{\translatioCantus}[1]{\vspace{1mm}%
{\noindent\footnotesize \nlfont{#1}}}

% pruznejsi varianta nasledujiciho - umoznuje nastavit sirku sloupce
% s prekladem
\newcommand{\psalmusEtTranslatioB}[3]{
  \vspace{0.5cm}
  \begin{parcolumns}[colwidths={2=#3}, nofirstindent=true]{2}
    \colchunk{
      \input{#1}
    }

    \colchunk{
      \vspace{-0.5cm}
      {\footnotesize \nlfont
        \input{#2}
      }
    }
  \end{parcolumns}
}

\newcommand{\psalmusEtTranslatio}[2]{
  \psalmusEtTranslatioB{#1}{#2}{8.5cm}
}


\newcommand{\canticumMagnificatEtTranslatio}[1]{
  \psalmusEtTranslatioB{#1}{temporalia/extra-adventum-vespers/magnificat-boh.tex}{12cm}
}
\newcommand{\canticumBenedictusEtTranslatio}[1]{
  \psalmusEtTranslatioB{#1}{temporalia/extra-adventum-laudes/benedictus-boh.tex}{10.5cm}
}

% volne misto nad antifonami, kam si zpevaci dokresli neumy
\newcommand{\hicSuntNeumae}{\vspace{0.5cm}}

% prepinani mista mezi notovymi osnovami: pro neumovane a neneumovane zpevy
\newcommand{\cantusCumNeumis}{
  \setgrefactor{17}
  \global\advance\grespaceabovelines by 5mm%
}
\newcommand{\cantusSineNeumas}{
  \setgrefactor{17}
  \global\advance\grespaceabovelines by -5mm%
}

% znaky k umisteni nad inicialu zpevu
\newcommand{\superInitialam}[1]{\gresetfirstlineaboveinitial{\small {\textbf{#1}}}{\small {\textbf{#1}}}}

% pars officii, i.e. "oratio", ...
\newcommand{\pars}[1]{\textbf{#1}}

\newenvironment{psalmus}{
  \setlength{\parindent}{0pt}
  \setlength{\parskip}{5pt}
}{
  \setlength{\parindent}{10pt}
  \setlength{\parskip}{10pt}
}

%%%% Prejmenovat na latinske:
\newcommand{\nadpisZalmu}[1]{
  \hspace{2cm}\textbf{#1}\vspace{2mm}%
  \nopagebreak%

}

% mode, score, translation
\newcommand{\antiphona}[3]{%
\hicSuntNeumae
\superInitialam{#1}
\includescore{#2}

#3
}
 % Often used macros

\setlength{\columnsep}{15pt} % prostor mezi sloupci

%%%%%%%%%%%%%%%%%%%%%%%%%%%%%%%%%%%%%%%%%%%%%%%%%%%%%%%%%%%%%%%%%%%%%%%%%%%%%%%%%%%%%%%%%%%%%%%%%%%%%%%%%%%%%
\begin{document}

% Here we set the space around the initial.
% Please report to http://home.gna.org/gregorio/gregoriotex/details for more details and options
\grechangedim{afterinitialshift}{2.2mm}{scalable}
\grechangedim{beforeinitialshift}{2.2mm}{scalable}
\grechangedim{interwordspacetext}{0.32 cm plus 0.15 cm minus 0.05 cm}{scalable}%
\grechangedim{annotationraise}{-0.2cm}{scalable}

% Here we set the initial font. Change 38 if you want a bigger initial.
% Emit the initials in red.
\grechangestyle{initial}{\color{red}\fontsize{36}{36}\selectfont}

\renewcommand{\headrulewidth}{0pt} % no horiz. rule at the header
\pagestyle{empty}

\grechangedim{spaceabovelines}{0.2cm}{scalable}%

\cantusSineNeumas

\vfill

\hbox{}

\vspace{5cm}

\begin{center}
{\huge \textit{AD II VESPERAS}}

\vspace{0.5cm}

{\huge \textsc{Dominica VIII Per Annum}}

\vspace{0.25cm}

{\textsc{(Anno C)}}

\vspace{0.5cm}

{\textsc{Secundum Antiphonale Romanum II}}
\end{center}

\vfill
\scriptura{}
\pagebreak

\pars{Introductio}

\gregorioscore{temporalia/deusinadiutorium-communis.gtex}

\vspace{0.5cm}

\pars{Hymnus}

\vspace{-0.3cm}

\antiphona{VIII}{temporalia/hym-OLuxBeata.gtex}

\vspace{0.5cm}

\vfill
\pagebreak

\pars{Antiphona I} \scriptura{Ps. 109, 3}

\antiphona{VIII g}{temporalia/ant-exuteroanteluciferum.gtex}

\vspace{0.5cm}

\pars{Psalmus} \scriptura{Ps. 109}

{
\grechangedim{interwordspacetext}{0.24 cm plus 0.15 cm minus 0.05 cm}{scalable}%
\gregorioscore{temporalia/ps109-initium-viii-G-auto.gtex}
\grechangedim{interwordspacetext}{0.32 cm plus 0.15 cm minus 0.05 cm}{scalable}%
}

{\setlength{\parindent}{0pt}\input{temporalia/ps109.tex}}

\vfill
\pagebreak

\pars{Antiphona II} \scriptura{Ps. 111, 1; \textbf{H92}}

\antiphona{IV* e}{temporalia/ant-inmandatiseius.gtex}

\vspace{0.5cm}

\pars{Psalmus} \scriptura{Ps. 111}

\gregorioscore{temporalia/ps111-initium-iv_-e-auto.gtex}

{\setlength{\parindent}{0pt}\input{temporalia/ps111.tex}}

\vfill
\pagebreak

\pars{Canticum} \scriptura{Cf. Ap. 19, 1-2.5-7}

\antiphona{d}{temporalia/canticum-ap19-d.gtex}

\vfill
\pagebreak

\pars{Lectio brevis} \scriptura{Hebr. 12, 22-24}

Accessístis ad Sion montem et civitátem Dei vivéntis, Ierúsalem cæléstem, et
multa mília angelórum, frequéntiam et ecclésiam primogenitórum, qui
conscrípti sunt in cælis, et iúdicem Deum ómnium, et spíritus iustórum, qui
consummáti sunt, et testaménti novi mediatórem Iesum, et sánguinem
aspersiónis, mélius loquéntem quam Abel.

\vspace{0.5cm}

\pars{Responsorium breve} \scriptura{Ps. 146, 5}

\vspace{-0.3cm}

\antiphona{\oldrbar.~ br.}{temporalia/resp-magnusdominusnoster.gtex}

\vfill
\pagebreak

\pars{Antiphona ad Magnificat} \scriptura{Mc. 7, 18}

\vspace{-0.3cm}

\antiphona{VIII g}{temporalia/ant-nonpotestarborbona.gtex}

\vspace{0.5cm}

\pars{Canticum Evangelicum} \scriptura{Lc. 1, 46-55}

\gresetinitiallines{0}
\gregorioscore{temporalia/magnificat-initium-viiisoll-g.gtex}

{\setlength{\parindent}{0pt}\input{temporalia/magnificat.tex}}

\vfill
\pagebreak

\pars{Preces}

{\setlength{\parindent}{0pt}
Gaudéntes in Dómino, a quo omne bonum descéndit, ipsum sincéris méntibus \textit{depre}\textbf{cé}mur:

\vspace{0.2cm}

\Rbardot{} Dómine, exáudi oratió\textit{nem} \textbf{nos}tram.

\vspace{0.2cm}

Universórum Pater et Dómine, qui Fílium tuum in mundum misísti, ut nomen tuum glorificarétur in \textit{omni} \textbf{lo}co,\\
-- testimónium Ecclésiæ tuæ róbora a\textit{pud} \textbf{gen}tes. \Rbardot{}

Fac nos Apostolórum prædicati\textit{óni} \textbf{dó}ciles,\\
-- et fídei nostræ veritáti \textit{con}\textbf{fór}mes. \Rbardot{}

Qui dí\textit{ligis} \textbf{ius}tos,\\
-- iudícium fac iniúriam pa\textit{ti}\textbf{én}tibus. \Rbardot{}

Compedítos solve, cæ\textit{cos} \textit{il}\textbf{lú}mina,\\
-- elísos érige, ádvenas \textit{cus}\textbf{tó}di. \Rbardot{}

In tua iam dormiéntium pace votum complé\textit{re} \textit{di}\textbf{gné}ris:\\
-- da eos per Fílium tuum ad resurrectiónem sanctam per\textit{ve}\textbf{ní}re. \Rbardot{}
}

\vspace{0.5cm}

\pars{Oratio conclusiva}

Da nobis, quǽsumus, Dómine,~\gredagger{}
ut et mundi cursus pacífico nobis tuo órdine dirigátur~\grestar{}
et Ecclésia tua tranquílla devotióne lætétur.
Per Dóminum nostrum Iesum Christum, Fílium tuum,
qui tecum vivit et regnat in unitáte Spíritus Sancti, Deus,
per ómnia sǽcula sæculórum. Amen.

\end{document}
