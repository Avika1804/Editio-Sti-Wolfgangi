% Commands used to produce a typical "Conventus" booklet

\newenvironment{titulusOfficii}{\begin{center}}{\end{center}}
\newcommand{\dies}[1]{#1

}
\newcommand{\nomenFesti}[1]{\textbf{\Large #1}

}
\newcommand{\celebratio}[1]{#1

}

%%%% Prejmenovat na latinske
\newcounter{Thumb}
\setcounter{Thumb}{0}
\newcommand{\thumbTitulus}{}

\newif\ifSideThumbsEnabled

\newlength\ThumbSize
\setlength\ThumbSize{2cm}

\SideThumbsEnabledfalse
\AddEverypageHook{
\ifSideThumbsEnabled
\ifodd\value{page}
\backgroundsetup{
angle=90,
position={current page.east|-current page text area.north east},
vshift=12pt,
hshift=-\value{Thumb}*\ThumbSize+\ThumbSize+\ThumbSize/4,
contents={\tikz\node[fill=gray!30,anchor=west,text width=\ThumbSize,
 align=center,text height=14pt,text depth=30pt]
{\small\thumbTitulus};
}
}
\else
\backgroundsetup{
angle=90,
position={current page.west|-current page text area.north west},
vshift=-16pt,
hshift=-\value{Thumb}*\ThumbSize+\ThumbSize+\ThumbSize/4,
contents={\tikz\node[fill=gray!30,anchor=west,text width=\ThumbSize,
 align=center,text height=28pt,text depth=6pt]
{\small\thumbTitulus};
}
}
\fi
\BgMaterial
\else\relax\fi}
\newcommand\sideThumbs[1]{\SideThumbsEnabledtrue%
\renewcommand{\thumbTitulus}{#1}%
\stepcounter{Thumb}
}
\newcommand\RemoveSideThumbs{\SideThumbsEnabledfalse}

\newcommand{\hora}[1]{%
\vspace{0.5cm}{\large \textbf{#1}}

\fancyhead[LE]{\thepage\ / #1}
\fancyhead[RO]{#1 / \thepage}
\addcontentsline{toc}{subsection}{#1}
}

% In Junicode, the default R/ with .35em has the bar too close to the
% letter.
\gresimpledefbarglyph{R}{.41em}

% Emit R/, V/, A/ and + in red
\let\oldrbar\Rbar
\let\oldvbar\Vbar
\let\oldabar\Abar
\let\oldgrecross\grecross
\let\oldgrestar\grestar
\let\oldgredagger\gredagger
\let\olddag\dag
\let\oldddag\ddag
\renewcommand{\Rbar}{{\color{red}\oldrbar}}
\renewcommand{\Vbar}{{\color{red}\oldvbar}}
\renewcommand{\Abar}{{\color{red}\oldabar}}
\renewcommand{\grecross}{{\color{red}\oldgrecross}}
\renewcommand{\grestar}{{\color{red}\oldgrestar}}
\renewcommand{\gredagger}{{\color{red}\oldgredagger}}
\renewcommand{\dag}{{\color{red}\olddag}}
\renewcommand{\ddag}{{\color{red}\oldddag}}
\newcommand{\Rbardot}{{\color{red}\oldrbar.~}}
\newcommand{\Vbardot}{{\color{red}\oldvbar.~}}
\newcommand{\Abardot}{{\color{red}\oldabar.~}}
\newcommand{\Psdot}{{\color{red}Ps.~}}

\let\quasiHora\hora

% larger unit than a hora
\newcommand{\divisio}[1]{%
\begin{center}
{\Large \textsc{#1}}
\end{center}
\fancyhead[CO,CE]{#1}
\addcontentsline{toc}{section}{#1}
}

% a part of a hora, larger than pars
\newcommand{\subhora}[1]{
\begin{center}
{\large \textit{#1}}
\end{center}
%\fancyhead[CO,CE]{#1}
\addcontentsline{toc}{subsubsection}{#1}
}

% rubricated inline text
\newcommand{\rubricatum}[1]{\textit{#1}}

% standalone rubric
\newcommand{\rubrica}[1]{\vspace{3mm}\rubricatum{#1}}

\newcommand{\notitia}[1]{\textcolor{red}{#1}}

\newcommand{\scriptura}[1]{\hfill \small\textit{#1}}

\newcommand{\translatioCantus}[1]{\vspace{1mm}%
{\noindent\footnotesize \nlfont{#1}}}

\newenvironment{translatioMulticol}[1]{
  \begin{multicols}{#1}
    \footnotesize
    \setlength{\parindent}{0pt}
    \setlength{\columnseprule}{0pt}
}{\end{multicols}}

% pruznejsi varianta nasledujiciho - umoznuje nastavit sirku sloupce
% s prekladem
\newcommand{\psalmusEtTranslatioB}[3]{
  \vspace{0.5cm}
  \begin{parcolumns}[colwidths={2=#3}, nofirstindent=true]{2}
    \colchunk{
      \input{#1}
    }

    \colchunk{
      \vspace{-0.5cm}
      {\footnotesize \nlfont
        \input{#2}
      }
    }
  \end{parcolumns}
}

\newcommand{\psalmusEtTranslatio}[2]{
  \psalmusEtTranslatioB{#1}{#2}{8.5cm}
}

\newcommand{\psalmusTr}[1]{{\footnotesize\nlfont\hspace*{-4pt} #1}}
\newcommand{\psalmusEtTranslatioT}[2]{{
  \setlength{\LTleft}{-0.25cm}
  \setlength{\LTright}{\fill}
  \setlength{\tabcolsep}{0.35cm}
  \begin{psalmus}
      \noindent
      \begin{longtable}{p{\textwidth-#2-1cm}p{#2}}
        \input{#1}
      \end{longtable}
  \end{psalmus}
}}
\newcommand{\psalmusEtTranslatioTS}[2]{{
  \setlength{\LTleft}{-0.25cm}
  \setlength{\LTright}{\fill}
  \setlength{\tabcolsep}{0.25cm}
  \begin{psalmus}
      \noindent
      \begin{longtable}{p{\textwidth-#2-0.5cm}p{#2}}
        \input{#1}
      \end{longtable}
  \end{psalmus}
}}

\newcommand{\canticumMagnificatEtTranslatio}[1]{
  \psalmusEtTranslatioB{#1}{temporalia/extra-adventum-vespers/magnificat-boh.tex}{12cm}
}
\newcommand{\canticumBenedictusEtTranslatio}[1]{
  \psalmusEtTranslatioB{#1}{temporalia/extra-adventum-laudes/benedictus-boh.tex}{10.5cm}
}

\newcommand{\textusEtTranslatio}[3]{
  \begin{parcolumns}[colwidths={2=#3}, nofirstindent=true]{2}
    \colchunk{
      #1
    }

    \colchunk{
      \vspace{-0.5cm}
      {\footnotesize \nlfont
        #2
      }
    }
  \end{parcolumns}
}

% volne misto nad antifonami, kam si zpevaci dokresli neumy
\newcommand{\hicSuntNeumae}{\vspace{0.5cm}}

% prepinani mista mezi notovymi osnovami: pro neumovane a neneumovane zpevy
\newcommand{\cantusCumNeumis}{
  \setgrefactor{14}
%  \global\advance\grespaceabovelines by 5mm%
%  \global\advance\grespaceabovelines by 2mm%
}
\newcommand{\cantusSineNeumas}{
  \setgrefactor{14}
%  \global\advance\grespaceabovelines by -5mm%
  \grechangedim{spaceabovelines}{0cm}{1}%
}

\renewcommand{\greabovelinestextstyle}[1]{%
{\color{red}\textit{#1}}\relax %
}

% znaky k umisteni nad inicialu zpevu
\newcommand{\superInitialam}[1]{\gresetfirstlineaboveinitial{\footnotesize {\textbf{#1}}}{\footnotesize {\textbf{#1}}}}

% pars officii, i.e. "oratio", ...
\newcommand{\pars}[1]{\textbf{#1}}

\newenvironment{psalmus}{
  \setlength{\parindent}{0pt}
  \setlength{\parskip}{5pt}
}{
  \setlength{\parindent}{10pt}
  \setlength{\parskip}{10pt}
}

%%%% Prejmenovat na latinske:
\newcommand{\nadpisZalmu}[1]{
  \hspace{2cm}\textbf{#1}\vspace{2mm}%
  \nopagebreak%

}

% mode, score, translation
\newcommand{\antiphona}[3]{%
\hicSuntNeumae
\superInitialam{#1}
\includescore{#2}

#3
}

% a semantic alias of the previous
\newcommand{\responsorium}[3]{\antiphona{#1}{#2}{#3}}
