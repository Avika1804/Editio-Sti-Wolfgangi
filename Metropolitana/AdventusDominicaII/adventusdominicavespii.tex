% LuaLaTeX

\documentclass[a4paper, twoside, 12pt]{article}
\usepackage[latin]{babel}
%\usepackage[landscape, left=3cm, right=1.5cm, top=2cm, bottom=1cm]{geometry} % okraje stranky
\usepackage[portrait, a4paper, mag=1500, truedimen, left=0.8cm, right=0.8cm, top=0.8cm, bottom=0.8cm]{geometry} % okraje stranky

\usepackage{fontspec}
\setmainfont[FeatureFile={junicode.fea}, Ligatures={Common, TeX}, RawFeature=+fixi]{Junicode}
%\setmainfont{Junicode}

% shortcut for Junicode without ligatures (for the Czech texts)
\newfontfamily\nlfont[FeatureFile={junicode.fea}, Ligatures={Common, TeX}, RawFeature=+fixi]{Junicode}

\usepackage{multicol}
\usepackage{color}
\usepackage{lettrine}
\usepackage{fancyhdr}

% usual packages loading:
\usepackage{luatextra}
\usepackage{graphicx} % support the \includegraphics command and options
\usepackage{gregoriotex} % for gregorio score inclusion
\usepackage{gregoriosyms}
\usepackage{wrapfig} % figures wrapped by the text
\usepackage{parcolumns}
\usepackage[contents={},opacity=1,scale=1,color=black]{background}
\usepackage{tikzpagenodes}
\usepackage{calc}
\usepackage{longtable}

\setlength{\headheight}{12pt}

% Commands used to produce a typical "Conventus" booklet

\newenvironment{titulusOfficii}{\begin{center}}{\end{center}}
\newcommand{\dies}[1]{#1

}
\newcommand{\nomenFesti}[1]{\textbf{\Large #1}

}
\newcommand{\celebratio}[1]{#1

}

\newcommand{\hora}[1]{%
\vspace{0.5cm}{\large \textbf{#1}}

\fancyhead[LE]{\thepage\ / #1}
\fancyhead[RO]{#1 / \thepage}
\addcontentsline{toc}{subsection}{#1}
}

% larger unit than a hora
\newcommand{\divisio}[1]{%
\begin{center}
{\Large \textsc{#1}}
\end{center}
\fancyhead[CO,CE]{#1}
\addcontentsline{toc}{section}{#1}
}

% a part of a hora, larger than pars
\newcommand{\subhora}[1]{
\begin{center}
{\large \textit{#1}}
\end{center}
%\fancyhead[CO,CE]{#1}
\addcontentsline{toc}{subsubsection}{#1}
}

% rubricated inline text
\newcommand{\rubricatum}[1]{\textit{#1}}

% standalone rubric
\newcommand{\rubrica}[1]{\vspace{3mm}\rubricatum{#1}}

\newcommand{\notitia}[1]{\textcolor{red}{#1}}

\newcommand{\scriptura}[1]{\hfill \small\textit{#1}}

\newcommand{\translatioCantus}[1]{\vspace{1mm}%
{\noindent\footnotesize \nlfont{#1}}}

% pruznejsi varianta nasledujiciho - umoznuje nastavit sirku sloupce
% s prekladem
\newcommand{\psalmusEtTranslatioB}[3]{
  \vspace{0.5cm}
  \begin{parcolumns}[colwidths={2=#3}, nofirstindent=true]{2}
    \colchunk{
      \input{#1}
    }

    \colchunk{
      \vspace{-0.5cm}
      {\footnotesize \nlfont
        \input{#2}
      }
    }
  \end{parcolumns}
}

\newcommand{\psalmusEtTranslatio}[2]{
  \psalmusEtTranslatioB{#1}{#2}{8.5cm}
}


\newcommand{\canticumMagnificatEtTranslatio}[1]{
  \psalmusEtTranslatioB{#1}{temporalia/extra-adventum-vespers/magnificat-boh.tex}{12cm}
}
\newcommand{\canticumBenedictusEtTranslatio}[1]{
  \psalmusEtTranslatioB{#1}{temporalia/extra-adventum-laudes/benedictus-boh.tex}{10.5cm}
}

% volne misto nad antifonami, kam si zpevaci dokresli neumy
\newcommand{\hicSuntNeumae}{\vspace{0.5cm}}

% prepinani mista mezi notovymi osnovami: pro neumovane a neneumovane zpevy
\newcommand{\cantusCumNeumis}{
  \setgrefactor{17}
  \global\advance\grespaceabovelines by 5mm%
}
\newcommand{\cantusSineNeumas}{
  \setgrefactor{17}
  \global\advance\grespaceabovelines by -5mm%
}

% znaky k umisteni nad inicialu zpevu
\newcommand{\superInitialam}[1]{\gresetfirstlineaboveinitial{\small {\textbf{#1}}}{\small {\textbf{#1}}}}

% pars officii, i.e. "oratio", ...
\newcommand{\pars}[1]{\textbf{#1}}

\newenvironment{psalmus}{
  \setlength{\parindent}{0pt}
  \setlength{\parskip}{5pt}
}{
  \setlength{\parindent}{10pt}
  \setlength{\parskip}{10pt}
}

%%%% Prejmenovat na latinske:
\newcommand{\nadpisZalmu}[1]{
  \hspace{2cm}\textbf{#1}\vspace{2mm}%
  \nopagebreak%

}

% mode, score, translation
\newcommand{\antiphona}[3]{%
\hicSuntNeumae
\superInitialam{#1}
\includescore{#2}

#3
}
 % Often used macros

\setlength{\columnsep}{15pt} % prostor mezi sloupci

%%%%%%%%%%%%%%%%%%%%%%%%%%%%%%%%%%%%%%%%%%%%%%%%%%%%%%%%%%%%%%%%%%%%%%%%%%%%%%%%%%%%%%%%%%%%%%%%%%%%%%%%%%%%%
\begin{document}

% Here we set the space around the initial.
% Please report to http://home.gna.org/gregorio/gregoriotex/details for more details and options
\grechangedim{afterinitialshift}{2.2mm}{scalable}
\grechangedim{beforeinitialshift}{2.2mm}{scalable}
\grechangedim{interwordspacetext}{0.32 cm plus 0.15 cm minus 0.05 cm}{scalable}%
\grechangedim{annotationraise}{-0.2cm}{scalable}

% Here we set the initial font. Change 38 if you want a bigger initial.
% Emit the initials in red.
\grechangestyle{initial}{\color{red}\fontsize{36}{36}\selectfont}

\renewcommand{\headrulewidth}{0pt} % no horiz. rule at the header
\pagestyle{empty}

\grechangedim{spaceabovelines}{0.2cm}{scalable}%

\cantusSineNeumas

\vfill

\hbox{}

\vspace{5cm}

\begin{center}
{\huge \textit{AD I VESPERAS}}

\vspace{0.5cm}

{\huge \textsc{Dominica II Adventus}}

\vspace{0.5cm}

{\textsc{Secundum Antiphonale Romanum II}}
\end{center}

\vfill
\scriptura{}
\pagebreak

\pars{Introductio}

\gregorioscore{temporalia/deusinadiutorium-communis.gtex}

\vspace{0.5cm}

\pars{Hymnus}

\vspace{-0.3cm}

\antiphona{II}{temporalia/hym-ConditorAlme.gtex}

\vspace{0.5cm}

\pars{Antiphona I} \scriptura{Cf. Zach. 9, 9}

\antiphona{VI f}{temporalia/ant1.gtex}

\vspace{0.5cm}

\pars{Psalmus} \scriptura{Ps. 118, 105-112}

{
\grechangedim{interwordspacetext}{0.24 cm plus 0.15 cm minus 0.05 cm}{scalable}%
\gregorioscore{temporalia/ps118xiv-initium-vi-F-auto.gtex}
\grechangedim{interwordspacetext}{0.32 cm plus 0.15 cm minus 0.05 cm}{scalable}%
}

{\setlength{\parindent}{0pt}\input{temporalia/ps118xiv.tex}}

\vfill
\pagebreak

\pars{Antiphona II} \scriptura{Is. 35, 3.4}

\antiphona{II d}{temporalia/ant2.gtex}

\vspace{0.5cm}

\pars{Psalmus} \scriptura{Ps. 15}

\gregorioscore{temporalia/ps15-initium-ii-D-auto.gtex}

{\setlength{\parindent}{0pt}\input{temporalia/ps15.tex}}

\vfill
\pagebreak

\pars{Antiphona III} \scriptura{Io. 1, 17; \textbf{H37}}

\vspace{-0.3cm}

\antiphona{I g}{temporalia/ant3.gtex}

\vspace{0.5cm}

\pars{Canticum} \scriptura{Phlp. 2, 6-11}

\gregorioscore{temporalia/phil2-initium-i-g-auto.gtex}

{\setlength{\parindent}{0pt}\input{temporalia/phil2.tex}}

\vfill
\pagebreak

\pars{Lectio brevis} \scriptura{1 Thes. 5, 23.24}

Ipse Deus pacis sanctíficet vos per ómnia, et ínteger spíritus vester et
ánima et corpus sine queréla in advéntu Dómini nostri Iesu Christi servétur.
Fidélis est qui vocávit vos, qui étiam fáciet.

\vspace{0.5cm}

\pars{Responsorium breve} \scriptura{Ps. 84, 8; \textbf{H20}}

\vspace{-0.3cm}

\antiphona{\oldrbar.~ br.}{temporalia/resp.gtex}

\vfill
\pagebreak

\pars{Antiphona ad Magnificat} \scriptura{\textbf{H21}}

\vspace{-0.3cm}

\antiphona{VII a}{temporalia/ant-magn-vesp.gtex}

\vspace{0.5cm}

\pars{Canticum Evangelicum} \scriptura{Lc. 1, 46-55}

\gregorioscore{temporalia/magnificat-initium-viisoll-a.gtex}

{\setlength{\parindent}{0pt}\input{temporalia/magnificat.tex}}

\vfill
\pagebreak

\pars{Preces}

{\setlength{\parindent}{0pt}
Christum Dóminum, fratres caríssimi, ex Vírgine María natum, húmiles
deprecémur lætantés\textit{que} \textit{vo}\textbf{cé}mus:

\vspace{0.2cm}

\Rbardot{} Veni, Dómi\textit{ne} \textbf{Ie}su.

\vspace{0.2cm}

Fili Dei unigénite, qui ventúrus es ut verus ángelus \textit{testa}\textbf{mén}ti,\\
-- fac ut mundus te suscípiat et \textit{ag}\textbf{nós}cat. \Rbardot{}

Qui génitus in sinu Patris, incarnári venísti ex Ma\textit{ría} \textbf{Vír}gine,\\
-- ab omni corruptióne humánæ condiciónis \textit{nos} \textbf{é}lue. \Rbardot{}

Qui, vita cum sis, venísti mor\textit{tem} \textit{per}\textbf{fér}re,\\
-- tríbue nobis nihil de mortis damnatióne \textit{sen}\textbf{tí}re. \Rbardot{}

Et quia merces tua ad iudícium tecum véniet, \gredagger{}\\
píetas tua consuéta tunc nobis occúr\textit{rat} \textit{in} \textbf{mú}nere,\\
-- quæ nostrum sólita est semper removére \textit{lan}\textbf{guó}rem. \Rbardot{}

Christe Dómine, qui morte tua mórtuis \textit{subve}\textbf{nís}ti,\\
-- pro defúnctis nos orántes \textit{ex}\textbf{áu}di. \Rbardot{}
}

\vspace{0.5cm}

\pars{Oratio conclusiva}

Omnípotens et miséricors Deus, \gredagger{}
in tui occúrsum Fílii festinántes nulla ópera terréni actus impédiant, \grestar{}
sed sapiéntiæ cæléstis erudítio nos fáciat eius esse consórtes.
Qui tecum vivit et regnat in unitáte Spíritus Sancti, Deus,
per ómnia sæcula sæculórum. Amen.

\end{document}
