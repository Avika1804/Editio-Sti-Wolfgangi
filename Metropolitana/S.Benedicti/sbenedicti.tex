% LuaLaTeX

\documentclass[a4paper, twoside, 12pt]{article}
\usepackage[latin]{babel}
%\usepackage[landscape, left=3cm, right=1.5cm, top=2cm, bottom=1cm]{geometry} % okraje stranky
\usepackage[portrait, a4paper, mag=1414, truedimen, left=0.8cm, right=0.8cm, top=0.8cm, bottom=0.8cm]{geometry} % okraje stranky

\usepackage{fontspec}
\setmainfont[FeatureFile={junicode.fea}, Ligatures={Common, TeX}, RawFeature=+fixi]{Junicode}
%\setmainfont{Junicode}

% shortcut for Junicode without ligatures (for the Czech texts)
\newfontfamily\nlfont[FeatureFile={junicode.fea}, Ligatures={Common, TeX}, RawFeature=+fixi]{Junicode}

\usepackage{multicol}
\usepackage{color}
\usepackage{lettrine}
\usepackage{fancyhdr}

% usual packages loading:
\usepackage{luatextra}
\usepackage{graphicx} % support the \includegraphics command and options
\usepackage{gregoriotex} % for gregorio score inclusion
\usepackage{gregoriosyms}
\usepackage{wrapfig} % figures wrapped by the text
\usepackage{parcolumns}
\usepackage[contents={},opacity=1,scale=1,color=black]{background}
\usepackage{tikzpagenodes}
\usepackage{calc}
\usepackage{longtable}

\setlength{\headheight}{12pt}

% Commands used to produce a typical "Conventus" booklet

\newenvironment{titulusOfficii}{\begin{center}}{\end{center}}
\newcommand{\dies}[1]{#1

}
\newcommand{\nomenFesti}[1]{\textbf{\Large #1}

}
\newcommand{\celebratio}[1]{#1

}

\newcommand{\hora}[1]{%
\vspace{0.5cm}{\large \textbf{#1}}

\fancyhead[LE]{\thepage\ / #1}
\fancyhead[RO]{#1 / \thepage}
\addcontentsline{toc}{subsection}{#1}
}

% larger unit than a hora
\newcommand{\divisio}[1]{%
\begin{center}
{\Large \textsc{#1}}
\end{center}
\fancyhead[CO,CE]{#1}
\addcontentsline{toc}{section}{#1}
}

% a part of a hora, larger than pars
\newcommand{\subhora}[1]{
\begin{center}
{\large \textit{#1}}
\end{center}
%\fancyhead[CO,CE]{#1}
\addcontentsline{toc}{subsubsection}{#1}
}

% rubricated inline text
\newcommand{\rubricatum}[1]{\textit{#1}}

% standalone rubric
\newcommand{\rubrica}[1]{\vspace{3mm}\rubricatum{#1}}

\newcommand{\notitia}[1]{\textcolor{red}{#1}}

\newcommand{\scriptura}[1]{\hfill \small\textit{#1}}

\newcommand{\translatioCantus}[1]{\vspace{1mm}%
{\noindent\footnotesize \nlfont{#1}}}

% pruznejsi varianta nasledujiciho - umoznuje nastavit sirku sloupce
% s prekladem
\newcommand{\psalmusEtTranslatioB}[3]{
  \vspace{0.5cm}
  \begin{parcolumns}[colwidths={2=#3}, nofirstindent=true]{2}
    \colchunk{
      \input{#1}
    }

    \colchunk{
      \vspace{-0.5cm}
      {\footnotesize \nlfont
        \input{#2}
      }
    }
  \end{parcolumns}
}

\newcommand{\psalmusEtTranslatio}[2]{
  \psalmusEtTranslatioB{#1}{#2}{8.5cm}
}


\newcommand{\canticumMagnificatEtTranslatio}[1]{
  \psalmusEtTranslatioB{#1}{temporalia/extra-adventum-vespers/magnificat-boh.tex}{12cm}
}
\newcommand{\canticumBenedictusEtTranslatio}[1]{
  \psalmusEtTranslatioB{#1}{temporalia/extra-adventum-laudes/benedictus-boh.tex}{10.5cm}
}

% volne misto nad antifonami, kam si zpevaci dokresli neumy
\newcommand{\hicSuntNeumae}{\vspace{0.5cm}}

% prepinani mista mezi notovymi osnovami: pro neumovane a neneumovane zpevy
\newcommand{\cantusCumNeumis}{
  \setgrefactor{17}
  \global\advance\grespaceabovelines by 5mm%
}
\newcommand{\cantusSineNeumas}{
  \setgrefactor{17}
  \global\advance\grespaceabovelines by -5mm%
}

% znaky k umisteni nad inicialu zpevu
\newcommand{\superInitialam}[1]{\gresetfirstlineaboveinitial{\small {\textbf{#1}}}{\small {\textbf{#1}}}}

% pars officii, i.e. "oratio", ...
\newcommand{\pars}[1]{\textbf{#1}}

\newenvironment{psalmus}{
  \setlength{\parindent}{0pt}
  \setlength{\parskip}{5pt}
}{
  \setlength{\parindent}{10pt}
  \setlength{\parskip}{10pt}
}

%%%% Prejmenovat na latinske:
\newcommand{\nadpisZalmu}[1]{
  \hspace{2cm}\textbf{#1}\vspace{2mm}%
  \nopagebreak%

}

% mode, score, translation
\newcommand{\antiphona}[3]{%
\hicSuntNeumae
\superInitialam{#1}
\includescore{#2}

#3
}
 % Often used macros

\setlength{\columnsep}{15pt} % prostor mezi sloupci

%%%%%%%%%%%%%%%%%%%%%%%%%%%%%%%%%%%%%%%%%%%%%%%%%%%%%%%%%%%%%%%%%%%%%%%%%%%%%%%%%%%%%%%%%%%%%%%%%%%%%%%%%%%%%
\begin{document}

% Here we set the space around the initial.
% Please report to http://home.gna.org/gregorio/gregoriotex/details for more details and options
\grechangedim{afterinitialshift}{2.2mm}{scalable}
\grechangedim{beforeinitialshift}{2.2mm}{scalable}
\grechangedim{interwordspacetext}{0.38 cm plus 0.15 cm minus 0.05 cm}{scalable}%
\grechangedim{annotationraise}{-0.2cm}{scalable}

% Here we set the initial font. Change 38 if you want a bigger initial.
% Emit the initials in red.
\grechangestyle{initial}{\color{red}\fontsize{36}{36}\selectfont}

\renewcommand{\headrulewidth}{0pt} % no horiz. rule at the header
\pagestyle{empty}

\grechangedim{spaceabovelines}{0.2cm}{scalable}%

\vfill

\hbox{}

\vspace{5cm}

\begin{center}
{\large \textit{DIE XI IULII}}

\vspace{0.5cm}

{\huge S. BENEDICTI, ABBATIS}

\vspace{0.5cm}

{\large \textsc{Europæ Patroni}}

\vspace{0.5cm}

\textsc{Secundum Antiphonale Romanum II}
\end{center}

\vfill
\pagebreak

\cantusSineNeumas

\pars{Introductio}

\gregorioscore{temporalia/deusinadiutorium-communis.gtex}

\vspace{0.5cm}

\pars{Hymnus}

\vspace{-0.5cm}

\antiphona{II}{temporalia/hym-InterAEternas.gtex}

\vspace{0.5cm}

\pars{Antiphona I} \scriptura{S. Gregorius}

\vspace{-0.2cm}

\antiphona{VIII g}{temporalia/ant1.gtex}

\scriptura{Ps. 14}

\gregorioscore{temporalia/ps14-initium-viii-G-auto.gtex}

{\setlength{\parindent}{0pt}\input{temporalia/ps14.tex}}

\vfill
\pagebreak

\pars{Antiphona II}

\vspace{-0.2cm}

\antiphona{I d}{temporalia/ant2.gtex}

\scriptura{Ps. 111}

\grechangedim{interwordspacetext}{0.25 cm plus 0.15 cm minus 0.05 cm}{scalable}%
\gregorioscore{temporalia/ps111-initium-i-D2-auto.gtex}
\grechangedim{interwordspacetext}{0.38 cm plus 0.15 cm minus 0.05 cm}{scalable}%

{\setlength{\parindent}{0pt}\input{temporalia/ps111.tex}}

\vfill
\pagebreak

\pars{Antiphona III}

\vspace{-0.5cm}

\antiphona{VIII g}{temporalia/ant3.gtex}

\scriptura{Ap. 15, 3-4}

\gregorioscore{temporalia/ap15-initium-viii-G-auto.gtex}

{\setlength{\parindent}{0pt}\input{temporalia/ap15.tex}}

\vfill
\pagebreak

\pars{Lectio brevis} \scriptura{1 Ptr. 5, 1-4}

Senióres, qui in vobis sunt, óbsecro consénior et testis Christi passiónum,
qui et eius, quæ in futúro revelánda est, glóriæ communicátor: páscite qui
in vobis est gregem Dei, providéntes non coácte, sed spontánee secúndúm
Deum, neque turpis lucri grátia, sed voluntárie; neque ut dominántes in
cleris, sed forma facti gregis ex ánimo.  Et, cum apparúerit princeps
pastórum, percipiétis immarcescíbilem glóriæ corónam.

\vspace{0.5cm}

\pars{Responsorium breve}

\vspace{-0.2cm}


\antiphona{\oldrbar.~ br.}{temporalia/resp.gtex}

\vfill
\pagebreak

\pars{Antiphona ad Magnificat}

\vspace{-0.2cm}

\antiphona{I f}{temporalia/ant-magn-vesp.gtex}

\scriptura{Lc. 1, 46-55}

\gregorioscore{temporalia/magnificat-initium-isoll-f.gtex}

{\setlength{\parindent}{0pt}\input{temporalia/magnificat.tex}}

\vfill
\pagebreak

\pars{Preces}

Christum, pro homínibus pontíficem constitútum in iis quæ sunt ad Deum,
dignis láudibus celebrémus, humíliter \textit{depre}\textbf{cán}tes:

\Rbardot{} Salvum fac pópulum tu\textit{um,} \textbf{Dó}mine.

Qui Ecclésiam tuam per sanctos et exímios rectóres mirabíliter
\textit{illus}\textbf{trás}ti,\\
-- fac ut christiáni eódem semper lætificéntur \textit{splen}\textbf{dó}re.
\Rbardot{}

Qui, cum sancti te pastóres sicut Móyses orárent pópuli peccáta
\textit{dimi}\textbf{sís}ti,\\
-- per intercessiónem eórum Ecclésiam tuam contínua purificatióne
\textit{sanc}\textbf{tí}fica. \Rbardot{}

Qui sanctos tuos unxísti in médio fratrum et Spíritum tuum in illos
\textit{direx}\textbf{ís}ti,\\
-- reple Spíritu Sancto omnes pópuli tui \textit{rec}\textbf{tó}res. \Rbardot{}

Qui pastórum sanctórum ipse posséssio \textit{exsti}\textbf{tís}ti,\\
-- tríbue nullum ex iis, quos sánguine acquisísti, sine te
\textit{ma}\textbf{né}re. \Rbardot{}

Qui, per Ecclésiæ pastóres, vitam ætérnam óvibus tuis præstas, ne rápiat eas
quisquam de \textit{manu} \textbf{tu}a,\\
-- salva defúnctos, pro quibus ánimam tuam po\textit{su}\textbf{ís}ti. \Rbardot{}

\vspace{0.5cm}

\pars{Oratio conclusiva}

Deus, qui beátum Benedíctum abbátem in schola divíni servítii præclárum
constituísti \\
magístrum, \gredagger{} tríbue, quǽsumus, \grestar{}
ut, amóri tuo nihil præponéntes,
viam mandatórum tuórum dilatáto corde currámus. Per Dóminum.

\end{document}
