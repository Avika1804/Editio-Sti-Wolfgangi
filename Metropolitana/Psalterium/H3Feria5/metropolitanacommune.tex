% Commands used to produce a typical "Metropolitana" booklet

\newenvironment{titulusOfficii}{\begin{center}}{\end{center}}
\newcommand{\dies}[1]{#1

}
\newcommand{\nomenFesti}[1]{\textbf{\Large #1}

}
\newcommand{\celebratio}[1]{#1

}

\newcommand{\hora}[1]{%
\vspace{0.5cm}{\large \textbf{#1}}

\fancyhead[LE]{\thepage\ / #1}
\fancyhead[RO]{#1 / \thepage}
\addcontentsline{toc}{subsection}{#1}
}

% larger unit than a hora
\newcommand{\divisio}[1]{%
\begin{center}
{\Large \textsc{#1}}
\end{center}
\fancyhead[CO,CE]{#1}
\addcontentsline{toc}{section}{#1}
}

% a part of a hora, larger than pars
\newcommand{\subhora}[1]{
\begin{center}
{\large \textit{#1}}
\end{center}
%\fancyhead[CO,CE]{#1}
\addcontentsline{toc}{subsubsection}{#1}
}

% rubricated inline text
\newcommand{\rubricatum}[1]{\textit{#1}}

% standalone rubric
\newcommand{\rubrica}[1]{
  \vspace{3mm}
  \noindent \rubricatum{#1}
  \vspace{3mm}
}

\newcommand{\notitia}[1]{\textcolor{red}{#1}}

\newcommand{\scriptura}[1]{\hfill \small\textit{#1}}

\newcommand{\translatioCantus}[1]{\vspace{1mm}%
{\noindent\footnotesize #1}}

\newenvironment{psalmiTranslatio}{\footnotesize}{}

% pruznejsi varianta nasledujiciho - umoznuje nastavit sirku sloupce
% s prekladem
\newcommand{\psalmusEtTranslatioB}[3]{
  \input{#1}
  %% \vspace{0.5cm}
  %% \begin{parcolumns}[colwidths={2=#3}, nofirstindent=true]{2}
  %%   \colchunk{
  %%     \input{#1}
  %%   }

  %%   \colchunk{
  %%     %\vspace{-0.5cm}
  %%     \input{#2}
  %%   }
  %% \end{parcolumns}
}

\newcommand{\psalmusEtTranslatio}[2]{
  \psalmusEtTranslatioB{#1}{#2}{5cm}
}

% znaky k umisteni nad inicialu zpevu
\newcommand{\superInitialam}[1]{\gresetfirstlineaboveinitial{\footnotesize {#1}}{}}

% pars officii, i.e. "oratio", ...
\newcommand{\pars}[1]{
  \hfill \textsc{\small{#1}} \hfill
}

%%%% Prejmenovat na latinske:
\newcommand{\nadpisZalmu}[1]{
  \hspace{2cm}\textbf{#1}\vspace{2mm}%
  \nopagebreak%

}

% mode, score, translation
\newcommand{\antiphona}[3]{%
\superInitialam{#1}
\includescore{#2}

#3
\vspace{2mm}
}

% test if we are in a specific environment
\makeatletter
\def\ifenv#1{
   \def\@tempa{#1}%
   \ifx\@tempa\@currenvir
      \expandafter\@firstoftwo
    \else
      \expandafter\@secondoftwo
   \fi
}
\makeatother

% for pslm
\newenvironment{psalmus}{\vspace{0.5cm}\begin{hangparas}{1em}{1}}{\end{hangparas}}
\newcommand{\flex}{~\dag\mbox{}\ifenv{psalmiTranslatio}{}{\\}}
\newcommand{\asterisk}{~*\ifenv{psalmiTranslatio}{}{\\}}
