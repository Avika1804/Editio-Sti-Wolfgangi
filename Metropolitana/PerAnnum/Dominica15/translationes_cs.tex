%%%% Preklady jednotlivych zpevu (nektere se opakuji, a je dobre mit je
% vsechny na jedne hromade)

\newcommand{\trIntroductio}{\translatioCantus{Bože pospěš mi na pomoc.
\Rbardot{} Slyš naše volání. Sláva Otci i~Synu i~Duchu svatému,
jako byla na počátku, i~nyní i~vždycky a~na věky věků. Amen. Aleluja.}}

\newcommand{\trAntI}{\translatioCantus{Ať je jméno Páně \grestar{}
požehnáno na věky.}}

\newcommand{\trAntII}{\translatioCantus{Vezmu kalich spásy a~budu vzývat
jméno Hospodinovo.}}

\newcommand{\trAntIII}{\translatioCantus{Pán Ježíš se ponížil,
proto také ho Bůh povýšil na věky.}}

\newcommand{\trResp}{\translatioCantus{Kterak jsou skvělá, \grestar{}
Pane, tvá díla. \Vbardot{} Všechno jsi stvořil a~učinil moudře.
\Vbardot{} Sláva Otci i~Synu i~Duchu Svatému.}}

\newcommand{\trRespII}{\translatioCantus{Zatmělo se, \grestar{}
když židé Ježíše ukřižovali, a kolem deváté hodiny vykřikl Ježíš mocným
hlasem: ,,Bože můj, Bože můj, proč jsi mě opustil?\mbox{}`` \gredagger{}
A naklonil hlavu a odevzdal ducha. Pak mu jeden z vojáků svým kopím probodl
bok a hned vyšla krev a voda. \Vbardot{}
Ježíš mocně vykřikl a řekl: ,,Otče, do tvých rukou odevzdávám svého ducha.``}}

\newcommand{\trAntMagnificat}{\translatioCantus{Ježíš dal \grestar{}
učedníkům sílu a~moc nade všemi zlými duchy, k~léčení nemocí a~k hlásání
Božího slova, aleluja.}}

\newcommand{\trLectioI}{{\footnotesize \nlfont
Z druhé knihy Makabejské

Nedlouho potom poslal král starého Athéňana, aby přinutil židy odvrátit se
od zákonů otců a~nežít podle Božích zákonů, zneuctít dokonce jeruzalémskou
svatyni a~pojmenovat ji po Diovi Olympském, svatyni na Gerizímu pak
pojmenovat po Diovi Pohostinném proto, že byli pohostinní obyvatelé toho
místa.

Těžký a~odporný byl všem nápor tolikerého zla. Vždyť i~svatyni naplnili
pohané pro\-sto\-páš\-nos\-tí a~nevázanou bujností, hledali povyražení u~nevěstek
a~na svatých nádvořích se pelešili s~ženštinami; dovnitř vnášeli, co tam
nepatřilo. I~zápalný oltář byl přeplněn nečistými věcmi, které zákony
zakazovaly. Nebylo dovoleno světit sobotu ani dodržovat svátky otců, ba
vůbec se přiznat k~tomu, že je někdo žid. Každý měsíc násilně nahnali lid na
obětní slavnost králových narozenin, a~když nadešla slavnost Dionýsova,
nutili židy, aby s~břečťanovými věnci oslavovali Dionýsa.

Z~popudu Ptolemaiova přišlo nařízení i~do sousedních helénských měst, aby se
s~židy jednalo stejným způsobem, kdykoli se budou pořádat obětní slavnosti;
ti, kdo se nerozhodnou přejít na helénský způsob života, měli být zabiti.
Všem bylo zřejmé, že nastává čas velikého utrpení.

Dvě ženy byly předvedeny před soud, že daly obřezat své syny. S~kojenci
u~prsů je vodili veřejně městem a~pak je svrhli z~hradeb. Židé se sešli do
blízkých jeskyní, aby tajně oslavili sobotu. Byli udáni Filipovi a~společně
zaživa upáleni, protože se z~úcty k~nejposvátnějšímu dni odmítli bránit.

Nyní vybízím ty, kdo při četbě knihy došli až sem, aby se nedali zkrušit
tím, co se stalo, ale aby pochopili, že to trestání nebylo ke zkáze, nýbrž
k~výchově našeho rodu. Je známkou velikého dobrodiní bezbožníkům, nejsou-li
dlouhý čas ponecháni v~pohodlí hříchu, ale postihne-li je vzápětí trest.
Zatímco u~jiných národů Hospodin trpělivě čeká, až se naplní míra jejich
hříchů, a~teprve pak je potrestá, s~námi se rozhodl naložit jinak, aby nás
nemusel trestat později, až naše hříchy dosáhnou nejzazší meze. Proto od nás
nikdy neoddaluje milosrdenství; svůj lid vychovává neštěstími, ale neopouští
jej. Jen tolik budiž řečeno na vysvětlenou; po tomto krátkém odbočení
pokračujeme ve vyprávění.

Jistému Eleazarovi, jednomu z~předních znalců Zákona, který byl pokročilého
věku a~uš\-lech\-ti\-lých rysů tváře, otevřeli násilím ústa a~nutili ho pozřít
vepřové maso. On však zvolil raději čestnou smrt než život s~potupou
a~dobrovolně šel na popraviště. Pokrm, který měl sníst, vyplivl. Těm, kdo
zůstanou, tím dal příklad, jak odmítat zakázaná jídla i~přes všechnu lásku
k~životu.

Dohlížitelé nad bezbožnými obětními slavnostmi, kteří se s~tím mužem znali
ještě z~dří\-věj\-ších dob, ho vzali stranou a~přemlouvali, aby si dal přinést
kusy masa, z~něhož má dovoleno jíst, aby si je sám připravil a~přitom
předstíral, že jí maso z~obětí nařízených králem. Když to prý udělá, unikne
smrti a~pro staré přátelství s~nimi dojde vlídného zacházení. On však učinil
významné rozhodnutí, hodné věku a~důstojnosti stáří, vznešených šedin a~od
mládí nejvzornějšího chování; zvláště pak dbal svatého a~Bohem daného
zákonodárství, když odpověděl, aby jej poslali do podsvětí. „Nesluší se,“
řekl, „abychom se ve svém věku přetvařovali, protože by mnozí z~mladých
soudili, že Eleazar se ve svých devadesáti letech stal odpadlíkem. Ty bych
pro své pokrytectví a~okamžik prchavého ži\-vo\-ta hanebně oklamal a~na své
stáří tak přivodil potupnou skvrnu. A~i kdybych pro přítomnost unikl trestu
lidí, neuniknu rukám Vševládného, ať živý, nebo mrtvý. Proto se nyní
statečně vzdávám ži\-vo\-ta, abych se ukázal hoden svého stáří a~mladým zanechal
ušlechtilý příklad, aby ochotně a~důstojně šli na smrt za vznešené a~svaté
zákony.“ To řekl a~hned vstoupil na popraviště.

Blahosklonnost, kterou mu ještě krátce předtím projevili jeho průvodci, se
změnila v~nepřátelství pro ta slova, která pokládali za šílenství. Když
skonával pod ranami, se sténáním pravil: „Hospodinu, který má svaté poznání,
je zjevné, že jsem mohl uniknout smrti; snáším na svém těle krutá muka
bičování a~v duši to rád vytrpím z~posvátné bázně k~němu.“

Tak skonal a~zanechal svou smrtí příklad hrdinství a~památku ctnosti nejen
mladým, ale i~většině národa.}}

\newcommand{\trLectioII}{{\footnotesize \nlfont
\noindent \textsc{Čas je drahocenným statkem}

\noindent \textit{Kázání P. Josefa Toufara, proslovené 31. prosince 1949 na sv. Silvestra při
dopolední bohoslužbě v chrámu Nanebevzetí Panny Marie v~Číhošti.}

Drazí v~Kristu! 

V~podvečer posledního dne roku jsme se shromáždili v~našem chrámu Páně,
abychom poděkovali Pánu Bohu za všechna dobrodiní, která jsme od něho za
těch 365 dní obdrželi pro duši a~pro tělo. Děkovat jsme přišli, ale také
hodně a~hodně odprosit dobrotivého Otce nebeského. Vždyť na konci uplynulého
roku s~pocitem bolestného smutku se ohlížíme nazpátek a~musíme doznat i~my:
Vykonal jsem v~tom roce mnoho činů a~práce, ale pro Boha a~moji duši málo
nebo nic. Namluvil jsem milióny slov, ale málo z~nich chválilo Boha -- prožil
jsem 365 dní, což činilo 8760 hodin, ale většinu z~nich jsem promrhal,
protože neměly cenu pro moji duši a~věčnou spásu. 

Perská pověst vypravuje: Jakýsi muž kráčel navečer po břehu mořském
a~tu našel sáček plný drobných kamínků. Probíral se prsty v~kamíncích
a~přitom pozoroval hejno bílých racků, kteří halekali na vlnách. Z~dlouhé
chvíle a~ze zábavy bral ze sáčku kaménky, o~nichž si myslel, že je tam
nějaké děti nechaly zapomenuté a~házel je po ptácích. Jen jeden jediný si
ponechal v~ruce a~donesl domů. Jak veliké bylo jeho překvapení, když při
světle domácího krbu poznal nádherně se třpytící diamant ohromné ceny.
Uvědomil si, jaký poklad tak lehkomyslně ztratil a~promarnil. Spěchal zpět
na mořský břeh, ale ani lítost, ani sebeobviňování nemohly mu vrátit
ztracený poklad. Ten ležel na dně mořském a~byl mu navždy ztracen. 

A~tobě, milý křesťane, dal Bůh v~právě skončeném roce poklad mnohem
cennější, nežli byl onen sáček plný diamantů onoho Peršana. To byly dny
tvého minulého života -- dny, které zapadly do moře minulosti a~více se
nevrátí -- dny, ve kterých jsi mohl hodně pro Boha a~duši vykonat, nebo
i~duši ztratit.

Dne 16. srpna 1851 měl být v~Londýně popraven muž Lucián Holl. Když
byl otázán, zda má před smrtí ještě nějaké přání, prosil úpěnlivě, aby mu
byl život prodloužen o~čtvrt hodiny, že chce svou duši na věčnost lépe
připravit. Přání bylo vyhověno, ale pak soudce zvolal s~hodinkami v~ruce:
„Čtvrt hodiny uplynulo, pozor, připravte se!“ A~tu začal odsouzený tím
úpěnlivěji volat a~prosil: „Ještě aspoň pět minut!“ Soudce povolil i~tuto
další lhůtu a~když povolený čas již uplynul, odsouzenec více prosil: „Aspoň
minutu, aspoň minutu, ještě jen jednu minutu!“ I~ta mu byla povolena, ale
pak již byla poprava vykonána. 

Moji drazí! Neotřese to vaší duší -- touha po čtvrthodině, pěti
minutách, po minutě života, aby jich mohl umírající plně využít pro věčnost?
A~my, co jsme promarnili těch minut, hodin, dní a~měsíců v~tomto roce
a~kolik už v~dosavadním našem životě? A~jak to musí být hrozná bilance pro
mnohého člověka, který žil bez Boha a~bez víry, který špatně hospodařil
s~časem svého života. Proto tak důtklivě nás napomíná sv. Pavel: „Bratři,
žijte pečlivě jako moudří, vykupujíce čas.“ 

Čas je drahocenným statkem, a~proto ho musíme dobře užívat. Každé
ráno je dar, který se nikdy nevrátí, a~škoda, věčná škoda každého
promarněného dne. Každý den je buď pramenem požehnání, nebo pramenem
neštěstí, neboť žít v~hříších znamená promrhat drahocenný čas, utíkat pravé
radosti aniž za živa si připravit zoufalé peklo. 

Ruský myslitel Tolstoj svého času špatným způsobem utrácel
drahocenný čas a~k jakým výsledkům dospěl? Na kraji zoufalství vidí
pošetilost svého počínání zahalenou v~indické bajce: Jeden poutník byl
pronásledován dravcem a~tu v~rychlosti skočil do prohlubně. Ale v~posledním
okamžiku v~hrůze spatřil, jak na dně propasti otvírá hrozný netvor svůj
chřtán, aby ho polapil a~pohltil. Rychle se zachytil keře, který vyrůstal ze
skalní štěrbiny. Tak visel mezi dvojím nebezpečím. Nahoře dravec -- dole
hrozná tlama netvora. Visí -- ruce mu umdlévají, síly docházejí a~cítí, že za
chvilku se zřítí do záhuby. A~když pohlédne ku keři, který jej na chvilku
zachraňuje, co vidí? Zpod kořenů vylezly dvě myši, jedna bílá, druhá černá,
a~ohlodávaly kořeny keře, kterého se držel. Poznal, že je ztracen. Tak jsem
visel i~já ve větvích života, i~na mne čuměla saň smrti, bílá a~černá myš,
den a~noc hlodaly větev, které jsem se držel. A~tu ho napadlo začít jiný
život a~zkusit nové krásné štěstí -- štěstí křesťana. 

Drazí v~Kristu! 

I~nám ustavičně hrozí otevřená tlama saně-smrti -- i~na větvích našich životů
ustavičně hlodají myši, bílá a~černá -- den a~noc -- a~nám ubíhají hodiny,
dny, měsíce, a~uplynul rok ...  Kolik smutku, kolik bolesti a~slz nám
přinesls, roku 1949 -- kolik radosti a~štěstí jsi nám také přinesl
-- i~otevřené huby se na nás dívaly -- a~jak šťastný byl člověk, který měl
hlubokou a~pevnou víru, který třeba se slzou v~oku mohl říci: Otče nebeský,
ne naše, ale Tvá svatá vůle se staň! 

Drazí v~Kristu! 

Už nikdy se nevrátí rok 1949. A~přece se vrátí! Vrátí se každému na soudu
Božím. Hříšníkům k~zahanbení, dítkám Božím k~oslavě. Byl-li nám rok 1949
časem k~pádu, přičiňme se a~prosme Boha, aby rok 1950 byl nám po celý rok
k~povstání, a~proto žijme tak, jako bychom již dnes měli zemřít, žijme pečlivě
jako moudří a~vykupujme si časem vezdejším život věčný. 

Drazí v~Kristu! 

Vložte do toho dnešního děkovného chvalozpěvu „Bože, chválíme Tebe“ celou
svou pokornou a~kajícnou duši i~veškeré odhodlání a~lásku své duše
s~nejsvětějším předsevzetím, že dny nastávajícího roku využijeme pro spásu své
duše, neboť ten každičký den příštího roku bude cenný jako drahokam. Bude
drahokamem ztraceným, když ho promarníme, bude však nádherně zářícím
drahokamem v~naší koruně věčné, když jej prosvítíme láskou k~Bohu, bližním
a~dobrými skutky. Čiňme dobré, dokud máme čas, aby Pán nás vzal na svou
pravici. Amen.}}

\newcommand{\trPreces}{Bratři a~sestry, jsme shromážděni na místě, kde
požehnaně působil a~krátce před svou smrtí trpěl kněz Josef Toufar.
V~duchovním spojení s ním prosme Boha za naši dnešní církev i~za celý svět
a~společně volejme:\\
\vspace{-0.2cm}\\
\Rbardot{} Kyrie, eleison.\\
\vspace{-0.2cm}\\
{\color{red}--} Prosme za oddanou víru, pevnou naději a~věrnou lásku všech křesťanů. \Rbardot{}\\
{\color{red}--} Prosme za svatost a~obětavost biskupů, kněží a~všech služebníků církve. \Rbardot{}\\
{\color{red}--} Prosme za pokoj a~úctu v rodinách a~dobrou budoucnost příštích generací. \Rbardot{}\\
{\color{red}--} Prosme za vzájemný respekt a~ohleduplnost lidí různých kultur a~různé víry. \Rbardot{}\\
{\color{red}--} Prosme ukončení válečných konfliktů a~nastolení trvalého míru. \Rbardot{}\\
{\color{red}--} Prosme za ty, kdo jsou nuceni uprchnout ze své vlasti. \Rbardot{}\\
{\color{red}--} Prosme za národy utištěné totalitním režimem. \Rbardot{}\\
{\color{red}--} Prosme za sílu a~statečnost těch, kdo trpí násilí. \Rbardot{}\\
{\color{red}--} Prosme za osvobození těch, kdo slouží zlu v jakékoli jeho podobě. \Rbardot{}\\
{\color{red}--} Prosme o~požehnání pro obyvatele této obce, tohoto kraje a~celé naší vlasti. \Rbardot{}}

\newcommand{\trOratioDominica}{\translatioCantus{Otče náš, jenž jsi na
nebesích, posvěť se jméno tvé. Přijď království tvé. Buď vůle tvá jako v~nebi,
tak i~na zemi. Chléb náš vezdejší dej nám dnes. A~odpusť nám naše
viny, jako i~my odpouštíme našim viníkům. A~neuveď nás v~pokušení, ale zbav
nás od zlého.}}

\newcommand{\trOratio}{Bože, ty vedeš svou církev, aby ze slavného svědectví
mučedníků čerpala sílu a~odvahu; pomáhej nám, ať také my žijeme z~víry
a~v~jejich přímluvě ať máme oporu. Prosíme o~to skrze tvého Syna Ježíše Krista,
našeho Pána, neboť on s~tebou v~jednotě Ducha svatého žije a~kraluje po
všechny věky věků.}

\newcommand{\trBenedictio}{\noindent Pán s~vámi.\\
\Rbardot{} I~s tebou.\\
Požehnej vás všemohoucí Bůh\\
Otec i~Syn \grealtcross{} i~Duch svatý.\\
\Rbardot{} Amen.}

\newcommand{\trAdDimissionem}{\noindent Jděte ve jménu Páně.\\
\Rbardot{} Bohu díky.}
