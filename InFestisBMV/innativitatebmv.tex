% LuaLaTeX

\documentclass[a4paper, twoside]{article}
\usepackage[latin]{babel} 
\usepackage[landscape, left=2cm, right=1.5cm, top=2cm, bottom=1cm]{geometry} % okraje stranky

\usepackage{fontspec}
% \setmainfont{Gentium}
\setmainfont{Gentium Book Basic}

\usepackage{multicol}
\usepackage{color}
\usepackage{lettrine}

\newenvironment{titulusOfficii}{\begin{center}}{\end{center}}
\newcommand{\dies}[1]{#1

}
\newcommand{\nomenFesti}[1]{\textbf{\Large #1}

}
\newcommand{\celebratio}[1]{#1

}

\newcommand{\hora}[1]{\textbf{#1}

}
\newcommand{\notitia}[1]{\textcolor{red}{#1}}

\setlength{\columnsep}{30pt} % prostor mezi sloupci

\begin{document}

\emph{Texty podle Antiphonale Romanum 1912. Překlady, není-li výslovně uvedeno
jinak, J.P.}

\begin{titulusOfficii}
\dies{Die 8. Septembris.}
\nomenFesti{In Nativitate B. Mariae Virginis.}
\celebratio{Duplex 2. classis cum Octava.}
\end{titulusOfficii}

\hora{In I. vesperis.}

\hora{Ad Laudes.}

Oratio.
\begin{multicols*}{2}
\lettrine{F}{amulis} tuis, quaésumus Dómine, coeléstis grátiae munus impertíre:~\dag\mbox{}
ut quibus beátae Vírginis partus éxstitit salútis exórdium,~*
Nativitátis ejus votíva solémnitas pacis tríbuat increméntum.
Per Dóminum.

\columnbreak

% preklad J.P.
Prosíme tě, Bože, 
uděl nám, svým služebníkům, dar nebeské milosti,
aby těm, kterým v porodu blahoslavené Panny vzešel počátek spásy,
\notitia{(co s \emph{votiva}?)} slavnost jejího narození přinesla
rozhojnění pokoje.
Skrze našeho Pána.
\end{multicols*}

\end{document}
