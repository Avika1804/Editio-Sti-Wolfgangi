% LuaLaTeX

\documentclass[a4paper, twoside, 12pt]{article}
\usepackage[latin]{babel} 
\usepackage[landscape, left=2cm, right=1.5cm, top=2cm, bottom=2cm]{geometry} % okraje stranky

\usepackage{fontspec}
% \setmainfont{Gentium Book Basic}
\setmainfont{Junicode}

\usepackage{multicol}
\usepackage{color}
\usepackage{lettrine}

\newenvironment{titulusOfficii}{\begin{center}}{\end{center}}
\newcommand{\dies}[1]{#1

}
\newcommand{\nomenFesti}[1]{\textbf{\Large #1}

}
\newcommand{\celebratio}[1]{#1

}

\newcommand{\hora}[1]{\vspace{0.5cm}{\large \textbf{#1}}

}

\newcommand{\rubrica}[1]{\vspace{3mm}\itshape{#1}}

\newcommand{\notitia}[1]{\textcolor{red}{#1}}

\newcommand{\titCapitulum}[1]{\hfill \textit{#1}}

\newenvironment{psalmus}{}{}

%%%% Prejmenovat na latinske:
\newcommand{\nadpisZalmu}[1]{
  \hspace{2cm}\textbf{#1}\vspace{2mm}%
  \nopagebreak%
}

\setlength{\columnsep}{30pt} % prostor mezi sloupci

\begin{document}

\emph{Texti secundum Antiphonale Romanum 1912. 
Translationes, nisi aliud notatum, J.P.
Translationes capitulorum sumptae sunt ex:
Jeruzalémská bible, Praha-Kostelní Vydří 2009.}

\begin{titulusOfficii}
\dies{Die 8. Septembris.}
\nomenFesti{In Nativitate B. Mariae Virginis.}
\celebratio{Duplex 2. classis.} % puvodne "cum Octava." Oktavy byly ale zruseny
\end{titulusOfficii}

\hora{In I. Vesperis.} %%%%%%%%%%%%%%%%%%%%%%%%%%%%%%%%%%%%%%%%%%%%%%%%%%%%%

\input{temporalia/ps109.tex}

\hora{Ad Laudes.} %%%%%%%%%%%%%%%%%%%%%%%%%%%%%%%%%%%%%%%%%%%%%%%%%%%%%%%%%%

Capitulum. \titCapitulum{Eccli. 24. b}
\begin{multicols*}{2}
\lettrine{A}{b} inítio et ante saécula creáta sum,~\dag\mbox{}
et usque ad futúrum saéculum non désinam,~*
et in habitatióne sancta coram ipso ministrávi.

\columnbreak

% preklad Jeruz. bible
Před věky, na počátku mě stvořil,
potrvám věčně. 
Ve svatém Stanu jsem před ním konala službu.
\end{multicols*}

%%% TODO: Hymnus

\begin{multicols*}{2}
℣. Natívitas est hódie sanctae Maríae \textbf{Vír}ginis.\\
\indent ℟. Cujus vita ínclyta cunctas illústrat ec\textbf{clé}sias.

\columnbreak

% preklad J.P.
Dnes je Narození svaté Panny Marie.\\
\indent Jejíž předrahý život osvěcuje všechny církve.
\end{multicols*}

Oratio.
\begin{multicols*}{2}
\lettrine{F}{amulis} tuis, quaésumus Dómine, coeléstis grátiae munus impertíre:~\dag\mbox{}
ut quibus beátae Vírginis partus éxstitit salútis exórdium,~*
Nativitátis ejus votíva solémnitas pacis tríbuat increméntum.
Per Dóminum.

\columnbreak

% preklad J.P.
Prosíme tě, Bože, 
uděl nám, svým služebníkům, dar nebeské milosti,
aby těm, kterým \notitia{?(}se porod blahoslavené Panny zjevil jako počátek spásy,\notitia{)?}
\notitia{(co s \emph{votiva}?)} slavnost jejího narození přinesla
rozhojnění pokoje.
Skrze našeho Pána.
\end{multicols*}

\hora{In II. Vesperis.} %%%%%%%%%%%%%%%%%%%%%%%%%%%%%%%%%%%%%%%%%%%%%%%%%%%%%

\rubrica{Omnia ut in I. Vesperis, praeter sequentia.

}

\end{document}
