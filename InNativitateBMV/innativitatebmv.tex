% LuaLaTeX

\documentclass[a4paper, twoside, 12pt]{article}
\usepackage[latin]{babel} 
\usepackage[landscape, left=3cm, right=1.5cm, top=2cm, bottom=1cm]{geometry} % okraje stranky

\usepackage{fontspec}
% \setmainfont{Gentium Book Basic}
\setmainfont{Junicode}

\usepackage{multicol}
\usepackage{color}
\usepackage{lettrine}
\usepackage{fancyhdr}

% usual packages loading:
\usepackage{luatextra}
\usepackage{graphicx} % support the \includegraphics command and options
\usepackage{gregoriotex} % for gregorio score inclusion
\usepackage{gregoriosyms}
\usepackage{wrapfig} % figures wrapped by the text
\usepackage{parcolumns}

\newenvironment{titulusOfficii}{\begin{center}}{\end{center}}
\newcommand{\dies}[1]{#1

}
\newcommand{\nomenFesti}[1]{\textbf{\Large #1}

}
\newcommand{\celebratio}[1]{#1

}

\newcommand{\hora}[1]{\vspace{0.5cm}{\large \textbf{#1}}
% from some reason the following doesn't work:
\fancyhead[LE]{\thepage\ / #1}
\fancyhead[RO]{#1 / \thepage}

}

\newcommand{\rubrica}[1]{\vspace{3mm}\textit{#1}}

\newcommand{\notitia}[1]{\textcolor{red}{#1}}

\newcommand{\titCapitulum}[1]{\hfill \textit{#1}}

\newcommand{\translatioCantus}[1]{\vspace{1mm}{\noindent\footnotesize #1}}

% pruznejsi varianta nasledujiciho - umoznuje nastavit sirku sloupce
% s prekladem
\newcommand{\psalmusEtTranslatioB}[3]{
  \vspace{0.5cm}
  \begin{parcolumns}[colwidths={2=#3}, nofirstindent=true]{2}
    \colchunk{\input{#1}}
    \colchunk{\vspace{-0.5cm}{\footnotesize \input{#2}}}
  \end{parcolumns}
}

\newcommand{\psalmusEtTranslatio}[2]{
  \psalmusEtTranslatioB{#1}{#2}{8.5cm}
}

% volne misto nad antifonami, kam si zpevaci dokresli neumy
\newcommand{\hicSuntNeumae}{\vspace{1.3cm}}

% prepinani mista mezi notovymi osnovami: pro neumovane a neneumovane zpevy
\newcommand{\cantusCumNeumis}{
  \setgrefactor{17}
  \global\advance\grespaceabovelines by 10mm%
}
\newcommand{\cantusSineNeumas}{
  \setgrefactor{17}
  \global\advance\grespaceabovelines by -10mm%
}

% pars officii, i.e. "oratio", ...
\newcommand{\pars}[1]{\textbf{#1}}

\newenvironment{psalmus}{
  \setlength{\parindent}{0pt}
  \setlength{\parskip}{5pt}
}{
  \setlength{\parindent}{10pt}
  \setlength{\parskip}{10pt}
}

%%%% Prejmenovat na latinske:
\newcommand{\nadpisZalmu}[1]{
  \hspace{2cm}\textbf{#1}\vspace{2mm}%
  \nopagebreak%

}

%%%% Preklady jednotlivych zpevu (nektere se opakuji, a je dobre mit je
% vsechny na jedne hromade)

\newcommand{\translatioAntI}{\translatioCantus{Jasné narození slavné Panny Marie,
z pokolení (dosl. ze semene) Abrahámova, vzešlé z kmene Judova, z rodu Davidova.}}
\newcommand{\translatioAntII}{\translatioCantus{Dnes je Narození svaté Panny 
Marie, jejíž předrahý život osvěcuje všechny církve.}}
\newcommand{\translatioAntIII}{\translatioCantus{Maria, jež vzešla 
z královského rodu, září; myslí i duchem ji v modlitbách zbožně prosíme, aby 
nám pomáhala. \notitia{(Překlad druhé části poněkud nejistý.)}}}
\newcommand{\translatioAntIV}{\translatioCantus{Srdcem i duchem pějme Kristu 
k slávě o této svaté slavnosti vznešené Rodičky Boží Marie.}}
\newcommand{\translatioAntV}{\translatioCantus{Příjemně \notitia{?} 
oslavujme Narození blahoslavené Marie,
aby se ona za nás přimlouvala u Pána Ježíše Krista.}}
\newcommand{\translatioRespVesp}{\translatioCantus{Buď zdráva, Maria,
plná milosti: * Pán s tebou. ℣. Požehnaná jsi mezi ženami,
a požehnaný plod života (ve smyslu lůna, břicha) tvého.}}
\newcommand{\translatioAntMagnificatI}{\translatioCantus{Konejme památku
nejdůstojnějšího narození slavné Panny Marie,
jíž se dostalo mateřské důstojnosti bez ztráty panenské cudnosti.}}
\newcommand{\translatioAntBenedictus}{\translatioCantus{Slavnostně slavme 
dnešní narození Marie, vždy Panny a Rodičky Boží: v něm se objevuje
vysokost trůnu (totiž Marie, trůnu Božího Syna), aleluja. \notitia{Překlad
krajně nejistý.}}}
\newcommand{\translatioAntMagnificatII}{\translatioCantus{Tvé narození,
Bohorodičko Panno, vyhlásilo radost celému světu:
z tebe totiž vzešlo Slunce spravedlnosti, Kristus, náš Bůh:
jenž zrušil kletbu a dal nám požehnání: přemohl smrt a dal nám život věčný.}}

\newcommand{\translatioIntroitus}{\translatioCantus{Radujme se všichni
v Pánu, slavíce svátek ke cti Panny Marie: z něj se radují andělé
a spoluchválí Božího Syna. \textit{Žalm:} Má ústa vydala dobré slovo,
přednáším svá díla králi.}}
\newcommand{\translatioGraduale}{\translatioCantus{Požehnaná a ctihodná jsi,
Panno Maria: nedotčená (co do panenství) jsi byla shledána matkou
Spasitele. ℣. Panno Boží Rodičko, ten, jehož nepojme ani celý svět,
se uzavřel do tvých útrob, když se stal člověkem.}}
\newcommand{\translatioAlleluia}{\translatioCantus{Aleluja. ℣. Skvělá slavnost
slavné Panny Marie, z pokolení (dosl. ze semene) Abrahámova, vzešlé z kmene 
Judova, z rodu Davidova.}}
\newcommand{\translatioOffertorium}{\translatioCantus{Blažená jsi, Panno Maria,
tys nosila Stvořitele všeho; porodila jsi toho, který tě utvořil,
a na věky zůstáváš Pannou.}}
\newcommand{\translatioCommunio}{\translatioCantus{Budou mě blahoslavit
všechna pokolení, protože mi učinil veliké věci ten, který je mocný.}}


%%%% Vicekrat opakovane kousky

\newcommand{\versusNativitasEst}{
  \pars{Versus.}

  % Versus. %%%
    \includescore{cantus/amon33/versus-nativitasest.tex}
    % \vspace{-5mm}
    \noindent ℟.\hspace{0.5mm}Cujus vita ínclyta cunctas illústrat ecclési\textit{as.}
    
    \noindent \translatioCantus{℣. Dnes je Narození svaté Panny Marie. ℟. Jejíž předrahý život osvěcuje všechny církve.}
}

\newcommand{\capitulumAbInitioEtResponsorium}{
  \pars{Capitulum.} \titCapitulum{Eccli. 24. b}

  \cantusSineNeumas
  \includescore{cantus/amon33/capitulum-AbInitio.tex}

  % preklad Jeruz. bible
  \translatioCantus{Před věky, na počátku mě stvořil, potrvám věčně. 
    Ve svatém Stanu jsem před ním konala službu.}

  \vspace{1cm}
  \pars{Responsorium breve.}

  \gresetfirstlineaboveinitial{\small \textsc{\textbf{VI.}}}{\small \textsc{\textbf{VI.}}}
  \includescore{cantus/amon33/resp1v.tex}

  \translatioRespVesp
}

\newcommand{\anteOrationem}{
  \rubrica{Ante Orationem, cantatur a Superiore:}

  \pars{Supplicatio Litaniae.}

  \includescore{../cantuscommunes/amon33/supplicatiolitaniae.tex}

  \pars{Oratio Dominica.}

  \includescore{../cantuscommunes/amon33/oratiodominica.tex}

  \rubrica{Deinde dicitur ab Hebdomadario:}

  \includescore{../cantuscommunes/amon33/dominusvobiscum-solemnis.tex}

  \rubrica{In choro monialium loco Dominus vobiscum dicitur:}

  \includescore{../cantuscommunes/amon33/domineexaudi.tex}
}

\setlength{\columnsep}{30pt} % prostor mezi sloupci

%%%%%%%%%%%%%%%%%%%%%%%%%%%%%%%%%%%%%%%%%%%%%%%%%%%%%%%%%%%%%%%%%%%%%%%%%%%%%%%%%%%%%%%%%%%%%%%%%%%%%%%%%%%%%
\begin{document}

% Here we set the space around the initial.
% Please report to http://home.gna.org/gregorio/gregoriotex/details for more details and options
\setspaceafterinitial{2.2mm plus 0em minus 0em}
\setspacebeforeinitial{2.2mm plus 0em minus 0em}

% Here we set the initial font. Change 43 if you want a bigger initial.
\def\greinitialformat#1{%
{\fontsize{43}{43}\selectfont #1}%
}

\pagestyle{empty}

%%%% Titulni stranka
\begin{titulusOfficii}
\dies{Die 8. Septembris.}
\nomenFesti{In Nativitate B. Mariae Virginis.}
\celebratio{Duplex 2. classis.} % puvodne "cum Octava." Oktavy byly ale zruseny
\end{titulusOfficii}

\vfill

\begin{center}
Ad usum et secundum consuetudines chori \guillemotright Conventus Choralis\guillemotleft.

Editio Sancti Wolfgangi 2012
\end{center}

\pagebreak

\renewcommand{\headrulewidth}{0pt} % no horiz. rule at the header
\fancyhf{}
\pagestyle{fancy}

\hora{In I. Vesperis.} %%%%%%%%%%%%%%%%%%%%%%%%%%%%%%%%%%%%%%%%%%%%%%%%%%%%%

\label{deusinadiutorium}

\includescore{../cantuscommunes/amon33/deusinadiutorium-communis.tex}

\vfill

\pagebreak

\cantusCumNeumis

\pars{psalmus 1.}

\hicSuntNeumae
\gresetfirstlineaboveinitial{\small \textsc{\textbf{VIII.G}}}{\small \textsc{\textbf{VIII.G}}}
\includescore{cantus/amon33/ant1.tex}

\translatioAntI

\titCapitulum{Psalmus 109.}

\includescore{temporalia/ps109-initium-viii-G-auto.tex}

%\translatioPsalmi{temporalia/ps109-boh.tex}
%\input{temporalia/ps109.tex}

\psalmusEtTranslatio{temporalia/ps109.tex}{temporalia/ps109-boh.tex}

\vfill

\pagebreak

\pars{psalmus 2.}

\hicSuntNeumae
\gresetfirstlineaboveinitial{\small \textsc{\textbf{VII.c}}}{\small \textsc{\textbf{VII.c}}}
\includescore{cantus/amon33/ant2.tex}

\translatioAntII

\titCapitulum{Psalmus 112.}

\includescore{temporalia/ps112-initium-vii-c-auto.tex}

% \translatioPsalmi{temporalia/ps112-boh.tex}
% \input{temporalia/ps112.tex}

\psalmusEtTranslatio{temporalia/ps112.tex}{temporalia/ps112-boh.tex}

\vfill

\pagebreak

\pars{psalmus 3.}

\hicSuntNeumae
\gresetfirstlineaboveinitial{\small \textsc{\textbf{VI.F}}}{\small \textsc{\textbf{VI.F}}}
\includescore{cantus/amon33/ant3.tex}

\translatioAntIII

\titCapitulum{Psalmus 121.}

\includescore{temporalia/ps121-initium-vi-F-auto.tex}

% \translatioPsalmi{temporalia/ps121-boh.tex}
% \input{temporalia/ps121.tex}

\psalmusEtTranslatio{temporalia/ps121.tex}{temporalia/ps121-boh.tex}

\vfill

\pagebreak

\pars{psalmus 4.}

\hicSuntNeumae
\gresetfirstlineaboveinitial{\small \textsc{\textbf{VIII.G}}}{\small \textsc{\textbf{VIII.G}}}
\includescore{cantus/amon33/ant4.tex}

\translatioAntIV

\titCapitulum{Psalmus 126.}

\includescore{temporalia/ps126-initium-viii-G-auto.tex}

%\translatioPsalmi{temporalia/ps126-boh.tex}
%\input{temporalia/ps126.tex}
\psalmusEtTranslatio{temporalia/ps126.tex}{temporalia/ps126-boh.tex}

\vfill

\pagebreak

\raggedcolumns

% Capitulum. %%%
\label{capitulum}
\capitulumAbInitioEtResponsorium

\vfill

\pagebreak

% Hymnus. %%%
\pars{Hymnus.}

\gresetfirstlineaboveinitial{\small \textsc{\textbf{I.}}}{\small \textsc{\textbf{I.}}}
\includescore{cantus/amon33/hym-AveMarisStella.tex}
\psalmusEtTranslatio{cantus/amon33/hym-AveMarisStella-text.tex}{cantus/amon33/hym-AveMarisStella-bohtext.tex}

\versusNativitasEst

\pagebreak

\cantusCumNeumis

\pars{Canticum B. Mariae V.}\rubrica{ - in I. vesperis:}

\hicSuntNeumae
\gresetfirstlineaboveinitial{\small \textsc{\textbf{I.D2}}}{\small \textsc{\textbf{I.D2}}}
\includescore{cantus/amon33/ant-magn-vesp1.tex}

\translatioAntMagnificatI

\cantusSineNeumas
\includescore{cantus/amon33/magnificat-initium-i-D2.tex}

\psalmusEtTranslatioB{temporalia/magnificat-iD2.tex}{temporalia/magnificat-boh.tex}{10cm}

\pagebreak

\pars{Canticum B. Mariae V.}\rubrica{ - in II. vesperis:}
\label{magnificatIIvesp}

\cantusCumNeumis

\hicSuntNeumae
\gresetfirstlineaboveinitial{\small \textsc{\textbf{I.f}}}{\small \textsc{\textbf{I.f}}}
\includescore{cantus/amon33/ant-magn-vesp2.tex}

\translatioAntMagnificatII

% antifona je dlouha a mista dost, dame Magnificat na novou stranku
\vfill
\pagebreak 

\cantusSineNeumas
\includescore{cantus/amon33/magnificat-initium-i-f.tex}

% maly svindl, ale prizvukova struktura Isoll-D2 a Isoll-f je stejna
\psalmusEtTranslatio{temporalia/magnificat-iD2.tex}{temporalia/magnificat-boh.tex}

\pagebreak

\label{oratio}
\anteOrationem

\pagebreak

% Oratio. %%%
\pars{Oratio.}

\includescore{cantus/amon33/oratio.tex}
\translatioCantus{Prosíme tě, Bože, 
uděl nám, svým služebníkům, dar nebeské milosti,
aby těm, kterým \notitia{?(}se porod blahoslavené Panny zjevil jako počátek spásy,\notitia{)?}
\notitia{(co s \emph{votiva}?)} slavnost jejího narození přinesla
rozhojnění pokoje.
Skrze našeho Pána.}

\vspace{1cm}
\rubrica{Hebdomadarius dicit iterum Dominus vobiscum. Postea cantatur a cantore:}
\vspace{2mm}

\includescore{../cantuscommunes/amon33/benedicamus-duplex-vesperae.tex}

\vfill

\pagebreak

\hora{Ad Laudes.} %%%%%%%%%%%%%%%%%%%%%%%%%%%%%%%%%%%%%%%%%%%%%%%%%%%%%%%%%%

% Psalmi festivi (AM33, pg. 721):
% 66 // 92, 99, 62, Dan3, 148+149+150

\vspace{1cm}
\includescore{../cantuscommunes/amon33/deusinadiutorium-communis.tex}
\vspace{1cm}

\cantusCumNeumis

\pars{psalmus 1.}

\titCapitulum{Psalmus 66.}

\includescore{temporalia/ps66-initium-dir-auto.tex}

%\translatioPsalmi{temporalia/ps66-boh.tex}
%\input{temporalia/ps66.tex}
\psalmusEtTranslatio{temporalia/ps66.tex}{temporalia/ps66-boh.tex}

\vfill

\pagebreak

\pars{psalmus 2.}

\hicSuntNeumae
\gresetfirstlineaboveinitial{\small \textsc{\textbf{VIII.G}}}{\small \textsc{\textbf{VIII.G}}}
\includescore{cantus/amon33/ant1.tex}

\translatioAntI

\titCapitulum{Psalmus 92.}

\includescore{temporalia/ps92-initium-viii-G-auto.tex}

%\translatioPsalmi{temporalia/ps92-boh.tex}
%\input{temporalia/ps92.tex}
\psalmusEtTranslatio{temporalia/ps92.tex}{temporalia/ps92-boh.tex}

\vfill

\pagebreak

\pars{psalmus 3.}

\hicSuntNeumae
\gresetfirstlineaboveinitial{\small \textsc{\textbf{VII.c}}}{\small \textsc{\textbf{VII.c}}}
\includescore{cantus/amon33/ant2.tex}

\translatioAntII

\titCapitulum{Psalmus 99.}

\includescore{temporalia/ps99-initium-vii-c-auto.tex}

%\translatioPsalmi{temporalia/ps99-boh.tex}
%\input{temporalia/ps99.tex}
\psalmusEtTranslatio{temporalia/ps99.tex}{temporalia/ps99-boh.tex}

\vfill

\pagebreak

\pars{psalmus 4.}

\hicSuntNeumae
\gresetfirstlineaboveinitial{\small \textsc{\textbf{VI.F}}}{\small \textsc{\textbf{VI.F}}}
\includescore{cantus/amon33/ant3.tex}

\translatioAntIII

\titCapitulum{Psalmus 62.}

\includescore{temporalia/ps62-initium-vi-F-auto.tex}

%\translatioPsalmi{temporalia/ps62-boh.tex}
%\input{temporalia/ps62.tex}
\psalmusEtTranslatio{temporalia/ps62.tex}{temporalia/ps62-boh.tex}

\vfill

\pagebreak

\pars{psalmus 5.}

\hicSuntNeumae
\gresetfirstlineaboveinitial{\small \textsc{\textbf{VIII.G}}}{\small \textsc{\textbf{VIII.G}}}
\includescore{cantus/amon33/ant4.tex}

\translatioAntIV

\titCapitulum{Canticum trium puerorum, Dan. 3, 57-88 et 56.}

\includescore{temporalia/dan3-initium-viii-G-auto.tex}

\psalmusEtTranslatioB{temporalia/dan3.tex}{temporalia/dan3-boh.tex}{10cm} % Canticum trium

\rubrica{Hic non dicitur Gloria Patri, neque Amen.}
\vspace{1cm}

\hicSuntNeumae
\includescore{cantus/amon33/ant4.tex} % repeat the antiphon - new page

\vfill

\pagebreak

\pars{psalmus 6.}

\hicSuntNeumae
\gresetfirstlineaboveinitial{\small \textsc{\textbf{VII.c}}}{\small \textsc{\textbf{VII.c}}}
\includescore{cantus/amon33/ant5.tex}

\translatioAntV

%
\titCapitulum{Psalmus 148.}

\includescore{temporalia/ps148-initium-vii-c-auto.tex}

\newlength{\psVItransW}
\setlength{\psVItransW}{10.5cm}

%\translatioPsalmi{temporalia/ps148-boh.tex}
%\input{temporalia/ps148.tex}
\psalmusEtTranslatioB{temporalia/ps148.tex}{temporalia/ps148-boh.tex}{\psVItransW}

\rubrica{Hic non dicitur Gloria Patri.}

%
\titCapitulum{Psalmus 149.}

\includescore{temporalia/ps149-initium-vii-c-auto.tex}

%\translatioPsalmi{temporalia/ps149-boh.tex}
%\input{temporalia/ps149.tex}
\psalmusEtTranslatioB{temporalia/ps149.tex}{temporalia/ps149-boh.tex}{\psVItransW}

\rubrica{Hic non dicitur Gloria Patri.}

%
\titCapitulum{Psalmus 150.}

\includescore{temporalia/ps150-initium-vii-c-auto.tex}

%\translatioPsalmi{temporalia/ps150-boh.tex}
%\input{temporalia/ps150.tex}
\psalmusEtTranslatioB{temporalia/ps150.tex}{temporalia/ps150-boh.tex}{\psVItransW}

\pagebreak

\capitulumAbInitioEtResponsorium
\vfill

\pagebreak

\cantusSineNeumas

\pars{Hymnus.}

\includescore{cantus/amon33/hym-OGloriosaDomina.tex}
\psalmusEtTranslatio{cantus/amon33/hym-OGloriosaDomina-text.tex}{cantus/amon33/hym-OGloriosaDomina-bohtext.tex}

\vspace{0.5cm}

\versusNativitasEst

\vfill

\pagebreak

\cantusCumNeumis

\pars{Canticum Zachariae.}

\hicSuntNeumae
\gresetfirstlineaboveinitial{\small \textsc{\textbf{VIII.G}}}{\small \textsc{\textbf{VIII.G}}}
\includescore{cantus/amon33/ant-ben-laud.tex}

\translatioAntBenedictus

\includescore{temporalia/benedictus-initium-viiiSoll-G-auto.tex}

\psalmusEtTranslatioB{temporalia/benedictus-viiiG.tex}{temporalia/benedictus-boh.tex}{10.5cm}

\vspace{0.5cm}

\vfill

\pagebreak

\label{oratio}
\anteOrationem

\pagebreak

% Oratio. %%%
\pars{Oratio.}

\includescore{cantus/amon33/oratio.tex}
\translatioCantus{Prosíme tě, Bože, 
uděl nám, svým služebníkům, dar nebeské milosti,
aby těm, kterým \notitia{?(}se porod blahoslavené Panny zjevil jako počátek spásy,\notitia{)?}
\notitia{(co s \emph{votiva}?)} slavnost jejího narození přinesla
rozhojnění pokoje.
Skrze našeho Pána.}

\vspace{1cm}
\rubrica{Hebdomadarius dicit iterum Dominus vobiscum. Postea cantatur a cantore:}
\vspace{2mm}

\includescore{../cantuscommunes/amon33/benedicamus-duplex-laudes.tex}

\vfill

\pagebreak

\hora{Ad Missam.} %%%%%%%%%%%%%%%%%%%%%%%%%%%%%%%%%%%%%%%%%%%%%%%%%%%%%

\pars{Antiphona ad introitum.}

\hicSuntNeumae
\gresetfirstlineaboveinitial{\small \textsc{\textbf{I}}}{\small \textsc{\textbf{I}}}
\includescore{cantus/triplex76/introitus-GaudeamusOmnes.tex}

\translatioIntroitus

\vfill
\pagebreak

\pars{Graduale.}

\hicSuntNeumae
\gresetfirstlineaboveinitial{\small \textsc{\textbf{IV}}}{\small \textsc{\textbf{IV}}}
\includescore{cantus/triplex76/graduale-BenedictaEtVenerabilis.tex}

\translatioGraduale

\vfill
\pagebreak

\pars{Alleluia.}

\hicSuntNeumae
\gresetfirstlineaboveinitial{\small \textsc{\textbf{VII}}}{\small \textsc{\textbf{VII}}}
\includescore{cantus/triplex76/alleluia-SolemnitasGloriosae.tex}

\translatioAlleluia

\vfill
\pagebreak

\pars{Offertorium.}

\hicSuntNeumae
\gresetfirstlineaboveinitial{\small \textsc{\textbf{VIII}}}{\small \textsc{\textbf{VIII}}}
\includescore{cantus/triplex76/offertorium-BeataEs.tex}

\translatioOffertorium

\vspace{3mm}

\pars{Communio.}

\hicSuntNeumae
\gresetfirstlineaboveinitial{\small \textsc{\textbf{VI}}}{\small \textsc{\textbf{VI}}}
\includescore{cantus/triplex76/communio-BeatamMeDicent.tex}

\translatioCommunio

\titCapitulum{Canticum Magnificat: Luc. 1,46-47.50.51.52.53.54.55}

\cantusSineNeumas
\includescore{cantus/triplex76/communio-versus-Magnificat.tex}

% \pagebreak

\hora{In II. Vesperis.} %%%%%%%%%%%%%%%%%%%%%%%%%%%%%%%%%%%%%%%%%%%%%%%%%%%%%

\rubrica{Omnia ut in I. Vesperis, pg. \pageref{deusinadiutorium}, 
praeter antiphonam ad Magnificat, quae est propria, 
pg. \pageref{magnificatIIvesp}.}

\newpage
\pagestyle{empty}

\mbox{} % musi tu byt, jinak by TeX prazdnou stranku ignoroval a vynechal ji 
\newpage % colophon chce byt na poslednim listu _vzadu_

%%% COLOPHON

\vspace*{5cm}

Fontes. 
Textus et cantus officii divini secundum 
Antiphonale Monasticum, Solesmis 1933. /
Textus et cantus missae secundum
Graduale triplex, Solesmis 1979. /
Versus ad communionem secundum
http://media.musicasacra.com/pdf/beatamme.pdf (12.8.2012). /
Translatio capituli sumpta est ex:
Jeruzalémská bible, Praha-Kostelní Vydří 2009. /
Translationes psalmorum ex
Hejčl Jan: Žaltář čili Kniha žalmů, Praha 1922.

Collaborantes.
Textus latinos cantusque transcripsit et omnem laborem typographicum peregit
Jakub Pavlík. /
Psalmos in lingua bohemica de libro supra dicto transcripsit
Barbora Maturová.

Instrumenta adhibita.
LuaTeX, http://www.luatex.org / 
Gregorio, http://home.gna.org/gregorio /
typi Junicode, http://junicode.sourceforge.net

\begin{center}
Liber hic imprimis ad usum chori 
\guillemotright Conventus Choralis\guillemotleft\ 
paratus est
et secundum eius consuetudines.
http://www.introitus.cz

\vspace{1cm}

{\large Editio Sancti Wolfgangi 2012.}

\vspace{2mm}

Series \guillemotright Conventus\guillemotleft, vol. I.

\vspace{1cm}

http://stiwolfgangi.xf.cz

\end{center}

\vfill

\end{document}
