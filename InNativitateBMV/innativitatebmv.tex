% LuaLaTeX

\documentclass[a4paper, twoside, 12pt]{article}
\usepackage[latin]{babel} 
\usepackage[landscape, left=3cm, right=1.5cm, top=2cm, bottom=2cm]{geometry} % okraje stranky

\usepackage{fontspec}
% \setmainfont{Gentium Book Basic}
\setmainfont{Junicode}

\usepackage{multicol}
\usepackage{color}
\usepackage{lettrine}
\usepackage{fancyhdr}

% usual packages loading:
\usepackage{luatextra}
\usepackage{graphicx} % support the \includegraphics command and options
\usepackage{gregoriotex} % for gregorio score inclusion

\newenvironment{titulusOfficii}{\begin{center}}{\end{center}}
\newcommand{\dies}[1]{#1

}
\newcommand{\nomenFesti}[1]{\textbf{\Large #1}

}
\newcommand{\celebratio}[1]{#1

}

\newcommand{\hora}[1]{\vspace{0.5cm}{\large \textbf{#1}}
% from some reason the following doesn't work:
\fancyhead[LE]{\thepage\ / #1}
\fancyhead[RO]{#1 / \thepage}

}

\newcommand{\rubrica}[1]{\vspace{3mm}\textit{#1}}

\newcommand{\notitia}[1]{\textcolor{red}{#1}}

\newcommand{\titCapitulum}[1]{\hfill \textit{#1}}

\newcommand{\translatioCantus}[1]{\vspace{1mm}{\small #1}}

% volne misto nad antifonami, kam si zpevaci dokresli neumy
\newcommand{\hicSuntNeumae}{\vspace{1.3cm}}

% prepinani mista mezi notovymi osnovami: pro neumovane a neneumovane zpevy
\newcommand{\cantusCumNeumis}{
  \setgrefactor{17}
  \global\advance\grespaceabovelines by 10mm%
}
\newcommand{\cantusSineNeumas}{
  \setgrefactor{17}
  \global\advance\grespaceabovelines by -10mm%
}

% pars officii, i.e. "oratio", ...
\newcommand{\pars}[1]{\textbf{#1}}

\newenvironment{psalmus}{
  \setlength{\parindent}{0pt}
  \setlength{\parskip}{5pt}
}{
  \setlength{\parindent}{10pt}
  \setlength{\parskip}{10pt}
}

%%%% Prejmenovat na latinske:
\newcommand{\nadpisZalmu}[1]{
  \hspace{2cm}\textbf{#1}\vspace{2mm}%
  \nopagebreak%

}

%%%% Preklady jednotlivych zpevu (nektere se opakuji, a je dobre mit je
% vsechny na jedne hromade)

\newcommand{\translatioAntI}{\translatioCantus{Jasné narození slavné Panny Marie,
z pokolení (dosl. ze semene) Abrahámova, vzešlé z kmene Judova, z rodu Davidova.}}
\newcommand{\translatioAntII}{\translatioCantus{Dnes je Narození svaté Panny 
Marie, jejíž předrahý život osvěcuje všechny církve.}}
\newcommand{\translatioAntIII}{\translatioCantus{Maria, jež vzešla 
z královského rodu, září; myslí i duchem ji v modlitbách zbožně prosíme, aby 
nám pomáhala. \notitia{(Překlad druhé části poněkud nejistý.)}}}
\newcommand{\translatioAntIV}{\translatioCantus{Srdcem i duchem pějme Kristu 
k slávě o této svaté slavnosti vznešené Rodičky Boží Marie.}}
\newcommand{\translatioAntV}{\translatioCantus{Příjemně \notitia{?} 
oslavujme Narození blahoslavené Marie,
aby se ona za nás přimlouvala u Pána Ježíše Krista.}}
\newcommand{\translatioRespVesp}{\translatioCantus{Buď zdráva, Maria,
plná milosti: * Pán s tebou. ℣. Požehnaná jsi mezi ženami,
a požehnaný plod tvého lůna.}}

%%%% Vicekrat opakovane kousky

\newcommand{\versusNativitasEst}{
  \pars{Versus.}

  % Versus. %%%
    \includescore{cantus/amon33/versus-nativitasest.tex}
    \noindent ℟.\hspace{0.5mm}Cujus vita ínclyta cunctas illústrat ecclési\textit{as.}

    \noindent \translatioCantus{℣. Dnes je Narození svaté Panny Marie. ℟. Jejíž předrahý život osvěcuje všechny církve.}
}

\setlength{\columnsep}{30pt} % prostor mezi sloupci

%%%%%%%%%%%%%%%%%%%%%%%%%%%%%%%%%%%%%%%%%%%%%%%%%%%%%%%%%%%%%%%%%%%%%%%%%%%%%%%%%%%%%%%%%%%%%%%%%%%%%%%%%%%%%
\begin{document}

% Here we set the space around the initial.
% Please report to http://home.gna.org/gregorio/gregoriotex/details for more details and options
\setspaceafterinitial{2.2mm plus 0em minus 0em}
\setspacebeforeinitial{2.2mm plus 0em minus 0em}

% Here we set the initial font. Change 43 if you want a bigger initial.
\def\greinitialformat#1{%
{\fontsize{43}{43}\selectfont #1}%
}

\pagestyle{empty}

%%%% Titulni stranka
\begin{titulusOfficii}
\dies{Die 8. Septembris.}
\nomenFesti{In Nativitate B. Mariae Virginis.}
\celebratio{Duplex 2. classis.} % puvodne "cum Octava." Oktavy byly ale zruseny
\end{titulusOfficii}

\vfill

\begin{center}
Ad usum et secundum consuetudines chori \guillemotright Conventus Choralis\guillemotleft.

Editio Sancti Wolfgangi 2012
\end{center}

\pagebreak

\renewcommand{\headrulewidth}{0pt} % no horiz. rule at the header
\fancyhf{}
\pagestyle{fancy}

\hora{In I. Vesperis.} %%%%%%%%%%%%%%%%%%%%%%%%%%%%%%%%%%%%%%%%%%%%%%%%%%%%%

\label{deusinadiutorium}

\includescore{../cantuscommunes/amon33/deusinadiutorium-communis.tex}

\vfill

\pagebreak

\cantusCumNeumis

\pars{psalmus 1.}

\hicSuntNeumae
\gresetfirstlineaboveinitial{\small \textsc{\textbf{VIII.G}}}{\small \textsc{\textbf{VIII.G}}}
\includescore{cantus/amon33/ant1.tex}

\translatioAntI

\titCapitulum{Psalmus 109.}

\includescore{temporalia/ps109-initium-viii-G-auto.tex}

\input{temporalia/ps109.tex}

\vfill

\pagebreak

\pars{psalmus 2.}

\hicSuntNeumae
\gresetfirstlineaboveinitial{\small \textsc{\textbf{VII.c}}}{\small \textsc{\textbf{VII.c}}}
\includescore{cantus/amon33/ant2.tex}

\translatioAntII

\titCapitulum{Psalmus 112.}

\includescore{temporalia/ps112-initium-vii-c-auto.tex}

\input{temporalia/ps112.tex}

\vfill

\pagebreak

\pars{psalmus 3.}

\hicSuntNeumae
\gresetfirstlineaboveinitial{\small \textsc{\textbf{VI.F}}}{\small \textsc{\textbf{VI.F}}}
\includescore{cantus/amon33/ant3.tex}

\translatioAntIII

\titCapitulum{Psalmus 121.}

\includescore{temporalia/ps121-initium-vi-F-auto.tex}

\input{temporalia/ps121.tex}

\vfill

\pagebreak

\pars{psalmus 4.}

\hicSuntNeumae
\gresetfirstlineaboveinitial{\small \textsc{\textbf{VIII.G}}}{\small \textsc{\textbf{VIII.G}}}
\includescore{cantus/amon33/ant4.tex}

\translatioAntIV

\titCapitulum{Psalmus 126.}

\includescore{temporalia/ps126-initium-viii-G-auto.tex}

\input{temporalia/ps126.tex}

\vfill

\pagebreak

\raggedcolumns

% Capitulum. %%%
\label{capitulum}
\pars{Capitulum.} \titCapitulum{Eccli. 24. b} 
\begin{multicols}{2}
\lettrine{A}{b} inítio et ante saécula creáta sum,~\dag\mbox{}
et usque ad futúrum saéculum non désinam,~*
et in habitatióne sancta coram ipso ministrávi.

\columnbreak

% preklad Jeruz. bible
Před věky, na počátku mě stvořil,
potrvám věčně. 
Ve svatém Stanu jsem před ním konala službu.
\end{multicols}

\cantusSineNeumas

\pars{Responsorium breve.}

\gresetfirstlineaboveinitial{\small \textsc{\textbf{VI.}}}{\small \textsc{\textbf{VI.}}}
\includescore{cantus/amon33/resp1v.tex}

\translatioRespVesp

\vfill

\pagebreak

% Hymnus. %%%
\pars{Hymnus.}

\gresetfirstlineaboveinitial{\small \textsc{\textbf{I.}}}{\small \textsc{\textbf{I.}}}
\includescore{cantus/amon33/hym-AveMarisStella.tex}
2. Sumens illud Ave Gabriélis ore,\\
Funda nos in pace, Mutans Hevae nomen.\\

3. Solve vincla reis, Profer lumen caecis:\\
Mala nostra pelle, Bona cuncta posce.\\

4. Monstra \textit{te} esse matrem: Sumat per te preces,\\
Qui pro nobis natus, Tulit esse tuus.\\

5. Virgo singuláris, Inter omnes mitis,\\
Nos culpis solútos, Mites fac et castos.\\

6. Vitam praesta puram, Iter para tuam:\\
Ut vidéntes Jesum, Semper collaetémur.\\

7. Sit laus Deo Patri, Summo Christo decus,\\
Spirítui Sancto, Tribus honor unus. Amen.\\


\versusNativitasEst

\pagebreak

\cantusCumNeumis

\pars{Canticum B. Mariae V.}

\hicSuntNeumae
\gresetfirstlineaboveinitial{\small \textsc{\textbf{I.D2}}}{\small \textsc{\textbf{I.D2}}}
\includescore{cantus/amon33/ant-magn-vesp1.tex}

\cantusSineNeumas
\includescore{cantus/amon33/magnificat-initium-i-D2.tex}

\input{temporalia/magnificat-iD2.tex}

\pagebreak

\label{oratio}
\rubrica{Ante Orationem, cantatur a Superiore:}

\pars{Supplicatio Litaniae.}

\includescore{../cantuscommunes/amon33/supplicatiolitaniae.tex}

\pars{Oratio Dominica.}

\includescore{../cantuscommunes/amon33/oratiodominica.tex}

\rubrica{Deinde dicitur ab Hebdomadario:}

\includescore{../cantuscommunes/amon33/dominusvobiscum-solemnis.tex}

\rubrica{In choro monialium loco Dominus vobiscum dicitur:}

\includescore{../cantuscommunes/amon33/domineexaudi.tex}

\pagebreak

% Oratio. %%%
\pars{Oratio.}
\begin{multicols}{2}
\lettrine{F}{amulis} tuis, quaésumus Dómine, coeléstis grátiae munus impertíre:~\dag\mbox{}
ut quibus beátae Vírginis partus éxstitit salútis exórdium,~*
Nativitátis ejus votíva solémnitas pacis tríbuat increméntum.
Per Dóminum.

\columnbreak

% preklad J.P.
Prosíme tě, Bože, 
uděl nám, svým služebníkům, dar nebeské milosti,
aby těm, kterým \notitia{?(}se porod blahoslavené Panny zjevil jako počátek spásy,\notitia{)?}
\notitia{(co s \emph{votiva}?)} slavnost jejího narození přinesla
rozhojnění pokoje.
Skrze našeho Pána.
\end{multicols}

\rubrica{Hebdomadarius dicit iterum Dominus vobiscum. Postea cantatur a cantore:}

\input{../cantuscommunes/amon33/benedicamus-duplex-vesperae.tex}

\vfill

\pagebreak

\hora{Ad Laudes.} %%%%%%%%%%%%%%%%%%%%%%%%%%%%%%%%%%%%%%%%%%%%%%%%%%%%%%%%%%

% Psalmi festivi (AM33, pg. 721):
% 66 // 92, 99, 62, Dan3, 148+149+150

\rubrica{Deus in adiutorium, pg. \pageref{deusinadiutorium}.}
\vspace{0.5cm}

\cantusCumNeumis

\pars{psalmus 1.}

\titCapitulum{Psalmus 66.}

\includescore{temporalia/ps66-initium-dir-auto.tex}

\input{temporalia/ps66.tex}

\vfill

\pagebreak

\pars{psalmus 2.}

\hicSuntNeumae
\gresetfirstlineaboveinitial{\small \textsc{\textbf{VIII.G}}}{\small \textsc{\textbf{VIII.G}}}
\includescore{cantus/amon33/ant1.tex}

\translatioAntI

\titCapitulum{Psalmus 92.}

\includescore{temporalia/ps92-initium-viii-G-auto.tex}

\input{temporalia/ps92.tex}

\vfill

\pagebreak

\pars{psalmus 3.}

\hicSuntNeumae
\gresetfirstlineaboveinitial{\small \textsc{\textbf{VII.c}}}{\small \textsc{\textbf{VII.c}}}
\includescore{cantus/amon33/ant2.tex}

\translatioAntII

\titCapitulum{Psalmus 99.}

\includescore{temporalia/ps99-initium-vii-c-auto.tex}

\input{temporalia/ps99.tex}

\vfill

\pagebreak

\pars{psalmus 4.}

\hicSuntNeumae
\gresetfirstlineaboveinitial{\small \textsc{\textbf{VI.F}}}{\small \textsc{\textbf{VI.F}}}
\includescore{cantus/amon33/ant3.tex}

\translatioAntIII

\titCapitulum{Psalmus 62.}

\includescore{temporalia/ps62-initium-vi-F-auto.tex}

\input{temporalia/ps62.tex}

\vfill

\pagebreak

\pars{psalmus 5.}

\hicSuntNeumae
\gresetfirstlineaboveinitial{\small \textsc{\textbf{VIII.G}}}{\small \textsc{\textbf{VIII.G}}}
\includescore{cantus/amon33/ant4.tex}

\translatioAntIV

\titCapitulum{Canticum trium puerorum, Dan. 3, 57-88 et 56.}

\includescore{temporalia/dan3-initium-viii-G-auto.tex}

\input{temporalia/dan3.tex} % Canticum trium

\rubrica{Hic non dicitur Gloria Patri, neque Amen.}

\vfill

\pagebreak

\pars{psalmus 6.}

\hicSuntNeumae
\gresetfirstlineaboveinitial{\small \textsc{\textbf{VII.c}}}{\small \textsc{\textbf{VII.c}}}
\includescore{cantus/amon33/ant5.tex}

\translatioAntV

%
\titCapitulum{Psalmus 148.}

\includescore{temporalia/ps148-initium-vii-c-auto.tex}

\input{temporalia/ps148.tex}

\rubrica{Hic non dicitur Gloria Patri.}

%
\titCapitulum{Psalmus 149.}

\includescore{temporalia/ps149-initium-vii-c-auto.tex}

\input{temporalia/ps149.tex}

\rubrica{Hic non dicitur Gloria Patri.}

%
\titCapitulum{Psalmus 150.}

\includescore{temporalia/ps150-initium-vii-c-auto.tex}

\input{temporalia/ps150.tex}

\pagebreak

\rubrica{Capitulum et responsorium breve de I. vesperis, pg. \pageref{capitulum}.}
\vspace{0.5cm}

\cantusSineNeumas

\pars{Hymnus.}

\includescore{cantus/amon33/hym-OGloriosaDomina.tex}
\includescore{cantus/amon33/hym-OGloriosaDomina-text.tex}

\vspace{0.5cm}

\versusNativitasEst

\vfill

\pagebreak

\cantusCumNeumis

\pars{Canticum Zachariae.}

\hicSuntNeumae
\gresetfirstlineaboveinitial{\small \textsc{\textbf{VIII.G}}}{\small \textsc{\textbf{VIII.G}}}
\includescore{cantus/amon33/ant-ben-laud.tex}

\includescore{temporalia/benedictus-initium-viiiSoll-G-auto.tex}

\input{temporalia/benedictus-viiiG.tex}

\vspace{0.5cm}

\rubrica{Supplicatio litaniae etc. usque ad finem horae
sicut in I. Vesperis, pg. \pageref{oratio}.}

\vfill

\pagebreak

\hora{In II. Vesperis.} %%%%%%%%%%%%%%%%%%%%%%%%%%%%%%%%%%%%%%%%%%%%%%%%%%%%%

\rubrica{Omnia ut in I. Vesperis, pg. \pageref{deusinadiutorium}, 
praeter sequentia.}
\vspace{0.5cm}

\pars{Canticum B. Mariae V.}

\hicSuntNeumae
\gresetfirstlineaboveinitial{\small \textsc{\textbf{I.f}}}{\small \textsc{\textbf{I.f}}}
\includescore{cantus/amon33/ant-magn-vesp2.tex}

% antifona je dlouha a mista dost, dame Magnificat na novou stranku
\vfill
\pagebreak 

\cantusSineNeumas
\includescore{cantus/amon33/magnificat-initium-i-f.tex}

% maly svindl, ale prizvukova struktura Isoll-D2 a Isoll-f je stejna
\input{temporalia/magnificat-iD2.tex}

\pagebreak

%%% TIRAZ
\pagestyle{empty}

\vspace*{5cm}

\begin{center}

Texti et cantus secundum Antiphonale Monasticum, Solesmis 1933. 

Translationes cantuum, nisi aliud notatum, J.P.

Translationes capitulorum sumptae sunt ex:
Jeruzalémská bible, Praha-Kostelní Vydří 2009.

Liber hic imprimis ad usum chori 
\guillemotright Conventus Choralis\guillemotleft\ 
paratus est
et secundum eius consuetudines.

\vspace{1cm}

{\large Editio Sancti Wolfgangi 2012.}

\vspace{2mm}

Series \guillemotright Conventus\guillemotleft, vol. I.

\vspace{1cm}

https://stiwolfgangi.xf.cz

\end{center}

\vfill

\end{document}
