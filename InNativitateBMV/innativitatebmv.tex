% LuaLaTeX

\documentclass[a4paper, twoside, 12pt]{article}
\usepackage[latin]{babel} 
\usepackage[landscape, left=3cm, right=1.5cm, top=2cm, bottom=2cm]{geometry} % okraje stranky

\usepackage{fontspec}
% \setmainfont{Gentium Book Basic}
\setmainfont{Junicode}

\usepackage{multicol}
\usepackage{color}
\usepackage{lettrine}
\usepackage{fancyhdr}

% usual packages loading:
\usepackage{luatextra}
\usepackage{graphicx} % support the \includegraphics command and options
\usepackage{gregoriotex} % for gregorio score inclusion

\newenvironment{titulusOfficii}{\begin{center}}{\end{center}}
\newcommand{\dies}[1]{#1

}
\newcommand{\nomenFesti}[1]{\textbf{\Large #1}

}
\newcommand{\celebratio}[1]{#1

}

\newcommand{\hora}[1]{\vspace{0.5cm}{\large \textbf{#1}}
% from some reason the following doesn't work:
\fancyhead[LE]{\thepage\ / #1}
\fancyhead[RO]{#1 / \thepage}

}

\newcommand{\rubrica}[1]{\vspace{3mm}\textit{#1}}

\newcommand{\notitia}[1]{\textcolor{red}{#1}}

\newcommand{\titCapitulum}[1]{\hfill \textit{#1}}

% volne misto nad antifonami, kam si zpevaci dokresli neumy
\newcommand{\hicSuntNeumae}{\vspace{1.3cm}}

% pars officii, i.e. "oratio", ...
\newcommand{\pars}[1]{\textbf{#1}}

\newenvironment{psalmus}{}{}

%%%% Prejmenovat na latinske:
\newcommand{\nadpisZalmu}[1]{
  \hspace{2cm}\textbf{#1}\vspace{2mm}%
  \nopagebreak%
}

\setlength{\columnsep}{30pt} % prostor mezi sloupci

\begin{document}

\emph{Texti et cantus secundum Antiphonale Monasticum, Solesmis 1933. 
Translationes, nisi aliud notatum, J.P.
Translationes capitulorum sumptae sunt ex:
Jeruzalémská bible, Praha-Kostelní Vydří 2009.}

\begin{titulusOfficii}
\dies{Die 8. Septembris.}
\nomenFesti{In Nativitate B. Mariae Virginis.}
\celebratio{Duplex 2. classis.} % puvodne "cum Octava." Oktavy byly ale zruseny
\end{titulusOfficii}

\renewcommand{\headrulewidth}{0pt} % no horiz. rule at the header
\fancyhf{}
\pagestyle{fancy}

\hora{In I. Vesperis.} %%%%%%%%%%%%%%%%%%%%%%%%%%%%%%%%%%%%%%%%%%%%%%%%%%%%%

% Here we set the space around the initial.
% Please report to http://home.gna.org/gregorio/gregoriotex/details for more details and options
\setspaceafterinitial{2.2mm plus 0em minus 0em}
\setspacebeforeinitial{2.2mm plus 0em minus 0em}

% Here we set the initial font. Change 43 if you want a bigger initial.
\def\greinitialformat#1{%
{\fontsize{43}{43}\selectfont #1}%
}

\includescore{../cantuscommunes/amon33/deusinadiutorium-communis.tex}

\pagebreak

\pars{psalmus 1.}

\hicSuntNeumae
\gresetfirstlineaboveinitial{\small \textsc{\textbf{VIII.G}}}{\small \textsc{\textbf{VIII.G}}}
\includescore{cantus/amon33/ant1.tex}
% \includescore{../tonipsalmorum/arom12/viii-G.tex}
\includescore{temporalia/ps109-initium-viii-G-auto.tex}

\input{temporalia/ps109.tex}

\pagebreak

\pars{psalmus 2.}

\hicSuntNeumae
\gresetfirstlineaboveinitial{\small \textsc{\textbf{VII.c}}}{\small \textsc{\textbf{VII.c}}}
\includescore{cantus/amon33/ant2.tex}
% \includescore{../tonipsalmorum/arom12/vii-c.tex}
\includescore{temporalia/ps112-initium-vii-c-auto.tex}

\input{temporalia/ps112.tex}

\pagebreak

\pars{psalmus 3.}

\hicSuntNeumae
\gresetfirstlineaboveinitial{\small \textsc{\textbf{VI.F}}}{\small \textsc{\textbf{VI.F}}}
\includescore{cantus/amon33/ant3.tex}
% \includescore{../tonipsalmorum/arom12/vi-F.tex}
\includescore{temporalia/ps121-initium-vi-F-auto.tex}

\input{temporalia/ps121.tex}

\pagebreak

\pars{psalmus 4.}

\hicSuntNeumae
\gresetfirstlineaboveinitial{\small \textsc{\textbf{VIII.G}}}{\small \textsc{\textbf{VIII.G}}}
\includescore{cantus/amon33/ant4.tex}
% \includescore{../tonipsalmorum/arom12/viii-G.tex}
\includescore{temporalia/ps126-initium-viii-G-auto.tex}

\input{temporalia/ps126.tex}

\pagebreak

\raggedcolumns

% Capitulum. %%%
\pars{Capitulum.} \titCapitulum{Eccli. 24. b} \label{capitulum}
\begin{multicols}{2}
\lettrine{A}{b} inítio et ante saécula creáta sum,~\dag\mbox{}
et usque ad futúrum saéculum non désinam,~*
et in habitatióne sancta coram ipso ministrávi.

\columnbreak

% preklad Jeruz. bible
Před věky, na počátku mě stvořil,
potrvám věčně. 
Ve svatém Stanu jsem před ním konala službu.
\end{multicols}

\pars{Responsorium breve.}

\gresetfirstlineaboveinitial{\small \textsc{\textbf{VI.}}}{\small \textsc{\textbf{VI.}}}
\includescore{cantus/amon33/resp1v.tex}

% Hymnus. %%%
\pars{Hymnus.}

\gresetfirstlineaboveinitial{\small \textsc{\textbf{I.}}}{\small \textsc{\textbf{I.}}}
\includescore{cantus/amon33/hym-AveMarisStella.tex}
2. Sumens illud Ave Gabriélis ore,\\
Funda nos in pace, Mutans Hevae nomen.\\

3. Solve vincla reis, Profer lumen caecis:\\
Mala nostra pelle, Bona cuncta posce.\\

4. Monstra \textit{te} esse matrem: Sumat per te preces,\\
Qui pro nobis natus, Tulit esse tuus.\\

5. Virgo singuláris, Inter omnes mitis,\\
Nos culpis solútos, Mites fac et castos.\\

6. Vitam praesta puram, Iter para tuam:\\
Ut vidéntes Jesum, Semper collaetémur.\\

7. Sit laus Deo Patri, Summo Christo decus,\\
Spirítui Sancto, Tribus honor unus. Amen.\\


% Versus. %%%
\label{versus}
\begin{multicols}{2}
℣. Natívitas est hódie sanctae Maríae \textbf{Vír}ginis.\\
\indent ℟. Cujus vita ínclyta cunctas illústrat ec\textbf{clé}sias.

\columnbreak

% preklad J.P.
Dnes je Narození svaté Panny Marie.\\
\indent Jejíž předrahý život osvěcuje všechny církve.
\end{multicols}

\pars{Canticum B. Mariae V.}

\hicSuntNeumae
\gresetfirstlineaboveinitial{\small \textsc{\textbf{I.D2}}}{\small \textsc{\textbf{I.D2}}}
\includescore{cantus/amon33/ant-magn-vesp1.tex}
\includescore{../tonipsalmorum/arom12/i-D2.tex}

\input{temporalia/magnificat-iD2.tex}

\rubrica{Ante Orationem, cantatur a Superiore:}

\pars{Supplicatio Litaniae.}

\includescore{../cantuscommunes/amon33/supplicatiolitaniae.tex}

\pars{Oratio Dominica.}

\includescore{../cantuscommunes/amon33/oratiodominica.tex}

\rubrica{Deinde dicitur ab Hebdomadario:}

\includescore{../cantuscommunes/amon33/dominusvobiscum-solemnis.tex}

\rubrica{In choro monialium loco Dominus vobiscum dicitur:}

\includescore{../cantuscommunes/amon33/domineexaudi.tex}

% Oratio. %%%
\pars{Oratio.}
\begin{multicols}{2}
\lettrine{F}{amulis} tuis, quaésumus Dómine, coeléstis grátiae munus impertíre:~\dag\mbox{}
ut quibus beátae Vírginis partus éxstitit salútis exórdium,~*
Nativitátis ejus votíva solémnitas pacis tríbuat increméntum.
Per Dóminum.

\columnbreak

% preklad J.P.
Prosíme tě, Bože, 
uděl nám, svým služebníkům, dar nebeské milosti,
aby těm, kterým \notitia{?(}se porod blahoslavené Panny zjevil jako počátek spásy,\notitia{)?}
\notitia{(co s \emph{votiva}?)} slavnost jejího narození přinesla
rozhojnění pokoje.
Skrze našeho Pána.
\end{multicols}

\rubrica{Hebdomadarius dicit iterum Dominus vobiscum. Postea cantatur a cantore:}

\input{../cantuscommunes/amon33/benedicamus-duplex-vesperae.tex}

\hora{Ad Laudes.} %%%%%%%%%%%%%%%%%%%%%%%%%%%%%%%%%%%%%%%%%%%%%%%%%%%%%%%%%%

% Psalmi festivi (AM33, pg. 721):
% 66 // 92, 99, 62, Dan3, 148+149+150

\pars{psalmus 1.}

\includescore{temporalia/ps66-initium-dir-auto.tex}

\input{temporalia/ps66.tex}

\pagebreak

\pars{psalmus 2.}

\hicSuntNeumae
\gresetfirstlineaboveinitial{\small \textsc{\textbf{VIII.G}}}{\small \textsc{\textbf{VIII.G}}}
\includescore{cantus/amon33/ant1.tex}
% \includescore{../tonipsalmorum/arom12/viii-G.tex}
\includescore{temporalia/ps92-initium-viii-G-auto.tex}

\input{temporalia/ps92.tex}

\pagebreak

\pars{psalmus 3.}

\hicSuntNeumae
\gresetfirstlineaboveinitial{\small \textsc{\textbf{VII.c}}}{\small \textsc{\textbf{VII.c}}}
\includescore{cantus/amon33/ant2.tex}
% \includescore{../tonipsalmorum/arom12/vii-c.tex}
\includescore{temporalia/ps99-initium-vii-c-auto.tex}

\input{temporalia/ps99.tex}

\pagebreak

\pars{psalmus 4.}

\hicSuntNeumae
\gresetfirstlineaboveinitial{\small \textsc{\textbf{VI.F}}}{\small \textsc{\textbf{VI.F}}}
\includescore{cantus/amon33/ant3.tex}
% \includescore{../tonipsalmorum/arom12/vi-F.tex}
\includescore{temporalia/ps62-initium-vi-F-auto.tex}

\input{temporalia/ps62.tex}

\pagebreak

\pars{psalmus 5.}

\hicSuntNeumae
\gresetfirstlineaboveinitial{\small \textsc{\textbf{VIII.G}}}{\small \textsc{\textbf{VIII.G}}}
\includescore{cantus/amon33/ant4.tex}
% \includescore{../tonipsalmorum/arom12/viii-G.tex}

% \input{temporalia/.tex} % Canticum trium

\pagebreak

\pars{psalmus 6.}

\gresetfirstlineaboveinitial{\small \textsc{\textbf{VII.c}}}{\small \textsc{\textbf{VII.c}}}
\includescore{cantus/amon33/ant5.tex}
% \includescore{../tonipsalmorum/arom12/vii-c.tex}

% \input{temporalia/.tex} % pss 148,149,150

\pagebreak

\rubrica{Capitulum et responsorium breve de I. vesperis.}

\pars{Hymnus.}

\includescore{cantus/amon33/hym-OGloriosaDomina.tex}
\includescore{cantus/amon33/hym-OGloriosaDomina-text.tex}

\rubrica{Versus de I. vesperis.}

\pars{Canticum Zachariae.}

\gresetfirstlineaboveinitial{\small \textsc{\textbf{VIII.G}}}{\small \textsc{\textbf{VIII.G}}}
\includescore{cantus/amon33/ant-ben-laud.tex}
\input{temporalia/benedictus-viiiG.tex}

\hora{In II. Vesperis.} %%%%%%%%%%%%%%%%%%%%%%%%%%%%%%%%%%%%%%%%%%%%%%%%%%%%%

\rubrica{Omnia ut in I. Vesperis, praeter sequentia.}

\pars{Canticum B. Mariae V.}

\hicSuntNeumae
\gresetfirstlineaboveinitial{\small \textsc{\textbf{I.f}}}{\small \textsc{\textbf{I.f}}}
\includescore{cantus/amon33/ant-magn-vesp2.tex}
\includescore{../tonipsalmorum/arom12/i-f.tex}

\input{temporalia/magnificat-iD2.tex}


\end{document}
